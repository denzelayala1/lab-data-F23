\documentclass[
reprint,
amsmath,amssymb,
aps,
tikz,
border=5pt
]{revtex4-1}

\usepackage{graphicx}% Include figure files
\usepackage{dcolumn}% Align table columns on decimal point
\usepackage{bm}% bold math
\usepackage{amsmath}
\usepackage{amssymb}
\usepackage{subcaption}
\usepackage{lipsum}
\usepackage{tikz}
\usepackage{pgf}
\usepackage{circuitikz}
\usepackage{multirow}
\usetikzlibrary{arrows}
\usepackage{mathtools}
\DeclarePairedDelimiter\bra{\langle}{\rvert}
\DeclarePairedDelimiter\ket{\lvert}{\rangle}
\DeclarePairedDelimiterX\braket[2]{\langle}{\rangle}{#1\,\delimsize\vert\,\mathopen{}#2}

\AtBeginDocument{\let\latexlabel\label}


\begin{document}


\preprint{APS/123-QED}

\title{Ferroic Materials: Understanding their Phases\\ and Multiferroic Potential}

\author{Denzel Ayala} 
 \email{Denzelay@buffalo.edu}
 \altaffiliation[Also at ]{University of Vermont}%Lines break automatically or can be forced with \\
\author{Mars Anderson}
 \email{anderse3@buffalo.edu}
\author{Serdar K. Gozpinar}

\affiliation{%
State University of New York at Buffalo\\Department of Physics
}%

%\collaboration{CLEO Collaboration}%\noaffiliation

\date{\today}% It is always \today, today,
             %  but any date may be explicitly specified

\begin{abstract}

\lipsum[1]

\end{abstract}
\maketitle

\section*{\label{sec:intro}Introduction \\\lowercase{to} Superconductivity}

Superconductors are technologically critical materials. Their most wide spread use is to create large magnetic fields at a minimal energy cost due to their extremely high conductivity. This has found applications across multiple economic sectors such as the medical field, fundamental physics research, energy production, next-generation computers, and various quantum devices.  In the medical field, Magnetic Resonance Imaging (MRI) machines use large magnetic fields in conjunction with radio waves to perturb the hydrogen in your body's water to produce images of your body. The contrast is due to slight differences in the electronic environments of different tissues. In particle accelerators the superconducting magnets are used to steer the fundamental particles that hold the secrets of the universe. This magnetic steering has also been used to contain plasma in fusion reactors which are intended to produce energy in the future.

\begin{figure}[b]
  \includegraphics[width=0.4 \textwidth]{figures/sc_tech.png}
  \caption{Cartoon representation of technology that use superconductors.~\cite{DoScience}}
  \label{fig:sc_tech}
\end{figure}

Superconductivity is an electronic phase of matter similar to how  gas, liquid and solid are physical phases of mater. However rather than the atomic arrangement undergoing a transition, the electrons are the participants. This phase transition can be of first or second order depending on the type of superconductivity. Over all the transition is the result of the system trying to minimize a Ginzburg-Landau (GL) free energy. Herein we set out to illuminate this phase transition using a van der Pauw measurement configuration to measure the resistivity of two high temperature superconductors. 



  \subsection*{Phenomenology of Superconductivity}

    The two defining hallmarks of a superconductor are perfect conductivity and perfect diamagnetism.\cite{tinkham}  When in the superconducting phase a material's resistivity disappears and current is able to flow freely nearly indefinitely.\cite{tinkham} Lower bounds on the characteristic decay time established by nuclear resonance is on the order of $10^5$ years, with some materials boasting possibility for unchanging fields or currents on times scales of $10^{10{^{10}}}$ years, (paraphrased from Tinkham).\cite{tinkham} 

    \begin{figure}[t]
      \includegraphics[width=0.4 \textwidth]{figures/pc_sc.png}
      \caption{Schematic diagram of a super and perfect conductor in a magnetic field in the normal phase (a), the perfect conducting phase (b), and after the removal of the applied magnetic field but still in the  perfect conducting phase (c).}
      \label{fig:sc_pc}
    \end{figure}


    The second characteristic of perfect diamagnetism is what separates a superconductor from a perfect conductor. While both a super and perfect conductor exclude magnetic field from entering the material, Meissner and Ochsenfeld showed that as a superconductor is cooled through its $T_c$ it expels any field that was originally in the normal sample.~\cite{MnO1933,tinkham} This expulsion is unique as a perfect conductor would trap flux in the material as seen in fig \ref{fig:sc_pc}c. Furthermore the reversibility of this \textbf{Meissner effect} implies that thermodynamic free energy goes into expelling the fields, and can be overwhelmed. This thermodynamic critical field  $H_c$ is related to the difference in the Helmholtz free energy per unit volume of the normal ($f_n$) and superconducting ($f_s$) states (eq \ref{eq:Hc}).
    
    \begin{gather}
      f_n(T) - f_s(T) = \frac{H_c^2(T)}{8\pi}
      \label{eq:Hc} 
    \end{gather}

    The two hallmarks have corresponding phenomenological parameters or characteristic lengths that describe them. In the case of the perfect diamagnetism, the London penetration depth ($\lambda_L$) as discovered by the London brothers in 1935 is indicative of the skin depth associated with superconductors. Pippard introduced the second parameter of coherence length ($\xi_0$) when trying to generalize the London  equations (eqs \ref{eq:LCurrent} -- \ref{eq:le2}) which describes the properties of supercurrent.~\cite{tinkham, pippard}

    \begin{gather}
      \mathbf{J}_s = \frac{-\mathbf{A}}{\Lambda c}
      \label{eq:LCurrent}\\
      \mathbf{E} = \frac{\partial}{\partial t}(\Lambda \mathbf{J}_s)  
      \label{eq:le1} \\
      \mathbf{B} = -c ~\nabla \times (\Lambda \mathbf{J}_s)  
      \label{eq:le2} \\
      \Lambda = \frac{4\pi \lambda^2}{c^2}  = \frac{m}{n_s e^2}
    \end{gather}

    The London brothers chose a special gauge (eq \ref{eq:LCurrent}) that insured the vector potential ($\mathbf{A}$) could be interpreted as a current density on the surface of the superconductor. The gauge required the vector potential is zero in the bulk of the superconductor, there is no normal components to the surface ($\hat{\mathbf{n}}\cdot \mathbf{A} = 0$), and no sources or sinks $\nabla \cdot \mathbf{A} = 0$ in the superconductor. 

    Equation \ref{eq:le1} is the equation that describes the perfect conductivity. In a regular conductor an applied electric field ($\mathbf{E}$) provides the energy needed to sustain a current ($\mathbf{J}$), in the material as Ohm's law indicates that any current would meet a resistance that dissipates the energy as heat. However, in super conductors the lack of resistance makes it so the electric field accelerates the superconducting electrons. 
 

    Ginzburg-Landau

    \begin{gather}
      n_s = |\psi(x)|^2 \\
      f = f_{n0} + \alpha |\psi|^2 + \frac{\beta}{2}|\psi|^4 + \nonumber \\ \frac{1}{2m^*} \lvert\left(\frac{\hbar}{i}\nabla - \frac{e^*}{c}\mathbf{A}\psi\right)\rvert^2 + \frac{h^2}{8\pi} \\
      f_s - f_n = \alpha |\psi|^2 + \frac{1}{2}\beta|\psi|^4
    \end{gather}


    
    \begin{figure}[t]
      \includegraphics[width=0.5 \textwidth]{figures/gl_free_E.png}
      \caption{Ginzburg-Landau free energy functions for $T>T_c$ (left) and $T<T_c$ (right). Dots indicate equilibrium positions.~\cite{tinkham}}
      \label{fig:sc_pc}
    \end{figure}


    TODO: 
    explain london equations, coherence length, GL equations. Introduce types of superconductors and our materials. Introduce and explain measurement technique.
    We are using YBCO and BSCCO both of which are cuperates, so discuss the differences in them.\cite{Cohen92} 
    
    Type 1 vs 2 super conductors. 
    
    Measurement technique van Der Pauw configuration and why it works. \cite{tinkham}


    \begin{figure}[h]
      \includegraphics[width=0.35 \textwidth]{figures/cryostat.png}
      \caption{Schematic of the double jacket cryostat surrounded by electromagnets.}
      \label{fig:cryostat}
    \end{figure}

\section*{Experimental }


    The sample was placed in a double jacket cryostat (fig \ref{fig:cryostat}). The inner layer is sealed using the sample probe and was connected to a  two-stage rotary vane vacuum pump system to remove the atmospheric gases. It was also connected to a helium line to replace the displaced gas. This inner jacket is the heat exchange layer between the cryogen of choice and the sample. The outer layer was filled with liquid nitrogen. When preparing your own samples remove and replace the atmospheric gas from the inner chamber BEFORE adding liquid N$_2$. Oxygen can condense in the inner jacket and cause explosions. 
    
    
    The cryostat was enveloped in large electromagnetic coils that could produce a uniform magnetic field with a field strength of up to 152 mT (green in fig \ref{fig:cryostat}). As our electromagnet is \textbf{not} actively cooled we applied a maximum field of 122 mT in order to prevent overheating from high currents.


    \begin{figure}[b]
      \includegraphics[width=0.5 \textwidth]{figures/lakeshore.png}
      \caption{Front panel of Lakeshore 330 temperature controller. PID values used were 250, 50, 0 respectively.}
      \label{fig:lakeshore}
    \end{figure}

    
    The temperature of the exchange gas was controlled using a Lakeshore 330 Autotuning temperature controller (fig \ref{fig:lakeshore}). It only provides heat to the environment through heater coils wrapped in white teflon and attached to the sample probe. Any cooling was accomplished by allowing the liquid nitrogen to drop the temperature of the exchange gas. This controller uses a Proportional-Integral-Derivative (PID) feedback control loop. The P, I, D, parameters used were 250, 50, and 0 respectively. This component continuously  calculates the difference between current temperature and the set point. The P-parameter multiplies the error directly. The I-parameter takes an integral over a time interval and multiplies that against the error. Minimizing this integral leads to a closer match between the set point and sample temperature. Finally the D-parameter multiplies the rate of change in error and controls how ``rapidly'' the set point is approached. Too high of a D-parameter results in overshooting the temperature. The sample temperature is measured using a temperature sensitive diode mounted on the back of the sample holder. The temperature controller sends a positive polarity $\mu$A current towards the diode resulting a voltage difference across the diode. This voltage is directly proportional to the sample temperature. 
    
    \begin{figure}[b]
      \includegraphics[width=0.25 \textwidth]{figures/sampleHolder.png}
      \caption{A sample (in black) mounted on the copper sample holder. The heating coils are under the white teflon.}
      \label{fig:lakeshore}
    \end{figure}


    The samples were approximately 12 mm in diameter and 2 mm thick. Thin copper wire were bonded to the edges of the sample as required by the Van der Pauw method with a carbon-based conductive paste. The numbers correspond to the wires that will connect to the lock-in amplifier. Take care not to touch the sample directly as the oils and contaminants on your skin can affect the readings. 

TODO: lockin amplifier (LA), VdP technique. First discribe the details of the instrument we are using. Then go into the procedure including best practices for sample preparation. 
To test the variation of the time constant we raised and lowered the frequency the LA was tuned too. 
  

  \begin{center}
  \begin{circuitikz}
    \draw 
      (0,0) to [short]   (0.5,0)
      to [V,l_=$V_{AC}$] (1, 0)
      to [short]         (1.5, 0)
      to [short,-*]      (1.5, 1)
      to [short, -*]     (1.5, 2)
      to [short]         (1.5, 3)
      to [short]         (1, 3)
      to [vsourcesin]    (0.5, 3)
      to [short]         (0.5, 3)
      to [short]         (0, 3)
      to [short,-*]      (0, 2)
      to [short, -*]     (0, 1);
      
    \draw 
      (0,0) to [short,-*]      (0,1)
      to [short, l_=3]         (1.5,1)
      to [short, l_=1]         (1.5,2)
      to [short, l_=2]         (0,2);
  \end{circuitikz}
  \end{center}


\section*{Results \lowercase{and} Discussion }

  \subsection*{Lock-in Amplifier Properties}

    When varying the LA's  sensitivity the most significant digit changed. At higher sensitivities the number of visible digits decreased but the values were more precise displaying a value of $XX.X\underbar{X} ~\mu V$. changed the  ``response time'' of the instrument. At low time constants the values changed rapidly with the changing reference frequency. However, the values fluctuated constantly and were not as stable. This lack of stability came from all sort of signal noise. In contrast when the time constant was set to (3x1 s) the displayed values changed slowly but the final value was relatively stable. This stability came from the time averaging of the noise during the data collection. As the majority of the input noise was random its average value is zero and its variance is constant. 


    \begin{figure}[h]
    
      \resizebox{0.45\textwidth}{!}{%% Creator: Matplotlib, PGF backend
%%
%% To include the figure in your LaTeX document, write
%%   \input{<filename>.pgf}
%%
%% Make sure the required packages are loaded in your preamble
%%   \usepackage{pgf}
%%
%% Also ensure that all the required font packages are loaded; for instance,
%% the lmodern package is sometimes necessary when using math font.
%%   \usepackage{lmodern}
%%
%% Figures using additional raster images can only be included by \input if
%% they are in the same directory as the main LaTeX file. For loading figures
%% from other directories you can use the `import` package
%%   \usepackage{import}
%%
%% and then include the figures with
%%   \import{<path to file>}{<filename>.pgf}
%%
%% Matplotlib used the following preamble
%%   
%%   \usepackage{fontspec}
%%   \setmainfont{DejaVuSerif.ttf}[Path=\detokenize{/home/dayala/.local/lib/python3.10/site-packages/matplotlib/mpl-data/fonts/ttf/}]
%%   \setsansfont{DejaVuSans.ttf}[Path=\detokenize{/home/dayala/.local/lib/python3.10/site-packages/matplotlib/mpl-data/fonts/ttf/}]
%%   \setmonofont{DejaVuSansMono.ttf}[Path=\detokenize{/home/dayala/.local/lib/python3.10/site-packages/matplotlib/mpl-data/fonts/ttf/}]
%%   \makeatletter\@ifpackageloaded{underscore}{}{\usepackage[strings]{underscore}}\makeatother
%%
\begingroup%
\makeatletter%
\begin{pgfpicture}%
\pgfpathrectangle{\pgfpointorigin}{\pgfqpoint{8.000000in}{5.500000in}}%
\pgfusepath{use as bounding box, clip}%
\begin{pgfscope}%
\pgfsetbuttcap%
\pgfsetmiterjoin%
\definecolor{currentfill}{rgb}{1.000000,1.000000,1.000000}%
\pgfsetfillcolor{currentfill}%
\pgfsetlinewidth{0.000000pt}%
\definecolor{currentstroke}{rgb}{1.000000,1.000000,1.000000}%
\pgfsetstrokecolor{currentstroke}%
\pgfsetdash{}{0pt}%
\pgfpathmoveto{\pgfqpoint{0.000000in}{0.000000in}}%
\pgfpathlineto{\pgfqpoint{8.000000in}{0.000000in}}%
\pgfpathlineto{\pgfqpoint{8.000000in}{5.500000in}}%
\pgfpathlineto{\pgfqpoint{0.000000in}{5.500000in}}%
\pgfpathlineto{\pgfqpoint{0.000000in}{0.000000in}}%
\pgfpathclose%
\pgfusepath{fill}%
\end{pgfscope}%
\begin{pgfscope}%
\pgfsetbuttcap%
\pgfsetmiterjoin%
\definecolor{currentfill}{rgb}{0.917647,0.917647,0.949020}%
\pgfsetfillcolor{currentfill}%
\pgfsetlinewidth{0.000000pt}%
\definecolor{currentstroke}{rgb}{0.000000,0.000000,0.000000}%
\pgfsetstrokecolor{currentstroke}%
\pgfsetstrokeopacity{0.000000}%
\pgfsetdash{}{0pt}%
\pgfpathmoveto{\pgfqpoint{0.670278in}{0.586684in}}%
\pgfpathlineto{\pgfqpoint{7.850000in}{0.586684in}}%
\pgfpathlineto{\pgfqpoint{7.850000in}{5.136667in}}%
\pgfpathlineto{\pgfqpoint{0.670278in}{5.136667in}}%
\pgfpathlineto{\pgfqpoint{0.670278in}{0.586684in}}%
\pgfpathclose%
\pgfusepath{fill}%
\end{pgfscope}%
\begin{pgfscope}%
\pgfpathrectangle{\pgfqpoint{0.670278in}{0.586684in}}{\pgfqpoint{7.179722in}{4.549983in}}%
\pgfusepath{clip}%
\pgfsetroundcap%
\pgfsetroundjoin%
\pgfsetlinewidth{1.003750pt}%
\definecolor{currentstroke}{rgb}{1.000000,1.000000,1.000000}%
\pgfsetstrokecolor{currentstroke}%
\pgfsetdash{}{0pt}%
\pgfpathmoveto{\pgfqpoint{1.498707in}{0.586684in}}%
\pgfpathlineto{\pgfqpoint{1.498707in}{5.136667in}}%
\pgfusepath{stroke}%
\end{pgfscope}%
\begin{pgfscope}%
\definecolor{textcolor}{rgb}{0.150000,0.150000,0.150000}%
\pgfsetstrokecolor{textcolor}%
\pgfsetfillcolor{textcolor}%
\pgftext[x=1.498707in,y=0.489462in,,top]{\color{textcolor}\rmfamily\fontsize{10.000000}{12.000000}\selectfont 60}%
\end{pgfscope}%
\begin{pgfscope}%
\pgfpathrectangle{\pgfqpoint{0.670278in}{0.586684in}}{\pgfqpoint{7.179722in}{4.549983in}}%
\pgfusepath{clip}%
\pgfsetroundcap%
\pgfsetroundjoin%
\pgfsetlinewidth{1.003750pt}%
\definecolor{currentstroke}{rgb}{1.000000,1.000000,1.000000}%
\pgfsetstrokecolor{currentstroke}%
\pgfsetdash{}{0pt}%
\pgfpathmoveto{\pgfqpoint{2.502864in}{0.586684in}}%
\pgfpathlineto{\pgfqpoint{2.502864in}{5.136667in}}%
\pgfusepath{stroke}%
\end{pgfscope}%
\begin{pgfscope}%
\definecolor{textcolor}{rgb}{0.150000,0.150000,0.150000}%
\pgfsetstrokecolor{textcolor}%
\pgfsetfillcolor{textcolor}%
\pgftext[x=2.502864in,y=0.489462in,,top]{\color{textcolor}\rmfamily\fontsize{10.000000}{12.000000}\selectfont 120}%
\end{pgfscope}%
\begin{pgfscope}%
\pgfpathrectangle{\pgfqpoint{0.670278in}{0.586684in}}{\pgfqpoint{7.179722in}{4.549983in}}%
\pgfusepath{clip}%
\pgfsetroundcap%
\pgfsetroundjoin%
\pgfsetlinewidth{1.003750pt}%
\definecolor{currentstroke}{rgb}{1.000000,1.000000,1.000000}%
\pgfsetstrokecolor{currentstroke}%
\pgfsetdash{}{0pt}%
\pgfpathmoveto{\pgfqpoint{3.507021in}{0.586684in}}%
\pgfpathlineto{\pgfqpoint{3.507021in}{5.136667in}}%
\pgfusepath{stroke}%
\end{pgfscope}%
\begin{pgfscope}%
\definecolor{textcolor}{rgb}{0.150000,0.150000,0.150000}%
\pgfsetstrokecolor{textcolor}%
\pgfsetfillcolor{textcolor}%
\pgftext[x=3.507021in,y=0.489462in,,top]{\color{textcolor}\rmfamily\fontsize{10.000000}{12.000000}\selectfont 180}%
\end{pgfscope}%
\begin{pgfscope}%
\pgfpathrectangle{\pgfqpoint{0.670278in}{0.586684in}}{\pgfqpoint{7.179722in}{4.549983in}}%
\pgfusepath{clip}%
\pgfsetroundcap%
\pgfsetroundjoin%
\pgfsetlinewidth{1.003750pt}%
\definecolor{currentstroke}{rgb}{1.000000,1.000000,1.000000}%
\pgfsetstrokecolor{currentstroke}%
\pgfsetdash{}{0pt}%
\pgfpathmoveto{\pgfqpoint{4.511178in}{0.586684in}}%
\pgfpathlineto{\pgfqpoint{4.511178in}{5.136667in}}%
\pgfusepath{stroke}%
\end{pgfscope}%
\begin{pgfscope}%
\definecolor{textcolor}{rgb}{0.150000,0.150000,0.150000}%
\pgfsetstrokecolor{textcolor}%
\pgfsetfillcolor{textcolor}%
\pgftext[x=4.511178in,y=0.489462in,,top]{\color{textcolor}\rmfamily\fontsize{10.000000}{12.000000}\selectfont 240}%
\end{pgfscope}%
\begin{pgfscope}%
\pgfpathrectangle{\pgfqpoint{0.670278in}{0.586684in}}{\pgfqpoint{7.179722in}{4.549983in}}%
\pgfusepath{clip}%
\pgfsetroundcap%
\pgfsetroundjoin%
\pgfsetlinewidth{1.003750pt}%
\definecolor{currentstroke}{rgb}{1.000000,1.000000,1.000000}%
\pgfsetstrokecolor{currentstroke}%
\pgfsetdash{}{0pt}%
\pgfpathmoveto{\pgfqpoint{5.515335in}{0.586684in}}%
\pgfpathlineto{\pgfqpoint{5.515335in}{5.136667in}}%
\pgfusepath{stroke}%
\end{pgfscope}%
\begin{pgfscope}%
\definecolor{textcolor}{rgb}{0.150000,0.150000,0.150000}%
\pgfsetstrokecolor{textcolor}%
\pgfsetfillcolor{textcolor}%
\pgftext[x=5.515335in,y=0.489462in,,top]{\color{textcolor}\rmfamily\fontsize{10.000000}{12.000000}\selectfont 300}%
\end{pgfscope}%
\begin{pgfscope}%
\pgfpathrectangle{\pgfqpoint{0.670278in}{0.586684in}}{\pgfqpoint{7.179722in}{4.549983in}}%
\pgfusepath{clip}%
\pgfsetroundcap%
\pgfsetroundjoin%
\pgfsetlinewidth{1.003750pt}%
\definecolor{currentstroke}{rgb}{1.000000,1.000000,1.000000}%
\pgfsetstrokecolor{currentstroke}%
\pgfsetdash{}{0pt}%
\pgfpathmoveto{\pgfqpoint{6.519492in}{0.586684in}}%
\pgfpathlineto{\pgfqpoint{6.519492in}{5.136667in}}%
\pgfusepath{stroke}%
\end{pgfscope}%
\begin{pgfscope}%
\definecolor{textcolor}{rgb}{0.150000,0.150000,0.150000}%
\pgfsetstrokecolor{textcolor}%
\pgfsetfillcolor{textcolor}%
\pgftext[x=6.519492in,y=0.489462in,,top]{\color{textcolor}\rmfamily\fontsize{10.000000}{12.000000}\selectfont 360}%
\end{pgfscope}%
\begin{pgfscope}%
\pgfpathrectangle{\pgfqpoint{0.670278in}{0.586684in}}{\pgfqpoint{7.179722in}{4.549983in}}%
\pgfusepath{clip}%
\pgfsetroundcap%
\pgfsetroundjoin%
\pgfsetlinewidth{1.003750pt}%
\definecolor{currentstroke}{rgb}{1.000000,1.000000,1.000000}%
\pgfsetstrokecolor{currentstroke}%
\pgfsetdash{}{0pt}%
\pgfpathmoveto{\pgfqpoint{7.523649in}{0.586684in}}%
\pgfpathlineto{\pgfqpoint{7.523649in}{5.136667in}}%
\pgfusepath{stroke}%
\end{pgfscope}%
\begin{pgfscope}%
\definecolor{textcolor}{rgb}{0.150000,0.150000,0.150000}%
\pgfsetstrokecolor{textcolor}%
\pgfsetfillcolor{textcolor}%
\pgftext[x=7.523649in,y=0.489462in,,top]{\color{textcolor}\rmfamily\fontsize{10.000000}{12.000000}\selectfont 420}%
\end{pgfscope}%
\begin{pgfscope}%
\definecolor{textcolor}{rgb}{0.150000,0.150000,0.150000}%
\pgfsetstrokecolor{textcolor}%
\pgfsetfillcolor{textcolor}%
\pgftext[x=4.260139in,y=0.299493in,,top]{\color{textcolor}\rmfamily\fontsize{11.000000}{13.200000}\selectfont Frequency (Hz)}%
\end{pgfscope}%
\begin{pgfscope}%
\pgfpathrectangle{\pgfqpoint{0.670278in}{0.586684in}}{\pgfqpoint{7.179722in}{4.549983in}}%
\pgfusepath{clip}%
\pgfsetroundcap%
\pgfsetroundjoin%
\pgfsetlinewidth{1.003750pt}%
\definecolor{currentstroke}{rgb}{1.000000,1.000000,1.000000}%
\pgfsetstrokecolor{currentstroke}%
\pgfsetdash{}{0pt}%
\pgfpathmoveto{\pgfqpoint{0.670278in}{0.640045in}}%
\pgfpathlineto{\pgfqpoint{7.850000in}{0.640045in}}%
\pgfusepath{stroke}%
\end{pgfscope}%
\begin{pgfscope}%
\definecolor{textcolor}{rgb}{0.150000,0.150000,0.150000}%
\pgfsetstrokecolor{textcolor}%
\pgfsetfillcolor{textcolor}%
\pgftext[x=0.396325in, y=0.587283in, left, base]{\color{textcolor}\rmfamily\fontsize{10.000000}{12.000000}\selectfont 10}%
\end{pgfscope}%
\begin{pgfscope}%
\pgfpathrectangle{\pgfqpoint{0.670278in}{0.586684in}}{\pgfqpoint{7.179722in}{4.549983in}}%
\pgfusepath{clip}%
\pgfsetroundcap%
\pgfsetroundjoin%
\pgfsetlinewidth{1.003750pt}%
\definecolor{currentstroke}{rgb}{1.000000,1.000000,1.000000}%
\pgfsetstrokecolor{currentstroke}%
\pgfsetdash{}{0pt}%
\pgfpathmoveto{\pgfqpoint{0.670278in}{1.337574in}}%
\pgfpathlineto{\pgfqpoint{7.850000in}{1.337574in}}%
\pgfusepath{stroke}%
\end{pgfscope}%
\begin{pgfscope}%
\definecolor{textcolor}{rgb}{0.150000,0.150000,0.150000}%
\pgfsetstrokecolor{textcolor}%
\pgfsetfillcolor{textcolor}%
\pgftext[x=0.396325in, y=1.284813in, left, base]{\color{textcolor}\rmfamily\fontsize{10.000000}{12.000000}\selectfont 12}%
\end{pgfscope}%
\begin{pgfscope}%
\pgfpathrectangle{\pgfqpoint{0.670278in}{0.586684in}}{\pgfqpoint{7.179722in}{4.549983in}}%
\pgfusepath{clip}%
\pgfsetroundcap%
\pgfsetroundjoin%
\pgfsetlinewidth{1.003750pt}%
\definecolor{currentstroke}{rgb}{1.000000,1.000000,1.000000}%
\pgfsetstrokecolor{currentstroke}%
\pgfsetdash{}{0pt}%
\pgfpathmoveto{\pgfqpoint{0.670278in}{2.035103in}}%
\pgfpathlineto{\pgfqpoint{7.850000in}{2.035103in}}%
\pgfusepath{stroke}%
\end{pgfscope}%
\begin{pgfscope}%
\definecolor{textcolor}{rgb}{0.150000,0.150000,0.150000}%
\pgfsetstrokecolor{textcolor}%
\pgfsetfillcolor{textcolor}%
\pgftext[x=0.396325in, y=1.982342in, left, base]{\color{textcolor}\rmfamily\fontsize{10.000000}{12.000000}\selectfont 14}%
\end{pgfscope}%
\begin{pgfscope}%
\pgfpathrectangle{\pgfqpoint{0.670278in}{0.586684in}}{\pgfqpoint{7.179722in}{4.549983in}}%
\pgfusepath{clip}%
\pgfsetroundcap%
\pgfsetroundjoin%
\pgfsetlinewidth{1.003750pt}%
\definecolor{currentstroke}{rgb}{1.000000,1.000000,1.000000}%
\pgfsetstrokecolor{currentstroke}%
\pgfsetdash{}{0pt}%
\pgfpathmoveto{\pgfqpoint{0.670278in}{2.732632in}}%
\pgfpathlineto{\pgfqpoint{7.850000in}{2.732632in}}%
\pgfusepath{stroke}%
\end{pgfscope}%
\begin{pgfscope}%
\definecolor{textcolor}{rgb}{0.150000,0.150000,0.150000}%
\pgfsetstrokecolor{textcolor}%
\pgfsetfillcolor{textcolor}%
\pgftext[x=0.396325in, y=2.679871in, left, base]{\color{textcolor}\rmfamily\fontsize{10.000000}{12.000000}\selectfont 16}%
\end{pgfscope}%
\begin{pgfscope}%
\pgfpathrectangle{\pgfqpoint{0.670278in}{0.586684in}}{\pgfqpoint{7.179722in}{4.549983in}}%
\pgfusepath{clip}%
\pgfsetroundcap%
\pgfsetroundjoin%
\pgfsetlinewidth{1.003750pt}%
\definecolor{currentstroke}{rgb}{1.000000,1.000000,1.000000}%
\pgfsetstrokecolor{currentstroke}%
\pgfsetdash{}{0pt}%
\pgfpathmoveto{\pgfqpoint{0.670278in}{3.430162in}}%
\pgfpathlineto{\pgfqpoint{7.850000in}{3.430162in}}%
\pgfusepath{stroke}%
\end{pgfscope}%
\begin{pgfscope}%
\definecolor{textcolor}{rgb}{0.150000,0.150000,0.150000}%
\pgfsetstrokecolor{textcolor}%
\pgfsetfillcolor{textcolor}%
\pgftext[x=0.396325in, y=3.377400in, left, base]{\color{textcolor}\rmfamily\fontsize{10.000000}{12.000000}\selectfont 18}%
\end{pgfscope}%
\begin{pgfscope}%
\pgfpathrectangle{\pgfqpoint{0.670278in}{0.586684in}}{\pgfqpoint{7.179722in}{4.549983in}}%
\pgfusepath{clip}%
\pgfsetroundcap%
\pgfsetroundjoin%
\pgfsetlinewidth{1.003750pt}%
\definecolor{currentstroke}{rgb}{1.000000,1.000000,1.000000}%
\pgfsetstrokecolor{currentstroke}%
\pgfsetdash{}{0pt}%
\pgfpathmoveto{\pgfqpoint{0.670278in}{4.127691in}}%
\pgfpathlineto{\pgfqpoint{7.850000in}{4.127691in}}%
\pgfusepath{stroke}%
\end{pgfscope}%
\begin{pgfscope}%
\definecolor{textcolor}{rgb}{0.150000,0.150000,0.150000}%
\pgfsetstrokecolor{textcolor}%
\pgfsetfillcolor{textcolor}%
\pgftext[x=0.396325in, y=4.074929in, left, base]{\color{textcolor}\rmfamily\fontsize{10.000000}{12.000000}\selectfont 20}%
\end{pgfscope}%
\begin{pgfscope}%
\pgfpathrectangle{\pgfqpoint{0.670278in}{0.586684in}}{\pgfqpoint{7.179722in}{4.549983in}}%
\pgfusepath{clip}%
\pgfsetroundcap%
\pgfsetroundjoin%
\pgfsetlinewidth{1.003750pt}%
\definecolor{currentstroke}{rgb}{1.000000,1.000000,1.000000}%
\pgfsetstrokecolor{currentstroke}%
\pgfsetdash{}{0pt}%
\pgfpathmoveto{\pgfqpoint{0.670278in}{4.825220in}}%
\pgfpathlineto{\pgfqpoint{7.850000in}{4.825220in}}%
\pgfusepath{stroke}%
\end{pgfscope}%
\begin{pgfscope}%
\definecolor{textcolor}{rgb}{0.150000,0.150000,0.150000}%
\pgfsetstrokecolor{textcolor}%
\pgfsetfillcolor{textcolor}%
\pgftext[x=0.396325in, y=4.772458in, left, base]{\color{textcolor}\rmfamily\fontsize{10.000000}{12.000000}\selectfont 22}%
\end{pgfscope}%
\begin{pgfscope}%
\definecolor{textcolor}{rgb}{0.150000,0.150000,0.150000}%
\pgfsetstrokecolor{textcolor}%
\pgfsetfillcolor{textcolor}%
\pgftext[x=0.340769in,y=2.861675in,,bottom,rotate=90.000000]{\color{textcolor}\rmfamily\fontsize{11.000000}{13.200000}\selectfont Voltage (\(\displaystyle \mu\)V)}%
\end{pgfscope}%
\begin{pgfscope}%
\pgfpathrectangle{\pgfqpoint{0.670278in}{0.586684in}}{\pgfqpoint{7.179722in}{4.549983in}}%
\pgfusepath{clip}%
\pgfsetbuttcap%
\pgfsetroundjoin%
\definecolor{currentfill}{rgb}{0.298039,0.447059,0.690196}%
\pgfsetfillcolor{currentfill}%
\pgfsetlinewidth{0.301125pt}%
\definecolor{currentstroke}{rgb}{0.298039,0.447059,0.690196}%
\pgfsetstrokecolor{currentstroke}%
\pgfsetdash{}{0pt}%
\pgfsys@defobject{currentmarker}{\pgfqpoint{-0.031056in}{-0.031056in}}{\pgfqpoint{0.031056in}{0.031056in}}{%
\pgfpathmoveto{\pgfqpoint{0.000000in}{-0.031056in}}%
\pgfpathcurveto{\pgfqpoint{0.008236in}{-0.031056in}}{\pgfqpoint{0.016136in}{-0.027784in}}{\pgfqpoint{0.021960in}{-0.021960in}}%
\pgfpathcurveto{\pgfqpoint{0.027784in}{-0.016136in}}{\pgfqpoint{0.031056in}{-0.008236in}}{\pgfqpoint{0.031056in}{0.000000in}}%
\pgfpathcurveto{\pgfqpoint{0.031056in}{0.008236in}}{\pgfqpoint{0.027784in}{0.016136in}}{\pgfqpoint{0.021960in}{0.021960in}}%
\pgfpathcurveto{\pgfqpoint{0.016136in}{0.027784in}}{\pgfqpoint{0.008236in}{0.031056in}}{\pgfqpoint{0.000000in}{0.031056in}}%
\pgfpathcurveto{\pgfqpoint{-0.008236in}{0.031056in}}{\pgfqpoint{-0.016136in}{0.027784in}}{\pgfqpoint{-0.021960in}{0.021960in}}%
\pgfpathcurveto{\pgfqpoint{-0.027784in}{0.016136in}}{\pgfqpoint{-0.031056in}{0.008236in}}{\pgfqpoint{-0.031056in}{0.000000in}}%
\pgfpathcurveto{\pgfqpoint{-0.031056in}{-0.008236in}}{\pgfqpoint{-0.027784in}{-0.016136in}}{\pgfqpoint{-0.021960in}{-0.021960in}}%
\pgfpathcurveto{\pgfqpoint{-0.016136in}{-0.027784in}}{\pgfqpoint{-0.008236in}{-0.031056in}}{\pgfqpoint{0.000000in}{-0.031056in}}%
\pgfpathlineto{\pgfqpoint{0.000000in}{-0.031056in}}%
\pgfpathclose%
\pgfusepath{stroke,fill}%
}%
\begin{pgfscope}%
\pgfsys@transformshift{0.996629in}{3.008156in}%
\pgfsys@useobject{currentmarker}{}%
\end{pgfscope}%
\begin{pgfscope}%
\pgfsys@transformshift{1.498707in}{3.276705in}%
\pgfsys@useobject{currentmarker}{}%
\end{pgfscope}%
\begin{pgfscope}%
\pgfsys@transformshift{2.000786in}{3.022107in}%
\pgfsys@useobject{currentmarker}{}%
\end{pgfscope}%
\begin{pgfscope}%
\pgfsys@transformshift{2.502864in}{3.020363in}%
\pgfsys@useobject{currentmarker}{}%
\end{pgfscope}%
\begin{pgfscope}%
\pgfsys@transformshift{3.004943in}{3.022107in}%
\pgfsys@useobject{currentmarker}{}%
\end{pgfscope}%
\begin{pgfscope}%
\pgfsys@transformshift{3.507021in}{3.046521in}%
\pgfsys@useobject{currentmarker}{}%
\end{pgfscope}%
\begin{pgfscope}%
\pgfsys@transformshift{4.009100in}{3.029082in}%
\pgfsys@useobject{currentmarker}{}%
\end{pgfscope}%
\begin{pgfscope}%
\pgfsys@transformshift{4.511178in}{3.029082in}%
\pgfsys@useobject{currentmarker}{}%
\end{pgfscope}%
\begin{pgfscope}%
\pgfsys@transformshift{5.013257in}{3.032570in}%
\pgfsys@useobject{currentmarker}{}%
\end{pgfscope}%
\begin{pgfscope}%
\pgfsys@transformshift{5.515335in}{3.131968in}%
\pgfsys@useobject{currentmarker}{}%
\end{pgfscope}%
\begin{pgfscope}%
\pgfsys@transformshift{6.017414in}{3.032570in}%
\pgfsys@useobject{currentmarker}{}%
\end{pgfscope}%
\begin{pgfscope}%
\pgfsys@transformshift{6.519492in}{3.027339in}%
\pgfsys@useobject{currentmarker}{}%
\end{pgfscope}%
\begin{pgfscope}%
\pgfsys@transformshift{7.021571in}{3.032570in}%
\pgfsys@useobject{currentmarker}{}%
\end{pgfscope}%
\begin{pgfscope}%
\pgfsys@transformshift{7.523649in}{3.030826in}%
\pgfsys@useobject{currentmarker}{}%
\end{pgfscope}%
\end{pgfscope}%
\begin{pgfscope}%
\pgfpathrectangle{\pgfqpoint{0.670278in}{0.586684in}}{\pgfqpoint{7.179722in}{4.549983in}}%
\pgfusepath{clip}%
\pgfsetbuttcap%
\pgfsetroundjoin%
\pgfsetlinewidth{1.756562pt}%
\definecolor{currentstroke}{rgb}{0.298039,0.447059,0.690196}%
\pgfsetstrokecolor{currentstroke}%
\pgfsetdash{}{0pt}%
\pgfpathmoveto{\pgfqpoint{0.996629in}{3.004669in}}%
\pgfpathlineto{\pgfqpoint{0.996629in}{3.011644in}}%
\pgfusepath{stroke}%
\end{pgfscope}%
\begin{pgfscope}%
\pgfpathrectangle{\pgfqpoint{0.670278in}{0.586684in}}{\pgfqpoint{7.179722in}{4.549983in}}%
\pgfusepath{clip}%
\pgfsetbuttcap%
\pgfsetroundjoin%
\pgfsetlinewidth{1.756562pt}%
\definecolor{currentstroke}{rgb}{0.298039,0.447059,0.690196}%
\pgfsetstrokecolor{currentstroke}%
\pgfsetdash{}{0pt}%
\pgfpathmoveto{\pgfqpoint{1.498707in}{1.623561in}}%
\pgfpathlineto{\pgfqpoint{1.498707in}{4.929849in}}%
\pgfusepath{stroke}%
\end{pgfscope}%
\begin{pgfscope}%
\pgfpathrectangle{\pgfqpoint{0.670278in}{0.586684in}}{\pgfqpoint{7.179722in}{4.549983in}}%
\pgfusepath{clip}%
\pgfsetbuttcap%
\pgfsetroundjoin%
\pgfsetlinewidth{1.756562pt}%
\definecolor{currentstroke}{rgb}{0.298039,0.447059,0.690196}%
\pgfsetstrokecolor{currentstroke}%
\pgfsetdash{}{0pt}%
\pgfpathmoveto{\pgfqpoint{2.000786in}{3.018619in}}%
\pgfpathlineto{\pgfqpoint{2.000786in}{3.025595in}}%
\pgfusepath{stroke}%
\end{pgfscope}%
\begin{pgfscope}%
\pgfpathrectangle{\pgfqpoint{0.670278in}{0.586684in}}{\pgfqpoint{7.179722in}{4.549983in}}%
\pgfusepath{clip}%
\pgfsetbuttcap%
\pgfsetroundjoin%
\pgfsetlinewidth{1.756562pt}%
\definecolor{currentstroke}{rgb}{0.298039,0.447059,0.690196}%
\pgfsetstrokecolor{currentstroke}%
\pgfsetdash{}{0pt}%
\pgfpathmoveto{\pgfqpoint{2.502864in}{3.008156in}}%
\pgfpathlineto{\pgfqpoint{2.502864in}{3.032570in}}%
\pgfusepath{stroke}%
\end{pgfscope}%
\begin{pgfscope}%
\pgfpathrectangle{\pgfqpoint{0.670278in}{0.586684in}}{\pgfqpoint{7.179722in}{4.549983in}}%
\pgfusepath{clip}%
\pgfsetbuttcap%
\pgfsetroundjoin%
\pgfsetlinewidth{1.756562pt}%
\definecolor{currentstroke}{rgb}{0.298039,0.447059,0.690196}%
\pgfsetstrokecolor{currentstroke}%
\pgfsetdash{}{0pt}%
\pgfpathmoveto{\pgfqpoint{3.004943in}{3.022107in}}%
\pgfpathlineto{\pgfqpoint{3.004943in}{3.022107in}}%
\pgfusepath{stroke}%
\end{pgfscope}%
\begin{pgfscope}%
\pgfpathrectangle{\pgfqpoint{0.670278in}{0.586684in}}{\pgfqpoint{7.179722in}{4.549983in}}%
\pgfusepath{clip}%
\pgfsetbuttcap%
\pgfsetroundjoin%
\pgfsetlinewidth{1.756562pt}%
\definecolor{currentstroke}{rgb}{0.298039,0.447059,0.690196}%
\pgfsetstrokecolor{currentstroke}%
\pgfsetdash{}{0pt}%
\pgfpathmoveto{\pgfqpoint{3.507021in}{0.793501in}}%
\pgfpathlineto{\pgfqpoint{3.507021in}{4.511332in}}%
\pgfusepath{stroke}%
\end{pgfscope}%
\begin{pgfscope}%
\pgfpathrectangle{\pgfqpoint{0.670278in}{0.586684in}}{\pgfqpoint{7.179722in}{4.549983in}}%
\pgfusepath{clip}%
\pgfsetbuttcap%
\pgfsetroundjoin%
\pgfsetlinewidth{1.756562pt}%
\definecolor{currentstroke}{rgb}{0.298039,0.447059,0.690196}%
\pgfsetstrokecolor{currentstroke}%
\pgfsetdash{}{0pt}%
\pgfpathmoveto{\pgfqpoint{4.009100in}{3.029082in}}%
\pgfpathlineto{\pgfqpoint{4.009100in}{3.029082in}}%
\pgfusepath{stroke}%
\end{pgfscope}%
\begin{pgfscope}%
\pgfpathrectangle{\pgfqpoint{0.670278in}{0.586684in}}{\pgfqpoint{7.179722in}{4.549983in}}%
\pgfusepath{clip}%
\pgfsetbuttcap%
\pgfsetroundjoin%
\pgfsetlinewidth{1.756562pt}%
\definecolor{currentstroke}{rgb}{0.298039,0.447059,0.690196}%
\pgfsetstrokecolor{currentstroke}%
\pgfsetdash{}{0pt}%
\pgfpathmoveto{\pgfqpoint{4.511178in}{3.029082in}}%
\pgfpathlineto{\pgfqpoint{4.511178in}{3.029082in}}%
\pgfusepath{stroke}%
\end{pgfscope}%
\begin{pgfscope}%
\pgfpathrectangle{\pgfqpoint{0.670278in}{0.586684in}}{\pgfqpoint{7.179722in}{4.549983in}}%
\pgfusepath{clip}%
\pgfsetbuttcap%
\pgfsetroundjoin%
\pgfsetlinewidth{1.756562pt}%
\definecolor{currentstroke}{rgb}{0.298039,0.447059,0.690196}%
\pgfsetstrokecolor{currentstroke}%
\pgfsetdash{}{0pt}%
\pgfpathmoveto{\pgfqpoint{5.013257in}{3.032570in}}%
\pgfpathlineto{\pgfqpoint{5.013257in}{3.032570in}}%
\pgfusepath{stroke}%
\end{pgfscope}%
\begin{pgfscope}%
\pgfpathrectangle{\pgfqpoint{0.670278in}{0.586684in}}{\pgfqpoint{7.179722in}{4.549983in}}%
\pgfusepath{clip}%
\pgfsetbuttcap%
\pgfsetroundjoin%
\pgfsetlinewidth{1.756562pt}%
\definecolor{currentstroke}{rgb}{0.298039,0.447059,0.690196}%
\pgfsetstrokecolor{currentstroke}%
\pgfsetdash{}{0pt}%
\pgfpathmoveto{\pgfqpoint{5.515335in}{2.160659in}}%
\pgfpathlineto{\pgfqpoint{5.515335in}{4.790343in}}%
\pgfusepath{stroke}%
\end{pgfscope}%
\begin{pgfscope}%
\pgfpathrectangle{\pgfqpoint{0.670278in}{0.586684in}}{\pgfqpoint{7.179722in}{4.549983in}}%
\pgfusepath{clip}%
\pgfsetbuttcap%
\pgfsetroundjoin%
\pgfsetlinewidth{1.756562pt}%
\definecolor{currentstroke}{rgb}{0.298039,0.447059,0.690196}%
\pgfsetstrokecolor{currentstroke}%
\pgfsetdash{}{0pt}%
\pgfpathmoveto{\pgfqpoint{6.017414in}{3.032570in}}%
\pgfpathlineto{\pgfqpoint{6.017414in}{3.032570in}}%
\pgfusepath{stroke}%
\end{pgfscope}%
\begin{pgfscope}%
\pgfpathrectangle{\pgfqpoint{0.670278in}{0.586684in}}{\pgfqpoint{7.179722in}{4.549983in}}%
\pgfusepath{clip}%
\pgfsetbuttcap%
\pgfsetroundjoin%
\pgfsetlinewidth{1.756562pt}%
\definecolor{currentstroke}{rgb}{0.298039,0.447059,0.690196}%
\pgfsetstrokecolor{currentstroke}%
\pgfsetdash{}{0pt}%
\pgfpathmoveto{\pgfqpoint{6.519492in}{3.015132in}}%
\pgfpathlineto{\pgfqpoint{6.519492in}{3.039545in}}%
\pgfusepath{stroke}%
\end{pgfscope}%
\begin{pgfscope}%
\pgfpathrectangle{\pgfqpoint{0.670278in}{0.586684in}}{\pgfqpoint{7.179722in}{4.549983in}}%
\pgfusepath{clip}%
\pgfsetbuttcap%
\pgfsetroundjoin%
\pgfsetlinewidth{1.756562pt}%
\definecolor{currentstroke}{rgb}{0.298039,0.447059,0.690196}%
\pgfsetstrokecolor{currentstroke}%
\pgfsetdash{}{0pt}%
\pgfpathmoveto{\pgfqpoint{7.021571in}{3.032570in}}%
\pgfpathlineto{\pgfqpoint{7.021571in}{3.032570in}}%
\pgfusepath{stroke}%
\end{pgfscope}%
\begin{pgfscope}%
\pgfpathrectangle{\pgfqpoint{0.670278in}{0.586684in}}{\pgfqpoint{7.179722in}{4.549983in}}%
\pgfusepath{clip}%
\pgfsetbuttcap%
\pgfsetroundjoin%
\pgfsetlinewidth{1.756562pt}%
\definecolor{currentstroke}{rgb}{0.298039,0.447059,0.690196}%
\pgfsetstrokecolor{currentstroke}%
\pgfsetdash{}{0pt}%
\pgfpathmoveto{\pgfqpoint{7.523649in}{3.022107in}}%
\pgfpathlineto{\pgfqpoint{7.523649in}{3.039545in}}%
\pgfusepath{stroke}%
\end{pgfscope}%
\begin{pgfscope}%
\pgfpathrectangle{\pgfqpoint{0.670278in}{0.586684in}}{\pgfqpoint{7.179722in}{4.549983in}}%
\pgfusepath{clip}%
\pgfsetbuttcap%
\pgfsetroundjoin%
\definecolor{currentfill}{rgb}{0.298039,0.447059,0.690196}%
\pgfsetfillcolor{currentfill}%
\pgfsetlinewidth{1.003750pt}%
\definecolor{currentstroke}{rgb}{0.298039,0.447059,0.690196}%
\pgfsetstrokecolor{currentstroke}%
\pgfsetdash{}{0pt}%
\pgfsys@defobject{currentmarker}{\pgfqpoint{-0.069444in}{-0.000000in}}{\pgfqpoint{0.069444in}{0.000000in}}{%
\pgfpathmoveto{\pgfqpoint{0.069444in}{-0.000000in}}%
\pgfpathlineto{\pgfqpoint{-0.069444in}{0.000000in}}%
\pgfusepath{stroke,fill}%
}%
\begin{pgfscope}%
\pgfsys@transformshift{0.996629in}{3.004669in}%
\pgfsys@useobject{currentmarker}{}%
\end{pgfscope}%
\begin{pgfscope}%
\pgfsys@transformshift{1.498707in}{1.623561in}%
\pgfsys@useobject{currentmarker}{}%
\end{pgfscope}%
\begin{pgfscope}%
\pgfsys@transformshift{2.000786in}{3.018619in}%
\pgfsys@useobject{currentmarker}{}%
\end{pgfscope}%
\begin{pgfscope}%
\pgfsys@transformshift{2.502864in}{3.008156in}%
\pgfsys@useobject{currentmarker}{}%
\end{pgfscope}%
\begin{pgfscope}%
\pgfsys@transformshift{3.004943in}{3.022107in}%
\pgfsys@useobject{currentmarker}{}%
\end{pgfscope}%
\begin{pgfscope}%
\pgfsys@transformshift{3.507021in}{0.793501in}%
\pgfsys@useobject{currentmarker}{}%
\end{pgfscope}%
\begin{pgfscope}%
\pgfsys@transformshift{4.009100in}{3.029082in}%
\pgfsys@useobject{currentmarker}{}%
\end{pgfscope}%
\begin{pgfscope}%
\pgfsys@transformshift{4.511178in}{3.029082in}%
\pgfsys@useobject{currentmarker}{}%
\end{pgfscope}%
\begin{pgfscope}%
\pgfsys@transformshift{5.013257in}{3.032570in}%
\pgfsys@useobject{currentmarker}{}%
\end{pgfscope}%
\begin{pgfscope}%
\pgfsys@transformshift{5.515335in}{2.160659in}%
\pgfsys@useobject{currentmarker}{}%
\end{pgfscope}%
\begin{pgfscope}%
\pgfsys@transformshift{6.017414in}{3.032570in}%
\pgfsys@useobject{currentmarker}{}%
\end{pgfscope}%
\begin{pgfscope}%
\pgfsys@transformshift{6.519492in}{3.015132in}%
\pgfsys@useobject{currentmarker}{}%
\end{pgfscope}%
\begin{pgfscope}%
\pgfsys@transformshift{7.021571in}{3.032570in}%
\pgfsys@useobject{currentmarker}{}%
\end{pgfscope}%
\begin{pgfscope}%
\pgfsys@transformshift{7.523649in}{3.022107in}%
\pgfsys@useobject{currentmarker}{}%
\end{pgfscope}%
\end{pgfscope}%
\begin{pgfscope}%
\pgfpathrectangle{\pgfqpoint{0.670278in}{0.586684in}}{\pgfqpoint{7.179722in}{4.549983in}}%
\pgfusepath{clip}%
\pgfsetbuttcap%
\pgfsetroundjoin%
\definecolor{currentfill}{rgb}{0.298039,0.447059,0.690196}%
\pgfsetfillcolor{currentfill}%
\pgfsetlinewidth{1.003750pt}%
\definecolor{currentstroke}{rgb}{0.298039,0.447059,0.690196}%
\pgfsetstrokecolor{currentstroke}%
\pgfsetdash{}{0pt}%
\pgfsys@defobject{currentmarker}{\pgfqpoint{-0.069444in}{-0.000000in}}{\pgfqpoint{0.069444in}{0.000000in}}{%
\pgfpathmoveto{\pgfqpoint{0.069444in}{-0.000000in}}%
\pgfpathlineto{\pgfqpoint{-0.069444in}{0.000000in}}%
\pgfusepath{stroke,fill}%
}%
\begin{pgfscope}%
\pgfsys@transformshift{0.996629in}{3.011644in}%
\pgfsys@useobject{currentmarker}{}%
\end{pgfscope}%
\begin{pgfscope}%
\pgfsys@transformshift{1.498707in}{4.929849in}%
\pgfsys@useobject{currentmarker}{}%
\end{pgfscope}%
\begin{pgfscope}%
\pgfsys@transformshift{2.000786in}{3.025595in}%
\pgfsys@useobject{currentmarker}{}%
\end{pgfscope}%
\begin{pgfscope}%
\pgfsys@transformshift{2.502864in}{3.032570in}%
\pgfsys@useobject{currentmarker}{}%
\end{pgfscope}%
\begin{pgfscope}%
\pgfsys@transformshift{3.004943in}{3.022107in}%
\pgfsys@useobject{currentmarker}{}%
\end{pgfscope}%
\begin{pgfscope}%
\pgfsys@transformshift{3.507021in}{4.511332in}%
\pgfsys@useobject{currentmarker}{}%
\end{pgfscope}%
\begin{pgfscope}%
\pgfsys@transformshift{4.009100in}{3.029082in}%
\pgfsys@useobject{currentmarker}{}%
\end{pgfscope}%
\begin{pgfscope}%
\pgfsys@transformshift{4.511178in}{3.029082in}%
\pgfsys@useobject{currentmarker}{}%
\end{pgfscope}%
\begin{pgfscope}%
\pgfsys@transformshift{5.013257in}{3.032570in}%
\pgfsys@useobject{currentmarker}{}%
\end{pgfscope}%
\begin{pgfscope}%
\pgfsys@transformshift{5.515335in}{4.790343in}%
\pgfsys@useobject{currentmarker}{}%
\end{pgfscope}%
\begin{pgfscope}%
\pgfsys@transformshift{6.017414in}{3.032570in}%
\pgfsys@useobject{currentmarker}{}%
\end{pgfscope}%
\begin{pgfscope}%
\pgfsys@transformshift{6.519492in}{3.039545in}%
\pgfsys@useobject{currentmarker}{}%
\end{pgfscope}%
\begin{pgfscope}%
\pgfsys@transformshift{7.021571in}{3.032570in}%
\pgfsys@useobject{currentmarker}{}%
\end{pgfscope}%
\begin{pgfscope}%
\pgfsys@transformshift{7.523649in}{3.039545in}%
\pgfsys@useobject{currentmarker}{}%
\end{pgfscope}%
\end{pgfscope}%
\begin{pgfscope}%
\pgfpathrectangle{\pgfqpoint{0.670278in}{0.586684in}}{\pgfqpoint{7.179722in}{4.549983in}}%
\pgfusepath{clip}%
\pgfsetroundcap%
\pgfsetroundjoin%
\pgfsetlinewidth{1.756562pt}%
\definecolor{currentstroke}{rgb}{0.298039,0.447059,0.690196}%
\pgfsetstrokecolor{currentstroke}%
\pgfsetdash{}{0pt}%
\pgfpathmoveto{\pgfqpoint{0.996629in}{3.008156in}}%
\pgfpathlineto{\pgfqpoint{1.498707in}{3.276705in}}%
\pgfpathlineto{\pgfqpoint{2.000786in}{3.022107in}}%
\pgfpathlineto{\pgfqpoint{2.502864in}{3.020363in}}%
\pgfpathlineto{\pgfqpoint{3.004943in}{3.022107in}}%
\pgfpathlineto{\pgfqpoint{3.507021in}{3.046521in}}%
\pgfpathlineto{\pgfqpoint{4.009100in}{3.029082in}}%
\pgfpathlineto{\pgfqpoint{4.511178in}{3.029082in}}%
\pgfpathlineto{\pgfqpoint{5.013257in}{3.032570in}}%
\pgfpathlineto{\pgfqpoint{5.515335in}{3.131968in}}%
\pgfpathlineto{\pgfqpoint{6.017414in}{3.032570in}}%
\pgfpathlineto{\pgfqpoint{6.519492in}{3.027339in}}%
\pgfpathlineto{\pgfqpoint{7.021571in}{3.032570in}}%
\pgfpathlineto{\pgfqpoint{7.523649in}{3.030826in}}%
\pgfusepath{stroke}%
\end{pgfscope}%
\begin{pgfscope}%
\pgfsetrectcap%
\pgfsetmiterjoin%
\pgfsetlinewidth{0.000000pt}%
\definecolor{currentstroke}{rgb}{1.000000,1.000000,1.000000}%
\pgfsetstrokecolor{currentstroke}%
\pgfsetdash{}{0pt}%
\pgfpathmoveto{\pgfqpoint{0.670278in}{0.586684in}}%
\pgfpathlineto{\pgfqpoint{0.670278in}{5.136667in}}%
\pgfusepath{}%
\end{pgfscope}%
\begin{pgfscope}%
\pgfsetrectcap%
\pgfsetmiterjoin%
\pgfsetlinewidth{0.000000pt}%
\definecolor{currentstroke}{rgb}{1.000000,1.000000,1.000000}%
\pgfsetstrokecolor{currentstroke}%
\pgfsetdash{}{0pt}%
\pgfpathmoveto{\pgfqpoint{7.850000in}{0.586684in}}%
\pgfpathlineto{\pgfqpoint{7.850000in}{5.136667in}}%
\pgfusepath{}%
\end{pgfscope}%
\begin{pgfscope}%
\pgfsetrectcap%
\pgfsetmiterjoin%
\pgfsetlinewidth{0.000000pt}%
\definecolor{currentstroke}{rgb}{1.000000,1.000000,1.000000}%
\pgfsetstrokecolor{currentstroke}%
\pgfsetdash{}{0pt}%
\pgfpathmoveto{\pgfqpoint{0.670278in}{0.586684in}}%
\pgfpathlineto{\pgfqpoint{7.850000in}{0.586684in}}%
\pgfusepath{}%
\end{pgfscope}%
\begin{pgfscope}%
\pgfsetrectcap%
\pgfsetmiterjoin%
\pgfsetlinewidth{0.000000pt}%
\definecolor{currentstroke}{rgb}{1.000000,1.000000,1.000000}%
\pgfsetstrokecolor{currentstroke}%
\pgfsetdash{}{0pt}%
\pgfpathmoveto{\pgfqpoint{0.670278in}{5.136667in}}%
\pgfpathlineto{\pgfqpoint{7.850000in}{5.136667in}}%
\pgfusepath{}%
\end{pgfscope}%
\begin{pgfscope}%
\definecolor{textcolor}{rgb}{0.150000,0.150000,0.150000}%
\pgfsetstrokecolor{textcolor}%
\pgfsetfillcolor{textcolor}%
\pgftext[x=4.260139in,y=5.220000in,,base]{\color{textcolor}\rmfamily\fontsize{12.000000}{14.400000}\selectfont Voltage at Reference Frequencies}%
\end{pgfscope}%
\end{pgfpicture}%
\makeatother%
\endgroup%
}
      \caption{this is a test of the captions to se if  HOLDER}
      \label{fig:freq_dep}
  
    \end{figure}

    The measured voltage during the frequency sweep was mostly stable at a value of $16.8(5) ~\mu V$. However, figure \ref{fig:freq_dep} shows a large deviation at $f_n$ = 60, 180, and 300 Hz. This is a systematic error that arises from the electrical outlets that we get our power from. The power supplied to the LA is AC current at 60Hz and the deviations at 180 and 300Hz are the overtones of this base frequency.\cite{harmonics} They follow the pattern of being amplified every odd-multiple of the base frequency (eq. \ref{eq:fn}). The harmonics stem from the current being a sin wave which is generated from the odd terms in series expansion.
    
    \begin{gather}
      f_n = 60(2n+1); ~~{n = 0, 1, 2, \dots}
      \label{eq:fn} \\
      \sin(x) = \sum_{n=0}^{\infty} \left(-1\right)^n \frac{x^{2n +1}}{(2n+1)!} \label{eq:sin_taylor}
    \end{gather}
    

    \begin{center}
      \begin{tabular}{ |c|c|c| }
        \hline
        Configuration & Voltage ($\mu V$) & Resistance  ($m\Omega$) \\ 
        \hline
        \multicolumn{3}{|c|}{BSCCO} \\
        \hline
        a & 16.9(5) & 3.4(0)  \\ 
        \hline
        b & 12.2(1) & 2.4(2)  \\ 
        \hline
        c & 16.9(4) & 3.4(2)  \\ 
        \hline
        d & 11.9(9) & 2.4(2)  \\ 
        \hline
        \multicolumn{3}{|c|}{YBCO} \\
        \hline
        a & 13.8(5) & 3.0(9)  \\ 
        \hline
        b & 14.0(0) & 2.4(5)  \\ 
        \hline
        c & 12.8(5) & 3.1(3) \\ 
        \hline
        d & 15.2(7) & 2.4(7)  \\ 
        \hline
      \end{tabular}
    \end{center}
    
    \begin{center}
      \begin{tabular}{ |c|c|c| }
        \hline
        Pair & Ratio & f(R$_i$/ R$_j$) \\
        \hline
        \multicolumn{3}{|c|}{BSCCO} \\
        \hline
        a-b & 1.405 & 0.99 \\
        \hline
        c-b & 1.412 & 0.99 \\
        \hline
        c-d & 1.414 & 0.99 \\
        \hline
        a-d & 1.406 & 0.99 \\
        \hline
        \multicolumn{3}{|c|}{YBCO} \\
        \hline
        a-b & 1.259 & 1.00 \\
        \hline
        c-b & 1.252 & 1.00 \\
        \hline
        c-d & 1.269 & 0.99 \\
        \hline
        a-d & 1.277 & 0.99 \\
        \hline
      \end{tabular}
    \end{center}
    \begin{gather}
      \rho = \frac{\pi t}{ln2}\frac{R_a}{}
    \end{gather}


  \subsection*{Temperature Dependence}

    As a type-2 superconductor the superconducting phase transition for Bi-2223 is of second order and expected  at $T_c \approx 108~K$.~\cite{bscco_tc} However, our sample began it's transition to the superconducting phase at around 111 K (fig.\ref{fig:bscco_temp}) and ended its transition at 97.5 K. Additionally, rather than being a singular smooth sigmoidal-like curve the resistivity plot has a shelf at around 105 K. This deviation can be explained as largely being from two sources, disorder and multiple chemical phases. In this case disorder specifically refers to lattice displacements and low-density vacancies that result in intra-grain disorder. This is distinct from chemical phase differences which are examples of inter-grain disorder. 
    

    \begin{figure}[t]
      \resizebox{0.45\textwidth}{!}{%% Creator: Matplotlib, PGF backend
%%
%% To include the figure in your LaTeX document, write
%%   \input{<filename>.pgf}
%%
%% Make sure the required packages are loaded in your preamble
%%   \usepackage{pgf}
%%
%% Also ensure that all the required font packages are loaded; for instance,
%% the lmodern package is sometimes necessary when using math font.
%%   \usepackage{lmodern}
%%
%% Figures using additional raster images can only be included by \input if
%% they are in the same directory as the main LaTeX file. For loading figures
%% from other directories you can use the `import` package
%%   \usepackage{import}
%%
%% and then include the figures with
%%   \import{<path to file>}{<filename>.pgf}
%%
%% Matplotlib used the following preamble
%%   \usepackage{fontspec}
%%   \setmainfont{DejaVuSerif.ttf}[Path=\detokenize{/usr/share/matplotlib/mpl-data/fonts/ttf/}]
%%   \setsansfont{DejaVuSans.ttf}[Path=\detokenize{/usr/share/matplotlib/mpl-data/fonts/ttf/}]
%%   \setmonofont{DejaVuSansMono.ttf}[Path=\detokenize{/usr/share/matplotlib/mpl-data/fonts/ttf/}]
%%
\begingroup%
\makeatletter%
\begin{pgfpicture}%
\pgfpathrectangle{\pgfpointorigin}{\pgfqpoint{8.000000in}{5.500000in}}%
\pgfusepath{use as bounding box, clip}%
\begin{pgfscope}%
\pgfsetbuttcap%
\pgfsetmiterjoin%
\definecolor{currentfill}{rgb}{1.000000,1.000000,1.000000}%
\pgfsetfillcolor{currentfill}%
\pgfsetlinewidth{0.000000pt}%
\definecolor{currentstroke}{rgb}{1.000000,1.000000,1.000000}%
\pgfsetstrokecolor{currentstroke}%
\pgfsetdash{}{0pt}%
\pgfpathmoveto{\pgfqpoint{0.000000in}{0.000000in}}%
\pgfpathlineto{\pgfqpoint{8.000000in}{0.000000in}}%
\pgfpathlineto{\pgfqpoint{8.000000in}{5.500000in}}%
\pgfpathlineto{\pgfqpoint{0.000000in}{5.500000in}}%
\pgfpathlineto{\pgfqpoint{0.000000in}{0.000000in}}%
\pgfpathclose%
\pgfusepath{fill}%
\end{pgfscope}%
\begin{pgfscope}%
\pgfsetbuttcap%
\pgfsetmiterjoin%
\definecolor{currentfill}{rgb}{0.917647,0.917647,0.949020}%
\pgfsetfillcolor{currentfill}%
\pgfsetlinewidth{0.000000pt}%
\definecolor{currentstroke}{rgb}{0.000000,0.000000,0.000000}%
\pgfsetstrokecolor{currentstroke}%
\pgfsetstrokeopacity{0.000000}%
\pgfsetdash{}{0pt}%
\pgfpathmoveto{\pgfqpoint{1.243858in}{0.894194in}}%
\pgfpathlineto{\pgfqpoint{7.850000in}{0.894194in}}%
\pgfpathlineto{\pgfqpoint{7.850000in}{5.013411in}}%
\pgfpathlineto{\pgfqpoint{1.243858in}{5.013411in}}%
\pgfpathlineto{\pgfqpoint{1.243858in}{0.894194in}}%
\pgfpathclose%
\pgfusepath{fill}%
\end{pgfscope}%
\begin{pgfscope}%
\pgfpathrectangle{\pgfqpoint{1.243858in}{0.894194in}}{\pgfqpoint{6.606142in}{4.119217in}}%
\pgfusepath{clip}%
\pgfsetroundcap%
\pgfsetroundjoin%
\pgfsetlinewidth{1.003750pt}%
\definecolor{currentstroke}{rgb}{1.000000,1.000000,1.000000}%
\pgfsetstrokecolor{currentstroke}%
\pgfsetdash{}{0pt}%
\pgfpathmoveto{\pgfqpoint{1.544137in}{0.894194in}}%
\pgfpathlineto{\pgfqpoint{1.544137in}{5.013411in}}%
\pgfusepath{stroke}%
\end{pgfscope}%
\begin{pgfscope}%
\definecolor{textcolor}{rgb}{0.150000,0.150000,0.150000}%
\pgfsetstrokecolor{textcolor}%
\pgfsetfillcolor{textcolor}%
\pgftext[x=1.544137in,y=0.796972in,,top]{\color{textcolor}\rmfamily\fontsize{20.000000}{24.000000}\selectfont 80}%
\end{pgfscope}%
\begin{pgfscope}%
\pgfpathrectangle{\pgfqpoint{1.243858in}{0.894194in}}{\pgfqpoint{6.606142in}{4.119217in}}%
\pgfusepath{clip}%
\pgfsetroundcap%
\pgfsetroundjoin%
\pgfsetlinewidth{1.003750pt}%
\definecolor{currentstroke}{rgb}{1.000000,1.000000,1.000000}%
\pgfsetstrokecolor{currentstroke}%
\pgfsetdash{}{0pt}%
\pgfpathmoveto{\pgfqpoint{2.294835in}{0.894194in}}%
\pgfpathlineto{\pgfqpoint{2.294835in}{5.013411in}}%
\pgfusepath{stroke}%
\end{pgfscope}%
\begin{pgfscope}%
\definecolor{textcolor}{rgb}{0.150000,0.150000,0.150000}%
\pgfsetstrokecolor{textcolor}%
\pgfsetfillcolor{textcolor}%
\pgftext[x=2.294835in,y=0.796972in,,top]{\color{textcolor}\rmfamily\fontsize{20.000000}{24.000000}\selectfont 90}%
\end{pgfscope}%
\begin{pgfscope}%
\pgfpathrectangle{\pgfqpoint{1.243858in}{0.894194in}}{\pgfqpoint{6.606142in}{4.119217in}}%
\pgfusepath{clip}%
\pgfsetroundcap%
\pgfsetroundjoin%
\pgfsetlinewidth{1.003750pt}%
\definecolor{currentstroke}{rgb}{1.000000,1.000000,1.000000}%
\pgfsetstrokecolor{currentstroke}%
\pgfsetdash{}{0pt}%
\pgfpathmoveto{\pgfqpoint{3.045533in}{0.894194in}}%
\pgfpathlineto{\pgfqpoint{3.045533in}{5.013411in}}%
\pgfusepath{stroke}%
\end{pgfscope}%
\begin{pgfscope}%
\definecolor{textcolor}{rgb}{0.150000,0.150000,0.150000}%
\pgfsetstrokecolor{textcolor}%
\pgfsetfillcolor{textcolor}%
\pgftext[x=3.045533in,y=0.796972in,,top]{\color{textcolor}\rmfamily\fontsize{20.000000}{24.000000}\selectfont 100}%
\end{pgfscope}%
\begin{pgfscope}%
\pgfpathrectangle{\pgfqpoint{1.243858in}{0.894194in}}{\pgfqpoint{6.606142in}{4.119217in}}%
\pgfusepath{clip}%
\pgfsetroundcap%
\pgfsetroundjoin%
\pgfsetlinewidth{1.003750pt}%
\definecolor{currentstroke}{rgb}{1.000000,1.000000,1.000000}%
\pgfsetstrokecolor{currentstroke}%
\pgfsetdash{}{0pt}%
\pgfpathmoveto{\pgfqpoint{3.796231in}{0.894194in}}%
\pgfpathlineto{\pgfqpoint{3.796231in}{5.013411in}}%
\pgfusepath{stroke}%
\end{pgfscope}%
\begin{pgfscope}%
\definecolor{textcolor}{rgb}{0.150000,0.150000,0.150000}%
\pgfsetstrokecolor{textcolor}%
\pgfsetfillcolor{textcolor}%
\pgftext[x=3.796231in,y=0.796972in,,top]{\color{textcolor}\rmfamily\fontsize{20.000000}{24.000000}\selectfont 110}%
\end{pgfscope}%
\begin{pgfscope}%
\pgfpathrectangle{\pgfqpoint{1.243858in}{0.894194in}}{\pgfqpoint{6.606142in}{4.119217in}}%
\pgfusepath{clip}%
\pgfsetroundcap%
\pgfsetroundjoin%
\pgfsetlinewidth{1.003750pt}%
\definecolor{currentstroke}{rgb}{1.000000,1.000000,1.000000}%
\pgfsetstrokecolor{currentstroke}%
\pgfsetdash{}{0pt}%
\pgfpathmoveto{\pgfqpoint{4.546929in}{0.894194in}}%
\pgfpathlineto{\pgfqpoint{4.546929in}{5.013411in}}%
\pgfusepath{stroke}%
\end{pgfscope}%
\begin{pgfscope}%
\definecolor{textcolor}{rgb}{0.150000,0.150000,0.150000}%
\pgfsetstrokecolor{textcolor}%
\pgfsetfillcolor{textcolor}%
\pgftext[x=4.546929in,y=0.796972in,,top]{\color{textcolor}\rmfamily\fontsize{20.000000}{24.000000}\selectfont 120}%
\end{pgfscope}%
\begin{pgfscope}%
\pgfpathrectangle{\pgfqpoint{1.243858in}{0.894194in}}{\pgfqpoint{6.606142in}{4.119217in}}%
\pgfusepath{clip}%
\pgfsetroundcap%
\pgfsetroundjoin%
\pgfsetlinewidth{1.003750pt}%
\definecolor{currentstroke}{rgb}{1.000000,1.000000,1.000000}%
\pgfsetstrokecolor{currentstroke}%
\pgfsetdash{}{0pt}%
\pgfpathmoveto{\pgfqpoint{5.297627in}{0.894194in}}%
\pgfpathlineto{\pgfqpoint{5.297627in}{5.013411in}}%
\pgfusepath{stroke}%
\end{pgfscope}%
\begin{pgfscope}%
\definecolor{textcolor}{rgb}{0.150000,0.150000,0.150000}%
\pgfsetstrokecolor{textcolor}%
\pgfsetfillcolor{textcolor}%
\pgftext[x=5.297627in,y=0.796972in,,top]{\color{textcolor}\rmfamily\fontsize{20.000000}{24.000000}\selectfont 130}%
\end{pgfscope}%
\begin{pgfscope}%
\pgfpathrectangle{\pgfqpoint{1.243858in}{0.894194in}}{\pgfqpoint{6.606142in}{4.119217in}}%
\pgfusepath{clip}%
\pgfsetroundcap%
\pgfsetroundjoin%
\pgfsetlinewidth{1.003750pt}%
\definecolor{currentstroke}{rgb}{1.000000,1.000000,1.000000}%
\pgfsetstrokecolor{currentstroke}%
\pgfsetdash{}{0pt}%
\pgfpathmoveto{\pgfqpoint{6.048325in}{0.894194in}}%
\pgfpathlineto{\pgfqpoint{6.048325in}{5.013411in}}%
\pgfusepath{stroke}%
\end{pgfscope}%
\begin{pgfscope}%
\definecolor{textcolor}{rgb}{0.150000,0.150000,0.150000}%
\pgfsetstrokecolor{textcolor}%
\pgfsetfillcolor{textcolor}%
\pgftext[x=6.048325in,y=0.796972in,,top]{\color{textcolor}\rmfamily\fontsize{20.000000}{24.000000}\selectfont 140}%
\end{pgfscope}%
\begin{pgfscope}%
\pgfpathrectangle{\pgfqpoint{1.243858in}{0.894194in}}{\pgfqpoint{6.606142in}{4.119217in}}%
\pgfusepath{clip}%
\pgfsetroundcap%
\pgfsetroundjoin%
\pgfsetlinewidth{1.003750pt}%
\definecolor{currentstroke}{rgb}{1.000000,1.000000,1.000000}%
\pgfsetstrokecolor{currentstroke}%
\pgfsetdash{}{0pt}%
\pgfpathmoveto{\pgfqpoint{6.799023in}{0.894194in}}%
\pgfpathlineto{\pgfqpoint{6.799023in}{5.013411in}}%
\pgfusepath{stroke}%
\end{pgfscope}%
\begin{pgfscope}%
\definecolor{textcolor}{rgb}{0.150000,0.150000,0.150000}%
\pgfsetstrokecolor{textcolor}%
\pgfsetfillcolor{textcolor}%
\pgftext[x=6.799023in,y=0.796972in,,top]{\color{textcolor}\rmfamily\fontsize{20.000000}{24.000000}\selectfont 150}%
\end{pgfscope}%
\begin{pgfscope}%
\pgfpathrectangle{\pgfqpoint{1.243858in}{0.894194in}}{\pgfqpoint{6.606142in}{4.119217in}}%
\pgfusepath{clip}%
\pgfsetroundcap%
\pgfsetroundjoin%
\pgfsetlinewidth{1.003750pt}%
\definecolor{currentstroke}{rgb}{1.000000,1.000000,1.000000}%
\pgfsetstrokecolor{currentstroke}%
\pgfsetdash{}{0pt}%
\pgfpathmoveto{\pgfqpoint{7.549721in}{0.894194in}}%
\pgfpathlineto{\pgfqpoint{7.549721in}{5.013411in}}%
\pgfusepath{stroke}%
\end{pgfscope}%
\begin{pgfscope}%
\definecolor{textcolor}{rgb}{0.150000,0.150000,0.150000}%
\pgfsetstrokecolor{textcolor}%
\pgfsetfillcolor{textcolor}%
\pgftext[x=7.549721in,y=0.796972in,,top]{\color{textcolor}\rmfamily\fontsize{20.000000}{24.000000}\selectfont 160}%
\end{pgfscope}%
\begin{pgfscope}%
\definecolor{textcolor}{rgb}{0.150000,0.150000,0.150000}%
\pgfsetstrokecolor{textcolor}%
\pgfsetfillcolor{textcolor}%
\pgftext[x=4.546929in,y=0.472591in,,top]{\color{textcolor}\rmfamily\fontsize{24.000000}{28.800000}\selectfont Temperature (K)}%
\end{pgfscope}%
\begin{pgfscope}%
\pgfpathrectangle{\pgfqpoint{1.243858in}{0.894194in}}{\pgfqpoint{6.606142in}{4.119217in}}%
\pgfusepath{clip}%
\pgfsetroundcap%
\pgfsetroundjoin%
\pgfsetlinewidth{1.003750pt}%
\definecolor{currentstroke}{rgb}{1.000000,1.000000,1.000000}%
\pgfsetstrokecolor{currentstroke}%
\pgfsetdash{}{0pt}%
\pgfpathmoveto{\pgfqpoint{1.243858in}{1.017419in}}%
\pgfpathlineto{\pgfqpoint{7.850000in}{1.017419in}}%
\pgfusepath{stroke}%
\end{pgfscope}%
\begin{pgfscope}%
\definecolor{textcolor}{rgb}{0.150000,0.150000,0.150000}%
\pgfsetstrokecolor{textcolor}%
\pgfsetfillcolor{textcolor}%
\pgftext[x=0.704877in, y=0.911896in, left, base]{\color{textcolor}\rmfamily\fontsize{20.000000}{24.000000}\selectfont 0.0}%
\end{pgfscope}%
\begin{pgfscope}%
\pgfpathrectangle{\pgfqpoint{1.243858in}{0.894194in}}{\pgfqpoint{6.606142in}{4.119217in}}%
\pgfusepath{clip}%
\pgfsetroundcap%
\pgfsetroundjoin%
\pgfsetlinewidth{1.003750pt}%
\definecolor{currentstroke}{rgb}{1.000000,1.000000,1.000000}%
\pgfsetstrokecolor{currentstroke}%
\pgfsetdash{}{0pt}%
\pgfpathmoveto{\pgfqpoint{1.243858in}{1.647931in}}%
\pgfpathlineto{\pgfqpoint{7.850000in}{1.647931in}}%
\pgfusepath{stroke}%
\end{pgfscope}%
\begin{pgfscope}%
\definecolor{textcolor}{rgb}{0.150000,0.150000,0.150000}%
\pgfsetstrokecolor{textcolor}%
\pgfsetfillcolor{textcolor}%
\pgftext[x=0.704877in, y=1.542408in, left, base]{\color{textcolor}\rmfamily\fontsize{20.000000}{24.000000}\selectfont 2.5}%
\end{pgfscope}%
\begin{pgfscope}%
\pgfpathrectangle{\pgfqpoint{1.243858in}{0.894194in}}{\pgfqpoint{6.606142in}{4.119217in}}%
\pgfusepath{clip}%
\pgfsetroundcap%
\pgfsetroundjoin%
\pgfsetlinewidth{1.003750pt}%
\definecolor{currentstroke}{rgb}{1.000000,1.000000,1.000000}%
\pgfsetstrokecolor{currentstroke}%
\pgfsetdash{}{0pt}%
\pgfpathmoveto{\pgfqpoint{1.243858in}{2.278444in}}%
\pgfpathlineto{\pgfqpoint{7.850000in}{2.278444in}}%
\pgfusepath{stroke}%
\end{pgfscope}%
\begin{pgfscope}%
\definecolor{textcolor}{rgb}{0.150000,0.150000,0.150000}%
\pgfsetstrokecolor{textcolor}%
\pgfsetfillcolor{textcolor}%
\pgftext[x=0.704877in, y=2.172921in, left, base]{\color{textcolor}\rmfamily\fontsize{20.000000}{24.000000}\selectfont 5.0}%
\end{pgfscope}%
\begin{pgfscope}%
\pgfpathrectangle{\pgfqpoint{1.243858in}{0.894194in}}{\pgfqpoint{6.606142in}{4.119217in}}%
\pgfusepath{clip}%
\pgfsetroundcap%
\pgfsetroundjoin%
\pgfsetlinewidth{1.003750pt}%
\definecolor{currentstroke}{rgb}{1.000000,1.000000,1.000000}%
\pgfsetstrokecolor{currentstroke}%
\pgfsetdash{}{0pt}%
\pgfpathmoveto{\pgfqpoint{1.243858in}{2.908956in}}%
\pgfpathlineto{\pgfqpoint{7.850000in}{2.908956in}}%
\pgfusepath{stroke}%
\end{pgfscope}%
\begin{pgfscope}%
\definecolor{textcolor}{rgb}{0.150000,0.150000,0.150000}%
\pgfsetstrokecolor{textcolor}%
\pgfsetfillcolor{textcolor}%
\pgftext[x=0.704877in, y=2.803433in, left, base]{\color{textcolor}\rmfamily\fontsize{20.000000}{24.000000}\selectfont 7.5}%
\end{pgfscope}%
\begin{pgfscope}%
\pgfpathrectangle{\pgfqpoint{1.243858in}{0.894194in}}{\pgfqpoint{6.606142in}{4.119217in}}%
\pgfusepath{clip}%
\pgfsetroundcap%
\pgfsetroundjoin%
\pgfsetlinewidth{1.003750pt}%
\definecolor{currentstroke}{rgb}{1.000000,1.000000,1.000000}%
\pgfsetstrokecolor{currentstroke}%
\pgfsetdash{}{0pt}%
\pgfpathmoveto{\pgfqpoint{1.243858in}{3.539469in}}%
\pgfpathlineto{\pgfqpoint{7.850000in}{3.539469in}}%
\pgfusepath{stroke}%
\end{pgfscope}%
\begin{pgfscope}%
\definecolor{textcolor}{rgb}{0.150000,0.150000,0.150000}%
\pgfsetstrokecolor{textcolor}%
\pgfsetfillcolor{textcolor}%
\pgftext[x=0.528147in, y=3.433946in, left, base]{\color{textcolor}\rmfamily\fontsize{20.000000}{24.000000}\selectfont 10.0}%
\end{pgfscope}%
\begin{pgfscope}%
\pgfpathrectangle{\pgfqpoint{1.243858in}{0.894194in}}{\pgfqpoint{6.606142in}{4.119217in}}%
\pgfusepath{clip}%
\pgfsetroundcap%
\pgfsetroundjoin%
\pgfsetlinewidth{1.003750pt}%
\definecolor{currentstroke}{rgb}{1.000000,1.000000,1.000000}%
\pgfsetstrokecolor{currentstroke}%
\pgfsetdash{}{0pt}%
\pgfpathmoveto{\pgfqpoint{1.243858in}{4.169981in}}%
\pgfpathlineto{\pgfqpoint{7.850000in}{4.169981in}}%
\pgfusepath{stroke}%
\end{pgfscope}%
\begin{pgfscope}%
\definecolor{textcolor}{rgb}{0.150000,0.150000,0.150000}%
\pgfsetstrokecolor{textcolor}%
\pgfsetfillcolor{textcolor}%
\pgftext[x=0.528147in, y=4.064458in, left, base]{\color{textcolor}\rmfamily\fontsize{20.000000}{24.000000}\selectfont 12.5}%
\end{pgfscope}%
\begin{pgfscope}%
\pgfpathrectangle{\pgfqpoint{1.243858in}{0.894194in}}{\pgfqpoint{6.606142in}{4.119217in}}%
\pgfusepath{clip}%
\pgfsetroundcap%
\pgfsetroundjoin%
\pgfsetlinewidth{1.003750pt}%
\definecolor{currentstroke}{rgb}{1.000000,1.000000,1.000000}%
\pgfsetstrokecolor{currentstroke}%
\pgfsetdash{}{0pt}%
\pgfpathmoveto{\pgfqpoint{1.243858in}{4.800494in}}%
\pgfpathlineto{\pgfqpoint{7.850000in}{4.800494in}}%
\pgfusepath{stroke}%
\end{pgfscope}%
\begin{pgfscope}%
\definecolor{textcolor}{rgb}{0.150000,0.150000,0.150000}%
\pgfsetstrokecolor{textcolor}%
\pgfsetfillcolor{textcolor}%
\pgftext[x=0.528147in, y=4.694971in, left, base]{\color{textcolor}\rmfamily\fontsize{20.000000}{24.000000}\selectfont 15.0}%
\end{pgfscope}%
\begin{pgfscope}%
\definecolor{textcolor}{rgb}{0.150000,0.150000,0.150000}%
\pgfsetstrokecolor{textcolor}%
\pgfsetfillcolor{textcolor}%
\pgftext[x=0.472591in,y=2.953803in,,bottom,rotate=90.000000]{\color{textcolor}\rmfamily\fontsize{24.000000}{28.800000}\selectfont \(\displaystyle \rho\) (\(\displaystyle \mu\Omega\cdot\) m)}%
\end{pgfscope}%
\begin{pgfscope}%
\pgfpathrectangle{\pgfqpoint{1.243858in}{0.894194in}}{\pgfqpoint{6.606142in}{4.119217in}}%
\pgfusepath{clip}%
\pgfsetbuttcap%
\pgfsetroundjoin%
\definecolor{currentfill}{rgb}{0.298039,0.447059,0.690196}%
\pgfsetfillcolor{currentfill}%
\pgfsetlinewidth{0.301125pt}%
\definecolor{currentstroke}{rgb}{0.298039,0.447059,0.690196}%
\pgfsetstrokecolor{currentstroke}%
\pgfsetdash{}{0pt}%
\pgfsys@defobject{currentmarker}{\pgfqpoint{-0.031056in}{-0.031056in}}{\pgfqpoint{0.031056in}{0.031056in}}{%
\pgfpathmoveto{\pgfqpoint{0.000000in}{-0.031056in}}%
\pgfpathcurveto{\pgfqpoint{0.008236in}{-0.031056in}}{\pgfqpoint{0.016136in}{-0.027784in}}{\pgfqpoint{0.021960in}{-0.021960in}}%
\pgfpathcurveto{\pgfqpoint{0.027784in}{-0.016136in}}{\pgfqpoint{0.031056in}{-0.008236in}}{\pgfqpoint{0.031056in}{0.000000in}}%
\pgfpathcurveto{\pgfqpoint{0.031056in}{0.008236in}}{\pgfqpoint{0.027784in}{0.016136in}}{\pgfqpoint{0.021960in}{0.021960in}}%
\pgfpathcurveto{\pgfqpoint{0.016136in}{0.027784in}}{\pgfqpoint{0.008236in}{0.031056in}}{\pgfqpoint{0.000000in}{0.031056in}}%
\pgfpathcurveto{\pgfqpoint{-0.008236in}{0.031056in}}{\pgfqpoint{-0.016136in}{0.027784in}}{\pgfqpoint{-0.021960in}{0.021960in}}%
\pgfpathcurveto{\pgfqpoint{-0.027784in}{0.016136in}}{\pgfqpoint{-0.031056in}{0.008236in}}{\pgfqpoint{-0.031056in}{0.000000in}}%
\pgfpathcurveto{\pgfqpoint{-0.031056in}{-0.008236in}}{\pgfqpoint{-0.027784in}{-0.016136in}}{\pgfqpoint{-0.021960in}{-0.021960in}}%
\pgfpathcurveto{\pgfqpoint{-0.016136in}{-0.027784in}}{\pgfqpoint{-0.008236in}{-0.031056in}}{\pgfqpoint{0.000000in}{-0.031056in}}%
\pgfpathlineto{\pgfqpoint{0.000000in}{-0.031056in}}%
\pgfpathclose%
\pgfusepath{stroke,fill}%
}%
\begin{pgfscope}%
\pgfsys@transformshift{1.544137in}{1.108866in}%
\pgfsys@useobject{currentmarker}{}%
\end{pgfscope}%
\begin{pgfscope}%
\pgfsys@transformshift{1.919486in}{1.108866in}%
\pgfsys@useobject{currentmarker}{}%
\end{pgfscope}%
\begin{pgfscope}%
\pgfsys@transformshift{2.294835in}{1.104293in}%
\pgfsys@useobject{currentmarker}{}%
\end{pgfscope}%
\begin{pgfscope}%
\pgfsys@transformshift{2.670184in}{1.095149in}%
\pgfsys@useobject{currentmarker}{}%
\end{pgfscope}%
\begin{pgfscope}%
\pgfsys@transformshift{2.857859in}{1.207171in}%
\pgfsys@useobject{currentmarker}{}%
\end{pgfscope}%
\begin{pgfscope}%
\pgfsys@transformshift{2.895394in}{1.271183in}%
\pgfsys@useobject{currentmarker}{}%
\end{pgfscope}%
\begin{pgfscope}%
\pgfsys@transformshift{2.932929in}{1.339769in}%
\pgfsys@useobject{currentmarker}{}%
\end{pgfscope}%
\begin{pgfscope}%
\pgfsys@transformshift{2.970464in}{1.417498in}%
\pgfsys@useobject{currentmarker}{}%
\end{pgfscope}%
\begin{pgfscope}%
\pgfsys@transformshift{3.009500in}{1.522662in}%
\pgfsys@useobject{currentmarker}{}%
\end{pgfscope}%
\begin{pgfscope}%
\pgfsys@transformshift{3.045533in}{1.604964in}%
\pgfsys@useobject{currentmarker}{}%
\end{pgfscope}%
\begin{pgfscope}%
\pgfsys@transformshift{3.083068in}{1.691838in}%
\pgfsys@useobject{currentmarker}{}%
\end{pgfscope}%
\begin{pgfscope}%
\pgfsys@transformshift{3.120603in}{1.771854in}%
\pgfsys@useobject{currentmarker}{}%
\end{pgfscope}%
\begin{pgfscope}%
\pgfsys@transformshift{3.158138in}{1.847298in}%
\pgfsys@useobject{currentmarker}{}%
\end{pgfscope}%
\begin{pgfscope}%
\pgfsys@transformshift{3.195673in}{1.909024in}%
\pgfsys@useobject{currentmarker}{}%
\end{pgfscope}%
\begin{pgfscope}%
\pgfsys@transformshift{3.233208in}{1.973037in}%
\pgfsys@useobject{currentmarker}{}%
\end{pgfscope}%
\begin{pgfscope}%
\pgfsys@transformshift{3.270743in}{2.014188in}%
\pgfsys@useobject{currentmarker}{}%
\end{pgfscope}%
\begin{pgfscope}%
\pgfsys@transformshift{3.308278in}{2.062198in}%
\pgfsys@useobject{currentmarker}{}%
\end{pgfscope}%
\begin{pgfscope}%
\pgfsys@transformshift{3.345813in}{2.098776in}%
\pgfsys@useobject{currentmarker}{}%
\end{pgfscope}%
\begin{pgfscope}%
\pgfsys@transformshift{3.383347in}{2.130783in}%
\pgfsys@useobject{currentmarker}{}%
\end{pgfscope}%
\begin{pgfscope}%
\pgfsys@transformshift{3.420882in}{2.169647in}%
\pgfsys@useobject{currentmarker}{}%
\end{pgfscope}%
\begin{pgfscope}%
\pgfsys@transformshift{3.458417in}{2.201654in}%
\pgfsys@useobject{currentmarker}{}%
\end{pgfscope}%
\begin{pgfscope}%
\pgfsys@transformshift{3.495952in}{2.226802in}%
\pgfsys@useobject{currentmarker}{}%
\end{pgfscope}%
\begin{pgfscope}%
\pgfsys@transformshift{3.533487in}{2.254236in}%
\pgfsys@useobject{currentmarker}{}%
\end{pgfscope}%
\begin{pgfscope}%
\pgfsys@transformshift{3.571022in}{2.306817in}%
\pgfsys@useobject{currentmarker}{}%
\end{pgfscope}%
\begin{pgfscope}%
\pgfsys@transformshift{3.608557in}{2.366258in}%
\pgfsys@useobject{currentmarker}{}%
\end{pgfscope}%
\begin{pgfscope}%
\pgfsys@transformshift{3.646092in}{2.441701in}%
\pgfsys@useobject{currentmarker}{}%
\end{pgfscope}%
\begin{pgfscope}%
\pgfsys@transformshift{3.683627in}{2.585730in}%
\pgfsys@useobject{currentmarker}{}%
\end{pgfscope}%
\begin{pgfscope}%
\pgfsys@transformshift{3.721161in}{2.757192in}%
\pgfsys@useobject{currentmarker}{}%
\end{pgfscope}%
\begin{pgfscope}%
\pgfsys@transformshift{3.758696in}{2.940086in}%
\pgfsys@useobject{currentmarker}{}%
\end{pgfscope}%
\begin{pgfscope}%
\pgfsys@transformshift{3.796231in}{3.097831in}%
\pgfsys@useobject{currentmarker}{}%
\end{pgfscope}%
\begin{pgfscope}%
\pgfsys@transformshift{3.833766in}{3.230429in}%
\pgfsys@useobject{currentmarker}{}%
\end{pgfscope}%
\begin{pgfscope}%
\pgfsys@transformshift{3.871301in}{3.340165in}%
\pgfsys@useobject{currentmarker}{}%
\end{pgfscope}%
\begin{pgfscope}%
\pgfsys@transformshift{3.908836in}{3.417895in}%
\pgfsys@useobject{currentmarker}{}%
\end{pgfscope}%
\begin{pgfscope}%
\pgfsys@transformshift{3.946371in}{3.509342in}%
\pgfsys@useobject{currentmarker}{}%
\end{pgfscope}%
\begin{pgfscope}%
\pgfsys@transformshift{3.983906in}{3.573354in}%
\pgfsys@useobject{currentmarker}{}%
\end{pgfscope}%
\begin{pgfscope}%
\pgfsys@transformshift{4.171580in}{3.795113in}%
\pgfsys@useobject{currentmarker}{}%
\end{pgfscope}%
\begin{pgfscope}%
\pgfsys@transformshift{4.546929in}{4.030588in}%
\pgfsys@useobject{currentmarker}{}%
\end{pgfscope}%
\begin{pgfscope}%
\pgfsys@transformshift{4.922278in}{4.172330in}%
\pgfsys@useobject{currentmarker}{}%
\end{pgfscope}%
\begin{pgfscope}%
\pgfsys@transformshift{5.297627in}{4.291211in}%
\pgfsys@useobject{currentmarker}{}%
\end{pgfscope}%
\begin{pgfscope}%
\pgfsys@transformshift{6.048325in}{4.487821in}%
\pgfsys@useobject{currentmarker}{}%
\end{pgfscope}%
\begin{pgfscope}%
\pgfsys@transformshift{6.799023in}{4.666143in}%
\pgfsys@useobject{currentmarker}{}%
\end{pgfscope}%
\begin{pgfscope}%
\pgfsys@transformshift{7.549721in}{4.823888in}%
\pgfsys@useobject{currentmarker}{}%
\end{pgfscope}%
\end{pgfscope}%
\begin{pgfscope}%
\pgfpathrectangle{\pgfqpoint{1.243858in}{0.894194in}}{\pgfqpoint{6.606142in}{4.119217in}}%
\pgfusepath{clip}%
\pgfsetbuttcap%
\pgfsetroundjoin%
\pgfsetlinewidth{1.756562pt}%
\definecolor{currentstroke}{rgb}{0.298039,0.447059,0.690196}%
\pgfsetstrokecolor{currentstroke}%
\pgfsetdash{}{0pt}%
\pgfpathmoveto{\pgfqpoint{1.544137in}{1.108866in}}%
\pgfpathlineto{\pgfqpoint{1.544137in}{1.108866in}}%
\pgfusepath{stroke}%
\end{pgfscope}%
\begin{pgfscope}%
\pgfpathrectangle{\pgfqpoint{1.243858in}{0.894194in}}{\pgfqpoint{6.606142in}{4.119217in}}%
\pgfusepath{clip}%
\pgfsetbuttcap%
\pgfsetroundjoin%
\pgfsetlinewidth{1.756562pt}%
\definecolor{currentstroke}{rgb}{0.298039,0.447059,0.690196}%
\pgfsetstrokecolor{currentstroke}%
\pgfsetdash{}{0pt}%
\pgfpathmoveto{\pgfqpoint{1.919486in}{1.108866in}}%
\pgfpathlineto{\pgfqpoint{1.919486in}{1.108866in}}%
\pgfusepath{stroke}%
\end{pgfscope}%
\begin{pgfscope}%
\pgfpathrectangle{\pgfqpoint{1.243858in}{0.894194in}}{\pgfqpoint{6.606142in}{4.119217in}}%
\pgfusepath{clip}%
\pgfsetbuttcap%
\pgfsetroundjoin%
\pgfsetlinewidth{1.756562pt}%
\definecolor{currentstroke}{rgb}{0.298039,0.447059,0.690196}%
\pgfsetstrokecolor{currentstroke}%
\pgfsetdash{}{0pt}%
\pgfpathmoveto{\pgfqpoint{2.294835in}{1.104293in}}%
\pgfpathlineto{\pgfqpoint{2.294835in}{1.104293in}}%
\pgfusepath{stroke}%
\end{pgfscope}%
\begin{pgfscope}%
\pgfpathrectangle{\pgfqpoint{1.243858in}{0.894194in}}{\pgfqpoint{6.606142in}{4.119217in}}%
\pgfusepath{clip}%
\pgfsetbuttcap%
\pgfsetroundjoin%
\pgfsetlinewidth{1.756562pt}%
\definecolor{currentstroke}{rgb}{0.298039,0.447059,0.690196}%
\pgfsetstrokecolor{currentstroke}%
\pgfsetdash{}{0pt}%
\pgfpathmoveto{\pgfqpoint{2.670184in}{1.081432in}}%
\pgfpathlineto{\pgfqpoint{2.670184in}{1.108866in}}%
\pgfusepath{stroke}%
\end{pgfscope}%
\begin{pgfscope}%
\pgfpathrectangle{\pgfqpoint{1.243858in}{0.894194in}}{\pgfqpoint{6.606142in}{4.119217in}}%
\pgfusepath{clip}%
\pgfsetbuttcap%
\pgfsetroundjoin%
\pgfsetlinewidth{1.756562pt}%
\definecolor{currentstroke}{rgb}{0.298039,0.447059,0.690196}%
\pgfsetstrokecolor{currentstroke}%
\pgfsetdash{}{0pt}%
\pgfpathmoveto{\pgfqpoint{2.857859in}{1.204885in}}%
\pgfpathlineto{\pgfqpoint{2.857859in}{1.209457in}}%
\pgfusepath{stroke}%
\end{pgfscope}%
\begin{pgfscope}%
\pgfpathrectangle{\pgfqpoint{1.243858in}{0.894194in}}{\pgfqpoint{6.606142in}{4.119217in}}%
\pgfusepath{clip}%
\pgfsetbuttcap%
\pgfsetroundjoin%
\pgfsetlinewidth{1.756562pt}%
\definecolor{currentstroke}{rgb}{0.298039,0.447059,0.690196}%
\pgfsetstrokecolor{currentstroke}%
\pgfsetdash{}{0pt}%
\pgfpathmoveto{\pgfqpoint{2.895394in}{1.264325in}}%
\pgfpathlineto{\pgfqpoint{2.895394in}{1.278042in}}%
\pgfusepath{stroke}%
\end{pgfscope}%
\begin{pgfscope}%
\pgfpathrectangle{\pgfqpoint{1.243858in}{0.894194in}}{\pgfqpoint{6.606142in}{4.119217in}}%
\pgfusepath{clip}%
\pgfsetbuttcap%
\pgfsetroundjoin%
\pgfsetlinewidth{1.756562pt}%
\definecolor{currentstroke}{rgb}{0.298039,0.447059,0.690196}%
\pgfsetstrokecolor{currentstroke}%
\pgfsetdash{}{0pt}%
\pgfpathmoveto{\pgfqpoint{2.932929in}{1.337482in}}%
\pgfpathlineto{\pgfqpoint{2.932929in}{1.342055in}}%
\pgfusepath{stroke}%
\end{pgfscope}%
\begin{pgfscope}%
\pgfpathrectangle{\pgfqpoint{1.243858in}{0.894194in}}{\pgfqpoint{6.606142in}{4.119217in}}%
\pgfusepath{clip}%
\pgfsetbuttcap%
\pgfsetroundjoin%
\pgfsetlinewidth{1.756562pt}%
\definecolor{currentstroke}{rgb}{0.298039,0.447059,0.690196}%
\pgfsetstrokecolor{currentstroke}%
\pgfsetdash{}{0pt}%
\pgfpathmoveto{\pgfqpoint{2.970464in}{1.415212in}}%
\pgfpathlineto{\pgfqpoint{2.970464in}{1.419784in}}%
\pgfusepath{stroke}%
\end{pgfscope}%
\begin{pgfscope}%
\pgfpathrectangle{\pgfqpoint{1.243858in}{0.894194in}}{\pgfqpoint{6.606142in}{4.119217in}}%
\pgfusepath{clip}%
\pgfsetbuttcap%
\pgfsetroundjoin%
\pgfsetlinewidth{1.756562pt}%
\definecolor{currentstroke}{rgb}{0.298039,0.447059,0.690196}%
\pgfsetstrokecolor{currentstroke}%
\pgfsetdash{}{0pt}%
\pgfpathmoveto{\pgfqpoint{3.009500in}{1.520376in}}%
\pgfpathlineto{\pgfqpoint{3.009500in}{1.524948in}}%
\pgfusepath{stroke}%
\end{pgfscope}%
\begin{pgfscope}%
\pgfpathrectangle{\pgfqpoint{1.243858in}{0.894194in}}{\pgfqpoint{6.606142in}{4.119217in}}%
\pgfusepath{clip}%
\pgfsetbuttcap%
\pgfsetroundjoin%
\pgfsetlinewidth{1.756562pt}%
\definecolor{currentstroke}{rgb}{0.298039,0.447059,0.690196}%
\pgfsetstrokecolor{currentstroke}%
\pgfsetdash{}{0pt}%
\pgfpathmoveto{\pgfqpoint{3.045533in}{1.593533in}}%
\pgfpathlineto{\pgfqpoint{3.045533in}{1.616395in}}%
\pgfusepath{stroke}%
\end{pgfscope}%
\begin{pgfscope}%
\pgfpathrectangle{\pgfqpoint{1.243858in}{0.894194in}}{\pgfqpoint{6.606142in}{4.119217in}}%
\pgfusepath{clip}%
\pgfsetbuttcap%
\pgfsetroundjoin%
\pgfsetlinewidth{1.756562pt}%
\definecolor{currentstroke}{rgb}{0.298039,0.447059,0.690196}%
\pgfsetstrokecolor{currentstroke}%
\pgfsetdash{}{0pt}%
\pgfpathmoveto{\pgfqpoint{3.083068in}{1.689552in}}%
\pgfpathlineto{\pgfqpoint{3.083068in}{1.694125in}}%
\pgfusepath{stroke}%
\end{pgfscope}%
\begin{pgfscope}%
\pgfpathrectangle{\pgfqpoint{1.243858in}{0.894194in}}{\pgfqpoint{6.606142in}{4.119217in}}%
\pgfusepath{clip}%
\pgfsetbuttcap%
\pgfsetroundjoin%
\pgfsetlinewidth{1.756562pt}%
\definecolor{currentstroke}{rgb}{0.298039,0.447059,0.690196}%
\pgfsetstrokecolor{currentstroke}%
\pgfsetdash{}{0pt}%
\pgfpathmoveto{\pgfqpoint{3.120603in}{1.771854in}}%
\pgfpathlineto{\pgfqpoint{3.120603in}{1.771854in}}%
\pgfusepath{stroke}%
\end{pgfscope}%
\begin{pgfscope}%
\pgfpathrectangle{\pgfqpoint{1.243858in}{0.894194in}}{\pgfqpoint{6.606142in}{4.119217in}}%
\pgfusepath{clip}%
\pgfsetbuttcap%
\pgfsetroundjoin%
\pgfsetlinewidth{1.756562pt}%
\definecolor{currentstroke}{rgb}{0.298039,0.447059,0.690196}%
\pgfsetstrokecolor{currentstroke}%
\pgfsetdash{}{0pt}%
\pgfpathmoveto{\pgfqpoint{3.158138in}{1.845012in}}%
\pgfpathlineto{\pgfqpoint{3.158138in}{1.849584in}}%
\pgfusepath{stroke}%
\end{pgfscope}%
\begin{pgfscope}%
\pgfpathrectangle{\pgfqpoint{1.243858in}{0.894194in}}{\pgfqpoint{6.606142in}{4.119217in}}%
\pgfusepath{clip}%
\pgfsetbuttcap%
\pgfsetroundjoin%
\pgfsetlinewidth{1.756562pt}%
\definecolor{currentstroke}{rgb}{0.298039,0.447059,0.690196}%
\pgfsetstrokecolor{currentstroke}%
\pgfsetdash{}{0pt}%
\pgfpathmoveto{\pgfqpoint{3.195673in}{1.904452in}}%
\pgfpathlineto{\pgfqpoint{3.195673in}{1.913597in}}%
\pgfusepath{stroke}%
\end{pgfscope}%
\begin{pgfscope}%
\pgfpathrectangle{\pgfqpoint{1.243858in}{0.894194in}}{\pgfqpoint{6.606142in}{4.119217in}}%
\pgfusepath{clip}%
\pgfsetbuttcap%
\pgfsetroundjoin%
\pgfsetlinewidth{1.756562pt}%
\definecolor{currentstroke}{rgb}{0.298039,0.447059,0.690196}%
\pgfsetstrokecolor{currentstroke}%
\pgfsetdash{}{0pt}%
\pgfpathmoveto{\pgfqpoint{3.233208in}{1.973037in}}%
\pgfpathlineto{\pgfqpoint{3.233208in}{1.973037in}}%
\pgfusepath{stroke}%
\end{pgfscope}%
\begin{pgfscope}%
\pgfpathrectangle{\pgfqpoint{1.243858in}{0.894194in}}{\pgfqpoint{6.606142in}{4.119217in}}%
\pgfusepath{clip}%
\pgfsetbuttcap%
\pgfsetroundjoin%
\pgfsetlinewidth{1.756562pt}%
\definecolor{currentstroke}{rgb}{0.298039,0.447059,0.690196}%
\pgfsetstrokecolor{currentstroke}%
\pgfsetdash{}{0pt}%
\pgfpathmoveto{\pgfqpoint{3.270743in}{2.014188in}}%
\pgfpathlineto{\pgfqpoint{3.270743in}{2.014188in}}%
\pgfusepath{stroke}%
\end{pgfscope}%
\begin{pgfscope}%
\pgfpathrectangle{\pgfqpoint{1.243858in}{0.894194in}}{\pgfqpoint{6.606142in}{4.119217in}}%
\pgfusepath{clip}%
\pgfsetbuttcap%
\pgfsetroundjoin%
\pgfsetlinewidth{1.756562pt}%
\definecolor{currentstroke}{rgb}{0.298039,0.447059,0.690196}%
\pgfsetstrokecolor{currentstroke}%
\pgfsetdash{}{0pt}%
\pgfpathmoveto{\pgfqpoint{3.308278in}{2.059911in}}%
\pgfpathlineto{\pgfqpoint{3.308278in}{2.064484in}}%
\pgfusepath{stroke}%
\end{pgfscope}%
\begin{pgfscope}%
\pgfpathrectangle{\pgfqpoint{1.243858in}{0.894194in}}{\pgfqpoint{6.606142in}{4.119217in}}%
\pgfusepath{clip}%
\pgfsetbuttcap%
\pgfsetroundjoin%
\pgfsetlinewidth{1.756562pt}%
\definecolor{currentstroke}{rgb}{0.298039,0.447059,0.690196}%
\pgfsetstrokecolor{currentstroke}%
\pgfsetdash{}{0pt}%
\pgfpathmoveto{\pgfqpoint{3.345813in}{2.096490in}}%
\pgfpathlineto{\pgfqpoint{3.345813in}{2.101062in}}%
\pgfusepath{stroke}%
\end{pgfscope}%
\begin{pgfscope}%
\pgfpathrectangle{\pgfqpoint{1.243858in}{0.894194in}}{\pgfqpoint{6.606142in}{4.119217in}}%
\pgfusepath{clip}%
\pgfsetbuttcap%
\pgfsetroundjoin%
\pgfsetlinewidth{1.756562pt}%
\definecolor{currentstroke}{rgb}{0.298039,0.447059,0.690196}%
\pgfsetstrokecolor{currentstroke}%
\pgfsetdash{}{0pt}%
\pgfpathmoveto{\pgfqpoint{3.383347in}{2.128496in}}%
\pgfpathlineto{\pgfqpoint{3.383347in}{2.133069in}}%
\pgfusepath{stroke}%
\end{pgfscope}%
\begin{pgfscope}%
\pgfpathrectangle{\pgfqpoint{1.243858in}{0.894194in}}{\pgfqpoint{6.606142in}{4.119217in}}%
\pgfusepath{clip}%
\pgfsetbuttcap%
\pgfsetroundjoin%
\pgfsetlinewidth{1.756562pt}%
\definecolor{currentstroke}{rgb}{0.298039,0.447059,0.690196}%
\pgfsetstrokecolor{currentstroke}%
\pgfsetdash{}{0pt}%
\pgfpathmoveto{\pgfqpoint{3.420882in}{2.169647in}}%
\pgfpathlineto{\pgfqpoint{3.420882in}{2.169647in}}%
\pgfusepath{stroke}%
\end{pgfscope}%
\begin{pgfscope}%
\pgfpathrectangle{\pgfqpoint{1.243858in}{0.894194in}}{\pgfqpoint{6.606142in}{4.119217in}}%
\pgfusepath{clip}%
\pgfsetbuttcap%
\pgfsetroundjoin%
\pgfsetlinewidth{1.756562pt}%
\definecolor{currentstroke}{rgb}{0.298039,0.447059,0.690196}%
\pgfsetstrokecolor{currentstroke}%
\pgfsetdash{}{0pt}%
\pgfpathmoveto{\pgfqpoint{3.458417in}{2.197081in}}%
\pgfpathlineto{\pgfqpoint{3.458417in}{2.206226in}}%
\pgfusepath{stroke}%
\end{pgfscope}%
\begin{pgfscope}%
\pgfpathrectangle{\pgfqpoint{1.243858in}{0.894194in}}{\pgfqpoint{6.606142in}{4.119217in}}%
\pgfusepath{clip}%
\pgfsetbuttcap%
\pgfsetroundjoin%
\pgfsetlinewidth{1.756562pt}%
\definecolor{currentstroke}{rgb}{0.298039,0.447059,0.690196}%
\pgfsetstrokecolor{currentstroke}%
\pgfsetdash{}{0pt}%
\pgfpathmoveto{\pgfqpoint{3.495952in}{2.219943in}}%
\pgfpathlineto{\pgfqpoint{3.495952in}{2.233660in}}%
\pgfusepath{stroke}%
\end{pgfscope}%
\begin{pgfscope}%
\pgfpathrectangle{\pgfqpoint{1.243858in}{0.894194in}}{\pgfqpoint{6.606142in}{4.119217in}}%
\pgfusepath{clip}%
\pgfsetbuttcap%
\pgfsetroundjoin%
\pgfsetlinewidth{1.756562pt}%
\definecolor{currentstroke}{rgb}{0.298039,0.447059,0.690196}%
\pgfsetstrokecolor{currentstroke}%
\pgfsetdash{}{0pt}%
\pgfpathmoveto{\pgfqpoint{3.533487in}{2.251949in}}%
\pgfpathlineto{\pgfqpoint{3.533487in}{2.256522in}}%
\pgfusepath{stroke}%
\end{pgfscope}%
\begin{pgfscope}%
\pgfpathrectangle{\pgfqpoint{1.243858in}{0.894194in}}{\pgfqpoint{6.606142in}{4.119217in}}%
\pgfusepath{clip}%
\pgfsetbuttcap%
\pgfsetroundjoin%
\pgfsetlinewidth{1.756562pt}%
\definecolor{currentstroke}{rgb}{0.298039,0.447059,0.690196}%
\pgfsetstrokecolor{currentstroke}%
\pgfsetdash{}{0pt}%
\pgfpathmoveto{\pgfqpoint{3.571022in}{2.306817in}}%
\pgfpathlineto{\pgfqpoint{3.571022in}{2.306817in}}%
\pgfusepath{stroke}%
\end{pgfscope}%
\begin{pgfscope}%
\pgfpathrectangle{\pgfqpoint{1.243858in}{0.894194in}}{\pgfqpoint{6.606142in}{4.119217in}}%
\pgfusepath{clip}%
\pgfsetbuttcap%
\pgfsetroundjoin%
\pgfsetlinewidth{1.756562pt}%
\definecolor{currentstroke}{rgb}{0.298039,0.447059,0.690196}%
\pgfsetstrokecolor{currentstroke}%
\pgfsetdash{}{0pt}%
\pgfpathmoveto{\pgfqpoint{3.608557in}{2.366258in}}%
\pgfpathlineto{\pgfqpoint{3.608557in}{2.366258in}}%
\pgfusepath{stroke}%
\end{pgfscope}%
\begin{pgfscope}%
\pgfpathrectangle{\pgfqpoint{1.243858in}{0.894194in}}{\pgfqpoint{6.606142in}{4.119217in}}%
\pgfusepath{clip}%
\pgfsetbuttcap%
\pgfsetroundjoin%
\pgfsetlinewidth{1.756562pt}%
\definecolor{currentstroke}{rgb}{0.298039,0.447059,0.690196}%
\pgfsetstrokecolor{currentstroke}%
\pgfsetdash{}{0pt}%
\pgfpathmoveto{\pgfqpoint{3.646092in}{2.434843in}}%
\pgfpathlineto{\pgfqpoint{3.646092in}{2.448560in}}%
\pgfusepath{stroke}%
\end{pgfscope}%
\begin{pgfscope}%
\pgfpathrectangle{\pgfqpoint{1.243858in}{0.894194in}}{\pgfqpoint{6.606142in}{4.119217in}}%
\pgfusepath{clip}%
\pgfsetbuttcap%
\pgfsetroundjoin%
\pgfsetlinewidth{1.756562pt}%
\definecolor{currentstroke}{rgb}{0.298039,0.447059,0.690196}%
\pgfsetstrokecolor{currentstroke}%
\pgfsetdash{}{0pt}%
\pgfpathmoveto{\pgfqpoint{3.683627in}{2.585730in}}%
\pgfpathlineto{\pgfqpoint{3.683627in}{2.585730in}}%
\pgfusepath{stroke}%
\end{pgfscope}%
\begin{pgfscope}%
\pgfpathrectangle{\pgfqpoint{1.243858in}{0.894194in}}{\pgfqpoint{6.606142in}{4.119217in}}%
\pgfusepath{clip}%
\pgfsetbuttcap%
\pgfsetroundjoin%
\pgfsetlinewidth{1.756562pt}%
\definecolor{currentstroke}{rgb}{0.298039,0.447059,0.690196}%
\pgfsetstrokecolor{currentstroke}%
\pgfsetdash{}{0pt}%
\pgfpathmoveto{\pgfqpoint{3.721161in}{2.754906in}}%
\pgfpathlineto{\pgfqpoint{3.721161in}{2.759479in}}%
\pgfusepath{stroke}%
\end{pgfscope}%
\begin{pgfscope}%
\pgfpathrectangle{\pgfqpoint{1.243858in}{0.894194in}}{\pgfqpoint{6.606142in}{4.119217in}}%
\pgfusepath{clip}%
\pgfsetbuttcap%
\pgfsetroundjoin%
\pgfsetlinewidth{1.756562pt}%
\definecolor{currentstroke}{rgb}{0.298039,0.447059,0.690196}%
\pgfsetstrokecolor{currentstroke}%
\pgfsetdash{}{0pt}%
\pgfpathmoveto{\pgfqpoint{3.758696in}{2.937800in}}%
\pgfpathlineto{\pgfqpoint{3.758696in}{2.942372in}}%
\pgfusepath{stroke}%
\end{pgfscope}%
\begin{pgfscope}%
\pgfpathrectangle{\pgfqpoint{1.243858in}{0.894194in}}{\pgfqpoint{6.606142in}{4.119217in}}%
\pgfusepath{clip}%
\pgfsetbuttcap%
\pgfsetroundjoin%
\pgfsetlinewidth{1.756562pt}%
\definecolor{currentstroke}{rgb}{0.298039,0.447059,0.690196}%
\pgfsetstrokecolor{currentstroke}%
\pgfsetdash{}{0pt}%
\pgfpathmoveto{\pgfqpoint{3.796231in}{3.093259in}}%
\pgfpathlineto{\pgfqpoint{3.796231in}{3.102404in}}%
\pgfusepath{stroke}%
\end{pgfscope}%
\begin{pgfscope}%
\pgfpathrectangle{\pgfqpoint{1.243858in}{0.894194in}}{\pgfqpoint{6.606142in}{4.119217in}}%
\pgfusepath{clip}%
\pgfsetbuttcap%
\pgfsetroundjoin%
\pgfsetlinewidth{1.756562pt}%
\definecolor{currentstroke}{rgb}{0.298039,0.447059,0.690196}%
\pgfsetstrokecolor{currentstroke}%
\pgfsetdash{}{0pt}%
\pgfpathmoveto{\pgfqpoint{3.833766in}{3.230429in}}%
\pgfpathlineto{\pgfqpoint{3.833766in}{3.230429in}}%
\pgfusepath{stroke}%
\end{pgfscope}%
\begin{pgfscope}%
\pgfpathrectangle{\pgfqpoint{1.243858in}{0.894194in}}{\pgfqpoint{6.606142in}{4.119217in}}%
\pgfusepath{clip}%
\pgfsetbuttcap%
\pgfsetroundjoin%
\pgfsetlinewidth{1.756562pt}%
\definecolor{currentstroke}{rgb}{0.298039,0.447059,0.690196}%
\pgfsetstrokecolor{currentstroke}%
\pgfsetdash{}{0pt}%
\pgfpathmoveto{\pgfqpoint{3.871301in}{3.340165in}}%
\pgfpathlineto{\pgfqpoint{3.871301in}{3.340165in}}%
\pgfusepath{stroke}%
\end{pgfscope}%
\begin{pgfscope}%
\pgfpathrectangle{\pgfqpoint{1.243858in}{0.894194in}}{\pgfqpoint{6.606142in}{4.119217in}}%
\pgfusepath{clip}%
\pgfsetbuttcap%
\pgfsetroundjoin%
\pgfsetlinewidth{1.756562pt}%
\definecolor{currentstroke}{rgb}{0.298039,0.447059,0.690196}%
\pgfsetstrokecolor{currentstroke}%
\pgfsetdash{}{0pt}%
\pgfpathmoveto{\pgfqpoint{3.908836in}{3.417895in}}%
\pgfpathlineto{\pgfqpoint{3.908836in}{3.417895in}}%
\pgfusepath{stroke}%
\end{pgfscope}%
\begin{pgfscope}%
\pgfpathrectangle{\pgfqpoint{1.243858in}{0.894194in}}{\pgfqpoint{6.606142in}{4.119217in}}%
\pgfusepath{clip}%
\pgfsetbuttcap%
\pgfsetroundjoin%
\pgfsetlinewidth{1.756562pt}%
\definecolor{currentstroke}{rgb}{0.298039,0.447059,0.690196}%
\pgfsetstrokecolor{currentstroke}%
\pgfsetdash{}{0pt}%
\pgfpathmoveto{\pgfqpoint{3.946371in}{3.509342in}}%
\pgfpathlineto{\pgfqpoint{3.946371in}{3.509342in}}%
\pgfusepath{stroke}%
\end{pgfscope}%
\begin{pgfscope}%
\pgfpathrectangle{\pgfqpoint{1.243858in}{0.894194in}}{\pgfqpoint{6.606142in}{4.119217in}}%
\pgfusepath{clip}%
\pgfsetbuttcap%
\pgfsetroundjoin%
\pgfsetlinewidth{1.756562pt}%
\definecolor{currentstroke}{rgb}{0.298039,0.447059,0.690196}%
\pgfsetstrokecolor{currentstroke}%
\pgfsetdash{}{0pt}%
\pgfpathmoveto{\pgfqpoint{3.983906in}{3.573354in}}%
\pgfpathlineto{\pgfqpoint{3.983906in}{3.573354in}}%
\pgfusepath{stroke}%
\end{pgfscope}%
\begin{pgfscope}%
\pgfpathrectangle{\pgfqpoint{1.243858in}{0.894194in}}{\pgfqpoint{6.606142in}{4.119217in}}%
\pgfusepath{clip}%
\pgfsetbuttcap%
\pgfsetroundjoin%
\pgfsetlinewidth{1.756562pt}%
\definecolor{currentstroke}{rgb}{0.298039,0.447059,0.690196}%
\pgfsetstrokecolor{currentstroke}%
\pgfsetdash{}{0pt}%
\pgfpathmoveto{\pgfqpoint{4.171580in}{3.792826in}}%
\pgfpathlineto{\pgfqpoint{4.171580in}{3.797399in}}%
\pgfusepath{stroke}%
\end{pgfscope}%
\begin{pgfscope}%
\pgfpathrectangle{\pgfqpoint{1.243858in}{0.894194in}}{\pgfqpoint{6.606142in}{4.119217in}}%
\pgfusepath{clip}%
\pgfsetbuttcap%
\pgfsetroundjoin%
\pgfsetlinewidth{1.756562pt}%
\definecolor{currentstroke}{rgb}{0.298039,0.447059,0.690196}%
\pgfsetstrokecolor{currentstroke}%
\pgfsetdash{}{0pt}%
\pgfpathmoveto{\pgfqpoint{4.546929in}{4.030588in}}%
\pgfpathlineto{\pgfqpoint{4.546929in}{4.030588in}}%
\pgfusepath{stroke}%
\end{pgfscope}%
\begin{pgfscope}%
\pgfpathrectangle{\pgfqpoint{1.243858in}{0.894194in}}{\pgfqpoint{6.606142in}{4.119217in}}%
\pgfusepath{clip}%
\pgfsetbuttcap%
\pgfsetroundjoin%
\pgfsetlinewidth{1.756562pt}%
\definecolor{currentstroke}{rgb}{0.298039,0.447059,0.690196}%
\pgfsetstrokecolor{currentstroke}%
\pgfsetdash{}{0pt}%
\pgfpathmoveto{\pgfqpoint{4.922278in}{4.172330in}}%
\pgfpathlineto{\pgfqpoint{4.922278in}{4.172330in}}%
\pgfusepath{stroke}%
\end{pgfscope}%
\begin{pgfscope}%
\pgfpathrectangle{\pgfqpoint{1.243858in}{0.894194in}}{\pgfqpoint{6.606142in}{4.119217in}}%
\pgfusepath{clip}%
\pgfsetbuttcap%
\pgfsetroundjoin%
\pgfsetlinewidth{1.756562pt}%
\definecolor{currentstroke}{rgb}{0.298039,0.447059,0.690196}%
\pgfsetstrokecolor{currentstroke}%
\pgfsetdash{}{0pt}%
\pgfpathmoveto{\pgfqpoint{5.297627in}{4.291211in}}%
\pgfpathlineto{\pgfqpoint{5.297627in}{4.291211in}}%
\pgfusepath{stroke}%
\end{pgfscope}%
\begin{pgfscope}%
\pgfpathrectangle{\pgfqpoint{1.243858in}{0.894194in}}{\pgfqpoint{6.606142in}{4.119217in}}%
\pgfusepath{clip}%
\pgfsetbuttcap%
\pgfsetroundjoin%
\pgfsetlinewidth{1.756562pt}%
\definecolor{currentstroke}{rgb}{0.298039,0.447059,0.690196}%
\pgfsetstrokecolor{currentstroke}%
\pgfsetdash{}{0pt}%
\pgfpathmoveto{\pgfqpoint{6.048325in}{4.487821in}}%
\pgfpathlineto{\pgfqpoint{6.048325in}{4.487821in}}%
\pgfusepath{stroke}%
\end{pgfscope}%
\begin{pgfscope}%
\pgfpathrectangle{\pgfqpoint{1.243858in}{0.894194in}}{\pgfqpoint{6.606142in}{4.119217in}}%
\pgfusepath{clip}%
\pgfsetbuttcap%
\pgfsetroundjoin%
\pgfsetlinewidth{1.756562pt}%
\definecolor{currentstroke}{rgb}{0.298039,0.447059,0.690196}%
\pgfsetstrokecolor{currentstroke}%
\pgfsetdash{}{0pt}%
\pgfpathmoveto{\pgfqpoint{6.799023in}{4.666143in}}%
\pgfpathlineto{\pgfqpoint{6.799023in}{4.666143in}}%
\pgfusepath{stroke}%
\end{pgfscope}%
\begin{pgfscope}%
\pgfpathrectangle{\pgfqpoint{1.243858in}{0.894194in}}{\pgfqpoint{6.606142in}{4.119217in}}%
\pgfusepath{clip}%
\pgfsetbuttcap%
\pgfsetroundjoin%
\pgfsetlinewidth{1.756562pt}%
\definecolor{currentstroke}{rgb}{0.298039,0.447059,0.690196}%
\pgfsetstrokecolor{currentstroke}%
\pgfsetdash{}{0pt}%
\pgfpathmoveto{\pgfqpoint{7.549721in}{4.821602in}}%
\pgfpathlineto{\pgfqpoint{7.549721in}{4.826174in}}%
\pgfusepath{stroke}%
\end{pgfscope}%
\begin{pgfscope}%
\pgfpathrectangle{\pgfqpoint{1.243858in}{0.894194in}}{\pgfqpoint{6.606142in}{4.119217in}}%
\pgfusepath{clip}%
\pgfsetbuttcap%
\pgfsetroundjoin%
\definecolor{currentfill}{rgb}{0.298039,0.447059,0.690196}%
\pgfsetfillcolor{currentfill}%
\pgfsetlinewidth{1.003750pt}%
\definecolor{currentstroke}{rgb}{0.298039,0.447059,0.690196}%
\pgfsetstrokecolor{currentstroke}%
\pgfsetdash{}{0pt}%
\pgfsys@defobject{currentmarker}{\pgfqpoint{-0.069444in}{-0.000000in}}{\pgfqpoint{0.069444in}{0.000000in}}{%
\pgfpathmoveto{\pgfqpoint{0.069444in}{-0.000000in}}%
\pgfpathlineto{\pgfqpoint{-0.069444in}{0.000000in}}%
\pgfusepath{stroke,fill}%
}%
\begin{pgfscope}%
\pgfsys@transformshift{1.544137in}{1.108866in}%
\pgfsys@useobject{currentmarker}{}%
\end{pgfscope}%
\begin{pgfscope}%
\pgfsys@transformshift{1.919486in}{1.108866in}%
\pgfsys@useobject{currentmarker}{}%
\end{pgfscope}%
\begin{pgfscope}%
\pgfsys@transformshift{2.294835in}{1.104293in}%
\pgfsys@useobject{currentmarker}{}%
\end{pgfscope}%
\begin{pgfscope}%
\pgfsys@transformshift{2.670184in}{1.081432in}%
\pgfsys@useobject{currentmarker}{}%
\end{pgfscope}%
\begin{pgfscope}%
\pgfsys@transformshift{2.857859in}{1.204885in}%
\pgfsys@useobject{currentmarker}{}%
\end{pgfscope}%
\begin{pgfscope}%
\pgfsys@transformshift{2.895394in}{1.264325in}%
\pgfsys@useobject{currentmarker}{}%
\end{pgfscope}%
\begin{pgfscope}%
\pgfsys@transformshift{2.932929in}{1.337482in}%
\pgfsys@useobject{currentmarker}{}%
\end{pgfscope}%
\begin{pgfscope}%
\pgfsys@transformshift{2.970464in}{1.415212in}%
\pgfsys@useobject{currentmarker}{}%
\end{pgfscope}%
\begin{pgfscope}%
\pgfsys@transformshift{3.009500in}{1.520376in}%
\pgfsys@useobject{currentmarker}{}%
\end{pgfscope}%
\begin{pgfscope}%
\pgfsys@transformshift{3.045533in}{1.593533in}%
\pgfsys@useobject{currentmarker}{}%
\end{pgfscope}%
\begin{pgfscope}%
\pgfsys@transformshift{3.083068in}{1.689552in}%
\pgfsys@useobject{currentmarker}{}%
\end{pgfscope}%
\begin{pgfscope}%
\pgfsys@transformshift{3.120603in}{1.771854in}%
\pgfsys@useobject{currentmarker}{}%
\end{pgfscope}%
\begin{pgfscope}%
\pgfsys@transformshift{3.158138in}{1.845012in}%
\pgfsys@useobject{currentmarker}{}%
\end{pgfscope}%
\begin{pgfscope}%
\pgfsys@transformshift{3.195673in}{1.904452in}%
\pgfsys@useobject{currentmarker}{}%
\end{pgfscope}%
\begin{pgfscope}%
\pgfsys@transformshift{3.233208in}{1.973037in}%
\pgfsys@useobject{currentmarker}{}%
\end{pgfscope}%
\begin{pgfscope}%
\pgfsys@transformshift{3.270743in}{2.014188in}%
\pgfsys@useobject{currentmarker}{}%
\end{pgfscope}%
\begin{pgfscope}%
\pgfsys@transformshift{3.308278in}{2.059911in}%
\pgfsys@useobject{currentmarker}{}%
\end{pgfscope}%
\begin{pgfscope}%
\pgfsys@transformshift{3.345813in}{2.096490in}%
\pgfsys@useobject{currentmarker}{}%
\end{pgfscope}%
\begin{pgfscope}%
\pgfsys@transformshift{3.383347in}{2.128496in}%
\pgfsys@useobject{currentmarker}{}%
\end{pgfscope}%
\begin{pgfscope}%
\pgfsys@transformshift{3.420882in}{2.169647in}%
\pgfsys@useobject{currentmarker}{}%
\end{pgfscope}%
\begin{pgfscope}%
\pgfsys@transformshift{3.458417in}{2.197081in}%
\pgfsys@useobject{currentmarker}{}%
\end{pgfscope}%
\begin{pgfscope}%
\pgfsys@transformshift{3.495952in}{2.219943in}%
\pgfsys@useobject{currentmarker}{}%
\end{pgfscope}%
\begin{pgfscope}%
\pgfsys@transformshift{3.533487in}{2.251949in}%
\pgfsys@useobject{currentmarker}{}%
\end{pgfscope}%
\begin{pgfscope}%
\pgfsys@transformshift{3.571022in}{2.306817in}%
\pgfsys@useobject{currentmarker}{}%
\end{pgfscope}%
\begin{pgfscope}%
\pgfsys@transformshift{3.608557in}{2.366258in}%
\pgfsys@useobject{currentmarker}{}%
\end{pgfscope}%
\begin{pgfscope}%
\pgfsys@transformshift{3.646092in}{2.434843in}%
\pgfsys@useobject{currentmarker}{}%
\end{pgfscope}%
\begin{pgfscope}%
\pgfsys@transformshift{3.683627in}{2.585730in}%
\pgfsys@useobject{currentmarker}{}%
\end{pgfscope}%
\begin{pgfscope}%
\pgfsys@transformshift{3.721161in}{2.754906in}%
\pgfsys@useobject{currentmarker}{}%
\end{pgfscope}%
\begin{pgfscope}%
\pgfsys@transformshift{3.758696in}{2.937800in}%
\pgfsys@useobject{currentmarker}{}%
\end{pgfscope}%
\begin{pgfscope}%
\pgfsys@transformshift{3.796231in}{3.093259in}%
\pgfsys@useobject{currentmarker}{}%
\end{pgfscope}%
\begin{pgfscope}%
\pgfsys@transformshift{3.833766in}{3.230429in}%
\pgfsys@useobject{currentmarker}{}%
\end{pgfscope}%
\begin{pgfscope}%
\pgfsys@transformshift{3.871301in}{3.340165in}%
\pgfsys@useobject{currentmarker}{}%
\end{pgfscope}%
\begin{pgfscope}%
\pgfsys@transformshift{3.908836in}{3.417895in}%
\pgfsys@useobject{currentmarker}{}%
\end{pgfscope}%
\begin{pgfscope}%
\pgfsys@transformshift{3.946371in}{3.509342in}%
\pgfsys@useobject{currentmarker}{}%
\end{pgfscope}%
\begin{pgfscope}%
\pgfsys@transformshift{3.983906in}{3.573354in}%
\pgfsys@useobject{currentmarker}{}%
\end{pgfscope}%
\begin{pgfscope}%
\pgfsys@transformshift{4.171580in}{3.792826in}%
\pgfsys@useobject{currentmarker}{}%
\end{pgfscope}%
\begin{pgfscope}%
\pgfsys@transformshift{4.546929in}{4.030588in}%
\pgfsys@useobject{currentmarker}{}%
\end{pgfscope}%
\begin{pgfscope}%
\pgfsys@transformshift{4.922278in}{4.172330in}%
\pgfsys@useobject{currentmarker}{}%
\end{pgfscope}%
\begin{pgfscope}%
\pgfsys@transformshift{5.297627in}{4.291211in}%
\pgfsys@useobject{currentmarker}{}%
\end{pgfscope}%
\begin{pgfscope}%
\pgfsys@transformshift{6.048325in}{4.487821in}%
\pgfsys@useobject{currentmarker}{}%
\end{pgfscope}%
\begin{pgfscope}%
\pgfsys@transformshift{6.799023in}{4.666143in}%
\pgfsys@useobject{currentmarker}{}%
\end{pgfscope}%
\begin{pgfscope}%
\pgfsys@transformshift{7.549721in}{4.821602in}%
\pgfsys@useobject{currentmarker}{}%
\end{pgfscope}%
\end{pgfscope}%
\begin{pgfscope}%
\pgfpathrectangle{\pgfqpoint{1.243858in}{0.894194in}}{\pgfqpoint{6.606142in}{4.119217in}}%
\pgfusepath{clip}%
\pgfsetbuttcap%
\pgfsetroundjoin%
\definecolor{currentfill}{rgb}{0.298039,0.447059,0.690196}%
\pgfsetfillcolor{currentfill}%
\pgfsetlinewidth{1.003750pt}%
\definecolor{currentstroke}{rgb}{0.298039,0.447059,0.690196}%
\pgfsetstrokecolor{currentstroke}%
\pgfsetdash{}{0pt}%
\pgfsys@defobject{currentmarker}{\pgfqpoint{-0.069444in}{-0.000000in}}{\pgfqpoint{0.069444in}{0.000000in}}{%
\pgfpathmoveto{\pgfqpoint{0.069444in}{-0.000000in}}%
\pgfpathlineto{\pgfqpoint{-0.069444in}{0.000000in}}%
\pgfusepath{stroke,fill}%
}%
\begin{pgfscope}%
\pgfsys@transformshift{1.544137in}{1.108866in}%
\pgfsys@useobject{currentmarker}{}%
\end{pgfscope}%
\begin{pgfscope}%
\pgfsys@transformshift{1.919486in}{1.108866in}%
\pgfsys@useobject{currentmarker}{}%
\end{pgfscope}%
\begin{pgfscope}%
\pgfsys@transformshift{2.294835in}{1.104293in}%
\pgfsys@useobject{currentmarker}{}%
\end{pgfscope}%
\begin{pgfscope}%
\pgfsys@transformshift{2.670184in}{1.108866in}%
\pgfsys@useobject{currentmarker}{}%
\end{pgfscope}%
\begin{pgfscope}%
\pgfsys@transformshift{2.857859in}{1.209457in}%
\pgfsys@useobject{currentmarker}{}%
\end{pgfscope}%
\begin{pgfscope}%
\pgfsys@transformshift{2.895394in}{1.278042in}%
\pgfsys@useobject{currentmarker}{}%
\end{pgfscope}%
\begin{pgfscope}%
\pgfsys@transformshift{2.932929in}{1.342055in}%
\pgfsys@useobject{currentmarker}{}%
\end{pgfscope}%
\begin{pgfscope}%
\pgfsys@transformshift{2.970464in}{1.419784in}%
\pgfsys@useobject{currentmarker}{}%
\end{pgfscope}%
\begin{pgfscope}%
\pgfsys@transformshift{3.009500in}{1.524948in}%
\pgfsys@useobject{currentmarker}{}%
\end{pgfscope}%
\begin{pgfscope}%
\pgfsys@transformshift{3.045533in}{1.616395in}%
\pgfsys@useobject{currentmarker}{}%
\end{pgfscope}%
\begin{pgfscope}%
\pgfsys@transformshift{3.083068in}{1.694125in}%
\pgfsys@useobject{currentmarker}{}%
\end{pgfscope}%
\begin{pgfscope}%
\pgfsys@transformshift{3.120603in}{1.771854in}%
\pgfsys@useobject{currentmarker}{}%
\end{pgfscope}%
\begin{pgfscope}%
\pgfsys@transformshift{3.158138in}{1.849584in}%
\pgfsys@useobject{currentmarker}{}%
\end{pgfscope}%
\begin{pgfscope}%
\pgfsys@transformshift{3.195673in}{1.913597in}%
\pgfsys@useobject{currentmarker}{}%
\end{pgfscope}%
\begin{pgfscope}%
\pgfsys@transformshift{3.233208in}{1.973037in}%
\pgfsys@useobject{currentmarker}{}%
\end{pgfscope}%
\begin{pgfscope}%
\pgfsys@transformshift{3.270743in}{2.014188in}%
\pgfsys@useobject{currentmarker}{}%
\end{pgfscope}%
\begin{pgfscope}%
\pgfsys@transformshift{3.308278in}{2.064484in}%
\pgfsys@useobject{currentmarker}{}%
\end{pgfscope}%
\begin{pgfscope}%
\pgfsys@transformshift{3.345813in}{2.101062in}%
\pgfsys@useobject{currentmarker}{}%
\end{pgfscope}%
\begin{pgfscope}%
\pgfsys@transformshift{3.383347in}{2.133069in}%
\pgfsys@useobject{currentmarker}{}%
\end{pgfscope}%
\begin{pgfscope}%
\pgfsys@transformshift{3.420882in}{2.169647in}%
\pgfsys@useobject{currentmarker}{}%
\end{pgfscope}%
\begin{pgfscope}%
\pgfsys@transformshift{3.458417in}{2.206226in}%
\pgfsys@useobject{currentmarker}{}%
\end{pgfscope}%
\begin{pgfscope}%
\pgfsys@transformshift{3.495952in}{2.233660in}%
\pgfsys@useobject{currentmarker}{}%
\end{pgfscope}%
\begin{pgfscope}%
\pgfsys@transformshift{3.533487in}{2.256522in}%
\pgfsys@useobject{currentmarker}{}%
\end{pgfscope}%
\begin{pgfscope}%
\pgfsys@transformshift{3.571022in}{2.306817in}%
\pgfsys@useobject{currentmarker}{}%
\end{pgfscope}%
\begin{pgfscope}%
\pgfsys@transformshift{3.608557in}{2.366258in}%
\pgfsys@useobject{currentmarker}{}%
\end{pgfscope}%
\begin{pgfscope}%
\pgfsys@transformshift{3.646092in}{2.448560in}%
\pgfsys@useobject{currentmarker}{}%
\end{pgfscope}%
\begin{pgfscope}%
\pgfsys@transformshift{3.683627in}{2.585730in}%
\pgfsys@useobject{currentmarker}{}%
\end{pgfscope}%
\begin{pgfscope}%
\pgfsys@transformshift{3.721161in}{2.759479in}%
\pgfsys@useobject{currentmarker}{}%
\end{pgfscope}%
\begin{pgfscope}%
\pgfsys@transformshift{3.758696in}{2.942372in}%
\pgfsys@useobject{currentmarker}{}%
\end{pgfscope}%
\begin{pgfscope}%
\pgfsys@transformshift{3.796231in}{3.102404in}%
\pgfsys@useobject{currentmarker}{}%
\end{pgfscope}%
\begin{pgfscope}%
\pgfsys@transformshift{3.833766in}{3.230429in}%
\pgfsys@useobject{currentmarker}{}%
\end{pgfscope}%
\begin{pgfscope}%
\pgfsys@transformshift{3.871301in}{3.340165in}%
\pgfsys@useobject{currentmarker}{}%
\end{pgfscope}%
\begin{pgfscope}%
\pgfsys@transformshift{3.908836in}{3.417895in}%
\pgfsys@useobject{currentmarker}{}%
\end{pgfscope}%
\begin{pgfscope}%
\pgfsys@transformshift{3.946371in}{3.509342in}%
\pgfsys@useobject{currentmarker}{}%
\end{pgfscope}%
\begin{pgfscope}%
\pgfsys@transformshift{3.983906in}{3.573354in}%
\pgfsys@useobject{currentmarker}{}%
\end{pgfscope}%
\begin{pgfscope}%
\pgfsys@transformshift{4.171580in}{3.797399in}%
\pgfsys@useobject{currentmarker}{}%
\end{pgfscope}%
\begin{pgfscope}%
\pgfsys@transformshift{4.546929in}{4.030588in}%
\pgfsys@useobject{currentmarker}{}%
\end{pgfscope}%
\begin{pgfscope}%
\pgfsys@transformshift{4.922278in}{4.172330in}%
\pgfsys@useobject{currentmarker}{}%
\end{pgfscope}%
\begin{pgfscope}%
\pgfsys@transformshift{5.297627in}{4.291211in}%
\pgfsys@useobject{currentmarker}{}%
\end{pgfscope}%
\begin{pgfscope}%
\pgfsys@transformshift{6.048325in}{4.487821in}%
\pgfsys@useobject{currentmarker}{}%
\end{pgfscope}%
\begin{pgfscope}%
\pgfsys@transformshift{6.799023in}{4.666143in}%
\pgfsys@useobject{currentmarker}{}%
\end{pgfscope}%
\begin{pgfscope}%
\pgfsys@transformshift{7.549721in}{4.826174in}%
\pgfsys@useobject{currentmarker}{}%
\end{pgfscope}%
\end{pgfscope}%
\begin{pgfscope}%
\pgfpathrectangle{\pgfqpoint{1.243858in}{0.894194in}}{\pgfqpoint{6.606142in}{4.119217in}}%
\pgfusepath{clip}%
\pgfsetroundcap%
\pgfsetroundjoin%
\pgfsetlinewidth{1.756562pt}%
\definecolor{currentstroke}{rgb}{0.298039,0.447059,0.690196}%
\pgfsetstrokecolor{currentstroke}%
\pgfsetdash{}{0pt}%
\pgfpathmoveto{\pgfqpoint{1.544137in}{1.108866in}}%
\pgfpathlineto{\pgfqpoint{1.919486in}{1.108866in}}%
\pgfpathlineto{\pgfqpoint{2.294835in}{1.104293in}}%
\pgfpathlineto{\pgfqpoint{2.670184in}{1.095149in}}%
\pgfpathlineto{\pgfqpoint{2.857859in}{1.207171in}}%
\pgfpathlineto{\pgfqpoint{2.895394in}{1.271183in}}%
\pgfpathlineto{\pgfqpoint{2.932929in}{1.339769in}}%
\pgfpathlineto{\pgfqpoint{2.970464in}{1.417498in}}%
\pgfpathlineto{\pgfqpoint{3.009500in}{1.522662in}}%
\pgfpathlineto{\pgfqpoint{3.045533in}{1.604964in}}%
\pgfpathlineto{\pgfqpoint{3.083068in}{1.691838in}}%
\pgfpathlineto{\pgfqpoint{3.120603in}{1.771854in}}%
\pgfpathlineto{\pgfqpoint{3.158138in}{1.847298in}}%
\pgfpathlineto{\pgfqpoint{3.195673in}{1.909024in}}%
\pgfpathlineto{\pgfqpoint{3.233208in}{1.973037in}}%
\pgfpathlineto{\pgfqpoint{3.270743in}{2.014188in}}%
\pgfpathlineto{\pgfqpoint{3.308278in}{2.062198in}}%
\pgfpathlineto{\pgfqpoint{3.345813in}{2.098776in}}%
\pgfpathlineto{\pgfqpoint{3.383347in}{2.130783in}}%
\pgfpathlineto{\pgfqpoint{3.420882in}{2.169647in}}%
\pgfpathlineto{\pgfqpoint{3.458417in}{2.201654in}}%
\pgfpathlineto{\pgfqpoint{3.495952in}{2.226802in}}%
\pgfpathlineto{\pgfqpoint{3.533487in}{2.254236in}}%
\pgfpathlineto{\pgfqpoint{3.571022in}{2.306817in}}%
\pgfpathlineto{\pgfqpoint{3.608557in}{2.366258in}}%
\pgfpathlineto{\pgfqpoint{3.646092in}{2.441701in}}%
\pgfpathlineto{\pgfqpoint{3.683627in}{2.585730in}}%
\pgfpathlineto{\pgfqpoint{3.721161in}{2.757192in}}%
\pgfpathlineto{\pgfqpoint{3.758696in}{2.940086in}}%
\pgfpathlineto{\pgfqpoint{3.796231in}{3.097831in}}%
\pgfpathlineto{\pgfqpoint{3.833766in}{3.230429in}}%
\pgfpathlineto{\pgfqpoint{3.871301in}{3.340165in}}%
\pgfpathlineto{\pgfqpoint{3.908836in}{3.417895in}}%
\pgfpathlineto{\pgfqpoint{3.946371in}{3.509342in}}%
\pgfpathlineto{\pgfqpoint{3.983906in}{3.573354in}}%
\pgfpathlineto{\pgfqpoint{4.171580in}{3.795113in}}%
\pgfpathlineto{\pgfqpoint{4.546929in}{4.030588in}}%
\pgfpathlineto{\pgfqpoint{4.922278in}{4.172330in}}%
\pgfpathlineto{\pgfqpoint{5.297627in}{4.291211in}}%
\pgfpathlineto{\pgfqpoint{6.048325in}{4.487821in}}%
\pgfpathlineto{\pgfqpoint{6.799023in}{4.666143in}}%
\pgfpathlineto{\pgfqpoint{7.549721in}{4.823888in}}%
\pgfusepath{stroke}%
\end{pgfscope}%
\begin{pgfscope}%
\pgfsetrectcap%
\pgfsetmiterjoin%
\pgfsetlinewidth{0.000000pt}%
\definecolor{currentstroke}{rgb}{1.000000,1.000000,1.000000}%
\pgfsetstrokecolor{currentstroke}%
\pgfsetdash{}{0pt}%
\pgfpathmoveto{\pgfqpoint{1.243858in}{0.894194in}}%
\pgfpathlineto{\pgfqpoint{1.243858in}{5.013411in}}%
\pgfusepath{}%
\end{pgfscope}%
\begin{pgfscope}%
\pgfsetrectcap%
\pgfsetmiterjoin%
\pgfsetlinewidth{0.000000pt}%
\definecolor{currentstroke}{rgb}{1.000000,1.000000,1.000000}%
\pgfsetstrokecolor{currentstroke}%
\pgfsetdash{}{0pt}%
\pgfpathmoveto{\pgfqpoint{7.850000in}{0.894194in}}%
\pgfpathlineto{\pgfqpoint{7.850000in}{5.013411in}}%
\pgfusepath{}%
\end{pgfscope}%
\begin{pgfscope}%
\pgfsetrectcap%
\pgfsetmiterjoin%
\pgfsetlinewidth{0.000000pt}%
\definecolor{currentstroke}{rgb}{1.000000,1.000000,1.000000}%
\pgfsetstrokecolor{currentstroke}%
\pgfsetdash{}{0pt}%
\pgfpathmoveto{\pgfqpoint{1.243858in}{0.894194in}}%
\pgfpathlineto{\pgfqpoint{7.850000in}{0.894194in}}%
\pgfusepath{}%
\end{pgfscope}%
\begin{pgfscope}%
\pgfsetrectcap%
\pgfsetmiterjoin%
\pgfsetlinewidth{0.000000pt}%
\definecolor{currentstroke}{rgb}{1.000000,1.000000,1.000000}%
\pgfsetstrokecolor{currentstroke}%
\pgfsetdash{}{0pt}%
\pgfpathmoveto{\pgfqpoint{1.243858in}{5.013411in}}%
\pgfpathlineto{\pgfqpoint{7.850000in}{5.013411in}}%
\pgfusepath{}%
\end{pgfscope}%
\begin{pgfscope}%
\definecolor{textcolor}{rgb}{0.150000,0.150000,0.150000}%
\pgfsetstrokecolor{textcolor}%
\pgfsetfillcolor{textcolor}%
\pgftext[x=4.546929in,y=5.096745in,,base]{\color{textcolor}\rmfamily\fontsize{24.000000}{28.800000}\selectfont Resistivity of Bi\(\displaystyle _2\)Sr\(\displaystyle _2\)Ca\(\displaystyle _2\)Cu\(\displaystyle _3\)O\(\displaystyle _{10}\)}%
\end{pgfscope}%
\end{pgfpicture}%
\makeatother%
\endgroup%
}
        \caption{this is a test of the captions to se if  HOLDER}
        \label{fig:bscco_temp}
    \end{figure}

    Lets cover the disorder effects first. As cuprates both BSCCO and YBCO undergo superconductor-insulator phase transition (SIT) at their respective $T_c$'s. This type of transition is suspected to have both fermionic~\cite{finkel1994suppression} and bosonic~\cite{fisher1990presence} mechanisms for the suppression of superconductivity.~\cite{Nat_SIT} In the case of fermionic suppression, Cooper pairs are destroyed prior to their formation due to the localization of electrons with increased disorder. In bosonic suppression, Cooper pairs that are formed at the transition become localized due to the disorder thus disrupting their conductivity and maintaining the insulating state. Both of these mechanisms hinge on electrons scattering off lattice distortions.  
    

    Seemingly in contrast, in 1959 Anderson noted that the $T_c$ of most superconductors is largely insensitive to physical and chemical impurities due to Cooper pairs being formed from time-reversible eigenstates.~\cite{anderson1959theory} In summery, boson formation is dependent on the averaging of scattered one-electron states which is identical to the average of unscattered plane waves if the spins were left unchanged, details in appendix. However, the introduction of magnetic impurities breaks this time-reversal symmetry. Atomic oxygen fills this role as cuprates experience oxygen vacancies and enrichment as a source of intra-grain disorder, and it has an effective magnetic moment of $1.71 \mu B$,~\cite{oxygenMag}. However, this type of defect in a ``homogenous'' system would result in the broadening of the $T_c$ range (both up and down) but not necessarily the aforementioned shelf. 

    The shelf most likely stems from the existence of domains comprised of different stoichiometric ratios that have aggregated. Most ceramics exist as a solid solution that include vacancies throughout the material. This is normally spread evenly through a sample but over the course of heating and cooling cycles vacancies and defects can cluster through dislocation creep in thus forming these larger domains. In the case of BSCCO the generalized formula is $Bi_2Sr_2Ca_{n-1}Cu_nO_{2n+4+x}$ where domains would depend on not only the oxygen vacancies (x) but also the ratios of Ca and Cu (n). These different stoichio-tropes (akin to allotropes) can have vastly different $T_c$ ranging from 33--108 K. This is why the resistivity initially drops at 110K and slightly plateaus before dropping again at roughly 104K. These $T_c$ might correspond to Bi-2223 at 108K with broadening from disorder and Bi-2234 with $T_c \approx 104~K$.
    
    
  
    \begin{figure}[t]
      \resizebox{0.45\textwidth}{!}{%% Creator: Matplotlib, PGF backend
%%
%% To include the figure in your LaTeX document, write
%%   \input{<filename>.pgf}
%%
%% Make sure the required packages are loaded in your preamble
%%   \usepackage{pgf}
%%
%% Also ensure that all the required font packages are loaded; for instance,
%% the lmodern package is sometimes necessary when using math font.
%%   \usepackage{lmodern}
%%
%% Figures using additional raster images can only be included by \input if
%% they are in the same directory as the main LaTeX file. For loading figures
%% from other directories you can use the `import` package
%%   \usepackage{import}
%%
%% and then include the figures with
%%   \import{<path to file>}{<filename>.pgf}
%%
%% Matplotlib used the following preamble
%%   \usepackage{fontspec}
%%   \setmainfont{DejaVuSerif.ttf}[Path=\detokenize{/usr/share/matplotlib/mpl-data/fonts/ttf/}]
%%   \setsansfont{DejaVuSans.ttf}[Path=\detokenize{/usr/share/matplotlib/mpl-data/fonts/ttf/}]
%%   \setmonofont{DejaVuSansMono.ttf}[Path=\detokenize{/usr/share/matplotlib/mpl-data/fonts/ttf/}]
%%
\begingroup%
\makeatletter%
\begin{pgfpicture}%
\pgfpathrectangle{\pgfpointorigin}{\pgfqpoint{8.000000in}{5.500000in}}%
\pgfusepath{use as bounding box, clip}%
\begin{pgfscope}%
\pgfsetbuttcap%
\pgfsetmiterjoin%
\definecolor{currentfill}{rgb}{1.000000,1.000000,1.000000}%
\pgfsetfillcolor{currentfill}%
\pgfsetlinewidth{0.000000pt}%
\definecolor{currentstroke}{rgb}{1.000000,1.000000,1.000000}%
\pgfsetstrokecolor{currentstroke}%
\pgfsetdash{}{0pt}%
\pgfpathmoveto{\pgfqpoint{0.000000in}{0.000000in}}%
\pgfpathlineto{\pgfqpoint{8.000000in}{0.000000in}}%
\pgfpathlineto{\pgfqpoint{8.000000in}{5.500000in}}%
\pgfpathlineto{\pgfqpoint{0.000000in}{5.500000in}}%
\pgfpathlineto{\pgfqpoint{0.000000in}{0.000000in}}%
\pgfpathclose%
\pgfusepath{fill}%
\end{pgfscope}%
\begin{pgfscope}%
\pgfsetbuttcap%
\pgfsetmiterjoin%
\definecolor{currentfill}{rgb}{0.917647,0.917647,0.949020}%
\pgfsetfillcolor{currentfill}%
\pgfsetlinewidth{0.000000pt}%
\definecolor{currentstroke}{rgb}{0.000000,0.000000,0.000000}%
\pgfsetstrokecolor{currentstroke}%
\pgfsetstrokeopacity{0.000000}%
\pgfsetdash{}{0pt}%
\pgfpathmoveto{\pgfqpoint{0.978830in}{0.894194in}}%
\pgfpathlineto{\pgfqpoint{7.850000in}{0.894194in}}%
\pgfpathlineto{\pgfqpoint{7.850000in}{5.013411in}}%
\pgfpathlineto{\pgfqpoint{0.978830in}{5.013411in}}%
\pgfpathlineto{\pgfqpoint{0.978830in}{0.894194in}}%
\pgfpathclose%
\pgfusepath{fill}%
\end{pgfscope}%
\begin{pgfscope}%
\pgfpathrectangle{\pgfqpoint{0.978830in}{0.894194in}}{\pgfqpoint{6.871170in}{4.119217in}}%
\pgfusepath{clip}%
\pgfsetroundcap%
\pgfsetroundjoin%
\pgfsetlinewidth{1.003750pt}%
\definecolor{currentstroke}{rgb}{1.000000,1.000000,1.000000}%
\pgfsetstrokecolor{currentstroke}%
\pgfsetdash{}{0pt}%
\pgfpathmoveto{\pgfqpoint{1.279422in}{0.894194in}}%
\pgfpathlineto{\pgfqpoint{1.279422in}{5.013411in}}%
\pgfusepath{stroke}%
\end{pgfscope}%
\begin{pgfscope}%
\definecolor{textcolor}{rgb}{0.150000,0.150000,0.150000}%
\pgfsetstrokecolor{textcolor}%
\pgfsetfillcolor{textcolor}%
\pgftext[x=1.279422in,y=0.796972in,,top]{\color{textcolor}\rmfamily\fontsize{20.000000}{24.000000}\selectfont 80}%
\end{pgfscope}%
\begin{pgfscope}%
\pgfpathrectangle{\pgfqpoint{0.978830in}{0.894194in}}{\pgfqpoint{6.871170in}{4.119217in}}%
\pgfusepath{clip}%
\pgfsetroundcap%
\pgfsetroundjoin%
\pgfsetlinewidth{1.003750pt}%
\definecolor{currentstroke}{rgb}{1.000000,1.000000,1.000000}%
\pgfsetstrokecolor{currentstroke}%
\pgfsetdash{}{0pt}%
\pgfpathmoveto{\pgfqpoint{2.061703in}{0.894194in}}%
\pgfpathlineto{\pgfqpoint{2.061703in}{5.013411in}}%
\pgfusepath{stroke}%
\end{pgfscope}%
\begin{pgfscope}%
\definecolor{textcolor}{rgb}{0.150000,0.150000,0.150000}%
\pgfsetstrokecolor{textcolor}%
\pgfsetfillcolor{textcolor}%
\pgftext[x=2.061703in,y=0.796972in,,top]{\color{textcolor}\rmfamily\fontsize{20.000000}{24.000000}\selectfont 90}%
\end{pgfscope}%
\begin{pgfscope}%
\pgfpathrectangle{\pgfqpoint{0.978830in}{0.894194in}}{\pgfqpoint{6.871170in}{4.119217in}}%
\pgfusepath{clip}%
\pgfsetroundcap%
\pgfsetroundjoin%
\pgfsetlinewidth{1.003750pt}%
\definecolor{currentstroke}{rgb}{1.000000,1.000000,1.000000}%
\pgfsetstrokecolor{currentstroke}%
\pgfsetdash{}{0pt}%
\pgfpathmoveto{\pgfqpoint{2.843985in}{0.894194in}}%
\pgfpathlineto{\pgfqpoint{2.843985in}{5.013411in}}%
\pgfusepath{stroke}%
\end{pgfscope}%
\begin{pgfscope}%
\definecolor{textcolor}{rgb}{0.150000,0.150000,0.150000}%
\pgfsetstrokecolor{textcolor}%
\pgfsetfillcolor{textcolor}%
\pgftext[x=2.843985in,y=0.796972in,,top]{\color{textcolor}\rmfamily\fontsize{20.000000}{24.000000}\selectfont 100}%
\end{pgfscope}%
\begin{pgfscope}%
\pgfpathrectangle{\pgfqpoint{0.978830in}{0.894194in}}{\pgfqpoint{6.871170in}{4.119217in}}%
\pgfusepath{clip}%
\pgfsetroundcap%
\pgfsetroundjoin%
\pgfsetlinewidth{1.003750pt}%
\definecolor{currentstroke}{rgb}{1.000000,1.000000,1.000000}%
\pgfsetstrokecolor{currentstroke}%
\pgfsetdash{}{0pt}%
\pgfpathmoveto{\pgfqpoint{3.626266in}{0.894194in}}%
\pgfpathlineto{\pgfqpoint{3.626266in}{5.013411in}}%
\pgfusepath{stroke}%
\end{pgfscope}%
\begin{pgfscope}%
\definecolor{textcolor}{rgb}{0.150000,0.150000,0.150000}%
\pgfsetstrokecolor{textcolor}%
\pgfsetfillcolor{textcolor}%
\pgftext[x=3.626266in,y=0.796972in,,top]{\color{textcolor}\rmfamily\fontsize{20.000000}{24.000000}\selectfont 110}%
\end{pgfscope}%
\begin{pgfscope}%
\pgfpathrectangle{\pgfqpoint{0.978830in}{0.894194in}}{\pgfqpoint{6.871170in}{4.119217in}}%
\pgfusepath{clip}%
\pgfsetroundcap%
\pgfsetroundjoin%
\pgfsetlinewidth{1.003750pt}%
\definecolor{currentstroke}{rgb}{1.000000,1.000000,1.000000}%
\pgfsetstrokecolor{currentstroke}%
\pgfsetdash{}{0pt}%
\pgfpathmoveto{\pgfqpoint{4.408548in}{0.894194in}}%
\pgfpathlineto{\pgfqpoint{4.408548in}{5.013411in}}%
\pgfusepath{stroke}%
\end{pgfscope}%
\begin{pgfscope}%
\definecolor{textcolor}{rgb}{0.150000,0.150000,0.150000}%
\pgfsetstrokecolor{textcolor}%
\pgfsetfillcolor{textcolor}%
\pgftext[x=4.408548in,y=0.796972in,,top]{\color{textcolor}\rmfamily\fontsize{20.000000}{24.000000}\selectfont 120}%
\end{pgfscope}%
\begin{pgfscope}%
\pgfpathrectangle{\pgfqpoint{0.978830in}{0.894194in}}{\pgfqpoint{6.871170in}{4.119217in}}%
\pgfusepath{clip}%
\pgfsetroundcap%
\pgfsetroundjoin%
\pgfsetlinewidth{1.003750pt}%
\definecolor{currentstroke}{rgb}{1.000000,1.000000,1.000000}%
\pgfsetstrokecolor{currentstroke}%
\pgfsetdash{}{0pt}%
\pgfpathmoveto{\pgfqpoint{5.190829in}{0.894194in}}%
\pgfpathlineto{\pgfqpoint{5.190829in}{5.013411in}}%
\pgfusepath{stroke}%
\end{pgfscope}%
\begin{pgfscope}%
\definecolor{textcolor}{rgb}{0.150000,0.150000,0.150000}%
\pgfsetstrokecolor{textcolor}%
\pgfsetfillcolor{textcolor}%
\pgftext[x=5.190829in,y=0.796972in,,top]{\color{textcolor}\rmfamily\fontsize{20.000000}{24.000000}\selectfont 130}%
\end{pgfscope}%
\begin{pgfscope}%
\pgfpathrectangle{\pgfqpoint{0.978830in}{0.894194in}}{\pgfqpoint{6.871170in}{4.119217in}}%
\pgfusepath{clip}%
\pgfsetroundcap%
\pgfsetroundjoin%
\pgfsetlinewidth{1.003750pt}%
\definecolor{currentstroke}{rgb}{1.000000,1.000000,1.000000}%
\pgfsetstrokecolor{currentstroke}%
\pgfsetdash{}{0pt}%
\pgfpathmoveto{\pgfqpoint{5.973111in}{0.894194in}}%
\pgfpathlineto{\pgfqpoint{5.973111in}{5.013411in}}%
\pgfusepath{stroke}%
\end{pgfscope}%
\begin{pgfscope}%
\definecolor{textcolor}{rgb}{0.150000,0.150000,0.150000}%
\pgfsetstrokecolor{textcolor}%
\pgfsetfillcolor{textcolor}%
\pgftext[x=5.973111in,y=0.796972in,,top]{\color{textcolor}\rmfamily\fontsize{20.000000}{24.000000}\selectfont 140}%
\end{pgfscope}%
\begin{pgfscope}%
\pgfpathrectangle{\pgfqpoint{0.978830in}{0.894194in}}{\pgfqpoint{6.871170in}{4.119217in}}%
\pgfusepath{clip}%
\pgfsetroundcap%
\pgfsetroundjoin%
\pgfsetlinewidth{1.003750pt}%
\definecolor{currentstroke}{rgb}{1.000000,1.000000,1.000000}%
\pgfsetstrokecolor{currentstroke}%
\pgfsetdash{}{0pt}%
\pgfpathmoveto{\pgfqpoint{6.755393in}{0.894194in}}%
\pgfpathlineto{\pgfqpoint{6.755393in}{5.013411in}}%
\pgfusepath{stroke}%
\end{pgfscope}%
\begin{pgfscope}%
\definecolor{textcolor}{rgb}{0.150000,0.150000,0.150000}%
\pgfsetstrokecolor{textcolor}%
\pgfsetfillcolor{textcolor}%
\pgftext[x=6.755393in,y=0.796972in,,top]{\color{textcolor}\rmfamily\fontsize{20.000000}{24.000000}\selectfont 150}%
\end{pgfscope}%
\begin{pgfscope}%
\pgfpathrectangle{\pgfqpoint{0.978830in}{0.894194in}}{\pgfqpoint{6.871170in}{4.119217in}}%
\pgfusepath{clip}%
\pgfsetroundcap%
\pgfsetroundjoin%
\pgfsetlinewidth{1.003750pt}%
\definecolor{currentstroke}{rgb}{1.000000,1.000000,1.000000}%
\pgfsetstrokecolor{currentstroke}%
\pgfsetdash{}{0pt}%
\pgfpathmoveto{\pgfqpoint{7.537674in}{0.894194in}}%
\pgfpathlineto{\pgfqpoint{7.537674in}{5.013411in}}%
\pgfusepath{stroke}%
\end{pgfscope}%
\begin{pgfscope}%
\definecolor{textcolor}{rgb}{0.150000,0.150000,0.150000}%
\pgfsetstrokecolor{textcolor}%
\pgfsetfillcolor{textcolor}%
\pgftext[x=7.537674in,y=0.796972in,,top]{\color{textcolor}\rmfamily\fontsize{20.000000}{24.000000}\selectfont 160}%
\end{pgfscope}%
\begin{pgfscope}%
\definecolor{textcolor}{rgb}{0.150000,0.150000,0.150000}%
\pgfsetstrokecolor{textcolor}%
\pgfsetfillcolor{textcolor}%
\pgftext[x=4.414415in,y=0.472591in,,top]{\color{textcolor}\rmfamily\fontsize{24.000000}{28.800000}\selectfont Temperature (K)}%
\end{pgfscope}%
\begin{pgfscope}%
\pgfpathrectangle{\pgfqpoint{0.978830in}{0.894194in}}{\pgfqpoint{6.871170in}{4.119217in}}%
\pgfusepath{clip}%
\pgfsetroundcap%
\pgfsetroundjoin%
\pgfsetlinewidth{1.003750pt}%
\definecolor{currentstroke}{rgb}{1.000000,1.000000,1.000000}%
\pgfsetstrokecolor{currentstroke}%
\pgfsetdash{}{0pt}%
\pgfpathmoveto{\pgfqpoint{0.978830in}{0.972138in}}%
\pgfpathlineto{\pgfqpoint{7.850000in}{0.972138in}}%
\pgfusepath{stroke}%
\end{pgfscope}%
\begin{pgfscope}%
\definecolor{textcolor}{rgb}{0.150000,0.150000,0.150000}%
\pgfsetstrokecolor{textcolor}%
\pgfsetfillcolor{textcolor}%
\pgftext[x=0.704877in, y=0.866615in, left, base]{\color{textcolor}\rmfamily\fontsize{20.000000}{24.000000}\selectfont 0}%
\end{pgfscope}%
\begin{pgfscope}%
\pgfpathrectangle{\pgfqpoint{0.978830in}{0.894194in}}{\pgfqpoint{6.871170in}{4.119217in}}%
\pgfusepath{clip}%
\pgfsetroundcap%
\pgfsetroundjoin%
\pgfsetlinewidth{1.003750pt}%
\definecolor{currentstroke}{rgb}{1.000000,1.000000,1.000000}%
\pgfsetstrokecolor{currentstroke}%
\pgfsetdash{}{0pt}%
\pgfpathmoveto{\pgfqpoint{0.978830in}{1.606718in}}%
\pgfpathlineto{\pgfqpoint{7.850000in}{1.606718in}}%
\pgfusepath{stroke}%
\end{pgfscope}%
\begin{pgfscope}%
\definecolor{textcolor}{rgb}{0.150000,0.150000,0.150000}%
\pgfsetstrokecolor{textcolor}%
\pgfsetfillcolor{textcolor}%
\pgftext[x=0.704877in, y=1.501195in, left, base]{\color{textcolor}\rmfamily\fontsize{20.000000}{24.000000}\selectfont 2}%
\end{pgfscope}%
\begin{pgfscope}%
\pgfpathrectangle{\pgfqpoint{0.978830in}{0.894194in}}{\pgfqpoint{6.871170in}{4.119217in}}%
\pgfusepath{clip}%
\pgfsetroundcap%
\pgfsetroundjoin%
\pgfsetlinewidth{1.003750pt}%
\definecolor{currentstroke}{rgb}{1.000000,1.000000,1.000000}%
\pgfsetstrokecolor{currentstroke}%
\pgfsetdash{}{0pt}%
\pgfpathmoveto{\pgfqpoint{0.978830in}{2.241299in}}%
\pgfpathlineto{\pgfqpoint{7.850000in}{2.241299in}}%
\pgfusepath{stroke}%
\end{pgfscope}%
\begin{pgfscope}%
\definecolor{textcolor}{rgb}{0.150000,0.150000,0.150000}%
\pgfsetstrokecolor{textcolor}%
\pgfsetfillcolor{textcolor}%
\pgftext[x=0.704877in, y=2.135776in, left, base]{\color{textcolor}\rmfamily\fontsize{20.000000}{24.000000}\selectfont 4}%
\end{pgfscope}%
\begin{pgfscope}%
\pgfpathrectangle{\pgfqpoint{0.978830in}{0.894194in}}{\pgfqpoint{6.871170in}{4.119217in}}%
\pgfusepath{clip}%
\pgfsetroundcap%
\pgfsetroundjoin%
\pgfsetlinewidth{1.003750pt}%
\definecolor{currentstroke}{rgb}{1.000000,1.000000,1.000000}%
\pgfsetstrokecolor{currentstroke}%
\pgfsetdash{}{0pt}%
\pgfpathmoveto{\pgfqpoint{0.978830in}{2.875879in}}%
\pgfpathlineto{\pgfqpoint{7.850000in}{2.875879in}}%
\pgfusepath{stroke}%
\end{pgfscope}%
\begin{pgfscope}%
\definecolor{textcolor}{rgb}{0.150000,0.150000,0.150000}%
\pgfsetstrokecolor{textcolor}%
\pgfsetfillcolor{textcolor}%
\pgftext[x=0.704877in, y=2.770356in, left, base]{\color{textcolor}\rmfamily\fontsize{20.000000}{24.000000}\selectfont 6}%
\end{pgfscope}%
\begin{pgfscope}%
\pgfpathrectangle{\pgfqpoint{0.978830in}{0.894194in}}{\pgfqpoint{6.871170in}{4.119217in}}%
\pgfusepath{clip}%
\pgfsetroundcap%
\pgfsetroundjoin%
\pgfsetlinewidth{1.003750pt}%
\definecolor{currentstroke}{rgb}{1.000000,1.000000,1.000000}%
\pgfsetstrokecolor{currentstroke}%
\pgfsetdash{}{0pt}%
\pgfpathmoveto{\pgfqpoint{0.978830in}{3.510459in}}%
\pgfpathlineto{\pgfqpoint{7.850000in}{3.510459in}}%
\pgfusepath{stroke}%
\end{pgfscope}%
\begin{pgfscope}%
\definecolor{textcolor}{rgb}{0.150000,0.150000,0.150000}%
\pgfsetstrokecolor{textcolor}%
\pgfsetfillcolor{textcolor}%
\pgftext[x=0.704877in, y=3.404936in, left, base]{\color{textcolor}\rmfamily\fontsize{20.000000}{24.000000}\selectfont 8}%
\end{pgfscope}%
\begin{pgfscope}%
\pgfpathrectangle{\pgfqpoint{0.978830in}{0.894194in}}{\pgfqpoint{6.871170in}{4.119217in}}%
\pgfusepath{clip}%
\pgfsetroundcap%
\pgfsetroundjoin%
\pgfsetlinewidth{1.003750pt}%
\definecolor{currentstroke}{rgb}{1.000000,1.000000,1.000000}%
\pgfsetstrokecolor{currentstroke}%
\pgfsetdash{}{0pt}%
\pgfpathmoveto{\pgfqpoint{0.978830in}{4.145040in}}%
\pgfpathlineto{\pgfqpoint{7.850000in}{4.145040in}}%
\pgfusepath{stroke}%
\end{pgfscope}%
\begin{pgfscope}%
\definecolor{textcolor}{rgb}{0.150000,0.150000,0.150000}%
\pgfsetstrokecolor{textcolor}%
\pgfsetfillcolor{textcolor}%
\pgftext[x=0.528147in, y=4.039517in, left, base]{\color{textcolor}\rmfamily\fontsize{20.000000}{24.000000}\selectfont 10}%
\end{pgfscope}%
\begin{pgfscope}%
\pgfpathrectangle{\pgfqpoint{0.978830in}{0.894194in}}{\pgfqpoint{6.871170in}{4.119217in}}%
\pgfusepath{clip}%
\pgfsetroundcap%
\pgfsetroundjoin%
\pgfsetlinewidth{1.003750pt}%
\definecolor{currentstroke}{rgb}{1.000000,1.000000,1.000000}%
\pgfsetstrokecolor{currentstroke}%
\pgfsetdash{}{0pt}%
\pgfpathmoveto{\pgfqpoint{0.978830in}{4.779620in}}%
\pgfpathlineto{\pgfqpoint{7.850000in}{4.779620in}}%
\pgfusepath{stroke}%
\end{pgfscope}%
\begin{pgfscope}%
\definecolor{textcolor}{rgb}{0.150000,0.150000,0.150000}%
\pgfsetstrokecolor{textcolor}%
\pgfsetfillcolor{textcolor}%
\pgftext[x=0.528147in, y=4.674097in, left, base]{\color{textcolor}\rmfamily\fontsize{20.000000}{24.000000}\selectfont 12}%
\end{pgfscope}%
\begin{pgfscope}%
\definecolor{textcolor}{rgb}{0.150000,0.150000,0.150000}%
\pgfsetstrokecolor{textcolor}%
\pgfsetfillcolor{textcolor}%
\pgftext[x=0.472591in,y=2.953803in,,bottom,rotate=90.000000]{\color{textcolor}\rmfamily\fontsize{24.000000}{28.800000}\selectfont \(\displaystyle \rho\) (\(\displaystyle \mu\Omega\cdot\) m)}%
\end{pgfscope}%
\begin{pgfscope}%
\pgfpathrectangle{\pgfqpoint{0.978830in}{0.894194in}}{\pgfqpoint{6.871170in}{4.119217in}}%
\pgfusepath{clip}%
\pgfsetbuttcap%
\pgfsetroundjoin%
\definecolor{currentfill}{rgb}{0.298039,0.447059,0.690196}%
\pgfsetfillcolor{currentfill}%
\pgfsetlinewidth{0.301125pt}%
\definecolor{currentstroke}{rgb}{0.298039,0.447059,0.690196}%
\pgfsetstrokecolor{currentstroke}%
\pgfsetdash{}{0pt}%
\pgfsys@defobject{currentmarker}{\pgfqpoint{-0.031056in}{-0.031056in}}{\pgfqpoint{0.031056in}{0.031056in}}{%
\pgfpathmoveto{\pgfqpoint{0.000000in}{-0.031056in}}%
\pgfpathcurveto{\pgfqpoint{0.008236in}{-0.031056in}}{\pgfqpoint{0.016136in}{-0.027784in}}{\pgfqpoint{0.021960in}{-0.021960in}}%
\pgfpathcurveto{\pgfqpoint{0.027784in}{-0.016136in}}{\pgfqpoint{0.031056in}{-0.008236in}}{\pgfqpoint{0.031056in}{0.000000in}}%
\pgfpathcurveto{\pgfqpoint{0.031056in}{0.008236in}}{\pgfqpoint{0.027784in}{0.016136in}}{\pgfqpoint{0.021960in}{0.021960in}}%
\pgfpathcurveto{\pgfqpoint{0.016136in}{0.027784in}}{\pgfqpoint{0.008236in}{0.031056in}}{\pgfqpoint{0.000000in}{0.031056in}}%
\pgfpathcurveto{\pgfqpoint{-0.008236in}{0.031056in}}{\pgfqpoint{-0.016136in}{0.027784in}}{\pgfqpoint{-0.021960in}{0.021960in}}%
\pgfpathcurveto{\pgfqpoint{-0.027784in}{0.016136in}}{\pgfqpoint{-0.031056in}{0.008236in}}{\pgfqpoint{-0.031056in}{0.000000in}}%
\pgfpathcurveto{\pgfqpoint{-0.031056in}{-0.008236in}}{\pgfqpoint{-0.027784in}{-0.016136in}}{\pgfqpoint{-0.021960in}{-0.021960in}}%
\pgfpathcurveto{\pgfqpoint{-0.016136in}{-0.027784in}}{\pgfqpoint{-0.008236in}{-0.031056in}}{\pgfqpoint{0.000000in}{-0.031056in}}%
\pgfpathlineto{\pgfqpoint{0.000000in}{-0.031056in}}%
\pgfpathclose%
\pgfusepath{stroke,fill}%
}%
\begin{pgfscope}%
\pgfsys@transformshift{1.291156in}{1.409312in}%
\pgfsys@useobject{currentmarker}{}%
\end{pgfscope}%
\begin{pgfscope}%
\pgfsys@transformshift{1.670563in}{1.383427in}%
\pgfsys@useobject{currentmarker}{}%
\end{pgfscope}%
\begin{pgfscope}%
\pgfsys@transformshift{1.866133in}{1.403560in}%
\pgfsys@useobject{currentmarker}{}%
\end{pgfscope}%
\begin{pgfscope}%
\pgfsys@transformshift{1.905247in}{1.386303in}%
\pgfsys@useobject{currentmarker}{}%
\end{pgfscope}%
\begin{pgfscope}%
\pgfsys@transformshift{1.944361in}{1.366170in}%
\pgfsys@useobject{currentmarker}{}%
\end{pgfscope}%
\begin{pgfscope}%
\pgfsys@transformshift{1.983475in}{1.320152in}%
\pgfsys@useobject{currentmarker}{}%
\end{pgfscope}%
\begin{pgfscope}%
\pgfsys@transformshift{2.022589in}{1.182097in}%
\pgfsys@useobject{currentmarker}{}%
\end{pgfscope}%
\begin{pgfscope}%
\pgfsys@transformshift{2.061703in}{1.090060in}%
\pgfsys@useobject{currentmarker}{}%
\end{pgfscope}%
\begin{pgfscope}%
\pgfsys@transformshift{2.100817in}{1.662413in}%
\pgfsys@useobject{currentmarker}{}%
\end{pgfscope}%
\begin{pgfscope}%
\pgfsys@transformshift{2.139932in}{2.266404in}%
\pgfsys@useobject{currentmarker}{}%
\end{pgfscope}%
\begin{pgfscope}%
\pgfsys@transformshift{2.179046in}{2.631675in}%
\pgfsys@useobject{currentmarker}{}%
\end{pgfscope}%
\begin{pgfscope}%
\pgfsys@transformshift{2.218160in}{2.804243in}%
\pgfsys@useobject{currentmarker}{}%
\end{pgfscope}%
\begin{pgfscope}%
\pgfsys@transformshift{2.257274in}{2.853138in}%
\pgfsys@useobject{currentmarker}{}%
\end{pgfscope}%
\begin{pgfscope}%
\pgfsys@transformshift{2.296388in}{2.902032in}%
\pgfsys@useobject{currentmarker}{}%
\end{pgfscope}%
\begin{pgfscope}%
\pgfsys@transformshift{2.335502in}{2.939422in}%
\pgfsys@useobject{currentmarker}{}%
\end{pgfscope}%
\begin{pgfscope}%
\pgfsys@transformshift{2.374616in}{2.965308in}%
\pgfsys@useobject{currentmarker}{}%
\end{pgfscope}%
\begin{pgfscope}%
\pgfsys@transformshift{2.413730in}{2.994069in}%
\pgfsys@useobject{currentmarker}{}%
\end{pgfscope}%
\begin{pgfscope}%
\pgfsys@transformshift{2.452844in}{3.051592in}%
\pgfsys@useobject{currentmarker}{}%
\end{pgfscope}%
\begin{pgfscope}%
\pgfsys@transformshift{2.843985in}{3.267303in}%
\pgfsys@useobject{currentmarker}{}%
\end{pgfscope}%
\begin{pgfscope}%
\pgfsys@transformshift{3.235126in}{3.460005in}%
\pgfsys@useobject{currentmarker}{}%
\end{pgfscope}%
\begin{pgfscope}%
\pgfsys@transformshift{3.626266in}{3.618193in}%
\pgfsys@useobject{currentmarker}{}%
\end{pgfscope}%
\begin{pgfscope}%
\pgfsys@transformshift{4.017407in}{3.773505in}%
\pgfsys@useobject{currentmarker}{}%
\end{pgfscope}%
\begin{pgfscope}%
\pgfsys@transformshift{4.408548in}{3.905807in}%
\pgfsys@useobject{currentmarker}{}%
\end{pgfscope}%
\begin{pgfscope}%
\pgfsys@transformshift{5.190829in}{4.158908in}%
\pgfsys@useobject{currentmarker}{}%
\end{pgfscope}%
\begin{pgfscope}%
\pgfsys@transformshift{5.973111in}{4.397628in}%
\pgfsys@useobject{currentmarker}{}%
\end{pgfscope}%
\begin{pgfscope}%
\pgfsys@transformshift{6.755393in}{4.613339in}%
\pgfsys@useobject{currentmarker}{}%
\end{pgfscope}%
\begin{pgfscope}%
\pgfsys@transformshift{7.537674in}{4.826174in}%
\pgfsys@useobject{currentmarker}{}%
\end{pgfscope}%
\end{pgfscope}%
\begin{pgfscope}%
\pgfpathrectangle{\pgfqpoint{0.978830in}{0.894194in}}{\pgfqpoint{6.871170in}{4.119217in}}%
\pgfusepath{clip}%
\pgfsetbuttcap%
\pgfsetroundjoin%
\pgfsetlinewidth{1.756562pt}%
\definecolor{currentstroke}{rgb}{0.298039,0.447059,0.690196}%
\pgfsetstrokecolor{currentstroke}%
\pgfsetdash{}{0pt}%
\pgfpathmoveto{\pgfqpoint{1.291156in}{1.409312in}}%
\pgfpathlineto{\pgfqpoint{1.291156in}{1.409312in}}%
\pgfusepath{stroke}%
\end{pgfscope}%
\begin{pgfscope}%
\pgfpathrectangle{\pgfqpoint{0.978830in}{0.894194in}}{\pgfqpoint{6.871170in}{4.119217in}}%
\pgfusepath{clip}%
\pgfsetbuttcap%
\pgfsetroundjoin%
\pgfsetlinewidth{1.756562pt}%
\definecolor{currentstroke}{rgb}{0.298039,0.447059,0.690196}%
\pgfsetstrokecolor{currentstroke}%
\pgfsetdash{}{0pt}%
\pgfpathmoveto{\pgfqpoint{1.670563in}{1.374799in}}%
\pgfpathlineto{\pgfqpoint{1.670563in}{1.392055in}}%
\pgfusepath{stroke}%
\end{pgfscope}%
\begin{pgfscope}%
\pgfpathrectangle{\pgfqpoint{0.978830in}{0.894194in}}{\pgfqpoint{6.871170in}{4.119217in}}%
\pgfusepath{clip}%
\pgfsetbuttcap%
\pgfsetroundjoin%
\pgfsetlinewidth{1.756562pt}%
\definecolor{currentstroke}{rgb}{0.298039,0.447059,0.690196}%
\pgfsetstrokecolor{currentstroke}%
\pgfsetdash{}{0pt}%
\pgfpathmoveto{\pgfqpoint{1.866133in}{1.397808in}}%
\pgfpathlineto{\pgfqpoint{1.866133in}{1.409312in}}%
\pgfusepath{stroke}%
\end{pgfscope}%
\begin{pgfscope}%
\pgfpathrectangle{\pgfqpoint{0.978830in}{0.894194in}}{\pgfqpoint{6.871170in}{4.119217in}}%
\pgfusepath{clip}%
\pgfsetbuttcap%
\pgfsetroundjoin%
\pgfsetlinewidth{1.756562pt}%
\definecolor{currentstroke}{rgb}{0.298039,0.447059,0.690196}%
\pgfsetstrokecolor{currentstroke}%
\pgfsetdash{}{0pt}%
\pgfpathmoveto{\pgfqpoint{1.905247in}{1.386303in}}%
\pgfpathlineto{\pgfqpoint{1.905247in}{1.386303in}}%
\pgfusepath{stroke}%
\end{pgfscope}%
\begin{pgfscope}%
\pgfpathrectangle{\pgfqpoint{0.978830in}{0.894194in}}{\pgfqpoint{6.871170in}{4.119217in}}%
\pgfusepath{clip}%
\pgfsetbuttcap%
\pgfsetroundjoin%
\pgfsetlinewidth{1.756562pt}%
\definecolor{currentstroke}{rgb}{0.298039,0.447059,0.690196}%
\pgfsetstrokecolor{currentstroke}%
\pgfsetdash{}{0pt}%
\pgfpathmoveto{\pgfqpoint{1.944361in}{1.363294in}}%
\pgfpathlineto{\pgfqpoint{1.944361in}{1.369046in}}%
\pgfusepath{stroke}%
\end{pgfscope}%
\begin{pgfscope}%
\pgfpathrectangle{\pgfqpoint{0.978830in}{0.894194in}}{\pgfqpoint{6.871170in}{4.119217in}}%
\pgfusepath{clip}%
\pgfsetbuttcap%
\pgfsetroundjoin%
\pgfsetlinewidth{1.756562pt}%
\definecolor{currentstroke}{rgb}{0.298039,0.447059,0.690196}%
\pgfsetstrokecolor{currentstroke}%
\pgfsetdash{}{0pt}%
\pgfpathmoveto{\pgfqpoint{1.983475in}{1.317276in}}%
\pgfpathlineto{\pgfqpoint{1.983475in}{1.323028in}}%
\pgfusepath{stroke}%
\end{pgfscope}%
\begin{pgfscope}%
\pgfpathrectangle{\pgfqpoint{0.978830in}{0.894194in}}{\pgfqpoint{6.871170in}{4.119217in}}%
\pgfusepath{clip}%
\pgfsetbuttcap%
\pgfsetroundjoin%
\pgfsetlinewidth{1.756562pt}%
\definecolor{currentstroke}{rgb}{0.298039,0.447059,0.690196}%
\pgfsetstrokecolor{currentstroke}%
\pgfsetdash{}{0pt}%
\pgfpathmoveto{\pgfqpoint{2.022589in}{1.179221in}}%
\pgfpathlineto{\pgfqpoint{2.022589in}{1.184973in}}%
\pgfusepath{stroke}%
\end{pgfscope}%
\begin{pgfscope}%
\pgfpathrectangle{\pgfqpoint{0.978830in}{0.894194in}}{\pgfqpoint{6.871170in}{4.119217in}}%
\pgfusepath{clip}%
\pgfsetbuttcap%
\pgfsetroundjoin%
\pgfsetlinewidth{1.756562pt}%
\definecolor{currentstroke}{rgb}{0.298039,0.447059,0.690196}%
\pgfsetstrokecolor{currentstroke}%
\pgfsetdash{}{0pt}%
\pgfpathmoveto{\pgfqpoint{2.061703in}{1.081432in}}%
\pgfpathlineto{\pgfqpoint{2.061703in}{1.098688in}}%
\pgfusepath{stroke}%
\end{pgfscope}%
\begin{pgfscope}%
\pgfpathrectangle{\pgfqpoint{0.978830in}{0.894194in}}{\pgfqpoint{6.871170in}{4.119217in}}%
\pgfusepath{clip}%
\pgfsetbuttcap%
\pgfsetroundjoin%
\pgfsetlinewidth{1.756562pt}%
\definecolor{currentstroke}{rgb}{0.298039,0.447059,0.690196}%
\pgfsetstrokecolor{currentstroke}%
\pgfsetdash{}{0pt}%
\pgfpathmoveto{\pgfqpoint{2.100817in}{1.662413in}}%
\pgfpathlineto{\pgfqpoint{2.100817in}{1.662413in}}%
\pgfusepath{stroke}%
\end{pgfscope}%
\begin{pgfscope}%
\pgfpathrectangle{\pgfqpoint{0.978830in}{0.894194in}}{\pgfqpoint{6.871170in}{4.119217in}}%
\pgfusepath{clip}%
\pgfsetbuttcap%
\pgfsetroundjoin%
\pgfsetlinewidth{1.756562pt}%
\definecolor{currentstroke}{rgb}{0.298039,0.447059,0.690196}%
\pgfsetstrokecolor{currentstroke}%
\pgfsetdash{}{0pt}%
\pgfpathmoveto{\pgfqpoint{2.139932in}{2.266404in}}%
\pgfpathlineto{\pgfqpoint{2.139932in}{2.266404in}}%
\pgfusepath{stroke}%
\end{pgfscope}%
\begin{pgfscope}%
\pgfpathrectangle{\pgfqpoint{0.978830in}{0.894194in}}{\pgfqpoint{6.871170in}{4.119217in}}%
\pgfusepath{clip}%
\pgfsetbuttcap%
\pgfsetroundjoin%
\pgfsetlinewidth{1.756562pt}%
\definecolor{currentstroke}{rgb}{0.298039,0.447059,0.690196}%
\pgfsetstrokecolor{currentstroke}%
\pgfsetdash{}{0pt}%
\pgfpathmoveto{\pgfqpoint{2.179046in}{2.628798in}}%
\pgfpathlineto{\pgfqpoint{2.179046in}{2.634551in}}%
\pgfusepath{stroke}%
\end{pgfscope}%
\begin{pgfscope}%
\pgfpathrectangle{\pgfqpoint{0.978830in}{0.894194in}}{\pgfqpoint{6.871170in}{4.119217in}}%
\pgfusepath{clip}%
\pgfsetbuttcap%
\pgfsetroundjoin%
\pgfsetlinewidth{1.756562pt}%
\definecolor{currentstroke}{rgb}{0.298039,0.447059,0.690196}%
\pgfsetstrokecolor{currentstroke}%
\pgfsetdash{}{0pt}%
\pgfpathmoveto{\pgfqpoint{2.218160in}{2.801367in}}%
\pgfpathlineto{\pgfqpoint{2.218160in}{2.807119in}}%
\pgfusepath{stroke}%
\end{pgfscope}%
\begin{pgfscope}%
\pgfpathrectangle{\pgfqpoint{0.978830in}{0.894194in}}{\pgfqpoint{6.871170in}{4.119217in}}%
\pgfusepath{clip}%
\pgfsetbuttcap%
\pgfsetroundjoin%
\pgfsetlinewidth{1.756562pt}%
\definecolor{currentstroke}{rgb}{0.298039,0.447059,0.690196}%
\pgfsetstrokecolor{currentstroke}%
\pgfsetdash{}{0pt}%
\pgfpathmoveto{\pgfqpoint{2.257274in}{2.853138in}}%
\pgfpathlineto{\pgfqpoint{2.257274in}{2.853138in}}%
\pgfusepath{stroke}%
\end{pgfscope}%
\begin{pgfscope}%
\pgfpathrectangle{\pgfqpoint{0.978830in}{0.894194in}}{\pgfqpoint{6.871170in}{4.119217in}}%
\pgfusepath{clip}%
\pgfsetbuttcap%
\pgfsetroundjoin%
\pgfsetlinewidth{1.756562pt}%
\definecolor{currentstroke}{rgb}{0.298039,0.447059,0.690196}%
\pgfsetstrokecolor{currentstroke}%
\pgfsetdash{}{0pt}%
\pgfpathmoveto{\pgfqpoint{2.296388in}{2.899156in}}%
\pgfpathlineto{\pgfqpoint{2.296388in}{2.904908in}}%
\pgfusepath{stroke}%
\end{pgfscope}%
\begin{pgfscope}%
\pgfpathrectangle{\pgfqpoint{0.978830in}{0.894194in}}{\pgfqpoint{6.871170in}{4.119217in}}%
\pgfusepath{clip}%
\pgfsetbuttcap%
\pgfsetroundjoin%
\pgfsetlinewidth{1.756562pt}%
\definecolor{currentstroke}{rgb}{0.298039,0.447059,0.690196}%
\pgfsetstrokecolor{currentstroke}%
\pgfsetdash{}{0pt}%
\pgfpathmoveto{\pgfqpoint{2.335502in}{2.939422in}}%
\pgfpathlineto{\pgfqpoint{2.335502in}{2.939422in}}%
\pgfusepath{stroke}%
\end{pgfscope}%
\begin{pgfscope}%
\pgfpathrectangle{\pgfqpoint{0.978830in}{0.894194in}}{\pgfqpoint{6.871170in}{4.119217in}}%
\pgfusepath{clip}%
\pgfsetbuttcap%
\pgfsetroundjoin%
\pgfsetlinewidth{1.756562pt}%
\definecolor{currentstroke}{rgb}{0.298039,0.447059,0.690196}%
\pgfsetstrokecolor{currentstroke}%
\pgfsetdash{}{0pt}%
\pgfpathmoveto{\pgfqpoint{2.374616in}{2.962431in}}%
\pgfpathlineto{\pgfqpoint{2.374616in}{2.968184in}}%
\pgfusepath{stroke}%
\end{pgfscope}%
\begin{pgfscope}%
\pgfpathrectangle{\pgfqpoint{0.978830in}{0.894194in}}{\pgfqpoint{6.871170in}{4.119217in}}%
\pgfusepath{clip}%
\pgfsetbuttcap%
\pgfsetroundjoin%
\pgfsetlinewidth{1.756562pt}%
\definecolor{currentstroke}{rgb}{0.298039,0.447059,0.690196}%
\pgfsetstrokecolor{currentstroke}%
\pgfsetdash{}{0pt}%
\pgfpathmoveto{\pgfqpoint{2.413730in}{2.991193in}}%
\pgfpathlineto{\pgfqpoint{2.413730in}{2.996945in}}%
\pgfusepath{stroke}%
\end{pgfscope}%
\begin{pgfscope}%
\pgfpathrectangle{\pgfqpoint{0.978830in}{0.894194in}}{\pgfqpoint{6.871170in}{4.119217in}}%
\pgfusepath{clip}%
\pgfsetbuttcap%
\pgfsetroundjoin%
\pgfsetlinewidth{1.756562pt}%
\definecolor{currentstroke}{rgb}{0.298039,0.447059,0.690196}%
\pgfsetstrokecolor{currentstroke}%
\pgfsetdash{}{0pt}%
\pgfpathmoveto{\pgfqpoint{2.452844in}{3.025707in}}%
\pgfpathlineto{\pgfqpoint{2.452844in}{3.077477in}}%
\pgfusepath{stroke}%
\end{pgfscope}%
\begin{pgfscope}%
\pgfpathrectangle{\pgfqpoint{0.978830in}{0.894194in}}{\pgfqpoint{6.871170in}{4.119217in}}%
\pgfusepath{clip}%
\pgfsetbuttcap%
\pgfsetroundjoin%
\pgfsetlinewidth{1.756562pt}%
\definecolor{currentstroke}{rgb}{0.298039,0.447059,0.690196}%
\pgfsetstrokecolor{currentstroke}%
\pgfsetdash{}{0pt}%
\pgfpathmoveto{\pgfqpoint{2.843985in}{3.267303in}}%
\pgfpathlineto{\pgfqpoint{2.843985in}{3.267303in}}%
\pgfusepath{stroke}%
\end{pgfscope}%
\begin{pgfscope}%
\pgfpathrectangle{\pgfqpoint{0.978830in}{0.894194in}}{\pgfqpoint{6.871170in}{4.119217in}}%
\pgfusepath{clip}%
\pgfsetbuttcap%
\pgfsetroundjoin%
\pgfsetlinewidth{1.756562pt}%
\definecolor{currentstroke}{rgb}{0.298039,0.447059,0.690196}%
\pgfsetstrokecolor{currentstroke}%
\pgfsetdash{}{0pt}%
\pgfpathmoveto{\pgfqpoint{3.235126in}{3.457129in}}%
\pgfpathlineto{\pgfqpoint{3.235126in}{3.462881in}}%
\pgfusepath{stroke}%
\end{pgfscope}%
\begin{pgfscope}%
\pgfpathrectangle{\pgfqpoint{0.978830in}{0.894194in}}{\pgfqpoint{6.871170in}{4.119217in}}%
\pgfusepath{clip}%
\pgfsetbuttcap%
\pgfsetroundjoin%
\pgfsetlinewidth{1.756562pt}%
\definecolor{currentstroke}{rgb}{0.298039,0.447059,0.690196}%
\pgfsetstrokecolor{currentstroke}%
\pgfsetdash{}{0pt}%
\pgfpathmoveto{\pgfqpoint{3.626266in}{3.618193in}}%
\pgfpathlineto{\pgfqpoint{3.626266in}{3.618193in}}%
\pgfusepath{stroke}%
\end{pgfscope}%
\begin{pgfscope}%
\pgfpathrectangle{\pgfqpoint{0.978830in}{0.894194in}}{\pgfqpoint{6.871170in}{4.119217in}}%
\pgfusepath{clip}%
\pgfsetbuttcap%
\pgfsetroundjoin%
\pgfsetlinewidth{1.756562pt}%
\definecolor{currentstroke}{rgb}{0.298039,0.447059,0.690196}%
\pgfsetstrokecolor{currentstroke}%
\pgfsetdash{}{0pt}%
\pgfpathmoveto{\pgfqpoint{4.017407in}{3.773505in}}%
\pgfpathlineto{\pgfqpoint{4.017407in}{3.773505in}}%
\pgfusepath{stroke}%
\end{pgfscope}%
\begin{pgfscope}%
\pgfpathrectangle{\pgfqpoint{0.978830in}{0.894194in}}{\pgfqpoint{6.871170in}{4.119217in}}%
\pgfusepath{clip}%
\pgfsetbuttcap%
\pgfsetroundjoin%
\pgfsetlinewidth{1.756562pt}%
\definecolor{currentstroke}{rgb}{0.298039,0.447059,0.690196}%
\pgfsetstrokecolor{currentstroke}%
\pgfsetdash{}{0pt}%
\pgfpathmoveto{\pgfqpoint{4.408548in}{3.905807in}}%
\pgfpathlineto{\pgfqpoint{4.408548in}{3.905807in}}%
\pgfusepath{stroke}%
\end{pgfscope}%
\begin{pgfscope}%
\pgfpathrectangle{\pgfqpoint{0.978830in}{0.894194in}}{\pgfqpoint{6.871170in}{4.119217in}}%
\pgfusepath{clip}%
\pgfsetbuttcap%
\pgfsetroundjoin%
\pgfsetlinewidth{1.756562pt}%
\definecolor{currentstroke}{rgb}{0.298039,0.447059,0.690196}%
\pgfsetstrokecolor{currentstroke}%
\pgfsetdash{}{0pt}%
\pgfpathmoveto{\pgfqpoint{5.190829in}{4.158908in}}%
\pgfpathlineto{\pgfqpoint{5.190829in}{4.158908in}}%
\pgfusepath{stroke}%
\end{pgfscope}%
\begin{pgfscope}%
\pgfpathrectangle{\pgfqpoint{0.978830in}{0.894194in}}{\pgfqpoint{6.871170in}{4.119217in}}%
\pgfusepath{clip}%
\pgfsetbuttcap%
\pgfsetroundjoin%
\pgfsetlinewidth{1.756562pt}%
\definecolor{currentstroke}{rgb}{0.298039,0.447059,0.690196}%
\pgfsetstrokecolor{currentstroke}%
\pgfsetdash{}{0pt}%
\pgfpathmoveto{\pgfqpoint{5.973111in}{4.394752in}}%
\pgfpathlineto{\pgfqpoint{5.973111in}{4.400505in}}%
\pgfusepath{stroke}%
\end{pgfscope}%
\begin{pgfscope}%
\pgfpathrectangle{\pgfqpoint{0.978830in}{0.894194in}}{\pgfqpoint{6.871170in}{4.119217in}}%
\pgfusepath{clip}%
\pgfsetbuttcap%
\pgfsetroundjoin%
\pgfsetlinewidth{1.756562pt}%
\definecolor{currentstroke}{rgb}{0.298039,0.447059,0.690196}%
\pgfsetstrokecolor{currentstroke}%
\pgfsetdash{}{0pt}%
\pgfpathmoveto{\pgfqpoint{6.755393in}{4.613339in}}%
\pgfpathlineto{\pgfqpoint{6.755393in}{4.613339in}}%
\pgfusepath{stroke}%
\end{pgfscope}%
\begin{pgfscope}%
\pgfpathrectangle{\pgfqpoint{0.978830in}{0.894194in}}{\pgfqpoint{6.871170in}{4.119217in}}%
\pgfusepath{clip}%
\pgfsetbuttcap%
\pgfsetroundjoin%
\pgfsetlinewidth{1.756562pt}%
\definecolor{currentstroke}{rgb}{0.298039,0.447059,0.690196}%
\pgfsetstrokecolor{currentstroke}%
\pgfsetdash{}{0pt}%
\pgfpathmoveto{\pgfqpoint{7.537674in}{4.826174in}}%
\pgfpathlineto{\pgfqpoint{7.537674in}{4.826174in}}%
\pgfusepath{stroke}%
\end{pgfscope}%
\begin{pgfscope}%
\pgfpathrectangle{\pgfqpoint{0.978830in}{0.894194in}}{\pgfqpoint{6.871170in}{4.119217in}}%
\pgfusepath{clip}%
\pgfsetbuttcap%
\pgfsetroundjoin%
\definecolor{currentfill}{rgb}{0.298039,0.447059,0.690196}%
\pgfsetfillcolor{currentfill}%
\pgfsetlinewidth{1.003750pt}%
\definecolor{currentstroke}{rgb}{0.298039,0.447059,0.690196}%
\pgfsetstrokecolor{currentstroke}%
\pgfsetdash{}{0pt}%
\pgfsys@defobject{currentmarker}{\pgfqpoint{-0.069444in}{-0.000000in}}{\pgfqpoint{0.069444in}{0.000000in}}{%
\pgfpathmoveto{\pgfqpoint{0.069444in}{-0.000000in}}%
\pgfpathlineto{\pgfqpoint{-0.069444in}{0.000000in}}%
\pgfusepath{stroke,fill}%
}%
\begin{pgfscope}%
\pgfsys@transformshift{1.291156in}{1.409312in}%
\pgfsys@useobject{currentmarker}{}%
\end{pgfscope}%
\begin{pgfscope}%
\pgfsys@transformshift{1.670563in}{1.374799in}%
\pgfsys@useobject{currentmarker}{}%
\end{pgfscope}%
\begin{pgfscope}%
\pgfsys@transformshift{1.866133in}{1.397808in}%
\pgfsys@useobject{currentmarker}{}%
\end{pgfscope}%
\begin{pgfscope}%
\pgfsys@transformshift{1.905247in}{1.386303in}%
\pgfsys@useobject{currentmarker}{}%
\end{pgfscope}%
\begin{pgfscope}%
\pgfsys@transformshift{1.944361in}{1.363294in}%
\pgfsys@useobject{currentmarker}{}%
\end{pgfscope}%
\begin{pgfscope}%
\pgfsys@transformshift{1.983475in}{1.317276in}%
\pgfsys@useobject{currentmarker}{}%
\end{pgfscope}%
\begin{pgfscope}%
\pgfsys@transformshift{2.022589in}{1.179221in}%
\pgfsys@useobject{currentmarker}{}%
\end{pgfscope}%
\begin{pgfscope}%
\pgfsys@transformshift{2.061703in}{1.081432in}%
\pgfsys@useobject{currentmarker}{}%
\end{pgfscope}%
\begin{pgfscope}%
\pgfsys@transformshift{2.100817in}{1.662413in}%
\pgfsys@useobject{currentmarker}{}%
\end{pgfscope}%
\begin{pgfscope}%
\pgfsys@transformshift{2.139932in}{2.266404in}%
\pgfsys@useobject{currentmarker}{}%
\end{pgfscope}%
\begin{pgfscope}%
\pgfsys@transformshift{2.179046in}{2.628798in}%
\pgfsys@useobject{currentmarker}{}%
\end{pgfscope}%
\begin{pgfscope}%
\pgfsys@transformshift{2.218160in}{2.801367in}%
\pgfsys@useobject{currentmarker}{}%
\end{pgfscope}%
\begin{pgfscope}%
\pgfsys@transformshift{2.257274in}{2.853138in}%
\pgfsys@useobject{currentmarker}{}%
\end{pgfscope}%
\begin{pgfscope}%
\pgfsys@transformshift{2.296388in}{2.899156in}%
\pgfsys@useobject{currentmarker}{}%
\end{pgfscope}%
\begin{pgfscope}%
\pgfsys@transformshift{2.335502in}{2.939422in}%
\pgfsys@useobject{currentmarker}{}%
\end{pgfscope}%
\begin{pgfscope}%
\pgfsys@transformshift{2.374616in}{2.962431in}%
\pgfsys@useobject{currentmarker}{}%
\end{pgfscope}%
\begin{pgfscope}%
\pgfsys@transformshift{2.413730in}{2.991193in}%
\pgfsys@useobject{currentmarker}{}%
\end{pgfscope}%
\begin{pgfscope}%
\pgfsys@transformshift{2.452844in}{3.025707in}%
\pgfsys@useobject{currentmarker}{}%
\end{pgfscope}%
\begin{pgfscope}%
\pgfsys@transformshift{2.843985in}{3.267303in}%
\pgfsys@useobject{currentmarker}{}%
\end{pgfscope}%
\begin{pgfscope}%
\pgfsys@transformshift{3.235126in}{3.457129in}%
\pgfsys@useobject{currentmarker}{}%
\end{pgfscope}%
\begin{pgfscope}%
\pgfsys@transformshift{3.626266in}{3.618193in}%
\pgfsys@useobject{currentmarker}{}%
\end{pgfscope}%
\begin{pgfscope}%
\pgfsys@transformshift{4.017407in}{3.773505in}%
\pgfsys@useobject{currentmarker}{}%
\end{pgfscope}%
\begin{pgfscope}%
\pgfsys@transformshift{4.408548in}{3.905807in}%
\pgfsys@useobject{currentmarker}{}%
\end{pgfscope}%
\begin{pgfscope}%
\pgfsys@transformshift{5.190829in}{4.158908in}%
\pgfsys@useobject{currentmarker}{}%
\end{pgfscope}%
\begin{pgfscope}%
\pgfsys@transformshift{5.973111in}{4.394752in}%
\pgfsys@useobject{currentmarker}{}%
\end{pgfscope}%
\begin{pgfscope}%
\pgfsys@transformshift{6.755393in}{4.613339in}%
\pgfsys@useobject{currentmarker}{}%
\end{pgfscope}%
\begin{pgfscope}%
\pgfsys@transformshift{7.537674in}{4.826174in}%
\pgfsys@useobject{currentmarker}{}%
\end{pgfscope}%
\end{pgfscope}%
\begin{pgfscope}%
\pgfpathrectangle{\pgfqpoint{0.978830in}{0.894194in}}{\pgfqpoint{6.871170in}{4.119217in}}%
\pgfusepath{clip}%
\pgfsetbuttcap%
\pgfsetroundjoin%
\definecolor{currentfill}{rgb}{0.298039,0.447059,0.690196}%
\pgfsetfillcolor{currentfill}%
\pgfsetlinewidth{1.003750pt}%
\definecolor{currentstroke}{rgb}{0.298039,0.447059,0.690196}%
\pgfsetstrokecolor{currentstroke}%
\pgfsetdash{}{0pt}%
\pgfsys@defobject{currentmarker}{\pgfqpoint{-0.069444in}{-0.000000in}}{\pgfqpoint{0.069444in}{0.000000in}}{%
\pgfpathmoveto{\pgfqpoint{0.069444in}{-0.000000in}}%
\pgfpathlineto{\pgfqpoint{-0.069444in}{0.000000in}}%
\pgfusepath{stroke,fill}%
}%
\begin{pgfscope}%
\pgfsys@transformshift{1.291156in}{1.409312in}%
\pgfsys@useobject{currentmarker}{}%
\end{pgfscope}%
\begin{pgfscope}%
\pgfsys@transformshift{1.670563in}{1.392055in}%
\pgfsys@useobject{currentmarker}{}%
\end{pgfscope}%
\begin{pgfscope}%
\pgfsys@transformshift{1.866133in}{1.409312in}%
\pgfsys@useobject{currentmarker}{}%
\end{pgfscope}%
\begin{pgfscope}%
\pgfsys@transformshift{1.905247in}{1.386303in}%
\pgfsys@useobject{currentmarker}{}%
\end{pgfscope}%
\begin{pgfscope}%
\pgfsys@transformshift{1.944361in}{1.369046in}%
\pgfsys@useobject{currentmarker}{}%
\end{pgfscope}%
\begin{pgfscope}%
\pgfsys@transformshift{1.983475in}{1.323028in}%
\pgfsys@useobject{currentmarker}{}%
\end{pgfscope}%
\begin{pgfscope}%
\pgfsys@transformshift{2.022589in}{1.184973in}%
\pgfsys@useobject{currentmarker}{}%
\end{pgfscope}%
\begin{pgfscope}%
\pgfsys@transformshift{2.061703in}{1.098688in}%
\pgfsys@useobject{currentmarker}{}%
\end{pgfscope}%
\begin{pgfscope}%
\pgfsys@transformshift{2.100817in}{1.662413in}%
\pgfsys@useobject{currentmarker}{}%
\end{pgfscope}%
\begin{pgfscope}%
\pgfsys@transformshift{2.139932in}{2.266404in}%
\pgfsys@useobject{currentmarker}{}%
\end{pgfscope}%
\begin{pgfscope}%
\pgfsys@transformshift{2.179046in}{2.634551in}%
\pgfsys@useobject{currentmarker}{}%
\end{pgfscope}%
\begin{pgfscope}%
\pgfsys@transformshift{2.218160in}{2.807119in}%
\pgfsys@useobject{currentmarker}{}%
\end{pgfscope}%
\begin{pgfscope}%
\pgfsys@transformshift{2.257274in}{2.853138in}%
\pgfsys@useobject{currentmarker}{}%
\end{pgfscope}%
\begin{pgfscope}%
\pgfsys@transformshift{2.296388in}{2.904908in}%
\pgfsys@useobject{currentmarker}{}%
\end{pgfscope}%
\begin{pgfscope}%
\pgfsys@transformshift{2.335502in}{2.939422in}%
\pgfsys@useobject{currentmarker}{}%
\end{pgfscope}%
\begin{pgfscope}%
\pgfsys@transformshift{2.374616in}{2.968184in}%
\pgfsys@useobject{currentmarker}{}%
\end{pgfscope}%
\begin{pgfscope}%
\pgfsys@transformshift{2.413730in}{2.996945in}%
\pgfsys@useobject{currentmarker}{}%
\end{pgfscope}%
\begin{pgfscope}%
\pgfsys@transformshift{2.452844in}{3.077477in}%
\pgfsys@useobject{currentmarker}{}%
\end{pgfscope}%
\begin{pgfscope}%
\pgfsys@transformshift{2.843985in}{3.267303in}%
\pgfsys@useobject{currentmarker}{}%
\end{pgfscope}%
\begin{pgfscope}%
\pgfsys@transformshift{3.235126in}{3.462881in}%
\pgfsys@useobject{currentmarker}{}%
\end{pgfscope}%
\begin{pgfscope}%
\pgfsys@transformshift{3.626266in}{3.618193in}%
\pgfsys@useobject{currentmarker}{}%
\end{pgfscope}%
\begin{pgfscope}%
\pgfsys@transformshift{4.017407in}{3.773505in}%
\pgfsys@useobject{currentmarker}{}%
\end{pgfscope}%
\begin{pgfscope}%
\pgfsys@transformshift{4.408548in}{3.905807in}%
\pgfsys@useobject{currentmarker}{}%
\end{pgfscope}%
\begin{pgfscope}%
\pgfsys@transformshift{5.190829in}{4.158908in}%
\pgfsys@useobject{currentmarker}{}%
\end{pgfscope}%
\begin{pgfscope}%
\pgfsys@transformshift{5.973111in}{4.400505in}%
\pgfsys@useobject{currentmarker}{}%
\end{pgfscope}%
\begin{pgfscope}%
\pgfsys@transformshift{6.755393in}{4.613339in}%
\pgfsys@useobject{currentmarker}{}%
\end{pgfscope}%
\begin{pgfscope}%
\pgfsys@transformshift{7.537674in}{4.826174in}%
\pgfsys@useobject{currentmarker}{}%
\end{pgfscope}%
\end{pgfscope}%
\begin{pgfscope}%
\pgfpathrectangle{\pgfqpoint{0.978830in}{0.894194in}}{\pgfqpoint{6.871170in}{4.119217in}}%
\pgfusepath{clip}%
\pgfsetroundcap%
\pgfsetroundjoin%
\pgfsetlinewidth{1.756562pt}%
\definecolor{currentstroke}{rgb}{0.298039,0.447059,0.690196}%
\pgfsetstrokecolor{currentstroke}%
\pgfsetdash{}{0pt}%
\pgfpathmoveto{\pgfqpoint{1.291156in}{1.409312in}}%
\pgfpathlineto{\pgfqpoint{1.670563in}{1.383427in}}%
\pgfpathlineto{\pgfqpoint{1.866133in}{1.403560in}}%
\pgfpathlineto{\pgfqpoint{1.905247in}{1.386303in}}%
\pgfpathlineto{\pgfqpoint{1.944361in}{1.366170in}}%
\pgfpathlineto{\pgfqpoint{1.983475in}{1.320152in}}%
\pgfpathlineto{\pgfqpoint{2.022589in}{1.182097in}}%
\pgfpathlineto{\pgfqpoint{2.061703in}{1.090060in}}%
\pgfpathlineto{\pgfqpoint{2.100817in}{1.662413in}}%
\pgfpathlineto{\pgfqpoint{2.139932in}{2.266404in}}%
\pgfpathlineto{\pgfqpoint{2.179046in}{2.631675in}}%
\pgfpathlineto{\pgfqpoint{2.218160in}{2.804243in}}%
\pgfpathlineto{\pgfqpoint{2.257274in}{2.853138in}}%
\pgfpathlineto{\pgfqpoint{2.296388in}{2.902032in}}%
\pgfpathlineto{\pgfqpoint{2.335502in}{2.939422in}}%
\pgfpathlineto{\pgfqpoint{2.374616in}{2.965308in}}%
\pgfpathlineto{\pgfqpoint{2.413730in}{2.994069in}}%
\pgfpathlineto{\pgfqpoint{2.452844in}{3.051592in}}%
\pgfpathlineto{\pgfqpoint{2.843985in}{3.267303in}}%
\pgfpathlineto{\pgfqpoint{3.235126in}{3.460005in}}%
\pgfpathlineto{\pgfqpoint{3.626266in}{3.618193in}}%
\pgfpathlineto{\pgfqpoint{4.017407in}{3.773505in}}%
\pgfpathlineto{\pgfqpoint{4.408548in}{3.905807in}}%
\pgfpathlineto{\pgfqpoint{5.190829in}{4.158908in}}%
\pgfpathlineto{\pgfqpoint{5.973111in}{4.397628in}}%
\pgfpathlineto{\pgfqpoint{6.755393in}{4.613339in}}%
\pgfpathlineto{\pgfqpoint{7.537674in}{4.826174in}}%
\pgfusepath{stroke}%
\end{pgfscope}%
\begin{pgfscope}%
\pgfsetrectcap%
\pgfsetmiterjoin%
\pgfsetlinewidth{0.000000pt}%
\definecolor{currentstroke}{rgb}{1.000000,1.000000,1.000000}%
\pgfsetstrokecolor{currentstroke}%
\pgfsetdash{}{0pt}%
\pgfpathmoveto{\pgfqpoint{0.978830in}{0.894194in}}%
\pgfpathlineto{\pgfqpoint{0.978830in}{5.013411in}}%
\pgfusepath{}%
\end{pgfscope}%
\begin{pgfscope}%
\pgfsetrectcap%
\pgfsetmiterjoin%
\pgfsetlinewidth{0.000000pt}%
\definecolor{currentstroke}{rgb}{1.000000,1.000000,1.000000}%
\pgfsetstrokecolor{currentstroke}%
\pgfsetdash{}{0pt}%
\pgfpathmoveto{\pgfqpoint{7.850000in}{0.894194in}}%
\pgfpathlineto{\pgfqpoint{7.850000in}{5.013411in}}%
\pgfusepath{}%
\end{pgfscope}%
\begin{pgfscope}%
\pgfsetrectcap%
\pgfsetmiterjoin%
\pgfsetlinewidth{0.000000pt}%
\definecolor{currentstroke}{rgb}{1.000000,1.000000,1.000000}%
\pgfsetstrokecolor{currentstroke}%
\pgfsetdash{}{0pt}%
\pgfpathmoveto{\pgfqpoint{0.978830in}{0.894194in}}%
\pgfpathlineto{\pgfqpoint{7.850000in}{0.894194in}}%
\pgfusepath{}%
\end{pgfscope}%
\begin{pgfscope}%
\pgfsetrectcap%
\pgfsetmiterjoin%
\pgfsetlinewidth{0.000000pt}%
\definecolor{currentstroke}{rgb}{1.000000,1.000000,1.000000}%
\pgfsetstrokecolor{currentstroke}%
\pgfsetdash{}{0pt}%
\pgfpathmoveto{\pgfqpoint{0.978830in}{5.013411in}}%
\pgfpathlineto{\pgfqpoint{7.850000in}{5.013411in}}%
\pgfusepath{}%
\end{pgfscope}%
\begin{pgfscope}%
\definecolor{textcolor}{rgb}{0.150000,0.150000,0.150000}%
\pgfsetstrokecolor{textcolor}%
\pgfsetfillcolor{textcolor}%
\pgftext[x=4.414415in,y=5.096745in,,base]{\color{textcolor}\rmfamily\fontsize{24.000000}{28.800000}\selectfont Resistivity of YBa\(\displaystyle _2\)Cu\(\displaystyle _3\)O\(\displaystyle _7\)}%
\end{pgfscope}%
\end{pgfpicture}%
\makeatother%
\endgroup%
}
        \caption{this is a test of the captions to se if  HOLDER}
        \label{fig:ybco_temp}
    \end{figure}

    In contrast to before, YBCO is a type-1 superconductor with a $T_c \approx 90~K$, so its SIT should be first order and sharp like a step function. This is largely reflected in the resistivity plot as its transition begins at 92 K and is complete at 90 K. However there is a spike in the resistivity \textit{after} the phase transition (fig \ref{fig:ybco_temp}). One possible source for this error is that a contact wasn't perfect, thus the residual resistivity that we normally see from the carbon paste was accented. Another possible source is that the setting on the LA were incorrect when taking those data points. 


  \subsection*{Applied Magnetic Field}

    After establishing a baseline of the phase transition over a large temperature sweep, we limited the temperature range and applied an external magnetic field  to see its influence on the SIT. The expected effect is that phase transition would shift to a lower temperature. 

    \begin{figure}[t]
      \resizebox{0.45\textwidth}{!}{%% Creator: Matplotlib, PGF backend
%%
%% To include the figure in your LaTeX document, write
%%   \input{<filename>.pgf}
%%
%% Make sure the required packages are loaded in your preamble
%%   \usepackage{pgf}
%%
%% Also ensure that all the required font packages are loaded; for instance,
%% the lmodern package is sometimes necessary when using math font.
%%   \usepackage{lmodern}
%%
%% Figures using additional raster images can only be included by \input if
%% they are in the same directory as the main LaTeX file. For loading figures
%% from other directories you can use the `import` package
%%   \usepackage{import}
%%
%% and then include the figures with
%%   \import{<path to file>}{<filename>.pgf}
%%
%% Matplotlib used the following preamble
%%   \usepackage{fontspec}
%%   \setmainfont{DejaVuSerif.ttf}[Path=\detokenize{/usr/share/matplotlib/mpl-data/fonts/ttf/}]
%%   \setsansfont{DejaVuSans.ttf}[Path=\detokenize{/usr/share/matplotlib/mpl-data/fonts/ttf/}]
%%   \setmonofont{DejaVuSansMono.ttf}[Path=\detokenize{/usr/share/matplotlib/mpl-data/fonts/ttf/}]
%%
\begingroup%
\makeatletter%
\begin{pgfpicture}%
\pgfpathrectangle{\pgfpointorigin}{\pgfqpoint{8.000000in}{5.500000in}}%
\pgfusepath{use as bounding box, clip}%
\begin{pgfscope}%
\pgfsetbuttcap%
\pgfsetmiterjoin%
\definecolor{currentfill}{rgb}{1.000000,1.000000,1.000000}%
\pgfsetfillcolor{currentfill}%
\pgfsetlinewidth{0.000000pt}%
\definecolor{currentstroke}{rgb}{1.000000,1.000000,1.000000}%
\pgfsetstrokecolor{currentstroke}%
\pgfsetdash{}{0pt}%
\pgfpathmoveto{\pgfqpoint{0.000000in}{0.000000in}}%
\pgfpathlineto{\pgfqpoint{8.000000in}{0.000000in}}%
\pgfpathlineto{\pgfqpoint{8.000000in}{5.500000in}}%
\pgfpathlineto{\pgfqpoint{0.000000in}{5.500000in}}%
\pgfpathlineto{\pgfqpoint{0.000000in}{0.000000in}}%
\pgfpathclose%
\pgfusepath{fill}%
\end{pgfscope}%
\begin{pgfscope}%
\pgfsetbuttcap%
\pgfsetmiterjoin%
\definecolor{currentfill}{rgb}{0.917647,0.917647,0.949020}%
\pgfsetfillcolor{currentfill}%
\pgfsetlinewidth{0.000000pt}%
\definecolor{currentstroke}{rgb}{0.000000,0.000000,0.000000}%
\pgfsetstrokecolor{currentstroke}%
\pgfsetstrokeopacity{0.000000}%
\pgfsetdash{}{0pt}%
\pgfpathmoveto{\pgfqpoint{0.978830in}{0.894194in}}%
\pgfpathlineto{\pgfqpoint{7.850000in}{0.894194in}}%
\pgfpathlineto{\pgfqpoint{7.850000in}{4.635449in}}%
\pgfpathlineto{\pgfqpoint{0.978830in}{4.635449in}}%
\pgfpathlineto{\pgfqpoint{0.978830in}{0.894194in}}%
\pgfpathclose%
\pgfusepath{fill}%
\end{pgfscope}%
\begin{pgfscope}%
\pgfpathrectangle{\pgfqpoint{0.978830in}{0.894194in}}{\pgfqpoint{6.871170in}{3.741255in}}%
\pgfusepath{clip}%
\pgfsetroundcap%
\pgfsetroundjoin%
\pgfsetlinewidth{1.003750pt}%
\definecolor{currentstroke}{rgb}{1.000000,1.000000,1.000000}%
\pgfsetstrokecolor{currentstroke}%
\pgfsetdash{}{0pt}%
\pgfpathmoveto{\pgfqpoint{1.291156in}{0.894194in}}%
\pgfpathlineto{\pgfqpoint{1.291156in}{4.635449in}}%
\pgfusepath{stroke}%
\end{pgfscope}%
\begin{pgfscope}%
\definecolor{textcolor}{rgb}{0.150000,0.150000,0.150000}%
\pgfsetstrokecolor{textcolor}%
\pgfsetfillcolor{textcolor}%
\pgftext[x=1.291156in,y=0.796972in,,top]{\color{textcolor}\rmfamily\fontsize{20.000000}{24.000000}\selectfont 80}%
\end{pgfscope}%
\begin{pgfscope}%
\pgfpathrectangle{\pgfqpoint{0.978830in}{0.894194in}}{\pgfqpoint{6.871170in}{3.741255in}}%
\pgfusepath{clip}%
\pgfsetroundcap%
\pgfsetroundjoin%
\pgfsetlinewidth{1.003750pt}%
\definecolor{currentstroke}{rgb}{1.000000,1.000000,1.000000}%
\pgfsetstrokecolor{currentstroke}%
\pgfsetdash{}{0pt}%
\pgfpathmoveto{\pgfqpoint{2.298659in}{0.894194in}}%
\pgfpathlineto{\pgfqpoint{2.298659in}{4.635449in}}%
\pgfusepath{stroke}%
\end{pgfscope}%
\begin{pgfscope}%
\definecolor{textcolor}{rgb}{0.150000,0.150000,0.150000}%
\pgfsetstrokecolor{textcolor}%
\pgfsetfillcolor{textcolor}%
\pgftext[x=2.298659in,y=0.796972in,,top]{\color{textcolor}\rmfamily\fontsize{20.000000}{24.000000}\selectfont 85}%
\end{pgfscope}%
\begin{pgfscope}%
\pgfpathrectangle{\pgfqpoint{0.978830in}{0.894194in}}{\pgfqpoint{6.871170in}{3.741255in}}%
\pgfusepath{clip}%
\pgfsetroundcap%
\pgfsetroundjoin%
\pgfsetlinewidth{1.003750pt}%
\definecolor{currentstroke}{rgb}{1.000000,1.000000,1.000000}%
\pgfsetstrokecolor{currentstroke}%
\pgfsetdash{}{0pt}%
\pgfpathmoveto{\pgfqpoint{3.306162in}{0.894194in}}%
\pgfpathlineto{\pgfqpoint{3.306162in}{4.635449in}}%
\pgfusepath{stroke}%
\end{pgfscope}%
\begin{pgfscope}%
\definecolor{textcolor}{rgb}{0.150000,0.150000,0.150000}%
\pgfsetstrokecolor{textcolor}%
\pgfsetfillcolor{textcolor}%
\pgftext[x=3.306162in,y=0.796972in,,top]{\color{textcolor}\rmfamily\fontsize{20.000000}{24.000000}\selectfont 90}%
\end{pgfscope}%
\begin{pgfscope}%
\pgfpathrectangle{\pgfqpoint{0.978830in}{0.894194in}}{\pgfqpoint{6.871170in}{3.741255in}}%
\pgfusepath{clip}%
\pgfsetroundcap%
\pgfsetroundjoin%
\pgfsetlinewidth{1.003750pt}%
\definecolor{currentstroke}{rgb}{1.000000,1.000000,1.000000}%
\pgfsetstrokecolor{currentstroke}%
\pgfsetdash{}{0pt}%
\pgfpathmoveto{\pgfqpoint{4.313665in}{0.894194in}}%
\pgfpathlineto{\pgfqpoint{4.313665in}{4.635449in}}%
\pgfusepath{stroke}%
\end{pgfscope}%
\begin{pgfscope}%
\definecolor{textcolor}{rgb}{0.150000,0.150000,0.150000}%
\pgfsetstrokecolor{textcolor}%
\pgfsetfillcolor{textcolor}%
\pgftext[x=4.313665in,y=0.796972in,,top]{\color{textcolor}\rmfamily\fontsize{20.000000}{24.000000}\selectfont 95}%
\end{pgfscope}%
\begin{pgfscope}%
\pgfpathrectangle{\pgfqpoint{0.978830in}{0.894194in}}{\pgfqpoint{6.871170in}{3.741255in}}%
\pgfusepath{clip}%
\pgfsetroundcap%
\pgfsetroundjoin%
\pgfsetlinewidth{1.003750pt}%
\definecolor{currentstroke}{rgb}{1.000000,1.000000,1.000000}%
\pgfsetstrokecolor{currentstroke}%
\pgfsetdash{}{0pt}%
\pgfpathmoveto{\pgfqpoint{5.321168in}{0.894194in}}%
\pgfpathlineto{\pgfqpoint{5.321168in}{4.635449in}}%
\pgfusepath{stroke}%
\end{pgfscope}%
\begin{pgfscope}%
\definecolor{textcolor}{rgb}{0.150000,0.150000,0.150000}%
\pgfsetstrokecolor{textcolor}%
\pgfsetfillcolor{textcolor}%
\pgftext[x=5.321168in,y=0.796972in,,top]{\color{textcolor}\rmfamily\fontsize{20.000000}{24.000000}\selectfont 100}%
\end{pgfscope}%
\begin{pgfscope}%
\pgfpathrectangle{\pgfqpoint{0.978830in}{0.894194in}}{\pgfqpoint{6.871170in}{3.741255in}}%
\pgfusepath{clip}%
\pgfsetroundcap%
\pgfsetroundjoin%
\pgfsetlinewidth{1.003750pt}%
\definecolor{currentstroke}{rgb}{1.000000,1.000000,1.000000}%
\pgfsetstrokecolor{currentstroke}%
\pgfsetdash{}{0pt}%
\pgfpathmoveto{\pgfqpoint{6.328671in}{0.894194in}}%
\pgfpathlineto{\pgfqpoint{6.328671in}{4.635449in}}%
\pgfusepath{stroke}%
\end{pgfscope}%
\begin{pgfscope}%
\definecolor{textcolor}{rgb}{0.150000,0.150000,0.150000}%
\pgfsetstrokecolor{textcolor}%
\pgfsetfillcolor{textcolor}%
\pgftext[x=6.328671in,y=0.796972in,,top]{\color{textcolor}\rmfamily\fontsize{20.000000}{24.000000}\selectfont 105}%
\end{pgfscope}%
\begin{pgfscope}%
\pgfpathrectangle{\pgfqpoint{0.978830in}{0.894194in}}{\pgfqpoint{6.871170in}{3.741255in}}%
\pgfusepath{clip}%
\pgfsetroundcap%
\pgfsetroundjoin%
\pgfsetlinewidth{1.003750pt}%
\definecolor{currentstroke}{rgb}{1.000000,1.000000,1.000000}%
\pgfsetstrokecolor{currentstroke}%
\pgfsetdash{}{0pt}%
\pgfpathmoveto{\pgfqpoint{7.336174in}{0.894194in}}%
\pgfpathlineto{\pgfqpoint{7.336174in}{4.635449in}}%
\pgfusepath{stroke}%
\end{pgfscope}%
\begin{pgfscope}%
\definecolor{textcolor}{rgb}{0.150000,0.150000,0.150000}%
\pgfsetstrokecolor{textcolor}%
\pgfsetfillcolor{textcolor}%
\pgftext[x=7.336174in,y=0.796972in,,top]{\color{textcolor}\rmfamily\fontsize{20.000000}{24.000000}\selectfont 110}%
\end{pgfscope}%
\begin{pgfscope}%
\definecolor{textcolor}{rgb}{0.150000,0.150000,0.150000}%
\pgfsetstrokecolor{textcolor}%
\pgfsetfillcolor{textcolor}%
\pgftext[x=4.414415in,y=0.472591in,,top]{\color{textcolor}\rmfamily\fontsize{24.000000}{28.800000}\selectfont Temperature (K)}%
\end{pgfscope}%
\begin{pgfscope}%
\pgfpathrectangle{\pgfqpoint{0.978830in}{0.894194in}}{\pgfqpoint{6.871170in}{3.741255in}}%
\pgfusepath{clip}%
\pgfsetroundcap%
\pgfsetroundjoin%
\pgfsetlinewidth{1.003750pt}%
\definecolor{currentstroke}{rgb}{1.000000,1.000000,1.000000}%
\pgfsetstrokecolor{currentstroke}%
\pgfsetdash{}{0pt}%
\pgfpathmoveto{\pgfqpoint{0.978830in}{1.029291in}}%
\pgfpathlineto{\pgfqpoint{7.850000in}{1.029291in}}%
\pgfusepath{stroke}%
\end{pgfscope}%
\begin{pgfscope}%
\definecolor{textcolor}{rgb}{0.150000,0.150000,0.150000}%
\pgfsetstrokecolor{textcolor}%
\pgfsetfillcolor{textcolor}%
\pgftext[x=0.439849in, y=0.923768in, left, base]{\color{textcolor}\rmfamily\fontsize{20.000000}{24.000000}\selectfont 0.0}%
\end{pgfscope}%
\begin{pgfscope}%
\pgfpathrectangle{\pgfqpoint{0.978830in}{0.894194in}}{\pgfqpoint{6.871170in}{3.741255in}}%
\pgfusepath{clip}%
\pgfsetroundcap%
\pgfsetroundjoin%
\pgfsetlinewidth{1.003750pt}%
\definecolor{currentstroke}{rgb}{1.000000,1.000000,1.000000}%
\pgfsetstrokecolor{currentstroke}%
\pgfsetdash{}{0pt}%
\pgfpathmoveto{\pgfqpoint{0.978830in}{1.717996in}}%
\pgfpathlineto{\pgfqpoint{7.850000in}{1.717996in}}%
\pgfusepath{stroke}%
\end{pgfscope}%
\begin{pgfscope}%
\definecolor{textcolor}{rgb}{0.150000,0.150000,0.150000}%
\pgfsetstrokecolor{textcolor}%
\pgfsetfillcolor{textcolor}%
\pgftext[x=0.439849in, y=1.612473in, left, base]{\color{textcolor}\rmfamily\fontsize{20.000000}{24.000000}\selectfont 2.5}%
\end{pgfscope}%
\begin{pgfscope}%
\pgfpathrectangle{\pgfqpoint{0.978830in}{0.894194in}}{\pgfqpoint{6.871170in}{3.741255in}}%
\pgfusepath{clip}%
\pgfsetroundcap%
\pgfsetroundjoin%
\pgfsetlinewidth{1.003750pt}%
\definecolor{currentstroke}{rgb}{1.000000,1.000000,1.000000}%
\pgfsetstrokecolor{currentstroke}%
\pgfsetdash{}{0pt}%
\pgfpathmoveto{\pgfqpoint{0.978830in}{2.406701in}}%
\pgfpathlineto{\pgfqpoint{7.850000in}{2.406701in}}%
\pgfusepath{stroke}%
\end{pgfscope}%
\begin{pgfscope}%
\definecolor{textcolor}{rgb}{0.150000,0.150000,0.150000}%
\pgfsetstrokecolor{textcolor}%
\pgfsetfillcolor{textcolor}%
\pgftext[x=0.439849in, y=2.301178in, left, base]{\color{textcolor}\rmfamily\fontsize{20.000000}{24.000000}\selectfont 5.0}%
\end{pgfscope}%
\begin{pgfscope}%
\pgfpathrectangle{\pgfqpoint{0.978830in}{0.894194in}}{\pgfqpoint{6.871170in}{3.741255in}}%
\pgfusepath{clip}%
\pgfsetroundcap%
\pgfsetroundjoin%
\pgfsetlinewidth{1.003750pt}%
\definecolor{currentstroke}{rgb}{1.000000,1.000000,1.000000}%
\pgfsetstrokecolor{currentstroke}%
\pgfsetdash{}{0pt}%
\pgfpathmoveto{\pgfqpoint{0.978830in}{3.095405in}}%
\pgfpathlineto{\pgfqpoint{7.850000in}{3.095405in}}%
\pgfusepath{stroke}%
\end{pgfscope}%
\begin{pgfscope}%
\definecolor{textcolor}{rgb}{0.150000,0.150000,0.150000}%
\pgfsetstrokecolor{textcolor}%
\pgfsetfillcolor{textcolor}%
\pgftext[x=0.439849in, y=2.989882in, left, base]{\color{textcolor}\rmfamily\fontsize{20.000000}{24.000000}\selectfont 7.5}%
\end{pgfscope}%
\begin{pgfscope}%
\pgfpathrectangle{\pgfqpoint{0.978830in}{0.894194in}}{\pgfqpoint{6.871170in}{3.741255in}}%
\pgfusepath{clip}%
\pgfsetroundcap%
\pgfsetroundjoin%
\pgfsetlinewidth{1.003750pt}%
\definecolor{currentstroke}{rgb}{1.000000,1.000000,1.000000}%
\pgfsetstrokecolor{currentstroke}%
\pgfsetdash{}{0pt}%
\pgfpathmoveto{\pgfqpoint{0.978830in}{3.784110in}}%
\pgfpathlineto{\pgfqpoint{7.850000in}{3.784110in}}%
\pgfusepath{stroke}%
\end{pgfscope}%
\begin{pgfscope}%
\definecolor{textcolor}{rgb}{0.150000,0.150000,0.150000}%
\pgfsetstrokecolor{textcolor}%
\pgfsetfillcolor{textcolor}%
\pgftext[x=0.263118in, y=3.678587in, left, base]{\color{textcolor}\rmfamily\fontsize{20.000000}{24.000000}\selectfont 10.0}%
\end{pgfscope}%
\begin{pgfscope}%
\pgfpathrectangle{\pgfqpoint{0.978830in}{0.894194in}}{\pgfqpoint{6.871170in}{3.741255in}}%
\pgfusepath{clip}%
\pgfsetroundcap%
\pgfsetroundjoin%
\pgfsetlinewidth{1.003750pt}%
\definecolor{currentstroke}{rgb}{1.000000,1.000000,1.000000}%
\pgfsetstrokecolor{currentstroke}%
\pgfsetdash{}{0pt}%
\pgfpathmoveto{\pgfqpoint{0.978830in}{4.472815in}}%
\pgfpathlineto{\pgfqpoint{7.850000in}{4.472815in}}%
\pgfusepath{stroke}%
\end{pgfscope}%
\begin{pgfscope}%
\definecolor{textcolor}{rgb}{0.150000,0.150000,0.150000}%
\pgfsetstrokecolor{textcolor}%
\pgfsetfillcolor{textcolor}%
\pgftext[x=0.263118in, y=4.367292in, left, base]{\color{textcolor}\rmfamily\fontsize{20.000000}{24.000000}\selectfont 12.5}%
\end{pgfscope}%
\begin{pgfscope}%
\definecolor{textcolor}{rgb}{0.150000,0.150000,0.150000}%
\pgfsetstrokecolor{textcolor}%
\pgfsetfillcolor{textcolor}%
\pgftext[x=0.207563in,y=2.764822in,,bottom,rotate=90.000000]{\color{textcolor}\rmfamily\fontsize{24.000000}{28.800000}\selectfont \(\displaystyle \rho\) (\(\displaystyle \mu\Omega\cdot\) m)}%
\end{pgfscope}%
\begin{pgfscope}%
\pgfpathrectangle{\pgfqpoint{0.978830in}{0.894194in}}{\pgfqpoint{6.871170in}{3.741255in}}%
\pgfusepath{clip}%
\pgfsetbuttcap%
\pgfsetroundjoin%
\definecolor{currentfill}{rgb}{0.298039,0.447059,0.690196}%
\pgfsetfillcolor{currentfill}%
\pgfsetlinewidth{0.301125pt}%
\definecolor{currentstroke}{rgb}{0.298039,0.447059,0.690196}%
\pgfsetstrokecolor{currentstroke}%
\pgfsetdash{}{0pt}%
\pgfsys@defobject{currentmarker}{\pgfqpoint{-0.048611in}{-0.048611in}}{\pgfqpoint{0.048611in}{0.048611in}}{%
\pgfpathmoveto{\pgfqpoint{0.000000in}{-0.048611in}}%
\pgfpathcurveto{\pgfqpoint{0.012892in}{-0.048611in}}{\pgfqpoint{0.025257in}{-0.043489in}}{\pgfqpoint{0.034373in}{-0.034373in}}%
\pgfpathcurveto{\pgfqpoint{0.043489in}{-0.025257in}}{\pgfqpoint{0.048611in}{-0.012892in}}{\pgfqpoint{0.048611in}{0.000000in}}%
\pgfpathcurveto{\pgfqpoint{0.048611in}{0.012892in}}{\pgfqpoint{0.043489in}{0.025257in}}{\pgfqpoint{0.034373in}{0.034373in}}%
\pgfpathcurveto{\pgfqpoint{0.025257in}{0.043489in}}{\pgfqpoint{0.012892in}{0.048611in}}{\pgfqpoint{0.000000in}{0.048611in}}%
\pgfpathcurveto{\pgfqpoint{-0.012892in}{0.048611in}}{\pgfqpoint{-0.025257in}{0.043489in}}{\pgfqpoint{-0.034373in}{0.034373in}}%
\pgfpathcurveto{\pgfqpoint{-0.043489in}{0.025257in}}{\pgfqpoint{-0.048611in}{0.012892in}}{\pgfqpoint{-0.048611in}{0.000000in}}%
\pgfpathcurveto{\pgfqpoint{-0.048611in}{-0.012892in}}{\pgfqpoint{-0.043489in}{-0.025257in}}{\pgfqpoint{-0.034373in}{-0.034373in}}%
\pgfpathcurveto{\pgfqpoint{-0.025257in}{-0.043489in}}{\pgfqpoint{-0.012892in}{-0.048611in}}{\pgfqpoint{0.000000in}{-0.048611in}}%
\pgfpathlineto{\pgfqpoint{0.000000in}{-0.048611in}}%
\pgfpathclose%
\pgfusepath{stroke,fill}%
}%
\begin{pgfscope}%
\pgfsys@transformshift{1.291156in}{1.284002in}%
\pgfsys@useobject{currentmarker}{}%
\end{pgfscope}%
\begin{pgfscope}%
\pgfsys@transformshift{2.234179in}{1.179121in}%
\pgfsys@useobject{currentmarker}{}%
\end{pgfscope}%
\begin{pgfscope}%
\pgfsys@transformshift{2.903161in}{1.089223in}%
\pgfsys@useobject{currentmarker}{}%
\end{pgfscope}%
\begin{pgfscope}%
\pgfsys@transformshift{3.306162in}{1.149155in}%
\pgfsys@useobject{currentmarker}{}%
\end{pgfscope}%
\begin{pgfscope}%
\pgfsys@transformshift{4.313665in}{1.134172in}%
\pgfsys@useobject{currentmarker}{}%
\end{pgfscope}%
\begin{pgfscope}%
\pgfsys@transformshift{4.817416in}{1.149155in}%
\pgfsys@useobject{currentmarker}{}%
\end{pgfscope}%
\begin{pgfscope}%
\pgfsys@transformshift{4.817416in}{1.363911in}%
\pgfsys@useobject{currentmarker}{}%
\end{pgfscope}%
\begin{pgfscope}%
\pgfsys@transformshift{4.918167in}{1.373900in}%
\pgfsys@useobject{currentmarker}{}%
\end{pgfscope}%
\begin{pgfscope}%
\pgfsys@transformshift{5.018917in}{1.373900in}%
\pgfsys@useobject{currentmarker}{}%
\end{pgfscope}%
\begin{pgfscope}%
\pgfsys@transformshift{5.119667in}{1.378894in}%
\pgfsys@useobject{currentmarker}{}%
\end{pgfscope}%
\begin{pgfscope}%
\pgfsys@transformshift{5.220417in}{1.413855in}%
\pgfsys@useobject{currentmarker}{}%
\end{pgfscope}%
\begin{pgfscope}%
\pgfsys@transformshift{5.321168in}{1.428838in}%
\pgfsys@useobject{currentmarker}{}%
\end{pgfscope}%
\begin{pgfscope}%
\pgfsys@transformshift{5.321168in}{1.373900in}%
\pgfsys@useobject{currentmarker}{}%
\end{pgfscope}%
\begin{pgfscope}%
\pgfsys@transformshift{5.421918in}{1.513741in}%
\pgfsys@useobject{currentmarker}{}%
\end{pgfscope}%
\begin{pgfscope}%
\pgfsys@transformshift{5.522668in}{1.583662in}%
\pgfsys@useobject{currentmarker}{}%
\end{pgfscope}%
\begin{pgfscope}%
\pgfsys@transformshift{5.623419in}{1.683549in}%
\pgfsys@useobject{currentmarker}{}%
\end{pgfscope}%
\begin{pgfscope}%
\pgfsys@transformshift{5.724169in}{1.748475in}%
\pgfsys@useobject{currentmarker}{}%
\end{pgfscope}%
\begin{pgfscope}%
\pgfsys@transformshift{5.824919in}{1.848362in}%
\pgfsys@useobject{currentmarker}{}%
\end{pgfscope}%
\begin{pgfscope}%
\pgfsys@transformshift{5.925669in}{1.943254in}%
\pgfsys@useobject{currentmarker}{}%
\end{pgfscope}%
\begin{pgfscope}%
\pgfsys@transformshift{6.026420in}{2.013175in}%
\pgfsys@useobject{currentmarker}{}%
\end{pgfscope}%
\begin{pgfscope}%
\pgfsys@transformshift{6.127170in}{2.083095in}%
\pgfsys@useobject{currentmarker}{}%
\end{pgfscope}%
\begin{pgfscope}%
\pgfsys@transformshift{6.227920in}{2.158010in}%
\pgfsys@useobject{currentmarker}{}%
\end{pgfscope}%
\begin{pgfscope}%
\pgfsys@transformshift{6.328671in}{2.262891in}%
\pgfsys@useobject{currentmarker}{}%
\end{pgfscope}%
\begin{pgfscope}%
\pgfsys@transformshift{6.328671in}{2.227931in}%
\pgfsys@useobject{currentmarker}{}%
\end{pgfscope}%
\begin{pgfscope}%
\pgfsys@transformshift{6.429421in}{2.292857in}%
\pgfsys@useobject{currentmarker}{}%
\end{pgfscope}%
\begin{pgfscope}%
\pgfsys@transformshift{6.530171in}{2.357784in}%
\pgfsys@useobject{currentmarker}{}%
\end{pgfscope}%
\begin{pgfscope}%
\pgfsys@transformshift{6.630921in}{2.437693in}%
\pgfsys@useobject{currentmarker}{}%
\end{pgfscope}%
\begin{pgfscope}%
\pgfsys@transformshift{6.731672in}{2.497625in}%
\pgfsys@useobject{currentmarker}{}%
\end{pgfscope}%
\begin{pgfscope}%
\pgfsys@transformshift{6.832422in}{2.577534in}%
\pgfsys@useobject{currentmarker}{}%
\end{pgfscope}%
\begin{pgfscope}%
\pgfsys@transformshift{6.933172in}{2.687410in}%
\pgfsys@useobject{currentmarker}{}%
\end{pgfscope}%
\begin{pgfscope}%
\pgfsys@transformshift{7.033923in}{2.832245in}%
\pgfsys@useobject{currentmarker}{}%
\end{pgfscope}%
\begin{pgfscope}%
\pgfsys@transformshift{7.134673in}{3.037013in}%
\pgfsys@useobject{currentmarker}{}%
\end{pgfscope}%
\begin{pgfscope}%
\pgfsys@transformshift{7.235423in}{3.351656in}%
\pgfsys@useobject{currentmarker}{}%
\end{pgfscope}%
\begin{pgfscope}%
\pgfsys@transformshift{7.336174in}{3.691271in}%
\pgfsys@useobject{currentmarker}{}%
\end{pgfscope}%
\begin{pgfscope}%
\pgfsys@transformshift{7.436924in}{4.020897in}%
\pgfsys@useobject{currentmarker}{}%
\end{pgfscope}%
\begin{pgfscope}%
\pgfsys@transformshift{7.537674in}{4.310568in}%
\pgfsys@useobject{currentmarker}{}%
\end{pgfscope}%
\end{pgfscope}%
\begin{pgfscope}%
\pgfpathrectangle{\pgfqpoint{0.978830in}{0.894194in}}{\pgfqpoint{6.871170in}{3.741255in}}%
\pgfusepath{clip}%
\pgfsetbuttcap%
\pgfsetroundjoin%
\definecolor{currentfill}{rgb}{0.333333,0.658824,0.407843}%
\pgfsetfillcolor{currentfill}%
\pgfsetlinewidth{0.301125pt}%
\definecolor{currentstroke}{rgb}{0.333333,0.658824,0.407843}%
\pgfsetstrokecolor{currentstroke}%
\pgfsetdash{}{0pt}%
\pgfsys@defobject{currentmarker}{\pgfqpoint{-0.048611in}{-0.048611in}}{\pgfqpoint{0.048611in}{0.048611in}}{%
\pgfpathmoveto{\pgfqpoint{0.000000in}{-0.048611in}}%
\pgfpathcurveto{\pgfqpoint{0.012892in}{-0.048611in}}{\pgfqpoint{0.025257in}{-0.043489in}}{\pgfqpoint{0.034373in}{-0.034373in}}%
\pgfpathcurveto{\pgfqpoint{0.043489in}{-0.025257in}}{\pgfqpoint{0.048611in}{-0.012892in}}{\pgfqpoint{0.048611in}{0.000000in}}%
\pgfpathcurveto{\pgfqpoint{0.048611in}{0.012892in}}{\pgfqpoint{0.043489in}{0.025257in}}{\pgfqpoint{0.034373in}{0.034373in}}%
\pgfpathcurveto{\pgfqpoint{0.025257in}{0.043489in}}{\pgfqpoint{0.012892in}{0.048611in}}{\pgfqpoint{0.000000in}{0.048611in}}%
\pgfpathcurveto{\pgfqpoint{-0.012892in}{0.048611in}}{\pgfqpoint{-0.025257in}{0.043489in}}{\pgfqpoint{-0.034373in}{0.034373in}}%
\pgfpathcurveto{\pgfqpoint{-0.043489in}{0.025257in}}{\pgfqpoint{-0.048611in}{0.012892in}}{\pgfqpoint{-0.048611in}{0.000000in}}%
\pgfpathcurveto{\pgfqpoint{-0.048611in}{-0.012892in}}{\pgfqpoint{-0.043489in}{-0.025257in}}{\pgfqpoint{-0.034373in}{-0.034373in}}%
\pgfpathcurveto{\pgfqpoint{-0.025257in}{-0.043489in}}{\pgfqpoint{-0.012892in}{-0.048611in}}{\pgfqpoint{0.000000in}{-0.048611in}}%
\pgfpathlineto{\pgfqpoint{0.000000in}{-0.048611in}}%
\pgfpathclose%
\pgfusepath{stroke,fill}%
}%
\begin{pgfscope}%
\pgfsys@transformshift{1.291156in}{1.269019in}%
\pgfsys@useobject{currentmarker}{}%
\end{pgfscope}%
\begin{pgfscope}%
\pgfsys@transformshift{2.234179in}{1.358917in}%
\pgfsys@useobject{currentmarker}{}%
\end{pgfscope}%
\begin{pgfscope}%
\pgfsys@transformshift{2.903161in}{1.533719in}%
\pgfsys@useobject{currentmarker}{}%
\end{pgfscope}%
\begin{pgfscope}%
\pgfsys@transformshift{3.306162in}{2.187976in}%
\pgfsys@useobject{currentmarker}{}%
\end{pgfscope}%
\begin{pgfscope}%
\pgfsys@transformshift{4.313665in}{2.187976in}%
\pgfsys@useobject{currentmarker}{}%
\end{pgfscope}%
\begin{pgfscope}%
\pgfsys@transformshift{4.817416in}{2.367772in}%
\pgfsys@useobject{currentmarker}{}%
\end{pgfscope}%
\begin{pgfscope}%
\pgfsys@transformshift{4.817416in}{2.602506in}%
\pgfsys@useobject{currentmarker}{}%
\end{pgfscope}%
\begin{pgfscope}%
\pgfsys@transformshift{4.918167in}{2.577534in}%
\pgfsys@useobject{currentmarker}{}%
\end{pgfscope}%
\begin{pgfscope}%
\pgfsys@transformshift{5.018917in}{2.622483in}%
\pgfsys@useobject{currentmarker}{}%
\end{pgfscope}%
\begin{pgfscope}%
\pgfsys@transformshift{5.119667in}{2.627478in}%
\pgfsys@useobject{currentmarker}{}%
\end{pgfscope}%
\begin{pgfscope}%
\pgfsys@transformshift{5.220417in}{2.632472in}%
\pgfsys@useobject{currentmarker}{}%
\end{pgfscope}%
\begin{pgfscope}%
\pgfsys@transformshift{5.321168in}{2.642461in}%
\pgfsys@useobject{currentmarker}{}%
\end{pgfscope}%
\begin{pgfscope}%
\pgfsys@transformshift{5.321168in}{2.532585in}%
\pgfsys@useobject{currentmarker}{}%
\end{pgfscope}%
\begin{pgfscope}%
\pgfsys@transformshift{5.421918in}{2.687410in}%
\pgfsys@useobject{currentmarker}{}%
\end{pgfscope}%
\begin{pgfscope}%
\pgfsys@transformshift{5.522668in}{2.697398in}%
\pgfsys@useobject{currentmarker}{}%
\end{pgfscope}%
\begin{pgfscope}%
\pgfsys@transformshift{5.623419in}{2.697398in}%
\pgfsys@useobject{currentmarker}{}%
\end{pgfscope}%
\begin{pgfscope}%
\pgfsys@transformshift{5.724169in}{2.727364in}%
\pgfsys@useobject{currentmarker}{}%
\end{pgfscope}%
\begin{pgfscope}%
\pgfsys@transformshift{5.824919in}{2.742347in}%
\pgfsys@useobject{currentmarker}{}%
\end{pgfscope}%
\begin{pgfscope}%
\pgfsys@transformshift{5.925669in}{2.782302in}%
\pgfsys@useobject{currentmarker}{}%
\end{pgfscope}%
\begin{pgfscope}%
\pgfsys@transformshift{6.026420in}{2.797285in}%
\pgfsys@useobject{currentmarker}{}%
\end{pgfscope}%
\begin{pgfscope}%
\pgfsys@transformshift{6.127170in}{2.847228in}%
\pgfsys@useobject{currentmarker}{}%
\end{pgfscope}%
\begin{pgfscope}%
\pgfsys@transformshift{6.227920in}{2.892177in}%
\pgfsys@useobject{currentmarker}{}%
\end{pgfscope}%
\begin{pgfscope}%
\pgfsys@transformshift{6.328671in}{2.977081in}%
\pgfsys@useobject{currentmarker}{}%
\end{pgfscope}%
\begin{pgfscope}%
\pgfsys@transformshift{6.328671in}{2.937126in}%
\pgfsys@useobject{currentmarker}{}%
\end{pgfscope}%
\begin{pgfscope}%
\pgfsys@transformshift{6.429421in}{2.997058in}%
\pgfsys@useobject{currentmarker}{}%
\end{pgfscope}%
\begin{pgfscope}%
\pgfsys@transformshift{6.530171in}{3.056990in}%
\pgfsys@useobject{currentmarker}{}%
\end{pgfscope}%
\begin{pgfscope}%
\pgfsys@transformshift{6.630921in}{3.116922in}%
\pgfsys@useobject{currentmarker}{}%
\end{pgfscope}%
\begin{pgfscope}%
\pgfsys@transformshift{6.731672in}{3.191837in}%
\pgfsys@useobject{currentmarker}{}%
\end{pgfscope}%
\begin{pgfscope}%
\pgfsys@transformshift{6.832422in}{3.291724in}%
\pgfsys@useobject{currentmarker}{}%
\end{pgfscope}%
\begin{pgfscope}%
\pgfsys@transformshift{6.933172in}{3.391611in}%
\pgfsys@useobject{currentmarker}{}%
\end{pgfscope}%
\begin{pgfscope}%
\pgfsys@transformshift{7.033923in}{3.526458in}%
\pgfsys@useobject{currentmarker}{}%
\end{pgfscope}%
\begin{pgfscope}%
\pgfsys@transformshift{7.134673in}{3.686276in}%
\pgfsys@useobject{currentmarker}{}%
\end{pgfscope}%
\begin{pgfscope}%
\pgfsys@transformshift{7.235423in}{3.836106in}%
\pgfsys@useobject{currentmarker}{}%
\end{pgfscope}%
\begin{pgfscope}%
\pgfsys@transformshift{7.336174in}{4.035880in}%
\pgfsys@useobject{currentmarker}{}%
\end{pgfscope}%
\begin{pgfscope}%
\pgfsys@transformshift{7.436924in}{4.225664in}%
\pgfsys@useobject{currentmarker}{}%
\end{pgfscope}%
\begin{pgfscope}%
\pgfsys@transformshift{7.537674in}{4.440421in}%
\pgfsys@useobject{currentmarker}{}%
\end{pgfscope}%
\end{pgfscope}%
\begin{pgfscope}%
\pgfpathrectangle{\pgfqpoint{0.978830in}{0.894194in}}{\pgfqpoint{6.871170in}{3.741255in}}%
\pgfusepath{clip}%
\pgfsetbuttcap%
\pgfsetroundjoin%
\pgfsetlinewidth{1.756562pt}%
\definecolor{currentstroke}{rgb}{0.298039,0.447059,0.690196}%
\pgfsetstrokecolor{currentstroke}%
\pgfsetdash{}{0pt}%
\pgfpathmoveto{\pgfqpoint{1.291156in}{1.259030in}}%
\pgfpathlineto{\pgfqpoint{1.291156in}{1.308974in}}%
\pgfusepath{stroke}%
\end{pgfscope}%
\begin{pgfscope}%
\pgfpathrectangle{\pgfqpoint{0.978830in}{0.894194in}}{\pgfqpoint{6.871170in}{3.741255in}}%
\pgfusepath{clip}%
\pgfsetbuttcap%
\pgfsetroundjoin%
\pgfsetlinewidth{1.756562pt}%
\definecolor{currentstroke}{rgb}{0.298039,0.447059,0.690196}%
\pgfsetstrokecolor{currentstroke}%
\pgfsetdash{}{0pt}%
\pgfpathmoveto{\pgfqpoint{2.234179in}{1.154149in}}%
\pgfpathlineto{\pgfqpoint{2.234179in}{1.204093in}}%
\pgfusepath{stroke}%
\end{pgfscope}%
\begin{pgfscope}%
\pgfpathrectangle{\pgfqpoint{0.978830in}{0.894194in}}{\pgfqpoint{6.871170in}{3.741255in}}%
\pgfusepath{clip}%
\pgfsetbuttcap%
\pgfsetroundjoin%
\pgfsetlinewidth{1.756562pt}%
\definecolor{currentstroke}{rgb}{0.298039,0.447059,0.690196}%
\pgfsetstrokecolor{currentstroke}%
\pgfsetdash{}{0pt}%
\pgfpathmoveto{\pgfqpoint{2.903161in}{1.064251in}}%
\pgfpathlineto{\pgfqpoint{2.903161in}{1.114195in}}%
\pgfusepath{stroke}%
\end{pgfscope}%
\begin{pgfscope}%
\pgfpathrectangle{\pgfqpoint{0.978830in}{0.894194in}}{\pgfqpoint{6.871170in}{3.741255in}}%
\pgfusepath{clip}%
\pgfsetbuttcap%
\pgfsetroundjoin%
\pgfsetlinewidth{1.756562pt}%
\definecolor{currentstroke}{rgb}{0.298039,0.447059,0.690196}%
\pgfsetstrokecolor{currentstroke}%
\pgfsetdash{}{0pt}%
\pgfpathmoveto{\pgfqpoint{3.306162in}{1.124183in}}%
\pgfpathlineto{\pgfqpoint{3.306162in}{1.174127in}}%
\pgfusepath{stroke}%
\end{pgfscope}%
\begin{pgfscope}%
\pgfpathrectangle{\pgfqpoint{0.978830in}{0.894194in}}{\pgfqpoint{6.871170in}{3.741255in}}%
\pgfusepath{clip}%
\pgfsetbuttcap%
\pgfsetroundjoin%
\pgfsetlinewidth{1.756562pt}%
\definecolor{currentstroke}{rgb}{0.298039,0.447059,0.690196}%
\pgfsetstrokecolor{currentstroke}%
\pgfsetdash{}{0pt}%
\pgfpathmoveto{\pgfqpoint{4.313665in}{1.109200in}}%
\pgfpathlineto{\pgfqpoint{4.313665in}{1.159144in}}%
\pgfusepath{stroke}%
\end{pgfscope}%
\begin{pgfscope}%
\pgfpathrectangle{\pgfqpoint{0.978830in}{0.894194in}}{\pgfqpoint{6.871170in}{3.741255in}}%
\pgfusepath{clip}%
\pgfsetbuttcap%
\pgfsetroundjoin%
\pgfsetlinewidth{1.756562pt}%
\definecolor{currentstroke}{rgb}{0.298039,0.447059,0.690196}%
\pgfsetstrokecolor{currentstroke}%
\pgfsetdash{}{0pt}%
\pgfpathmoveto{\pgfqpoint{4.817416in}{1.124183in}}%
\pgfpathlineto{\pgfqpoint{4.817416in}{1.174127in}}%
\pgfusepath{stroke}%
\end{pgfscope}%
\begin{pgfscope}%
\pgfpathrectangle{\pgfqpoint{0.978830in}{0.894194in}}{\pgfqpoint{6.871170in}{3.741255in}}%
\pgfusepath{clip}%
\pgfsetbuttcap%
\pgfsetroundjoin%
\pgfsetlinewidth{1.756562pt}%
\definecolor{currentstroke}{rgb}{0.298039,0.447059,0.690196}%
\pgfsetstrokecolor{currentstroke}%
\pgfsetdash{}{0pt}%
\pgfpathmoveto{\pgfqpoint{4.817416in}{1.338940in}}%
\pgfpathlineto{\pgfqpoint{4.817416in}{1.388883in}}%
\pgfusepath{stroke}%
\end{pgfscope}%
\begin{pgfscope}%
\pgfpathrectangle{\pgfqpoint{0.978830in}{0.894194in}}{\pgfqpoint{6.871170in}{3.741255in}}%
\pgfusepath{clip}%
\pgfsetbuttcap%
\pgfsetroundjoin%
\pgfsetlinewidth{1.756562pt}%
\definecolor{currentstroke}{rgb}{0.298039,0.447059,0.690196}%
\pgfsetstrokecolor{currentstroke}%
\pgfsetdash{}{0pt}%
\pgfpathmoveto{\pgfqpoint{4.918167in}{1.348928in}}%
\pgfpathlineto{\pgfqpoint{4.918167in}{1.398872in}}%
\pgfusepath{stroke}%
\end{pgfscope}%
\begin{pgfscope}%
\pgfpathrectangle{\pgfqpoint{0.978830in}{0.894194in}}{\pgfqpoint{6.871170in}{3.741255in}}%
\pgfusepath{clip}%
\pgfsetbuttcap%
\pgfsetroundjoin%
\pgfsetlinewidth{1.756562pt}%
\definecolor{currentstroke}{rgb}{0.298039,0.447059,0.690196}%
\pgfsetstrokecolor{currentstroke}%
\pgfsetdash{}{0pt}%
\pgfpathmoveto{\pgfqpoint{5.018917in}{1.348928in}}%
\pgfpathlineto{\pgfqpoint{5.018917in}{1.398872in}}%
\pgfusepath{stroke}%
\end{pgfscope}%
\begin{pgfscope}%
\pgfpathrectangle{\pgfqpoint{0.978830in}{0.894194in}}{\pgfqpoint{6.871170in}{3.741255in}}%
\pgfusepath{clip}%
\pgfsetbuttcap%
\pgfsetroundjoin%
\pgfsetlinewidth{1.756562pt}%
\definecolor{currentstroke}{rgb}{0.298039,0.447059,0.690196}%
\pgfsetstrokecolor{currentstroke}%
\pgfsetdash{}{0pt}%
\pgfpathmoveto{\pgfqpoint{5.119667in}{1.353923in}}%
\pgfpathlineto{\pgfqpoint{5.119667in}{1.403866in}}%
\pgfusepath{stroke}%
\end{pgfscope}%
\begin{pgfscope}%
\pgfpathrectangle{\pgfqpoint{0.978830in}{0.894194in}}{\pgfqpoint{6.871170in}{3.741255in}}%
\pgfusepath{clip}%
\pgfsetbuttcap%
\pgfsetroundjoin%
\pgfsetlinewidth{1.756562pt}%
\definecolor{currentstroke}{rgb}{0.298039,0.447059,0.690196}%
\pgfsetstrokecolor{currentstroke}%
\pgfsetdash{}{0pt}%
\pgfpathmoveto{\pgfqpoint{5.220417in}{1.388883in}}%
\pgfpathlineto{\pgfqpoint{5.220417in}{1.438826in}}%
\pgfusepath{stroke}%
\end{pgfscope}%
\begin{pgfscope}%
\pgfpathrectangle{\pgfqpoint{0.978830in}{0.894194in}}{\pgfqpoint{6.871170in}{3.741255in}}%
\pgfusepath{clip}%
\pgfsetbuttcap%
\pgfsetroundjoin%
\pgfsetlinewidth{1.756562pt}%
\definecolor{currentstroke}{rgb}{0.298039,0.447059,0.690196}%
\pgfsetstrokecolor{currentstroke}%
\pgfsetdash{}{0pt}%
\pgfpathmoveto{\pgfqpoint{5.321168in}{1.403866in}}%
\pgfpathlineto{\pgfqpoint{5.321168in}{1.453809in}}%
\pgfusepath{stroke}%
\end{pgfscope}%
\begin{pgfscope}%
\pgfpathrectangle{\pgfqpoint{0.978830in}{0.894194in}}{\pgfqpoint{6.871170in}{3.741255in}}%
\pgfusepath{clip}%
\pgfsetbuttcap%
\pgfsetroundjoin%
\pgfsetlinewidth{1.756562pt}%
\definecolor{currentstroke}{rgb}{0.298039,0.447059,0.690196}%
\pgfsetstrokecolor{currentstroke}%
\pgfsetdash{}{0pt}%
\pgfpathmoveto{\pgfqpoint{5.321168in}{1.348928in}}%
\pgfpathlineto{\pgfqpoint{5.321168in}{1.398872in}}%
\pgfusepath{stroke}%
\end{pgfscope}%
\begin{pgfscope}%
\pgfpathrectangle{\pgfqpoint{0.978830in}{0.894194in}}{\pgfqpoint{6.871170in}{3.741255in}}%
\pgfusepath{clip}%
\pgfsetbuttcap%
\pgfsetroundjoin%
\pgfsetlinewidth{1.756562pt}%
\definecolor{currentstroke}{rgb}{0.298039,0.447059,0.690196}%
\pgfsetstrokecolor{currentstroke}%
\pgfsetdash{}{0pt}%
\pgfpathmoveto{\pgfqpoint{5.421918in}{1.488770in}}%
\pgfpathlineto{\pgfqpoint{5.421918in}{1.538713in}}%
\pgfusepath{stroke}%
\end{pgfscope}%
\begin{pgfscope}%
\pgfpathrectangle{\pgfqpoint{0.978830in}{0.894194in}}{\pgfqpoint{6.871170in}{3.741255in}}%
\pgfusepath{clip}%
\pgfsetbuttcap%
\pgfsetroundjoin%
\pgfsetlinewidth{1.756562pt}%
\definecolor{currentstroke}{rgb}{0.298039,0.447059,0.690196}%
\pgfsetstrokecolor{currentstroke}%
\pgfsetdash{}{0pt}%
\pgfpathmoveto{\pgfqpoint{5.522668in}{1.558690in}}%
\pgfpathlineto{\pgfqpoint{5.522668in}{1.608634in}}%
\pgfusepath{stroke}%
\end{pgfscope}%
\begin{pgfscope}%
\pgfpathrectangle{\pgfqpoint{0.978830in}{0.894194in}}{\pgfqpoint{6.871170in}{3.741255in}}%
\pgfusepath{clip}%
\pgfsetbuttcap%
\pgfsetroundjoin%
\pgfsetlinewidth{1.756562pt}%
\definecolor{currentstroke}{rgb}{0.298039,0.447059,0.690196}%
\pgfsetstrokecolor{currentstroke}%
\pgfsetdash{}{0pt}%
\pgfpathmoveto{\pgfqpoint{5.623419in}{1.658577in}}%
\pgfpathlineto{\pgfqpoint{5.623419in}{1.708520in}}%
\pgfusepath{stroke}%
\end{pgfscope}%
\begin{pgfscope}%
\pgfpathrectangle{\pgfqpoint{0.978830in}{0.894194in}}{\pgfqpoint{6.871170in}{3.741255in}}%
\pgfusepath{clip}%
\pgfsetbuttcap%
\pgfsetroundjoin%
\pgfsetlinewidth{1.756562pt}%
\definecolor{currentstroke}{rgb}{0.298039,0.447059,0.690196}%
\pgfsetstrokecolor{currentstroke}%
\pgfsetdash{}{0pt}%
\pgfpathmoveto{\pgfqpoint{5.724169in}{1.723503in}}%
\pgfpathlineto{\pgfqpoint{5.724169in}{1.773447in}}%
\pgfusepath{stroke}%
\end{pgfscope}%
\begin{pgfscope}%
\pgfpathrectangle{\pgfqpoint{0.978830in}{0.894194in}}{\pgfqpoint{6.871170in}{3.741255in}}%
\pgfusepath{clip}%
\pgfsetbuttcap%
\pgfsetroundjoin%
\pgfsetlinewidth{1.756562pt}%
\definecolor{currentstroke}{rgb}{0.298039,0.447059,0.690196}%
\pgfsetstrokecolor{currentstroke}%
\pgfsetdash{}{0pt}%
\pgfpathmoveto{\pgfqpoint{5.824919in}{1.823390in}}%
\pgfpathlineto{\pgfqpoint{5.824919in}{1.873333in}}%
\pgfusepath{stroke}%
\end{pgfscope}%
\begin{pgfscope}%
\pgfpathrectangle{\pgfqpoint{0.978830in}{0.894194in}}{\pgfqpoint{6.871170in}{3.741255in}}%
\pgfusepath{clip}%
\pgfsetbuttcap%
\pgfsetroundjoin%
\pgfsetlinewidth{1.756562pt}%
\definecolor{currentstroke}{rgb}{0.298039,0.447059,0.690196}%
\pgfsetstrokecolor{currentstroke}%
\pgfsetdash{}{0pt}%
\pgfpathmoveto{\pgfqpoint{5.925669in}{1.918282in}}%
\pgfpathlineto{\pgfqpoint{5.925669in}{1.968226in}}%
\pgfusepath{stroke}%
\end{pgfscope}%
\begin{pgfscope}%
\pgfpathrectangle{\pgfqpoint{0.978830in}{0.894194in}}{\pgfqpoint{6.871170in}{3.741255in}}%
\pgfusepath{clip}%
\pgfsetbuttcap%
\pgfsetroundjoin%
\pgfsetlinewidth{1.756562pt}%
\definecolor{currentstroke}{rgb}{0.298039,0.447059,0.690196}%
\pgfsetstrokecolor{currentstroke}%
\pgfsetdash{}{0pt}%
\pgfpathmoveto{\pgfqpoint{6.026420in}{1.988203in}}%
\pgfpathlineto{\pgfqpoint{6.026420in}{2.038146in}}%
\pgfusepath{stroke}%
\end{pgfscope}%
\begin{pgfscope}%
\pgfpathrectangle{\pgfqpoint{0.978830in}{0.894194in}}{\pgfqpoint{6.871170in}{3.741255in}}%
\pgfusepath{clip}%
\pgfsetbuttcap%
\pgfsetroundjoin%
\pgfsetlinewidth{1.756562pt}%
\definecolor{currentstroke}{rgb}{0.298039,0.447059,0.690196}%
\pgfsetstrokecolor{currentstroke}%
\pgfsetdash{}{0pt}%
\pgfpathmoveto{\pgfqpoint{6.127170in}{2.058124in}}%
\pgfpathlineto{\pgfqpoint{6.127170in}{2.108067in}}%
\pgfusepath{stroke}%
\end{pgfscope}%
\begin{pgfscope}%
\pgfpathrectangle{\pgfqpoint{0.978830in}{0.894194in}}{\pgfqpoint{6.871170in}{3.741255in}}%
\pgfusepath{clip}%
\pgfsetbuttcap%
\pgfsetroundjoin%
\pgfsetlinewidth{1.756562pt}%
\definecolor{currentstroke}{rgb}{0.298039,0.447059,0.690196}%
\pgfsetstrokecolor{currentstroke}%
\pgfsetdash{}{0pt}%
\pgfpathmoveto{\pgfqpoint{6.227920in}{2.133039in}}%
\pgfpathlineto{\pgfqpoint{6.227920in}{2.182982in}}%
\pgfusepath{stroke}%
\end{pgfscope}%
\begin{pgfscope}%
\pgfpathrectangle{\pgfqpoint{0.978830in}{0.894194in}}{\pgfqpoint{6.871170in}{3.741255in}}%
\pgfusepath{clip}%
\pgfsetbuttcap%
\pgfsetroundjoin%
\pgfsetlinewidth{1.756562pt}%
\definecolor{currentstroke}{rgb}{0.298039,0.447059,0.690196}%
\pgfsetstrokecolor{currentstroke}%
\pgfsetdash{}{0pt}%
\pgfpathmoveto{\pgfqpoint{6.328671in}{2.237920in}}%
\pgfpathlineto{\pgfqpoint{6.328671in}{2.287863in}}%
\pgfusepath{stroke}%
\end{pgfscope}%
\begin{pgfscope}%
\pgfpathrectangle{\pgfqpoint{0.978830in}{0.894194in}}{\pgfqpoint{6.871170in}{3.741255in}}%
\pgfusepath{clip}%
\pgfsetbuttcap%
\pgfsetroundjoin%
\pgfsetlinewidth{1.756562pt}%
\definecolor{currentstroke}{rgb}{0.298039,0.447059,0.690196}%
\pgfsetstrokecolor{currentstroke}%
\pgfsetdash{}{0pt}%
\pgfpathmoveto{\pgfqpoint{6.328671in}{2.202959in}}%
\pgfpathlineto{\pgfqpoint{6.328671in}{2.252903in}}%
\pgfusepath{stroke}%
\end{pgfscope}%
\begin{pgfscope}%
\pgfpathrectangle{\pgfqpoint{0.978830in}{0.894194in}}{\pgfqpoint{6.871170in}{3.741255in}}%
\pgfusepath{clip}%
\pgfsetbuttcap%
\pgfsetroundjoin%
\pgfsetlinewidth{1.756562pt}%
\definecolor{currentstroke}{rgb}{0.298039,0.447059,0.690196}%
\pgfsetstrokecolor{currentstroke}%
\pgfsetdash{}{0pt}%
\pgfpathmoveto{\pgfqpoint{6.429421in}{2.267886in}}%
\pgfpathlineto{\pgfqpoint{6.429421in}{2.317829in}}%
\pgfusepath{stroke}%
\end{pgfscope}%
\begin{pgfscope}%
\pgfpathrectangle{\pgfqpoint{0.978830in}{0.894194in}}{\pgfqpoint{6.871170in}{3.741255in}}%
\pgfusepath{clip}%
\pgfsetbuttcap%
\pgfsetroundjoin%
\pgfsetlinewidth{1.756562pt}%
\definecolor{currentstroke}{rgb}{0.298039,0.447059,0.690196}%
\pgfsetstrokecolor{currentstroke}%
\pgfsetdash{}{0pt}%
\pgfpathmoveto{\pgfqpoint{6.530171in}{2.332812in}}%
\pgfpathlineto{\pgfqpoint{6.530171in}{2.382755in}}%
\pgfusepath{stroke}%
\end{pgfscope}%
\begin{pgfscope}%
\pgfpathrectangle{\pgfqpoint{0.978830in}{0.894194in}}{\pgfqpoint{6.871170in}{3.741255in}}%
\pgfusepath{clip}%
\pgfsetbuttcap%
\pgfsetroundjoin%
\pgfsetlinewidth{1.756562pt}%
\definecolor{currentstroke}{rgb}{0.298039,0.447059,0.690196}%
\pgfsetstrokecolor{currentstroke}%
\pgfsetdash{}{0pt}%
\pgfpathmoveto{\pgfqpoint{6.630921in}{2.412721in}}%
\pgfpathlineto{\pgfqpoint{6.630921in}{2.462665in}}%
\pgfusepath{stroke}%
\end{pgfscope}%
\begin{pgfscope}%
\pgfpathrectangle{\pgfqpoint{0.978830in}{0.894194in}}{\pgfqpoint{6.871170in}{3.741255in}}%
\pgfusepath{clip}%
\pgfsetbuttcap%
\pgfsetroundjoin%
\pgfsetlinewidth{1.756562pt}%
\definecolor{currentstroke}{rgb}{0.298039,0.447059,0.690196}%
\pgfsetstrokecolor{currentstroke}%
\pgfsetdash{}{0pt}%
\pgfpathmoveto{\pgfqpoint{6.731672in}{2.472653in}}%
\pgfpathlineto{\pgfqpoint{6.731672in}{2.522597in}}%
\pgfusepath{stroke}%
\end{pgfscope}%
\begin{pgfscope}%
\pgfpathrectangle{\pgfqpoint{0.978830in}{0.894194in}}{\pgfqpoint{6.871170in}{3.741255in}}%
\pgfusepath{clip}%
\pgfsetbuttcap%
\pgfsetroundjoin%
\pgfsetlinewidth{1.756562pt}%
\definecolor{currentstroke}{rgb}{0.298039,0.447059,0.690196}%
\pgfsetstrokecolor{currentstroke}%
\pgfsetdash{}{0pt}%
\pgfpathmoveto{\pgfqpoint{6.832422in}{2.552563in}}%
\pgfpathlineto{\pgfqpoint{6.832422in}{2.602506in}}%
\pgfusepath{stroke}%
\end{pgfscope}%
\begin{pgfscope}%
\pgfpathrectangle{\pgfqpoint{0.978830in}{0.894194in}}{\pgfqpoint{6.871170in}{3.741255in}}%
\pgfusepath{clip}%
\pgfsetbuttcap%
\pgfsetroundjoin%
\pgfsetlinewidth{1.756562pt}%
\definecolor{currentstroke}{rgb}{0.298039,0.447059,0.690196}%
\pgfsetstrokecolor{currentstroke}%
\pgfsetdash{}{0pt}%
\pgfpathmoveto{\pgfqpoint{6.933172in}{2.662438in}}%
\pgfpathlineto{\pgfqpoint{6.933172in}{2.712381in}}%
\pgfusepath{stroke}%
\end{pgfscope}%
\begin{pgfscope}%
\pgfpathrectangle{\pgfqpoint{0.978830in}{0.894194in}}{\pgfqpoint{6.871170in}{3.741255in}}%
\pgfusepath{clip}%
\pgfsetbuttcap%
\pgfsetroundjoin%
\pgfsetlinewidth{1.756562pt}%
\definecolor{currentstroke}{rgb}{0.298039,0.447059,0.690196}%
\pgfsetstrokecolor{currentstroke}%
\pgfsetdash{}{0pt}%
\pgfpathmoveto{\pgfqpoint{7.033923in}{2.807274in}}%
\pgfpathlineto{\pgfqpoint{7.033923in}{2.857217in}}%
\pgfusepath{stroke}%
\end{pgfscope}%
\begin{pgfscope}%
\pgfpathrectangle{\pgfqpoint{0.978830in}{0.894194in}}{\pgfqpoint{6.871170in}{3.741255in}}%
\pgfusepath{clip}%
\pgfsetbuttcap%
\pgfsetroundjoin%
\pgfsetlinewidth{1.756562pt}%
\definecolor{currentstroke}{rgb}{0.298039,0.447059,0.690196}%
\pgfsetstrokecolor{currentstroke}%
\pgfsetdash{}{0pt}%
\pgfpathmoveto{\pgfqpoint{7.134673in}{3.012041in}}%
\pgfpathlineto{\pgfqpoint{7.134673in}{3.061985in}}%
\pgfusepath{stroke}%
\end{pgfscope}%
\begin{pgfscope}%
\pgfpathrectangle{\pgfqpoint{0.978830in}{0.894194in}}{\pgfqpoint{6.871170in}{3.741255in}}%
\pgfusepath{clip}%
\pgfsetbuttcap%
\pgfsetroundjoin%
\pgfsetlinewidth{1.756562pt}%
\definecolor{currentstroke}{rgb}{0.298039,0.447059,0.690196}%
\pgfsetstrokecolor{currentstroke}%
\pgfsetdash{}{0pt}%
\pgfpathmoveto{\pgfqpoint{7.235423in}{3.326684in}}%
\pgfpathlineto{\pgfqpoint{7.235423in}{3.376628in}}%
\pgfusepath{stroke}%
\end{pgfscope}%
\begin{pgfscope}%
\pgfpathrectangle{\pgfqpoint{0.978830in}{0.894194in}}{\pgfqpoint{6.871170in}{3.741255in}}%
\pgfusepath{clip}%
\pgfsetbuttcap%
\pgfsetroundjoin%
\pgfsetlinewidth{1.756562pt}%
\definecolor{currentstroke}{rgb}{0.298039,0.447059,0.690196}%
\pgfsetstrokecolor{currentstroke}%
\pgfsetdash{}{0pt}%
\pgfpathmoveto{\pgfqpoint{7.336174in}{3.666299in}}%
\pgfpathlineto{\pgfqpoint{7.336174in}{3.716242in}}%
\pgfusepath{stroke}%
\end{pgfscope}%
\begin{pgfscope}%
\pgfpathrectangle{\pgfqpoint{0.978830in}{0.894194in}}{\pgfqpoint{6.871170in}{3.741255in}}%
\pgfusepath{clip}%
\pgfsetbuttcap%
\pgfsetroundjoin%
\pgfsetlinewidth{1.756562pt}%
\definecolor{currentstroke}{rgb}{0.298039,0.447059,0.690196}%
\pgfsetstrokecolor{currentstroke}%
\pgfsetdash{}{0pt}%
\pgfpathmoveto{\pgfqpoint{7.436924in}{3.995925in}}%
\pgfpathlineto{\pgfqpoint{7.436924in}{4.045868in}}%
\pgfusepath{stroke}%
\end{pgfscope}%
\begin{pgfscope}%
\pgfpathrectangle{\pgfqpoint{0.978830in}{0.894194in}}{\pgfqpoint{6.871170in}{3.741255in}}%
\pgfusepath{clip}%
\pgfsetbuttcap%
\pgfsetroundjoin%
\pgfsetlinewidth{1.756562pt}%
\definecolor{currentstroke}{rgb}{0.298039,0.447059,0.690196}%
\pgfsetstrokecolor{currentstroke}%
\pgfsetdash{}{0pt}%
\pgfpathmoveto{\pgfqpoint{7.537674in}{4.285596in}}%
\pgfpathlineto{\pgfqpoint{7.537674in}{4.335540in}}%
\pgfusepath{stroke}%
\end{pgfscope}%
\begin{pgfscope}%
\pgfpathrectangle{\pgfqpoint{0.978830in}{0.894194in}}{\pgfqpoint{6.871170in}{3.741255in}}%
\pgfusepath{clip}%
\pgfsetbuttcap%
\pgfsetroundjoin%
\definecolor{currentfill}{rgb}{0.298039,0.447059,0.690196}%
\pgfsetfillcolor{currentfill}%
\pgfsetlinewidth{1.003750pt}%
\definecolor{currentstroke}{rgb}{0.298039,0.447059,0.690196}%
\pgfsetstrokecolor{currentstroke}%
\pgfsetdash{}{0pt}%
\pgfsys@defobject{currentmarker}{\pgfqpoint{-0.069444in}{-0.000000in}}{\pgfqpoint{0.069444in}{0.000000in}}{%
\pgfpathmoveto{\pgfqpoint{0.069444in}{-0.000000in}}%
\pgfpathlineto{\pgfqpoint{-0.069444in}{0.000000in}}%
\pgfusepath{stroke,fill}%
}%
\begin{pgfscope}%
\pgfsys@transformshift{1.291156in}{1.259030in}%
\pgfsys@useobject{currentmarker}{}%
\end{pgfscope}%
\begin{pgfscope}%
\pgfsys@transformshift{2.234179in}{1.154149in}%
\pgfsys@useobject{currentmarker}{}%
\end{pgfscope}%
\begin{pgfscope}%
\pgfsys@transformshift{2.903161in}{1.064251in}%
\pgfsys@useobject{currentmarker}{}%
\end{pgfscope}%
\begin{pgfscope}%
\pgfsys@transformshift{3.306162in}{1.124183in}%
\pgfsys@useobject{currentmarker}{}%
\end{pgfscope}%
\begin{pgfscope}%
\pgfsys@transformshift{4.313665in}{1.109200in}%
\pgfsys@useobject{currentmarker}{}%
\end{pgfscope}%
\begin{pgfscope}%
\pgfsys@transformshift{4.817416in}{1.124183in}%
\pgfsys@useobject{currentmarker}{}%
\end{pgfscope}%
\begin{pgfscope}%
\pgfsys@transformshift{4.817416in}{1.338940in}%
\pgfsys@useobject{currentmarker}{}%
\end{pgfscope}%
\begin{pgfscope}%
\pgfsys@transformshift{4.918167in}{1.348928in}%
\pgfsys@useobject{currentmarker}{}%
\end{pgfscope}%
\begin{pgfscope}%
\pgfsys@transformshift{5.018917in}{1.348928in}%
\pgfsys@useobject{currentmarker}{}%
\end{pgfscope}%
\begin{pgfscope}%
\pgfsys@transformshift{5.119667in}{1.353923in}%
\pgfsys@useobject{currentmarker}{}%
\end{pgfscope}%
\begin{pgfscope}%
\pgfsys@transformshift{5.220417in}{1.388883in}%
\pgfsys@useobject{currentmarker}{}%
\end{pgfscope}%
\begin{pgfscope}%
\pgfsys@transformshift{5.321168in}{1.403866in}%
\pgfsys@useobject{currentmarker}{}%
\end{pgfscope}%
\begin{pgfscope}%
\pgfsys@transformshift{5.321168in}{1.348928in}%
\pgfsys@useobject{currentmarker}{}%
\end{pgfscope}%
\begin{pgfscope}%
\pgfsys@transformshift{5.421918in}{1.488770in}%
\pgfsys@useobject{currentmarker}{}%
\end{pgfscope}%
\begin{pgfscope}%
\pgfsys@transformshift{5.522668in}{1.558690in}%
\pgfsys@useobject{currentmarker}{}%
\end{pgfscope}%
\begin{pgfscope}%
\pgfsys@transformshift{5.623419in}{1.658577in}%
\pgfsys@useobject{currentmarker}{}%
\end{pgfscope}%
\begin{pgfscope}%
\pgfsys@transformshift{5.724169in}{1.723503in}%
\pgfsys@useobject{currentmarker}{}%
\end{pgfscope}%
\begin{pgfscope}%
\pgfsys@transformshift{5.824919in}{1.823390in}%
\pgfsys@useobject{currentmarker}{}%
\end{pgfscope}%
\begin{pgfscope}%
\pgfsys@transformshift{5.925669in}{1.918282in}%
\pgfsys@useobject{currentmarker}{}%
\end{pgfscope}%
\begin{pgfscope}%
\pgfsys@transformshift{6.026420in}{1.988203in}%
\pgfsys@useobject{currentmarker}{}%
\end{pgfscope}%
\begin{pgfscope}%
\pgfsys@transformshift{6.127170in}{2.058124in}%
\pgfsys@useobject{currentmarker}{}%
\end{pgfscope}%
\begin{pgfscope}%
\pgfsys@transformshift{6.227920in}{2.133039in}%
\pgfsys@useobject{currentmarker}{}%
\end{pgfscope}%
\begin{pgfscope}%
\pgfsys@transformshift{6.328671in}{2.237920in}%
\pgfsys@useobject{currentmarker}{}%
\end{pgfscope}%
\begin{pgfscope}%
\pgfsys@transformshift{6.328671in}{2.202959in}%
\pgfsys@useobject{currentmarker}{}%
\end{pgfscope}%
\begin{pgfscope}%
\pgfsys@transformshift{6.429421in}{2.267886in}%
\pgfsys@useobject{currentmarker}{}%
\end{pgfscope}%
\begin{pgfscope}%
\pgfsys@transformshift{6.530171in}{2.332812in}%
\pgfsys@useobject{currentmarker}{}%
\end{pgfscope}%
\begin{pgfscope}%
\pgfsys@transformshift{6.630921in}{2.412721in}%
\pgfsys@useobject{currentmarker}{}%
\end{pgfscope}%
\begin{pgfscope}%
\pgfsys@transformshift{6.731672in}{2.472653in}%
\pgfsys@useobject{currentmarker}{}%
\end{pgfscope}%
\begin{pgfscope}%
\pgfsys@transformshift{6.832422in}{2.552563in}%
\pgfsys@useobject{currentmarker}{}%
\end{pgfscope}%
\begin{pgfscope}%
\pgfsys@transformshift{6.933172in}{2.662438in}%
\pgfsys@useobject{currentmarker}{}%
\end{pgfscope}%
\begin{pgfscope}%
\pgfsys@transformshift{7.033923in}{2.807274in}%
\pgfsys@useobject{currentmarker}{}%
\end{pgfscope}%
\begin{pgfscope}%
\pgfsys@transformshift{7.134673in}{3.012041in}%
\pgfsys@useobject{currentmarker}{}%
\end{pgfscope}%
\begin{pgfscope}%
\pgfsys@transformshift{7.235423in}{3.326684in}%
\pgfsys@useobject{currentmarker}{}%
\end{pgfscope}%
\begin{pgfscope}%
\pgfsys@transformshift{7.336174in}{3.666299in}%
\pgfsys@useobject{currentmarker}{}%
\end{pgfscope}%
\begin{pgfscope}%
\pgfsys@transformshift{7.436924in}{3.995925in}%
\pgfsys@useobject{currentmarker}{}%
\end{pgfscope}%
\begin{pgfscope}%
\pgfsys@transformshift{7.537674in}{4.285596in}%
\pgfsys@useobject{currentmarker}{}%
\end{pgfscope}%
\end{pgfscope}%
\begin{pgfscope}%
\pgfpathrectangle{\pgfqpoint{0.978830in}{0.894194in}}{\pgfqpoint{6.871170in}{3.741255in}}%
\pgfusepath{clip}%
\pgfsetbuttcap%
\pgfsetroundjoin%
\definecolor{currentfill}{rgb}{0.298039,0.447059,0.690196}%
\pgfsetfillcolor{currentfill}%
\pgfsetlinewidth{1.003750pt}%
\definecolor{currentstroke}{rgb}{0.298039,0.447059,0.690196}%
\pgfsetstrokecolor{currentstroke}%
\pgfsetdash{}{0pt}%
\pgfsys@defobject{currentmarker}{\pgfqpoint{-0.069444in}{-0.000000in}}{\pgfqpoint{0.069444in}{0.000000in}}{%
\pgfpathmoveto{\pgfqpoint{0.069444in}{-0.000000in}}%
\pgfpathlineto{\pgfqpoint{-0.069444in}{0.000000in}}%
\pgfusepath{stroke,fill}%
}%
\begin{pgfscope}%
\pgfsys@transformshift{1.291156in}{1.308974in}%
\pgfsys@useobject{currentmarker}{}%
\end{pgfscope}%
\begin{pgfscope}%
\pgfsys@transformshift{2.234179in}{1.204093in}%
\pgfsys@useobject{currentmarker}{}%
\end{pgfscope}%
\begin{pgfscope}%
\pgfsys@transformshift{2.903161in}{1.114195in}%
\pgfsys@useobject{currentmarker}{}%
\end{pgfscope}%
\begin{pgfscope}%
\pgfsys@transformshift{3.306162in}{1.174127in}%
\pgfsys@useobject{currentmarker}{}%
\end{pgfscope}%
\begin{pgfscope}%
\pgfsys@transformshift{4.313665in}{1.159144in}%
\pgfsys@useobject{currentmarker}{}%
\end{pgfscope}%
\begin{pgfscope}%
\pgfsys@transformshift{4.817416in}{1.174127in}%
\pgfsys@useobject{currentmarker}{}%
\end{pgfscope}%
\begin{pgfscope}%
\pgfsys@transformshift{4.817416in}{1.388883in}%
\pgfsys@useobject{currentmarker}{}%
\end{pgfscope}%
\begin{pgfscope}%
\pgfsys@transformshift{4.918167in}{1.398872in}%
\pgfsys@useobject{currentmarker}{}%
\end{pgfscope}%
\begin{pgfscope}%
\pgfsys@transformshift{5.018917in}{1.398872in}%
\pgfsys@useobject{currentmarker}{}%
\end{pgfscope}%
\begin{pgfscope}%
\pgfsys@transformshift{5.119667in}{1.403866in}%
\pgfsys@useobject{currentmarker}{}%
\end{pgfscope}%
\begin{pgfscope}%
\pgfsys@transformshift{5.220417in}{1.438826in}%
\pgfsys@useobject{currentmarker}{}%
\end{pgfscope}%
\begin{pgfscope}%
\pgfsys@transformshift{5.321168in}{1.453809in}%
\pgfsys@useobject{currentmarker}{}%
\end{pgfscope}%
\begin{pgfscope}%
\pgfsys@transformshift{5.321168in}{1.398872in}%
\pgfsys@useobject{currentmarker}{}%
\end{pgfscope}%
\begin{pgfscope}%
\pgfsys@transformshift{5.421918in}{1.538713in}%
\pgfsys@useobject{currentmarker}{}%
\end{pgfscope}%
\begin{pgfscope}%
\pgfsys@transformshift{5.522668in}{1.608634in}%
\pgfsys@useobject{currentmarker}{}%
\end{pgfscope}%
\begin{pgfscope}%
\pgfsys@transformshift{5.623419in}{1.708520in}%
\pgfsys@useobject{currentmarker}{}%
\end{pgfscope}%
\begin{pgfscope}%
\pgfsys@transformshift{5.724169in}{1.773447in}%
\pgfsys@useobject{currentmarker}{}%
\end{pgfscope}%
\begin{pgfscope}%
\pgfsys@transformshift{5.824919in}{1.873333in}%
\pgfsys@useobject{currentmarker}{}%
\end{pgfscope}%
\begin{pgfscope}%
\pgfsys@transformshift{5.925669in}{1.968226in}%
\pgfsys@useobject{currentmarker}{}%
\end{pgfscope}%
\begin{pgfscope}%
\pgfsys@transformshift{6.026420in}{2.038146in}%
\pgfsys@useobject{currentmarker}{}%
\end{pgfscope}%
\begin{pgfscope}%
\pgfsys@transformshift{6.127170in}{2.108067in}%
\pgfsys@useobject{currentmarker}{}%
\end{pgfscope}%
\begin{pgfscope}%
\pgfsys@transformshift{6.227920in}{2.182982in}%
\pgfsys@useobject{currentmarker}{}%
\end{pgfscope}%
\begin{pgfscope}%
\pgfsys@transformshift{6.328671in}{2.287863in}%
\pgfsys@useobject{currentmarker}{}%
\end{pgfscope}%
\begin{pgfscope}%
\pgfsys@transformshift{6.328671in}{2.252903in}%
\pgfsys@useobject{currentmarker}{}%
\end{pgfscope}%
\begin{pgfscope}%
\pgfsys@transformshift{6.429421in}{2.317829in}%
\pgfsys@useobject{currentmarker}{}%
\end{pgfscope}%
\begin{pgfscope}%
\pgfsys@transformshift{6.530171in}{2.382755in}%
\pgfsys@useobject{currentmarker}{}%
\end{pgfscope}%
\begin{pgfscope}%
\pgfsys@transformshift{6.630921in}{2.462665in}%
\pgfsys@useobject{currentmarker}{}%
\end{pgfscope}%
\begin{pgfscope}%
\pgfsys@transformshift{6.731672in}{2.522597in}%
\pgfsys@useobject{currentmarker}{}%
\end{pgfscope}%
\begin{pgfscope}%
\pgfsys@transformshift{6.832422in}{2.602506in}%
\pgfsys@useobject{currentmarker}{}%
\end{pgfscope}%
\begin{pgfscope}%
\pgfsys@transformshift{6.933172in}{2.712381in}%
\pgfsys@useobject{currentmarker}{}%
\end{pgfscope}%
\begin{pgfscope}%
\pgfsys@transformshift{7.033923in}{2.857217in}%
\pgfsys@useobject{currentmarker}{}%
\end{pgfscope}%
\begin{pgfscope}%
\pgfsys@transformshift{7.134673in}{3.061985in}%
\pgfsys@useobject{currentmarker}{}%
\end{pgfscope}%
\begin{pgfscope}%
\pgfsys@transformshift{7.235423in}{3.376628in}%
\pgfsys@useobject{currentmarker}{}%
\end{pgfscope}%
\begin{pgfscope}%
\pgfsys@transformshift{7.336174in}{3.716242in}%
\pgfsys@useobject{currentmarker}{}%
\end{pgfscope}%
\begin{pgfscope}%
\pgfsys@transformshift{7.436924in}{4.045868in}%
\pgfsys@useobject{currentmarker}{}%
\end{pgfscope}%
\begin{pgfscope}%
\pgfsys@transformshift{7.537674in}{4.335540in}%
\pgfsys@useobject{currentmarker}{}%
\end{pgfscope}%
\end{pgfscope}%
\begin{pgfscope}%
\pgfpathrectangle{\pgfqpoint{0.978830in}{0.894194in}}{\pgfqpoint{6.871170in}{3.741255in}}%
\pgfusepath{clip}%
\pgfsetbuttcap%
\pgfsetroundjoin%
\pgfsetlinewidth{1.756562pt}%
\definecolor{currentstroke}{rgb}{0.333333,0.658824,0.407843}%
\pgfsetstrokecolor{currentstroke}%
\pgfsetdash{}{0pt}%
\pgfpathmoveto{\pgfqpoint{1.291156in}{1.244047in}}%
\pgfpathlineto{\pgfqpoint{1.291156in}{1.293991in}}%
\pgfusepath{stroke}%
\end{pgfscope}%
\begin{pgfscope}%
\pgfpathrectangle{\pgfqpoint{0.978830in}{0.894194in}}{\pgfqpoint{6.871170in}{3.741255in}}%
\pgfusepath{clip}%
\pgfsetbuttcap%
\pgfsetroundjoin%
\pgfsetlinewidth{1.756562pt}%
\definecolor{currentstroke}{rgb}{0.333333,0.658824,0.407843}%
\pgfsetstrokecolor{currentstroke}%
\pgfsetdash{}{0pt}%
\pgfpathmoveto{\pgfqpoint{2.234179in}{1.333945in}}%
\pgfpathlineto{\pgfqpoint{2.234179in}{1.383889in}}%
\pgfusepath{stroke}%
\end{pgfscope}%
\begin{pgfscope}%
\pgfpathrectangle{\pgfqpoint{0.978830in}{0.894194in}}{\pgfqpoint{6.871170in}{3.741255in}}%
\pgfusepath{clip}%
\pgfsetbuttcap%
\pgfsetroundjoin%
\pgfsetlinewidth{1.756562pt}%
\definecolor{currentstroke}{rgb}{0.333333,0.658824,0.407843}%
\pgfsetstrokecolor{currentstroke}%
\pgfsetdash{}{0pt}%
\pgfpathmoveto{\pgfqpoint{2.903161in}{1.508747in}}%
\pgfpathlineto{\pgfqpoint{2.903161in}{1.558690in}}%
\pgfusepath{stroke}%
\end{pgfscope}%
\begin{pgfscope}%
\pgfpathrectangle{\pgfqpoint{0.978830in}{0.894194in}}{\pgfqpoint{6.871170in}{3.741255in}}%
\pgfusepath{clip}%
\pgfsetbuttcap%
\pgfsetroundjoin%
\pgfsetlinewidth{1.756562pt}%
\definecolor{currentstroke}{rgb}{0.333333,0.658824,0.407843}%
\pgfsetstrokecolor{currentstroke}%
\pgfsetdash{}{0pt}%
\pgfpathmoveto{\pgfqpoint{3.306162in}{2.163005in}}%
\pgfpathlineto{\pgfqpoint{3.306162in}{2.212948in}}%
\pgfusepath{stroke}%
\end{pgfscope}%
\begin{pgfscope}%
\pgfpathrectangle{\pgfqpoint{0.978830in}{0.894194in}}{\pgfqpoint{6.871170in}{3.741255in}}%
\pgfusepath{clip}%
\pgfsetbuttcap%
\pgfsetroundjoin%
\pgfsetlinewidth{1.756562pt}%
\definecolor{currentstroke}{rgb}{0.333333,0.658824,0.407843}%
\pgfsetstrokecolor{currentstroke}%
\pgfsetdash{}{0pt}%
\pgfpathmoveto{\pgfqpoint{4.313665in}{2.163005in}}%
\pgfpathlineto{\pgfqpoint{4.313665in}{2.212948in}}%
\pgfusepath{stroke}%
\end{pgfscope}%
\begin{pgfscope}%
\pgfpathrectangle{\pgfqpoint{0.978830in}{0.894194in}}{\pgfqpoint{6.871170in}{3.741255in}}%
\pgfusepath{clip}%
\pgfsetbuttcap%
\pgfsetroundjoin%
\pgfsetlinewidth{1.756562pt}%
\definecolor{currentstroke}{rgb}{0.333333,0.658824,0.407843}%
\pgfsetstrokecolor{currentstroke}%
\pgfsetdash{}{0pt}%
\pgfpathmoveto{\pgfqpoint{4.817416in}{2.342801in}}%
\pgfpathlineto{\pgfqpoint{4.817416in}{2.392744in}}%
\pgfusepath{stroke}%
\end{pgfscope}%
\begin{pgfscope}%
\pgfpathrectangle{\pgfqpoint{0.978830in}{0.894194in}}{\pgfqpoint{6.871170in}{3.741255in}}%
\pgfusepath{clip}%
\pgfsetbuttcap%
\pgfsetroundjoin%
\pgfsetlinewidth{1.756562pt}%
\definecolor{currentstroke}{rgb}{0.333333,0.658824,0.407843}%
\pgfsetstrokecolor{currentstroke}%
\pgfsetdash{}{0pt}%
\pgfpathmoveto{\pgfqpoint{4.817416in}{2.577534in}}%
\pgfpathlineto{\pgfqpoint{4.817416in}{2.627478in}}%
\pgfusepath{stroke}%
\end{pgfscope}%
\begin{pgfscope}%
\pgfpathrectangle{\pgfqpoint{0.978830in}{0.894194in}}{\pgfqpoint{6.871170in}{3.741255in}}%
\pgfusepath{clip}%
\pgfsetbuttcap%
\pgfsetroundjoin%
\pgfsetlinewidth{1.756562pt}%
\definecolor{currentstroke}{rgb}{0.333333,0.658824,0.407843}%
\pgfsetstrokecolor{currentstroke}%
\pgfsetdash{}{0pt}%
\pgfpathmoveto{\pgfqpoint{4.918167in}{2.552563in}}%
\pgfpathlineto{\pgfqpoint{4.918167in}{2.602506in}}%
\pgfusepath{stroke}%
\end{pgfscope}%
\begin{pgfscope}%
\pgfpathrectangle{\pgfqpoint{0.978830in}{0.894194in}}{\pgfqpoint{6.871170in}{3.741255in}}%
\pgfusepath{clip}%
\pgfsetbuttcap%
\pgfsetroundjoin%
\pgfsetlinewidth{1.756562pt}%
\definecolor{currentstroke}{rgb}{0.333333,0.658824,0.407843}%
\pgfsetstrokecolor{currentstroke}%
\pgfsetdash{}{0pt}%
\pgfpathmoveto{\pgfqpoint{5.018917in}{2.597512in}}%
\pgfpathlineto{\pgfqpoint{5.018917in}{2.647455in}}%
\pgfusepath{stroke}%
\end{pgfscope}%
\begin{pgfscope}%
\pgfpathrectangle{\pgfqpoint{0.978830in}{0.894194in}}{\pgfqpoint{6.871170in}{3.741255in}}%
\pgfusepath{clip}%
\pgfsetbuttcap%
\pgfsetroundjoin%
\pgfsetlinewidth{1.756562pt}%
\definecolor{currentstroke}{rgb}{0.333333,0.658824,0.407843}%
\pgfsetstrokecolor{currentstroke}%
\pgfsetdash{}{0pt}%
\pgfpathmoveto{\pgfqpoint{5.119667in}{2.602506in}}%
\pgfpathlineto{\pgfqpoint{5.119667in}{2.652449in}}%
\pgfusepath{stroke}%
\end{pgfscope}%
\begin{pgfscope}%
\pgfpathrectangle{\pgfqpoint{0.978830in}{0.894194in}}{\pgfqpoint{6.871170in}{3.741255in}}%
\pgfusepath{clip}%
\pgfsetbuttcap%
\pgfsetroundjoin%
\pgfsetlinewidth{1.756562pt}%
\definecolor{currentstroke}{rgb}{0.333333,0.658824,0.407843}%
\pgfsetstrokecolor{currentstroke}%
\pgfsetdash{}{0pt}%
\pgfpathmoveto{\pgfqpoint{5.220417in}{2.607500in}}%
\pgfpathlineto{\pgfqpoint{5.220417in}{2.657444in}}%
\pgfusepath{stroke}%
\end{pgfscope}%
\begin{pgfscope}%
\pgfpathrectangle{\pgfqpoint{0.978830in}{0.894194in}}{\pgfqpoint{6.871170in}{3.741255in}}%
\pgfusepath{clip}%
\pgfsetbuttcap%
\pgfsetroundjoin%
\pgfsetlinewidth{1.756562pt}%
\definecolor{currentstroke}{rgb}{0.333333,0.658824,0.407843}%
\pgfsetstrokecolor{currentstroke}%
\pgfsetdash{}{0pt}%
\pgfpathmoveto{\pgfqpoint{5.321168in}{2.617489in}}%
\pgfpathlineto{\pgfqpoint{5.321168in}{2.667432in}}%
\pgfusepath{stroke}%
\end{pgfscope}%
\begin{pgfscope}%
\pgfpathrectangle{\pgfqpoint{0.978830in}{0.894194in}}{\pgfqpoint{6.871170in}{3.741255in}}%
\pgfusepath{clip}%
\pgfsetbuttcap%
\pgfsetroundjoin%
\pgfsetlinewidth{1.756562pt}%
\definecolor{currentstroke}{rgb}{0.333333,0.658824,0.407843}%
\pgfsetstrokecolor{currentstroke}%
\pgfsetdash{}{0pt}%
\pgfpathmoveto{\pgfqpoint{5.321168in}{2.507614in}}%
\pgfpathlineto{\pgfqpoint{5.321168in}{2.557557in}}%
\pgfusepath{stroke}%
\end{pgfscope}%
\begin{pgfscope}%
\pgfpathrectangle{\pgfqpoint{0.978830in}{0.894194in}}{\pgfqpoint{6.871170in}{3.741255in}}%
\pgfusepath{clip}%
\pgfsetbuttcap%
\pgfsetroundjoin%
\pgfsetlinewidth{1.756562pt}%
\definecolor{currentstroke}{rgb}{0.333333,0.658824,0.407843}%
\pgfsetstrokecolor{currentstroke}%
\pgfsetdash{}{0pt}%
\pgfpathmoveto{\pgfqpoint{5.421918in}{2.662438in}}%
\pgfpathlineto{\pgfqpoint{5.421918in}{2.712381in}}%
\pgfusepath{stroke}%
\end{pgfscope}%
\begin{pgfscope}%
\pgfpathrectangle{\pgfqpoint{0.978830in}{0.894194in}}{\pgfqpoint{6.871170in}{3.741255in}}%
\pgfusepath{clip}%
\pgfsetbuttcap%
\pgfsetroundjoin%
\pgfsetlinewidth{1.756562pt}%
\definecolor{currentstroke}{rgb}{0.333333,0.658824,0.407843}%
\pgfsetstrokecolor{currentstroke}%
\pgfsetdash{}{0pt}%
\pgfpathmoveto{\pgfqpoint{5.522668in}{2.672427in}}%
\pgfpathlineto{\pgfqpoint{5.522668in}{2.722370in}}%
\pgfusepath{stroke}%
\end{pgfscope}%
\begin{pgfscope}%
\pgfpathrectangle{\pgfqpoint{0.978830in}{0.894194in}}{\pgfqpoint{6.871170in}{3.741255in}}%
\pgfusepath{clip}%
\pgfsetbuttcap%
\pgfsetroundjoin%
\pgfsetlinewidth{1.756562pt}%
\definecolor{currentstroke}{rgb}{0.333333,0.658824,0.407843}%
\pgfsetstrokecolor{currentstroke}%
\pgfsetdash{}{0pt}%
\pgfpathmoveto{\pgfqpoint{5.623419in}{2.672427in}}%
\pgfpathlineto{\pgfqpoint{5.623419in}{2.722370in}}%
\pgfusepath{stroke}%
\end{pgfscope}%
\begin{pgfscope}%
\pgfpathrectangle{\pgfqpoint{0.978830in}{0.894194in}}{\pgfqpoint{6.871170in}{3.741255in}}%
\pgfusepath{clip}%
\pgfsetbuttcap%
\pgfsetroundjoin%
\pgfsetlinewidth{1.756562pt}%
\definecolor{currentstroke}{rgb}{0.333333,0.658824,0.407843}%
\pgfsetstrokecolor{currentstroke}%
\pgfsetdash{}{0pt}%
\pgfpathmoveto{\pgfqpoint{5.724169in}{2.702393in}}%
\pgfpathlineto{\pgfqpoint{5.724169in}{2.752336in}}%
\pgfusepath{stroke}%
\end{pgfscope}%
\begin{pgfscope}%
\pgfpathrectangle{\pgfqpoint{0.978830in}{0.894194in}}{\pgfqpoint{6.871170in}{3.741255in}}%
\pgfusepath{clip}%
\pgfsetbuttcap%
\pgfsetroundjoin%
\pgfsetlinewidth{1.756562pt}%
\definecolor{currentstroke}{rgb}{0.333333,0.658824,0.407843}%
\pgfsetstrokecolor{currentstroke}%
\pgfsetdash{}{0pt}%
\pgfpathmoveto{\pgfqpoint{5.824919in}{2.717376in}}%
\pgfpathlineto{\pgfqpoint{5.824919in}{2.767319in}}%
\pgfusepath{stroke}%
\end{pgfscope}%
\begin{pgfscope}%
\pgfpathrectangle{\pgfqpoint{0.978830in}{0.894194in}}{\pgfqpoint{6.871170in}{3.741255in}}%
\pgfusepath{clip}%
\pgfsetbuttcap%
\pgfsetroundjoin%
\pgfsetlinewidth{1.756562pt}%
\definecolor{currentstroke}{rgb}{0.333333,0.658824,0.407843}%
\pgfsetstrokecolor{currentstroke}%
\pgfsetdash{}{0pt}%
\pgfpathmoveto{\pgfqpoint{5.925669in}{2.757330in}}%
\pgfpathlineto{\pgfqpoint{5.925669in}{2.807274in}}%
\pgfusepath{stroke}%
\end{pgfscope}%
\begin{pgfscope}%
\pgfpathrectangle{\pgfqpoint{0.978830in}{0.894194in}}{\pgfqpoint{6.871170in}{3.741255in}}%
\pgfusepath{clip}%
\pgfsetbuttcap%
\pgfsetroundjoin%
\pgfsetlinewidth{1.756562pt}%
\definecolor{currentstroke}{rgb}{0.333333,0.658824,0.407843}%
\pgfsetstrokecolor{currentstroke}%
\pgfsetdash{}{0pt}%
\pgfpathmoveto{\pgfqpoint{6.026420in}{2.772313in}}%
\pgfpathlineto{\pgfqpoint{6.026420in}{2.822257in}}%
\pgfusepath{stroke}%
\end{pgfscope}%
\begin{pgfscope}%
\pgfpathrectangle{\pgfqpoint{0.978830in}{0.894194in}}{\pgfqpoint{6.871170in}{3.741255in}}%
\pgfusepath{clip}%
\pgfsetbuttcap%
\pgfsetroundjoin%
\pgfsetlinewidth{1.756562pt}%
\definecolor{currentstroke}{rgb}{0.333333,0.658824,0.407843}%
\pgfsetstrokecolor{currentstroke}%
\pgfsetdash{}{0pt}%
\pgfpathmoveto{\pgfqpoint{6.127170in}{2.822257in}}%
\pgfpathlineto{\pgfqpoint{6.127170in}{2.872200in}}%
\pgfusepath{stroke}%
\end{pgfscope}%
\begin{pgfscope}%
\pgfpathrectangle{\pgfqpoint{0.978830in}{0.894194in}}{\pgfqpoint{6.871170in}{3.741255in}}%
\pgfusepath{clip}%
\pgfsetbuttcap%
\pgfsetroundjoin%
\pgfsetlinewidth{1.756562pt}%
\definecolor{currentstroke}{rgb}{0.333333,0.658824,0.407843}%
\pgfsetstrokecolor{currentstroke}%
\pgfsetdash{}{0pt}%
\pgfpathmoveto{\pgfqpoint{6.227920in}{2.867206in}}%
\pgfpathlineto{\pgfqpoint{6.227920in}{2.917149in}}%
\pgfusepath{stroke}%
\end{pgfscope}%
\begin{pgfscope}%
\pgfpathrectangle{\pgfqpoint{0.978830in}{0.894194in}}{\pgfqpoint{6.871170in}{3.741255in}}%
\pgfusepath{clip}%
\pgfsetbuttcap%
\pgfsetroundjoin%
\pgfsetlinewidth{1.756562pt}%
\definecolor{currentstroke}{rgb}{0.333333,0.658824,0.407843}%
\pgfsetstrokecolor{currentstroke}%
\pgfsetdash{}{0pt}%
\pgfpathmoveto{\pgfqpoint{6.328671in}{2.952109in}}%
\pgfpathlineto{\pgfqpoint{6.328671in}{3.002053in}}%
\pgfusepath{stroke}%
\end{pgfscope}%
\begin{pgfscope}%
\pgfpathrectangle{\pgfqpoint{0.978830in}{0.894194in}}{\pgfqpoint{6.871170in}{3.741255in}}%
\pgfusepath{clip}%
\pgfsetbuttcap%
\pgfsetroundjoin%
\pgfsetlinewidth{1.756562pt}%
\definecolor{currentstroke}{rgb}{0.333333,0.658824,0.407843}%
\pgfsetstrokecolor{currentstroke}%
\pgfsetdash{}{0pt}%
\pgfpathmoveto{\pgfqpoint{6.328671in}{2.912155in}}%
\pgfpathlineto{\pgfqpoint{6.328671in}{2.962098in}}%
\pgfusepath{stroke}%
\end{pgfscope}%
\begin{pgfscope}%
\pgfpathrectangle{\pgfqpoint{0.978830in}{0.894194in}}{\pgfqpoint{6.871170in}{3.741255in}}%
\pgfusepath{clip}%
\pgfsetbuttcap%
\pgfsetroundjoin%
\pgfsetlinewidth{1.756562pt}%
\definecolor{currentstroke}{rgb}{0.333333,0.658824,0.407843}%
\pgfsetstrokecolor{currentstroke}%
\pgfsetdash{}{0pt}%
\pgfpathmoveto{\pgfqpoint{6.429421in}{2.972087in}}%
\pgfpathlineto{\pgfqpoint{6.429421in}{3.022030in}}%
\pgfusepath{stroke}%
\end{pgfscope}%
\begin{pgfscope}%
\pgfpathrectangle{\pgfqpoint{0.978830in}{0.894194in}}{\pgfqpoint{6.871170in}{3.741255in}}%
\pgfusepath{clip}%
\pgfsetbuttcap%
\pgfsetroundjoin%
\pgfsetlinewidth{1.756562pt}%
\definecolor{currentstroke}{rgb}{0.333333,0.658824,0.407843}%
\pgfsetstrokecolor{currentstroke}%
\pgfsetdash{}{0pt}%
\pgfpathmoveto{\pgfqpoint{6.530171in}{3.032019in}}%
\pgfpathlineto{\pgfqpoint{6.530171in}{3.081962in}}%
\pgfusepath{stroke}%
\end{pgfscope}%
\begin{pgfscope}%
\pgfpathrectangle{\pgfqpoint{0.978830in}{0.894194in}}{\pgfqpoint{6.871170in}{3.741255in}}%
\pgfusepath{clip}%
\pgfsetbuttcap%
\pgfsetroundjoin%
\pgfsetlinewidth{1.756562pt}%
\definecolor{currentstroke}{rgb}{0.333333,0.658824,0.407843}%
\pgfsetstrokecolor{currentstroke}%
\pgfsetdash{}{0pt}%
\pgfpathmoveto{\pgfqpoint{6.630921in}{3.091951in}}%
\pgfpathlineto{\pgfqpoint{6.630921in}{3.141894in}}%
\pgfusepath{stroke}%
\end{pgfscope}%
\begin{pgfscope}%
\pgfpathrectangle{\pgfqpoint{0.978830in}{0.894194in}}{\pgfqpoint{6.871170in}{3.741255in}}%
\pgfusepath{clip}%
\pgfsetbuttcap%
\pgfsetroundjoin%
\pgfsetlinewidth{1.756562pt}%
\definecolor{currentstroke}{rgb}{0.333333,0.658824,0.407843}%
\pgfsetstrokecolor{currentstroke}%
\pgfsetdash{}{0pt}%
\pgfpathmoveto{\pgfqpoint{6.731672in}{3.166866in}}%
\pgfpathlineto{\pgfqpoint{6.731672in}{3.216809in}}%
\pgfusepath{stroke}%
\end{pgfscope}%
\begin{pgfscope}%
\pgfpathrectangle{\pgfqpoint{0.978830in}{0.894194in}}{\pgfqpoint{6.871170in}{3.741255in}}%
\pgfusepath{clip}%
\pgfsetbuttcap%
\pgfsetroundjoin%
\pgfsetlinewidth{1.756562pt}%
\definecolor{currentstroke}{rgb}{0.333333,0.658824,0.407843}%
\pgfsetstrokecolor{currentstroke}%
\pgfsetdash{}{0pt}%
\pgfpathmoveto{\pgfqpoint{6.832422in}{3.266752in}}%
\pgfpathlineto{\pgfqpoint{6.832422in}{3.316696in}}%
\pgfusepath{stroke}%
\end{pgfscope}%
\begin{pgfscope}%
\pgfpathrectangle{\pgfqpoint{0.978830in}{0.894194in}}{\pgfqpoint{6.871170in}{3.741255in}}%
\pgfusepath{clip}%
\pgfsetbuttcap%
\pgfsetroundjoin%
\pgfsetlinewidth{1.756562pt}%
\definecolor{currentstroke}{rgb}{0.333333,0.658824,0.407843}%
\pgfsetstrokecolor{currentstroke}%
\pgfsetdash{}{0pt}%
\pgfpathmoveto{\pgfqpoint{6.933172in}{3.366639in}}%
\pgfpathlineto{\pgfqpoint{6.933172in}{3.416582in}}%
\pgfusepath{stroke}%
\end{pgfscope}%
\begin{pgfscope}%
\pgfpathrectangle{\pgfqpoint{0.978830in}{0.894194in}}{\pgfqpoint{6.871170in}{3.741255in}}%
\pgfusepath{clip}%
\pgfsetbuttcap%
\pgfsetroundjoin%
\pgfsetlinewidth{1.756562pt}%
\definecolor{currentstroke}{rgb}{0.333333,0.658824,0.407843}%
\pgfsetstrokecolor{currentstroke}%
\pgfsetdash{}{0pt}%
\pgfpathmoveto{\pgfqpoint{7.033923in}{3.501486in}}%
\pgfpathlineto{\pgfqpoint{7.033923in}{3.551429in}}%
\pgfusepath{stroke}%
\end{pgfscope}%
\begin{pgfscope}%
\pgfpathrectangle{\pgfqpoint{0.978830in}{0.894194in}}{\pgfqpoint{6.871170in}{3.741255in}}%
\pgfusepath{clip}%
\pgfsetbuttcap%
\pgfsetroundjoin%
\pgfsetlinewidth{1.756562pt}%
\definecolor{currentstroke}{rgb}{0.333333,0.658824,0.407843}%
\pgfsetstrokecolor{currentstroke}%
\pgfsetdash{}{0pt}%
\pgfpathmoveto{\pgfqpoint{7.134673in}{3.661305in}}%
\pgfpathlineto{\pgfqpoint{7.134673in}{3.711248in}}%
\pgfusepath{stroke}%
\end{pgfscope}%
\begin{pgfscope}%
\pgfpathrectangle{\pgfqpoint{0.978830in}{0.894194in}}{\pgfqpoint{6.871170in}{3.741255in}}%
\pgfusepath{clip}%
\pgfsetbuttcap%
\pgfsetroundjoin%
\pgfsetlinewidth{1.756562pt}%
\definecolor{currentstroke}{rgb}{0.333333,0.658824,0.407843}%
\pgfsetstrokecolor{currentstroke}%
\pgfsetdash{}{0pt}%
\pgfpathmoveto{\pgfqpoint{7.235423in}{3.811135in}}%
\pgfpathlineto{\pgfqpoint{7.235423in}{3.861078in}}%
\pgfusepath{stroke}%
\end{pgfscope}%
\begin{pgfscope}%
\pgfpathrectangle{\pgfqpoint{0.978830in}{0.894194in}}{\pgfqpoint{6.871170in}{3.741255in}}%
\pgfusepath{clip}%
\pgfsetbuttcap%
\pgfsetroundjoin%
\pgfsetlinewidth{1.756562pt}%
\definecolor{currentstroke}{rgb}{0.333333,0.658824,0.407843}%
\pgfsetstrokecolor{currentstroke}%
\pgfsetdash{}{0pt}%
\pgfpathmoveto{\pgfqpoint{7.336174in}{4.010908in}}%
\pgfpathlineto{\pgfqpoint{7.336174in}{4.060851in}}%
\pgfusepath{stroke}%
\end{pgfscope}%
\begin{pgfscope}%
\pgfpathrectangle{\pgfqpoint{0.978830in}{0.894194in}}{\pgfqpoint{6.871170in}{3.741255in}}%
\pgfusepath{clip}%
\pgfsetbuttcap%
\pgfsetroundjoin%
\pgfsetlinewidth{1.756562pt}%
\definecolor{currentstroke}{rgb}{0.333333,0.658824,0.407843}%
\pgfsetstrokecolor{currentstroke}%
\pgfsetdash{}{0pt}%
\pgfpathmoveto{\pgfqpoint{7.436924in}{4.200693in}}%
\pgfpathlineto{\pgfqpoint{7.436924in}{4.250636in}}%
\pgfusepath{stroke}%
\end{pgfscope}%
\begin{pgfscope}%
\pgfpathrectangle{\pgfqpoint{0.978830in}{0.894194in}}{\pgfqpoint{6.871170in}{3.741255in}}%
\pgfusepath{clip}%
\pgfsetbuttcap%
\pgfsetroundjoin%
\pgfsetlinewidth{1.756562pt}%
\definecolor{currentstroke}{rgb}{0.333333,0.658824,0.407843}%
\pgfsetstrokecolor{currentstroke}%
\pgfsetdash{}{0pt}%
\pgfpathmoveto{\pgfqpoint{7.537674in}{4.415449in}}%
\pgfpathlineto{\pgfqpoint{7.537674in}{4.465392in}}%
\pgfusepath{stroke}%
\end{pgfscope}%
\begin{pgfscope}%
\pgfpathrectangle{\pgfqpoint{0.978830in}{0.894194in}}{\pgfqpoint{6.871170in}{3.741255in}}%
\pgfusepath{clip}%
\pgfsetbuttcap%
\pgfsetroundjoin%
\definecolor{currentfill}{rgb}{0.333333,0.658824,0.407843}%
\pgfsetfillcolor{currentfill}%
\pgfsetlinewidth{1.003750pt}%
\definecolor{currentstroke}{rgb}{0.333333,0.658824,0.407843}%
\pgfsetstrokecolor{currentstroke}%
\pgfsetdash{}{0pt}%
\pgfsys@defobject{currentmarker}{\pgfqpoint{-0.069444in}{-0.000000in}}{\pgfqpoint{0.069444in}{0.000000in}}{%
\pgfpathmoveto{\pgfqpoint{0.069444in}{-0.000000in}}%
\pgfpathlineto{\pgfqpoint{-0.069444in}{0.000000in}}%
\pgfusepath{stroke,fill}%
}%
\begin{pgfscope}%
\pgfsys@transformshift{1.291156in}{1.244047in}%
\pgfsys@useobject{currentmarker}{}%
\end{pgfscope}%
\begin{pgfscope}%
\pgfsys@transformshift{2.234179in}{1.333945in}%
\pgfsys@useobject{currentmarker}{}%
\end{pgfscope}%
\begin{pgfscope}%
\pgfsys@transformshift{2.903161in}{1.508747in}%
\pgfsys@useobject{currentmarker}{}%
\end{pgfscope}%
\begin{pgfscope}%
\pgfsys@transformshift{3.306162in}{2.163005in}%
\pgfsys@useobject{currentmarker}{}%
\end{pgfscope}%
\begin{pgfscope}%
\pgfsys@transformshift{4.313665in}{2.163005in}%
\pgfsys@useobject{currentmarker}{}%
\end{pgfscope}%
\begin{pgfscope}%
\pgfsys@transformshift{4.817416in}{2.342801in}%
\pgfsys@useobject{currentmarker}{}%
\end{pgfscope}%
\begin{pgfscope}%
\pgfsys@transformshift{4.817416in}{2.577534in}%
\pgfsys@useobject{currentmarker}{}%
\end{pgfscope}%
\begin{pgfscope}%
\pgfsys@transformshift{4.918167in}{2.552563in}%
\pgfsys@useobject{currentmarker}{}%
\end{pgfscope}%
\begin{pgfscope}%
\pgfsys@transformshift{5.018917in}{2.597512in}%
\pgfsys@useobject{currentmarker}{}%
\end{pgfscope}%
\begin{pgfscope}%
\pgfsys@transformshift{5.119667in}{2.602506in}%
\pgfsys@useobject{currentmarker}{}%
\end{pgfscope}%
\begin{pgfscope}%
\pgfsys@transformshift{5.220417in}{2.607500in}%
\pgfsys@useobject{currentmarker}{}%
\end{pgfscope}%
\begin{pgfscope}%
\pgfsys@transformshift{5.321168in}{2.617489in}%
\pgfsys@useobject{currentmarker}{}%
\end{pgfscope}%
\begin{pgfscope}%
\pgfsys@transformshift{5.321168in}{2.507614in}%
\pgfsys@useobject{currentmarker}{}%
\end{pgfscope}%
\begin{pgfscope}%
\pgfsys@transformshift{5.421918in}{2.662438in}%
\pgfsys@useobject{currentmarker}{}%
\end{pgfscope}%
\begin{pgfscope}%
\pgfsys@transformshift{5.522668in}{2.672427in}%
\pgfsys@useobject{currentmarker}{}%
\end{pgfscope}%
\begin{pgfscope}%
\pgfsys@transformshift{5.623419in}{2.672427in}%
\pgfsys@useobject{currentmarker}{}%
\end{pgfscope}%
\begin{pgfscope}%
\pgfsys@transformshift{5.724169in}{2.702393in}%
\pgfsys@useobject{currentmarker}{}%
\end{pgfscope}%
\begin{pgfscope}%
\pgfsys@transformshift{5.824919in}{2.717376in}%
\pgfsys@useobject{currentmarker}{}%
\end{pgfscope}%
\begin{pgfscope}%
\pgfsys@transformshift{5.925669in}{2.757330in}%
\pgfsys@useobject{currentmarker}{}%
\end{pgfscope}%
\begin{pgfscope}%
\pgfsys@transformshift{6.026420in}{2.772313in}%
\pgfsys@useobject{currentmarker}{}%
\end{pgfscope}%
\begin{pgfscope}%
\pgfsys@transformshift{6.127170in}{2.822257in}%
\pgfsys@useobject{currentmarker}{}%
\end{pgfscope}%
\begin{pgfscope}%
\pgfsys@transformshift{6.227920in}{2.867206in}%
\pgfsys@useobject{currentmarker}{}%
\end{pgfscope}%
\begin{pgfscope}%
\pgfsys@transformshift{6.328671in}{2.952109in}%
\pgfsys@useobject{currentmarker}{}%
\end{pgfscope}%
\begin{pgfscope}%
\pgfsys@transformshift{6.328671in}{2.912155in}%
\pgfsys@useobject{currentmarker}{}%
\end{pgfscope}%
\begin{pgfscope}%
\pgfsys@transformshift{6.429421in}{2.972087in}%
\pgfsys@useobject{currentmarker}{}%
\end{pgfscope}%
\begin{pgfscope}%
\pgfsys@transformshift{6.530171in}{3.032019in}%
\pgfsys@useobject{currentmarker}{}%
\end{pgfscope}%
\begin{pgfscope}%
\pgfsys@transformshift{6.630921in}{3.091951in}%
\pgfsys@useobject{currentmarker}{}%
\end{pgfscope}%
\begin{pgfscope}%
\pgfsys@transformshift{6.731672in}{3.166866in}%
\pgfsys@useobject{currentmarker}{}%
\end{pgfscope}%
\begin{pgfscope}%
\pgfsys@transformshift{6.832422in}{3.266752in}%
\pgfsys@useobject{currentmarker}{}%
\end{pgfscope}%
\begin{pgfscope}%
\pgfsys@transformshift{6.933172in}{3.366639in}%
\pgfsys@useobject{currentmarker}{}%
\end{pgfscope}%
\begin{pgfscope}%
\pgfsys@transformshift{7.033923in}{3.501486in}%
\pgfsys@useobject{currentmarker}{}%
\end{pgfscope}%
\begin{pgfscope}%
\pgfsys@transformshift{7.134673in}{3.661305in}%
\pgfsys@useobject{currentmarker}{}%
\end{pgfscope}%
\begin{pgfscope}%
\pgfsys@transformshift{7.235423in}{3.811135in}%
\pgfsys@useobject{currentmarker}{}%
\end{pgfscope}%
\begin{pgfscope}%
\pgfsys@transformshift{7.336174in}{4.010908in}%
\pgfsys@useobject{currentmarker}{}%
\end{pgfscope}%
\begin{pgfscope}%
\pgfsys@transformshift{7.436924in}{4.200693in}%
\pgfsys@useobject{currentmarker}{}%
\end{pgfscope}%
\begin{pgfscope}%
\pgfsys@transformshift{7.537674in}{4.415449in}%
\pgfsys@useobject{currentmarker}{}%
\end{pgfscope}%
\end{pgfscope}%
\begin{pgfscope}%
\pgfpathrectangle{\pgfqpoint{0.978830in}{0.894194in}}{\pgfqpoint{6.871170in}{3.741255in}}%
\pgfusepath{clip}%
\pgfsetbuttcap%
\pgfsetroundjoin%
\definecolor{currentfill}{rgb}{0.333333,0.658824,0.407843}%
\pgfsetfillcolor{currentfill}%
\pgfsetlinewidth{1.003750pt}%
\definecolor{currentstroke}{rgb}{0.333333,0.658824,0.407843}%
\pgfsetstrokecolor{currentstroke}%
\pgfsetdash{}{0pt}%
\pgfsys@defobject{currentmarker}{\pgfqpoint{-0.069444in}{-0.000000in}}{\pgfqpoint{0.069444in}{0.000000in}}{%
\pgfpathmoveto{\pgfqpoint{0.069444in}{-0.000000in}}%
\pgfpathlineto{\pgfqpoint{-0.069444in}{0.000000in}}%
\pgfusepath{stroke,fill}%
}%
\begin{pgfscope}%
\pgfsys@transformshift{1.291156in}{1.293991in}%
\pgfsys@useobject{currentmarker}{}%
\end{pgfscope}%
\begin{pgfscope}%
\pgfsys@transformshift{2.234179in}{1.383889in}%
\pgfsys@useobject{currentmarker}{}%
\end{pgfscope}%
\begin{pgfscope}%
\pgfsys@transformshift{2.903161in}{1.558690in}%
\pgfsys@useobject{currentmarker}{}%
\end{pgfscope}%
\begin{pgfscope}%
\pgfsys@transformshift{3.306162in}{2.212948in}%
\pgfsys@useobject{currentmarker}{}%
\end{pgfscope}%
\begin{pgfscope}%
\pgfsys@transformshift{4.313665in}{2.212948in}%
\pgfsys@useobject{currentmarker}{}%
\end{pgfscope}%
\begin{pgfscope}%
\pgfsys@transformshift{4.817416in}{2.392744in}%
\pgfsys@useobject{currentmarker}{}%
\end{pgfscope}%
\begin{pgfscope}%
\pgfsys@transformshift{4.817416in}{2.627478in}%
\pgfsys@useobject{currentmarker}{}%
\end{pgfscope}%
\begin{pgfscope}%
\pgfsys@transformshift{4.918167in}{2.602506in}%
\pgfsys@useobject{currentmarker}{}%
\end{pgfscope}%
\begin{pgfscope}%
\pgfsys@transformshift{5.018917in}{2.647455in}%
\pgfsys@useobject{currentmarker}{}%
\end{pgfscope}%
\begin{pgfscope}%
\pgfsys@transformshift{5.119667in}{2.652449in}%
\pgfsys@useobject{currentmarker}{}%
\end{pgfscope}%
\begin{pgfscope}%
\pgfsys@transformshift{5.220417in}{2.657444in}%
\pgfsys@useobject{currentmarker}{}%
\end{pgfscope}%
\begin{pgfscope}%
\pgfsys@transformshift{5.321168in}{2.667432in}%
\pgfsys@useobject{currentmarker}{}%
\end{pgfscope}%
\begin{pgfscope}%
\pgfsys@transformshift{5.321168in}{2.557557in}%
\pgfsys@useobject{currentmarker}{}%
\end{pgfscope}%
\begin{pgfscope}%
\pgfsys@transformshift{5.421918in}{2.712381in}%
\pgfsys@useobject{currentmarker}{}%
\end{pgfscope}%
\begin{pgfscope}%
\pgfsys@transformshift{5.522668in}{2.722370in}%
\pgfsys@useobject{currentmarker}{}%
\end{pgfscope}%
\begin{pgfscope}%
\pgfsys@transformshift{5.623419in}{2.722370in}%
\pgfsys@useobject{currentmarker}{}%
\end{pgfscope}%
\begin{pgfscope}%
\pgfsys@transformshift{5.724169in}{2.752336in}%
\pgfsys@useobject{currentmarker}{}%
\end{pgfscope}%
\begin{pgfscope}%
\pgfsys@transformshift{5.824919in}{2.767319in}%
\pgfsys@useobject{currentmarker}{}%
\end{pgfscope}%
\begin{pgfscope}%
\pgfsys@transformshift{5.925669in}{2.807274in}%
\pgfsys@useobject{currentmarker}{}%
\end{pgfscope}%
\begin{pgfscope}%
\pgfsys@transformshift{6.026420in}{2.822257in}%
\pgfsys@useobject{currentmarker}{}%
\end{pgfscope}%
\begin{pgfscope}%
\pgfsys@transformshift{6.127170in}{2.872200in}%
\pgfsys@useobject{currentmarker}{}%
\end{pgfscope}%
\begin{pgfscope}%
\pgfsys@transformshift{6.227920in}{2.917149in}%
\pgfsys@useobject{currentmarker}{}%
\end{pgfscope}%
\begin{pgfscope}%
\pgfsys@transformshift{6.328671in}{3.002053in}%
\pgfsys@useobject{currentmarker}{}%
\end{pgfscope}%
\begin{pgfscope}%
\pgfsys@transformshift{6.328671in}{2.962098in}%
\pgfsys@useobject{currentmarker}{}%
\end{pgfscope}%
\begin{pgfscope}%
\pgfsys@transformshift{6.429421in}{3.022030in}%
\pgfsys@useobject{currentmarker}{}%
\end{pgfscope}%
\begin{pgfscope}%
\pgfsys@transformshift{6.530171in}{3.081962in}%
\pgfsys@useobject{currentmarker}{}%
\end{pgfscope}%
\begin{pgfscope}%
\pgfsys@transformshift{6.630921in}{3.141894in}%
\pgfsys@useobject{currentmarker}{}%
\end{pgfscope}%
\begin{pgfscope}%
\pgfsys@transformshift{6.731672in}{3.216809in}%
\pgfsys@useobject{currentmarker}{}%
\end{pgfscope}%
\begin{pgfscope}%
\pgfsys@transformshift{6.832422in}{3.316696in}%
\pgfsys@useobject{currentmarker}{}%
\end{pgfscope}%
\begin{pgfscope}%
\pgfsys@transformshift{6.933172in}{3.416582in}%
\pgfsys@useobject{currentmarker}{}%
\end{pgfscope}%
\begin{pgfscope}%
\pgfsys@transformshift{7.033923in}{3.551429in}%
\pgfsys@useobject{currentmarker}{}%
\end{pgfscope}%
\begin{pgfscope}%
\pgfsys@transformshift{7.134673in}{3.711248in}%
\pgfsys@useobject{currentmarker}{}%
\end{pgfscope}%
\begin{pgfscope}%
\pgfsys@transformshift{7.235423in}{3.861078in}%
\pgfsys@useobject{currentmarker}{}%
\end{pgfscope}%
\begin{pgfscope}%
\pgfsys@transformshift{7.336174in}{4.060851in}%
\pgfsys@useobject{currentmarker}{}%
\end{pgfscope}%
\begin{pgfscope}%
\pgfsys@transformshift{7.436924in}{4.250636in}%
\pgfsys@useobject{currentmarker}{}%
\end{pgfscope}%
\begin{pgfscope}%
\pgfsys@transformshift{7.537674in}{4.465392in}%
\pgfsys@useobject{currentmarker}{}%
\end{pgfscope}%
\end{pgfscope}%
\begin{pgfscope}%
\pgfpathrectangle{\pgfqpoint{0.978830in}{0.894194in}}{\pgfqpoint{6.871170in}{3.741255in}}%
\pgfusepath{clip}%
\pgfsetroundcap%
\pgfsetroundjoin%
\pgfsetlinewidth{1.756562pt}%
\definecolor{currentstroke}{rgb}{0.298039,0.447059,0.690196}%
\pgfsetstrokecolor{currentstroke}%
\pgfsetdash{}{0pt}%
\pgfpathmoveto{\pgfqpoint{1.291156in}{1.284002in}}%
\pgfpathlineto{\pgfqpoint{2.234179in}{1.179121in}}%
\pgfpathlineto{\pgfqpoint{2.903161in}{1.089223in}}%
\pgfpathlineto{\pgfqpoint{3.306162in}{1.149155in}}%
\pgfpathlineto{\pgfqpoint{4.313665in}{1.134172in}}%
\pgfpathlineto{\pgfqpoint{4.817416in}{1.149155in}}%
\pgfpathlineto{\pgfqpoint{4.817416in}{1.363911in}}%
\pgfpathlineto{\pgfqpoint{4.918167in}{1.373900in}}%
\pgfpathlineto{\pgfqpoint{5.018917in}{1.373900in}}%
\pgfpathlineto{\pgfqpoint{5.119667in}{1.378894in}}%
\pgfpathlineto{\pgfqpoint{5.220417in}{1.413855in}}%
\pgfpathlineto{\pgfqpoint{5.321168in}{1.428838in}}%
\pgfpathlineto{\pgfqpoint{5.321168in}{1.373900in}}%
\pgfpathlineto{\pgfqpoint{5.421918in}{1.513741in}}%
\pgfpathlineto{\pgfqpoint{5.522668in}{1.583662in}}%
\pgfpathlineto{\pgfqpoint{5.623419in}{1.683549in}}%
\pgfpathlineto{\pgfqpoint{5.724169in}{1.748475in}}%
\pgfpathlineto{\pgfqpoint{5.824919in}{1.848362in}}%
\pgfpathlineto{\pgfqpoint{5.925669in}{1.943254in}}%
\pgfpathlineto{\pgfqpoint{6.026420in}{2.013175in}}%
\pgfpathlineto{\pgfqpoint{6.127170in}{2.083095in}}%
\pgfpathlineto{\pgfqpoint{6.227920in}{2.158010in}}%
\pgfpathlineto{\pgfqpoint{6.328671in}{2.262891in}}%
\pgfpathlineto{\pgfqpoint{6.328671in}{2.227931in}}%
\pgfpathlineto{\pgfqpoint{6.429421in}{2.292857in}}%
\pgfpathlineto{\pgfqpoint{6.530171in}{2.357784in}}%
\pgfpathlineto{\pgfqpoint{6.630921in}{2.437693in}}%
\pgfpathlineto{\pgfqpoint{6.731672in}{2.497625in}}%
\pgfpathlineto{\pgfqpoint{6.832422in}{2.577534in}}%
\pgfpathlineto{\pgfqpoint{6.933172in}{2.687410in}}%
\pgfpathlineto{\pgfqpoint{7.033923in}{2.832245in}}%
\pgfpathlineto{\pgfqpoint{7.134673in}{3.037013in}}%
\pgfpathlineto{\pgfqpoint{7.235423in}{3.351656in}}%
\pgfpathlineto{\pgfqpoint{7.336174in}{3.691271in}}%
\pgfpathlineto{\pgfqpoint{7.436924in}{4.020897in}}%
\pgfpathlineto{\pgfqpoint{7.537674in}{4.310568in}}%
\pgfusepath{stroke}%
\end{pgfscope}%
\begin{pgfscope}%
\pgfpathrectangle{\pgfqpoint{0.978830in}{0.894194in}}{\pgfqpoint{6.871170in}{3.741255in}}%
\pgfusepath{clip}%
\pgfsetroundcap%
\pgfsetroundjoin%
\pgfsetlinewidth{1.756562pt}%
\definecolor{currentstroke}{rgb}{0.333333,0.658824,0.407843}%
\pgfsetstrokecolor{currentstroke}%
\pgfsetdash{}{0pt}%
\pgfpathmoveto{\pgfqpoint{1.291156in}{1.269019in}}%
\pgfpathlineto{\pgfqpoint{2.234179in}{1.358917in}}%
\pgfpathlineto{\pgfqpoint{2.903161in}{1.533719in}}%
\pgfpathlineto{\pgfqpoint{3.306162in}{2.187976in}}%
\pgfpathlineto{\pgfqpoint{4.313665in}{2.187976in}}%
\pgfpathlineto{\pgfqpoint{4.817416in}{2.367772in}}%
\pgfpathlineto{\pgfqpoint{4.817416in}{2.602506in}}%
\pgfpathlineto{\pgfqpoint{4.918167in}{2.577534in}}%
\pgfpathlineto{\pgfqpoint{5.018917in}{2.622483in}}%
\pgfpathlineto{\pgfqpoint{5.119667in}{2.627478in}}%
\pgfpathlineto{\pgfqpoint{5.220417in}{2.632472in}}%
\pgfpathlineto{\pgfqpoint{5.321168in}{2.642461in}}%
\pgfpathlineto{\pgfqpoint{5.321168in}{2.532585in}}%
\pgfpathlineto{\pgfqpoint{5.421918in}{2.687410in}}%
\pgfpathlineto{\pgfqpoint{5.522668in}{2.697398in}}%
\pgfpathlineto{\pgfqpoint{5.623419in}{2.697398in}}%
\pgfpathlineto{\pgfqpoint{5.724169in}{2.727364in}}%
\pgfpathlineto{\pgfqpoint{5.824919in}{2.742347in}}%
\pgfpathlineto{\pgfqpoint{5.925669in}{2.782302in}}%
\pgfpathlineto{\pgfqpoint{6.026420in}{2.797285in}}%
\pgfpathlineto{\pgfqpoint{6.127170in}{2.847228in}}%
\pgfpathlineto{\pgfqpoint{6.227920in}{2.892177in}}%
\pgfpathlineto{\pgfqpoint{6.328671in}{2.977081in}}%
\pgfpathlineto{\pgfqpoint{6.328671in}{2.937126in}}%
\pgfpathlineto{\pgfqpoint{6.429421in}{2.997058in}}%
\pgfpathlineto{\pgfqpoint{6.530171in}{3.056990in}}%
\pgfpathlineto{\pgfqpoint{6.630921in}{3.116922in}}%
\pgfpathlineto{\pgfqpoint{6.731672in}{3.191837in}}%
\pgfpathlineto{\pgfqpoint{6.832422in}{3.291724in}}%
\pgfpathlineto{\pgfqpoint{6.933172in}{3.391611in}}%
\pgfpathlineto{\pgfqpoint{7.033923in}{3.526458in}}%
\pgfpathlineto{\pgfqpoint{7.134673in}{3.686276in}}%
\pgfpathlineto{\pgfqpoint{7.235423in}{3.836106in}}%
\pgfpathlineto{\pgfqpoint{7.336174in}{4.035880in}}%
\pgfpathlineto{\pgfqpoint{7.436924in}{4.225664in}}%
\pgfpathlineto{\pgfqpoint{7.537674in}{4.440421in}}%
\pgfusepath{stroke}%
\end{pgfscope}%
\begin{pgfscope}%
\pgfsetrectcap%
\pgfsetmiterjoin%
\pgfsetlinewidth{0.000000pt}%
\definecolor{currentstroke}{rgb}{1.000000,1.000000,1.000000}%
\pgfsetstrokecolor{currentstroke}%
\pgfsetdash{}{0pt}%
\pgfpathmoveto{\pgfqpoint{0.978830in}{0.894194in}}%
\pgfpathlineto{\pgfqpoint{0.978830in}{4.635449in}}%
\pgfusepath{}%
\end{pgfscope}%
\begin{pgfscope}%
\pgfsetrectcap%
\pgfsetmiterjoin%
\pgfsetlinewidth{0.000000pt}%
\definecolor{currentstroke}{rgb}{1.000000,1.000000,1.000000}%
\pgfsetstrokecolor{currentstroke}%
\pgfsetdash{}{0pt}%
\pgfpathmoveto{\pgfqpoint{7.850000in}{0.894194in}}%
\pgfpathlineto{\pgfqpoint{7.850000in}{4.635449in}}%
\pgfusepath{}%
\end{pgfscope}%
\begin{pgfscope}%
\pgfsetrectcap%
\pgfsetmiterjoin%
\pgfsetlinewidth{0.000000pt}%
\definecolor{currentstroke}{rgb}{1.000000,1.000000,1.000000}%
\pgfsetstrokecolor{currentstroke}%
\pgfsetdash{}{0pt}%
\pgfpathmoveto{\pgfqpoint{0.978830in}{0.894194in}}%
\pgfpathlineto{\pgfqpoint{7.850000in}{0.894194in}}%
\pgfusepath{}%
\end{pgfscope}%
\begin{pgfscope}%
\pgfsetrectcap%
\pgfsetmiterjoin%
\pgfsetlinewidth{0.000000pt}%
\definecolor{currentstroke}{rgb}{1.000000,1.000000,1.000000}%
\pgfsetstrokecolor{currentstroke}%
\pgfsetdash{}{0pt}%
\pgfpathmoveto{\pgfqpoint{0.978830in}{4.635449in}}%
\pgfpathlineto{\pgfqpoint{7.850000in}{4.635449in}}%
\pgfusepath{}%
\end{pgfscope}%
\begin{pgfscope}%
\definecolor{textcolor}{rgb}{0.150000,0.150000,0.150000}%
\pgfsetstrokecolor{textcolor}%
\pgfsetfillcolor{textcolor}%
\pgftext[x=2.107042in, y=5.096745in, left, base]{\color{textcolor}\rmfamily\fontsize{24.000000}{28.800000}\selectfont Magnetic Field Dependent }%
\end{pgfscope}%
\begin{pgfscope}%
\definecolor{textcolor}{rgb}{0.150000,0.150000,0.150000}%
\pgfsetstrokecolor{textcolor}%
\pgfsetfillcolor{textcolor}%
\pgftext[x=1.953487in, y=4.718783in, left, base]{\color{textcolor}\rmfamily\fontsize{24.000000}{28.800000}\selectfont  Resistivity of Bi\(\displaystyle _2\)Sr\(\displaystyle _2\)Ca\(\displaystyle _2\)Cu\(\displaystyle _3\)O\(\displaystyle _{10}\)}%
\end{pgfscope}%
\begin{pgfscope}%
\pgfsetbuttcap%
\pgfsetroundjoin%
\definecolor{currentfill}{rgb}{0.298039,0.447059,0.690196}%
\pgfsetfillcolor{currentfill}%
\pgfsetlinewidth{0.301125pt}%
\definecolor{currentstroke}{rgb}{0.298039,0.447059,0.690196}%
\pgfsetstrokecolor{currentstroke}%
\pgfsetdash{}{0pt}%
\pgfsys@defobject{currentmarker}{\pgfqpoint{-0.048611in}{-0.048611in}}{\pgfqpoint{0.048611in}{0.048611in}}{%
\pgfpathmoveto{\pgfqpoint{0.000000in}{-0.048611in}}%
\pgfpathcurveto{\pgfqpoint{0.012892in}{-0.048611in}}{\pgfqpoint{0.025257in}{-0.043489in}}{\pgfqpoint{0.034373in}{-0.034373in}}%
\pgfpathcurveto{\pgfqpoint{0.043489in}{-0.025257in}}{\pgfqpoint{0.048611in}{-0.012892in}}{\pgfqpoint{0.048611in}{0.000000in}}%
\pgfpathcurveto{\pgfqpoint{0.048611in}{0.012892in}}{\pgfqpoint{0.043489in}{0.025257in}}{\pgfqpoint{0.034373in}{0.034373in}}%
\pgfpathcurveto{\pgfqpoint{0.025257in}{0.043489in}}{\pgfqpoint{0.012892in}{0.048611in}}{\pgfqpoint{0.000000in}{0.048611in}}%
\pgfpathcurveto{\pgfqpoint{-0.012892in}{0.048611in}}{\pgfqpoint{-0.025257in}{0.043489in}}{\pgfqpoint{-0.034373in}{0.034373in}}%
\pgfpathcurveto{\pgfqpoint{-0.043489in}{0.025257in}}{\pgfqpoint{-0.048611in}{0.012892in}}{\pgfqpoint{-0.048611in}{0.000000in}}%
\pgfpathcurveto{\pgfqpoint{-0.048611in}{-0.012892in}}{\pgfqpoint{-0.043489in}{-0.025257in}}{\pgfqpoint{-0.034373in}{-0.034373in}}%
\pgfpathcurveto{\pgfqpoint{-0.025257in}{-0.043489in}}{\pgfqpoint{-0.012892in}{-0.048611in}}{\pgfqpoint{0.000000in}{-0.048611in}}%
\pgfpathlineto{\pgfqpoint{0.000000in}{-0.048611in}}%
\pgfpathclose%
\pgfusepath{stroke,fill}%
}%
\begin{pgfscope}%
\pgfsys@transformshift{1.506608in}{4.247320in}%
\pgfsys@useobject{currentmarker}{}%
\end{pgfscope}%
\end{pgfscope}%
\begin{pgfscope}%
\definecolor{textcolor}{rgb}{0.150000,0.150000,0.150000}%
\pgfsetstrokecolor{textcolor}%
\pgfsetfillcolor{textcolor}%
\pgftext[x=2.006608in,y=4.174403in,left,base]{\color{textcolor}\rmfamily\fontsize{20.000000}{24.000000}\selectfont B = 0 mT}%
\end{pgfscope}%
\begin{pgfscope}%
\pgfsetbuttcap%
\pgfsetroundjoin%
\definecolor{currentfill}{rgb}{0.333333,0.658824,0.407843}%
\pgfsetfillcolor{currentfill}%
\pgfsetlinewidth{0.301125pt}%
\definecolor{currentstroke}{rgb}{0.333333,0.658824,0.407843}%
\pgfsetstrokecolor{currentstroke}%
\pgfsetdash{}{0pt}%
\pgfsys@defobject{currentmarker}{\pgfqpoint{-0.048611in}{-0.048611in}}{\pgfqpoint{0.048611in}{0.048611in}}{%
\pgfpathmoveto{\pgfqpoint{0.000000in}{-0.048611in}}%
\pgfpathcurveto{\pgfqpoint{0.012892in}{-0.048611in}}{\pgfqpoint{0.025257in}{-0.043489in}}{\pgfqpoint{0.034373in}{-0.034373in}}%
\pgfpathcurveto{\pgfqpoint{0.043489in}{-0.025257in}}{\pgfqpoint{0.048611in}{-0.012892in}}{\pgfqpoint{0.048611in}{0.000000in}}%
\pgfpathcurveto{\pgfqpoint{0.048611in}{0.012892in}}{\pgfqpoint{0.043489in}{0.025257in}}{\pgfqpoint{0.034373in}{0.034373in}}%
\pgfpathcurveto{\pgfqpoint{0.025257in}{0.043489in}}{\pgfqpoint{0.012892in}{0.048611in}}{\pgfqpoint{0.000000in}{0.048611in}}%
\pgfpathcurveto{\pgfqpoint{-0.012892in}{0.048611in}}{\pgfqpoint{-0.025257in}{0.043489in}}{\pgfqpoint{-0.034373in}{0.034373in}}%
\pgfpathcurveto{\pgfqpoint{-0.043489in}{0.025257in}}{\pgfqpoint{-0.048611in}{0.012892in}}{\pgfqpoint{-0.048611in}{0.000000in}}%
\pgfpathcurveto{\pgfqpoint{-0.048611in}{-0.012892in}}{\pgfqpoint{-0.043489in}{-0.025257in}}{\pgfqpoint{-0.034373in}{-0.034373in}}%
\pgfpathcurveto{\pgfqpoint{-0.025257in}{-0.043489in}}{\pgfqpoint{-0.012892in}{-0.048611in}}{\pgfqpoint{0.000000in}{-0.048611in}}%
\pgfpathlineto{\pgfqpoint{0.000000in}{-0.048611in}}%
\pgfpathclose%
\pgfusepath{stroke,fill}%
}%
\begin{pgfscope}%
\pgfsys@transformshift{1.506608in}{3.839605in}%
\pgfsys@useobject{currentmarker}{}%
\end{pgfscope}%
\end{pgfscope}%
\begin{pgfscope}%
\definecolor{textcolor}{rgb}{0.150000,0.150000,0.150000}%
\pgfsetstrokecolor{textcolor}%
\pgfsetfillcolor{textcolor}%
\pgftext[x=2.006608in,y=3.766689in,left,base]{\color{textcolor}\rmfamily\fontsize{20.000000}{24.000000}\selectfont B = 8 mT}%
\end{pgfscope}%
\end{pgfpicture}%
\makeatother%
\endgroup%
}
        \caption{this is a test of the captions to se if  HOLDER}
        \label{fig:bscco_mag}
    \end{figure}

    The expected shift was seen in both samples. However, when taking the measurements for BSCCO the temperature range was  started from the bottom of the unadulterated superconductor temperature range at 97.5 K. Since the magnetic field pushed lowered the transition temperature we did not see the complete transition. The data points below 95K were taken on a separate day with larger temperature steps between readings and some overlap with the previous spectrum. The overlap points were plotted as individual points rather than averaging them into the previous measurements to highlight the two sources of error this introduced. Here the onset $T_c$ is at 111 K and the end of the transition range is at 84.5 K. 
    
    The sample underwent a cool-heat-cool cycle thus slightly changing the read values. This is reflected in the vertical steps at 97.5 and 100 K which have readings on both days. The lower values are from the second day. Additionally the large step-size has reduced resolution to resolve features in the spectrum. Due to this lack of resolution it isn't possible to determine if the step feature seen in the $B=0mT$ plot is still present in the $B=8mT$.


    \begin{figure}[t]
      \resizebox{0.45\textwidth}{!}{%% Creator: Matplotlib, PGF backend
%%
%% To include the figure in your LaTeX document, write
%%   \input{<filename>.pgf}
%%
%% Make sure the required packages are loaded in your preamble
%%   \usepackage{pgf}
%%
%% Also ensure that all the required font packages are loaded; for instance,
%% the lmodern package is sometimes necessary when using math font.
%%   \usepackage{lmodern}
%%
%% Figures using additional raster images can only be included by \input if
%% they are in the same directory as the main LaTeX file. For loading figures
%% from other directories you can use the `import` package
%%   \usepackage{import}
%%
%% and then include the figures with
%%   \import{<path to file>}{<filename>.pgf}
%%
%% Matplotlib used the following preamble
%%   
%%   \usepackage{fontspec}
%%   \setmainfont{DejaVuSerif.ttf}[Path=\detokenize{/home/dayala/.local/lib/python3.10/site-packages/matplotlib/mpl-data/fonts/ttf/}]
%%   \setsansfont{DejaVuSans.ttf}[Path=\detokenize{/home/dayala/.local/lib/python3.10/site-packages/matplotlib/mpl-data/fonts/ttf/}]
%%   \setmonofont{DejaVuSansMono.ttf}[Path=\detokenize{/home/dayala/.local/lib/python3.10/site-packages/matplotlib/mpl-data/fonts/ttf/}]
%%   \makeatletter\@ifpackageloaded{underscore}{}{\usepackage[strings]{underscore}}\makeatother
%%
\begingroup%
\makeatletter%
\begin{pgfpicture}%
\pgfpathrectangle{\pgfpointorigin}{\pgfqpoint{8.000000in}{5.500000in}}%
\pgfusepath{use as bounding box, clip}%
\begin{pgfscope}%
\pgfsetbuttcap%
\pgfsetmiterjoin%
\definecolor{currentfill}{rgb}{1.000000,1.000000,1.000000}%
\pgfsetfillcolor{currentfill}%
\pgfsetlinewidth{0.000000pt}%
\definecolor{currentstroke}{rgb}{1.000000,1.000000,1.000000}%
\pgfsetstrokecolor{currentstroke}%
\pgfsetdash{}{0pt}%
\pgfpathmoveto{\pgfqpoint{0.000000in}{0.000000in}}%
\pgfpathlineto{\pgfqpoint{8.000000in}{0.000000in}}%
\pgfpathlineto{\pgfqpoint{8.000000in}{5.500000in}}%
\pgfpathlineto{\pgfqpoint{0.000000in}{5.500000in}}%
\pgfpathlineto{\pgfqpoint{0.000000in}{0.000000in}}%
\pgfpathclose%
\pgfusepath{fill}%
\end{pgfscope}%
\begin{pgfscope}%
\pgfsetbuttcap%
\pgfsetmiterjoin%
\definecolor{currentfill}{rgb}{0.917647,0.917647,0.949020}%
\pgfsetfillcolor{currentfill}%
\pgfsetlinewidth{0.000000pt}%
\definecolor{currentstroke}{rgb}{0.000000,0.000000,0.000000}%
\pgfsetstrokecolor{currentstroke}%
\pgfsetstrokeopacity{0.000000}%
\pgfsetdash{}{0pt}%
\pgfpathmoveto{\pgfqpoint{1.000000in}{0.605000in}}%
\pgfpathlineto{\pgfqpoint{7.200000in}{0.605000in}}%
\pgfpathlineto{\pgfqpoint{7.200000in}{4.840000in}}%
\pgfpathlineto{\pgfqpoint{1.000000in}{4.840000in}}%
\pgfpathlineto{\pgfqpoint{1.000000in}{0.605000in}}%
\pgfpathclose%
\pgfusepath{fill}%
\end{pgfscope}%
\begin{pgfscope}%
\pgfpathrectangle{\pgfqpoint{1.000000in}{0.605000in}}{\pgfqpoint{6.200000in}{4.235000in}}%
\pgfusepath{clip}%
\pgfsetroundcap%
\pgfsetroundjoin%
\pgfsetlinewidth{1.003750pt}%
\definecolor{currentstroke}{rgb}{1.000000,1.000000,1.000000}%
\pgfsetstrokecolor{currentstroke}%
\pgfsetdash{}{0pt}%
\pgfpathmoveto{\pgfqpoint{1.657576in}{0.605000in}}%
\pgfpathlineto{\pgfqpoint{1.657576in}{4.840000in}}%
\pgfusepath{stroke}%
\end{pgfscope}%
\begin{pgfscope}%
\definecolor{textcolor}{rgb}{0.150000,0.150000,0.150000}%
\pgfsetstrokecolor{textcolor}%
\pgfsetfillcolor{textcolor}%
\pgftext[x=1.657576in,y=0.507778in,,top]{\color{textcolor}\rmfamily\fontsize{10.000000}{12.000000}\selectfont 88}%
\end{pgfscope}%
\begin{pgfscope}%
\pgfpathrectangle{\pgfqpoint{1.000000in}{0.605000in}}{\pgfqpoint{6.200000in}{4.235000in}}%
\pgfusepath{clip}%
\pgfsetroundcap%
\pgfsetroundjoin%
\pgfsetlinewidth{1.003750pt}%
\definecolor{currentstroke}{rgb}{1.000000,1.000000,1.000000}%
\pgfsetstrokecolor{currentstroke}%
\pgfsetdash{}{0pt}%
\pgfpathmoveto{\pgfqpoint{2.409091in}{0.605000in}}%
\pgfpathlineto{\pgfqpoint{2.409091in}{4.840000in}}%
\pgfusepath{stroke}%
\end{pgfscope}%
\begin{pgfscope}%
\definecolor{textcolor}{rgb}{0.150000,0.150000,0.150000}%
\pgfsetstrokecolor{textcolor}%
\pgfsetfillcolor{textcolor}%
\pgftext[x=2.409091in,y=0.507778in,,top]{\color{textcolor}\rmfamily\fontsize{10.000000}{12.000000}\selectfont 89}%
\end{pgfscope}%
\begin{pgfscope}%
\pgfpathrectangle{\pgfqpoint{1.000000in}{0.605000in}}{\pgfqpoint{6.200000in}{4.235000in}}%
\pgfusepath{clip}%
\pgfsetroundcap%
\pgfsetroundjoin%
\pgfsetlinewidth{1.003750pt}%
\definecolor{currentstroke}{rgb}{1.000000,1.000000,1.000000}%
\pgfsetstrokecolor{currentstroke}%
\pgfsetdash{}{0pt}%
\pgfpathmoveto{\pgfqpoint{3.160606in}{0.605000in}}%
\pgfpathlineto{\pgfqpoint{3.160606in}{4.840000in}}%
\pgfusepath{stroke}%
\end{pgfscope}%
\begin{pgfscope}%
\definecolor{textcolor}{rgb}{0.150000,0.150000,0.150000}%
\pgfsetstrokecolor{textcolor}%
\pgfsetfillcolor{textcolor}%
\pgftext[x=3.160606in,y=0.507778in,,top]{\color{textcolor}\rmfamily\fontsize{10.000000}{12.000000}\selectfont 90}%
\end{pgfscope}%
\begin{pgfscope}%
\pgfpathrectangle{\pgfqpoint{1.000000in}{0.605000in}}{\pgfqpoint{6.200000in}{4.235000in}}%
\pgfusepath{clip}%
\pgfsetroundcap%
\pgfsetroundjoin%
\pgfsetlinewidth{1.003750pt}%
\definecolor{currentstroke}{rgb}{1.000000,1.000000,1.000000}%
\pgfsetstrokecolor{currentstroke}%
\pgfsetdash{}{0pt}%
\pgfpathmoveto{\pgfqpoint{3.912121in}{0.605000in}}%
\pgfpathlineto{\pgfqpoint{3.912121in}{4.840000in}}%
\pgfusepath{stroke}%
\end{pgfscope}%
\begin{pgfscope}%
\definecolor{textcolor}{rgb}{0.150000,0.150000,0.150000}%
\pgfsetstrokecolor{textcolor}%
\pgfsetfillcolor{textcolor}%
\pgftext[x=3.912121in,y=0.507778in,,top]{\color{textcolor}\rmfamily\fontsize{10.000000}{12.000000}\selectfont 91}%
\end{pgfscope}%
\begin{pgfscope}%
\pgfpathrectangle{\pgfqpoint{1.000000in}{0.605000in}}{\pgfqpoint{6.200000in}{4.235000in}}%
\pgfusepath{clip}%
\pgfsetroundcap%
\pgfsetroundjoin%
\pgfsetlinewidth{1.003750pt}%
\definecolor{currentstroke}{rgb}{1.000000,1.000000,1.000000}%
\pgfsetstrokecolor{currentstroke}%
\pgfsetdash{}{0pt}%
\pgfpathmoveto{\pgfqpoint{4.663636in}{0.605000in}}%
\pgfpathlineto{\pgfqpoint{4.663636in}{4.840000in}}%
\pgfusepath{stroke}%
\end{pgfscope}%
\begin{pgfscope}%
\definecolor{textcolor}{rgb}{0.150000,0.150000,0.150000}%
\pgfsetstrokecolor{textcolor}%
\pgfsetfillcolor{textcolor}%
\pgftext[x=4.663636in,y=0.507778in,,top]{\color{textcolor}\rmfamily\fontsize{10.000000}{12.000000}\selectfont 92}%
\end{pgfscope}%
\begin{pgfscope}%
\pgfpathrectangle{\pgfqpoint{1.000000in}{0.605000in}}{\pgfqpoint{6.200000in}{4.235000in}}%
\pgfusepath{clip}%
\pgfsetroundcap%
\pgfsetroundjoin%
\pgfsetlinewidth{1.003750pt}%
\definecolor{currentstroke}{rgb}{1.000000,1.000000,1.000000}%
\pgfsetstrokecolor{currentstroke}%
\pgfsetdash{}{0pt}%
\pgfpathmoveto{\pgfqpoint{5.415152in}{0.605000in}}%
\pgfpathlineto{\pgfqpoint{5.415152in}{4.840000in}}%
\pgfusepath{stroke}%
\end{pgfscope}%
\begin{pgfscope}%
\definecolor{textcolor}{rgb}{0.150000,0.150000,0.150000}%
\pgfsetstrokecolor{textcolor}%
\pgfsetfillcolor{textcolor}%
\pgftext[x=5.415152in,y=0.507778in,,top]{\color{textcolor}\rmfamily\fontsize{10.000000}{12.000000}\selectfont 93}%
\end{pgfscope}%
\begin{pgfscope}%
\pgfpathrectangle{\pgfqpoint{1.000000in}{0.605000in}}{\pgfqpoint{6.200000in}{4.235000in}}%
\pgfusepath{clip}%
\pgfsetroundcap%
\pgfsetroundjoin%
\pgfsetlinewidth{1.003750pt}%
\definecolor{currentstroke}{rgb}{1.000000,1.000000,1.000000}%
\pgfsetstrokecolor{currentstroke}%
\pgfsetdash{}{0pt}%
\pgfpathmoveto{\pgfqpoint{6.166667in}{0.605000in}}%
\pgfpathlineto{\pgfqpoint{6.166667in}{4.840000in}}%
\pgfusepath{stroke}%
\end{pgfscope}%
\begin{pgfscope}%
\definecolor{textcolor}{rgb}{0.150000,0.150000,0.150000}%
\pgfsetstrokecolor{textcolor}%
\pgfsetfillcolor{textcolor}%
\pgftext[x=6.166667in,y=0.507778in,,top]{\color{textcolor}\rmfamily\fontsize{10.000000}{12.000000}\selectfont 94}%
\end{pgfscope}%
\begin{pgfscope}%
\pgfpathrectangle{\pgfqpoint{1.000000in}{0.605000in}}{\pgfqpoint{6.200000in}{4.235000in}}%
\pgfusepath{clip}%
\pgfsetroundcap%
\pgfsetroundjoin%
\pgfsetlinewidth{1.003750pt}%
\definecolor{currentstroke}{rgb}{1.000000,1.000000,1.000000}%
\pgfsetstrokecolor{currentstroke}%
\pgfsetdash{}{0pt}%
\pgfpathmoveto{\pgfqpoint{6.918182in}{0.605000in}}%
\pgfpathlineto{\pgfqpoint{6.918182in}{4.840000in}}%
\pgfusepath{stroke}%
\end{pgfscope}%
\begin{pgfscope}%
\definecolor{textcolor}{rgb}{0.150000,0.150000,0.150000}%
\pgfsetstrokecolor{textcolor}%
\pgfsetfillcolor{textcolor}%
\pgftext[x=6.918182in,y=0.507778in,,top]{\color{textcolor}\rmfamily\fontsize{10.000000}{12.000000}\selectfont 95}%
\end{pgfscope}%
\begin{pgfscope}%
\definecolor{textcolor}{rgb}{0.150000,0.150000,0.150000}%
\pgfsetstrokecolor{textcolor}%
\pgfsetfillcolor{textcolor}%
\pgftext[x=4.100000in,y=0.317809in,,top]{\color{textcolor}\rmfamily\fontsize{11.000000}{13.200000}\selectfont Temperature (K)}%
\end{pgfscope}%
\begin{pgfscope}%
\pgfpathrectangle{\pgfqpoint{1.000000in}{0.605000in}}{\pgfqpoint{6.200000in}{4.235000in}}%
\pgfusepath{clip}%
\pgfsetroundcap%
\pgfsetroundjoin%
\pgfsetlinewidth{1.003750pt}%
\definecolor{currentstroke}{rgb}{1.000000,1.000000,1.000000}%
\pgfsetstrokecolor{currentstroke}%
\pgfsetdash{}{0pt}%
\pgfpathmoveto{\pgfqpoint{1.000000in}{0.736389in}}%
\pgfpathlineto{\pgfqpoint{7.200000in}{0.736389in}}%
\pgfusepath{stroke}%
\end{pgfscope}%
\begin{pgfscope}%
\definecolor{textcolor}{rgb}{0.150000,0.150000,0.150000}%
\pgfsetstrokecolor{textcolor}%
\pgfsetfillcolor{textcolor}%
\pgftext[x=0.814412in, y=0.683627in, left, base]{\color{textcolor}\rmfamily\fontsize{10.000000}{12.000000}\selectfont 0}%
\end{pgfscope}%
\begin{pgfscope}%
\pgfpathrectangle{\pgfqpoint{1.000000in}{0.605000in}}{\pgfqpoint{6.200000in}{4.235000in}}%
\pgfusepath{clip}%
\pgfsetroundcap%
\pgfsetroundjoin%
\pgfsetlinewidth{1.003750pt}%
\definecolor{currentstroke}{rgb}{1.000000,1.000000,1.000000}%
\pgfsetstrokecolor{currentstroke}%
\pgfsetdash{}{0pt}%
\pgfpathmoveto{\pgfqpoint{1.000000in}{1.217935in}}%
\pgfpathlineto{\pgfqpoint{7.200000in}{1.217935in}}%
\pgfusepath{stroke}%
\end{pgfscope}%
\begin{pgfscope}%
\definecolor{textcolor}{rgb}{0.150000,0.150000,0.150000}%
\pgfsetstrokecolor{textcolor}%
\pgfsetfillcolor{textcolor}%
\pgftext[x=0.814412in, y=1.165173in, left, base]{\color{textcolor}\rmfamily\fontsize{10.000000}{12.000000}\selectfont 1}%
\end{pgfscope}%
\begin{pgfscope}%
\pgfpathrectangle{\pgfqpoint{1.000000in}{0.605000in}}{\pgfqpoint{6.200000in}{4.235000in}}%
\pgfusepath{clip}%
\pgfsetroundcap%
\pgfsetroundjoin%
\pgfsetlinewidth{1.003750pt}%
\definecolor{currentstroke}{rgb}{1.000000,1.000000,1.000000}%
\pgfsetstrokecolor{currentstroke}%
\pgfsetdash{}{0pt}%
\pgfpathmoveto{\pgfqpoint{1.000000in}{1.699481in}}%
\pgfpathlineto{\pgfqpoint{7.200000in}{1.699481in}}%
\pgfusepath{stroke}%
\end{pgfscope}%
\begin{pgfscope}%
\definecolor{textcolor}{rgb}{0.150000,0.150000,0.150000}%
\pgfsetstrokecolor{textcolor}%
\pgfsetfillcolor{textcolor}%
\pgftext[x=0.814412in, y=1.646719in, left, base]{\color{textcolor}\rmfamily\fontsize{10.000000}{12.000000}\selectfont 2}%
\end{pgfscope}%
\begin{pgfscope}%
\pgfpathrectangle{\pgfqpoint{1.000000in}{0.605000in}}{\pgfqpoint{6.200000in}{4.235000in}}%
\pgfusepath{clip}%
\pgfsetroundcap%
\pgfsetroundjoin%
\pgfsetlinewidth{1.003750pt}%
\definecolor{currentstroke}{rgb}{1.000000,1.000000,1.000000}%
\pgfsetstrokecolor{currentstroke}%
\pgfsetdash{}{0pt}%
\pgfpathmoveto{\pgfqpoint{1.000000in}{2.181027in}}%
\pgfpathlineto{\pgfqpoint{7.200000in}{2.181027in}}%
\pgfusepath{stroke}%
\end{pgfscope}%
\begin{pgfscope}%
\definecolor{textcolor}{rgb}{0.150000,0.150000,0.150000}%
\pgfsetstrokecolor{textcolor}%
\pgfsetfillcolor{textcolor}%
\pgftext[x=0.814412in, y=2.128265in, left, base]{\color{textcolor}\rmfamily\fontsize{10.000000}{12.000000}\selectfont 3}%
\end{pgfscope}%
\begin{pgfscope}%
\pgfpathrectangle{\pgfqpoint{1.000000in}{0.605000in}}{\pgfqpoint{6.200000in}{4.235000in}}%
\pgfusepath{clip}%
\pgfsetroundcap%
\pgfsetroundjoin%
\pgfsetlinewidth{1.003750pt}%
\definecolor{currentstroke}{rgb}{1.000000,1.000000,1.000000}%
\pgfsetstrokecolor{currentstroke}%
\pgfsetdash{}{0pt}%
\pgfpathmoveto{\pgfqpoint{1.000000in}{2.662573in}}%
\pgfpathlineto{\pgfqpoint{7.200000in}{2.662573in}}%
\pgfusepath{stroke}%
\end{pgfscope}%
\begin{pgfscope}%
\definecolor{textcolor}{rgb}{0.150000,0.150000,0.150000}%
\pgfsetstrokecolor{textcolor}%
\pgfsetfillcolor{textcolor}%
\pgftext[x=0.814412in, y=2.609811in, left, base]{\color{textcolor}\rmfamily\fontsize{10.000000}{12.000000}\selectfont 4}%
\end{pgfscope}%
\begin{pgfscope}%
\pgfpathrectangle{\pgfqpoint{1.000000in}{0.605000in}}{\pgfqpoint{6.200000in}{4.235000in}}%
\pgfusepath{clip}%
\pgfsetroundcap%
\pgfsetroundjoin%
\pgfsetlinewidth{1.003750pt}%
\definecolor{currentstroke}{rgb}{1.000000,1.000000,1.000000}%
\pgfsetstrokecolor{currentstroke}%
\pgfsetdash{}{0pt}%
\pgfpathmoveto{\pgfqpoint{1.000000in}{3.144119in}}%
\pgfpathlineto{\pgfqpoint{7.200000in}{3.144119in}}%
\pgfusepath{stroke}%
\end{pgfscope}%
\begin{pgfscope}%
\definecolor{textcolor}{rgb}{0.150000,0.150000,0.150000}%
\pgfsetstrokecolor{textcolor}%
\pgfsetfillcolor{textcolor}%
\pgftext[x=0.814412in, y=3.091357in, left, base]{\color{textcolor}\rmfamily\fontsize{10.000000}{12.000000}\selectfont 5}%
\end{pgfscope}%
\begin{pgfscope}%
\pgfpathrectangle{\pgfqpoint{1.000000in}{0.605000in}}{\pgfqpoint{6.200000in}{4.235000in}}%
\pgfusepath{clip}%
\pgfsetroundcap%
\pgfsetroundjoin%
\pgfsetlinewidth{1.003750pt}%
\definecolor{currentstroke}{rgb}{1.000000,1.000000,1.000000}%
\pgfsetstrokecolor{currentstroke}%
\pgfsetdash{}{0pt}%
\pgfpathmoveto{\pgfqpoint{1.000000in}{3.625665in}}%
\pgfpathlineto{\pgfqpoint{7.200000in}{3.625665in}}%
\pgfusepath{stroke}%
\end{pgfscope}%
\begin{pgfscope}%
\definecolor{textcolor}{rgb}{0.150000,0.150000,0.150000}%
\pgfsetstrokecolor{textcolor}%
\pgfsetfillcolor{textcolor}%
\pgftext[x=0.814412in, y=3.572903in, left, base]{\color{textcolor}\rmfamily\fontsize{10.000000}{12.000000}\selectfont 6}%
\end{pgfscope}%
\begin{pgfscope}%
\pgfpathrectangle{\pgfqpoint{1.000000in}{0.605000in}}{\pgfqpoint{6.200000in}{4.235000in}}%
\pgfusepath{clip}%
\pgfsetroundcap%
\pgfsetroundjoin%
\pgfsetlinewidth{1.003750pt}%
\definecolor{currentstroke}{rgb}{1.000000,1.000000,1.000000}%
\pgfsetstrokecolor{currentstroke}%
\pgfsetdash{}{0pt}%
\pgfpathmoveto{\pgfqpoint{1.000000in}{4.107211in}}%
\pgfpathlineto{\pgfqpoint{7.200000in}{4.107211in}}%
\pgfusepath{stroke}%
\end{pgfscope}%
\begin{pgfscope}%
\definecolor{textcolor}{rgb}{0.150000,0.150000,0.150000}%
\pgfsetstrokecolor{textcolor}%
\pgfsetfillcolor{textcolor}%
\pgftext[x=0.814412in, y=4.054449in, left, base]{\color{textcolor}\rmfamily\fontsize{10.000000}{12.000000}\selectfont 7}%
\end{pgfscope}%
\begin{pgfscope}%
\pgfpathrectangle{\pgfqpoint{1.000000in}{0.605000in}}{\pgfqpoint{6.200000in}{4.235000in}}%
\pgfusepath{clip}%
\pgfsetroundcap%
\pgfsetroundjoin%
\pgfsetlinewidth{1.003750pt}%
\definecolor{currentstroke}{rgb}{1.000000,1.000000,1.000000}%
\pgfsetstrokecolor{currentstroke}%
\pgfsetdash{}{0pt}%
\pgfpathmoveto{\pgfqpoint{1.000000in}{4.588757in}}%
\pgfpathlineto{\pgfqpoint{7.200000in}{4.588757in}}%
\pgfusepath{stroke}%
\end{pgfscope}%
\begin{pgfscope}%
\definecolor{textcolor}{rgb}{0.150000,0.150000,0.150000}%
\pgfsetstrokecolor{textcolor}%
\pgfsetfillcolor{textcolor}%
\pgftext[x=0.814412in, y=4.535995in, left, base]{\color{textcolor}\rmfamily\fontsize{10.000000}{12.000000}\selectfont 8}%
\end{pgfscope}%
\begin{pgfscope}%
\definecolor{textcolor}{rgb}{0.150000,0.150000,0.150000}%
\pgfsetstrokecolor{textcolor}%
\pgfsetfillcolor{textcolor}%
\pgftext[x=0.758857in,y=2.722500in,,bottom,rotate=90.000000]{\color{textcolor}\rmfamily\fontsize{11.000000}{13.200000}\selectfont \(\displaystyle \rho\) (\(\displaystyle \Omega\cdot\) m)}%
\end{pgfscope}%
\begin{pgfscope}%
\pgfpathrectangle{\pgfqpoint{1.000000in}{0.605000in}}{\pgfqpoint{6.200000in}{4.235000in}}%
\pgfusepath{clip}%
\pgfsetbuttcap%
\pgfsetroundjoin%
\definecolor{currentfill}{rgb}{0.298039,0.447059,0.690196}%
\pgfsetfillcolor{currentfill}%
\pgfsetlinewidth{0.301125pt}%
\definecolor{currentstroke}{rgb}{0.298039,0.447059,0.690196}%
\pgfsetstrokecolor{currentstroke}%
\pgfsetdash{}{0pt}%
\pgfsys@defobject{currentmarker}{\pgfqpoint{-0.048611in}{-0.048611in}}{\pgfqpoint{0.048611in}{0.048611in}}{%
\pgfpathmoveto{\pgfqpoint{0.000000in}{-0.048611in}}%
\pgfpathcurveto{\pgfqpoint{0.012892in}{-0.048611in}}{\pgfqpoint{0.025257in}{-0.043489in}}{\pgfqpoint{0.034373in}{-0.034373in}}%
\pgfpathcurveto{\pgfqpoint{0.043489in}{-0.025257in}}{\pgfqpoint{0.048611in}{-0.012892in}}{\pgfqpoint{0.048611in}{0.000000in}}%
\pgfpathcurveto{\pgfqpoint{0.048611in}{0.012892in}}{\pgfqpoint{0.043489in}{0.025257in}}{\pgfqpoint{0.034373in}{0.034373in}}%
\pgfpathcurveto{\pgfqpoint{0.025257in}{0.043489in}}{\pgfqpoint{0.012892in}{0.048611in}}{\pgfqpoint{0.000000in}{0.048611in}}%
\pgfpathcurveto{\pgfqpoint{-0.012892in}{0.048611in}}{\pgfqpoint{-0.025257in}{0.043489in}}{\pgfqpoint{-0.034373in}{0.034373in}}%
\pgfpathcurveto{\pgfqpoint{-0.043489in}{0.025257in}}{\pgfqpoint{-0.048611in}{0.012892in}}{\pgfqpoint{-0.048611in}{0.000000in}}%
\pgfpathcurveto{\pgfqpoint{-0.048611in}{-0.012892in}}{\pgfqpoint{-0.043489in}{-0.025257in}}{\pgfqpoint{-0.034373in}{-0.034373in}}%
\pgfpathcurveto{\pgfqpoint{-0.025257in}{-0.043489in}}{\pgfqpoint{-0.012892in}{-0.048611in}}{\pgfqpoint{0.000000in}{-0.048611in}}%
\pgfpathlineto{\pgfqpoint{0.000000in}{-0.048611in}}%
\pgfpathclose%
\pgfusepath{stroke,fill}%
}%
\begin{pgfscope}%
\pgfsys@transformshift{1.281818in}{0.841151in}%
\pgfsys@useobject{currentmarker}{}%
\end{pgfscope}%
\begin{pgfscope}%
\pgfsys@transformshift{1.657576in}{0.867341in}%
\pgfsys@useobject{currentmarker}{}%
\end{pgfscope}%
\begin{pgfscope}%
\pgfsys@transformshift{2.033333in}{0.884802in}%
\pgfsys@useobject{currentmarker}{}%
\end{pgfscope}%
\begin{pgfscope}%
\pgfsys@transformshift{2.409091in}{0.954643in}%
\pgfsys@useobject{currentmarker}{}%
\end{pgfscope}%
\begin{pgfscope}%
\pgfsys@transformshift{2.784848in}{1.155437in}%
\pgfsys@useobject{currentmarker}{}%
\end{pgfscope}%
\begin{pgfscope}%
\pgfsys@transformshift{3.160606in}{1.565754in}%
\pgfsys@useobject{currentmarker}{}%
\end{pgfscope}%
\begin{pgfscope}%
\pgfsys@transformshift{3.536364in}{2.368929in}%
\pgfsys@useobject{currentmarker}{}%
\end{pgfscope}%
\begin{pgfscope}%
\pgfsys@transformshift{3.912121in}{3.311786in}%
\pgfsys@useobject{currentmarker}{}%
\end{pgfscope}%
\begin{pgfscope}%
\pgfsys@transformshift{4.287879in}{3.992738in}%
\pgfsys@useobject{currentmarker}{}%
\end{pgfscope}%
\begin{pgfscope}%
\pgfsys@transformshift{4.663636in}{4.228452in}%
\pgfsys@useobject{currentmarker}{}%
\end{pgfscope}%
\begin{pgfscope}%
\pgfsys@transformshift{5.039394in}{4.385595in}%
\pgfsys@useobject{currentmarker}{}%
\end{pgfscope}%
\begin{pgfscope}%
\pgfsys@transformshift{5.415152in}{4.411786in}%
\pgfsys@useobject{currentmarker}{}%
\end{pgfscope}%
\begin{pgfscope}%
\pgfsys@transformshift{6.166667in}{4.481627in}%
\pgfsys@useobject{currentmarker}{}%
\end{pgfscope}%
\begin{pgfscope}%
\pgfsys@transformshift{6.918182in}{4.603849in}%
\pgfsys@useobject{currentmarker}{}%
\end{pgfscope}%
\end{pgfscope}%
\begin{pgfscope}%
\pgfpathrectangle{\pgfqpoint{1.000000in}{0.605000in}}{\pgfqpoint{6.200000in}{4.235000in}}%
\pgfusepath{clip}%
\pgfsetbuttcap%
\pgfsetroundjoin%
\definecolor{currentfill}{rgb}{0.333333,0.658824,0.407843}%
\pgfsetfillcolor{currentfill}%
\pgfsetlinewidth{0.301125pt}%
\definecolor{currentstroke}{rgb}{0.333333,0.658824,0.407843}%
\pgfsetstrokecolor{currentstroke}%
\pgfsetdash{}{0pt}%
\pgfsys@defobject{currentmarker}{\pgfqpoint{-0.048611in}{-0.048611in}}{\pgfqpoint{0.048611in}{0.048611in}}{%
\pgfpathmoveto{\pgfqpoint{0.000000in}{-0.048611in}}%
\pgfpathcurveto{\pgfqpoint{0.012892in}{-0.048611in}}{\pgfqpoint{0.025257in}{-0.043489in}}{\pgfqpoint{0.034373in}{-0.034373in}}%
\pgfpathcurveto{\pgfqpoint{0.043489in}{-0.025257in}}{\pgfqpoint{0.048611in}{-0.012892in}}{\pgfqpoint{0.048611in}{0.000000in}}%
\pgfpathcurveto{\pgfqpoint{0.048611in}{0.012892in}}{\pgfqpoint{0.043489in}{0.025257in}}{\pgfqpoint{0.034373in}{0.034373in}}%
\pgfpathcurveto{\pgfqpoint{0.025257in}{0.043489in}}{\pgfqpoint{0.012892in}{0.048611in}}{\pgfqpoint{0.000000in}{0.048611in}}%
\pgfpathcurveto{\pgfqpoint{-0.012892in}{0.048611in}}{\pgfqpoint{-0.025257in}{0.043489in}}{\pgfqpoint{-0.034373in}{0.034373in}}%
\pgfpathcurveto{\pgfqpoint{-0.043489in}{0.025257in}}{\pgfqpoint{-0.048611in}{0.012892in}}{\pgfqpoint{-0.048611in}{0.000000in}}%
\pgfpathcurveto{\pgfqpoint{-0.048611in}{-0.012892in}}{\pgfqpoint{-0.043489in}{-0.025257in}}{\pgfqpoint{-0.034373in}{-0.034373in}}%
\pgfpathcurveto{\pgfqpoint{-0.025257in}{-0.043489in}}{\pgfqpoint{-0.012892in}{-0.048611in}}{\pgfqpoint{0.000000in}{-0.048611in}}%
\pgfpathlineto{\pgfqpoint{0.000000in}{-0.048611in}}%
\pgfpathclose%
\pgfusepath{stroke,fill}%
}%
\begin{pgfscope}%
\pgfsys@transformshift{1.281818in}{0.867341in}%
\pgfsys@useobject{currentmarker}{}%
\end{pgfscope}%
\begin{pgfscope}%
\pgfsys@transformshift{1.657576in}{0.954643in}%
\pgfsys@useobject{currentmarker}{}%
\end{pgfscope}%
\begin{pgfscope}%
\pgfsys@transformshift{2.033333in}{1.059405in}%
\pgfsys@useobject{currentmarker}{}%
\end{pgfscope}%
\begin{pgfscope}%
\pgfsys@transformshift{2.409091in}{1.207817in}%
\pgfsys@useobject{currentmarker}{}%
\end{pgfscope}%
\begin{pgfscope}%
\pgfsys@transformshift{2.784848in}{1.679246in}%
\pgfsys@useobject{currentmarker}{}%
\end{pgfscope}%
\begin{pgfscope}%
\pgfsys@transformshift{3.160606in}{2.368929in}%
\pgfsys@useobject{currentmarker}{}%
\end{pgfscope}%
\begin{pgfscope}%
\pgfsys@transformshift{3.536364in}{3.198294in}%
\pgfsys@useobject{currentmarker}{}%
\end{pgfscope}%
\begin{pgfscope}%
\pgfsys@transformshift{3.912121in}{3.757024in}%
\pgfsys@useobject{currentmarker}{}%
\end{pgfscope}%
\begin{pgfscope}%
\pgfsys@transformshift{4.287879in}{4.123690in}%
\pgfsys@useobject{currentmarker}{}%
\end{pgfscope}%
\begin{pgfscope}%
\pgfsys@transformshift{4.663636in}{4.298294in}%
\pgfsys@useobject{currentmarker}{}%
\end{pgfscope}%
\begin{pgfscope}%
\pgfsys@transformshift{5.039394in}{4.333214in}%
\pgfsys@useobject{currentmarker}{}%
\end{pgfscope}%
\begin{pgfscope}%
\pgfsys@transformshift{5.415152in}{4.420516in}%
\pgfsys@useobject{currentmarker}{}%
\end{pgfscope}%
\begin{pgfscope}%
\pgfsys@transformshift{6.166667in}{4.490357in}%
\pgfsys@useobject{currentmarker}{}%
\end{pgfscope}%
\begin{pgfscope}%
\pgfsys@transformshift{6.918182in}{4.577659in}%
\pgfsys@useobject{currentmarker}{}%
\end{pgfscope}%
\end{pgfscope}%
\begin{pgfscope}%
\pgfpathrectangle{\pgfqpoint{1.000000in}{0.605000in}}{\pgfqpoint{6.200000in}{4.235000in}}%
\pgfusepath{clip}%
\pgfsetbuttcap%
\pgfsetroundjoin%
\pgfsetlinewidth{1.756562pt}%
\definecolor{currentstroke}{rgb}{0.298039,0.447059,0.690196}%
\pgfsetstrokecolor{currentstroke}%
\pgfsetdash{}{0pt}%
\pgfpathmoveto{\pgfqpoint{1.281818in}{0.797500in}}%
\pgfpathlineto{\pgfqpoint{1.281818in}{0.884802in}}%
\pgfusepath{stroke}%
\end{pgfscope}%
\begin{pgfscope}%
\pgfpathrectangle{\pgfqpoint{1.000000in}{0.605000in}}{\pgfqpoint{6.200000in}{4.235000in}}%
\pgfusepath{clip}%
\pgfsetbuttcap%
\pgfsetroundjoin%
\pgfsetlinewidth{1.756562pt}%
\definecolor{currentstroke}{rgb}{0.298039,0.447059,0.690196}%
\pgfsetstrokecolor{currentstroke}%
\pgfsetdash{}{0pt}%
\pgfpathmoveto{\pgfqpoint{1.657576in}{0.823690in}}%
\pgfpathlineto{\pgfqpoint{1.657576in}{0.910992in}}%
\pgfusepath{stroke}%
\end{pgfscope}%
\begin{pgfscope}%
\pgfpathrectangle{\pgfqpoint{1.000000in}{0.605000in}}{\pgfqpoint{6.200000in}{4.235000in}}%
\pgfusepath{clip}%
\pgfsetbuttcap%
\pgfsetroundjoin%
\pgfsetlinewidth{1.756562pt}%
\definecolor{currentstroke}{rgb}{0.298039,0.447059,0.690196}%
\pgfsetstrokecolor{currentstroke}%
\pgfsetdash{}{0pt}%
\pgfpathmoveto{\pgfqpoint{2.033333in}{0.841151in}}%
\pgfpathlineto{\pgfqpoint{2.033333in}{0.928452in}}%
\pgfusepath{stroke}%
\end{pgfscope}%
\begin{pgfscope}%
\pgfpathrectangle{\pgfqpoint{1.000000in}{0.605000in}}{\pgfqpoint{6.200000in}{4.235000in}}%
\pgfusepath{clip}%
\pgfsetbuttcap%
\pgfsetroundjoin%
\pgfsetlinewidth{1.756562pt}%
\definecolor{currentstroke}{rgb}{0.298039,0.447059,0.690196}%
\pgfsetstrokecolor{currentstroke}%
\pgfsetdash{}{0pt}%
\pgfpathmoveto{\pgfqpoint{2.409091in}{0.910992in}}%
\pgfpathlineto{\pgfqpoint{2.409091in}{0.998294in}}%
\pgfusepath{stroke}%
\end{pgfscope}%
\begin{pgfscope}%
\pgfpathrectangle{\pgfqpoint{1.000000in}{0.605000in}}{\pgfqpoint{6.200000in}{4.235000in}}%
\pgfusepath{clip}%
\pgfsetbuttcap%
\pgfsetroundjoin%
\pgfsetlinewidth{1.756562pt}%
\definecolor{currentstroke}{rgb}{0.298039,0.447059,0.690196}%
\pgfsetstrokecolor{currentstroke}%
\pgfsetdash{}{0pt}%
\pgfpathmoveto{\pgfqpoint{2.784848in}{1.111786in}}%
\pgfpathlineto{\pgfqpoint{2.784848in}{1.199087in}}%
\pgfusepath{stroke}%
\end{pgfscope}%
\begin{pgfscope}%
\pgfpathrectangle{\pgfqpoint{1.000000in}{0.605000in}}{\pgfqpoint{6.200000in}{4.235000in}}%
\pgfusepath{clip}%
\pgfsetbuttcap%
\pgfsetroundjoin%
\pgfsetlinewidth{1.756562pt}%
\definecolor{currentstroke}{rgb}{0.298039,0.447059,0.690196}%
\pgfsetstrokecolor{currentstroke}%
\pgfsetdash{}{0pt}%
\pgfpathmoveto{\pgfqpoint{3.160606in}{1.522103in}}%
\pgfpathlineto{\pgfqpoint{3.160606in}{1.609405in}}%
\pgfusepath{stroke}%
\end{pgfscope}%
\begin{pgfscope}%
\pgfpathrectangle{\pgfqpoint{1.000000in}{0.605000in}}{\pgfqpoint{6.200000in}{4.235000in}}%
\pgfusepath{clip}%
\pgfsetbuttcap%
\pgfsetroundjoin%
\pgfsetlinewidth{1.756562pt}%
\definecolor{currentstroke}{rgb}{0.298039,0.447059,0.690196}%
\pgfsetstrokecolor{currentstroke}%
\pgfsetdash{}{0pt}%
\pgfpathmoveto{\pgfqpoint{3.536364in}{2.325278in}}%
\pgfpathlineto{\pgfqpoint{3.536364in}{2.412579in}}%
\pgfusepath{stroke}%
\end{pgfscope}%
\begin{pgfscope}%
\pgfpathrectangle{\pgfqpoint{1.000000in}{0.605000in}}{\pgfqpoint{6.200000in}{4.235000in}}%
\pgfusepath{clip}%
\pgfsetbuttcap%
\pgfsetroundjoin%
\pgfsetlinewidth{1.756562pt}%
\definecolor{currentstroke}{rgb}{0.298039,0.447059,0.690196}%
\pgfsetstrokecolor{currentstroke}%
\pgfsetdash{}{0pt}%
\pgfpathmoveto{\pgfqpoint{3.912121in}{3.268135in}}%
\pgfpathlineto{\pgfqpoint{3.912121in}{3.355437in}}%
\pgfusepath{stroke}%
\end{pgfscope}%
\begin{pgfscope}%
\pgfpathrectangle{\pgfqpoint{1.000000in}{0.605000in}}{\pgfqpoint{6.200000in}{4.235000in}}%
\pgfusepath{clip}%
\pgfsetbuttcap%
\pgfsetroundjoin%
\pgfsetlinewidth{1.756562pt}%
\definecolor{currentstroke}{rgb}{0.298039,0.447059,0.690196}%
\pgfsetstrokecolor{currentstroke}%
\pgfsetdash{}{0pt}%
\pgfpathmoveto{\pgfqpoint{4.287879in}{3.949087in}}%
\pgfpathlineto{\pgfqpoint{4.287879in}{4.036389in}}%
\pgfusepath{stroke}%
\end{pgfscope}%
\begin{pgfscope}%
\pgfpathrectangle{\pgfqpoint{1.000000in}{0.605000in}}{\pgfqpoint{6.200000in}{4.235000in}}%
\pgfusepath{clip}%
\pgfsetbuttcap%
\pgfsetroundjoin%
\pgfsetlinewidth{1.756562pt}%
\definecolor{currentstroke}{rgb}{0.298039,0.447059,0.690196}%
\pgfsetstrokecolor{currentstroke}%
\pgfsetdash{}{0pt}%
\pgfpathmoveto{\pgfqpoint{4.663636in}{4.184802in}}%
\pgfpathlineto{\pgfqpoint{4.663636in}{4.272103in}}%
\pgfusepath{stroke}%
\end{pgfscope}%
\begin{pgfscope}%
\pgfpathrectangle{\pgfqpoint{1.000000in}{0.605000in}}{\pgfqpoint{6.200000in}{4.235000in}}%
\pgfusepath{clip}%
\pgfsetbuttcap%
\pgfsetroundjoin%
\pgfsetlinewidth{1.756562pt}%
\definecolor{currentstroke}{rgb}{0.298039,0.447059,0.690196}%
\pgfsetstrokecolor{currentstroke}%
\pgfsetdash{}{0pt}%
\pgfpathmoveto{\pgfqpoint{5.039394in}{4.341944in}}%
\pgfpathlineto{\pgfqpoint{5.039394in}{4.429246in}}%
\pgfusepath{stroke}%
\end{pgfscope}%
\begin{pgfscope}%
\pgfpathrectangle{\pgfqpoint{1.000000in}{0.605000in}}{\pgfqpoint{6.200000in}{4.235000in}}%
\pgfusepath{clip}%
\pgfsetbuttcap%
\pgfsetroundjoin%
\pgfsetlinewidth{1.756562pt}%
\definecolor{currentstroke}{rgb}{0.298039,0.447059,0.690196}%
\pgfsetstrokecolor{currentstroke}%
\pgfsetdash{}{0pt}%
\pgfpathmoveto{\pgfqpoint{5.415152in}{4.368135in}}%
\pgfpathlineto{\pgfqpoint{5.415152in}{4.455437in}}%
\pgfusepath{stroke}%
\end{pgfscope}%
\begin{pgfscope}%
\pgfpathrectangle{\pgfqpoint{1.000000in}{0.605000in}}{\pgfqpoint{6.200000in}{4.235000in}}%
\pgfusepath{clip}%
\pgfsetbuttcap%
\pgfsetroundjoin%
\pgfsetlinewidth{1.756562pt}%
\definecolor{currentstroke}{rgb}{0.298039,0.447059,0.690196}%
\pgfsetstrokecolor{currentstroke}%
\pgfsetdash{}{0pt}%
\pgfpathmoveto{\pgfqpoint{6.166667in}{4.437976in}}%
\pgfpathlineto{\pgfqpoint{6.166667in}{4.525278in}}%
\pgfusepath{stroke}%
\end{pgfscope}%
\begin{pgfscope}%
\pgfpathrectangle{\pgfqpoint{1.000000in}{0.605000in}}{\pgfqpoint{6.200000in}{4.235000in}}%
\pgfusepath{clip}%
\pgfsetbuttcap%
\pgfsetroundjoin%
\pgfsetlinewidth{1.756562pt}%
\definecolor{currentstroke}{rgb}{0.298039,0.447059,0.690196}%
\pgfsetstrokecolor{currentstroke}%
\pgfsetdash{}{0pt}%
\pgfpathmoveto{\pgfqpoint{6.918182in}{4.560198in}}%
\pgfpathlineto{\pgfqpoint{6.918182in}{4.647500in}}%
\pgfusepath{stroke}%
\end{pgfscope}%
\begin{pgfscope}%
\pgfpathrectangle{\pgfqpoint{1.000000in}{0.605000in}}{\pgfqpoint{6.200000in}{4.235000in}}%
\pgfusepath{clip}%
\pgfsetbuttcap%
\pgfsetroundjoin%
\definecolor{currentfill}{rgb}{0.298039,0.447059,0.690196}%
\pgfsetfillcolor{currentfill}%
\pgfsetlinewidth{1.003750pt}%
\definecolor{currentstroke}{rgb}{0.298039,0.447059,0.690196}%
\pgfsetstrokecolor{currentstroke}%
\pgfsetdash{}{0pt}%
\pgfsys@defobject{currentmarker}{\pgfqpoint{-0.069444in}{-0.000000in}}{\pgfqpoint{0.069444in}{0.000000in}}{%
\pgfpathmoveto{\pgfqpoint{0.069444in}{-0.000000in}}%
\pgfpathlineto{\pgfqpoint{-0.069444in}{0.000000in}}%
\pgfusepath{stroke,fill}%
}%
\begin{pgfscope}%
\pgfsys@transformshift{1.281818in}{0.797500in}%
\pgfsys@useobject{currentmarker}{}%
\end{pgfscope}%
\begin{pgfscope}%
\pgfsys@transformshift{1.657576in}{0.823690in}%
\pgfsys@useobject{currentmarker}{}%
\end{pgfscope}%
\begin{pgfscope}%
\pgfsys@transformshift{2.033333in}{0.841151in}%
\pgfsys@useobject{currentmarker}{}%
\end{pgfscope}%
\begin{pgfscope}%
\pgfsys@transformshift{2.409091in}{0.910992in}%
\pgfsys@useobject{currentmarker}{}%
\end{pgfscope}%
\begin{pgfscope}%
\pgfsys@transformshift{2.784848in}{1.111786in}%
\pgfsys@useobject{currentmarker}{}%
\end{pgfscope}%
\begin{pgfscope}%
\pgfsys@transformshift{3.160606in}{1.522103in}%
\pgfsys@useobject{currentmarker}{}%
\end{pgfscope}%
\begin{pgfscope}%
\pgfsys@transformshift{3.536364in}{2.325278in}%
\pgfsys@useobject{currentmarker}{}%
\end{pgfscope}%
\begin{pgfscope}%
\pgfsys@transformshift{3.912121in}{3.268135in}%
\pgfsys@useobject{currentmarker}{}%
\end{pgfscope}%
\begin{pgfscope}%
\pgfsys@transformshift{4.287879in}{3.949087in}%
\pgfsys@useobject{currentmarker}{}%
\end{pgfscope}%
\begin{pgfscope}%
\pgfsys@transformshift{4.663636in}{4.184802in}%
\pgfsys@useobject{currentmarker}{}%
\end{pgfscope}%
\begin{pgfscope}%
\pgfsys@transformshift{5.039394in}{4.341944in}%
\pgfsys@useobject{currentmarker}{}%
\end{pgfscope}%
\begin{pgfscope}%
\pgfsys@transformshift{5.415152in}{4.368135in}%
\pgfsys@useobject{currentmarker}{}%
\end{pgfscope}%
\begin{pgfscope}%
\pgfsys@transformshift{6.166667in}{4.437976in}%
\pgfsys@useobject{currentmarker}{}%
\end{pgfscope}%
\begin{pgfscope}%
\pgfsys@transformshift{6.918182in}{4.560198in}%
\pgfsys@useobject{currentmarker}{}%
\end{pgfscope}%
\end{pgfscope}%
\begin{pgfscope}%
\pgfpathrectangle{\pgfqpoint{1.000000in}{0.605000in}}{\pgfqpoint{6.200000in}{4.235000in}}%
\pgfusepath{clip}%
\pgfsetbuttcap%
\pgfsetroundjoin%
\definecolor{currentfill}{rgb}{0.298039,0.447059,0.690196}%
\pgfsetfillcolor{currentfill}%
\pgfsetlinewidth{1.003750pt}%
\definecolor{currentstroke}{rgb}{0.298039,0.447059,0.690196}%
\pgfsetstrokecolor{currentstroke}%
\pgfsetdash{}{0pt}%
\pgfsys@defobject{currentmarker}{\pgfqpoint{-0.069444in}{-0.000000in}}{\pgfqpoint{0.069444in}{0.000000in}}{%
\pgfpathmoveto{\pgfqpoint{0.069444in}{-0.000000in}}%
\pgfpathlineto{\pgfqpoint{-0.069444in}{0.000000in}}%
\pgfusepath{stroke,fill}%
}%
\begin{pgfscope}%
\pgfsys@transformshift{1.281818in}{0.884802in}%
\pgfsys@useobject{currentmarker}{}%
\end{pgfscope}%
\begin{pgfscope}%
\pgfsys@transformshift{1.657576in}{0.910992in}%
\pgfsys@useobject{currentmarker}{}%
\end{pgfscope}%
\begin{pgfscope}%
\pgfsys@transformshift{2.033333in}{0.928452in}%
\pgfsys@useobject{currentmarker}{}%
\end{pgfscope}%
\begin{pgfscope}%
\pgfsys@transformshift{2.409091in}{0.998294in}%
\pgfsys@useobject{currentmarker}{}%
\end{pgfscope}%
\begin{pgfscope}%
\pgfsys@transformshift{2.784848in}{1.199087in}%
\pgfsys@useobject{currentmarker}{}%
\end{pgfscope}%
\begin{pgfscope}%
\pgfsys@transformshift{3.160606in}{1.609405in}%
\pgfsys@useobject{currentmarker}{}%
\end{pgfscope}%
\begin{pgfscope}%
\pgfsys@transformshift{3.536364in}{2.412579in}%
\pgfsys@useobject{currentmarker}{}%
\end{pgfscope}%
\begin{pgfscope}%
\pgfsys@transformshift{3.912121in}{3.355437in}%
\pgfsys@useobject{currentmarker}{}%
\end{pgfscope}%
\begin{pgfscope}%
\pgfsys@transformshift{4.287879in}{4.036389in}%
\pgfsys@useobject{currentmarker}{}%
\end{pgfscope}%
\begin{pgfscope}%
\pgfsys@transformshift{4.663636in}{4.272103in}%
\pgfsys@useobject{currentmarker}{}%
\end{pgfscope}%
\begin{pgfscope}%
\pgfsys@transformshift{5.039394in}{4.429246in}%
\pgfsys@useobject{currentmarker}{}%
\end{pgfscope}%
\begin{pgfscope}%
\pgfsys@transformshift{5.415152in}{4.455437in}%
\pgfsys@useobject{currentmarker}{}%
\end{pgfscope}%
\begin{pgfscope}%
\pgfsys@transformshift{6.166667in}{4.525278in}%
\pgfsys@useobject{currentmarker}{}%
\end{pgfscope}%
\begin{pgfscope}%
\pgfsys@transformshift{6.918182in}{4.647500in}%
\pgfsys@useobject{currentmarker}{}%
\end{pgfscope}%
\end{pgfscope}%
\begin{pgfscope}%
\pgfpathrectangle{\pgfqpoint{1.000000in}{0.605000in}}{\pgfqpoint{6.200000in}{4.235000in}}%
\pgfusepath{clip}%
\pgfsetbuttcap%
\pgfsetroundjoin%
\pgfsetlinewidth{1.756562pt}%
\definecolor{currentstroke}{rgb}{0.333333,0.658824,0.407843}%
\pgfsetstrokecolor{currentstroke}%
\pgfsetdash{}{0pt}%
\pgfpathmoveto{\pgfqpoint{1.281818in}{0.823690in}}%
\pgfpathlineto{\pgfqpoint{1.281818in}{0.910992in}}%
\pgfusepath{stroke}%
\end{pgfscope}%
\begin{pgfscope}%
\pgfpathrectangle{\pgfqpoint{1.000000in}{0.605000in}}{\pgfqpoint{6.200000in}{4.235000in}}%
\pgfusepath{clip}%
\pgfsetbuttcap%
\pgfsetroundjoin%
\pgfsetlinewidth{1.756562pt}%
\definecolor{currentstroke}{rgb}{0.333333,0.658824,0.407843}%
\pgfsetstrokecolor{currentstroke}%
\pgfsetdash{}{0pt}%
\pgfpathmoveto{\pgfqpoint{1.657576in}{0.910992in}}%
\pgfpathlineto{\pgfqpoint{1.657576in}{0.998294in}}%
\pgfusepath{stroke}%
\end{pgfscope}%
\begin{pgfscope}%
\pgfpathrectangle{\pgfqpoint{1.000000in}{0.605000in}}{\pgfqpoint{6.200000in}{4.235000in}}%
\pgfusepath{clip}%
\pgfsetbuttcap%
\pgfsetroundjoin%
\pgfsetlinewidth{1.756562pt}%
\definecolor{currentstroke}{rgb}{0.333333,0.658824,0.407843}%
\pgfsetstrokecolor{currentstroke}%
\pgfsetdash{}{0pt}%
\pgfpathmoveto{\pgfqpoint{2.033333in}{1.015754in}}%
\pgfpathlineto{\pgfqpoint{2.033333in}{1.103056in}}%
\pgfusepath{stroke}%
\end{pgfscope}%
\begin{pgfscope}%
\pgfpathrectangle{\pgfqpoint{1.000000in}{0.605000in}}{\pgfqpoint{6.200000in}{4.235000in}}%
\pgfusepath{clip}%
\pgfsetbuttcap%
\pgfsetroundjoin%
\pgfsetlinewidth{1.756562pt}%
\definecolor{currentstroke}{rgb}{0.333333,0.658824,0.407843}%
\pgfsetstrokecolor{currentstroke}%
\pgfsetdash{}{0pt}%
\pgfpathmoveto{\pgfqpoint{2.409091in}{1.164167in}}%
\pgfpathlineto{\pgfqpoint{2.409091in}{1.251468in}}%
\pgfusepath{stroke}%
\end{pgfscope}%
\begin{pgfscope}%
\pgfpathrectangle{\pgfqpoint{1.000000in}{0.605000in}}{\pgfqpoint{6.200000in}{4.235000in}}%
\pgfusepath{clip}%
\pgfsetbuttcap%
\pgfsetroundjoin%
\pgfsetlinewidth{1.756562pt}%
\definecolor{currentstroke}{rgb}{0.333333,0.658824,0.407843}%
\pgfsetstrokecolor{currentstroke}%
\pgfsetdash{}{0pt}%
\pgfpathmoveto{\pgfqpoint{2.784848in}{1.635595in}}%
\pgfpathlineto{\pgfqpoint{2.784848in}{1.722897in}}%
\pgfusepath{stroke}%
\end{pgfscope}%
\begin{pgfscope}%
\pgfpathrectangle{\pgfqpoint{1.000000in}{0.605000in}}{\pgfqpoint{6.200000in}{4.235000in}}%
\pgfusepath{clip}%
\pgfsetbuttcap%
\pgfsetroundjoin%
\pgfsetlinewidth{1.756562pt}%
\definecolor{currentstroke}{rgb}{0.333333,0.658824,0.407843}%
\pgfsetstrokecolor{currentstroke}%
\pgfsetdash{}{0pt}%
\pgfpathmoveto{\pgfqpoint{3.160606in}{2.325278in}}%
\pgfpathlineto{\pgfqpoint{3.160606in}{2.412579in}}%
\pgfusepath{stroke}%
\end{pgfscope}%
\begin{pgfscope}%
\pgfpathrectangle{\pgfqpoint{1.000000in}{0.605000in}}{\pgfqpoint{6.200000in}{4.235000in}}%
\pgfusepath{clip}%
\pgfsetbuttcap%
\pgfsetroundjoin%
\pgfsetlinewidth{1.756562pt}%
\definecolor{currentstroke}{rgb}{0.333333,0.658824,0.407843}%
\pgfsetstrokecolor{currentstroke}%
\pgfsetdash{}{0pt}%
\pgfpathmoveto{\pgfqpoint{3.536364in}{3.154643in}}%
\pgfpathlineto{\pgfqpoint{3.536364in}{3.241944in}}%
\pgfusepath{stroke}%
\end{pgfscope}%
\begin{pgfscope}%
\pgfpathrectangle{\pgfqpoint{1.000000in}{0.605000in}}{\pgfqpoint{6.200000in}{4.235000in}}%
\pgfusepath{clip}%
\pgfsetbuttcap%
\pgfsetroundjoin%
\pgfsetlinewidth{1.756562pt}%
\definecolor{currentstroke}{rgb}{0.333333,0.658824,0.407843}%
\pgfsetstrokecolor{currentstroke}%
\pgfsetdash{}{0pt}%
\pgfpathmoveto{\pgfqpoint{3.912121in}{3.713373in}}%
\pgfpathlineto{\pgfqpoint{3.912121in}{3.800675in}}%
\pgfusepath{stroke}%
\end{pgfscope}%
\begin{pgfscope}%
\pgfpathrectangle{\pgfqpoint{1.000000in}{0.605000in}}{\pgfqpoint{6.200000in}{4.235000in}}%
\pgfusepath{clip}%
\pgfsetbuttcap%
\pgfsetroundjoin%
\pgfsetlinewidth{1.756562pt}%
\definecolor{currentstroke}{rgb}{0.333333,0.658824,0.407843}%
\pgfsetstrokecolor{currentstroke}%
\pgfsetdash{}{0pt}%
\pgfpathmoveto{\pgfqpoint{4.287879in}{4.080040in}}%
\pgfpathlineto{\pgfqpoint{4.287879in}{4.167341in}}%
\pgfusepath{stroke}%
\end{pgfscope}%
\begin{pgfscope}%
\pgfpathrectangle{\pgfqpoint{1.000000in}{0.605000in}}{\pgfqpoint{6.200000in}{4.235000in}}%
\pgfusepath{clip}%
\pgfsetbuttcap%
\pgfsetroundjoin%
\pgfsetlinewidth{1.756562pt}%
\definecolor{currentstroke}{rgb}{0.333333,0.658824,0.407843}%
\pgfsetstrokecolor{currentstroke}%
\pgfsetdash{}{0pt}%
\pgfpathmoveto{\pgfqpoint{4.663636in}{4.254643in}}%
\pgfpathlineto{\pgfqpoint{4.663636in}{4.341944in}}%
\pgfusepath{stroke}%
\end{pgfscope}%
\begin{pgfscope}%
\pgfpathrectangle{\pgfqpoint{1.000000in}{0.605000in}}{\pgfqpoint{6.200000in}{4.235000in}}%
\pgfusepath{clip}%
\pgfsetbuttcap%
\pgfsetroundjoin%
\pgfsetlinewidth{1.756562pt}%
\definecolor{currentstroke}{rgb}{0.333333,0.658824,0.407843}%
\pgfsetstrokecolor{currentstroke}%
\pgfsetdash{}{0pt}%
\pgfpathmoveto{\pgfqpoint{5.039394in}{4.289563in}}%
\pgfpathlineto{\pgfqpoint{5.039394in}{4.376865in}}%
\pgfusepath{stroke}%
\end{pgfscope}%
\begin{pgfscope}%
\pgfpathrectangle{\pgfqpoint{1.000000in}{0.605000in}}{\pgfqpoint{6.200000in}{4.235000in}}%
\pgfusepath{clip}%
\pgfsetbuttcap%
\pgfsetroundjoin%
\pgfsetlinewidth{1.756562pt}%
\definecolor{currentstroke}{rgb}{0.333333,0.658824,0.407843}%
\pgfsetstrokecolor{currentstroke}%
\pgfsetdash{}{0pt}%
\pgfpathmoveto{\pgfqpoint{5.415152in}{4.376865in}}%
\pgfpathlineto{\pgfqpoint{5.415152in}{4.464167in}}%
\pgfusepath{stroke}%
\end{pgfscope}%
\begin{pgfscope}%
\pgfpathrectangle{\pgfqpoint{1.000000in}{0.605000in}}{\pgfqpoint{6.200000in}{4.235000in}}%
\pgfusepath{clip}%
\pgfsetbuttcap%
\pgfsetroundjoin%
\pgfsetlinewidth{1.756562pt}%
\definecolor{currentstroke}{rgb}{0.333333,0.658824,0.407843}%
\pgfsetstrokecolor{currentstroke}%
\pgfsetdash{}{0pt}%
\pgfpathmoveto{\pgfqpoint{6.166667in}{4.446706in}}%
\pgfpathlineto{\pgfqpoint{6.166667in}{4.534008in}}%
\pgfusepath{stroke}%
\end{pgfscope}%
\begin{pgfscope}%
\pgfpathrectangle{\pgfqpoint{1.000000in}{0.605000in}}{\pgfqpoint{6.200000in}{4.235000in}}%
\pgfusepath{clip}%
\pgfsetbuttcap%
\pgfsetroundjoin%
\pgfsetlinewidth{1.756562pt}%
\definecolor{currentstroke}{rgb}{0.333333,0.658824,0.407843}%
\pgfsetstrokecolor{currentstroke}%
\pgfsetdash{}{0pt}%
\pgfpathmoveto{\pgfqpoint{6.918182in}{4.534008in}}%
\pgfpathlineto{\pgfqpoint{6.918182in}{4.621310in}}%
\pgfusepath{stroke}%
\end{pgfscope}%
\begin{pgfscope}%
\pgfpathrectangle{\pgfqpoint{1.000000in}{0.605000in}}{\pgfqpoint{6.200000in}{4.235000in}}%
\pgfusepath{clip}%
\pgfsetbuttcap%
\pgfsetroundjoin%
\definecolor{currentfill}{rgb}{0.333333,0.658824,0.407843}%
\pgfsetfillcolor{currentfill}%
\pgfsetlinewidth{1.003750pt}%
\definecolor{currentstroke}{rgb}{0.333333,0.658824,0.407843}%
\pgfsetstrokecolor{currentstroke}%
\pgfsetdash{}{0pt}%
\pgfsys@defobject{currentmarker}{\pgfqpoint{-0.069444in}{-0.000000in}}{\pgfqpoint{0.069444in}{0.000000in}}{%
\pgfpathmoveto{\pgfqpoint{0.069444in}{-0.000000in}}%
\pgfpathlineto{\pgfqpoint{-0.069444in}{0.000000in}}%
\pgfusepath{stroke,fill}%
}%
\begin{pgfscope}%
\pgfsys@transformshift{1.281818in}{0.823690in}%
\pgfsys@useobject{currentmarker}{}%
\end{pgfscope}%
\begin{pgfscope}%
\pgfsys@transformshift{1.657576in}{0.910992in}%
\pgfsys@useobject{currentmarker}{}%
\end{pgfscope}%
\begin{pgfscope}%
\pgfsys@transformshift{2.033333in}{1.015754in}%
\pgfsys@useobject{currentmarker}{}%
\end{pgfscope}%
\begin{pgfscope}%
\pgfsys@transformshift{2.409091in}{1.164167in}%
\pgfsys@useobject{currentmarker}{}%
\end{pgfscope}%
\begin{pgfscope}%
\pgfsys@transformshift{2.784848in}{1.635595in}%
\pgfsys@useobject{currentmarker}{}%
\end{pgfscope}%
\begin{pgfscope}%
\pgfsys@transformshift{3.160606in}{2.325278in}%
\pgfsys@useobject{currentmarker}{}%
\end{pgfscope}%
\begin{pgfscope}%
\pgfsys@transformshift{3.536364in}{3.154643in}%
\pgfsys@useobject{currentmarker}{}%
\end{pgfscope}%
\begin{pgfscope}%
\pgfsys@transformshift{3.912121in}{3.713373in}%
\pgfsys@useobject{currentmarker}{}%
\end{pgfscope}%
\begin{pgfscope}%
\pgfsys@transformshift{4.287879in}{4.080040in}%
\pgfsys@useobject{currentmarker}{}%
\end{pgfscope}%
\begin{pgfscope}%
\pgfsys@transformshift{4.663636in}{4.254643in}%
\pgfsys@useobject{currentmarker}{}%
\end{pgfscope}%
\begin{pgfscope}%
\pgfsys@transformshift{5.039394in}{4.289563in}%
\pgfsys@useobject{currentmarker}{}%
\end{pgfscope}%
\begin{pgfscope}%
\pgfsys@transformshift{5.415152in}{4.376865in}%
\pgfsys@useobject{currentmarker}{}%
\end{pgfscope}%
\begin{pgfscope}%
\pgfsys@transformshift{6.166667in}{4.446706in}%
\pgfsys@useobject{currentmarker}{}%
\end{pgfscope}%
\begin{pgfscope}%
\pgfsys@transformshift{6.918182in}{4.534008in}%
\pgfsys@useobject{currentmarker}{}%
\end{pgfscope}%
\end{pgfscope}%
\begin{pgfscope}%
\pgfpathrectangle{\pgfqpoint{1.000000in}{0.605000in}}{\pgfqpoint{6.200000in}{4.235000in}}%
\pgfusepath{clip}%
\pgfsetbuttcap%
\pgfsetroundjoin%
\definecolor{currentfill}{rgb}{0.333333,0.658824,0.407843}%
\pgfsetfillcolor{currentfill}%
\pgfsetlinewidth{1.003750pt}%
\definecolor{currentstroke}{rgb}{0.333333,0.658824,0.407843}%
\pgfsetstrokecolor{currentstroke}%
\pgfsetdash{}{0pt}%
\pgfsys@defobject{currentmarker}{\pgfqpoint{-0.069444in}{-0.000000in}}{\pgfqpoint{0.069444in}{0.000000in}}{%
\pgfpathmoveto{\pgfqpoint{0.069444in}{-0.000000in}}%
\pgfpathlineto{\pgfqpoint{-0.069444in}{0.000000in}}%
\pgfusepath{stroke,fill}%
}%
\begin{pgfscope}%
\pgfsys@transformshift{1.281818in}{0.910992in}%
\pgfsys@useobject{currentmarker}{}%
\end{pgfscope}%
\begin{pgfscope}%
\pgfsys@transformshift{1.657576in}{0.998294in}%
\pgfsys@useobject{currentmarker}{}%
\end{pgfscope}%
\begin{pgfscope}%
\pgfsys@transformshift{2.033333in}{1.103056in}%
\pgfsys@useobject{currentmarker}{}%
\end{pgfscope}%
\begin{pgfscope}%
\pgfsys@transformshift{2.409091in}{1.251468in}%
\pgfsys@useobject{currentmarker}{}%
\end{pgfscope}%
\begin{pgfscope}%
\pgfsys@transformshift{2.784848in}{1.722897in}%
\pgfsys@useobject{currentmarker}{}%
\end{pgfscope}%
\begin{pgfscope}%
\pgfsys@transformshift{3.160606in}{2.412579in}%
\pgfsys@useobject{currentmarker}{}%
\end{pgfscope}%
\begin{pgfscope}%
\pgfsys@transformshift{3.536364in}{3.241944in}%
\pgfsys@useobject{currentmarker}{}%
\end{pgfscope}%
\begin{pgfscope}%
\pgfsys@transformshift{3.912121in}{3.800675in}%
\pgfsys@useobject{currentmarker}{}%
\end{pgfscope}%
\begin{pgfscope}%
\pgfsys@transformshift{4.287879in}{4.167341in}%
\pgfsys@useobject{currentmarker}{}%
\end{pgfscope}%
\begin{pgfscope}%
\pgfsys@transformshift{4.663636in}{4.341944in}%
\pgfsys@useobject{currentmarker}{}%
\end{pgfscope}%
\begin{pgfscope}%
\pgfsys@transformshift{5.039394in}{4.376865in}%
\pgfsys@useobject{currentmarker}{}%
\end{pgfscope}%
\begin{pgfscope}%
\pgfsys@transformshift{5.415152in}{4.464167in}%
\pgfsys@useobject{currentmarker}{}%
\end{pgfscope}%
\begin{pgfscope}%
\pgfsys@transformshift{6.166667in}{4.534008in}%
\pgfsys@useobject{currentmarker}{}%
\end{pgfscope}%
\begin{pgfscope}%
\pgfsys@transformshift{6.918182in}{4.621310in}%
\pgfsys@useobject{currentmarker}{}%
\end{pgfscope}%
\end{pgfscope}%
\begin{pgfscope}%
\pgfpathrectangle{\pgfqpoint{1.000000in}{0.605000in}}{\pgfqpoint{6.200000in}{4.235000in}}%
\pgfusepath{clip}%
\pgfsetroundcap%
\pgfsetroundjoin%
\pgfsetlinewidth{1.756562pt}%
\definecolor{currentstroke}{rgb}{0.298039,0.447059,0.690196}%
\pgfsetstrokecolor{currentstroke}%
\pgfsetdash{}{0pt}%
\pgfpathmoveto{\pgfqpoint{1.281818in}{0.841151in}}%
\pgfpathlineto{\pgfqpoint{1.657576in}{0.867341in}}%
\pgfpathlineto{\pgfqpoint{2.033333in}{0.884802in}}%
\pgfpathlineto{\pgfqpoint{2.409091in}{0.954643in}}%
\pgfpathlineto{\pgfqpoint{2.784848in}{1.155437in}}%
\pgfpathlineto{\pgfqpoint{3.160606in}{1.565754in}}%
\pgfpathlineto{\pgfqpoint{3.536364in}{2.368929in}}%
\pgfpathlineto{\pgfqpoint{3.912121in}{3.311786in}}%
\pgfpathlineto{\pgfqpoint{4.287879in}{3.992738in}}%
\pgfpathlineto{\pgfqpoint{4.663636in}{4.228452in}}%
\pgfpathlineto{\pgfqpoint{5.039394in}{4.385595in}}%
\pgfpathlineto{\pgfqpoint{5.415152in}{4.411786in}}%
\pgfpathlineto{\pgfqpoint{6.166667in}{4.481627in}}%
\pgfpathlineto{\pgfqpoint{6.918182in}{4.603849in}}%
\pgfusepath{stroke}%
\end{pgfscope}%
\begin{pgfscope}%
\pgfpathrectangle{\pgfqpoint{1.000000in}{0.605000in}}{\pgfqpoint{6.200000in}{4.235000in}}%
\pgfusepath{clip}%
\pgfsetroundcap%
\pgfsetroundjoin%
\pgfsetlinewidth{1.756562pt}%
\definecolor{currentstroke}{rgb}{0.333333,0.658824,0.407843}%
\pgfsetstrokecolor{currentstroke}%
\pgfsetdash{}{0pt}%
\pgfpathmoveto{\pgfqpoint{1.281818in}{0.867341in}}%
\pgfpathlineto{\pgfqpoint{1.657576in}{0.954643in}}%
\pgfpathlineto{\pgfqpoint{2.033333in}{1.059405in}}%
\pgfpathlineto{\pgfqpoint{2.409091in}{1.207817in}}%
\pgfpathlineto{\pgfqpoint{2.784848in}{1.679246in}}%
\pgfpathlineto{\pgfqpoint{3.160606in}{2.368929in}}%
\pgfpathlineto{\pgfqpoint{3.536364in}{3.198294in}}%
\pgfpathlineto{\pgfqpoint{3.912121in}{3.757024in}}%
\pgfpathlineto{\pgfqpoint{4.287879in}{4.123690in}}%
\pgfpathlineto{\pgfqpoint{4.663636in}{4.298294in}}%
\pgfpathlineto{\pgfqpoint{5.039394in}{4.333214in}}%
\pgfpathlineto{\pgfqpoint{5.415152in}{4.420516in}}%
\pgfpathlineto{\pgfqpoint{6.166667in}{4.490357in}}%
\pgfpathlineto{\pgfqpoint{6.918182in}{4.577659in}}%
\pgfusepath{stroke}%
\end{pgfscope}%
\begin{pgfscope}%
\pgfsetrectcap%
\pgfsetmiterjoin%
\pgfsetlinewidth{0.000000pt}%
\definecolor{currentstroke}{rgb}{1.000000,1.000000,1.000000}%
\pgfsetstrokecolor{currentstroke}%
\pgfsetdash{}{0pt}%
\pgfpathmoveto{\pgfqpoint{1.000000in}{0.605000in}}%
\pgfpathlineto{\pgfqpoint{1.000000in}{4.840000in}}%
\pgfusepath{}%
\end{pgfscope}%
\begin{pgfscope}%
\pgfsetrectcap%
\pgfsetmiterjoin%
\pgfsetlinewidth{0.000000pt}%
\definecolor{currentstroke}{rgb}{1.000000,1.000000,1.000000}%
\pgfsetstrokecolor{currentstroke}%
\pgfsetdash{}{0pt}%
\pgfpathmoveto{\pgfqpoint{7.200000in}{0.605000in}}%
\pgfpathlineto{\pgfqpoint{7.200000in}{4.840000in}}%
\pgfusepath{}%
\end{pgfscope}%
\begin{pgfscope}%
\pgfsetrectcap%
\pgfsetmiterjoin%
\pgfsetlinewidth{0.000000pt}%
\definecolor{currentstroke}{rgb}{1.000000,1.000000,1.000000}%
\pgfsetstrokecolor{currentstroke}%
\pgfsetdash{}{0pt}%
\pgfpathmoveto{\pgfqpoint{1.000000in}{0.605000in}}%
\pgfpathlineto{\pgfqpoint{7.200000in}{0.605000in}}%
\pgfusepath{}%
\end{pgfscope}%
\begin{pgfscope}%
\pgfsetrectcap%
\pgfsetmiterjoin%
\pgfsetlinewidth{0.000000pt}%
\definecolor{currentstroke}{rgb}{1.000000,1.000000,1.000000}%
\pgfsetstrokecolor{currentstroke}%
\pgfsetdash{}{0pt}%
\pgfpathmoveto{\pgfqpoint{1.000000in}{4.840000in}}%
\pgfpathlineto{\pgfqpoint{7.200000in}{4.840000in}}%
\pgfusepath{}%
\end{pgfscope}%
\begin{pgfscope}%
\definecolor{textcolor}{rgb}{0.150000,0.150000,0.150000}%
\pgfsetstrokecolor{textcolor}%
\pgfsetfillcolor{textcolor}%
\pgftext[x=4.100000in,y=4.923333in,,base]{\color{textcolor}\rmfamily\fontsize{12.000000}{14.400000}\selectfont Magnetic Field Dependent Resistivity of YBa\(\displaystyle _2\)Cu\(\displaystyle _3\)O\(\displaystyle _7\)}%
\end{pgfscope}%
\begin{pgfscope}%
\pgfsetbuttcap%
\pgfsetroundjoin%
\definecolor{currentfill}{rgb}{0.298039,0.447059,0.690196}%
\pgfsetfillcolor{currentfill}%
\pgfsetlinewidth{0.301125pt}%
\definecolor{currentstroke}{rgb}{0.298039,0.447059,0.690196}%
\pgfsetstrokecolor{currentstroke}%
\pgfsetdash{}{0pt}%
\pgfsys@defobject{currentmarker}{\pgfqpoint{-0.048611in}{-0.048611in}}{\pgfqpoint{0.048611in}{0.048611in}}{%
\pgfpathmoveto{\pgfqpoint{0.000000in}{-0.048611in}}%
\pgfpathcurveto{\pgfqpoint{0.012892in}{-0.048611in}}{\pgfqpoint{0.025257in}{-0.043489in}}{\pgfqpoint{0.034373in}{-0.034373in}}%
\pgfpathcurveto{\pgfqpoint{0.043489in}{-0.025257in}}{\pgfqpoint{0.048611in}{-0.012892in}}{\pgfqpoint{0.048611in}{0.000000in}}%
\pgfpathcurveto{\pgfqpoint{0.048611in}{0.012892in}}{\pgfqpoint{0.043489in}{0.025257in}}{\pgfqpoint{0.034373in}{0.034373in}}%
\pgfpathcurveto{\pgfqpoint{0.025257in}{0.043489in}}{\pgfqpoint{0.012892in}{0.048611in}}{\pgfqpoint{0.000000in}{0.048611in}}%
\pgfpathcurveto{\pgfqpoint{-0.012892in}{0.048611in}}{\pgfqpoint{-0.025257in}{0.043489in}}{\pgfqpoint{-0.034373in}{0.034373in}}%
\pgfpathcurveto{\pgfqpoint{-0.043489in}{0.025257in}}{\pgfqpoint{-0.048611in}{0.012892in}}{\pgfqpoint{-0.048611in}{0.000000in}}%
\pgfpathcurveto{\pgfqpoint{-0.048611in}{-0.012892in}}{\pgfqpoint{-0.043489in}{-0.025257in}}{\pgfqpoint{-0.034373in}{-0.034373in}}%
\pgfpathcurveto{\pgfqpoint{-0.025257in}{-0.043489in}}{\pgfqpoint{-0.012892in}{-0.048611in}}{\pgfqpoint{0.000000in}{-0.048611in}}%
\pgfpathlineto{\pgfqpoint{0.000000in}{-0.048611in}}%
\pgfpathclose%
\pgfusepath{stroke,fill}%
}%
\begin{pgfscope}%
\pgfsys@transformshift{1.263889in}{4.645935in}%
\pgfsys@useobject{currentmarker}{}%
\end{pgfscope}%
\end{pgfscope}%
\begin{pgfscope}%
\definecolor{textcolor}{rgb}{0.150000,0.150000,0.150000}%
\pgfsetstrokecolor{textcolor}%
\pgfsetfillcolor{textcolor}%
\pgftext[x=1.513889in,y=4.609477in,left,base]{\color{textcolor}\rmfamily\fontsize{10.000000}{12.000000}\selectfont B = 0 mT}%
\end{pgfscope}%
\begin{pgfscope}%
\pgfsetbuttcap%
\pgfsetroundjoin%
\definecolor{currentfill}{rgb}{0.333333,0.658824,0.407843}%
\pgfsetfillcolor{currentfill}%
\pgfsetlinewidth{0.301125pt}%
\definecolor{currentstroke}{rgb}{0.333333,0.658824,0.407843}%
\pgfsetstrokecolor{currentstroke}%
\pgfsetdash{}{0pt}%
\pgfsys@defobject{currentmarker}{\pgfqpoint{-0.048611in}{-0.048611in}}{\pgfqpoint{0.048611in}{0.048611in}}{%
\pgfpathmoveto{\pgfqpoint{0.000000in}{-0.048611in}}%
\pgfpathcurveto{\pgfqpoint{0.012892in}{-0.048611in}}{\pgfqpoint{0.025257in}{-0.043489in}}{\pgfqpoint{0.034373in}{-0.034373in}}%
\pgfpathcurveto{\pgfqpoint{0.043489in}{-0.025257in}}{\pgfqpoint{0.048611in}{-0.012892in}}{\pgfqpoint{0.048611in}{0.000000in}}%
\pgfpathcurveto{\pgfqpoint{0.048611in}{0.012892in}}{\pgfqpoint{0.043489in}{0.025257in}}{\pgfqpoint{0.034373in}{0.034373in}}%
\pgfpathcurveto{\pgfqpoint{0.025257in}{0.043489in}}{\pgfqpoint{0.012892in}{0.048611in}}{\pgfqpoint{0.000000in}{0.048611in}}%
\pgfpathcurveto{\pgfqpoint{-0.012892in}{0.048611in}}{\pgfqpoint{-0.025257in}{0.043489in}}{\pgfqpoint{-0.034373in}{0.034373in}}%
\pgfpathcurveto{\pgfqpoint{-0.043489in}{0.025257in}}{\pgfqpoint{-0.048611in}{0.012892in}}{\pgfqpoint{-0.048611in}{0.000000in}}%
\pgfpathcurveto{\pgfqpoint{-0.048611in}{-0.012892in}}{\pgfqpoint{-0.043489in}{-0.025257in}}{\pgfqpoint{-0.034373in}{-0.034373in}}%
\pgfpathcurveto{\pgfqpoint{-0.025257in}{-0.043489in}}{\pgfqpoint{-0.012892in}{-0.048611in}}{\pgfqpoint{0.000000in}{-0.048611in}}%
\pgfpathlineto{\pgfqpoint{0.000000in}{-0.048611in}}%
\pgfpathclose%
\pgfusepath{stroke,fill}%
}%
\begin{pgfscope}%
\pgfsys@transformshift{1.263889in}{4.442078in}%
\pgfsys@useobject{currentmarker}{}%
\end{pgfscope}%
\end{pgfscope}%
\begin{pgfscope}%
\definecolor{textcolor}{rgb}{0.150000,0.150000,0.150000}%
\pgfsetstrokecolor{textcolor}%
\pgfsetfillcolor{textcolor}%
\pgftext[x=1.513889in,y=4.405620in,left,base]{\color{textcolor}\rmfamily\fontsize{10.000000}{12.000000}\selectfont B = 8 mT}%
\end{pgfscope}%
\end{pgfpicture}%
\makeatother%
\endgroup%
}
        \caption{this is a test of the captions to se if  HOLDER}
        \label{fig:ybco_mag}
    \end{figure}
  
    For YBCO the shift towards lower temperature is clearly visible. Notably the shift is less drastic than in the case of the type-2 superconductor. Here the $T_c$ onset is still at 92 K but the transition ends at 85 K. Additionally, the resistivity plot doesn't show the same spike that was present in the previous plot. As in the BSCCO case these measurements were taken on different days so there was a thermal cycle between the measurements. This difference probably wouldn't account for the large spike seen on the day before. The only way it would is if during the cool-heat-cool cycle the carbon conductive paste reset itself and made better contact on the second cooling.
    
    The discrepancy in the shift values can be explained by the london penetration depth. In type-1 superconductors the magnetic field is nearly entirely expelled as the transition begins. The stronger the applied field the more energy that must go into expelling the field thus shifting the entire curve towards 0 K. However in type-2 superconductors an intermediate stage that contains normal and superconducting phases can exist. This  makes the transition take place over a larger temperature range as there is an energetically favorable configuration that preserves flux.
  
 


\section*{Summary \lowercase{and} Conclusion}


\lipsum[2]



\section*{Appendix}
The general form of the single electron wave function in the presence of scatterers is given by $ \psi_{n\sigma}$, where $\sigma$ is the spin index.
    
    \begin{gather}
      \psi_{n\sigma} = \sum_{k}\braket{n}{k} \phi_{k\sigma}
      \label{eq:scatE}\\
      \left(\psi_{n\sigma}\right)^* = \sum_{k}\braket{n}{k}^*\phi_{-k,-\sigma}
      \label{eq:tscatE}
    \end{gather}

    where $\phi_{k\sigma}$ are the Bloch waves, $\braket{n}{k}$ is the scattering unitary transformation and equation \ref{eq:tscatE} is the time-reversed wave function. When considering the phonon mediated electron interaction its important to note that this is a second order interaction between the scattered states NOT the initial plane wave states. Thus when using second order perturbation theory the interaction strength is dependent on the scattered energy ($E_n$) rather than the plane wave energies. 

    \begin{gather}
      H_{Cooper Pair} = -\sum_{n, n'} V_{n, n'} c_n^*c_{-n}^*c_{-n}c_{n} \\
      V_{n, n'} = \sum_{k, k'}\frac{|\braket{n}{k}|^2|\braket{n'}{k'}|^2|M_{k-k'}|^2\hbar\omega_{k-k'}}{(\hbar\omega_{k-k'})^2 - (E_n - E_{n'})^2}
    \end{gather}



    \begin{center}
      \begin{tabular}{ |c|c|c| }
        \hline
        \multicolumn{3}{|c|}{BSCCO} \\
        \hline
        Configuration & Voltage ($\mu V$) & Resistance  ($m\Omega$) \\ 
        \hline
        a & 16.9(5) & 3.4(0)  \\ 
        \hline
        b & 12.2(1) & 2.4(2)  \\ 
        \hline
        c & 16.9(4) & 3.4(2)  \\ 
        \hline
        d & 11.9(9) & 2.4(2)  \\ 
        \hline
      \end{tabular}
    \end{center}

    \begin{center}
        \begin{tabular}{ |c|c|c| }
        \hline
        \multicolumn{3}{|c|}{YBCO} \\
        \hline
        Configuration & Voltage ($\mu V$) & Resistance  ($m\Omega$) \\ 
        \hline
        a & 13.8(5) & 3.0(9)  \\ 
        \hline
        b & 14.0(0) & 2.4(5)  \\ 
        \hline
        c & 12.8(5) & 3.1(3) \\ 
        \hline
        d & 15.2(7) & 2.4(7)  \\ 
        \hline
      \end{tabular}
    \end{center}

    \begin{center}
      \begin{tabular}{ |c|c|c| }
        \hline
        Configuration & Voltage ($\mu V$) & Resistance  ($m\Omega$) \\ 
        \hline
        \multicolumn{3}{|c|}{BSCCO} \\
        \hline
        a & 16.9(5) & 3.4(0)  \\ 
        \hline
        b & 12.2(1) & 2.4(2)  \\ 
        \hline
        c & 16.9(4) & 3.4(2)  \\ 
        \hline
        d & 11.9(9) & 2.4(2)  \\ 
        \hline
        \multicolumn{3}{|c|}{YBCO} \\
        \hline
        a & 13.8(5) & 3.0(9)  \\ 
        \hline
        b & 14.0(0) & 2.4(5)  \\ 
        \hline
        c & 12.8(5) & 3.1(3) \\ 
        \hline
        d & 15.2(7) & 2.4(7)  \\ 
        \hline
      \end{tabular}
    \end{center}
\bibliography{superconductivity_Denzel_Ayala}{}

\end{document}