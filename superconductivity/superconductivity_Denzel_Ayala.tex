\documentclass[
reprint,
amsmath,amssymb,
aps,
tikz,
border=5pt
]{revtex4-1}

\usepackage{graphicx}% Include figure files
\usepackage{dcolumn}% Align table columns on decimal point
\usepackage{bm}% bold math
\usepackage{amsmath}
\usepackage{amssymb}
\usepackage{subcaption}
\usepackage{lipsum}
\usepackage{tikz}

\usetikzlibrary{arrows}

\AtBeginDocument{\let\latexlabel\label}


\begin{document}


\preprint{APS/123-QED}

\title{Ferroic Materials: Understanding their Phases\\ and Multiferroic Potential}

\author{Denzel Ayala} 
 \email{Denzelay@buffalo.edu}
 \altaffiliation[Also at ]{University of Vermont}%Lines break automatically or can be forced with \\
\author{Mars Anderson}
\author{Serdar K. Gozpinar}

\affiliation{%
State University of New York at Buffalo\\Department of Physics
}%

%\collaboration{CLEO Collaboration}%\noaffiliation

\date{\today}% It is always \today, today,
             %  but any date may be explicitly specified

\begin{abstract}

  Ferroic materials, exhibit switchable long range ordered states that are energetically equivalent. This paper explores the characteristics and properties of ferro- magnetic, elastic, and electric phases with the goal of combining existing ferroics into multiferroics. With  the increasing spread of information technologies, magnetoelectric multiferroics hold great potential for revolutionizing modern electronics with low-power consumption and high-switch-rate capabilities. The fundamentally different chemical species required for ferroelectrics and ferromagnets makes their synthesis elusive but various approaches, such as multicomponent composites and structurally driven multiferroics, have been explored and successfully overcome this challenge.
\end{abstract}
\maketitle

\section*{\label{sec:intro}Introduction \\\lowercase{to} Superconductivity}
\end{document}