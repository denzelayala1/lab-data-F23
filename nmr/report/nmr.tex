\documentclass[
reprint,
amsmath,amssymb,
aps,
tikz,
border=5pt
]{revtex4-1}

\usepackage{graphicx}% Include figure files
\usepackage{dcolumn}% Align table columns on decimal point
\usepackage{bm}% bold math
\usepackage{amsmath}
\usepackage{amssymb}
\usepackage{subcaption}
\usepackage{lipsum}
\usepackage{tikz}
\usepackage{pgf}
\usepackage{circuitikz}
\usepackage{multirow}
\usetikzlibrary{arrows}
\usepackage{mathtools}
\DeclarePairedDelimiter\bra{\langle}{\rvert}
\DeclarePairedDelimiter\ket{\lvert}{\rangle}
\DeclarePairedDelimiterX\braket[2]{\langle}{\rangle}{#1\,\delimsize\vert\,\mathopen{}#2}
\AtBeginDocument{\let\latexlabel\label}


\begin{document}


\preprint{APS/123-QED}

\title{Spin-Lattice and Spin-Spin Relaxation Time Measurements \\ for Mineral Oil and Aqueous CuSO$_{4} \boldsymbol{\cdot} $H$_2$O}

\author{Denzel Ayala} 
 \email{Denzelay@buffalo.edu}
 \altaffiliation[Also at ]{University of Vermont}%Lines break automatically or can be forced with \\
\author{Jonathan DeMaria}
 \email{jmdemari@buffalo.edu}
\author{Serdar K. Gozpinar}

\affiliation{%
State University of New York at Buffalo\\Department of Physics
}%

%\collaboration{CLEO Collaboration}%\noaffiliation

\date{\today}% It is always \today, today,
             %  but any date may be explicitly specified

\begin{abstract}

The resistivity of two cuprates $Bi_2Sr_2Ca_{2}Cu_3O_{10}$ (BSCCO) and $YBa_2Cu_3O_7$ (YBCO) was measured between 80--160 K to identify their critical temperature as superconductors with and without the presence of a static 122 mT magnetic field. The samples were measured using the Van Der Pauw method. Both materials displayed a high temperature superconductivity with BSCCO showing a $ T_c = 105\pm 7 K$ when $B=0 mT$ and $T_c = 98\pm 14 K$ when $B=122 mT$. YBCO had a lower temperature range at $T_c = 91\pm 2 K$ when $B=0 mT$ and $T_c = 88.5\pm 4 K$ when $B=122 mT$.

\end{abstract}

\maketitle

\section*{\label{sec:intro}Introduction \\\lowercase{to} NMR}

  This is jhust rankdom test asdfl;kjfdsa ppijls.~\cite{ashcroft76}


  Broad intro with applications

  basics of physics i.e. nucelar spin and magnetic field alignment

\subsection*{Pulse Sequences}

\subsubsection*{T1 sequence}

\subsubsection*{spin echo}

\subsubsection*{CP method}

\subsubsection*{MG method}


\section*{Experimental }

\subsection*{Equipment}

    \begin{figure}[h]
      \includegraphics[width=0.4 \textwidth]{figures/block_diagram.png}
      \caption{Block diagram of experimental set up.}
      \label{fig:block}
    \end{figure}


    A TeachSpin PS1-A system was used to produce/amplify the pulse sequences, and pickup/amplify the output signal. The signal was then processed/synced and displayed on a Tektronix TBS 1052B Digital Oscilloscope. A simplified block diagram of the system is shown in figure~\ref{fig:block}. The pulse sequence begins at the \textbf{pulse programer}. The exact number of pulses, their widths, and time characteristics are defined in this module. This sends two signals, the first goes to the \textbf{oscilloscope}, the second to the \textbf{RF synthesized oscillator}. This first signal is to let the oscilloscope know when a output signal is coming. The second signal tells the RF synthesized oscillator what pulse sequence to generate. From there the pulse sequence is sent to an \textbf{RF amplifier} and the \textbf{mixer}. The amplifier ensure that a strong enough signal sequence is transmitted to the transmitter Helmholtz coils in the sample chamber such that it influences the nuclear spins. The mixer combines the reference signal with the sample output signal. Before getting to the mixer the sample output signal goes to the \textbf{receiver}. The receiver also sends the output signal to a \textbf{RF amplifier detector}. The mixer and RF amplifier detector both send the signal to the oscilloscope. 
    
    The \textbf{pulse programer}
    
    
  
    
    
    P1: block diagram and overview of set usepackage
    
    P2: 3module system

    p3: sample holder

    p4: oscilloscope


Discribe 3 module system,  oscilloscope. 


\subsection*{Procedure}

\subsubsection*{Sample Preparation} 
    The two chemical species of interest were mineral oil, and CuSO$_{4} \boldsymbol{\cdot} $H$_2$O. The mineral oil was used as purchased from supplier. As CuSO$_{4}$ is a blue crystalline solid it was dissolved in deionized water prior to measurement.Seven sample concentrations of 0.005M, 0.01M, 0.05M, 0.1M, 0.2M 0.5M, \& 1M CuSO$_{4} \boldsymbol{\cdot} $H$_2$O were prepared. 

    Chemical species were loaded into their own glass sample vials. Samples should be approximately cubical (~5mm tall) so that the magnetic field is experienced uniformly in the entire sample. Sample vials are caped with a rubber stop to ensure that concentrations are consistent and there is minimal evaporation or contamination. A rubber o-ring was used to adjust the sample depth inside of the probe. The o-ring was placed at the same height for all sample. 


\subsubsection*{Single pulse FID}
    talk about capaciter, gain, i.e. the stuff that must be done for ALL pulse sequences
\subsubsection*{Magnetic Field Contours}

    In order to maximize signal amplitude the spatial dependence of the magnetic field was mapped so the sample can be placed in the section with the most uniform field. This was done by finding the resonant frequency at every point in a regular grid. Field uniformity is determined by a region with minimal change in the resonant frequency. As the resonant frequency of the magnet drift over time due to thermal effects it is crucial to take these measurements rapidly. 
    
    The direction of the permanent magnetic field $(B_0)$ will be called the z-axis. The sample holder allowed for freedom of movement in the x and y directions. Here the y-axis moved the sample up and down while the x-direction moved the sample perpendicular to the sample table. The values of the y-axis range from $0\ldots20~cm$ and begin at the bottom of the magnet and finish at the top. The x-axis places its zero point at the center of the magnet and ranges from $-10\ldots10~cm$ The field was mapped from $-1~cm \leq x \leq +1~cm$ with step sizes of $0.2~cm$ and from $9~cm \leq y \leq 11~cm$ with step sizes of $0.5~cm$. 

\subsubsection*{Spin-Lattice Relaxation: T1 Sequence}
    A quick estimate of T$_1$ was acquired. First the spectrometer was set to resonance for a single pulse FID with a large repetition time. The repetition time was slowly lowered until the maximum amplitude of the FID decreased. The point just before the amplitude decrease is $\sim 4-5\times$T$_1$.

    For accurate measurement of T$_1$ a two pulse sequence of: $180^{\circ}- \tau \text{(variable)} - 90^{\circ}\text{ (FID)}$

\subsubsection*{Spin Echo}

\subsubsection*{Carr-Purcell method}

\subsubsection*{Meiboom-Gill method}




\section*{Results \lowercase{and} Discussion }




  \begin{tabular}[b]{ |c|c|c| }
    \hline
    \multicolumn{3}{|c|}{Mineral Oil} \\ \hline
     Method & T$_1\pm$ error (ms) & T$_2\pm$ error (ms) \\ \hline
     Single Pulse & $47\pm 4$ & - \\ \hline 
     Spin Echo & - & $28\pm 1$ \\ \hline
     Carl-Purcell & - & $80\pm 20$ \\ \hline 
     Meiboom-Gill & - & $61\pm 6$ \\ \hline
    \multicolumn{3}{|c|}{CuSO$_{4} \boldsymbol{\cdot} $H$_2$O} \\ \hline
    Molarity (mol/L) & T$_1\pm$ error (ms) & T$_2\pm$ error (ms) \\ \hline
    0.005 M & $240\pm 40 $ & $130 \pm 10$ \\ \hline 
    0.01 M & $140 \pm 30$ & $100 \pm 10$ \\ \hline
    0.05 M & $17.4 \pm 0.7 $ & $ 19.0 \pm 0.5$ \\ \hline 
    0.1 M & $8.2 \pm 0.6$ & $8.0 \pm 0.4$ \\ \hline
    0.2 M & $4.1 \pm 0.3$ & $4.9 \pm 0.4$ \\ \hline 
    0.5 M & $1.7 \pm 0.1$ & $2.5 \pm 0.5$ \\ \hline
    1.0 M & $0.75 \pm 0.05$ & $1.01 \pm 0.04$ \\ \hline 
  \end{tabular}
  

  
\subsubsection*{Single pulse FID}

\subsubsection*{Magnetic Field Contours}

  \begin{figure}[h]
    \resizebox{0.55\textwidth}{!}{%% Creator: Matplotlib, PGF backend
%%
%% To include the figure in your LaTeX document, write
%%   \input{<filename>.pgf}
%%
%% Make sure the required packages are loaded in your preamble
%%   \usepackage{pgf}
%%
%% Also ensure that all the required font packages are loaded; for instance,
%% the lmodern package is sometimes necessary when using math font.
%%   \usepackage{lmodern}
%%
%% Figures using additional raster images can only be included by \input if
%% they are in the same directory as the main LaTeX file. For loading figures
%% from other directories you can use the `import` package
%%   \usepackage{import}
%%
%% and then include the figures with
%%   \import{<path to file>}{<filename>.pgf}
%%
%% Matplotlib used the following preamble
%%   \usepackage{fontspec}
%%   \setmainfont{DejaVuSerif.ttf}[Path=\detokenize{/usr/share/matplotlib/mpl-data/fonts/ttf/}]
%%   \setsansfont{DejaVuSans.ttf}[Path=\detokenize{/usr/share/matplotlib/mpl-data/fonts/ttf/}]
%%   \setmonofont{DejaVuSansMono.ttf}[Path=\detokenize{/usr/share/matplotlib/mpl-data/fonts/ttf/}]
%%
\begingroup%
\makeatletter%
\begin{pgfpicture}%
\pgfpathrectangle{\pgfpointorigin}{\pgfqpoint{8.000000in}{5.500000in}}%
\pgfusepath{use as bounding box, clip}%
\begin{pgfscope}%
\pgfsetbuttcap%
\pgfsetmiterjoin%
\definecolor{currentfill}{rgb}{1.000000,1.000000,1.000000}%
\pgfsetfillcolor{currentfill}%
\pgfsetlinewidth{0.000000pt}%
\definecolor{currentstroke}{rgb}{1.000000,1.000000,1.000000}%
\pgfsetstrokecolor{currentstroke}%
\pgfsetdash{}{0pt}%
\pgfpathmoveto{\pgfqpoint{0.000000in}{0.000000in}}%
\pgfpathlineto{\pgfqpoint{8.000000in}{0.000000in}}%
\pgfpathlineto{\pgfqpoint{8.000000in}{5.500000in}}%
\pgfpathlineto{\pgfqpoint{0.000000in}{5.500000in}}%
\pgfpathlineto{\pgfqpoint{0.000000in}{0.000000in}}%
\pgfpathclose%
\pgfusepath{fill}%
\end{pgfscope}%
\begin{pgfscope}%
\pgfsetbuttcap%
\pgfsetmiterjoin%
\definecolor{currentfill}{rgb}{0.917647,0.917647,0.949020}%
\pgfsetfillcolor{currentfill}%
\pgfsetlinewidth{0.000000pt}%
\definecolor{currentstroke}{rgb}{0.000000,0.000000,0.000000}%
\pgfsetstrokecolor{currentstroke}%
\pgfsetstrokeopacity{0.000000}%
\pgfsetdash{}{0pt}%
\pgfpathmoveto{\pgfqpoint{2.099978in}{0.840429in}}%
\pgfpathlineto{\pgfqpoint{6.272961in}{0.840429in}}%
\pgfpathlineto{\pgfqpoint{6.272961in}{5.013411in}}%
\pgfpathlineto{\pgfqpoint{2.099978in}{5.013411in}}%
\pgfpathlineto{\pgfqpoint{2.099978in}{0.840429in}}%
\pgfpathclose%
\pgfusepath{fill}%
\end{pgfscope}%
\begin{pgfscope}%
\pgfpathrectangle{\pgfqpoint{2.099978in}{0.840429in}}{\pgfqpoint{4.172982in}{4.172982in}}%
\pgfusepath{clip}%
\pgfsetroundcap%
\pgfsetroundjoin%
\pgfsetlinewidth{1.003750pt}%
\definecolor{currentstroke}{rgb}{1.000000,1.000000,1.000000}%
\pgfsetstrokecolor{currentstroke}%
\pgfsetdash{}{0pt}%
\pgfpathmoveto{\pgfqpoint{2.099978in}{0.840429in}}%
\pgfpathlineto{\pgfqpoint{2.099978in}{5.013411in}}%
\pgfusepath{stroke}%
\end{pgfscope}%
\begin{pgfscope}%
\definecolor{textcolor}{rgb}{0.150000,0.150000,0.150000}%
\pgfsetstrokecolor{textcolor}%
\pgfsetfillcolor{textcolor}%
\pgftext[x=2.099978in,y=0.743207in,,top]{\color{textcolor}\rmfamily\fontsize{20.000000}{24.000000}\selectfont \ensuremath{-}1.0}%
\end{pgfscope}%
\begin{pgfscope}%
\pgfpathrectangle{\pgfqpoint{2.099978in}{0.840429in}}{\pgfqpoint{4.172982in}{4.172982in}}%
\pgfusepath{clip}%
\pgfsetroundcap%
\pgfsetroundjoin%
\pgfsetlinewidth{1.003750pt}%
\definecolor{currentstroke}{rgb}{1.000000,1.000000,1.000000}%
\pgfsetstrokecolor{currentstroke}%
\pgfsetdash{}{0pt}%
\pgfpathmoveto{\pgfqpoint{3.143224in}{0.840429in}}%
\pgfpathlineto{\pgfqpoint{3.143224in}{5.013411in}}%
\pgfusepath{stroke}%
\end{pgfscope}%
\begin{pgfscope}%
\definecolor{textcolor}{rgb}{0.150000,0.150000,0.150000}%
\pgfsetstrokecolor{textcolor}%
\pgfsetfillcolor{textcolor}%
\pgftext[x=3.143224in,y=0.743207in,,top]{\color{textcolor}\rmfamily\fontsize{20.000000}{24.000000}\selectfont \ensuremath{-}0.5}%
\end{pgfscope}%
\begin{pgfscope}%
\pgfpathrectangle{\pgfqpoint{2.099978in}{0.840429in}}{\pgfqpoint{4.172982in}{4.172982in}}%
\pgfusepath{clip}%
\pgfsetroundcap%
\pgfsetroundjoin%
\pgfsetlinewidth{1.003750pt}%
\definecolor{currentstroke}{rgb}{1.000000,1.000000,1.000000}%
\pgfsetstrokecolor{currentstroke}%
\pgfsetdash{}{0pt}%
\pgfpathmoveto{\pgfqpoint{4.186470in}{0.840429in}}%
\pgfpathlineto{\pgfqpoint{4.186470in}{5.013411in}}%
\pgfusepath{stroke}%
\end{pgfscope}%
\begin{pgfscope}%
\definecolor{textcolor}{rgb}{0.150000,0.150000,0.150000}%
\pgfsetstrokecolor{textcolor}%
\pgfsetfillcolor{textcolor}%
\pgftext[x=4.186470in,y=0.743207in,,top]{\color{textcolor}\rmfamily\fontsize{20.000000}{24.000000}\selectfont 0.0}%
\end{pgfscope}%
\begin{pgfscope}%
\pgfpathrectangle{\pgfqpoint{2.099978in}{0.840429in}}{\pgfqpoint{4.172982in}{4.172982in}}%
\pgfusepath{clip}%
\pgfsetroundcap%
\pgfsetroundjoin%
\pgfsetlinewidth{1.003750pt}%
\definecolor{currentstroke}{rgb}{1.000000,1.000000,1.000000}%
\pgfsetstrokecolor{currentstroke}%
\pgfsetdash{}{0pt}%
\pgfpathmoveto{\pgfqpoint{5.229715in}{0.840429in}}%
\pgfpathlineto{\pgfqpoint{5.229715in}{5.013411in}}%
\pgfusepath{stroke}%
\end{pgfscope}%
\begin{pgfscope}%
\definecolor{textcolor}{rgb}{0.150000,0.150000,0.150000}%
\pgfsetstrokecolor{textcolor}%
\pgfsetfillcolor{textcolor}%
\pgftext[x=5.229715in,y=0.743207in,,top]{\color{textcolor}\rmfamily\fontsize{20.000000}{24.000000}\selectfont 0.5}%
\end{pgfscope}%
\begin{pgfscope}%
\pgfpathrectangle{\pgfqpoint{2.099978in}{0.840429in}}{\pgfqpoint{4.172982in}{4.172982in}}%
\pgfusepath{clip}%
\pgfsetroundcap%
\pgfsetroundjoin%
\pgfsetlinewidth{1.003750pt}%
\definecolor{currentstroke}{rgb}{1.000000,1.000000,1.000000}%
\pgfsetstrokecolor{currentstroke}%
\pgfsetdash{}{0pt}%
\pgfpathmoveto{\pgfqpoint{6.272961in}{0.840429in}}%
\pgfpathlineto{\pgfqpoint{6.272961in}{5.013411in}}%
\pgfusepath{stroke}%
\end{pgfscope}%
\begin{pgfscope}%
\definecolor{textcolor}{rgb}{0.150000,0.150000,0.150000}%
\pgfsetstrokecolor{textcolor}%
\pgfsetfillcolor{textcolor}%
\pgftext[x=6.272961in,y=0.743207in,,top]{\color{textcolor}\rmfamily\fontsize{20.000000}{24.000000}\selectfont 1.0}%
\end{pgfscope}%
\begin{pgfscope}%
\definecolor{textcolor}{rgb}{0.150000,0.150000,0.150000}%
\pgfsetstrokecolor{textcolor}%
\pgfsetfillcolor{textcolor}%
\pgftext[x=4.186470in,y=0.418826in,,top]{\color{textcolor}\rmfamily\fontsize{20.000000}{24.000000}\selectfont X-axis (cm)}%
\end{pgfscope}%
\begin{pgfscope}%
\pgfpathrectangle{\pgfqpoint{2.099978in}{0.840429in}}{\pgfqpoint{4.172982in}{4.172982in}}%
\pgfusepath{clip}%
\pgfsetroundcap%
\pgfsetroundjoin%
\pgfsetlinewidth{1.003750pt}%
\definecolor{currentstroke}{rgb}{1.000000,1.000000,1.000000}%
\pgfsetstrokecolor{currentstroke}%
\pgfsetdash{}{0pt}%
\pgfpathmoveto{\pgfqpoint{2.099978in}{0.840429in}}%
\pgfpathlineto{\pgfqpoint{6.272961in}{0.840429in}}%
\pgfusepath{stroke}%
\end{pgfscope}%
\begin{pgfscope}%
\definecolor{textcolor}{rgb}{0.150000,0.150000,0.150000}%
\pgfsetstrokecolor{textcolor}%
\pgfsetfillcolor{textcolor}%
\pgftext[x=1.560997in, y=0.734906in, left, base]{\color{textcolor}\rmfamily\fontsize{20.000000}{24.000000}\selectfont 9.0}%
\end{pgfscope}%
\begin{pgfscope}%
\pgfpathrectangle{\pgfqpoint{2.099978in}{0.840429in}}{\pgfqpoint{4.172982in}{4.172982in}}%
\pgfusepath{clip}%
\pgfsetroundcap%
\pgfsetroundjoin%
\pgfsetlinewidth{1.003750pt}%
\definecolor{currentstroke}{rgb}{1.000000,1.000000,1.000000}%
\pgfsetstrokecolor{currentstroke}%
\pgfsetdash{}{0pt}%
\pgfpathmoveto{\pgfqpoint{2.099978in}{1.883675in}}%
\pgfpathlineto{\pgfqpoint{6.272961in}{1.883675in}}%
\pgfusepath{stroke}%
\end{pgfscope}%
\begin{pgfscope}%
\definecolor{textcolor}{rgb}{0.150000,0.150000,0.150000}%
\pgfsetstrokecolor{textcolor}%
\pgfsetfillcolor{textcolor}%
\pgftext[x=1.560997in, y=1.778152in, left, base]{\color{textcolor}\rmfamily\fontsize{20.000000}{24.000000}\selectfont 9.5}%
\end{pgfscope}%
\begin{pgfscope}%
\pgfpathrectangle{\pgfqpoint{2.099978in}{0.840429in}}{\pgfqpoint{4.172982in}{4.172982in}}%
\pgfusepath{clip}%
\pgfsetroundcap%
\pgfsetroundjoin%
\pgfsetlinewidth{1.003750pt}%
\definecolor{currentstroke}{rgb}{1.000000,1.000000,1.000000}%
\pgfsetstrokecolor{currentstroke}%
\pgfsetdash{}{0pt}%
\pgfpathmoveto{\pgfqpoint{2.099978in}{2.926920in}}%
\pgfpathlineto{\pgfqpoint{6.272961in}{2.926920in}}%
\pgfusepath{stroke}%
\end{pgfscope}%
\begin{pgfscope}%
\definecolor{textcolor}{rgb}{0.150000,0.150000,0.150000}%
\pgfsetstrokecolor{textcolor}%
\pgfsetfillcolor{textcolor}%
\pgftext[x=1.384267in, y=2.821397in, left, base]{\color{textcolor}\rmfamily\fontsize{20.000000}{24.000000}\selectfont 10.0}%
\end{pgfscope}%
\begin{pgfscope}%
\pgfpathrectangle{\pgfqpoint{2.099978in}{0.840429in}}{\pgfqpoint{4.172982in}{4.172982in}}%
\pgfusepath{clip}%
\pgfsetroundcap%
\pgfsetroundjoin%
\pgfsetlinewidth{1.003750pt}%
\definecolor{currentstroke}{rgb}{1.000000,1.000000,1.000000}%
\pgfsetstrokecolor{currentstroke}%
\pgfsetdash{}{0pt}%
\pgfpathmoveto{\pgfqpoint{2.099978in}{3.970166in}}%
\pgfpathlineto{\pgfqpoint{6.272961in}{3.970166in}}%
\pgfusepath{stroke}%
\end{pgfscope}%
\begin{pgfscope}%
\definecolor{textcolor}{rgb}{0.150000,0.150000,0.150000}%
\pgfsetstrokecolor{textcolor}%
\pgfsetfillcolor{textcolor}%
\pgftext[x=1.384267in, y=3.864643in, left, base]{\color{textcolor}\rmfamily\fontsize{20.000000}{24.000000}\selectfont 10.5}%
\end{pgfscope}%
\begin{pgfscope}%
\pgfpathrectangle{\pgfqpoint{2.099978in}{0.840429in}}{\pgfqpoint{4.172982in}{4.172982in}}%
\pgfusepath{clip}%
\pgfsetroundcap%
\pgfsetroundjoin%
\pgfsetlinewidth{1.003750pt}%
\definecolor{currentstroke}{rgb}{1.000000,1.000000,1.000000}%
\pgfsetstrokecolor{currentstroke}%
\pgfsetdash{}{0pt}%
\pgfpathmoveto{\pgfqpoint{2.099978in}{5.013411in}}%
\pgfpathlineto{\pgfqpoint{6.272961in}{5.013411in}}%
\pgfusepath{stroke}%
\end{pgfscope}%
\begin{pgfscope}%
\definecolor{textcolor}{rgb}{0.150000,0.150000,0.150000}%
\pgfsetstrokecolor{textcolor}%
\pgfsetfillcolor{textcolor}%
\pgftext[x=1.384267in, y=4.907888in, left, base]{\color{textcolor}\rmfamily\fontsize{20.000000}{24.000000}\selectfont 11.0}%
\end{pgfscope}%
\begin{pgfscope}%
\definecolor{textcolor}{rgb}{0.150000,0.150000,0.150000}%
\pgfsetstrokecolor{textcolor}%
\pgfsetfillcolor{textcolor}%
\pgftext[x=1.328711in,y=2.926920in,,bottom,rotate=90.000000]{\color{textcolor}\rmfamily\fontsize{20.000000}{24.000000}\selectfont Y-axis (cm)}%
\end{pgfscope}%
\begin{pgfscope}%
\pgfpathrectangle{\pgfqpoint{2.099978in}{0.840429in}}{\pgfqpoint{4.172982in}{4.172982in}}%
\pgfusepath{clip}%
\pgfsetbuttcap%
\pgfsetroundjoin%
\definecolor{currentfill}{rgb}{0.020588,0.020588,0.028645}%
\pgfsetfillcolor{currentfill}%
\pgfsetlinewidth{0.000000pt}%
\definecolor{currentstroke}{rgb}{0.000000,0.000000,0.000000}%
\pgfsetstrokecolor{currentstroke}%
\pgfsetdash{}{0pt}%
\pgfpathmoveto{\pgfqpoint{6.272961in}{4.720692in}}%
\pgfpathlineto{\pgfqpoint{6.272961in}{5.013411in}}%
\pgfpathlineto{\pgfqpoint{5.983323in}{5.013411in}}%
\pgfpathlineto{\pgfqpoint{6.272961in}{4.720692in}}%
\pgfpathclose%
\pgfusepath{fill}%
\end{pgfscope}%
\begin{pgfscope}%
\pgfpathrectangle{\pgfqpoint{2.099978in}{0.840429in}}{\pgfqpoint{4.172982in}{4.172982in}}%
\pgfusepath{clip}%
\pgfsetbuttcap%
\pgfsetroundjoin%
\definecolor{currentfill}{rgb}{0.065196,0.065196,0.090708}%
\pgfsetfillcolor{currentfill}%
\pgfsetlinewidth{0.000000pt}%
\definecolor{currentstroke}{rgb}{0.000000,0.000000,0.000000}%
\pgfsetstrokecolor{currentstroke}%
\pgfsetdash{}{0pt}%
\pgfpathmoveto{\pgfqpoint{5.855662in}{4.641755in}}%
\pgfpathlineto{\pgfqpoint{6.272961in}{4.275370in}}%
\pgfpathlineto{\pgfqpoint{6.272961in}{4.720692in}}%
\pgfpathlineto{\pgfqpoint{5.983323in}{5.013411in}}%
\pgfpathlineto{\pgfqpoint{5.855662in}{5.013411in}}%
\pgfpathlineto{\pgfqpoint{5.635422in}{5.013411in}}%
\pgfpathlineto{\pgfqpoint{5.855662in}{4.641755in}}%
\pgfpathclose%
\pgfusepath{fill}%
\end{pgfscope}%
\begin{pgfscope}%
\pgfpathrectangle{\pgfqpoint{2.099978in}{0.840429in}}{\pgfqpoint{4.172982in}{4.172982in}}%
\pgfusepath{clip}%
\pgfsetbuttcap%
\pgfsetroundjoin%
\definecolor{currentfill}{rgb}{0.109804,0.109804,0.152771}%
\pgfsetfillcolor{currentfill}%
\pgfsetlinewidth{0.000000pt}%
\definecolor{currentstroke}{rgb}{0.000000,0.000000,0.000000}%
\pgfsetstrokecolor{currentstroke}%
\pgfsetdash{}{0pt}%
\pgfpathmoveto{\pgfqpoint{6.272961in}{2.809019in}}%
\pgfpathlineto{\pgfqpoint{6.272961in}{2.926920in}}%
\pgfpathlineto{\pgfqpoint{6.272961in}{3.970166in}}%
\pgfpathlineto{\pgfqpoint{6.272961in}{4.275370in}}%
\pgfpathlineto{\pgfqpoint{5.855662in}{4.641755in}}%
\pgfpathlineto{\pgfqpoint{5.635422in}{5.013411in}}%
\pgfpathlineto{\pgfqpoint{5.438364in}{5.013411in}}%
\pgfpathlineto{\pgfqpoint{5.376089in}{5.013411in}}%
\pgfpathlineto{\pgfqpoint{5.438364in}{4.801502in}}%
\pgfpathlineto{\pgfqpoint{5.855662in}{4.118503in}}%
\pgfpathlineto{\pgfqpoint{6.053444in}{3.970166in}}%
\pgfpathlineto{\pgfqpoint{6.124568in}{2.926920in}}%
\pgfpathlineto{\pgfqpoint{6.272961in}{2.809019in}}%
\pgfpathclose%
\pgfusepath{fill}%
\end{pgfscope}%
\begin{pgfscope}%
\pgfpathrectangle{\pgfqpoint{2.099978in}{0.840429in}}{\pgfqpoint{4.172982in}{4.172982in}}%
\pgfusepath{clip}%
\pgfsetbuttcap%
\pgfsetroundjoin%
\definecolor{currentfill}{rgb}{0.150980,0.150980,0.210060}%
\pgfsetfillcolor{currentfill}%
\pgfsetlinewidth{0.000000pt}%
\definecolor{currentstroke}{rgb}{0.000000,0.000000,0.000000}%
\pgfsetstrokecolor{currentstroke}%
\pgfsetdash{}{0pt}%
\pgfpathmoveto{\pgfqpoint{5.855662in}{2.908705in}}%
\pgfpathlineto{\pgfqpoint{6.272961in}{2.579648in}}%
\pgfpathlineto{\pgfqpoint{6.272961in}{2.809019in}}%
\pgfpathlineto{\pgfqpoint{6.124568in}{2.926920in}}%
\pgfpathlineto{\pgfqpoint{6.053444in}{3.970166in}}%
\pgfpathlineto{\pgfqpoint{5.855662in}{4.118503in}}%
\pgfpathlineto{\pgfqpoint{5.438364in}{4.801502in}}%
\pgfpathlineto{\pgfqpoint{5.376089in}{5.013411in}}%
\pgfpathlineto{\pgfqpoint{5.205231in}{5.013411in}}%
\pgfpathlineto{\pgfqpoint{5.438364in}{4.220110in}}%
\pgfpathlineto{\pgfqpoint{5.594851in}{3.970166in}}%
\pgfpathlineto{\pgfqpoint{5.827673in}{2.926920in}}%
\pgfpathlineto{\pgfqpoint{5.855662in}{2.908705in}}%
\pgfpathclose%
\pgfusepath{fill}%
\end{pgfscope}%
\begin{pgfscope}%
\pgfpathrectangle{\pgfqpoint{2.099978in}{0.840429in}}{\pgfqpoint{4.172982in}{4.172982in}}%
\pgfusepath{clip}%
\pgfsetbuttcap%
\pgfsetroundjoin%
\definecolor{currentfill}{rgb}{0.195588,0.195588,0.272123}%
\pgfsetfillcolor{currentfill}%
\pgfsetlinewidth{0.000000pt}%
\definecolor{currentstroke}{rgb}{0.000000,0.000000,0.000000}%
\pgfsetstrokecolor{currentstroke}%
\pgfsetdash{}{0pt}%
\pgfpathmoveto{\pgfqpoint{5.438364in}{2.913748in}}%
\pgfpathlineto{\pgfqpoint{5.855662in}{2.642925in}}%
\pgfpathlineto{\pgfqpoint{6.272961in}{2.350277in}}%
\pgfpathlineto{\pgfqpoint{6.272961in}{2.579648in}}%
\pgfpathlineto{\pgfqpoint{5.855662in}{2.908705in}}%
\pgfpathlineto{\pgfqpoint{5.827673in}{2.926920in}}%
\pgfpathlineto{\pgfqpoint{5.594851in}{3.970166in}}%
\pgfpathlineto{\pgfqpoint{5.438364in}{4.220110in}}%
\pgfpathlineto{\pgfqpoint{5.205231in}{5.013411in}}%
\pgfpathlineto{\pgfqpoint{5.034372in}{5.013411in}}%
\pgfpathlineto{\pgfqpoint{5.291507in}{3.970166in}}%
\pgfpathlineto{\pgfqpoint{5.423460in}{2.926920in}}%
\pgfpathlineto{\pgfqpoint{5.438364in}{2.913748in}}%
\pgfpathclose%
\pgfusepath{fill}%
\end{pgfscope}%
\begin{pgfscope}%
\pgfpathrectangle{\pgfqpoint{2.099978in}{0.840429in}}{\pgfqpoint{4.172982in}{4.172982in}}%
\pgfusepath{clip}%
\pgfsetbuttcap%
\pgfsetroundjoin%
\definecolor{currentfill}{rgb}{0.240196,0.240196,0.334186}%
\pgfsetfillcolor{currentfill}%
\pgfsetlinewidth{0.000000pt}%
\definecolor{currentstroke}{rgb}{0.000000,0.000000,0.000000}%
\pgfsetstrokecolor{currentstroke}%
\pgfsetdash{}{0pt}%
\pgfpathmoveto{\pgfqpoint{5.438364in}{2.631861in}}%
\pgfpathlineto{\pgfqpoint{5.855662in}{2.377146in}}%
\pgfpathlineto{\pgfqpoint{6.272961in}{2.120906in}}%
\pgfpathlineto{\pgfqpoint{6.272961in}{2.350277in}}%
\pgfpathlineto{\pgfqpoint{5.855662in}{2.642925in}}%
\pgfpathlineto{\pgfqpoint{5.438364in}{2.913748in}}%
\pgfpathlineto{\pgfqpoint{5.423460in}{2.926920in}}%
\pgfpathlineto{\pgfqpoint{5.291507in}{3.970166in}}%
\pgfpathlineto{\pgfqpoint{5.034372in}{5.013411in}}%
\pgfpathlineto{\pgfqpoint{5.021066in}{5.013411in}}%
\pgfpathlineto{\pgfqpoint{4.630180in}{5.013411in}}%
\pgfpathlineto{\pgfqpoint{5.021066in}{4.023665in}}%
\pgfpathlineto{\pgfqpoint{5.033906in}{3.970166in}}%
\pgfpathlineto{\pgfqpoint{5.104526in}{2.926920in}}%
\pgfpathlineto{\pgfqpoint{5.438364in}{2.631861in}}%
\pgfpathclose%
\pgfusepath{fill}%
\end{pgfscope}%
\begin{pgfscope}%
\pgfpathrectangle{\pgfqpoint{2.099978in}{0.840429in}}{\pgfqpoint{4.172982in}{4.172982in}}%
\pgfusepath{clip}%
\pgfsetbuttcap%
\pgfsetroundjoin%
\definecolor{currentfill}{rgb}{0.240196,0.240196,0.334186}%
\pgfsetfillcolor{currentfill}%
\pgfsetlinewidth{0.000000pt}%
\definecolor{currentstroke}{rgb}{0.000000,0.000000,0.000000}%
\pgfsetstrokecolor{currentstroke}%
\pgfsetdash{}{0pt}%
\pgfpathmoveto{\pgfqpoint{2.144115in}{5.013411in}}%
\pgfpathlineto{\pgfqpoint{2.099978in}{5.013411in}}%
\pgfpathlineto{\pgfqpoint{2.099978in}{5.002736in}}%
\pgfpathlineto{\pgfqpoint{2.144115in}{5.013411in}}%
\pgfpathclose%
\pgfusepath{fill}%
\end{pgfscope}%
\begin{pgfscope}%
\pgfpathrectangle{\pgfqpoint{2.099978in}{0.840429in}}{\pgfqpoint{4.172982in}{4.172982in}}%
\pgfusepath{clip}%
\pgfsetbuttcap%
\pgfsetroundjoin%
\definecolor{currentfill}{rgb}{0.284804,0.284804,0.396249}%
\pgfsetfillcolor{currentfill}%
\pgfsetlinewidth{0.000000pt}%
\definecolor{currentstroke}{rgb}{0.000000,0.000000,0.000000}%
\pgfsetstrokecolor{currentstroke}%
\pgfsetdash{}{0pt}%
\pgfpathmoveto{\pgfqpoint{6.272961in}{0.840429in}}%
\pgfpathlineto{\pgfqpoint{6.272961in}{1.822636in}}%
\pgfpathlineto{\pgfqpoint{6.047772in}{0.840429in}}%
\pgfpathlineto{\pgfqpoint{6.272961in}{0.840429in}}%
\pgfpathclose%
\pgfusepath{fill}%
\end{pgfscope}%
\begin{pgfscope}%
\pgfpathrectangle{\pgfqpoint{2.099978in}{0.840429in}}{\pgfqpoint{4.172982in}{4.172982in}}%
\pgfusepath{clip}%
\pgfsetbuttcap%
\pgfsetroundjoin%
\definecolor{currentfill}{rgb}{0.284804,0.284804,0.396249}%
\pgfsetfillcolor{currentfill}%
\pgfsetlinewidth{0.000000pt}%
\definecolor{currentstroke}{rgb}{0.000000,0.000000,0.000000}%
\pgfsetstrokecolor{currentstroke}%
\pgfsetdash{}{0pt}%
\pgfpathmoveto{\pgfqpoint{5.021066in}{2.678678in}}%
\pgfpathlineto{\pgfqpoint{5.438364in}{2.349974in}}%
\pgfpathlineto{\pgfqpoint{5.855662in}{2.111367in}}%
\pgfpathlineto{\pgfqpoint{6.272961in}{1.891535in}}%
\pgfpathlineto{\pgfqpoint{6.272961in}{2.120906in}}%
\pgfpathlineto{\pgfqpoint{5.855662in}{2.377146in}}%
\pgfpathlineto{\pgfqpoint{5.438364in}{2.631861in}}%
\pgfpathlineto{\pgfqpoint{5.104526in}{2.926920in}}%
\pgfpathlineto{\pgfqpoint{5.033906in}{3.970166in}}%
\pgfpathlineto{\pgfqpoint{5.021066in}{4.023665in}}%
\pgfpathlineto{\pgfqpoint{4.630180in}{5.013411in}}%
\pgfpathlineto{\pgfqpoint{4.603768in}{5.013411in}}%
\pgfpathlineto{\pgfqpoint{4.186470in}{5.013411in}}%
\pgfpathlineto{\pgfqpoint{4.041839in}{5.013411in}}%
\pgfpathlineto{\pgfqpoint{4.186470in}{4.779449in}}%
\pgfpathlineto{\pgfqpoint{4.603768in}{4.100572in}}%
\pgfpathlineto{\pgfqpoint{4.655331in}{3.970166in}}%
\pgfpathlineto{\pgfqpoint{4.908167in}{2.926920in}}%
\pgfpathlineto{\pgfqpoint{5.021066in}{2.678678in}}%
\pgfpathclose%
\pgfusepath{fill}%
\end{pgfscope}%
\begin{pgfscope}%
\pgfpathrectangle{\pgfqpoint{2.099978in}{0.840429in}}{\pgfqpoint{4.172982in}{4.172982in}}%
\pgfusepath{clip}%
\pgfsetbuttcap%
\pgfsetroundjoin%
\definecolor{currentfill}{rgb}{0.284804,0.284804,0.396249}%
\pgfsetfillcolor{currentfill}%
\pgfsetlinewidth{0.000000pt}%
\definecolor{currentstroke}{rgb}{0.000000,0.000000,0.000000}%
\pgfsetstrokecolor{currentstroke}%
\pgfsetdash{}{0pt}%
\pgfpathmoveto{\pgfqpoint{2.466117in}{5.013411in}}%
\pgfpathlineto{\pgfqpoint{2.144115in}{5.013411in}}%
\pgfpathlineto{\pgfqpoint{2.099978in}{5.002736in}}%
\pgfpathlineto{\pgfqpoint{2.099978in}{4.924857in}}%
\pgfpathlineto{\pgfqpoint{2.466117in}{5.013411in}}%
\pgfpathclose%
\pgfusepath{fill}%
\end{pgfscope}%
\begin{pgfscope}%
\pgfpathrectangle{\pgfqpoint{2.099978in}{0.840429in}}{\pgfqpoint{4.172982in}{4.172982in}}%
\pgfusepath{clip}%
\pgfsetbuttcap%
\pgfsetroundjoin%
\definecolor{currentfill}{rgb}{0.329412,0.333150,0.454412}%
\pgfsetfillcolor{currentfill}%
\pgfsetlinewidth{0.000000pt}%
\definecolor{currentstroke}{rgb}{0.000000,0.000000,0.000000}%
\pgfsetstrokecolor{currentstroke}%
\pgfsetdash{}{0pt}%
\pgfpathmoveto{\pgfqpoint{5.855662in}{0.840429in}}%
\pgfpathlineto{\pgfqpoint{6.047772in}{0.840429in}}%
\pgfpathlineto{\pgfqpoint{6.272961in}{1.822636in}}%
\pgfpathlineto{\pgfqpoint{6.272961in}{1.883675in}}%
\pgfpathlineto{\pgfqpoint{6.272961in}{1.891535in}}%
\pgfpathlineto{\pgfqpoint{5.855662in}{2.111367in}}%
\pgfpathlineto{\pgfqpoint{5.438364in}{2.349974in}}%
\pgfpathlineto{\pgfqpoint{5.021066in}{2.678678in}}%
\pgfpathlineto{\pgfqpoint{4.908167in}{2.926920in}}%
\pgfpathlineto{\pgfqpoint{4.655331in}{3.970166in}}%
\pgfpathlineto{\pgfqpoint{4.603768in}{4.100572in}}%
\pgfpathlineto{\pgfqpoint{4.186470in}{4.779449in}}%
\pgfpathlineto{\pgfqpoint{4.041839in}{5.013411in}}%
\pgfpathlineto{\pgfqpoint{3.769171in}{5.013411in}}%
\pgfpathlineto{\pgfqpoint{3.351873in}{5.013411in}}%
\pgfpathlineto{\pgfqpoint{2.934575in}{5.013411in}}%
\pgfpathlineto{\pgfqpoint{2.517277in}{5.013411in}}%
\pgfpathlineto{\pgfqpoint{2.466117in}{5.013411in}}%
\pgfpathlineto{\pgfqpoint{2.099978in}{4.924857in}}%
\pgfpathlineto{\pgfqpoint{2.099978in}{4.846977in}}%
\pgfpathlineto{\pgfqpoint{2.500647in}{3.970166in}}%
\pgfpathlineto{\pgfqpoint{2.517277in}{3.783872in}}%
\pgfpathlineto{\pgfqpoint{2.934575in}{3.764547in}}%
\pgfpathlineto{\pgfqpoint{3.351873in}{3.848001in}}%
\pgfpathlineto{\pgfqpoint{3.650621in}{3.970166in}}%
\pgfpathlineto{\pgfqpoint{3.769171in}{4.173925in}}%
\pgfpathlineto{\pgfqpoint{3.899577in}{3.970166in}}%
\pgfpathlineto{\pgfqpoint{4.186470in}{3.807158in}}%
\pgfpathlineto{\pgfqpoint{4.603768in}{3.483551in}}%
\pgfpathlineto{\pgfqpoint{4.755253in}{2.926920in}}%
\pgfpathlineto{\pgfqpoint{5.021066in}{2.342452in}}%
\pgfpathlineto{\pgfqpoint{5.438364in}{2.068087in}}%
\pgfpathlineto{\pgfqpoint{5.780679in}{1.883675in}}%
\pgfpathlineto{\pgfqpoint{5.691448in}{0.840429in}}%
\pgfpathlineto{\pgfqpoint{5.855662in}{0.840429in}}%
\pgfpathclose%
\pgfusepath{fill}%
\end{pgfscope}%
\begin{pgfscope}%
\pgfpathrectangle{\pgfqpoint{2.099978in}{0.840429in}}{\pgfqpoint{4.172982in}{4.172982in}}%
\pgfusepath{clip}%
\pgfsetbuttcap%
\pgfsetroundjoin%
\definecolor{currentfill}{rgb}{0.370588,0.389767,0.495588}%
\pgfsetfillcolor{currentfill}%
\pgfsetlinewidth{0.000000pt}%
\definecolor{currentstroke}{rgb}{0.000000,0.000000,0.000000}%
\pgfsetstrokecolor{currentstroke}%
\pgfsetdash{}{0pt}%
\pgfpathmoveto{\pgfqpoint{5.438364in}{0.840429in}}%
\pgfpathlineto{\pgfqpoint{5.691448in}{0.840429in}}%
\pgfpathlineto{\pgfqpoint{5.780679in}{1.883675in}}%
\pgfpathlineto{\pgfqpoint{5.438364in}{2.068087in}}%
\pgfpathlineto{\pgfqpoint{5.021066in}{2.342452in}}%
\pgfpathlineto{\pgfqpoint{4.755253in}{2.926920in}}%
\pgfpathlineto{\pgfqpoint{4.603768in}{3.483551in}}%
\pgfpathlineto{\pgfqpoint{4.186470in}{3.807158in}}%
\pgfpathlineto{\pgfqpoint{3.899577in}{3.970166in}}%
\pgfpathlineto{\pgfqpoint{3.769171in}{4.173925in}}%
\pgfpathlineto{\pgfqpoint{3.650621in}{3.970166in}}%
\pgfpathlineto{\pgfqpoint{3.351873in}{3.848001in}}%
\pgfpathlineto{\pgfqpoint{2.934575in}{3.764547in}}%
\pgfpathlineto{\pgfqpoint{2.517277in}{3.783872in}}%
\pgfpathlineto{\pgfqpoint{2.500647in}{3.970166in}}%
\pgfpathlineto{\pgfqpoint{2.099978in}{4.846977in}}%
\pgfpathlineto{\pgfqpoint{2.099978in}{4.769098in}}%
\pgfpathlineto{\pgfqpoint{2.465059in}{3.970166in}}%
\pgfpathlineto{\pgfqpoint{2.517277in}{3.385203in}}%
\pgfpathlineto{\pgfqpoint{2.934575in}{3.417159in}}%
\pgfpathlineto{\pgfqpoint{3.351873in}{3.536773in}}%
\pgfpathlineto{\pgfqpoint{3.769171in}{3.596198in}}%
\pgfpathlineto{\pgfqpoint{4.186470in}{3.331474in}}%
\pgfpathlineto{\pgfqpoint{4.599232in}{2.926920in}}%
\pgfpathlineto{\pgfqpoint{4.603768in}{2.923247in}}%
\pgfpathlineto{\pgfqpoint{5.021066in}{2.006225in}}%
\pgfpathlineto{\pgfqpoint{5.235206in}{1.883675in}}%
\pgfpathlineto{\pgfqpoint{5.383408in}{0.840429in}}%
\pgfpathlineto{\pgfqpoint{5.438364in}{0.840429in}}%
\pgfpathclose%
\pgfusepath{fill}%
\end{pgfscope}%
\begin{pgfscope}%
\pgfpathrectangle{\pgfqpoint{2.099978in}{0.840429in}}{\pgfqpoint{4.172982in}{4.172982in}}%
\pgfusepath{clip}%
\pgfsetbuttcap%
\pgfsetroundjoin%
\definecolor{currentfill}{rgb}{0.415196,0.451103,0.540196}%
\pgfsetfillcolor{currentfill}%
\pgfsetlinewidth{0.000000pt}%
\definecolor{currentstroke}{rgb}{0.000000,0.000000,0.000000}%
\pgfsetstrokecolor{currentstroke}%
\pgfsetdash{}{0pt}%
\pgfpathmoveto{\pgfqpoint{2.143147in}{1.883675in}}%
\pgfpathlineto{\pgfqpoint{2.159595in}{2.926920in}}%
\pgfpathlineto{\pgfqpoint{2.099978in}{2.929721in}}%
\pgfpathlineto{\pgfqpoint{2.099978in}{2.926920in}}%
\pgfpathlineto{\pgfqpoint{2.099978in}{1.883675in}}%
\pgfpathlineto{\pgfqpoint{2.099978in}{1.861104in}}%
\pgfpathlineto{\pgfqpoint{2.143147in}{1.883675in}}%
\pgfpathclose%
\pgfusepath{fill}%
\end{pgfscope}%
\begin{pgfscope}%
\pgfpathrectangle{\pgfqpoint{2.099978in}{0.840429in}}{\pgfqpoint{4.172982in}{4.172982in}}%
\pgfusepath{clip}%
\pgfsetbuttcap%
\pgfsetroundjoin%
\definecolor{currentfill}{rgb}{0.415196,0.451103,0.540196}%
\pgfsetfillcolor{currentfill}%
\pgfsetlinewidth{0.000000pt}%
\definecolor{currentstroke}{rgb}{0.000000,0.000000,0.000000}%
\pgfsetstrokecolor{currentstroke}%
\pgfsetdash{}{0pt}%
\pgfpathmoveto{\pgfqpoint{5.021066in}{1.101240in}}%
\pgfpathlineto{\pgfqpoint{5.084406in}{0.840429in}}%
\pgfpathlineto{\pgfqpoint{5.383408in}{0.840429in}}%
\pgfpathlineto{\pgfqpoint{5.235206in}{1.883675in}}%
\pgfpathlineto{\pgfqpoint{5.021066in}{2.006225in}}%
\pgfpathlineto{\pgfqpoint{4.603768in}{2.923247in}}%
\pgfpathlineto{\pgfqpoint{4.599232in}{2.926920in}}%
\pgfpathlineto{\pgfqpoint{4.186470in}{3.331474in}}%
\pgfpathlineto{\pgfqpoint{3.769171in}{3.596198in}}%
\pgfpathlineto{\pgfqpoint{3.351873in}{3.536773in}}%
\pgfpathlineto{\pgfqpoint{2.934575in}{3.417159in}}%
\pgfpathlineto{\pgfqpoint{2.517277in}{3.385203in}}%
\pgfpathlineto{\pgfqpoint{2.465059in}{3.970166in}}%
\pgfpathlineto{\pgfqpoint{2.099978in}{4.769098in}}%
\pgfpathlineto{\pgfqpoint{2.099978in}{4.691218in}}%
\pgfpathlineto{\pgfqpoint{2.429471in}{3.970166in}}%
\pgfpathlineto{\pgfqpoint{2.517277in}{2.986535in}}%
\pgfpathlineto{\pgfqpoint{2.934575in}{3.069771in}}%
\pgfpathlineto{\pgfqpoint{3.351873in}{3.225544in}}%
\pgfpathlineto{\pgfqpoint{3.769171in}{3.153233in}}%
\pgfpathlineto{\pgfqpoint{4.091987in}{2.926920in}}%
\pgfpathlineto{\pgfqpoint{4.186470in}{2.864011in}}%
\pgfpathlineto{\pgfqpoint{4.603768in}{2.530193in}}%
\pgfpathlineto{\pgfqpoint{4.904770in}{1.883675in}}%
\pgfpathlineto{\pgfqpoint{5.021066in}{1.101240in}}%
\pgfpathclose%
\pgfusepath{fill}%
\end{pgfscope}%
\begin{pgfscope}%
\pgfpathrectangle{\pgfqpoint{2.099978in}{0.840429in}}{\pgfqpoint{4.172982in}{4.172982in}}%
\pgfusepath{clip}%
\pgfsetbuttcap%
\pgfsetroundjoin%
\definecolor{currentfill}{rgb}{0.459804,0.512439,0.584804}%
\pgfsetfillcolor{currentfill}%
\pgfsetlinewidth{0.000000pt}%
\definecolor{currentstroke}{rgb}{0.000000,0.000000,0.000000}%
\pgfsetstrokecolor{currentstroke}%
\pgfsetdash{}{0pt}%
\pgfpathmoveto{\pgfqpoint{2.517277in}{1.877261in}}%
\pgfpathlineto{\pgfqpoint{2.530317in}{1.883675in}}%
\pgfpathlineto{\pgfqpoint{2.934575in}{2.505609in}}%
\pgfpathlineto{\pgfqpoint{3.351873in}{2.907092in}}%
\pgfpathlineto{\pgfqpoint{3.769171in}{2.558025in}}%
\pgfpathlineto{\pgfqpoint{4.186470in}{2.443305in}}%
\pgfpathlineto{\pgfqpoint{4.603768in}{2.137139in}}%
\pgfpathlineto{\pgfqpoint{4.721774in}{1.883675in}}%
\pgfpathlineto{\pgfqpoint{4.764812in}{0.840429in}}%
\pgfpathlineto{\pgfqpoint{5.021066in}{0.840429in}}%
\pgfpathlineto{\pgfqpoint{5.084406in}{0.840429in}}%
\pgfpathlineto{\pgfqpoint{5.021066in}{1.101240in}}%
\pgfpathlineto{\pgfqpoint{4.904770in}{1.883675in}}%
\pgfpathlineto{\pgfqpoint{4.603768in}{2.530193in}}%
\pgfpathlineto{\pgfqpoint{4.186470in}{2.864011in}}%
\pgfpathlineto{\pgfqpoint{4.091987in}{2.926920in}}%
\pgfpathlineto{\pgfqpoint{3.769171in}{3.153233in}}%
\pgfpathlineto{\pgfqpoint{3.351873in}{3.225544in}}%
\pgfpathlineto{\pgfqpoint{2.934575in}{3.069771in}}%
\pgfpathlineto{\pgfqpoint{2.517277in}{2.986535in}}%
\pgfpathlineto{\pgfqpoint{2.429471in}{3.970166in}}%
\pgfpathlineto{\pgfqpoint{2.099978in}{4.691218in}}%
\pgfpathlineto{\pgfqpoint{2.099978in}{4.613339in}}%
\pgfpathlineto{\pgfqpoint{2.393883in}{3.970166in}}%
\pgfpathlineto{\pgfqpoint{2.099978in}{3.042098in}}%
\pgfpathlineto{\pgfqpoint{2.099978in}{2.929721in}}%
\pgfpathlineto{\pgfqpoint{2.159595in}{2.926920in}}%
\pgfpathlineto{\pgfqpoint{2.143147in}{1.883675in}}%
\pgfpathlineto{\pgfqpoint{2.099978in}{1.861104in}}%
\pgfpathlineto{\pgfqpoint{2.099978in}{1.659853in}}%
\pgfpathlineto{\pgfqpoint{2.517277in}{1.877261in}}%
\pgfpathclose%
\pgfusepath{fill}%
\end{pgfscope}%
\begin{pgfscope}%
\pgfpathrectangle{\pgfqpoint{2.099978in}{0.840429in}}{\pgfqpoint{4.172982in}{4.172982in}}%
\pgfusepath{clip}%
\pgfsetbuttcap%
\pgfsetroundjoin%
\definecolor{currentfill}{rgb}{0.504412,0.573774,0.629412}%
\pgfsetfillcolor{currentfill}%
\pgfsetlinewidth{0.000000pt}%
\definecolor{currentstroke}{rgb}{0.000000,0.000000,0.000000}%
\pgfsetstrokecolor{currentstroke}%
\pgfsetdash{}{0pt}%
\pgfpathmoveto{\pgfqpoint{2.517277in}{1.648517in}}%
\pgfpathlineto{\pgfqpoint{2.934575in}{1.845640in}}%
\pgfpathlineto{\pgfqpoint{2.979285in}{1.883675in}}%
\pgfpathlineto{\pgfqpoint{3.351873in}{2.417499in}}%
\pgfpathlineto{\pgfqpoint{3.732223in}{1.883675in}}%
\pgfpathlineto{\pgfqpoint{3.769171in}{1.833856in}}%
\pgfpathlineto{\pgfqpoint{3.870516in}{1.883675in}}%
\pgfpathlineto{\pgfqpoint{4.186470in}{2.022599in}}%
\pgfpathlineto{\pgfqpoint{4.387531in}{1.883675in}}%
\pgfpathlineto{\pgfqpoint{4.544468in}{0.840429in}}%
\pgfpathlineto{\pgfqpoint{4.603768in}{0.840429in}}%
\pgfpathlineto{\pgfqpoint{4.764812in}{0.840429in}}%
\pgfpathlineto{\pgfqpoint{4.721774in}{1.883675in}}%
\pgfpathlineto{\pgfqpoint{4.603768in}{2.137139in}}%
\pgfpathlineto{\pgfqpoint{4.186470in}{2.443305in}}%
\pgfpathlineto{\pgfqpoint{3.769171in}{2.558025in}}%
\pgfpathlineto{\pgfqpoint{3.351873in}{2.907092in}}%
\pgfpathlineto{\pgfqpoint{2.934575in}{2.505609in}}%
\pgfpathlineto{\pgfqpoint{2.530317in}{1.883675in}}%
\pgfpathlineto{\pgfqpoint{2.517277in}{1.877261in}}%
\pgfpathlineto{\pgfqpoint{2.099978in}{1.659853in}}%
\pgfpathlineto{\pgfqpoint{2.099978in}{1.458602in}}%
\pgfpathlineto{\pgfqpoint{2.517277in}{1.648517in}}%
\pgfpathclose%
\pgfusepath{fill}%
\end{pgfscope}%
\begin{pgfscope}%
\pgfpathrectangle{\pgfqpoint{2.099978in}{0.840429in}}{\pgfqpoint{4.172982in}{4.172982in}}%
\pgfusepath{clip}%
\pgfsetbuttcap%
\pgfsetroundjoin%
\definecolor{currentfill}{rgb}{0.504412,0.573774,0.629412}%
\pgfsetfillcolor{currentfill}%
\pgfsetlinewidth{0.000000pt}%
\definecolor{currentstroke}{rgb}{0.000000,0.000000,0.000000}%
\pgfsetstrokecolor{currentstroke}%
\pgfsetdash{}{0pt}%
\pgfpathmoveto{\pgfqpoint{3.351873in}{0.840429in}}%
\pgfpathlineto{\pgfqpoint{3.458943in}{0.840429in}}%
\pgfpathlineto{\pgfqpoint{3.351873in}{1.258442in}}%
\pgfpathlineto{\pgfqpoint{3.279218in}{0.840429in}}%
\pgfpathlineto{\pgfqpoint{3.351873in}{0.840429in}}%
\pgfpathclose%
\pgfusepath{fill}%
\end{pgfscope}%
\begin{pgfscope}%
\pgfpathrectangle{\pgfqpoint{2.099978in}{0.840429in}}{\pgfqpoint{4.172982in}{4.172982in}}%
\pgfusepath{clip}%
\pgfsetbuttcap%
\pgfsetroundjoin%
\definecolor{currentfill}{rgb}{0.504412,0.573774,0.629412}%
\pgfsetfillcolor{currentfill}%
\pgfsetlinewidth{0.000000pt}%
\definecolor{currentstroke}{rgb}{0.000000,0.000000,0.000000}%
\pgfsetstrokecolor{currentstroke}%
\pgfsetdash{}{0pt}%
\pgfpathmoveto{\pgfqpoint{2.393883in}{3.970166in}}%
\pgfpathlineto{\pgfqpoint{2.099978in}{4.613339in}}%
\pgfpathlineto{\pgfqpoint{2.099978in}{4.535459in}}%
\pgfpathlineto{\pgfqpoint{2.358295in}{3.970166in}}%
\pgfpathlineto{\pgfqpoint{2.099978in}{3.154474in}}%
\pgfpathlineto{\pgfqpoint{2.099978in}{3.042098in}}%
\pgfpathlineto{\pgfqpoint{2.393883in}{3.970166in}}%
\pgfpathclose%
\pgfusepath{fill}%
\end{pgfscope}%
\begin{pgfscope}%
\pgfpathrectangle{\pgfqpoint{2.099978in}{0.840429in}}{\pgfqpoint{4.172982in}{4.172982in}}%
\pgfusepath{clip}%
\pgfsetbuttcap%
\pgfsetroundjoin%
\definecolor{currentfill}{rgb}{0.549020,0.635110,0.674020}%
\pgfsetfillcolor{currentfill}%
\pgfsetlinewidth{0.000000pt}%
\definecolor{currentstroke}{rgb}{0.000000,0.000000,0.000000}%
\pgfsetstrokecolor{currentstroke}%
\pgfsetdash{}{0pt}%
\pgfpathmoveto{\pgfqpoint{2.517277in}{1.419773in}}%
\pgfpathlineto{\pgfqpoint{2.934575in}{1.554944in}}%
\pgfpathlineto{\pgfqpoint{3.179551in}{0.840429in}}%
\pgfpathlineto{\pgfqpoint{3.279218in}{0.840429in}}%
\pgfpathlineto{\pgfqpoint{3.351873in}{1.258442in}}%
\pgfpathlineto{\pgfqpoint{3.458943in}{0.840429in}}%
\pgfpathlineto{\pgfqpoint{3.605821in}{0.840429in}}%
\pgfpathlineto{\pgfqpoint{3.769171in}{1.363517in}}%
\pgfpathlineto{\pgfqpoint{4.186470in}{1.626452in}}%
\pgfpathlineto{\pgfqpoint{4.426965in}{0.840429in}}%
\pgfpathlineto{\pgfqpoint{4.544468in}{0.840429in}}%
\pgfpathlineto{\pgfqpoint{4.387531in}{1.883675in}}%
\pgfpathlineto{\pgfqpoint{4.186470in}{2.022599in}}%
\pgfpathlineto{\pgfqpoint{3.870516in}{1.883675in}}%
\pgfpathlineto{\pgfqpoint{3.769171in}{1.833856in}}%
\pgfpathlineto{\pgfqpoint{3.732223in}{1.883675in}}%
\pgfpathlineto{\pgfqpoint{3.351873in}{2.417499in}}%
\pgfpathlineto{\pgfqpoint{2.979285in}{1.883675in}}%
\pgfpathlineto{\pgfqpoint{2.934575in}{1.845640in}}%
\pgfpathlineto{\pgfqpoint{2.517277in}{1.648517in}}%
\pgfpathlineto{\pgfqpoint{2.099978in}{1.458602in}}%
\pgfpathlineto{\pgfqpoint{2.099978in}{1.257351in}}%
\pgfpathlineto{\pgfqpoint{2.517277in}{1.419773in}}%
\pgfpathclose%
\pgfpathmoveto{\pgfqpoint{3.321001in}{1.883675in}}%
\pgfpathlineto{\pgfqpoint{3.351873in}{1.927906in}}%
\pgfpathlineto{\pgfqpoint{3.383388in}{1.883675in}}%
\pgfpathlineto{\pgfqpoint{3.351873in}{1.831869in}}%
\pgfpathlineto{\pgfqpoint{3.321001in}{1.883675in}}%
\pgfpathclose%
\pgfusepath{fill}%
\end{pgfscope}%
\begin{pgfscope}%
\pgfpathrectangle{\pgfqpoint{2.099978in}{0.840429in}}{\pgfqpoint{4.172982in}{4.172982in}}%
\pgfusepath{clip}%
\pgfsetbuttcap%
\pgfsetroundjoin%
\definecolor{currentfill}{rgb}{0.549020,0.635110,0.674020}%
\pgfsetfillcolor{currentfill}%
\pgfsetlinewidth{0.000000pt}%
\definecolor{currentstroke}{rgb}{0.000000,0.000000,0.000000}%
\pgfsetstrokecolor{currentstroke}%
\pgfsetdash{}{0pt}%
\pgfpathmoveto{\pgfqpoint{2.358295in}{3.970166in}}%
\pgfpathlineto{\pgfqpoint{2.099978in}{4.535459in}}%
\pgfpathlineto{\pgfqpoint{2.099978in}{4.457580in}}%
\pgfpathlineto{\pgfqpoint{2.322708in}{3.970166in}}%
\pgfpathlineto{\pgfqpoint{2.099978in}{3.266850in}}%
\pgfpathlineto{\pgfqpoint{2.099978in}{3.154474in}}%
\pgfpathlineto{\pgfqpoint{2.358295in}{3.970166in}}%
\pgfpathclose%
\pgfusepath{fill}%
\end{pgfscope}%
\begin{pgfscope}%
\pgfpathrectangle{\pgfqpoint{2.099978in}{0.840429in}}{\pgfqpoint{4.172982in}{4.172982in}}%
\pgfusepath{clip}%
\pgfsetbuttcap%
\pgfsetroundjoin%
\definecolor{currentfill}{rgb}{0.590196,0.691728,0.715196}%
\pgfsetfillcolor{currentfill}%
\pgfsetlinewidth{0.000000pt}%
\definecolor{currentstroke}{rgb}{0.000000,0.000000,0.000000}%
\pgfsetstrokecolor{currentstroke}%
\pgfsetdash{}{0pt}%
\pgfpathmoveto{\pgfqpoint{2.517277in}{1.191028in}}%
\pgfpathlineto{\pgfqpoint{2.934575in}{1.264248in}}%
\pgfpathlineto{\pgfqpoint{3.079884in}{0.840429in}}%
\pgfpathlineto{\pgfqpoint{3.179551in}{0.840429in}}%
\pgfpathlineto{\pgfqpoint{2.934575in}{1.554944in}}%
\pgfpathlineto{\pgfqpoint{2.517277in}{1.419773in}}%
\pgfpathlineto{\pgfqpoint{2.099978in}{1.257351in}}%
\pgfpathlineto{\pgfqpoint{2.099978in}{1.056100in}}%
\pgfpathlineto{\pgfqpoint{2.517277in}{1.191028in}}%
\pgfpathclose%
\pgfusepath{fill}%
\end{pgfscope}%
\begin{pgfscope}%
\pgfpathrectangle{\pgfqpoint{2.099978in}{0.840429in}}{\pgfqpoint{4.172982in}{4.172982in}}%
\pgfusepath{clip}%
\pgfsetbuttcap%
\pgfsetroundjoin%
\definecolor{currentfill}{rgb}{0.590196,0.691728,0.715196}%
\pgfsetfillcolor{currentfill}%
\pgfsetlinewidth{0.000000pt}%
\definecolor{currentstroke}{rgb}{0.000000,0.000000,0.000000}%
\pgfsetstrokecolor{currentstroke}%
\pgfsetdash{}{0pt}%
\pgfpathmoveto{\pgfqpoint{3.351873in}{1.831869in}}%
\pgfpathlineto{\pgfqpoint{3.383388in}{1.883675in}}%
\pgfpathlineto{\pgfqpoint{3.351873in}{1.927906in}}%
\pgfpathlineto{\pgfqpoint{3.321001in}{1.883675in}}%
\pgfpathlineto{\pgfqpoint{3.351873in}{1.831869in}}%
\pgfpathclose%
\pgfusepath{fill}%
\end{pgfscope}%
\begin{pgfscope}%
\pgfpathrectangle{\pgfqpoint{2.099978in}{0.840429in}}{\pgfqpoint{4.172982in}{4.172982in}}%
\pgfusepath{clip}%
\pgfsetbuttcap%
\pgfsetroundjoin%
\definecolor{currentfill}{rgb}{0.590196,0.691728,0.715196}%
\pgfsetfillcolor{currentfill}%
\pgfsetlinewidth{0.000000pt}%
\definecolor{currentstroke}{rgb}{0.000000,0.000000,0.000000}%
\pgfsetstrokecolor{currentstroke}%
\pgfsetdash{}{0pt}%
\pgfpathmoveto{\pgfqpoint{3.769171in}{0.893178in}}%
\pgfpathlineto{\pgfqpoint{4.186470in}{1.242414in}}%
\pgfpathlineto{\pgfqpoint{4.309463in}{0.840429in}}%
\pgfpathlineto{\pgfqpoint{4.426965in}{0.840429in}}%
\pgfpathlineto{\pgfqpoint{4.186470in}{1.626452in}}%
\pgfpathlineto{\pgfqpoint{3.769171in}{1.363517in}}%
\pgfpathlineto{\pgfqpoint{3.605821in}{0.840429in}}%
\pgfpathlineto{\pgfqpoint{3.752699in}{0.840429in}}%
\pgfpathlineto{\pgfqpoint{3.769171in}{0.893178in}}%
\pgfpathclose%
\pgfusepath{fill}%
\end{pgfscope}%
\begin{pgfscope}%
\pgfpathrectangle{\pgfqpoint{2.099978in}{0.840429in}}{\pgfqpoint{4.172982in}{4.172982in}}%
\pgfusepath{clip}%
\pgfsetbuttcap%
\pgfsetroundjoin%
\definecolor{currentfill}{rgb}{0.590196,0.691728,0.715196}%
\pgfsetfillcolor{currentfill}%
\pgfsetlinewidth{0.000000pt}%
\definecolor{currentstroke}{rgb}{0.000000,0.000000,0.000000}%
\pgfsetstrokecolor{currentstroke}%
\pgfsetdash{}{0pt}%
\pgfpathmoveto{\pgfqpoint{2.322708in}{3.970166in}}%
\pgfpathlineto{\pgfqpoint{2.099978in}{4.457580in}}%
\pgfpathlineto{\pgfqpoint{2.099978in}{4.379700in}}%
\pgfpathlineto{\pgfqpoint{2.287120in}{3.970166in}}%
\pgfpathlineto{\pgfqpoint{2.099978in}{3.379227in}}%
\pgfpathlineto{\pgfqpoint{2.099978in}{3.266850in}}%
\pgfpathlineto{\pgfqpoint{2.322708in}{3.970166in}}%
\pgfpathclose%
\pgfusepath{fill}%
\end{pgfscope}%
\begin{pgfscope}%
\pgfpathrectangle{\pgfqpoint{2.099978in}{0.840429in}}{\pgfqpoint{4.172982in}{4.172982in}}%
\pgfusepath{clip}%
\pgfsetbuttcap%
\pgfsetroundjoin%
\definecolor{currentfill}{rgb}{0.634804,0.753064,0.759804}%
\pgfsetfillcolor{currentfill}%
\pgfsetlinewidth{0.000000pt}%
\definecolor{currentstroke}{rgb}{0.000000,0.000000,0.000000}%
\pgfsetstrokecolor{currentstroke}%
\pgfsetdash{}{0pt}%
\pgfpathmoveto{\pgfqpoint{2.517277in}{0.962284in}}%
\pgfpathlineto{\pgfqpoint{2.934575in}{0.973552in}}%
\pgfpathlineto{\pgfqpoint{2.980217in}{0.840429in}}%
\pgfpathlineto{\pgfqpoint{3.079884in}{0.840429in}}%
\pgfpathlineto{\pgfqpoint{2.934575in}{1.264248in}}%
\pgfpathlineto{\pgfqpoint{2.517277in}{1.191028in}}%
\pgfpathlineto{\pgfqpoint{2.099978in}{1.056100in}}%
\pgfpathlineto{\pgfqpoint{2.099978in}{0.854849in}}%
\pgfpathlineto{\pgfqpoint{2.517277in}{0.962284in}}%
\pgfpathclose%
\pgfusepath{fill}%
\end{pgfscope}%
\begin{pgfscope}%
\pgfpathrectangle{\pgfqpoint{2.099978in}{0.840429in}}{\pgfqpoint{4.172982in}{4.172982in}}%
\pgfusepath{clip}%
\pgfsetbuttcap%
\pgfsetroundjoin%
\definecolor{currentfill}{rgb}{0.634804,0.753064,0.759804}%
\pgfsetfillcolor{currentfill}%
\pgfsetlinewidth{0.000000pt}%
\definecolor{currentstroke}{rgb}{0.000000,0.000000,0.000000}%
\pgfsetstrokecolor{currentstroke}%
\pgfsetdash{}{0pt}%
\pgfpathmoveto{\pgfqpoint{3.769171in}{0.840429in}}%
\pgfpathlineto{\pgfqpoint{4.165604in}{0.840429in}}%
\pgfpathlineto{\pgfqpoint{4.186470in}{0.858375in}}%
\pgfpathlineto{\pgfqpoint{4.191960in}{0.840429in}}%
\pgfpathlineto{\pgfqpoint{4.309463in}{0.840429in}}%
\pgfpathlineto{\pgfqpoint{4.186470in}{1.242414in}}%
\pgfpathlineto{\pgfqpoint{3.769171in}{0.893178in}}%
\pgfpathlineto{\pgfqpoint{3.752699in}{0.840429in}}%
\pgfpathlineto{\pgfqpoint{3.769171in}{0.840429in}}%
\pgfpathclose%
\pgfusepath{fill}%
\end{pgfscope}%
\begin{pgfscope}%
\pgfpathrectangle{\pgfqpoint{2.099978in}{0.840429in}}{\pgfqpoint{4.172982in}{4.172982in}}%
\pgfusepath{clip}%
\pgfsetbuttcap%
\pgfsetroundjoin%
\definecolor{currentfill}{rgb}{0.634804,0.753064,0.759804}%
\pgfsetfillcolor{currentfill}%
\pgfsetlinewidth{0.000000pt}%
\definecolor{currentstroke}{rgb}{0.000000,0.000000,0.000000}%
\pgfsetstrokecolor{currentstroke}%
\pgfsetdash{}{0pt}%
\pgfpathmoveto{\pgfqpoint{2.287120in}{3.970166in}}%
\pgfpathlineto{\pgfqpoint{2.099978in}{4.379700in}}%
\pgfpathlineto{\pgfqpoint{2.099978in}{4.301821in}}%
\pgfpathlineto{\pgfqpoint{2.251532in}{3.970166in}}%
\pgfpathlineto{\pgfqpoint{2.099978in}{3.491603in}}%
\pgfpathlineto{\pgfqpoint{2.099978in}{3.379227in}}%
\pgfpathlineto{\pgfqpoint{2.287120in}{3.970166in}}%
\pgfpathclose%
\pgfusepath{fill}%
\end{pgfscope}%
\begin{pgfscope}%
\pgfpathrectangle{\pgfqpoint{2.099978in}{0.840429in}}{\pgfqpoint{4.172982in}{4.172982in}}%
\pgfusepath{clip}%
\pgfsetbuttcap%
\pgfsetroundjoin%
\definecolor{currentfill}{rgb}{0.694393,0.804412,0.804412}%
\pgfsetfillcolor{currentfill}%
\pgfsetlinewidth{0.000000pt}%
\definecolor{currentstroke}{rgb}{0.000000,0.000000,0.000000}%
\pgfsetstrokecolor{currentstroke}%
\pgfsetdash{}{0pt}%
\pgfpathmoveto{\pgfqpoint{2.517277in}{0.840429in}}%
\pgfpathlineto{\pgfqpoint{2.934575in}{0.840429in}}%
\pgfpathlineto{\pgfqpoint{2.980217in}{0.840429in}}%
\pgfpathlineto{\pgfqpoint{2.934575in}{0.973552in}}%
\pgfpathlineto{\pgfqpoint{2.517277in}{0.962284in}}%
\pgfpathlineto{\pgfqpoint{2.099978in}{0.854849in}}%
\pgfpathlineto{\pgfqpoint{2.099978in}{0.840429in}}%
\pgfpathlineto{\pgfqpoint{2.517277in}{0.840429in}}%
\pgfpathclose%
\pgfusepath{fill}%
\end{pgfscope}%
\begin{pgfscope}%
\pgfpathrectangle{\pgfqpoint{2.099978in}{0.840429in}}{\pgfqpoint{4.172982in}{4.172982in}}%
\pgfusepath{clip}%
\pgfsetbuttcap%
\pgfsetroundjoin%
\definecolor{currentfill}{rgb}{0.694393,0.804412,0.804412}%
\pgfsetfillcolor{currentfill}%
\pgfsetlinewidth{0.000000pt}%
\definecolor{currentstroke}{rgb}{0.000000,0.000000,0.000000}%
\pgfsetstrokecolor{currentstroke}%
\pgfsetdash{}{0pt}%
\pgfpathmoveto{\pgfqpoint{4.186470in}{0.840429in}}%
\pgfpathlineto{\pgfqpoint{4.191960in}{0.840429in}}%
\pgfpathlineto{\pgfqpoint{4.186470in}{0.858375in}}%
\pgfpathlineto{\pgfqpoint{4.165604in}{0.840429in}}%
\pgfpathlineto{\pgfqpoint{4.186470in}{0.840429in}}%
\pgfpathclose%
\pgfusepath{fill}%
\end{pgfscope}%
\begin{pgfscope}%
\pgfpathrectangle{\pgfqpoint{2.099978in}{0.840429in}}{\pgfqpoint{4.172982in}{4.172982in}}%
\pgfusepath{clip}%
\pgfsetbuttcap%
\pgfsetroundjoin%
\definecolor{currentfill}{rgb}{0.694393,0.804412,0.804412}%
\pgfsetfillcolor{currentfill}%
\pgfsetlinewidth{0.000000pt}%
\definecolor{currentstroke}{rgb}{0.000000,0.000000,0.000000}%
\pgfsetstrokecolor{currentstroke}%
\pgfsetdash{}{0pt}%
\pgfpathmoveto{\pgfqpoint{2.251532in}{3.970166in}}%
\pgfpathlineto{\pgfqpoint{2.099978in}{4.301821in}}%
\pgfpathlineto{\pgfqpoint{2.099978in}{4.223941in}}%
\pgfpathlineto{\pgfqpoint{2.215944in}{3.970166in}}%
\pgfpathlineto{\pgfqpoint{2.099978in}{3.603980in}}%
\pgfpathlineto{\pgfqpoint{2.099978in}{3.491603in}}%
\pgfpathlineto{\pgfqpoint{2.251532in}{3.970166in}}%
\pgfpathclose%
\pgfusepath{fill}%
\end{pgfscope}%
\begin{pgfscope}%
\pgfpathrectangle{\pgfqpoint{2.099978in}{0.840429in}}{\pgfqpoint{4.172982in}{4.172982in}}%
\pgfusepath{clip}%
\pgfsetbuttcap%
\pgfsetroundjoin%
\definecolor{currentfill}{rgb}{0.764093,0.849020,0.849020}%
\pgfsetfillcolor{currentfill}%
\pgfsetlinewidth{0.000000pt}%
\definecolor{currentstroke}{rgb}{0.000000,0.000000,0.000000}%
\pgfsetstrokecolor{currentstroke}%
\pgfsetdash{}{0pt}%
\pgfpathmoveto{\pgfqpoint{2.215944in}{3.970166in}}%
\pgfpathlineto{\pgfqpoint{2.099978in}{4.223941in}}%
\pgfpathlineto{\pgfqpoint{2.099978in}{4.146062in}}%
\pgfpathlineto{\pgfqpoint{2.180356in}{3.970166in}}%
\pgfpathlineto{\pgfqpoint{2.099978in}{3.716356in}}%
\pgfpathlineto{\pgfqpoint{2.099978in}{3.603980in}}%
\pgfpathlineto{\pgfqpoint{2.215944in}{3.970166in}}%
\pgfpathclose%
\pgfusepath{fill}%
\end{pgfscope}%
\begin{pgfscope}%
\pgfpathrectangle{\pgfqpoint{2.099978in}{0.840429in}}{\pgfqpoint{4.172982in}{4.172982in}}%
\pgfusepath{clip}%
\pgfsetbuttcap%
\pgfsetroundjoin%
\definecolor{currentfill}{rgb}{0.833793,0.893627,0.893627}%
\pgfsetfillcolor{currentfill}%
\pgfsetlinewidth{0.000000pt}%
\definecolor{currentstroke}{rgb}{0.000000,0.000000,0.000000}%
\pgfsetstrokecolor{currentstroke}%
\pgfsetdash{}{0pt}%
\pgfpathmoveto{\pgfqpoint{2.180356in}{3.970166in}}%
\pgfpathlineto{\pgfqpoint{2.099978in}{4.146062in}}%
\pgfpathlineto{\pgfqpoint{2.099978in}{4.068182in}}%
\pgfpathlineto{\pgfqpoint{2.144768in}{3.970166in}}%
\pgfpathlineto{\pgfqpoint{2.099978in}{3.828733in}}%
\pgfpathlineto{\pgfqpoint{2.099978in}{3.716356in}}%
\pgfpathlineto{\pgfqpoint{2.180356in}{3.970166in}}%
\pgfpathclose%
\pgfusepath{fill}%
\end{pgfscope}%
\begin{pgfscope}%
\pgfpathrectangle{\pgfqpoint{2.099978in}{0.840429in}}{\pgfqpoint{4.172982in}{4.172982in}}%
\pgfusepath{clip}%
\pgfsetbuttcap%
\pgfsetroundjoin%
\definecolor{currentfill}{rgb}{0.898131,0.934804,0.934804}%
\pgfsetfillcolor{currentfill}%
\pgfsetlinewidth{0.000000pt}%
\definecolor{currentstroke}{rgb}{0.000000,0.000000,0.000000}%
\pgfsetstrokecolor{currentstroke}%
\pgfsetdash{}{0pt}%
\pgfpathmoveto{\pgfqpoint{2.144768in}{3.970166in}}%
\pgfpathlineto{\pgfqpoint{2.099978in}{4.068182in}}%
\pgfpathlineto{\pgfqpoint{2.099978in}{3.990303in}}%
\pgfpathlineto{\pgfqpoint{2.109180in}{3.970166in}}%
\pgfpathlineto{\pgfqpoint{2.099978in}{3.941109in}}%
\pgfpathlineto{\pgfqpoint{2.099978in}{3.828733in}}%
\pgfpathlineto{\pgfqpoint{2.144768in}{3.970166in}}%
\pgfpathclose%
\pgfusepath{fill}%
\end{pgfscope}%
\begin{pgfscope}%
\pgfpathrectangle{\pgfqpoint{2.099978in}{0.840429in}}{\pgfqpoint{4.172982in}{4.172982in}}%
\pgfusepath{clip}%
\pgfsetbuttcap%
\pgfsetroundjoin%
\definecolor{currentfill}{rgb}{0.967831,0.979412,0.979412}%
\pgfsetfillcolor{currentfill}%
\pgfsetlinewidth{0.000000pt}%
\definecolor{currentstroke}{rgb}{0.000000,0.000000,0.000000}%
\pgfsetstrokecolor{currentstroke}%
\pgfsetdash{}{0pt}%
\pgfpathmoveto{\pgfqpoint{2.109180in}{3.970166in}}%
\pgfpathlineto{\pgfqpoint{2.099978in}{3.990303in}}%
\pgfpathlineto{\pgfqpoint{2.099978in}{3.970166in}}%
\pgfpathlineto{\pgfqpoint{2.099978in}{3.941109in}}%
\pgfpathlineto{\pgfqpoint{2.109180in}{3.970166in}}%
\pgfpathclose%
\pgfusepath{fill}%
\end{pgfscope}%
\begin{pgfscope}%
\pgfsetrectcap%
\pgfsetmiterjoin%
\pgfsetlinewidth{0.000000pt}%
\definecolor{currentstroke}{rgb}{1.000000,1.000000,1.000000}%
\pgfsetstrokecolor{currentstroke}%
\pgfsetdash{}{0pt}%
\pgfpathmoveto{\pgfqpoint{2.099978in}{0.840429in}}%
\pgfpathlineto{\pgfqpoint{2.099978in}{5.013411in}}%
\pgfusepath{}%
\end{pgfscope}%
\begin{pgfscope}%
\pgfsetrectcap%
\pgfsetmiterjoin%
\pgfsetlinewidth{0.000000pt}%
\definecolor{currentstroke}{rgb}{1.000000,1.000000,1.000000}%
\pgfsetstrokecolor{currentstroke}%
\pgfsetdash{}{0pt}%
\pgfpathmoveto{\pgfqpoint{6.272961in}{0.840429in}}%
\pgfpathlineto{\pgfqpoint{6.272961in}{5.013411in}}%
\pgfusepath{}%
\end{pgfscope}%
\begin{pgfscope}%
\pgfsetrectcap%
\pgfsetmiterjoin%
\pgfsetlinewidth{0.000000pt}%
\definecolor{currentstroke}{rgb}{1.000000,1.000000,1.000000}%
\pgfsetstrokecolor{currentstroke}%
\pgfsetdash{}{0pt}%
\pgfpathmoveto{\pgfqpoint{2.099978in}{0.840429in}}%
\pgfpathlineto{\pgfqpoint{6.272961in}{0.840429in}}%
\pgfusepath{}%
\end{pgfscope}%
\begin{pgfscope}%
\pgfsetrectcap%
\pgfsetmiterjoin%
\pgfsetlinewidth{0.000000pt}%
\definecolor{currentstroke}{rgb}{1.000000,1.000000,1.000000}%
\pgfsetstrokecolor{currentstroke}%
\pgfsetdash{}{0pt}%
\pgfpathmoveto{\pgfqpoint{2.099978in}{5.013411in}}%
\pgfpathlineto{\pgfqpoint{6.272961in}{5.013411in}}%
\pgfusepath{}%
\end{pgfscope}%
\begin{pgfscope}%
\definecolor{textcolor}{rgb}{0.150000,0.150000,0.150000}%
\pgfsetstrokecolor{textcolor}%
\pgfsetfillcolor{textcolor}%
\pgftext[x=4.186470in,y=5.096745in,,base]{\color{textcolor}\rmfamily\fontsize{24.000000}{28.800000}\selectfont Magnetic Field Contour}%
\end{pgfscope}%
\begin{pgfscope}%
\pgfsetbuttcap%
\pgfsetmiterjoin%
\definecolor{currentfill}{rgb}{0.917647,0.917647,0.949020}%
\pgfsetfillcolor{currentfill}%
\pgfsetlinewidth{0.000000pt}%
\definecolor{currentstroke}{rgb}{0.000000,0.000000,0.000000}%
\pgfsetstrokecolor{currentstroke}%
\pgfsetstrokeopacity{0.000000}%
\pgfsetdash{}{0pt}%
\pgfpathmoveto{\pgfqpoint{6.635952in}{0.840429in}}%
\pgfpathlineto{\pgfqpoint{6.844602in}{0.840429in}}%
\pgfpathlineto{\pgfqpoint{6.844602in}{5.013411in}}%
\pgfpathlineto{\pgfqpoint{6.635952in}{5.013411in}}%
\pgfpathlineto{\pgfqpoint{6.635952in}{0.840429in}}%
\pgfpathclose%
\pgfusepath{fill}%
\end{pgfscope}%
\begin{pgfscope}%
\pgfpathrectangle{\pgfqpoint{6.635952in}{0.840429in}}{\pgfqpoint{0.208649in}{4.172982in}}%
\pgfusepath{clip}%
\pgfsetbuttcap%
\pgfsetmiterjoin%
\definecolor{currentfill}{rgb}{0.917647,0.917647,0.949020}%
\pgfsetfillcolor{currentfill}%
\pgfsetlinewidth{0.010037pt}%
\definecolor{currentstroke}{rgb}{0.917647,0.917647,0.949020}%
\pgfsetstrokecolor{currentstroke}%
\pgfsetdash{}{0pt}%
\pgfusepath{stroke,fill}%
\end{pgfscope}%
\begin{pgfscope}%
\definecolor{textcolor}{rgb}{0.150000,0.150000,0.150000}%
\pgfsetstrokecolor{textcolor}%
\pgfsetfillcolor{textcolor}%
\pgftext[x=6.941824in, y=0.787668in, left, base]{\color{textcolor}\rmfamily\fontsize{10.000000}{12.000000}\selectfont 57.658}%
\end{pgfscope}%
\begin{pgfscope}%
\definecolor{textcolor}{rgb}{0.150000,0.150000,0.150000}%
\pgfsetstrokecolor{textcolor}%
\pgfsetfillcolor{textcolor}%
\pgftext[x=6.941824in, y=1.413615in, left, base]{\color{textcolor}\rmfamily\fontsize{10.000000}{12.000000}\selectfont 57.667}%
\end{pgfscope}%
\begin{pgfscope}%
\definecolor{textcolor}{rgb}{0.150000,0.150000,0.150000}%
\pgfsetstrokecolor{textcolor}%
\pgfsetfillcolor{textcolor}%
\pgftext[x=6.941824in, y=2.039562in, left, base]{\color{textcolor}\rmfamily\fontsize{10.000000}{12.000000}\selectfont 57.676}%
\end{pgfscope}%
\begin{pgfscope}%
\definecolor{textcolor}{rgb}{0.150000,0.150000,0.150000}%
\pgfsetstrokecolor{textcolor}%
\pgfsetfillcolor{textcolor}%
\pgftext[x=6.941824in, y=2.665510in, left, base]{\color{textcolor}\rmfamily\fontsize{10.000000}{12.000000}\selectfont 57.685}%
\end{pgfscope}%
\begin{pgfscope}%
\definecolor{textcolor}{rgb}{0.150000,0.150000,0.150000}%
\pgfsetstrokecolor{textcolor}%
\pgfsetfillcolor{textcolor}%
\pgftext[x=6.941824in, y=3.291457in, left, base]{\color{textcolor}\rmfamily\fontsize{10.000000}{12.000000}\selectfont 57.694}%
\end{pgfscope}%
\begin{pgfscope}%
\definecolor{textcolor}{rgb}{0.150000,0.150000,0.150000}%
\pgfsetstrokecolor{textcolor}%
\pgfsetfillcolor{textcolor}%
\pgftext[x=6.941824in, y=3.917404in, left, base]{\color{textcolor}\rmfamily\fontsize{10.000000}{12.000000}\selectfont 57.703}%
\end{pgfscope}%
\begin{pgfscope}%
\definecolor{textcolor}{rgb}{0.150000,0.150000,0.150000}%
\pgfsetstrokecolor{textcolor}%
\pgfsetfillcolor{textcolor}%
\pgftext[x=6.941824in, y=4.543352in, left, base]{\color{textcolor}\rmfamily\fontsize{10.000000}{12.000000}\selectfont 57.712}%
\end{pgfscope}%
\begin{pgfscope}%
\definecolor{textcolor}{rgb}{0.150000,0.150000,0.150000}%
\pgfsetstrokecolor{textcolor}%
\pgfsetfillcolor{textcolor}%
\pgftext[x=7.483355in,y=2.926920in,,top,rotate=90.000000]{\color{textcolor}\rmfamily\fontsize{15.000000}{18.000000}\selectfont Resonant Frequency (MHz)}%
\end{pgfscope}%
\begin{pgfscope}%
\pgfpathrectangle{\pgfqpoint{6.635952in}{0.840429in}}{\pgfqpoint{0.208649in}{4.172982in}}%
\pgfusepath{clip}%
\pgfsetbuttcap%
\pgfsetroundjoin%
\definecolor{currentfill}{rgb}{0.020588,0.020588,0.028645}%
\pgfsetfillcolor{currentfill}%
\pgfsetlinewidth{0.000000pt}%
\definecolor{currentstroke}{rgb}{0.000000,0.000000,0.000000}%
\pgfsetstrokecolor{currentstroke}%
\pgfsetdash{}{0pt}%
\pgfpathmoveto{\pgfqpoint{6.635952in}{0.840429in}}%
\pgfpathlineto{\pgfqpoint{6.844602in}{0.840429in}}%
\pgfpathlineto{\pgfqpoint{6.844602in}{1.049078in}}%
\pgfpathlineto{\pgfqpoint{6.635952in}{1.049078in}}%
\pgfpathlineto{\pgfqpoint{6.635952in}{0.840429in}}%
\pgfusepath{fill}%
\end{pgfscope}%
\begin{pgfscope}%
\pgfpathrectangle{\pgfqpoint{6.635952in}{0.840429in}}{\pgfqpoint{0.208649in}{4.172982in}}%
\pgfusepath{clip}%
\pgfsetbuttcap%
\pgfsetroundjoin%
\definecolor{currentfill}{rgb}{0.065196,0.065196,0.090708}%
\pgfsetfillcolor{currentfill}%
\pgfsetlinewidth{0.000000pt}%
\definecolor{currentstroke}{rgb}{0.000000,0.000000,0.000000}%
\pgfsetstrokecolor{currentstroke}%
\pgfsetdash{}{0pt}%
\pgfpathmoveto{\pgfqpoint{6.635952in}{1.049078in}}%
\pgfpathlineto{\pgfqpoint{6.844602in}{1.049078in}}%
\pgfpathlineto{\pgfqpoint{6.844602in}{1.257727in}}%
\pgfpathlineto{\pgfqpoint{6.635952in}{1.257727in}}%
\pgfpathlineto{\pgfqpoint{6.635952in}{1.049078in}}%
\pgfusepath{fill}%
\end{pgfscope}%
\begin{pgfscope}%
\pgfpathrectangle{\pgfqpoint{6.635952in}{0.840429in}}{\pgfqpoint{0.208649in}{4.172982in}}%
\pgfusepath{clip}%
\pgfsetbuttcap%
\pgfsetroundjoin%
\definecolor{currentfill}{rgb}{0.109804,0.109804,0.152771}%
\pgfsetfillcolor{currentfill}%
\pgfsetlinewidth{0.000000pt}%
\definecolor{currentstroke}{rgb}{0.000000,0.000000,0.000000}%
\pgfsetstrokecolor{currentstroke}%
\pgfsetdash{}{0pt}%
\pgfpathmoveto{\pgfqpoint{6.635952in}{1.257727in}}%
\pgfpathlineto{\pgfqpoint{6.844602in}{1.257727in}}%
\pgfpathlineto{\pgfqpoint{6.844602in}{1.466376in}}%
\pgfpathlineto{\pgfqpoint{6.635952in}{1.466376in}}%
\pgfpathlineto{\pgfqpoint{6.635952in}{1.257727in}}%
\pgfusepath{fill}%
\end{pgfscope}%
\begin{pgfscope}%
\pgfpathrectangle{\pgfqpoint{6.635952in}{0.840429in}}{\pgfqpoint{0.208649in}{4.172982in}}%
\pgfusepath{clip}%
\pgfsetbuttcap%
\pgfsetroundjoin%
\definecolor{currentfill}{rgb}{0.150980,0.150980,0.210060}%
\pgfsetfillcolor{currentfill}%
\pgfsetlinewidth{0.000000pt}%
\definecolor{currentstroke}{rgb}{0.000000,0.000000,0.000000}%
\pgfsetstrokecolor{currentstroke}%
\pgfsetdash{}{0pt}%
\pgfpathmoveto{\pgfqpoint{6.635952in}{1.466376in}}%
\pgfpathlineto{\pgfqpoint{6.844602in}{1.466376in}}%
\pgfpathlineto{\pgfqpoint{6.844602in}{1.675026in}}%
\pgfpathlineto{\pgfqpoint{6.635952in}{1.675026in}}%
\pgfpathlineto{\pgfqpoint{6.635952in}{1.466376in}}%
\pgfusepath{fill}%
\end{pgfscope}%
\begin{pgfscope}%
\pgfpathrectangle{\pgfqpoint{6.635952in}{0.840429in}}{\pgfqpoint{0.208649in}{4.172982in}}%
\pgfusepath{clip}%
\pgfsetbuttcap%
\pgfsetroundjoin%
\definecolor{currentfill}{rgb}{0.195588,0.195588,0.272123}%
\pgfsetfillcolor{currentfill}%
\pgfsetlinewidth{0.000000pt}%
\definecolor{currentstroke}{rgb}{0.000000,0.000000,0.000000}%
\pgfsetstrokecolor{currentstroke}%
\pgfsetdash{}{0pt}%
\pgfpathmoveto{\pgfqpoint{6.635952in}{1.675026in}}%
\pgfpathlineto{\pgfqpoint{6.844602in}{1.675026in}}%
\pgfpathlineto{\pgfqpoint{6.844602in}{1.883675in}}%
\pgfpathlineto{\pgfqpoint{6.635952in}{1.883675in}}%
\pgfpathlineto{\pgfqpoint{6.635952in}{1.675026in}}%
\pgfusepath{fill}%
\end{pgfscope}%
\begin{pgfscope}%
\pgfpathrectangle{\pgfqpoint{6.635952in}{0.840429in}}{\pgfqpoint{0.208649in}{4.172982in}}%
\pgfusepath{clip}%
\pgfsetbuttcap%
\pgfsetroundjoin%
\definecolor{currentfill}{rgb}{0.240196,0.240196,0.334186}%
\pgfsetfillcolor{currentfill}%
\pgfsetlinewidth{0.000000pt}%
\definecolor{currentstroke}{rgb}{0.000000,0.000000,0.000000}%
\pgfsetstrokecolor{currentstroke}%
\pgfsetdash{}{0pt}%
\pgfpathmoveto{\pgfqpoint{6.635952in}{1.883675in}}%
\pgfpathlineto{\pgfqpoint{6.844602in}{1.883675in}}%
\pgfpathlineto{\pgfqpoint{6.844602in}{2.092324in}}%
\pgfpathlineto{\pgfqpoint{6.635952in}{2.092324in}}%
\pgfpathlineto{\pgfqpoint{6.635952in}{1.883675in}}%
\pgfusepath{fill}%
\end{pgfscope}%
\begin{pgfscope}%
\pgfpathrectangle{\pgfqpoint{6.635952in}{0.840429in}}{\pgfqpoint{0.208649in}{4.172982in}}%
\pgfusepath{clip}%
\pgfsetbuttcap%
\pgfsetroundjoin%
\definecolor{currentfill}{rgb}{0.284804,0.284804,0.396249}%
\pgfsetfillcolor{currentfill}%
\pgfsetlinewidth{0.000000pt}%
\definecolor{currentstroke}{rgb}{0.000000,0.000000,0.000000}%
\pgfsetstrokecolor{currentstroke}%
\pgfsetdash{}{0pt}%
\pgfpathmoveto{\pgfqpoint{6.635952in}{2.092324in}}%
\pgfpathlineto{\pgfqpoint{6.844602in}{2.092324in}}%
\pgfpathlineto{\pgfqpoint{6.844602in}{2.300973in}}%
\pgfpathlineto{\pgfqpoint{6.635952in}{2.300973in}}%
\pgfpathlineto{\pgfqpoint{6.635952in}{2.092324in}}%
\pgfusepath{fill}%
\end{pgfscope}%
\begin{pgfscope}%
\pgfpathrectangle{\pgfqpoint{6.635952in}{0.840429in}}{\pgfqpoint{0.208649in}{4.172982in}}%
\pgfusepath{clip}%
\pgfsetbuttcap%
\pgfsetroundjoin%
\definecolor{currentfill}{rgb}{0.329412,0.333150,0.454412}%
\pgfsetfillcolor{currentfill}%
\pgfsetlinewidth{0.000000pt}%
\definecolor{currentstroke}{rgb}{0.000000,0.000000,0.000000}%
\pgfsetstrokecolor{currentstroke}%
\pgfsetdash{}{0pt}%
\pgfpathmoveto{\pgfqpoint{6.635952in}{2.300973in}}%
\pgfpathlineto{\pgfqpoint{6.844602in}{2.300973in}}%
\pgfpathlineto{\pgfqpoint{6.844602in}{2.509622in}}%
\pgfpathlineto{\pgfqpoint{6.635952in}{2.509622in}}%
\pgfpathlineto{\pgfqpoint{6.635952in}{2.300973in}}%
\pgfusepath{fill}%
\end{pgfscope}%
\begin{pgfscope}%
\pgfpathrectangle{\pgfqpoint{6.635952in}{0.840429in}}{\pgfqpoint{0.208649in}{4.172982in}}%
\pgfusepath{clip}%
\pgfsetbuttcap%
\pgfsetroundjoin%
\definecolor{currentfill}{rgb}{0.370588,0.389767,0.495588}%
\pgfsetfillcolor{currentfill}%
\pgfsetlinewidth{0.000000pt}%
\definecolor{currentstroke}{rgb}{0.000000,0.000000,0.000000}%
\pgfsetstrokecolor{currentstroke}%
\pgfsetdash{}{0pt}%
\pgfpathmoveto{\pgfqpoint{6.635952in}{2.509622in}}%
\pgfpathlineto{\pgfqpoint{6.844602in}{2.509622in}}%
\pgfpathlineto{\pgfqpoint{6.844602in}{2.718271in}}%
\pgfpathlineto{\pgfqpoint{6.635952in}{2.718271in}}%
\pgfpathlineto{\pgfqpoint{6.635952in}{2.509622in}}%
\pgfusepath{fill}%
\end{pgfscope}%
\begin{pgfscope}%
\pgfpathrectangle{\pgfqpoint{6.635952in}{0.840429in}}{\pgfqpoint{0.208649in}{4.172982in}}%
\pgfusepath{clip}%
\pgfsetbuttcap%
\pgfsetroundjoin%
\definecolor{currentfill}{rgb}{0.415196,0.451103,0.540196}%
\pgfsetfillcolor{currentfill}%
\pgfsetlinewidth{0.000000pt}%
\definecolor{currentstroke}{rgb}{0.000000,0.000000,0.000000}%
\pgfsetstrokecolor{currentstroke}%
\pgfsetdash{}{0pt}%
\pgfpathmoveto{\pgfqpoint{6.635952in}{2.718271in}}%
\pgfpathlineto{\pgfqpoint{6.844602in}{2.718271in}}%
\pgfpathlineto{\pgfqpoint{6.844602in}{2.926920in}}%
\pgfpathlineto{\pgfqpoint{6.635952in}{2.926920in}}%
\pgfpathlineto{\pgfqpoint{6.635952in}{2.718271in}}%
\pgfusepath{fill}%
\end{pgfscope}%
\begin{pgfscope}%
\pgfpathrectangle{\pgfqpoint{6.635952in}{0.840429in}}{\pgfqpoint{0.208649in}{4.172982in}}%
\pgfusepath{clip}%
\pgfsetbuttcap%
\pgfsetroundjoin%
\definecolor{currentfill}{rgb}{0.459804,0.512439,0.584804}%
\pgfsetfillcolor{currentfill}%
\pgfsetlinewidth{0.000000pt}%
\definecolor{currentstroke}{rgb}{0.000000,0.000000,0.000000}%
\pgfsetstrokecolor{currentstroke}%
\pgfsetdash{}{0pt}%
\pgfpathmoveto{\pgfqpoint{6.635952in}{2.926920in}}%
\pgfpathlineto{\pgfqpoint{6.844602in}{2.926920in}}%
\pgfpathlineto{\pgfqpoint{6.844602in}{3.135569in}}%
\pgfpathlineto{\pgfqpoint{6.635952in}{3.135569in}}%
\pgfpathlineto{\pgfqpoint{6.635952in}{2.926920in}}%
\pgfusepath{fill}%
\end{pgfscope}%
\begin{pgfscope}%
\pgfpathrectangle{\pgfqpoint{6.635952in}{0.840429in}}{\pgfqpoint{0.208649in}{4.172982in}}%
\pgfusepath{clip}%
\pgfsetbuttcap%
\pgfsetroundjoin%
\definecolor{currentfill}{rgb}{0.504412,0.573774,0.629412}%
\pgfsetfillcolor{currentfill}%
\pgfsetlinewidth{0.000000pt}%
\definecolor{currentstroke}{rgb}{0.000000,0.000000,0.000000}%
\pgfsetstrokecolor{currentstroke}%
\pgfsetdash{}{0pt}%
\pgfpathmoveto{\pgfqpoint{6.635952in}{3.135569in}}%
\pgfpathlineto{\pgfqpoint{6.844602in}{3.135569in}}%
\pgfpathlineto{\pgfqpoint{6.844602in}{3.344219in}}%
\pgfpathlineto{\pgfqpoint{6.635952in}{3.344219in}}%
\pgfpathlineto{\pgfqpoint{6.635952in}{3.135569in}}%
\pgfusepath{fill}%
\end{pgfscope}%
\begin{pgfscope}%
\pgfpathrectangle{\pgfqpoint{6.635952in}{0.840429in}}{\pgfqpoint{0.208649in}{4.172982in}}%
\pgfusepath{clip}%
\pgfsetbuttcap%
\pgfsetroundjoin%
\definecolor{currentfill}{rgb}{0.549020,0.635110,0.674020}%
\pgfsetfillcolor{currentfill}%
\pgfsetlinewidth{0.000000pt}%
\definecolor{currentstroke}{rgb}{0.000000,0.000000,0.000000}%
\pgfsetstrokecolor{currentstroke}%
\pgfsetdash{}{0pt}%
\pgfpathmoveto{\pgfqpoint{6.635952in}{3.344219in}}%
\pgfpathlineto{\pgfqpoint{6.844602in}{3.344219in}}%
\pgfpathlineto{\pgfqpoint{6.844602in}{3.552868in}}%
\pgfpathlineto{\pgfqpoint{6.635952in}{3.552868in}}%
\pgfpathlineto{\pgfqpoint{6.635952in}{3.344219in}}%
\pgfusepath{fill}%
\end{pgfscope}%
\begin{pgfscope}%
\pgfpathrectangle{\pgfqpoint{6.635952in}{0.840429in}}{\pgfqpoint{0.208649in}{4.172982in}}%
\pgfusepath{clip}%
\pgfsetbuttcap%
\pgfsetroundjoin%
\definecolor{currentfill}{rgb}{0.590196,0.691728,0.715196}%
\pgfsetfillcolor{currentfill}%
\pgfsetlinewidth{0.000000pt}%
\definecolor{currentstroke}{rgb}{0.000000,0.000000,0.000000}%
\pgfsetstrokecolor{currentstroke}%
\pgfsetdash{}{0pt}%
\pgfpathmoveto{\pgfqpoint{6.635952in}{3.552868in}}%
\pgfpathlineto{\pgfqpoint{6.844602in}{3.552868in}}%
\pgfpathlineto{\pgfqpoint{6.844602in}{3.761517in}}%
\pgfpathlineto{\pgfqpoint{6.635952in}{3.761517in}}%
\pgfpathlineto{\pgfqpoint{6.635952in}{3.552868in}}%
\pgfusepath{fill}%
\end{pgfscope}%
\begin{pgfscope}%
\pgfpathrectangle{\pgfqpoint{6.635952in}{0.840429in}}{\pgfqpoint{0.208649in}{4.172982in}}%
\pgfusepath{clip}%
\pgfsetbuttcap%
\pgfsetroundjoin%
\definecolor{currentfill}{rgb}{0.634804,0.753064,0.759804}%
\pgfsetfillcolor{currentfill}%
\pgfsetlinewidth{0.000000pt}%
\definecolor{currentstroke}{rgb}{0.000000,0.000000,0.000000}%
\pgfsetstrokecolor{currentstroke}%
\pgfsetdash{}{0pt}%
\pgfpathmoveto{\pgfqpoint{6.635952in}{3.761517in}}%
\pgfpathlineto{\pgfqpoint{6.844602in}{3.761517in}}%
\pgfpathlineto{\pgfqpoint{6.844602in}{3.970166in}}%
\pgfpathlineto{\pgfqpoint{6.635952in}{3.970166in}}%
\pgfpathlineto{\pgfqpoint{6.635952in}{3.761517in}}%
\pgfusepath{fill}%
\end{pgfscope}%
\begin{pgfscope}%
\pgfpathrectangle{\pgfqpoint{6.635952in}{0.840429in}}{\pgfqpoint{0.208649in}{4.172982in}}%
\pgfusepath{clip}%
\pgfsetbuttcap%
\pgfsetroundjoin%
\definecolor{currentfill}{rgb}{0.694393,0.804412,0.804412}%
\pgfsetfillcolor{currentfill}%
\pgfsetlinewidth{0.000000pt}%
\definecolor{currentstroke}{rgb}{0.000000,0.000000,0.000000}%
\pgfsetstrokecolor{currentstroke}%
\pgfsetdash{}{0pt}%
\pgfpathmoveto{\pgfqpoint{6.635952in}{3.970166in}}%
\pgfpathlineto{\pgfqpoint{6.844602in}{3.970166in}}%
\pgfpathlineto{\pgfqpoint{6.844602in}{4.178815in}}%
\pgfpathlineto{\pgfqpoint{6.635952in}{4.178815in}}%
\pgfpathlineto{\pgfqpoint{6.635952in}{3.970166in}}%
\pgfusepath{fill}%
\end{pgfscope}%
\begin{pgfscope}%
\pgfpathrectangle{\pgfqpoint{6.635952in}{0.840429in}}{\pgfqpoint{0.208649in}{4.172982in}}%
\pgfusepath{clip}%
\pgfsetbuttcap%
\pgfsetroundjoin%
\definecolor{currentfill}{rgb}{0.764093,0.849020,0.849020}%
\pgfsetfillcolor{currentfill}%
\pgfsetlinewidth{0.000000pt}%
\definecolor{currentstroke}{rgb}{0.000000,0.000000,0.000000}%
\pgfsetstrokecolor{currentstroke}%
\pgfsetdash{}{0pt}%
\pgfpathmoveto{\pgfqpoint{6.635952in}{4.178815in}}%
\pgfpathlineto{\pgfqpoint{6.844602in}{4.178815in}}%
\pgfpathlineto{\pgfqpoint{6.844602in}{4.387464in}}%
\pgfpathlineto{\pgfqpoint{6.635952in}{4.387464in}}%
\pgfpathlineto{\pgfqpoint{6.635952in}{4.178815in}}%
\pgfusepath{fill}%
\end{pgfscope}%
\begin{pgfscope}%
\pgfpathrectangle{\pgfqpoint{6.635952in}{0.840429in}}{\pgfqpoint{0.208649in}{4.172982in}}%
\pgfusepath{clip}%
\pgfsetbuttcap%
\pgfsetroundjoin%
\definecolor{currentfill}{rgb}{0.833793,0.893627,0.893627}%
\pgfsetfillcolor{currentfill}%
\pgfsetlinewidth{0.000000pt}%
\definecolor{currentstroke}{rgb}{0.000000,0.000000,0.000000}%
\pgfsetstrokecolor{currentstroke}%
\pgfsetdash{}{0pt}%
\pgfpathmoveto{\pgfqpoint{6.635952in}{4.387464in}}%
\pgfpathlineto{\pgfqpoint{6.844602in}{4.387464in}}%
\pgfpathlineto{\pgfqpoint{6.844602in}{4.596113in}}%
\pgfpathlineto{\pgfqpoint{6.635952in}{4.596113in}}%
\pgfpathlineto{\pgfqpoint{6.635952in}{4.387464in}}%
\pgfusepath{fill}%
\end{pgfscope}%
\begin{pgfscope}%
\pgfpathrectangle{\pgfqpoint{6.635952in}{0.840429in}}{\pgfqpoint{0.208649in}{4.172982in}}%
\pgfusepath{clip}%
\pgfsetbuttcap%
\pgfsetroundjoin%
\definecolor{currentfill}{rgb}{0.898131,0.934804,0.934804}%
\pgfsetfillcolor{currentfill}%
\pgfsetlinewidth{0.000000pt}%
\definecolor{currentstroke}{rgb}{0.000000,0.000000,0.000000}%
\pgfsetstrokecolor{currentstroke}%
\pgfsetdash{}{0pt}%
\pgfpathmoveto{\pgfqpoint{6.635952in}{4.596113in}}%
\pgfpathlineto{\pgfqpoint{6.844602in}{4.596113in}}%
\pgfpathlineto{\pgfqpoint{6.844602in}{4.804762in}}%
\pgfpathlineto{\pgfqpoint{6.635952in}{4.804762in}}%
\pgfpathlineto{\pgfqpoint{6.635952in}{4.596113in}}%
\pgfusepath{fill}%
\end{pgfscope}%
\begin{pgfscope}%
\pgfpathrectangle{\pgfqpoint{6.635952in}{0.840429in}}{\pgfqpoint{0.208649in}{4.172982in}}%
\pgfusepath{clip}%
\pgfsetbuttcap%
\pgfsetroundjoin%
\definecolor{currentfill}{rgb}{0.967831,0.979412,0.979412}%
\pgfsetfillcolor{currentfill}%
\pgfsetlinewidth{0.000000pt}%
\definecolor{currentstroke}{rgb}{0.000000,0.000000,0.000000}%
\pgfsetstrokecolor{currentstroke}%
\pgfsetdash{}{0pt}%
\pgfpathmoveto{\pgfqpoint{6.635952in}{4.804762in}}%
\pgfpathlineto{\pgfqpoint{6.844602in}{4.804762in}}%
\pgfpathlineto{\pgfqpoint{6.844602in}{5.013411in}}%
\pgfpathlineto{\pgfqpoint{6.635952in}{5.013411in}}%
\pgfpathlineto{\pgfqpoint{6.635952in}{4.804762in}}%
\pgfusepath{fill}%
\end{pgfscope}%
\begin{pgfscope}%
\pgfsetrectcap%
\pgfsetmiterjoin%
\pgfsetlinewidth{0.000000pt}%
\definecolor{currentstroke}{rgb}{1.000000,1.000000,1.000000}%
\pgfsetstrokecolor{currentstroke}%
\pgfsetdash{}{0pt}%
\pgfpathmoveto{\pgfqpoint{6.635952in}{0.840429in}}%
\pgfpathlineto{\pgfqpoint{6.740277in}{0.840429in}}%
\pgfpathlineto{\pgfqpoint{6.844602in}{0.840429in}}%
\pgfpathlineto{\pgfqpoint{6.844602in}{5.013411in}}%
\pgfpathlineto{\pgfqpoint{6.740277in}{5.013411in}}%
\pgfpathlineto{\pgfqpoint{6.635952in}{5.013411in}}%
\pgfpathlineto{\pgfqpoint{6.635952in}{0.840429in}}%
\pgfpathclose%
\pgfusepath{}%
\end{pgfscope}%
\end{pgfpicture}%
\makeatother%
\endgroup%
}
    \caption{Contour plot of magnetic field produced by the permanent magnet}
    \label{fig:mo_t1}
  \end{figure}
\subsubsection*{Spin Echo}

\subsubsection*{Carr-Purcell method}

\subsubsection*{Meiboom-Gill method}

\section*{Summary \lowercase{and} Conclusion}



\section*{Appendix}


  
  \begin{figure}[h]
    \resizebox{0.45\textwidth}{!}{%% Creator: Matplotlib, PGF backend
%%
%% To include the figure in your LaTeX document, write
%%   \input{<filename>.pgf}
%%
%% Make sure the required packages are loaded in your preamble
%%   \usepackage{pgf}
%%
%% Also ensure that all the required font packages are loaded; for instance,
%% the lmodern package is sometimes necessary when using math font.
%%   \usepackage{lmodern}
%%
%% Figures using additional raster images can only be included by \input if
%% they are in the same directory as the main LaTeX file. For loading figures
%% from other directories you can use the `import` package
%%   \usepackage{import}
%%
%% and then include the figures with
%%   \import{<path to file>}{<filename>.pgf}
%%
%% Matplotlib used the following preamble
%%   \usepackage{fontspec}
%%   \setmainfont{DejaVuSerif.ttf}[Path=\detokenize{/usr/share/matplotlib/mpl-data/fonts/ttf/}]
%%   \setsansfont{DejaVuSans.ttf}[Path=\detokenize{/usr/share/matplotlib/mpl-data/fonts/ttf/}]
%%   \setmonofont{DejaVuSansMono.ttf}[Path=\detokenize{/usr/share/matplotlib/mpl-data/fonts/ttf/}]
%%
\begingroup%
\makeatletter%
\begin{pgfpicture}%
\pgfpathrectangle{\pgfpointorigin}{\pgfqpoint{8.000000in}{5.500000in}}%
\pgfusepath{use as bounding box, clip}%
\begin{pgfscope}%
\pgfsetbuttcap%
\pgfsetmiterjoin%
\definecolor{currentfill}{rgb}{1.000000,1.000000,1.000000}%
\pgfsetfillcolor{currentfill}%
\pgfsetlinewidth{0.000000pt}%
\definecolor{currentstroke}{rgb}{1.000000,1.000000,1.000000}%
\pgfsetstrokecolor{currentstroke}%
\pgfsetdash{}{0pt}%
\pgfpathmoveto{\pgfqpoint{0.000000in}{0.000000in}}%
\pgfpathlineto{\pgfqpoint{8.000000in}{0.000000in}}%
\pgfpathlineto{\pgfqpoint{8.000000in}{5.500000in}}%
\pgfpathlineto{\pgfqpoint{0.000000in}{5.500000in}}%
\pgfpathlineto{\pgfqpoint{0.000000in}{0.000000in}}%
\pgfpathclose%
\pgfusepath{fill}%
\end{pgfscope}%
\begin{pgfscope}%
\pgfsetbuttcap%
\pgfsetmiterjoin%
\definecolor{currentfill}{rgb}{0.917647,0.917647,0.949020}%
\pgfsetfillcolor{currentfill}%
\pgfsetlinewidth{0.000000pt}%
\definecolor{currentstroke}{rgb}{0.000000,0.000000,0.000000}%
\pgfsetstrokecolor{currentstroke}%
\pgfsetstrokeopacity{0.000000}%
\pgfsetdash{}{0pt}%
\pgfpathmoveto{\pgfqpoint{1.026143in}{0.898915in}}%
\pgfpathlineto{\pgfqpoint{7.850000in}{0.898915in}}%
\pgfpathlineto{\pgfqpoint{7.850000in}{5.013411in}}%
\pgfpathlineto{\pgfqpoint{1.026143in}{5.013411in}}%
\pgfpathlineto{\pgfqpoint{1.026143in}{0.898915in}}%
\pgfpathclose%
\pgfusepath{fill}%
\end{pgfscope}%
\begin{pgfscope}%
\pgfpathrectangle{\pgfqpoint{1.026143in}{0.898915in}}{\pgfqpoint{6.823857in}{4.114497in}}%
\pgfusepath{clip}%
\pgfsetroundcap%
\pgfsetroundjoin%
\pgfsetlinewidth{1.003750pt}%
\definecolor{currentstroke}{rgb}{1.000000,1.000000,1.000000}%
\pgfsetstrokecolor{currentstroke}%
\pgfsetdash{}{0pt}%
\pgfpathmoveto{\pgfqpoint{1.273657in}{0.898915in}}%
\pgfpathlineto{\pgfqpoint{1.273657in}{5.013411in}}%
\pgfusepath{stroke}%
\end{pgfscope}%
\begin{pgfscope}%
\definecolor{textcolor}{rgb}{0.150000,0.150000,0.150000}%
\pgfsetstrokecolor{textcolor}%
\pgfsetfillcolor{textcolor}%
\pgftext[x=1.273657in,y=0.801692in,,top]{\color{textcolor}\rmfamily\fontsize{20.000000}{24.000000}\selectfont 0}%
\end{pgfscope}%
\begin{pgfscope}%
\pgfpathrectangle{\pgfqpoint{1.026143in}{0.898915in}}{\pgfqpoint{6.823857in}{4.114497in}}%
\pgfusepath{clip}%
\pgfsetroundcap%
\pgfsetroundjoin%
\pgfsetlinewidth{1.003750pt}%
\definecolor{currentstroke}{rgb}{1.000000,1.000000,1.000000}%
\pgfsetstrokecolor{currentstroke}%
\pgfsetdash{}{0pt}%
\pgfpathmoveto{\pgfqpoint{2.526891in}{0.898915in}}%
\pgfpathlineto{\pgfqpoint{2.526891in}{5.013411in}}%
\pgfusepath{stroke}%
\end{pgfscope}%
\begin{pgfscope}%
\definecolor{textcolor}{rgb}{0.150000,0.150000,0.150000}%
\pgfsetstrokecolor{textcolor}%
\pgfsetfillcolor{textcolor}%
\pgftext[x=2.526891in,y=0.801692in,,top]{\color{textcolor}\rmfamily\fontsize{20.000000}{24.000000}\selectfont 20}%
\end{pgfscope}%
\begin{pgfscope}%
\pgfpathrectangle{\pgfqpoint{1.026143in}{0.898915in}}{\pgfqpoint{6.823857in}{4.114497in}}%
\pgfusepath{clip}%
\pgfsetroundcap%
\pgfsetroundjoin%
\pgfsetlinewidth{1.003750pt}%
\definecolor{currentstroke}{rgb}{1.000000,1.000000,1.000000}%
\pgfsetstrokecolor{currentstroke}%
\pgfsetdash{}{0pt}%
\pgfpathmoveto{\pgfqpoint{3.780124in}{0.898915in}}%
\pgfpathlineto{\pgfqpoint{3.780124in}{5.013411in}}%
\pgfusepath{stroke}%
\end{pgfscope}%
\begin{pgfscope}%
\definecolor{textcolor}{rgb}{0.150000,0.150000,0.150000}%
\pgfsetstrokecolor{textcolor}%
\pgfsetfillcolor{textcolor}%
\pgftext[x=3.780124in,y=0.801692in,,top]{\color{textcolor}\rmfamily\fontsize{20.000000}{24.000000}\selectfont 40}%
\end{pgfscope}%
\begin{pgfscope}%
\pgfpathrectangle{\pgfqpoint{1.026143in}{0.898915in}}{\pgfqpoint{6.823857in}{4.114497in}}%
\pgfusepath{clip}%
\pgfsetroundcap%
\pgfsetroundjoin%
\pgfsetlinewidth{1.003750pt}%
\definecolor{currentstroke}{rgb}{1.000000,1.000000,1.000000}%
\pgfsetstrokecolor{currentstroke}%
\pgfsetdash{}{0pt}%
\pgfpathmoveto{\pgfqpoint{5.033358in}{0.898915in}}%
\pgfpathlineto{\pgfqpoint{5.033358in}{5.013411in}}%
\pgfusepath{stroke}%
\end{pgfscope}%
\begin{pgfscope}%
\definecolor{textcolor}{rgb}{0.150000,0.150000,0.150000}%
\pgfsetstrokecolor{textcolor}%
\pgfsetfillcolor{textcolor}%
\pgftext[x=5.033358in,y=0.801692in,,top]{\color{textcolor}\rmfamily\fontsize{20.000000}{24.000000}\selectfont 60}%
\end{pgfscope}%
\begin{pgfscope}%
\pgfpathrectangle{\pgfqpoint{1.026143in}{0.898915in}}{\pgfqpoint{6.823857in}{4.114497in}}%
\pgfusepath{clip}%
\pgfsetroundcap%
\pgfsetroundjoin%
\pgfsetlinewidth{1.003750pt}%
\definecolor{currentstroke}{rgb}{1.000000,1.000000,1.000000}%
\pgfsetstrokecolor{currentstroke}%
\pgfsetdash{}{0pt}%
\pgfpathmoveto{\pgfqpoint{6.286591in}{0.898915in}}%
\pgfpathlineto{\pgfqpoint{6.286591in}{5.013411in}}%
\pgfusepath{stroke}%
\end{pgfscope}%
\begin{pgfscope}%
\definecolor{textcolor}{rgb}{0.150000,0.150000,0.150000}%
\pgfsetstrokecolor{textcolor}%
\pgfsetfillcolor{textcolor}%
\pgftext[x=6.286591in,y=0.801692in,,top]{\color{textcolor}\rmfamily\fontsize{20.000000}{24.000000}\selectfont 80}%
\end{pgfscope}%
\begin{pgfscope}%
\pgfpathrectangle{\pgfqpoint{1.026143in}{0.898915in}}{\pgfqpoint{6.823857in}{4.114497in}}%
\pgfusepath{clip}%
\pgfsetroundcap%
\pgfsetroundjoin%
\pgfsetlinewidth{1.003750pt}%
\definecolor{currentstroke}{rgb}{1.000000,1.000000,1.000000}%
\pgfsetstrokecolor{currentstroke}%
\pgfsetdash{}{0pt}%
\pgfpathmoveto{\pgfqpoint{7.539825in}{0.898915in}}%
\pgfpathlineto{\pgfqpoint{7.539825in}{5.013411in}}%
\pgfusepath{stroke}%
\end{pgfscope}%
\begin{pgfscope}%
\definecolor{textcolor}{rgb}{0.150000,0.150000,0.150000}%
\pgfsetstrokecolor{textcolor}%
\pgfsetfillcolor{textcolor}%
\pgftext[x=7.539825in,y=0.801692in,,top]{\color{textcolor}\rmfamily\fontsize{20.000000}{24.000000}\selectfont 100}%
\end{pgfscope}%
\begin{pgfscope}%
\definecolor{textcolor}{rgb}{0.150000,0.150000,0.150000}%
\pgfsetstrokecolor{textcolor}%
\pgfsetfillcolor{textcolor}%
\pgftext[x=4.438072in,y=0.477311in,,top]{\color{textcolor}\rmfamily\fontsize{24.000000}{28.800000}\selectfont Delay Time (ms)}%
\end{pgfscope}%
\begin{pgfscope}%
\pgfpathrectangle{\pgfqpoint{1.026143in}{0.898915in}}{\pgfqpoint{6.823857in}{4.114497in}}%
\pgfusepath{clip}%
\pgfsetroundcap%
\pgfsetroundjoin%
\pgfsetlinewidth{1.003750pt}%
\definecolor{currentstroke}{rgb}{1.000000,1.000000,1.000000}%
\pgfsetstrokecolor{currentstroke}%
\pgfsetdash{}{0pt}%
\pgfpathmoveto{\pgfqpoint{1.026143in}{1.222157in}}%
\pgfpathlineto{\pgfqpoint{7.850000in}{1.222157in}}%
\pgfusepath{stroke}%
\end{pgfscope}%
\begin{pgfscope}%
\definecolor{textcolor}{rgb}{0.150000,0.150000,0.150000}%
\pgfsetstrokecolor{textcolor}%
\pgfsetfillcolor{textcolor}%
\pgftext[x=0.528147in, y=1.116634in, left, base]{\color{textcolor}\rmfamily\fontsize{20.000000}{24.000000}\selectfont \ensuremath{-}6}%
\end{pgfscope}%
\begin{pgfscope}%
\pgfpathrectangle{\pgfqpoint{1.026143in}{0.898915in}}{\pgfqpoint{6.823857in}{4.114497in}}%
\pgfusepath{clip}%
\pgfsetroundcap%
\pgfsetroundjoin%
\pgfsetlinewidth{1.003750pt}%
\definecolor{currentstroke}{rgb}{1.000000,1.000000,1.000000}%
\pgfsetstrokecolor{currentstroke}%
\pgfsetdash{}{0pt}%
\pgfpathmoveto{\pgfqpoint{1.026143in}{1.928444in}}%
\pgfpathlineto{\pgfqpoint{7.850000in}{1.928444in}}%
\pgfusepath{stroke}%
\end{pgfscope}%
\begin{pgfscope}%
\definecolor{textcolor}{rgb}{0.150000,0.150000,0.150000}%
\pgfsetstrokecolor{textcolor}%
\pgfsetfillcolor{textcolor}%
\pgftext[x=0.528147in, y=1.822921in, left, base]{\color{textcolor}\rmfamily\fontsize{20.000000}{24.000000}\selectfont \ensuremath{-}4}%
\end{pgfscope}%
\begin{pgfscope}%
\pgfpathrectangle{\pgfqpoint{1.026143in}{0.898915in}}{\pgfqpoint{6.823857in}{4.114497in}}%
\pgfusepath{clip}%
\pgfsetroundcap%
\pgfsetroundjoin%
\pgfsetlinewidth{1.003750pt}%
\definecolor{currentstroke}{rgb}{1.000000,1.000000,1.000000}%
\pgfsetstrokecolor{currentstroke}%
\pgfsetdash{}{0pt}%
\pgfpathmoveto{\pgfqpoint{1.026143in}{2.634730in}}%
\pgfpathlineto{\pgfqpoint{7.850000in}{2.634730in}}%
\pgfusepath{stroke}%
\end{pgfscope}%
\begin{pgfscope}%
\definecolor{textcolor}{rgb}{0.150000,0.150000,0.150000}%
\pgfsetstrokecolor{textcolor}%
\pgfsetfillcolor{textcolor}%
\pgftext[x=0.528147in, y=2.529207in, left, base]{\color{textcolor}\rmfamily\fontsize{20.000000}{24.000000}\selectfont \ensuremath{-}2}%
\end{pgfscope}%
\begin{pgfscope}%
\pgfpathrectangle{\pgfqpoint{1.026143in}{0.898915in}}{\pgfqpoint{6.823857in}{4.114497in}}%
\pgfusepath{clip}%
\pgfsetroundcap%
\pgfsetroundjoin%
\pgfsetlinewidth{1.003750pt}%
\definecolor{currentstroke}{rgb}{1.000000,1.000000,1.000000}%
\pgfsetstrokecolor{currentstroke}%
\pgfsetdash{}{0pt}%
\pgfpathmoveto{\pgfqpoint{1.026143in}{3.341017in}}%
\pgfpathlineto{\pgfqpoint{7.850000in}{3.341017in}}%
\pgfusepath{stroke}%
\end{pgfscope}%
\begin{pgfscope}%
\definecolor{textcolor}{rgb}{0.150000,0.150000,0.150000}%
\pgfsetstrokecolor{textcolor}%
\pgfsetfillcolor{textcolor}%
\pgftext[x=0.752190in, y=3.235494in, left, base]{\color{textcolor}\rmfamily\fontsize{20.000000}{24.000000}\selectfont 0}%
\end{pgfscope}%
\begin{pgfscope}%
\pgfpathrectangle{\pgfqpoint{1.026143in}{0.898915in}}{\pgfqpoint{6.823857in}{4.114497in}}%
\pgfusepath{clip}%
\pgfsetroundcap%
\pgfsetroundjoin%
\pgfsetlinewidth{1.003750pt}%
\definecolor{currentstroke}{rgb}{1.000000,1.000000,1.000000}%
\pgfsetstrokecolor{currentstroke}%
\pgfsetdash{}{0pt}%
\pgfpathmoveto{\pgfqpoint{1.026143in}{4.047303in}}%
\pgfpathlineto{\pgfqpoint{7.850000in}{4.047303in}}%
\pgfusepath{stroke}%
\end{pgfscope}%
\begin{pgfscope}%
\definecolor{textcolor}{rgb}{0.150000,0.150000,0.150000}%
\pgfsetstrokecolor{textcolor}%
\pgfsetfillcolor{textcolor}%
\pgftext[x=0.752190in, y=3.941780in, left, base]{\color{textcolor}\rmfamily\fontsize{20.000000}{24.000000}\selectfont 2}%
\end{pgfscope}%
\begin{pgfscope}%
\pgfpathrectangle{\pgfqpoint{1.026143in}{0.898915in}}{\pgfqpoint{6.823857in}{4.114497in}}%
\pgfusepath{clip}%
\pgfsetroundcap%
\pgfsetroundjoin%
\pgfsetlinewidth{1.003750pt}%
\definecolor{currentstroke}{rgb}{1.000000,1.000000,1.000000}%
\pgfsetstrokecolor{currentstroke}%
\pgfsetdash{}{0pt}%
\pgfpathmoveto{\pgfqpoint{1.026143in}{4.753590in}}%
\pgfpathlineto{\pgfqpoint{7.850000in}{4.753590in}}%
\pgfusepath{stroke}%
\end{pgfscope}%
\begin{pgfscope}%
\definecolor{textcolor}{rgb}{0.150000,0.150000,0.150000}%
\pgfsetstrokecolor{textcolor}%
\pgfsetfillcolor{textcolor}%
\pgftext[x=0.752190in, y=4.648067in, left, base]{\color{textcolor}\rmfamily\fontsize{20.000000}{24.000000}\selectfont 4}%
\end{pgfscope}%
\begin{pgfscope}%
\definecolor{textcolor}{rgb}{0.150000,0.150000,0.150000}%
\pgfsetstrokecolor{textcolor}%
\pgfsetfillcolor{textcolor}%
\pgftext[x=0.472591in,y=2.956163in,,bottom,rotate=90.000000]{\color{textcolor}\rmfamily\fontsize{24.000000}{28.800000}\selectfont Pulse Amplitude (V)}%
\end{pgfscope}%
\begin{pgfscope}%
\pgfpathrectangle{\pgfqpoint{1.026143in}{0.898915in}}{\pgfqpoint{6.823857in}{4.114497in}}%
\pgfusepath{clip}%
\pgfsetbuttcap%
\pgfsetroundjoin%
\definecolor{currentfill}{rgb}{0.392157,0.584314,0.929412}%
\pgfsetfillcolor{currentfill}%
\pgfsetlinewidth{0.301125pt}%
\definecolor{currentstroke}{rgb}{0.392157,0.584314,0.929412}%
\pgfsetstrokecolor{currentstroke}%
\pgfsetdash{}{0pt}%
\pgfsys@defobject{currentmarker}{\pgfqpoint{-0.048611in}{-0.048611in}}{\pgfqpoint{0.048611in}{0.048611in}}{%
\pgfpathmoveto{\pgfqpoint{0.000000in}{-0.048611in}}%
\pgfpathcurveto{\pgfqpoint{0.012892in}{-0.048611in}}{\pgfqpoint{0.025257in}{-0.043489in}}{\pgfqpoint{0.034373in}{-0.034373in}}%
\pgfpathcurveto{\pgfqpoint{0.043489in}{-0.025257in}}{\pgfqpoint{0.048611in}{-0.012892in}}{\pgfqpoint{0.048611in}{0.000000in}}%
\pgfpathcurveto{\pgfqpoint{0.048611in}{0.012892in}}{\pgfqpoint{0.043489in}{0.025257in}}{\pgfqpoint{0.034373in}{0.034373in}}%
\pgfpathcurveto{\pgfqpoint{0.025257in}{0.043489in}}{\pgfqpoint{0.012892in}{0.048611in}}{\pgfqpoint{0.000000in}{0.048611in}}%
\pgfpathcurveto{\pgfqpoint{-0.012892in}{0.048611in}}{\pgfqpoint{-0.025257in}{0.043489in}}{\pgfqpoint{-0.034373in}{0.034373in}}%
\pgfpathcurveto{\pgfqpoint{-0.043489in}{0.025257in}}{\pgfqpoint{-0.048611in}{0.012892in}}{\pgfqpoint{-0.048611in}{0.000000in}}%
\pgfpathcurveto{\pgfqpoint{-0.048611in}{-0.012892in}}{\pgfqpoint{-0.043489in}{-0.025257in}}{\pgfqpoint{-0.034373in}{-0.034373in}}%
\pgfpathcurveto{\pgfqpoint{-0.025257in}{-0.043489in}}{\pgfqpoint{-0.012892in}{-0.048611in}}{\pgfqpoint{0.000000in}{-0.048611in}}%
\pgfpathlineto{\pgfqpoint{0.000000in}{-0.048611in}}%
\pgfpathclose%
\pgfusepath{stroke,fill}%
}%
\begin{pgfscope}%
\pgfsys@transformshift{1.336319in}{1.123277in}%
\pgfsys@useobject{currentmarker}{}%
\end{pgfscope}%
\begin{pgfscope}%
\pgfsys@transformshift{1.586965in}{1.504672in}%
\pgfsys@useobject{currentmarker}{}%
\end{pgfscope}%
\begin{pgfscope}%
\pgfsys@transformshift{1.900274in}{1.843689in}%
\pgfsys@useobject{currentmarker}{}%
\end{pgfscope}%
\begin{pgfscope}%
\pgfsys@transformshift{2.213582in}{2.154455in}%
\pgfsys@useobject{currentmarker}{}%
\end{pgfscope}%
\begin{pgfscope}%
\pgfsys@transformshift{2.526891in}{2.422844in}%
\pgfsys@useobject{currentmarker}{}%
\end{pgfscope}%
\begin{pgfscope}%
\pgfsys@transformshift{2.840199in}{2.648856in}%
\pgfsys@useobject{currentmarker}{}%
\end{pgfscope}%
\begin{pgfscope}%
\pgfsys@transformshift{3.153507in}{2.903119in}%
\pgfsys@useobject{currentmarker}{}%
\end{pgfscope}%
\begin{pgfscope}%
\pgfsys@transformshift{3.466816in}{3.228011in}%
\pgfsys@useobject{currentmarker}{}%
\end{pgfscope}%
\begin{pgfscope}%
\pgfsys@transformshift{3.780124in}{3.581154in}%
\pgfsys@useobject{currentmarker}{}%
\end{pgfscope}%
\begin{pgfscope}%
\pgfsys@transformshift{4.093432in}{3.821292in}%
\pgfsys@useobject{currentmarker}{}%
\end{pgfscope}%
\begin{pgfscope}%
\pgfsys@transformshift{4.406741in}{3.990800in}%
\pgfsys@useobject{currentmarker}{}%
\end{pgfscope}%
\begin{pgfscope}%
\pgfsys@transformshift{5.033358in}{4.202686in}%
\pgfsys@useobject{currentmarker}{}%
\end{pgfscope}%
\begin{pgfscope}%
\pgfsys@transformshift{5.659974in}{4.372195in}%
\pgfsys@useobject{currentmarker}{}%
\end{pgfscope}%
\begin{pgfscope}%
\pgfsys@transformshift{6.286591in}{4.527578in}%
\pgfsys@useobject{currentmarker}{}%
\end{pgfscope}%
\begin{pgfscope}%
\pgfsys@transformshift{6.913208in}{4.654710in}%
\pgfsys@useobject{currentmarker}{}%
\end{pgfscope}%
\begin{pgfscope}%
\pgfsys@transformshift{7.539825in}{4.767715in}%
\pgfsys@useobject{currentmarker}{}%
\end{pgfscope}%
\end{pgfscope}%
\begin{pgfscope}%
\pgfpathrectangle{\pgfqpoint{1.026143in}{0.898915in}}{\pgfqpoint{6.823857in}{4.114497in}}%
\pgfusepath{clip}%
\pgfsetroundcap%
\pgfsetroundjoin%
\pgfsetlinewidth{1.756562pt}%
\definecolor{currentstroke}{rgb}{0.298039,0.447059,0.690196}%
\pgfsetstrokecolor{currentstroke}%
\pgfsetdash{}{0pt}%
\pgfpathmoveto{\pgfqpoint{1.336319in}{1.085937in}}%
\pgfpathlineto{\pgfqpoint{1.398980in}{1.175730in}}%
\pgfpathlineto{\pgfqpoint{1.461642in}{1.263628in}}%
\pgfpathlineto{\pgfqpoint{1.524304in}{1.349672in}}%
\pgfpathlineto{\pgfqpoint{1.586965in}{1.433900in}}%
\pgfpathlineto{\pgfqpoint{1.649627in}{1.516351in}}%
\pgfpathlineto{\pgfqpoint{1.712289in}{1.597063in}}%
\pgfpathlineto{\pgfqpoint{1.774950in}{1.676071in}}%
\pgfpathlineto{\pgfqpoint{1.837612in}{1.753413in}}%
\pgfpathlineto{\pgfqpoint{1.900274in}{1.829123in}}%
\pgfpathlineto{\pgfqpoint{1.962935in}{1.903235in}}%
\pgfpathlineto{\pgfqpoint{2.025597in}{1.975784in}}%
\pgfpathlineto{\pgfqpoint{2.088259in}{2.046802in}}%
\pgfpathlineto{\pgfqpoint{2.150920in}{2.116322in}}%
\pgfpathlineto{\pgfqpoint{2.213582in}{2.184375in}}%
\pgfpathlineto{\pgfqpoint{2.276244in}{2.250992in}}%
\pgfpathlineto{\pgfqpoint{2.338905in}{2.316203in}}%
\pgfpathlineto{\pgfqpoint{2.401567in}{2.380039in}}%
\pgfpathlineto{\pgfqpoint{2.464229in}{2.442528in}}%
\pgfpathlineto{\pgfqpoint{2.526891in}{2.503698in}}%
\pgfpathlineto{\pgfqpoint{2.589552in}{2.563578in}}%
\pgfpathlineto{\pgfqpoint{2.652214in}{2.622194in}}%
\pgfpathlineto{\pgfqpoint{2.714876in}{2.679574in}}%
\pgfpathlineto{\pgfqpoint{2.777537in}{2.735743in}}%
\pgfpathlineto{\pgfqpoint{2.840199in}{2.790727in}}%
\pgfpathlineto{\pgfqpoint{2.902861in}{2.844551in}}%
\pgfpathlineto{\pgfqpoint{2.965522in}{2.897239in}}%
\pgfpathlineto{\pgfqpoint{3.028184in}{2.948816in}}%
\pgfpathlineto{\pgfqpoint{3.090846in}{2.999304in}}%
\pgfpathlineto{\pgfqpoint{3.153507in}{3.048727in}}%
\pgfpathlineto{\pgfqpoint{3.216169in}{3.097108in}}%
\pgfpathlineto{\pgfqpoint{3.278831in}{3.144468in}}%
\pgfpathlineto{\pgfqpoint{3.341492in}{3.190828in}}%
\pgfpathlineto{\pgfqpoint{3.404154in}{3.236210in}}%
\pgfpathlineto{\pgfqpoint{3.466816in}{3.280635in}}%
\pgfpathlineto{\pgfqpoint{3.529477in}{3.324122in}}%
\pgfpathlineto{\pgfqpoint{3.592139in}{3.366692in}}%
\pgfpathlineto{\pgfqpoint{3.654801in}{3.408364in}}%
\pgfpathlineto{\pgfqpoint{3.717462in}{3.449157in}}%
\pgfpathlineto{\pgfqpoint{3.780124in}{3.489089in}}%
\pgfpathlineto{\pgfqpoint{3.842786in}{3.528178in}}%
\pgfpathlineto{\pgfqpoint{3.905447in}{3.566443in}}%
\pgfpathlineto{\pgfqpoint{3.968109in}{3.603900in}}%
\pgfpathlineto{\pgfqpoint{4.030771in}{3.640567in}}%
\pgfpathlineto{\pgfqpoint{4.093432in}{3.676460in}}%
\pgfpathlineto{\pgfqpoint{4.156094in}{3.711596in}}%
\pgfpathlineto{\pgfqpoint{4.218756in}{3.745991in}}%
\pgfpathlineto{\pgfqpoint{4.281417in}{3.779660in}}%
\pgfpathlineto{\pgfqpoint{4.344079in}{3.812619in}}%
\pgfpathlineto{\pgfqpoint{4.406741in}{3.844882in}}%
\pgfpathlineto{\pgfqpoint{4.469403in}{3.876465in}}%
\pgfpathlineto{\pgfqpoint{4.532064in}{3.907381in}}%
\pgfpathlineto{\pgfqpoint{4.594726in}{3.937645in}}%
\pgfpathlineto{\pgfqpoint{4.657388in}{3.967270in}}%
\pgfpathlineto{\pgfqpoint{4.720049in}{3.996271in}}%
\pgfpathlineto{\pgfqpoint{4.782711in}{4.024659in}}%
\pgfpathlineto{\pgfqpoint{4.845373in}{4.052449in}}%
\pgfpathlineto{\pgfqpoint{4.908034in}{4.079652in}}%
\pgfpathlineto{\pgfqpoint{4.970696in}{4.106281in}}%
\pgfpathlineto{\pgfqpoint{5.033358in}{4.132349in}}%
\pgfpathlineto{\pgfqpoint{5.096019in}{4.157866in}}%
\pgfpathlineto{\pgfqpoint{5.158681in}{4.182845in}}%
\pgfpathlineto{\pgfqpoint{5.221343in}{4.207297in}}%
\pgfpathlineto{\pgfqpoint{5.284004in}{4.231233in}}%
\pgfpathlineto{\pgfqpoint{5.346666in}{4.254664in}}%
\pgfpathlineto{\pgfqpoint{5.409328in}{4.277601in}}%
\pgfpathlineto{\pgfqpoint{5.471989in}{4.300054in}}%
\pgfpathlineto{\pgfqpoint{5.534651in}{4.322033in}}%
\pgfpathlineto{\pgfqpoint{5.597313in}{4.343548in}}%
\pgfpathlineto{\pgfqpoint{5.659974in}{4.364610in}}%
\pgfpathlineto{\pgfqpoint{5.722636in}{4.385227in}}%
\pgfpathlineto{\pgfqpoint{5.785298in}{4.405409in}}%
\pgfpathlineto{\pgfqpoint{5.847959in}{4.425165in}}%
\pgfpathlineto{\pgfqpoint{5.910621in}{4.444505in}}%
\pgfpathlineto{\pgfqpoint{5.973283in}{4.463436in}}%
\pgfpathlineto{\pgfqpoint{6.035944in}{4.481968in}}%
\pgfpathlineto{\pgfqpoint{6.098606in}{4.500109in}}%
\pgfpathlineto{\pgfqpoint{6.161268in}{4.517867in}}%
\pgfpathlineto{\pgfqpoint{6.223929in}{4.535250in}}%
\pgfpathlineto{\pgfqpoint{6.286591in}{4.552267in}}%
\pgfpathlineto{\pgfqpoint{6.349253in}{4.568925in}}%
\pgfpathlineto{\pgfqpoint{6.411915in}{4.585231in}}%
\pgfpathlineto{\pgfqpoint{6.474576in}{4.601193in}}%
\pgfpathlineto{\pgfqpoint{6.537238in}{4.616819in}}%
\pgfpathlineto{\pgfqpoint{6.599900in}{4.632115in}}%
\pgfpathlineto{\pgfqpoint{6.662561in}{4.647088in}}%
\pgfpathlineto{\pgfqpoint{6.725223in}{4.661745in}}%
\pgfpathlineto{\pgfqpoint{6.787885in}{4.676093in}}%
\pgfpathlineto{\pgfqpoint{6.850546in}{4.690138in}}%
\pgfpathlineto{\pgfqpoint{6.913208in}{4.703887in}}%
\pgfpathlineto{\pgfqpoint{6.975870in}{4.717345in}}%
\pgfpathlineto{\pgfqpoint{7.038531in}{4.730520in}}%
\pgfpathlineto{\pgfqpoint{7.101193in}{4.743417in}}%
\pgfpathlineto{\pgfqpoint{7.163855in}{4.756042in}}%
\pgfpathlineto{\pgfqpoint{7.226516in}{4.768400in}}%
\pgfpathlineto{\pgfqpoint{7.289178in}{4.780498in}}%
\pgfpathlineto{\pgfqpoint{7.351840in}{4.792340in}}%
\pgfpathlineto{\pgfqpoint{7.414501in}{4.803932in}}%
\pgfpathlineto{\pgfqpoint{7.477163in}{4.815280in}}%
\pgfpathlineto{\pgfqpoint{7.539825in}{4.826389in}}%
\pgfusepath{stroke}%
\end{pgfscope}%
\begin{pgfscope}%
\pgfsetrectcap%
\pgfsetmiterjoin%
\pgfsetlinewidth{0.000000pt}%
\definecolor{currentstroke}{rgb}{1.000000,1.000000,1.000000}%
\pgfsetstrokecolor{currentstroke}%
\pgfsetdash{}{0pt}%
\pgfpathmoveto{\pgfqpoint{1.026143in}{0.898915in}}%
\pgfpathlineto{\pgfqpoint{1.026143in}{5.013411in}}%
\pgfusepath{}%
\end{pgfscope}%
\begin{pgfscope}%
\pgfsetrectcap%
\pgfsetmiterjoin%
\pgfsetlinewidth{0.000000pt}%
\definecolor{currentstroke}{rgb}{1.000000,1.000000,1.000000}%
\pgfsetstrokecolor{currentstroke}%
\pgfsetdash{}{0pt}%
\pgfpathmoveto{\pgfqpoint{7.850000in}{0.898915in}}%
\pgfpathlineto{\pgfqpoint{7.850000in}{5.013411in}}%
\pgfusepath{}%
\end{pgfscope}%
\begin{pgfscope}%
\pgfsetrectcap%
\pgfsetmiterjoin%
\pgfsetlinewidth{0.000000pt}%
\definecolor{currentstroke}{rgb}{1.000000,1.000000,1.000000}%
\pgfsetstrokecolor{currentstroke}%
\pgfsetdash{}{0pt}%
\pgfpathmoveto{\pgfqpoint{1.026143in}{0.898915in}}%
\pgfpathlineto{\pgfqpoint{7.850000in}{0.898915in}}%
\pgfusepath{}%
\end{pgfscope}%
\begin{pgfscope}%
\pgfsetrectcap%
\pgfsetmiterjoin%
\pgfsetlinewidth{0.000000pt}%
\definecolor{currentstroke}{rgb}{1.000000,1.000000,1.000000}%
\pgfsetstrokecolor{currentstroke}%
\pgfsetdash{}{0pt}%
\pgfpathmoveto{\pgfqpoint{1.026143in}{5.013411in}}%
\pgfpathlineto{\pgfqpoint{7.850000in}{5.013411in}}%
\pgfusepath{}%
\end{pgfscope}%
\begin{pgfscope}%
\definecolor{textcolor}{rgb}{0.150000,0.150000,0.150000}%
\pgfsetstrokecolor{textcolor}%
\pgfsetfillcolor{textcolor}%
\pgftext[x=2.526891in, y=1.716213in, left, base]{\color{textcolor}\rmfamily\fontsize{20.000000}{24.000000}\selectfont M\(\displaystyle _z\)(t) = 6.16(1\(\displaystyle -\)2\(\displaystyle exp[\frac{-t}{T_1}]\))\(\displaystyle +\)-0.49}%
\end{pgfscope}%
\begin{pgfscope}%
\definecolor{textcolor}{rgb}{0.150000,0.150000,0.150000}%
\pgfsetstrokecolor{textcolor}%
\pgfsetfillcolor{textcolor}%
\pgftext[x=2.526891in, y=1.222157in, left, base]{\color{textcolor}\rmfamily\fontsize{20.000000}{24.000000}\selectfont T\(\displaystyle _1\) = 46.8\(\displaystyle \pm\)3.7 ms}%
\end{pgfscope}%
\begin{pgfscope}%
\definecolor{textcolor}{rgb}{0.150000,0.150000,0.150000}%
\pgfsetstrokecolor{textcolor}%
\pgfsetfillcolor{textcolor}%
\pgftext[x=4.438072in,y=5.096745in,,base]{\color{textcolor}\rmfamily\fontsize{24.000000}{28.800000}\selectfont T\(\displaystyle _1\) measurement of Mineral Oil}%
\end{pgfscope}%
\begin{pgfscope}%
\pgfsetbuttcap%
\pgfsetroundjoin%
\definecolor{currentfill}{rgb}{0.392157,0.584314,0.929412}%
\pgfsetfillcolor{currentfill}%
\pgfsetlinewidth{0.301125pt}%
\definecolor{currentstroke}{rgb}{0.392157,0.584314,0.929412}%
\pgfsetstrokecolor{currentstroke}%
\pgfsetdash{}{0pt}%
\pgfsys@defobject{currentmarker}{\pgfqpoint{-0.048611in}{-0.048611in}}{\pgfqpoint{0.048611in}{0.048611in}}{%
\pgfpathmoveto{\pgfqpoint{0.000000in}{-0.048611in}}%
\pgfpathcurveto{\pgfqpoint{0.012892in}{-0.048611in}}{\pgfqpoint{0.025257in}{-0.043489in}}{\pgfqpoint{0.034373in}{-0.034373in}}%
\pgfpathcurveto{\pgfqpoint{0.043489in}{-0.025257in}}{\pgfqpoint{0.048611in}{-0.012892in}}{\pgfqpoint{0.048611in}{0.000000in}}%
\pgfpathcurveto{\pgfqpoint{0.048611in}{0.012892in}}{\pgfqpoint{0.043489in}{0.025257in}}{\pgfqpoint{0.034373in}{0.034373in}}%
\pgfpathcurveto{\pgfqpoint{0.025257in}{0.043489in}}{\pgfqpoint{0.012892in}{0.048611in}}{\pgfqpoint{0.000000in}{0.048611in}}%
\pgfpathcurveto{\pgfqpoint{-0.012892in}{0.048611in}}{\pgfqpoint{-0.025257in}{0.043489in}}{\pgfqpoint{-0.034373in}{0.034373in}}%
\pgfpathcurveto{\pgfqpoint{-0.043489in}{0.025257in}}{\pgfqpoint{-0.048611in}{0.012892in}}{\pgfqpoint{-0.048611in}{0.000000in}}%
\pgfpathcurveto{\pgfqpoint{-0.048611in}{-0.012892in}}{\pgfqpoint{-0.043489in}{-0.025257in}}{\pgfqpoint{-0.034373in}{-0.034373in}}%
\pgfpathcurveto{\pgfqpoint{-0.025257in}{-0.043489in}}{\pgfqpoint{-0.012892in}{-0.048611in}}{\pgfqpoint{0.000000in}{-0.048611in}}%
\pgfpathlineto{\pgfqpoint{0.000000in}{-0.048611in}}%
\pgfpathclose%
\pgfusepath{stroke,fill}%
}%
\begin{pgfscope}%
\pgfsys@transformshift{1.553921in}{4.625282in}%
\pgfsys@useobject{currentmarker}{}%
\end{pgfscope}%
\end{pgfscope}%
\begin{pgfscope}%
\definecolor{textcolor}{rgb}{0.150000,0.150000,0.150000}%
\pgfsetstrokecolor{textcolor}%
\pgfsetfillcolor{textcolor}%
\pgftext[x=2.053921in,y=4.552365in,left,base]{\color{textcolor}\rmfamily\fontsize{20.000000}{24.000000}\selectfont Data}%
\end{pgfscope}%
\begin{pgfscope}%
\pgfsetroundcap%
\pgfsetroundjoin%
\pgfsetlinewidth{1.756562pt}%
\definecolor{currentstroke}{rgb}{0.298039,0.447059,0.690196}%
\pgfsetstrokecolor{currentstroke}%
\pgfsetdash{}{0pt}%
\pgfpathmoveto{\pgfqpoint{1.276143in}{4.241873in}}%
\pgfpathlineto{\pgfqpoint{1.553921in}{4.241873in}}%
\pgfpathlineto{\pgfqpoint{1.831699in}{4.241873in}}%
\pgfusepath{stroke}%
\end{pgfscope}%
\begin{pgfscope}%
\definecolor{textcolor}{rgb}{0.150000,0.150000,0.150000}%
\pgfsetstrokecolor{textcolor}%
\pgfsetfillcolor{textcolor}%
\pgftext[x=2.053921in,y=4.144651in,left,base]{\color{textcolor}\rmfamily\fontsize{20.000000}{24.000000}\selectfont Fitted Curve}%
\end{pgfscope}%
\end{pgfpicture}%
\makeatother%
\endgroup%
}
    \caption{T1 measurement of mineral oil using the two-pulse method }
    \label{fig:mo_t1}
  \end{figure}

  \begin{figure}[h]
    \resizebox{0.45\textwidth}{!}{%% Creator: Matplotlib, PGF backend
%%
%% To include the figure in your LaTeX document, write
%%   \input{<filename>.pgf}
%%
%% Make sure the required packages are loaded in your preamble
%%   \usepackage{pgf}
%%
%% Also ensure that all the required font packages are loaded; for instance,
%% the lmodern package is sometimes necessary when using math font.
%%   \usepackage{lmodern}
%%
%% Figures using additional raster images can only be included by \input if
%% they are in the same directory as the main LaTeX file. For loading figures
%% from other directories you can use the `import` package
%%   \usepackage{import}
%%
%% and then include the figures with
%%   \import{<path to file>}{<filename>.pgf}
%%
%% Matplotlib used the following preamble
%%   \usepackage{fontspec}
%%   \setmainfont{DejaVuSerif.ttf}[Path=\detokenize{/usr/share/matplotlib/mpl-data/fonts/ttf/}]
%%   \setsansfont{DejaVuSans.ttf}[Path=\detokenize{/usr/share/matplotlib/mpl-data/fonts/ttf/}]
%%   \setmonofont{DejaVuSansMono.ttf}[Path=\detokenize{/usr/share/matplotlib/mpl-data/fonts/ttf/}]
%%
\begingroup%
\makeatletter%
\begin{pgfpicture}%
\pgfpathrectangle{\pgfpointorigin}{\pgfqpoint{8.000000in}{5.500000in}}%
\pgfusepath{use as bounding box, clip}%
\begin{pgfscope}%
\pgfsetbuttcap%
\pgfsetmiterjoin%
\definecolor{currentfill}{rgb}{1.000000,1.000000,1.000000}%
\pgfsetfillcolor{currentfill}%
\pgfsetlinewidth{0.000000pt}%
\definecolor{currentstroke}{rgb}{1.000000,1.000000,1.000000}%
\pgfsetstrokecolor{currentstroke}%
\pgfsetdash{}{0pt}%
\pgfpathmoveto{\pgfqpoint{0.000000in}{0.000000in}}%
\pgfpathlineto{\pgfqpoint{8.000000in}{0.000000in}}%
\pgfpathlineto{\pgfqpoint{8.000000in}{5.500000in}}%
\pgfpathlineto{\pgfqpoint{0.000000in}{5.500000in}}%
\pgfpathlineto{\pgfqpoint{0.000000in}{0.000000in}}%
\pgfpathclose%
\pgfusepath{fill}%
\end{pgfscope}%
\begin{pgfscope}%
\pgfsetbuttcap%
\pgfsetmiterjoin%
\definecolor{currentfill}{rgb}{0.917647,0.917647,0.949020}%
\pgfsetfillcolor{currentfill}%
\pgfsetlinewidth{0.000000pt}%
\definecolor{currentstroke}{rgb}{0.000000,0.000000,0.000000}%
\pgfsetstrokecolor{currentstroke}%
\pgfsetstrokeopacity{0.000000}%
\pgfsetdash{}{0pt}%
\pgfpathmoveto{\pgfqpoint{0.802099in}{0.894194in}}%
\pgfpathlineto{\pgfqpoint{7.675188in}{0.894194in}}%
\pgfpathlineto{\pgfqpoint{7.675188in}{5.013411in}}%
\pgfpathlineto{\pgfqpoint{0.802099in}{5.013411in}}%
\pgfpathlineto{\pgfqpoint{0.802099in}{0.894194in}}%
\pgfpathclose%
\pgfusepath{fill}%
\end{pgfscope}%
\begin{pgfscope}%
\pgfpathrectangle{\pgfqpoint{0.802099in}{0.894194in}}{\pgfqpoint{6.873088in}{4.119217in}}%
\pgfusepath{clip}%
\pgfsetroundcap%
\pgfsetroundjoin%
\pgfsetlinewidth{1.003750pt}%
\definecolor{currentstroke}{rgb}{1.000000,1.000000,1.000000}%
\pgfsetstrokecolor{currentstroke}%
\pgfsetdash{}{0pt}%
\pgfpathmoveto{\pgfqpoint{1.008610in}{0.894194in}}%
\pgfpathlineto{\pgfqpoint{1.008610in}{5.013411in}}%
\pgfusepath{stroke}%
\end{pgfscope}%
\begin{pgfscope}%
\definecolor{textcolor}{rgb}{0.150000,0.150000,0.150000}%
\pgfsetstrokecolor{textcolor}%
\pgfsetfillcolor{textcolor}%
\pgftext[x=1.008610in,y=0.796972in,,top]{\color{textcolor}\rmfamily\fontsize{20.000000}{24.000000}\selectfont 0}%
\end{pgfscope}%
\begin{pgfscope}%
\pgfpathrectangle{\pgfqpoint{0.802099in}{0.894194in}}{\pgfqpoint{6.873088in}{4.119217in}}%
\pgfusepath{clip}%
\pgfsetroundcap%
\pgfsetroundjoin%
\pgfsetlinewidth{1.003750pt}%
\definecolor{currentstroke}{rgb}{1.000000,1.000000,1.000000}%
\pgfsetstrokecolor{currentstroke}%
\pgfsetdash{}{0pt}%
\pgfpathmoveto{\pgfqpoint{2.420646in}{0.894194in}}%
\pgfpathlineto{\pgfqpoint{2.420646in}{5.013411in}}%
\pgfusepath{stroke}%
\end{pgfscope}%
\begin{pgfscope}%
\definecolor{textcolor}{rgb}{0.150000,0.150000,0.150000}%
\pgfsetstrokecolor{textcolor}%
\pgfsetfillcolor{textcolor}%
\pgftext[x=2.420646in,y=0.796972in,,top]{\color{textcolor}\rmfamily\fontsize{20.000000}{24.000000}\selectfont 20}%
\end{pgfscope}%
\begin{pgfscope}%
\pgfpathrectangle{\pgfqpoint{0.802099in}{0.894194in}}{\pgfqpoint{6.873088in}{4.119217in}}%
\pgfusepath{clip}%
\pgfsetroundcap%
\pgfsetroundjoin%
\pgfsetlinewidth{1.003750pt}%
\definecolor{currentstroke}{rgb}{1.000000,1.000000,1.000000}%
\pgfsetstrokecolor{currentstroke}%
\pgfsetdash{}{0pt}%
\pgfpathmoveto{\pgfqpoint{3.832683in}{0.894194in}}%
\pgfpathlineto{\pgfqpoint{3.832683in}{5.013411in}}%
\pgfusepath{stroke}%
\end{pgfscope}%
\begin{pgfscope}%
\definecolor{textcolor}{rgb}{0.150000,0.150000,0.150000}%
\pgfsetstrokecolor{textcolor}%
\pgfsetfillcolor{textcolor}%
\pgftext[x=3.832683in,y=0.796972in,,top]{\color{textcolor}\rmfamily\fontsize{20.000000}{24.000000}\selectfont 40}%
\end{pgfscope}%
\begin{pgfscope}%
\pgfpathrectangle{\pgfqpoint{0.802099in}{0.894194in}}{\pgfqpoint{6.873088in}{4.119217in}}%
\pgfusepath{clip}%
\pgfsetroundcap%
\pgfsetroundjoin%
\pgfsetlinewidth{1.003750pt}%
\definecolor{currentstroke}{rgb}{1.000000,1.000000,1.000000}%
\pgfsetstrokecolor{currentstroke}%
\pgfsetdash{}{0pt}%
\pgfpathmoveto{\pgfqpoint{5.244720in}{0.894194in}}%
\pgfpathlineto{\pgfqpoint{5.244720in}{5.013411in}}%
\pgfusepath{stroke}%
\end{pgfscope}%
\begin{pgfscope}%
\definecolor{textcolor}{rgb}{0.150000,0.150000,0.150000}%
\pgfsetstrokecolor{textcolor}%
\pgfsetfillcolor{textcolor}%
\pgftext[x=5.244720in,y=0.796972in,,top]{\color{textcolor}\rmfamily\fontsize{20.000000}{24.000000}\selectfont 60}%
\end{pgfscope}%
\begin{pgfscope}%
\pgfpathrectangle{\pgfqpoint{0.802099in}{0.894194in}}{\pgfqpoint{6.873088in}{4.119217in}}%
\pgfusepath{clip}%
\pgfsetroundcap%
\pgfsetroundjoin%
\pgfsetlinewidth{1.003750pt}%
\definecolor{currentstroke}{rgb}{1.000000,1.000000,1.000000}%
\pgfsetstrokecolor{currentstroke}%
\pgfsetdash{}{0pt}%
\pgfpathmoveto{\pgfqpoint{6.656756in}{0.894194in}}%
\pgfpathlineto{\pgfqpoint{6.656756in}{5.013411in}}%
\pgfusepath{stroke}%
\end{pgfscope}%
\begin{pgfscope}%
\definecolor{textcolor}{rgb}{0.150000,0.150000,0.150000}%
\pgfsetstrokecolor{textcolor}%
\pgfsetfillcolor{textcolor}%
\pgftext[x=6.656756in,y=0.796972in,,top]{\color{textcolor}\rmfamily\fontsize{20.000000}{24.000000}\selectfont 80}%
\end{pgfscope}%
\begin{pgfscope}%
\definecolor{textcolor}{rgb}{0.150000,0.150000,0.150000}%
\pgfsetstrokecolor{textcolor}%
\pgfsetfillcolor{textcolor}%
\pgftext[x=4.238644in,y=0.472591in,,top]{\color{textcolor}\rmfamily\fontsize{24.000000}{28.800000}\selectfont Time (ms)}%
\end{pgfscope}%
\begin{pgfscope}%
\pgfpathrectangle{\pgfqpoint{0.802099in}{0.894194in}}{\pgfqpoint{6.873088in}{4.119217in}}%
\pgfusepath{clip}%
\pgfsetroundcap%
\pgfsetroundjoin%
\pgfsetlinewidth{1.003750pt}%
\definecolor{currentstroke}{rgb}{1.000000,1.000000,1.000000}%
\pgfsetstrokecolor{currentstroke}%
\pgfsetdash{}{0pt}%
\pgfpathmoveto{\pgfqpoint{0.802099in}{1.529746in}}%
\pgfpathlineto{\pgfqpoint{7.675188in}{1.529746in}}%
\pgfusepath{stroke}%
\end{pgfscope}%
\begin{pgfscope}%
\definecolor{textcolor}{rgb}{0.150000,0.150000,0.150000}%
\pgfsetstrokecolor{textcolor}%
\pgfsetfillcolor{textcolor}%
\pgftext[x=0.528147in, y=1.424223in, left, base]{\color{textcolor}\rmfamily\fontsize{20.000000}{24.000000}\selectfont 1}%
\end{pgfscope}%
\begin{pgfscope}%
\pgfpathrectangle{\pgfqpoint{0.802099in}{0.894194in}}{\pgfqpoint{6.873088in}{4.119217in}}%
\pgfusepath{clip}%
\pgfsetroundcap%
\pgfsetroundjoin%
\pgfsetlinewidth{1.003750pt}%
\definecolor{currentstroke}{rgb}{1.000000,1.000000,1.000000}%
\pgfsetstrokecolor{currentstroke}%
\pgfsetdash{}{0pt}%
\pgfpathmoveto{\pgfqpoint{0.802099in}{2.189032in}}%
\pgfpathlineto{\pgfqpoint{7.675188in}{2.189032in}}%
\pgfusepath{stroke}%
\end{pgfscope}%
\begin{pgfscope}%
\definecolor{textcolor}{rgb}{0.150000,0.150000,0.150000}%
\pgfsetstrokecolor{textcolor}%
\pgfsetfillcolor{textcolor}%
\pgftext[x=0.528147in, y=2.083509in, left, base]{\color{textcolor}\rmfamily\fontsize{20.000000}{24.000000}\selectfont 2}%
\end{pgfscope}%
\begin{pgfscope}%
\pgfpathrectangle{\pgfqpoint{0.802099in}{0.894194in}}{\pgfqpoint{6.873088in}{4.119217in}}%
\pgfusepath{clip}%
\pgfsetroundcap%
\pgfsetroundjoin%
\pgfsetlinewidth{1.003750pt}%
\definecolor{currentstroke}{rgb}{1.000000,1.000000,1.000000}%
\pgfsetstrokecolor{currentstroke}%
\pgfsetdash{}{0pt}%
\pgfpathmoveto{\pgfqpoint{0.802099in}{2.848317in}}%
\pgfpathlineto{\pgfqpoint{7.675188in}{2.848317in}}%
\pgfusepath{stroke}%
\end{pgfscope}%
\begin{pgfscope}%
\definecolor{textcolor}{rgb}{0.150000,0.150000,0.150000}%
\pgfsetstrokecolor{textcolor}%
\pgfsetfillcolor{textcolor}%
\pgftext[x=0.528147in, y=2.742794in, left, base]{\color{textcolor}\rmfamily\fontsize{20.000000}{24.000000}\selectfont 3}%
\end{pgfscope}%
\begin{pgfscope}%
\pgfpathrectangle{\pgfqpoint{0.802099in}{0.894194in}}{\pgfqpoint{6.873088in}{4.119217in}}%
\pgfusepath{clip}%
\pgfsetroundcap%
\pgfsetroundjoin%
\pgfsetlinewidth{1.003750pt}%
\definecolor{currentstroke}{rgb}{1.000000,1.000000,1.000000}%
\pgfsetstrokecolor{currentstroke}%
\pgfsetdash{}{0pt}%
\pgfpathmoveto{\pgfqpoint{0.802099in}{3.507603in}}%
\pgfpathlineto{\pgfqpoint{7.675188in}{3.507603in}}%
\pgfusepath{stroke}%
\end{pgfscope}%
\begin{pgfscope}%
\definecolor{textcolor}{rgb}{0.150000,0.150000,0.150000}%
\pgfsetstrokecolor{textcolor}%
\pgfsetfillcolor{textcolor}%
\pgftext[x=0.528147in, y=3.402080in, left, base]{\color{textcolor}\rmfamily\fontsize{20.000000}{24.000000}\selectfont 4}%
\end{pgfscope}%
\begin{pgfscope}%
\pgfpathrectangle{\pgfqpoint{0.802099in}{0.894194in}}{\pgfqpoint{6.873088in}{4.119217in}}%
\pgfusepath{clip}%
\pgfsetroundcap%
\pgfsetroundjoin%
\pgfsetlinewidth{1.003750pt}%
\definecolor{currentstroke}{rgb}{1.000000,1.000000,1.000000}%
\pgfsetstrokecolor{currentstroke}%
\pgfsetdash{}{0pt}%
\pgfpathmoveto{\pgfqpoint{0.802099in}{4.166889in}}%
\pgfpathlineto{\pgfqpoint{7.675188in}{4.166889in}}%
\pgfusepath{stroke}%
\end{pgfscope}%
\begin{pgfscope}%
\definecolor{textcolor}{rgb}{0.150000,0.150000,0.150000}%
\pgfsetstrokecolor{textcolor}%
\pgfsetfillcolor{textcolor}%
\pgftext[x=0.528147in, y=4.061366in, left, base]{\color{textcolor}\rmfamily\fontsize{20.000000}{24.000000}\selectfont 5}%
\end{pgfscope}%
\begin{pgfscope}%
\pgfpathrectangle{\pgfqpoint{0.802099in}{0.894194in}}{\pgfqpoint{6.873088in}{4.119217in}}%
\pgfusepath{clip}%
\pgfsetroundcap%
\pgfsetroundjoin%
\pgfsetlinewidth{1.003750pt}%
\definecolor{currentstroke}{rgb}{1.000000,1.000000,1.000000}%
\pgfsetstrokecolor{currentstroke}%
\pgfsetdash{}{0pt}%
\pgfpathmoveto{\pgfqpoint{0.802099in}{4.826174in}}%
\pgfpathlineto{\pgfqpoint{7.675188in}{4.826174in}}%
\pgfusepath{stroke}%
\end{pgfscope}%
\begin{pgfscope}%
\definecolor{textcolor}{rgb}{0.150000,0.150000,0.150000}%
\pgfsetstrokecolor{textcolor}%
\pgfsetfillcolor{textcolor}%
\pgftext[x=0.528147in, y=4.720651in, left, base]{\color{textcolor}\rmfamily\fontsize{20.000000}{24.000000}\selectfont 6}%
\end{pgfscope}%
\begin{pgfscope}%
\definecolor{textcolor}{rgb}{0.150000,0.150000,0.150000}%
\pgfsetstrokecolor{textcolor}%
\pgfsetfillcolor{textcolor}%
\pgftext[x=0.472591in,y=2.953803in,,bottom,rotate=90.000000]{\color{textcolor}\rmfamily\fontsize{24.000000}{28.800000}\selectfont Pulse Amplitude (V)}%
\end{pgfscope}%
\begin{pgfscope}%
\pgfpathrectangle{\pgfqpoint{0.802099in}{0.894194in}}{\pgfqpoint{6.873088in}{4.119217in}}%
\pgfusepath{clip}%
\pgfsetbuttcap%
\pgfsetroundjoin%
\definecolor{currentfill}{rgb}{0.698039,0.133333,0.133333}%
\pgfsetfillcolor{currentfill}%
\pgfsetlinewidth{0.301125pt}%
\definecolor{currentstroke}{rgb}{0.698039,0.133333,0.133333}%
\pgfsetstrokecolor{currentstroke}%
\pgfsetdash{}{0pt}%
\pgfsys@defobject{currentmarker}{\pgfqpoint{-0.048611in}{-0.048611in}}{\pgfqpoint{0.048611in}{0.048611in}}{%
\pgfpathmoveto{\pgfqpoint{0.000000in}{-0.048611in}}%
\pgfpathcurveto{\pgfqpoint{0.012892in}{-0.048611in}}{\pgfqpoint{0.025257in}{-0.043489in}}{\pgfqpoint{0.034373in}{-0.034373in}}%
\pgfpathcurveto{\pgfqpoint{0.043489in}{-0.025257in}}{\pgfqpoint{0.048611in}{-0.012892in}}{\pgfqpoint{0.048611in}{0.000000in}}%
\pgfpathcurveto{\pgfqpoint{0.048611in}{0.012892in}}{\pgfqpoint{0.043489in}{0.025257in}}{\pgfqpoint{0.034373in}{0.034373in}}%
\pgfpathcurveto{\pgfqpoint{0.025257in}{0.043489in}}{\pgfqpoint{0.012892in}{0.048611in}}{\pgfqpoint{0.000000in}{0.048611in}}%
\pgfpathcurveto{\pgfqpoint{-0.012892in}{0.048611in}}{\pgfqpoint{-0.025257in}{0.043489in}}{\pgfqpoint{-0.034373in}{0.034373in}}%
\pgfpathcurveto{\pgfqpoint{-0.043489in}{0.025257in}}{\pgfqpoint{-0.048611in}{0.012892in}}{\pgfqpoint{-0.048611in}{0.000000in}}%
\pgfpathcurveto{\pgfqpoint{-0.048611in}{-0.012892in}}{\pgfqpoint{-0.043489in}{-0.025257in}}{\pgfqpoint{-0.034373in}{-0.034373in}}%
\pgfpathcurveto{\pgfqpoint{-0.025257in}{-0.043489in}}{\pgfqpoint{-0.012892in}{-0.048611in}}{\pgfqpoint{0.000000in}{-0.048611in}}%
\pgfpathlineto{\pgfqpoint{0.000000in}{-0.048611in}}%
\pgfpathclose%
\pgfusepath{stroke,fill}%
}%
\begin{pgfscope}%
\pgfsys@transformshift{1.114513in}{4.826174in}%
\pgfsys@useobject{currentmarker}{}%
\end{pgfscope}%
\begin{pgfscope}%
\pgfsys@transformshift{1.714628in}{3.692203in}%
\pgfsys@useobject{currentmarker}{}%
\end{pgfscope}%
\begin{pgfscope}%
\pgfsys@transformshift{2.420646in}{2.874689in}%
\pgfsys@useobject{currentmarker}{}%
\end{pgfscope}%
\begin{pgfscope}%
\pgfsys@transformshift{3.126665in}{2.347260in}%
\pgfsys@useobject{currentmarker}{}%
\end{pgfscope}%
\begin{pgfscope}%
\pgfsys@transformshift{3.832683in}{1.978060in}%
\pgfsys@useobject{currentmarker}{}%
\end{pgfscope}%
\begin{pgfscope}%
\pgfsys@transformshift{4.538701in}{1.740717in}%
\pgfsys@useobject{currentmarker}{}%
\end{pgfscope}%
\begin{pgfscope}%
\pgfsys@transformshift{5.244720in}{1.450632in}%
\pgfsys@useobject{currentmarker}{}%
\end{pgfscope}%
\begin{pgfscope}%
\pgfsys@transformshift{5.950738in}{1.318774in}%
\pgfsys@useobject{currentmarker}{}%
\end{pgfscope}%
\begin{pgfscope}%
\pgfsys@transformshift{6.656756in}{1.213289in}%
\pgfsys@useobject{currentmarker}{}%
\end{pgfscope}%
\begin{pgfscope}%
\pgfsys@transformshift{7.362775in}{1.081432in}%
\pgfsys@useobject{currentmarker}{}%
\end{pgfscope}%
\end{pgfscope}%
\begin{pgfscope}%
\pgfpathrectangle{\pgfqpoint{0.802099in}{0.894194in}}{\pgfqpoint{6.873088in}{4.119217in}}%
\pgfusepath{clip}%
\pgfsetroundcap%
\pgfsetroundjoin%
\pgfsetlinewidth{1.756562pt}%
\definecolor{currentstroke}{rgb}{0.298039,0.447059,0.690196}%
\pgfsetstrokecolor{currentstroke}%
\pgfsetdash{}{0pt}%
\pgfpathmoveto{\pgfqpoint{1.114513in}{4.767111in}}%
\pgfpathlineto{\pgfqpoint{1.177626in}{4.647254in}}%
\pgfpathlineto{\pgfqpoint{1.240740in}{4.531207in}}%
\pgfpathlineto{\pgfqpoint{1.303854in}{4.418848in}}%
\pgfpathlineto{\pgfqpoint{1.366968in}{4.310059in}}%
\pgfpathlineto{\pgfqpoint{1.430081in}{4.204728in}}%
\pgfpathlineto{\pgfqpoint{1.493195in}{4.102744in}}%
\pgfpathlineto{\pgfqpoint{1.556309in}{4.004001in}}%
\pgfpathlineto{\pgfqpoint{1.619423in}{3.908396in}}%
\pgfpathlineto{\pgfqpoint{1.682536in}{3.815830in}}%
\pgfpathlineto{\pgfqpoint{1.745650in}{3.726205in}}%
\pgfpathlineto{\pgfqpoint{1.808764in}{3.639428in}}%
\pgfpathlineto{\pgfqpoint{1.871878in}{3.555409in}}%
\pgfpathlineto{\pgfqpoint{1.934991in}{3.474061in}}%
\pgfpathlineto{\pgfqpoint{1.998105in}{3.395297in}}%
\pgfpathlineto{\pgfqpoint{2.061219in}{3.319037in}}%
\pgfpathlineto{\pgfqpoint{2.124333in}{3.245200in}}%
\pgfpathlineto{\pgfqpoint{2.187446in}{3.173710in}}%
\pgfpathlineto{\pgfqpoint{2.250560in}{3.104491in}}%
\pgfpathlineto{\pgfqpoint{2.313674in}{3.037473in}}%
\pgfpathlineto{\pgfqpoint{2.376788in}{2.972584in}}%
\pgfpathlineto{\pgfqpoint{2.439901in}{2.909757in}}%
\pgfpathlineto{\pgfqpoint{2.503015in}{2.848927in}}%
\pgfpathlineto{\pgfqpoint{2.566129in}{2.790031in}}%
\pgfpathlineto{\pgfqpoint{2.629243in}{2.733006in}}%
\pgfpathlineto{\pgfqpoint{2.692357in}{2.677793in}}%
\pgfpathlineto{\pgfqpoint{2.755470in}{2.624334in}}%
\pgfpathlineto{\pgfqpoint{2.818584in}{2.572575in}}%
\pgfpathlineto{\pgfqpoint{2.881698in}{2.522461in}}%
\pgfpathlineto{\pgfqpoint{2.944812in}{2.473939in}}%
\pgfpathlineto{\pgfqpoint{3.007925in}{2.426959in}}%
\pgfpathlineto{\pgfqpoint{3.071039in}{2.381472in}}%
\pgfpathlineto{\pgfqpoint{3.134153in}{2.337431in}}%
\pgfpathlineto{\pgfqpoint{3.197267in}{2.294790in}}%
\pgfpathlineto{\pgfqpoint{3.260380in}{2.253503in}}%
\pgfpathlineto{\pgfqpoint{3.323494in}{2.213529in}}%
\pgfpathlineto{\pgfqpoint{3.386608in}{2.174825in}}%
\pgfpathlineto{\pgfqpoint{3.449722in}{2.137351in}}%
\pgfpathlineto{\pgfqpoint{3.512835in}{2.101068in}}%
\pgfpathlineto{\pgfqpoint{3.575949in}{2.065938in}}%
\pgfpathlineto{\pgfqpoint{3.639063in}{2.031924in}}%
\pgfpathlineto{\pgfqpoint{3.702177in}{1.998992in}}%
\pgfpathlineto{\pgfqpoint{3.765290in}{1.967106in}}%
\pgfpathlineto{\pgfqpoint{3.828404in}{1.936233in}}%
\pgfpathlineto{\pgfqpoint{3.891518in}{1.906341in}}%
\pgfpathlineto{\pgfqpoint{3.954632in}{1.877400in}}%
\pgfpathlineto{\pgfqpoint{4.017745in}{1.849378in}}%
\pgfpathlineto{\pgfqpoint{4.080859in}{1.822247in}}%
\pgfpathlineto{\pgfqpoint{4.143973in}{1.795977in}}%
\pgfpathlineto{\pgfqpoint{4.207087in}{1.770543in}}%
\pgfpathlineto{\pgfqpoint{4.270200in}{1.745917in}}%
\pgfpathlineto{\pgfqpoint{4.333314in}{1.722074in}}%
\pgfpathlineto{\pgfqpoint{4.396428in}{1.698988in}}%
\pgfpathlineto{\pgfqpoint{4.459542in}{1.676636in}}%
\pgfpathlineto{\pgfqpoint{4.522656in}{1.654995in}}%
\pgfpathlineto{\pgfqpoint{4.585769in}{1.634041in}}%
\pgfpathlineto{\pgfqpoint{4.648883in}{1.613753in}}%
\pgfpathlineto{\pgfqpoint{4.711997in}{1.594110in}}%
\pgfpathlineto{\pgfqpoint{4.775111in}{1.575091in}}%
\pgfpathlineto{\pgfqpoint{4.838224in}{1.556676in}}%
\pgfpathlineto{\pgfqpoint{4.901338in}{1.538847in}}%
\pgfpathlineto{\pgfqpoint{4.964452in}{1.521584in}}%
\pgfpathlineto{\pgfqpoint{5.027566in}{1.504870in}}%
\pgfpathlineto{\pgfqpoint{5.090679in}{1.488687in}}%
\pgfpathlineto{\pgfqpoint{5.153793in}{1.473018in}}%
\pgfpathlineto{\pgfqpoint{5.216907in}{1.457848in}}%
\pgfpathlineto{\pgfqpoint{5.280021in}{1.443159in}}%
\pgfpathlineto{\pgfqpoint{5.343134in}{1.428937in}}%
\pgfpathlineto{\pgfqpoint{5.406248in}{1.415168in}}%
\pgfpathlineto{\pgfqpoint{5.469362in}{1.401835in}}%
\pgfpathlineto{\pgfqpoint{5.532476in}{1.388927in}}%
\pgfpathlineto{\pgfqpoint{5.595589in}{1.376429in}}%
\pgfpathlineto{\pgfqpoint{5.658703in}{1.364327in}}%
\pgfpathlineto{\pgfqpoint{5.721817in}{1.352611in}}%
\pgfpathlineto{\pgfqpoint{5.784931in}{1.341267in}}%
\pgfpathlineto{\pgfqpoint{5.848044in}{1.330283in}}%
\pgfpathlineto{\pgfqpoint{5.911158in}{1.319648in}}%
\pgfpathlineto{\pgfqpoint{5.974272in}{1.309352in}}%
\pgfpathlineto{\pgfqpoint{6.037386in}{1.299382in}}%
\pgfpathlineto{\pgfqpoint{6.100499in}{1.289730in}}%
\pgfpathlineto{\pgfqpoint{6.163613in}{1.280384in}}%
\pgfpathlineto{\pgfqpoint{6.226727in}{1.271335in}}%
\pgfpathlineto{\pgfqpoint{6.289841in}{1.262574in}}%
\pgfpathlineto{\pgfqpoint{6.352955in}{1.254091in}}%
\pgfpathlineto{\pgfqpoint{6.416068in}{1.245878in}}%
\pgfpathlineto{\pgfqpoint{6.479182in}{1.237926in}}%
\pgfpathlineto{\pgfqpoint{6.542296in}{1.230226in}}%
\pgfpathlineto{\pgfqpoint{6.605410in}{1.222771in}}%
\pgfpathlineto{\pgfqpoint{6.668523in}{1.215554in}}%
\pgfpathlineto{\pgfqpoint{6.731637in}{1.208565in}}%
\pgfpathlineto{\pgfqpoint{6.794751in}{1.201799in}}%
\pgfpathlineto{\pgfqpoint{6.857865in}{1.195247in}}%
\pgfpathlineto{\pgfqpoint{6.920978in}{1.188904in}}%
\pgfpathlineto{\pgfqpoint{6.984092in}{1.182762in}}%
\pgfpathlineto{\pgfqpoint{7.047206in}{1.176816in}}%
\pgfpathlineto{\pgfqpoint{7.110320in}{1.171059in}}%
\pgfpathlineto{\pgfqpoint{7.173433in}{1.165484in}}%
\pgfpathlineto{\pgfqpoint{7.236547in}{1.160087in}}%
\pgfpathlineto{\pgfqpoint{7.299661in}{1.154861in}}%
\pgfpathlineto{\pgfqpoint{7.362775in}{1.149801in}}%
\pgfusepath{stroke}%
\end{pgfscope}%
\begin{pgfscope}%
\pgfsetrectcap%
\pgfsetmiterjoin%
\pgfsetlinewidth{0.000000pt}%
\definecolor{currentstroke}{rgb}{1.000000,1.000000,1.000000}%
\pgfsetstrokecolor{currentstroke}%
\pgfsetdash{}{0pt}%
\pgfpathmoveto{\pgfqpoint{0.802099in}{0.894194in}}%
\pgfpathlineto{\pgfqpoint{0.802099in}{5.013411in}}%
\pgfusepath{}%
\end{pgfscope}%
\begin{pgfscope}%
\pgfsetrectcap%
\pgfsetmiterjoin%
\pgfsetlinewidth{0.000000pt}%
\definecolor{currentstroke}{rgb}{1.000000,1.000000,1.000000}%
\pgfsetstrokecolor{currentstroke}%
\pgfsetdash{}{0pt}%
\pgfpathmoveto{\pgfqpoint{7.675188in}{0.894194in}}%
\pgfpathlineto{\pgfqpoint{7.675188in}{5.013411in}}%
\pgfusepath{}%
\end{pgfscope}%
\begin{pgfscope}%
\pgfsetrectcap%
\pgfsetmiterjoin%
\pgfsetlinewidth{0.000000pt}%
\definecolor{currentstroke}{rgb}{1.000000,1.000000,1.000000}%
\pgfsetstrokecolor{currentstroke}%
\pgfsetdash{}{0pt}%
\pgfpathmoveto{\pgfqpoint{0.802099in}{0.894194in}}%
\pgfpathlineto{\pgfqpoint{7.675188in}{0.894194in}}%
\pgfusepath{}%
\end{pgfscope}%
\begin{pgfscope}%
\pgfsetrectcap%
\pgfsetmiterjoin%
\pgfsetlinewidth{0.000000pt}%
\definecolor{currentstroke}{rgb}{1.000000,1.000000,1.000000}%
\pgfsetstrokecolor{currentstroke}%
\pgfsetdash{}{0pt}%
\pgfpathmoveto{\pgfqpoint{0.802099in}{5.013411in}}%
\pgfpathlineto{\pgfqpoint{7.675188in}{5.013411in}}%
\pgfusepath{}%
\end{pgfscope}%
\begin{pgfscope}%
\definecolor{textcolor}{rgb}{0.150000,0.150000,0.150000}%
\pgfsetstrokecolor{textcolor}%
\pgfsetfillcolor{textcolor}%
\pgftext[x=3.832683in, y=3.507194in, left, base]{\color{textcolor}\rmfamily\fontsize{20.000000}{24.000000}\selectfont M\(\displaystyle _z\)(t) = 6.04\(\displaystyle exp[\frac{-2t}{T_2}]+\)0.19}%
\end{pgfscope}%
\begin{pgfscope}%
\definecolor{textcolor}{rgb}{0.150000,0.150000,0.150000}%
\pgfsetstrokecolor{textcolor}%
\pgfsetfillcolor{textcolor}%
\pgftext[x=3.832683in, y=3.013139in, left, base]{\color{textcolor}\rmfamily\fontsize{20.000000}{24.000000}\selectfont T\(\displaystyle _2\) = 55.3\(\displaystyle \pm\)2.8 ms}%
\end{pgfscope}%
\begin{pgfscope}%
\definecolor{textcolor}{rgb}{0.150000,0.150000,0.150000}%
\pgfsetstrokecolor{textcolor}%
\pgfsetfillcolor{textcolor}%
\pgftext[x=4.238644in,y=5.096745in,,base]{\color{textcolor}\rmfamily\fontsize{24.000000}{28.800000}\selectfont  Spin Echo T\(\displaystyle _2\) measurement of Mineral Oil}%
\end{pgfscope}%
\begin{pgfscope}%
\pgfsetbuttcap%
\pgfsetroundjoin%
\definecolor{currentfill}{rgb}{0.698039,0.133333,0.133333}%
\pgfsetfillcolor{currentfill}%
\pgfsetlinewidth{0.301125pt}%
\definecolor{currentstroke}{rgb}{0.698039,0.133333,0.133333}%
\pgfsetstrokecolor{currentstroke}%
\pgfsetdash{}{0pt}%
\pgfsys@defobject{currentmarker}{\pgfqpoint{-0.048611in}{-0.048611in}}{\pgfqpoint{0.048611in}{0.048611in}}{%
\pgfpathmoveto{\pgfqpoint{0.000000in}{-0.048611in}}%
\pgfpathcurveto{\pgfqpoint{0.012892in}{-0.048611in}}{\pgfqpoint{0.025257in}{-0.043489in}}{\pgfqpoint{0.034373in}{-0.034373in}}%
\pgfpathcurveto{\pgfqpoint{0.043489in}{-0.025257in}}{\pgfqpoint{0.048611in}{-0.012892in}}{\pgfqpoint{0.048611in}{0.000000in}}%
\pgfpathcurveto{\pgfqpoint{0.048611in}{0.012892in}}{\pgfqpoint{0.043489in}{0.025257in}}{\pgfqpoint{0.034373in}{0.034373in}}%
\pgfpathcurveto{\pgfqpoint{0.025257in}{0.043489in}}{\pgfqpoint{0.012892in}{0.048611in}}{\pgfqpoint{0.000000in}{0.048611in}}%
\pgfpathcurveto{\pgfqpoint{-0.012892in}{0.048611in}}{\pgfqpoint{-0.025257in}{0.043489in}}{\pgfqpoint{-0.034373in}{0.034373in}}%
\pgfpathcurveto{\pgfqpoint{-0.043489in}{0.025257in}}{\pgfqpoint{-0.048611in}{0.012892in}}{\pgfqpoint{-0.048611in}{0.000000in}}%
\pgfpathcurveto{\pgfqpoint{-0.048611in}{-0.012892in}}{\pgfqpoint{-0.043489in}{-0.025257in}}{\pgfqpoint{-0.034373in}{-0.034373in}}%
\pgfpathcurveto{\pgfqpoint{-0.025257in}{-0.043489in}}{\pgfqpoint{-0.012892in}{-0.048611in}}{\pgfqpoint{0.000000in}{-0.048611in}}%
\pgfpathlineto{\pgfqpoint{0.000000in}{-0.048611in}}%
\pgfpathclose%
\pgfusepath{stroke,fill}%
}%
\begin{pgfscope}%
\pgfsys@transformshift{5.144317in}{4.625282in}%
\pgfsys@useobject{currentmarker}{}%
\end{pgfscope}%
\end{pgfscope}%
\begin{pgfscope}%
\definecolor{textcolor}{rgb}{0.150000,0.150000,0.150000}%
\pgfsetstrokecolor{textcolor}%
\pgfsetfillcolor{textcolor}%
\pgftext[x=5.644317in,y=4.552365in,left,base]{\color{textcolor}\rmfamily\fontsize{20.000000}{24.000000}\selectfont Data}%
\end{pgfscope}%
\begin{pgfscope}%
\pgfsetroundcap%
\pgfsetroundjoin%
\pgfsetlinewidth{1.756562pt}%
\definecolor{currentstroke}{rgb}{0.298039,0.447059,0.690196}%
\pgfsetstrokecolor{currentstroke}%
\pgfsetdash{}{0pt}%
\pgfpathmoveto{\pgfqpoint{4.866540in}{4.241873in}}%
\pgfpathlineto{\pgfqpoint{5.144317in}{4.241873in}}%
\pgfpathlineto{\pgfqpoint{5.422095in}{4.241873in}}%
\pgfusepath{stroke}%
\end{pgfscope}%
\begin{pgfscope}%
\definecolor{textcolor}{rgb}{0.150000,0.150000,0.150000}%
\pgfsetstrokecolor{textcolor}%
\pgfsetfillcolor{textcolor}%
\pgftext[x=5.644317in,y=4.144651in,left,base]{\color{textcolor}\rmfamily\fontsize{20.000000}{24.000000}\selectfont Fitted Curve}%
\end{pgfscope}%
\end{pgfpicture}%
\makeatother%
\endgroup%
}
    \caption{T2 measurement of mineral oil using the spin-echo pulse sequence }
    \label{fig:mo_se}
  \end{figure}

  \begin{figure}[h]
    \resizebox{0.45\textwidth}{!}{%% Creator: Matplotlib, PGF backend
%%
%% To include the figure in your LaTeX document, write
%%   \input{<filename>.pgf}
%%
%% Make sure the required packages are loaded in your preamble
%%   \usepackage{pgf}
%%
%% Also ensure that all the required font packages are loaded; for instance,
%% the lmodern package is sometimes necessary when using math font.
%%   \usepackage{lmodern}
%%
%% Figures using additional raster images can only be included by \input if
%% they are in the same directory as the main LaTeX file. For loading figures
%% from other directories you can use the `import` package
%%   \usepackage{import}
%%
%% and then include the figures with
%%   \import{<path to file>}{<filename>.pgf}
%%
%% Matplotlib used the following preamble
%%   \usepackage{fontspec}
%%   \setmainfont{DejaVuSerif.ttf}[Path=\detokenize{/usr/share/matplotlib/mpl-data/fonts/ttf/}]
%%   \setsansfont{DejaVuSans.ttf}[Path=\detokenize{/usr/share/matplotlib/mpl-data/fonts/ttf/}]
%%   \setmonofont{DejaVuSansMono.ttf}[Path=\detokenize{/usr/share/matplotlib/mpl-data/fonts/ttf/}]
%%
\begingroup%
\makeatletter%
\begin{pgfpicture}%
\pgfpathrectangle{\pgfpointorigin}{\pgfqpoint{8.000000in}{5.500000in}}%
\pgfusepath{use as bounding box, clip}%
\begin{pgfscope}%
\pgfsetbuttcap%
\pgfsetmiterjoin%
\definecolor{currentfill}{rgb}{1.000000,1.000000,1.000000}%
\pgfsetfillcolor{currentfill}%
\pgfsetlinewidth{0.000000pt}%
\definecolor{currentstroke}{rgb}{1.000000,1.000000,1.000000}%
\pgfsetstrokecolor{currentstroke}%
\pgfsetdash{}{0pt}%
\pgfpathmoveto{\pgfqpoint{0.000000in}{0.000000in}}%
\pgfpathlineto{\pgfqpoint{8.000000in}{0.000000in}}%
\pgfpathlineto{\pgfqpoint{8.000000in}{5.500000in}}%
\pgfpathlineto{\pgfqpoint{0.000000in}{5.500000in}}%
\pgfpathlineto{\pgfqpoint{0.000000in}{0.000000in}}%
\pgfpathclose%
\pgfusepath{fill}%
\end{pgfscope}%
\begin{pgfscope}%
\pgfsetbuttcap%
\pgfsetmiterjoin%
\definecolor{currentfill}{rgb}{0.917647,0.917647,0.949020}%
\pgfsetfillcolor{currentfill}%
\pgfsetlinewidth{0.000000pt}%
\definecolor{currentstroke}{rgb}{0.000000,0.000000,0.000000}%
\pgfsetstrokecolor{currentstroke}%
\pgfsetstrokeopacity{0.000000}%
\pgfsetdash{}{0pt}%
\pgfpathmoveto{\pgfqpoint{0.802099in}{0.894194in}}%
\pgfpathlineto{\pgfqpoint{7.761936in}{0.894194in}}%
\pgfpathlineto{\pgfqpoint{7.761936in}{5.013411in}}%
\pgfpathlineto{\pgfqpoint{0.802099in}{5.013411in}}%
\pgfpathlineto{\pgfqpoint{0.802099in}{0.894194in}}%
\pgfpathclose%
\pgfusepath{fill}%
\end{pgfscope}%
\begin{pgfscope}%
\pgfpathrectangle{\pgfqpoint{0.802099in}{0.894194in}}{\pgfqpoint{6.959837in}{4.119217in}}%
\pgfusepath{clip}%
\pgfsetroundcap%
\pgfsetroundjoin%
\pgfsetlinewidth{1.003750pt}%
\definecolor{currentstroke}{rgb}{1.000000,1.000000,1.000000}%
\pgfsetstrokecolor{currentstroke}%
\pgfsetdash{}{0pt}%
\pgfpathmoveto{\pgfqpoint{0.835152in}{0.894194in}}%
\pgfpathlineto{\pgfqpoint{0.835152in}{5.013411in}}%
\pgfusepath{stroke}%
\end{pgfscope}%
\begin{pgfscope}%
\definecolor{textcolor}{rgb}{0.150000,0.150000,0.150000}%
\pgfsetstrokecolor{textcolor}%
\pgfsetfillcolor{textcolor}%
\pgftext[x=0.835152in,y=0.796972in,,top]{\color{textcolor}\rmfamily\fontsize{20.000000}{24.000000}\selectfont 0}%
\end{pgfscope}%
\begin{pgfscope}%
\pgfpathrectangle{\pgfqpoint{0.802099in}{0.894194in}}{\pgfqpoint{6.959837in}{4.119217in}}%
\pgfusepath{clip}%
\pgfsetroundcap%
\pgfsetroundjoin%
\pgfsetlinewidth{1.003750pt}%
\definecolor{currentstroke}{rgb}{1.000000,1.000000,1.000000}%
\pgfsetstrokecolor{currentstroke}%
\pgfsetdash{}{0pt}%
\pgfpathmoveto{\pgfqpoint{1.779499in}{0.894194in}}%
\pgfpathlineto{\pgfqpoint{1.779499in}{5.013411in}}%
\pgfusepath{stroke}%
\end{pgfscope}%
\begin{pgfscope}%
\definecolor{textcolor}{rgb}{0.150000,0.150000,0.150000}%
\pgfsetstrokecolor{textcolor}%
\pgfsetfillcolor{textcolor}%
\pgftext[x=1.779499in,y=0.796972in,,top]{\color{textcolor}\rmfamily\fontsize{20.000000}{24.000000}\selectfont 20}%
\end{pgfscope}%
\begin{pgfscope}%
\pgfpathrectangle{\pgfqpoint{0.802099in}{0.894194in}}{\pgfqpoint{6.959837in}{4.119217in}}%
\pgfusepath{clip}%
\pgfsetroundcap%
\pgfsetroundjoin%
\pgfsetlinewidth{1.003750pt}%
\definecolor{currentstroke}{rgb}{1.000000,1.000000,1.000000}%
\pgfsetstrokecolor{currentstroke}%
\pgfsetdash{}{0pt}%
\pgfpathmoveto{\pgfqpoint{2.723845in}{0.894194in}}%
\pgfpathlineto{\pgfqpoint{2.723845in}{5.013411in}}%
\pgfusepath{stroke}%
\end{pgfscope}%
\begin{pgfscope}%
\definecolor{textcolor}{rgb}{0.150000,0.150000,0.150000}%
\pgfsetstrokecolor{textcolor}%
\pgfsetfillcolor{textcolor}%
\pgftext[x=2.723845in,y=0.796972in,,top]{\color{textcolor}\rmfamily\fontsize{20.000000}{24.000000}\selectfont 40}%
\end{pgfscope}%
\begin{pgfscope}%
\pgfpathrectangle{\pgfqpoint{0.802099in}{0.894194in}}{\pgfqpoint{6.959837in}{4.119217in}}%
\pgfusepath{clip}%
\pgfsetroundcap%
\pgfsetroundjoin%
\pgfsetlinewidth{1.003750pt}%
\definecolor{currentstroke}{rgb}{1.000000,1.000000,1.000000}%
\pgfsetstrokecolor{currentstroke}%
\pgfsetdash{}{0pt}%
\pgfpathmoveto{\pgfqpoint{3.668192in}{0.894194in}}%
\pgfpathlineto{\pgfqpoint{3.668192in}{5.013411in}}%
\pgfusepath{stroke}%
\end{pgfscope}%
\begin{pgfscope}%
\definecolor{textcolor}{rgb}{0.150000,0.150000,0.150000}%
\pgfsetstrokecolor{textcolor}%
\pgfsetfillcolor{textcolor}%
\pgftext[x=3.668192in,y=0.796972in,,top]{\color{textcolor}\rmfamily\fontsize{20.000000}{24.000000}\selectfont 60}%
\end{pgfscope}%
\begin{pgfscope}%
\pgfpathrectangle{\pgfqpoint{0.802099in}{0.894194in}}{\pgfqpoint{6.959837in}{4.119217in}}%
\pgfusepath{clip}%
\pgfsetroundcap%
\pgfsetroundjoin%
\pgfsetlinewidth{1.003750pt}%
\definecolor{currentstroke}{rgb}{1.000000,1.000000,1.000000}%
\pgfsetstrokecolor{currentstroke}%
\pgfsetdash{}{0pt}%
\pgfpathmoveto{\pgfqpoint{4.612539in}{0.894194in}}%
\pgfpathlineto{\pgfqpoint{4.612539in}{5.013411in}}%
\pgfusepath{stroke}%
\end{pgfscope}%
\begin{pgfscope}%
\definecolor{textcolor}{rgb}{0.150000,0.150000,0.150000}%
\pgfsetstrokecolor{textcolor}%
\pgfsetfillcolor{textcolor}%
\pgftext[x=4.612539in,y=0.796972in,,top]{\color{textcolor}\rmfamily\fontsize{20.000000}{24.000000}\selectfont 80}%
\end{pgfscope}%
\begin{pgfscope}%
\pgfpathrectangle{\pgfqpoint{0.802099in}{0.894194in}}{\pgfqpoint{6.959837in}{4.119217in}}%
\pgfusepath{clip}%
\pgfsetroundcap%
\pgfsetroundjoin%
\pgfsetlinewidth{1.003750pt}%
\definecolor{currentstroke}{rgb}{1.000000,1.000000,1.000000}%
\pgfsetstrokecolor{currentstroke}%
\pgfsetdash{}{0pt}%
\pgfpathmoveto{\pgfqpoint{5.556886in}{0.894194in}}%
\pgfpathlineto{\pgfqpoint{5.556886in}{5.013411in}}%
\pgfusepath{stroke}%
\end{pgfscope}%
\begin{pgfscope}%
\definecolor{textcolor}{rgb}{0.150000,0.150000,0.150000}%
\pgfsetstrokecolor{textcolor}%
\pgfsetfillcolor{textcolor}%
\pgftext[x=5.556886in,y=0.796972in,,top]{\color{textcolor}\rmfamily\fontsize{20.000000}{24.000000}\selectfont 100}%
\end{pgfscope}%
\begin{pgfscope}%
\pgfpathrectangle{\pgfqpoint{0.802099in}{0.894194in}}{\pgfqpoint{6.959837in}{4.119217in}}%
\pgfusepath{clip}%
\pgfsetroundcap%
\pgfsetroundjoin%
\pgfsetlinewidth{1.003750pt}%
\definecolor{currentstroke}{rgb}{1.000000,1.000000,1.000000}%
\pgfsetstrokecolor{currentstroke}%
\pgfsetdash{}{0pt}%
\pgfpathmoveto{\pgfqpoint{6.501233in}{0.894194in}}%
\pgfpathlineto{\pgfqpoint{6.501233in}{5.013411in}}%
\pgfusepath{stroke}%
\end{pgfscope}%
\begin{pgfscope}%
\definecolor{textcolor}{rgb}{0.150000,0.150000,0.150000}%
\pgfsetstrokecolor{textcolor}%
\pgfsetfillcolor{textcolor}%
\pgftext[x=6.501233in,y=0.796972in,,top]{\color{textcolor}\rmfamily\fontsize{20.000000}{24.000000}\selectfont 120}%
\end{pgfscope}%
\begin{pgfscope}%
\pgfpathrectangle{\pgfqpoint{0.802099in}{0.894194in}}{\pgfqpoint{6.959837in}{4.119217in}}%
\pgfusepath{clip}%
\pgfsetroundcap%
\pgfsetroundjoin%
\pgfsetlinewidth{1.003750pt}%
\definecolor{currentstroke}{rgb}{1.000000,1.000000,1.000000}%
\pgfsetstrokecolor{currentstroke}%
\pgfsetdash{}{0pt}%
\pgfpathmoveto{\pgfqpoint{7.445580in}{0.894194in}}%
\pgfpathlineto{\pgfqpoint{7.445580in}{5.013411in}}%
\pgfusepath{stroke}%
\end{pgfscope}%
\begin{pgfscope}%
\definecolor{textcolor}{rgb}{0.150000,0.150000,0.150000}%
\pgfsetstrokecolor{textcolor}%
\pgfsetfillcolor{textcolor}%
\pgftext[x=7.445580in,y=0.796972in,,top]{\color{textcolor}\rmfamily\fontsize{20.000000}{24.000000}\selectfont 140}%
\end{pgfscope}%
\begin{pgfscope}%
\definecolor{textcolor}{rgb}{0.150000,0.150000,0.150000}%
\pgfsetstrokecolor{textcolor}%
\pgfsetfillcolor{textcolor}%
\pgftext[x=4.282018in,y=0.472591in,,top]{\color{textcolor}\rmfamily\fontsize{24.000000}{28.800000}\selectfont Time (ms)}%
\end{pgfscope}%
\begin{pgfscope}%
\pgfpathrectangle{\pgfqpoint{0.802099in}{0.894194in}}{\pgfqpoint{6.959837in}{4.119217in}}%
\pgfusepath{clip}%
\pgfsetroundcap%
\pgfsetroundjoin%
\pgfsetlinewidth{1.003750pt}%
\definecolor{currentstroke}{rgb}{1.000000,1.000000,1.000000}%
\pgfsetstrokecolor{currentstroke}%
\pgfsetdash{}{0pt}%
\pgfpathmoveto{\pgfqpoint{0.802099in}{1.621214in}}%
\pgfpathlineto{\pgfqpoint{7.761936in}{1.621214in}}%
\pgfusepath{stroke}%
\end{pgfscope}%
\begin{pgfscope}%
\definecolor{textcolor}{rgb}{0.150000,0.150000,0.150000}%
\pgfsetstrokecolor{textcolor}%
\pgfsetfillcolor{textcolor}%
\pgftext[x=0.528147in, y=1.515691in, left, base]{\color{textcolor}\rmfamily\fontsize{20.000000}{24.000000}\selectfont 1}%
\end{pgfscope}%
\begin{pgfscope}%
\pgfpathrectangle{\pgfqpoint{0.802099in}{0.894194in}}{\pgfqpoint{6.959837in}{4.119217in}}%
\pgfusepath{clip}%
\pgfsetroundcap%
\pgfsetroundjoin%
\pgfsetlinewidth{1.003750pt}%
\definecolor{currentstroke}{rgb}{1.000000,1.000000,1.000000}%
\pgfsetstrokecolor{currentstroke}%
\pgfsetdash{}{0pt}%
\pgfpathmoveto{\pgfqpoint{0.802099in}{2.464625in}}%
\pgfpathlineto{\pgfqpoint{7.761936in}{2.464625in}}%
\pgfusepath{stroke}%
\end{pgfscope}%
\begin{pgfscope}%
\definecolor{textcolor}{rgb}{0.150000,0.150000,0.150000}%
\pgfsetstrokecolor{textcolor}%
\pgfsetfillcolor{textcolor}%
\pgftext[x=0.528147in, y=2.359102in, left, base]{\color{textcolor}\rmfamily\fontsize{20.000000}{24.000000}\selectfont 2}%
\end{pgfscope}%
\begin{pgfscope}%
\pgfpathrectangle{\pgfqpoint{0.802099in}{0.894194in}}{\pgfqpoint{6.959837in}{4.119217in}}%
\pgfusepath{clip}%
\pgfsetroundcap%
\pgfsetroundjoin%
\pgfsetlinewidth{1.003750pt}%
\definecolor{currentstroke}{rgb}{1.000000,1.000000,1.000000}%
\pgfsetstrokecolor{currentstroke}%
\pgfsetdash{}{0pt}%
\pgfpathmoveto{\pgfqpoint{0.802099in}{3.308035in}}%
\pgfpathlineto{\pgfqpoint{7.761936in}{3.308035in}}%
\pgfusepath{stroke}%
\end{pgfscope}%
\begin{pgfscope}%
\definecolor{textcolor}{rgb}{0.150000,0.150000,0.150000}%
\pgfsetstrokecolor{textcolor}%
\pgfsetfillcolor{textcolor}%
\pgftext[x=0.528147in, y=3.202512in, left, base]{\color{textcolor}\rmfamily\fontsize{20.000000}{24.000000}\selectfont 3}%
\end{pgfscope}%
\begin{pgfscope}%
\pgfpathrectangle{\pgfqpoint{0.802099in}{0.894194in}}{\pgfqpoint{6.959837in}{4.119217in}}%
\pgfusepath{clip}%
\pgfsetroundcap%
\pgfsetroundjoin%
\pgfsetlinewidth{1.003750pt}%
\definecolor{currentstroke}{rgb}{1.000000,1.000000,1.000000}%
\pgfsetstrokecolor{currentstroke}%
\pgfsetdash{}{0pt}%
\pgfpathmoveto{\pgfqpoint{0.802099in}{4.151446in}}%
\pgfpathlineto{\pgfqpoint{7.761936in}{4.151446in}}%
\pgfusepath{stroke}%
\end{pgfscope}%
\begin{pgfscope}%
\definecolor{textcolor}{rgb}{0.150000,0.150000,0.150000}%
\pgfsetstrokecolor{textcolor}%
\pgfsetfillcolor{textcolor}%
\pgftext[x=0.528147in, y=4.045923in, left, base]{\color{textcolor}\rmfamily\fontsize{20.000000}{24.000000}\selectfont 4}%
\end{pgfscope}%
\begin{pgfscope}%
\pgfpathrectangle{\pgfqpoint{0.802099in}{0.894194in}}{\pgfqpoint{6.959837in}{4.119217in}}%
\pgfusepath{clip}%
\pgfsetroundcap%
\pgfsetroundjoin%
\pgfsetlinewidth{1.003750pt}%
\definecolor{currentstroke}{rgb}{1.000000,1.000000,1.000000}%
\pgfsetstrokecolor{currentstroke}%
\pgfsetdash{}{0pt}%
\pgfpathmoveto{\pgfqpoint{0.802099in}{4.994856in}}%
\pgfpathlineto{\pgfqpoint{7.761936in}{4.994856in}}%
\pgfusepath{stroke}%
\end{pgfscope}%
\begin{pgfscope}%
\definecolor{textcolor}{rgb}{0.150000,0.150000,0.150000}%
\pgfsetstrokecolor{textcolor}%
\pgfsetfillcolor{textcolor}%
\pgftext[x=0.528147in, y=4.889333in, left, base]{\color{textcolor}\rmfamily\fontsize{20.000000}{24.000000}\selectfont 5}%
\end{pgfscope}%
\begin{pgfscope}%
\definecolor{textcolor}{rgb}{0.150000,0.150000,0.150000}%
\pgfsetstrokecolor{textcolor}%
\pgfsetfillcolor{textcolor}%
\pgftext[x=0.472591in,y=2.953803in,,bottom,rotate=90.000000]{\color{textcolor}\rmfamily\fontsize{24.000000}{28.800000}\selectfont Pulse Amplitude (V)}%
\end{pgfscope}%
\begin{pgfscope}%
\pgfpathrectangle{\pgfqpoint{0.802099in}{0.894194in}}{\pgfqpoint{6.959837in}{4.119217in}}%
\pgfusepath{clip}%
\pgfsetbuttcap%
\pgfsetroundjoin%
\definecolor{currentfill}{rgb}{0.698039,0.133333,0.133333}%
\pgfsetfillcolor{currentfill}%
\pgfsetlinewidth{0.301125pt}%
\definecolor{currentstroke}{rgb}{0.698039,0.133333,0.133333}%
\pgfsetstrokecolor{currentstroke}%
\pgfsetdash{}{0pt}%
\pgfsys@defobject{currentmarker}{\pgfqpoint{-0.048611in}{-0.048611in}}{\pgfqpoint{0.048611in}{0.048611in}}{%
\pgfpathmoveto{\pgfqpoint{0.000000in}{-0.048611in}}%
\pgfpathcurveto{\pgfqpoint{0.012892in}{-0.048611in}}{\pgfqpoint{0.025257in}{-0.043489in}}{\pgfqpoint{0.034373in}{-0.034373in}}%
\pgfpathcurveto{\pgfqpoint{0.043489in}{-0.025257in}}{\pgfqpoint{0.048611in}{-0.012892in}}{\pgfqpoint{0.048611in}{0.000000in}}%
\pgfpathcurveto{\pgfqpoint{0.048611in}{0.012892in}}{\pgfqpoint{0.043489in}{0.025257in}}{\pgfqpoint{0.034373in}{0.034373in}}%
\pgfpathcurveto{\pgfqpoint{0.025257in}{0.043489in}}{\pgfqpoint{0.012892in}{0.048611in}}{\pgfqpoint{0.000000in}{0.048611in}}%
\pgfpathcurveto{\pgfqpoint{-0.012892in}{0.048611in}}{\pgfqpoint{-0.025257in}{0.043489in}}{\pgfqpoint{-0.034373in}{0.034373in}}%
\pgfpathcurveto{\pgfqpoint{-0.043489in}{0.025257in}}{\pgfqpoint{-0.048611in}{0.012892in}}{\pgfqpoint{-0.048611in}{0.000000in}}%
\pgfpathcurveto{\pgfqpoint{-0.048611in}{-0.012892in}}{\pgfqpoint{-0.043489in}{-0.025257in}}{\pgfqpoint{-0.034373in}{-0.034373in}}%
\pgfpathcurveto{\pgfqpoint{-0.025257in}{-0.043489in}}{\pgfqpoint{-0.012892in}{-0.048611in}}{\pgfqpoint{0.000000in}{-0.048611in}}%
\pgfpathlineto{\pgfqpoint{0.000000in}{-0.048611in}}%
\pgfpathclose%
\pgfusepath{stroke,fill}%
}%
\begin{pgfscope}%
\pgfsys@transformshift{1.118456in}{4.826174in}%
\pgfsys@useobject{currentmarker}{}%
\end{pgfscope}%
\begin{pgfscope}%
\pgfsys@transformshift{1.873933in}{3.915291in}%
\pgfsys@useobject{currentmarker}{}%
\end{pgfscope}%
\begin{pgfscope}%
\pgfsys@transformshift{2.534976in}{3.308035in}%
\pgfsys@useobject{currentmarker}{}%
\end{pgfscope}%
\begin{pgfscope}%
\pgfsys@transformshift{3.243236in}{2.869462in}%
\pgfsys@useobject{currentmarker}{}%
\end{pgfscope}%
\begin{pgfscope}%
\pgfsys@transformshift{3.951496in}{2.532098in}%
\pgfsys@useobject{currentmarker}{}%
\end{pgfscope}%
\begin{pgfscope}%
\pgfsys@transformshift{4.706974in}{2.295943in}%
\pgfsys@useobject{currentmarker}{}%
\end{pgfscope}%
\begin{pgfscope}%
\pgfsys@transformshift{5.368017in}{2.127261in}%
\pgfsys@useobject{currentmarker}{}%
\end{pgfscope}%
\begin{pgfscope}%
\pgfsys@transformshift{6.029060in}{1.823633in}%
\pgfsys@useobject{currentmarker}{}%
\end{pgfscope}%
\begin{pgfscope}%
\pgfsys@transformshift{6.784537in}{1.654951in}%
\pgfsys@useobject{currentmarker}{}%
\end{pgfscope}%
\begin{pgfscope}%
\pgfsys@transformshift{7.445580in}{1.081432in}%
\pgfsys@useobject{currentmarker}{}%
\end{pgfscope}%
\end{pgfscope}%
\begin{pgfscope}%
\pgfpathrectangle{\pgfqpoint{0.802099in}{0.894194in}}{\pgfqpoint{6.959837in}{4.119217in}}%
\pgfusepath{clip}%
\pgfsetroundcap%
\pgfsetroundjoin%
\pgfsetlinewidth{1.756562pt}%
\definecolor{currentstroke}{rgb}{0.298039,0.447059,0.690196}%
\pgfsetstrokecolor{currentstroke}%
\pgfsetdash{}{0pt}%
\pgfpathmoveto{\pgfqpoint{1.118456in}{4.711724in}}%
\pgfpathlineto{\pgfqpoint{1.182366in}{4.643839in}}%
\pgfpathlineto{\pgfqpoint{1.246276in}{4.577064in}}%
\pgfpathlineto{\pgfqpoint{1.310187in}{4.511383in}}%
\pgfpathlineto{\pgfqpoint{1.374097in}{4.446777in}}%
\pgfpathlineto{\pgfqpoint{1.438007in}{4.383229in}}%
\pgfpathlineto{\pgfqpoint{1.501918in}{4.320720in}}%
\pgfpathlineto{\pgfqpoint{1.565828in}{4.259235in}}%
\pgfpathlineto{\pgfqpoint{1.629738in}{4.198757in}}%
\pgfpathlineto{\pgfqpoint{1.693649in}{4.139268in}}%
\pgfpathlineto{\pgfqpoint{1.757559in}{4.080753in}}%
\pgfpathlineto{\pgfqpoint{1.821469in}{4.023196in}}%
\pgfpathlineto{\pgfqpoint{1.885380in}{3.966581in}}%
\pgfpathlineto{\pgfqpoint{1.949290in}{3.910892in}}%
\pgfpathlineto{\pgfqpoint{2.013201in}{3.856116in}}%
\pgfpathlineto{\pgfqpoint{2.077111in}{3.802236in}}%
\pgfpathlineto{\pgfqpoint{2.141021in}{3.749238in}}%
\pgfpathlineto{\pgfqpoint{2.204932in}{3.697107in}}%
\pgfpathlineto{\pgfqpoint{2.268842in}{3.645830in}}%
\pgfpathlineto{\pgfqpoint{2.332752in}{3.595392in}}%
\pgfpathlineto{\pgfqpoint{2.396663in}{3.545779in}}%
\pgfpathlineto{\pgfqpoint{2.460573in}{3.496979in}}%
\pgfpathlineto{\pgfqpoint{2.524483in}{3.448978in}}%
\pgfpathlineto{\pgfqpoint{2.588394in}{3.401762in}}%
\pgfpathlineto{\pgfqpoint{2.652304in}{3.355319in}}%
\pgfpathlineto{\pgfqpoint{2.716214in}{3.309637in}}%
\pgfpathlineto{\pgfqpoint{2.780125in}{3.264702in}}%
\pgfpathlineto{\pgfqpoint{2.844035in}{3.220502in}}%
\pgfpathlineto{\pgfqpoint{2.907945in}{3.177027in}}%
\pgfpathlineto{\pgfqpoint{2.971856in}{3.134262in}}%
\pgfpathlineto{\pgfqpoint{3.035766in}{3.092198in}}%
\pgfpathlineto{\pgfqpoint{3.099676in}{3.050823in}}%
\pgfpathlineto{\pgfqpoint{3.163587in}{3.010124in}}%
\pgfpathlineto{\pgfqpoint{3.227497in}{2.970092in}}%
\pgfpathlineto{\pgfqpoint{3.291407in}{2.930715in}}%
\pgfpathlineto{\pgfqpoint{3.355318in}{2.891983in}}%
\pgfpathlineto{\pgfqpoint{3.419228in}{2.853884in}}%
\pgfpathlineto{\pgfqpoint{3.483139in}{2.816410in}}%
\pgfpathlineto{\pgfqpoint{3.547049in}{2.779548in}}%
\pgfpathlineto{\pgfqpoint{3.610959in}{2.743290in}}%
\pgfpathlineto{\pgfqpoint{3.674870in}{2.707626in}}%
\pgfpathlineto{\pgfqpoint{3.738780in}{2.672545in}}%
\pgfpathlineto{\pgfqpoint{3.802690in}{2.638039in}}%
\pgfpathlineto{\pgfqpoint{3.866601in}{2.604097in}}%
\pgfpathlineto{\pgfqpoint{3.930511in}{2.570711in}}%
\pgfpathlineto{\pgfqpoint{3.994421in}{2.537872in}}%
\pgfpathlineto{\pgfqpoint{4.058332in}{2.505570in}}%
\pgfpathlineto{\pgfqpoint{4.122242in}{2.473797in}}%
\pgfpathlineto{\pgfqpoint{4.186152in}{2.442544in}}%
\pgfpathlineto{\pgfqpoint{4.250063in}{2.411802in}}%
\pgfpathlineto{\pgfqpoint{4.313973in}{2.381564in}}%
\pgfpathlineto{\pgfqpoint{4.377883in}{2.351820in}}%
\pgfpathlineto{\pgfqpoint{4.441794in}{2.322564in}}%
\pgfpathlineto{\pgfqpoint{4.505704in}{2.293786in}}%
\pgfpathlineto{\pgfqpoint{4.569614in}{2.265480in}}%
\pgfpathlineto{\pgfqpoint{4.633525in}{2.237636in}}%
\pgfpathlineto{\pgfqpoint{4.697435in}{2.210249in}}%
\pgfpathlineto{\pgfqpoint{4.761345in}{2.183310in}}%
\pgfpathlineto{\pgfqpoint{4.825256in}{2.156812in}}%
\pgfpathlineto{\pgfqpoint{4.889166in}{2.130747in}}%
\pgfpathlineto{\pgfqpoint{4.953076in}{2.105109in}}%
\pgfpathlineto{\pgfqpoint{5.016987in}{2.079891in}}%
\pgfpathlineto{\pgfqpoint{5.080897in}{2.055086in}}%
\pgfpathlineto{\pgfqpoint{5.144808in}{2.030687in}}%
\pgfpathlineto{\pgfqpoint{5.208718in}{2.006687in}}%
\pgfpathlineto{\pgfqpoint{5.272628in}{1.983080in}}%
\pgfpathlineto{\pgfqpoint{5.336539in}{1.959859in}}%
\pgfpathlineto{\pgfqpoint{5.400449in}{1.937018in}}%
\pgfpathlineto{\pgfqpoint{5.464359in}{1.914552in}}%
\pgfpathlineto{\pgfqpoint{5.528270in}{1.892453in}}%
\pgfpathlineto{\pgfqpoint{5.592180in}{1.870716in}}%
\pgfpathlineto{\pgfqpoint{5.656090in}{1.849334in}}%
\pgfpathlineto{\pgfqpoint{5.720001in}{1.828303in}}%
\pgfpathlineto{\pgfqpoint{5.783911in}{1.807616in}}%
\pgfpathlineto{\pgfqpoint{5.847821in}{1.787267in}}%
\pgfpathlineto{\pgfqpoint{5.911732in}{1.767252in}}%
\pgfpathlineto{\pgfqpoint{5.975642in}{1.747564in}}%
\pgfpathlineto{\pgfqpoint{6.039552in}{1.728198in}}%
\pgfpathlineto{\pgfqpoint{6.103463in}{1.709150in}}%
\pgfpathlineto{\pgfqpoint{6.167373in}{1.690413in}}%
\pgfpathlineto{\pgfqpoint{6.231283in}{1.671983in}}%
\pgfpathlineto{\pgfqpoint{6.295194in}{1.653855in}}%
\pgfpathlineto{\pgfqpoint{6.359104in}{1.636023in}}%
\pgfpathlineto{\pgfqpoint{6.423014in}{1.618483in}}%
\pgfpathlineto{\pgfqpoint{6.486925in}{1.601231in}}%
\pgfpathlineto{\pgfqpoint{6.550835in}{1.584260in}}%
\pgfpathlineto{\pgfqpoint{6.614746in}{1.567568in}}%
\pgfpathlineto{\pgfqpoint{6.678656in}{1.551149in}}%
\pgfpathlineto{\pgfqpoint{6.742566in}{1.534998in}}%
\pgfpathlineto{\pgfqpoint{6.806477in}{1.519112in}}%
\pgfpathlineto{\pgfqpoint{6.870387in}{1.503486in}}%
\pgfpathlineto{\pgfqpoint{6.934297in}{1.488116in}}%
\pgfpathlineto{\pgfqpoint{6.998208in}{1.472997in}}%
\pgfpathlineto{\pgfqpoint{7.062118in}{1.458126in}}%
\pgfpathlineto{\pgfqpoint{7.126028in}{1.443498in}}%
\pgfpathlineto{\pgfqpoint{7.189939in}{1.429110in}}%
\pgfpathlineto{\pgfqpoint{7.253849in}{1.414957in}}%
\pgfpathlineto{\pgfqpoint{7.317759in}{1.401036in}}%
\pgfpathlineto{\pgfqpoint{7.381670in}{1.387343in}}%
\pgfpathlineto{\pgfqpoint{7.445580in}{1.373873in}}%
\pgfusepath{stroke}%
\end{pgfscope}%
\begin{pgfscope}%
\pgfsetrectcap%
\pgfsetmiterjoin%
\pgfsetlinewidth{0.000000pt}%
\definecolor{currentstroke}{rgb}{1.000000,1.000000,1.000000}%
\pgfsetstrokecolor{currentstroke}%
\pgfsetdash{}{0pt}%
\pgfpathmoveto{\pgfqpoint{0.802099in}{0.894194in}}%
\pgfpathlineto{\pgfqpoint{0.802099in}{5.013411in}}%
\pgfusepath{}%
\end{pgfscope}%
\begin{pgfscope}%
\pgfsetrectcap%
\pgfsetmiterjoin%
\pgfsetlinewidth{0.000000pt}%
\definecolor{currentstroke}{rgb}{1.000000,1.000000,1.000000}%
\pgfsetstrokecolor{currentstroke}%
\pgfsetdash{}{0pt}%
\pgfpathmoveto{\pgfqpoint{7.761936in}{0.894194in}}%
\pgfpathlineto{\pgfqpoint{7.761936in}{5.013411in}}%
\pgfusepath{}%
\end{pgfscope}%
\begin{pgfscope}%
\pgfsetrectcap%
\pgfsetmiterjoin%
\pgfsetlinewidth{0.000000pt}%
\definecolor{currentstroke}{rgb}{1.000000,1.000000,1.000000}%
\pgfsetstrokecolor{currentstroke}%
\pgfsetdash{}{0pt}%
\pgfpathmoveto{\pgfqpoint{0.802099in}{0.894194in}}%
\pgfpathlineto{\pgfqpoint{7.761936in}{0.894194in}}%
\pgfusepath{}%
\end{pgfscope}%
\begin{pgfscope}%
\pgfsetrectcap%
\pgfsetmiterjoin%
\pgfsetlinewidth{0.000000pt}%
\definecolor{currentstroke}{rgb}{1.000000,1.000000,1.000000}%
\pgfsetstrokecolor{currentstroke}%
\pgfsetdash{}{0pt}%
\pgfpathmoveto{\pgfqpoint{0.802099in}{5.013411in}}%
\pgfpathlineto{\pgfqpoint{7.761936in}{5.013411in}}%
\pgfusepath{}%
\end{pgfscope}%
\begin{pgfscope}%
\definecolor{textcolor}{rgb}{0.150000,0.150000,0.150000}%
\pgfsetstrokecolor{textcolor}%
\pgfsetfillcolor{textcolor}%
\pgftext[x=3.668192in, y=3.633409in, left, base]{\color{textcolor}\rmfamily\fontsize{20.000000}{24.000000}\selectfont M\(\displaystyle _z\)(t) = 5.29\(\displaystyle exp[\frac{-2t}{T_2}]+\)-0.25}%
\end{pgfscope}%
\begin{pgfscope}%
\definecolor{textcolor}{rgb}{0.150000,0.150000,0.150000}%
\pgfsetstrokecolor{textcolor}%
\pgfsetfillcolor{textcolor}%
\pgftext[x=3.668192in, y=3.139353in, left, base]{\color{textcolor}\rmfamily\fontsize{20.000000}{24.000000}\selectfont T\(\displaystyle _2\) = 164.\(\displaystyle \pm\)38. ms}%
\end{pgfscope}%
\begin{pgfscope}%
\definecolor{textcolor}{rgb}{0.150000,0.150000,0.150000}%
\pgfsetstrokecolor{textcolor}%
\pgfsetfillcolor{textcolor}%
\pgftext[x=4.282018in,y=5.096745in,,base]{\color{textcolor}\rmfamily\fontsize{24.000000}{28.800000}\selectfont Carl-Purcell T\(\displaystyle _2\) measurement of Mineral Oil}%
\end{pgfscope}%
\begin{pgfscope}%
\pgfsetbuttcap%
\pgfsetroundjoin%
\definecolor{currentfill}{rgb}{0.698039,0.133333,0.133333}%
\pgfsetfillcolor{currentfill}%
\pgfsetlinewidth{0.301125pt}%
\definecolor{currentstroke}{rgb}{0.698039,0.133333,0.133333}%
\pgfsetstrokecolor{currentstroke}%
\pgfsetdash{}{0pt}%
\pgfsys@defobject{currentmarker}{\pgfqpoint{-0.048611in}{-0.048611in}}{\pgfqpoint{0.048611in}{0.048611in}}{%
\pgfpathmoveto{\pgfqpoint{0.000000in}{-0.048611in}}%
\pgfpathcurveto{\pgfqpoint{0.012892in}{-0.048611in}}{\pgfqpoint{0.025257in}{-0.043489in}}{\pgfqpoint{0.034373in}{-0.034373in}}%
\pgfpathcurveto{\pgfqpoint{0.043489in}{-0.025257in}}{\pgfqpoint{0.048611in}{-0.012892in}}{\pgfqpoint{0.048611in}{0.000000in}}%
\pgfpathcurveto{\pgfqpoint{0.048611in}{0.012892in}}{\pgfqpoint{0.043489in}{0.025257in}}{\pgfqpoint{0.034373in}{0.034373in}}%
\pgfpathcurveto{\pgfqpoint{0.025257in}{0.043489in}}{\pgfqpoint{0.012892in}{0.048611in}}{\pgfqpoint{0.000000in}{0.048611in}}%
\pgfpathcurveto{\pgfqpoint{-0.012892in}{0.048611in}}{\pgfqpoint{-0.025257in}{0.043489in}}{\pgfqpoint{-0.034373in}{0.034373in}}%
\pgfpathcurveto{\pgfqpoint{-0.043489in}{0.025257in}}{\pgfqpoint{-0.048611in}{0.012892in}}{\pgfqpoint{-0.048611in}{0.000000in}}%
\pgfpathcurveto{\pgfqpoint{-0.048611in}{-0.012892in}}{\pgfqpoint{-0.043489in}{-0.025257in}}{\pgfqpoint{-0.034373in}{-0.034373in}}%
\pgfpathcurveto{\pgfqpoint{-0.025257in}{-0.043489in}}{\pgfqpoint{-0.012892in}{-0.048611in}}{\pgfqpoint{0.000000in}{-0.048611in}}%
\pgfpathlineto{\pgfqpoint{0.000000in}{-0.048611in}}%
\pgfpathclose%
\pgfusepath{stroke,fill}%
}%
\begin{pgfscope}%
\pgfsys@transformshift{5.231066in}{4.625282in}%
\pgfsys@useobject{currentmarker}{}%
\end{pgfscope}%
\end{pgfscope}%
\begin{pgfscope}%
\definecolor{textcolor}{rgb}{0.150000,0.150000,0.150000}%
\pgfsetstrokecolor{textcolor}%
\pgfsetfillcolor{textcolor}%
\pgftext[x=5.731066in,y=4.552365in,left,base]{\color{textcolor}\rmfamily\fontsize{20.000000}{24.000000}\selectfont Data}%
\end{pgfscope}%
\begin{pgfscope}%
\pgfsetroundcap%
\pgfsetroundjoin%
\pgfsetlinewidth{1.756562pt}%
\definecolor{currentstroke}{rgb}{0.298039,0.447059,0.690196}%
\pgfsetstrokecolor{currentstroke}%
\pgfsetdash{}{0pt}%
\pgfpathmoveto{\pgfqpoint{4.953288in}{4.241873in}}%
\pgfpathlineto{\pgfqpoint{5.231066in}{4.241873in}}%
\pgfpathlineto{\pgfqpoint{5.508844in}{4.241873in}}%
\pgfusepath{stroke}%
\end{pgfscope}%
\begin{pgfscope}%
\definecolor{textcolor}{rgb}{0.150000,0.150000,0.150000}%
\pgfsetstrokecolor{textcolor}%
\pgfsetfillcolor{textcolor}%
\pgftext[x=5.731066in,y=4.144651in,left,base]{\color{textcolor}\rmfamily\fontsize{20.000000}{24.000000}\selectfont Fitted Curve}%
\end{pgfscope}%
\end{pgfpicture}%
\makeatother%
\endgroup%
}
    \caption{T2 measurement of  using the Carl-Purcell pulse sequence }
    \label{fig:mo_cp}
  \end{figure}

  \begin{figure}[h]
    \resizebox{0.45\textwidth}{!}{%% Creator: Matplotlib, PGF backend
%%
%% To include the figure in your LaTeX document, write
%%   \input{<filename>.pgf}
%%
%% Make sure the required packages are loaded in your preamble
%%   \usepackage{pgf}
%%
%% Also ensure that all the required font packages are loaded; for instance,
%% the lmodern package is sometimes necessary when using math font.
%%   \usepackage{lmodern}
%%
%% Figures using additional raster images can only be included by \input if
%% they are in the same directory as the main LaTeX file. For loading figures
%% from other directories you can use the `import` package
%%   \usepackage{import}
%%
%% and then include the figures with
%%   \import{<path to file>}{<filename>.pgf}
%%
%% Matplotlib used the following preamble
%%   \usepackage{fontspec}
%%   \setmainfont{DejaVuSerif.ttf}[Path=\detokenize{/usr/share/matplotlib/mpl-data/fonts/ttf/}]
%%   \setsansfont{DejaVuSans.ttf}[Path=\detokenize{/usr/share/matplotlib/mpl-data/fonts/ttf/}]
%%   \setmonofont{DejaVuSansMono.ttf}[Path=\detokenize{/usr/share/matplotlib/mpl-data/fonts/ttf/}]
%%
\begingroup%
\makeatletter%
\begin{pgfpicture}%
\pgfpathrectangle{\pgfpointorigin}{\pgfqpoint{8.000000in}{5.500000in}}%
\pgfusepath{use as bounding box, clip}%
\begin{pgfscope}%
\pgfsetbuttcap%
\pgfsetmiterjoin%
\definecolor{currentfill}{rgb}{1.000000,1.000000,1.000000}%
\pgfsetfillcolor{currentfill}%
\pgfsetlinewidth{0.000000pt}%
\definecolor{currentstroke}{rgb}{1.000000,1.000000,1.000000}%
\pgfsetstrokecolor{currentstroke}%
\pgfsetdash{}{0pt}%
\pgfpathmoveto{\pgfqpoint{0.000000in}{0.000000in}}%
\pgfpathlineto{\pgfqpoint{8.000000in}{0.000000in}}%
\pgfpathlineto{\pgfqpoint{8.000000in}{5.500000in}}%
\pgfpathlineto{\pgfqpoint{0.000000in}{5.500000in}}%
\pgfpathlineto{\pgfqpoint{0.000000in}{0.000000in}}%
\pgfpathclose%
\pgfusepath{fill}%
\end{pgfscope}%
\begin{pgfscope}%
\pgfsetbuttcap%
\pgfsetmiterjoin%
\definecolor{currentfill}{rgb}{0.917647,0.917647,0.949020}%
\pgfsetfillcolor{currentfill}%
\pgfsetlinewidth{0.000000pt}%
\definecolor{currentstroke}{rgb}{0.000000,0.000000,0.000000}%
\pgfsetstrokecolor{currentstroke}%
\pgfsetstrokeopacity{0.000000}%
\pgfsetdash{}{0pt}%
\pgfpathmoveto{\pgfqpoint{0.802099in}{0.894194in}}%
\pgfpathlineto{\pgfqpoint{7.626498in}{0.894194in}}%
\pgfpathlineto{\pgfqpoint{7.626498in}{5.013411in}}%
\pgfpathlineto{\pgfqpoint{0.802099in}{5.013411in}}%
\pgfpathlineto{\pgfqpoint{0.802099in}{0.894194in}}%
\pgfpathclose%
\pgfusepath{fill}%
\end{pgfscope}%
\begin{pgfscope}%
\pgfpathrectangle{\pgfqpoint{0.802099in}{0.894194in}}{\pgfqpoint{6.824398in}{4.119217in}}%
\pgfusepath{clip}%
\pgfsetroundcap%
\pgfsetroundjoin%
\pgfsetlinewidth{1.003750pt}%
\definecolor{currentstroke}{rgb}{1.000000,1.000000,1.000000}%
\pgfsetstrokecolor{currentstroke}%
\pgfsetdash{}{0pt}%
\pgfpathmoveto{\pgfqpoint{0.949036in}{0.894194in}}%
\pgfpathlineto{\pgfqpoint{0.949036in}{5.013411in}}%
\pgfusepath{stroke}%
\end{pgfscope}%
\begin{pgfscope}%
\definecolor{textcolor}{rgb}{0.150000,0.150000,0.150000}%
\pgfsetstrokecolor{textcolor}%
\pgfsetfillcolor{textcolor}%
\pgftext[x=0.949036in,y=0.796972in,,top]{\color{textcolor}\rmfamily\fontsize{20.000000}{24.000000}\selectfont 0}%
\end{pgfscope}%
\begin{pgfscope}%
\pgfpathrectangle{\pgfqpoint{0.802099in}{0.894194in}}{\pgfqpoint{6.824398in}{4.119217in}}%
\pgfusepath{clip}%
\pgfsetroundcap%
\pgfsetroundjoin%
\pgfsetlinewidth{1.003750pt}%
\definecolor{currentstroke}{rgb}{1.000000,1.000000,1.000000}%
\pgfsetstrokecolor{currentstroke}%
\pgfsetdash{}{0pt}%
\pgfpathmoveto{\pgfqpoint{1.969431in}{0.894194in}}%
\pgfpathlineto{\pgfqpoint{1.969431in}{5.013411in}}%
\pgfusepath{stroke}%
\end{pgfscope}%
\begin{pgfscope}%
\definecolor{textcolor}{rgb}{0.150000,0.150000,0.150000}%
\pgfsetstrokecolor{textcolor}%
\pgfsetfillcolor{textcolor}%
\pgftext[x=1.969431in,y=0.796972in,,top]{\color{textcolor}\rmfamily\fontsize{20.000000}{24.000000}\selectfont 25}%
\end{pgfscope}%
\begin{pgfscope}%
\pgfpathrectangle{\pgfqpoint{0.802099in}{0.894194in}}{\pgfqpoint{6.824398in}{4.119217in}}%
\pgfusepath{clip}%
\pgfsetroundcap%
\pgfsetroundjoin%
\pgfsetlinewidth{1.003750pt}%
\definecolor{currentstroke}{rgb}{1.000000,1.000000,1.000000}%
\pgfsetstrokecolor{currentstroke}%
\pgfsetdash{}{0pt}%
\pgfpathmoveto{\pgfqpoint{2.989825in}{0.894194in}}%
\pgfpathlineto{\pgfqpoint{2.989825in}{5.013411in}}%
\pgfusepath{stroke}%
\end{pgfscope}%
\begin{pgfscope}%
\definecolor{textcolor}{rgb}{0.150000,0.150000,0.150000}%
\pgfsetstrokecolor{textcolor}%
\pgfsetfillcolor{textcolor}%
\pgftext[x=2.989825in,y=0.796972in,,top]{\color{textcolor}\rmfamily\fontsize{20.000000}{24.000000}\selectfont 50}%
\end{pgfscope}%
\begin{pgfscope}%
\pgfpathrectangle{\pgfqpoint{0.802099in}{0.894194in}}{\pgfqpoint{6.824398in}{4.119217in}}%
\pgfusepath{clip}%
\pgfsetroundcap%
\pgfsetroundjoin%
\pgfsetlinewidth{1.003750pt}%
\definecolor{currentstroke}{rgb}{1.000000,1.000000,1.000000}%
\pgfsetstrokecolor{currentstroke}%
\pgfsetdash{}{0pt}%
\pgfpathmoveto{\pgfqpoint{4.010220in}{0.894194in}}%
\pgfpathlineto{\pgfqpoint{4.010220in}{5.013411in}}%
\pgfusepath{stroke}%
\end{pgfscope}%
\begin{pgfscope}%
\definecolor{textcolor}{rgb}{0.150000,0.150000,0.150000}%
\pgfsetstrokecolor{textcolor}%
\pgfsetfillcolor{textcolor}%
\pgftext[x=4.010220in,y=0.796972in,,top]{\color{textcolor}\rmfamily\fontsize{20.000000}{24.000000}\selectfont 75}%
\end{pgfscope}%
\begin{pgfscope}%
\pgfpathrectangle{\pgfqpoint{0.802099in}{0.894194in}}{\pgfqpoint{6.824398in}{4.119217in}}%
\pgfusepath{clip}%
\pgfsetroundcap%
\pgfsetroundjoin%
\pgfsetlinewidth{1.003750pt}%
\definecolor{currentstroke}{rgb}{1.000000,1.000000,1.000000}%
\pgfsetstrokecolor{currentstroke}%
\pgfsetdash{}{0pt}%
\pgfpathmoveto{\pgfqpoint{5.030614in}{0.894194in}}%
\pgfpathlineto{\pgfqpoint{5.030614in}{5.013411in}}%
\pgfusepath{stroke}%
\end{pgfscope}%
\begin{pgfscope}%
\definecolor{textcolor}{rgb}{0.150000,0.150000,0.150000}%
\pgfsetstrokecolor{textcolor}%
\pgfsetfillcolor{textcolor}%
\pgftext[x=5.030614in,y=0.796972in,,top]{\color{textcolor}\rmfamily\fontsize{20.000000}{24.000000}\selectfont 100}%
\end{pgfscope}%
\begin{pgfscope}%
\pgfpathrectangle{\pgfqpoint{0.802099in}{0.894194in}}{\pgfqpoint{6.824398in}{4.119217in}}%
\pgfusepath{clip}%
\pgfsetroundcap%
\pgfsetroundjoin%
\pgfsetlinewidth{1.003750pt}%
\definecolor{currentstroke}{rgb}{1.000000,1.000000,1.000000}%
\pgfsetstrokecolor{currentstroke}%
\pgfsetdash{}{0pt}%
\pgfpathmoveto{\pgfqpoint{6.051008in}{0.894194in}}%
\pgfpathlineto{\pgfqpoint{6.051008in}{5.013411in}}%
\pgfusepath{stroke}%
\end{pgfscope}%
\begin{pgfscope}%
\definecolor{textcolor}{rgb}{0.150000,0.150000,0.150000}%
\pgfsetstrokecolor{textcolor}%
\pgfsetfillcolor{textcolor}%
\pgftext[x=6.051008in,y=0.796972in,,top]{\color{textcolor}\rmfamily\fontsize{20.000000}{24.000000}\selectfont 125}%
\end{pgfscope}%
\begin{pgfscope}%
\pgfpathrectangle{\pgfqpoint{0.802099in}{0.894194in}}{\pgfqpoint{6.824398in}{4.119217in}}%
\pgfusepath{clip}%
\pgfsetroundcap%
\pgfsetroundjoin%
\pgfsetlinewidth{1.003750pt}%
\definecolor{currentstroke}{rgb}{1.000000,1.000000,1.000000}%
\pgfsetstrokecolor{currentstroke}%
\pgfsetdash{}{0pt}%
\pgfpathmoveto{\pgfqpoint{7.071403in}{0.894194in}}%
\pgfpathlineto{\pgfqpoint{7.071403in}{5.013411in}}%
\pgfusepath{stroke}%
\end{pgfscope}%
\begin{pgfscope}%
\definecolor{textcolor}{rgb}{0.150000,0.150000,0.150000}%
\pgfsetstrokecolor{textcolor}%
\pgfsetfillcolor{textcolor}%
\pgftext[x=7.071403in,y=0.796972in,,top]{\color{textcolor}\rmfamily\fontsize{20.000000}{24.000000}\selectfont 150}%
\end{pgfscope}%
\begin{pgfscope}%
\definecolor{textcolor}{rgb}{0.150000,0.150000,0.150000}%
\pgfsetstrokecolor{textcolor}%
\pgfsetfillcolor{textcolor}%
\pgftext[x=4.214298in,y=0.472591in,,top]{\color{textcolor}\rmfamily\fontsize{24.000000}{28.800000}\selectfont Time (ms)}%
\end{pgfscope}%
\begin{pgfscope}%
\pgfpathrectangle{\pgfqpoint{0.802099in}{0.894194in}}{\pgfqpoint{6.824398in}{4.119217in}}%
\pgfusepath{clip}%
\pgfsetroundcap%
\pgfsetroundjoin%
\pgfsetlinewidth{1.003750pt}%
\definecolor{currentstroke}{rgb}{1.000000,1.000000,1.000000}%
\pgfsetstrokecolor{currentstroke}%
\pgfsetdash{}{0pt}%
\pgfpathmoveto{\pgfqpoint{0.802099in}{1.301711in}}%
\pgfpathlineto{\pgfqpoint{7.626498in}{1.301711in}}%
\pgfusepath{stroke}%
\end{pgfscope}%
\begin{pgfscope}%
\definecolor{textcolor}{rgb}{0.150000,0.150000,0.150000}%
\pgfsetstrokecolor{textcolor}%
\pgfsetfillcolor{textcolor}%
\pgftext[x=0.528147in, y=1.196188in, left, base]{\color{textcolor}\rmfamily\fontsize{20.000000}{24.000000}\selectfont 1}%
\end{pgfscope}%
\begin{pgfscope}%
\pgfpathrectangle{\pgfqpoint{0.802099in}{0.894194in}}{\pgfqpoint{6.824398in}{4.119217in}}%
\pgfusepath{clip}%
\pgfsetroundcap%
\pgfsetroundjoin%
\pgfsetlinewidth{1.003750pt}%
\definecolor{currentstroke}{rgb}{1.000000,1.000000,1.000000}%
\pgfsetstrokecolor{currentstroke}%
\pgfsetdash{}{0pt}%
\pgfpathmoveto{\pgfqpoint{0.802099in}{2.219540in}}%
\pgfpathlineto{\pgfqpoint{7.626498in}{2.219540in}}%
\pgfusepath{stroke}%
\end{pgfscope}%
\begin{pgfscope}%
\definecolor{textcolor}{rgb}{0.150000,0.150000,0.150000}%
\pgfsetstrokecolor{textcolor}%
\pgfsetfillcolor{textcolor}%
\pgftext[x=0.528147in, y=2.114017in, left, base]{\color{textcolor}\rmfamily\fontsize{20.000000}{24.000000}\selectfont 2}%
\end{pgfscope}%
\begin{pgfscope}%
\pgfpathrectangle{\pgfqpoint{0.802099in}{0.894194in}}{\pgfqpoint{6.824398in}{4.119217in}}%
\pgfusepath{clip}%
\pgfsetroundcap%
\pgfsetroundjoin%
\pgfsetlinewidth{1.003750pt}%
\definecolor{currentstroke}{rgb}{1.000000,1.000000,1.000000}%
\pgfsetstrokecolor{currentstroke}%
\pgfsetdash{}{0pt}%
\pgfpathmoveto{\pgfqpoint{0.802099in}{3.137369in}}%
\pgfpathlineto{\pgfqpoint{7.626498in}{3.137369in}}%
\pgfusepath{stroke}%
\end{pgfscope}%
\begin{pgfscope}%
\definecolor{textcolor}{rgb}{0.150000,0.150000,0.150000}%
\pgfsetstrokecolor{textcolor}%
\pgfsetfillcolor{textcolor}%
\pgftext[x=0.528147in, y=3.031846in, left, base]{\color{textcolor}\rmfamily\fontsize{20.000000}{24.000000}\selectfont 3}%
\end{pgfscope}%
\begin{pgfscope}%
\pgfpathrectangle{\pgfqpoint{0.802099in}{0.894194in}}{\pgfqpoint{6.824398in}{4.119217in}}%
\pgfusepath{clip}%
\pgfsetroundcap%
\pgfsetroundjoin%
\pgfsetlinewidth{1.003750pt}%
\definecolor{currentstroke}{rgb}{1.000000,1.000000,1.000000}%
\pgfsetstrokecolor{currentstroke}%
\pgfsetdash{}{0pt}%
\pgfpathmoveto{\pgfqpoint{0.802099in}{4.055198in}}%
\pgfpathlineto{\pgfqpoint{7.626498in}{4.055198in}}%
\pgfusepath{stroke}%
\end{pgfscope}%
\begin{pgfscope}%
\definecolor{textcolor}{rgb}{0.150000,0.150000,0.150000}%
\pgfsetstrokecolor{textcolor}%
\pgfsetfillcolor{textcolor}%
\pgftext[x=0.528147in, y=3.949675in, left, base]{\color{textcolor}\rmfamily\fontsize{20.000000}{24.000000}\selectfont 4}%
\end{pgfscope}%
\begin{pgfscope}%
\pgfpathrectangle{\pgfqpoint{0.802099in}{0.894194in}}{\pgfqpoint{6.824398in}{4.119217in}}%
\pgfusepath{clip}%
\pgfsetroundcap%
\pgfsetroundjoin%
\pgfsetlinewidth{1.003750pt}%
\definecolor{currentstroke}{rgb}{1.000000,1.000000,1.000000}%
\pgfsetstrokecolor{currentstroke}%
\pgfsetdash{}{0pt}%
\pgfpathmoveto{\pgfqpoint{0.802099in}{4.973027in}}%
\pgfpathlineto{\pgfqpoint{7.626498in}{4.973027in}}%
\pgfusepath{stroke}%
\end{pgfscope}%
\begin{pgfscope}%
\definecolor{textcolor}{rgb}{0.150000,0.150000,0.150000}%
\pgfsetstrokecolor{textcolor}%
\pgfsetfillcolor{textcolor}%
\pgftext[x=0.528147in, y=4.867504in, left, base]{\color{textcolor}\rmfamily\fontsize{20.000000}{24.000000}\selectfont 5}%
\end{pgfscope}%
\begin{pgfscope}%
\definecolor{textcolor}{rgb}{0.150000,0.150000,0.150000}%
\pgfsetstrokecolor{textcolor}%
\pgfsetfillcolor{textcolor}%
\pgftext[x=0.472591in,y=2.953803in,,bottom,rotate=90.000000]{\color{textcolor}\rmfamily\fontsize{24.000000}{28.800000}\selectfont Pulse Amplitude (V)}%
\end{pgfscope}%
\begin{pgfscope}%
\pgfpathrectangle{\pgfqpoint{0.802099in}{0.894194in}}{\pgfqpoint{6.824398in}{4.119217in}}%
\pgfusepath{clip}%
\pgfsetbuttcap%
\pgfsetroundjoin%
\definecolor{currentfill}{rgb}{0.698039,0.133333,0.133333}%
\pgfsetfillcolor{currentfill}%
\pgfsetlinewidth{0.301125pt}%
\definecolor{currentstroke}{rgb}{0.698039,0.133333,0.133333}%
\pgfsetstrokecolor{currentstroke}%
\pgfsetdash{}{0pt}%
\pgfsys@defobject{currentmarker}{\pgfqpoint{-0.048611in}{-0.048611in}}{\pgfqpoint{0.048611in}{0.048611in}}{%
\pgfpathmoveto{\pgfqpoint{0.000000in}{-0.048611in}}%
\pgfpathcurveto{\pgfqpoint{0.012892in}{-0.048611in}}{\pgfqpoint{0.025257in}{-0.043489in}}{\pgfqpoint{0.034373in}{-0.034373in}}%
\pgfpathcurveto{\pgfqpoint{0.043489in}{-0.025257in}}{\pgfqpoint{0.048611in}{-0.012892in}}{\pgfqpoint{0.048611in}{0.000000in}}%
\pgfpathcurveto{\pgfqpoint{0.048611in}{0.012892in}}{\pgfqpoint{0.043489in}{0.025257in}}{\pgfqpoint{0.034373in}{0.034373in}}%
\pgfpathcurveto{\pgfqpoint{0.025257in}{0.043489in}}{\pgfqpoint{0.012892in}{0.048611in}}{\pgfqpoint{0.000000in}{0.048611in}}%
\pgfpathcurveto{\pgfqpoint{-0.012892in}{0.048611in}}{\pgfqpoint{-0.025257in}{0.043489in}}{\pgfqpoint{-0.034373in}{0.034373in}}%
\pgfpathcurveto{\pgfqpoint{-0.043489in}{0.025257in}}{\pgfqpoint{-0.048611in}{0.012892in}}{\pgfqpoint{-0.048611in}{0.000000in}}%
\pgfpathcurveto{\pgfqpoint{-0.048611in}{-0.012892in}}{\pgfqpoint{-0.043489in}{-0.025257in}}{\pgfqpoint{-0.034373in}{-0.034373in}}%
\pgfpathcurveto{\pgfqpoint{-0.025257in}{-0.043489in}}{\pgfqpoint{-0.012892in}{-0.048611in}}{\pgfqpoint{0.000000in}{-0.048611in}}%
\pgfpathlineto{\pgfqpoint{0.000000in}{-0.048611in}}%
\pgfpathclose%
\pgfusepath{stroke,fill}%
}%
\begin{pgfscope}%
\pgfsys@transformshift{1.112299in}{4.826174in}%
\pgfsys@useobject{currentmarker}{}%
\end{pgfscope}%
\begin{pgfscope}%
\pgfsys@transformshift{1.765352in}{3.761493in}%
\pgfsys@useobject{currentmarker}{}%
\end{pgfscope}%
\begin{pgfscope}%
\pgfsys@transformshift{2.418404in}{2.990516in}%
\pgfsys@useobject{currentmarker}{}%
\end{pgfscope}%
\begin{pgfscope}%
\pgfsys@transformshift{3.071457in}{2.660098in}%
\pgfsys@useobject{currentmarker}{}%
\end{pgfscope}%
\begin{pgfscope}%
\pgfsys@transformshift{3.724509in}{2.292966in}%
\pgfsys@useobject{currentmarker}{}%
\end{pgfscope}%
\begin{pgfscope}%
\pgfsys@transformshift{4.377562in}{1.999261in}%
\pgfsys@useobject{currentmarker}{}%
\end{pgfscope}%
\begin{pgfscope}%
\pgfsys@transformshift{5.030614in}{1.705555in}%
\pgfsys@useobject{currentmarker}{}%
\end{pgfscope}%
\begin{pgfscope}%
\pgfsys@transformshift{5.683666in}{1.595416in}%
\pgfsys@useobject{currentmarker}{}%
\end{pgfscope}%
\begin{pgfscope}%
\pgfsys@transformshift{6.336719in}{1.411850in}%
\pgfsys@useobject{currentmarker}{}%
\end{pgfscope}%
\begin{pgfscope}%
\pgfsys@transformshift{6.989771in}{1.191571in}%
\pgfsys@useobject{currentmarker}{}%
\end{pgfscope}%
\begin{pgfscope}%
\pgfsys@transformshift{7.316298in}{1.081432in}%
\pgfsys@useobject{currentmarker}{}%
\end{pgfscope}%
\end{pgfscope}%
\begin{pgfscope}%
\pgfpathrectangle{\pgfqpoint{0.802099in}{0.894194in}}{\pgfqpoint{6.824398in}{4.119217in}}%
\pgfusepath{clip}%
\pgfsetroundcap%
\pgfsetroundjoin%
\pgfsetlinewidth{1.756562pt}%
\definecolor{currentstroke}{rgb}{0.298039,0.447059,0.690196}%
\pgfsetstrokecolor{currentstroke}%
\pgfsetdash{}{0pt}%
\pgfpathmoveto{\pgfqpoint{1.112299in}{4.727949in}}%
\pgfpathlineto{\pgfqpoint{1.174966in}{4.632474in}}%
\pgfpathlineto{\pgfqpoint{1.237633in}{4.539373in}}%
\pgfpathlineto{\pgfqpoint{1.300299in}{4.448587in}}%
\pgfpathlineto{\pgfqpoint{1.362966in}{4.360058in}}%
\pgfpathlineto{\pgfqpoint{1.425633in}{4.273730in}}%
\pgfpathlineto{\pgfqpoint{1.488299in}{4.189548in}}%
\pgfpathlineto{\pgfqpoint{1.550966in}{4.107459in}}%
\pgfpathlineto{\pgfqpoint{1.613633in}{4.027410in}}%
\pgfpathlineto{\pgfqpoint{1.676299in}{3.949352in}}%
\pgfpathlineto{\pgfqpoint{1.738966in}{3.873235in}}%
\pgfpathlineto{\pgfqpoint{1.801632in}{3.799009in}}%
\pgfpathlineto{\pgfqpoint{1.864299in}{3.726630in}}%
\pgfpathlineto{\pgfqpoint{1.926966in}{3.656049in}}%
\pgfpathlineto{\pgfqpoint{1.989632in}{3.587223in}}%
\pgfpathlineto{\pgfqpoint{2.052299in}{3.520109in}}%
\pgfpathlineto{\pgfqpoint{2.114966in}{3.454663in}}%
\pgfpathlineto{\pgfqpoint{2.177632in}{3.390844in}}%
\pgfpathlineto{\pgfqpoint{2.240299in}{3.328611in}}%
\pgfpathlineto{\pgfqpoint{2.302966in}{3.267926in}}%
\pgfpathlineto{\pgfqpoint{2.365632in}{3.208750in}}%
\pgfpathlineto{\pgfqpoint{2.428299in}{3.151044in}}%
\pgfpathlineto{\pgfqpoint{2.490966in}{3.094774in}}%
\pgfpathlineto{\pgfqpoint{2.553632in}{3.039902in}}%
\pgfpathlineto{\pgfqpoint{2.616299in}{2.986394in}}%
\pgfpathlineto{\pgfqpoint{2.678966in}{2.934217in}}%
\pgfpathlineto{\pgfqpoint{2.741632in}{2.883337in}}%
\pgfpathlineto{\pgfqpoint{2.804299in}{2.833722in}}%
\pgfpathlineto{\pgfqpoint{2.866966in}{2.785340in}}%
\pgfpathlineto{\pgfqpoint{2.929632in}{2.738161in}}%
\pgfpathlineto{\pgfqpoint{2.992299in}{2.692155in}}%
\pgfpathlineto{\pgfqpoint{3.054965in}{2.647293in}}%
\pgfpathlineto{\pgfqpoint{3.117632in}{2.603546in}}%
\pgfpathlineto{\pgfqpoint{3.180299in}{2.560887in}}%
\pgfpathlineto{\pgfqpoint{3.242965in}{2.519288in}}%
\pgfpathlineto{\pgfqpoint{3.305632in}{2.478724in}}%
\pgfpathlineto{\pgfqpoint{3.368299in}{2.439168in}}%
\pgfpathlineto{\pgfqpoint{3.430965in}{2.400595in}}%
\pgfpathlineto{\pgfqpoint{3.493632in}{2.362982in}}%
\pgfpathlineto{\pgfqpoint{3.556299in}{2.326303in}}%
\pgfpathlineto{\pgfqpoint{3.618965in}{2.290536in}}%
\pgfpathlineto{\pgfqpoint{3.681632in}{2.255659in}}%
\pgfpathlineto{\pgfqpoint{3.744299in}{2.221649in}}%
\pgfpathlineto{\pgfqpoint{3.806965in}{2.188484in}}%
\pgfpathlineto{\pgfqpoint{3.869632in}{2.156143in}}%
\pgfpathlineto{\pgfqpoint{3.932299in}{2.124607in}}%
\pgfpathlineto{\pgfqpoint{3.994965in}{2.093855in}}%
\pgfpathlineto{\pgfqpoint{4.057632in}{2.063867in}}%
\pgfpathlineto{\pgfqpoint{4.120299in}{2.034625in}}%
\pgfpathlineto{\pgfqpoint{4.182965in}{2.006110in}}%
\pgfpathlineto{\pgfqpoint{4.245632in}{1.978304in}}%
\pgfpathlineto{\pgfqpoint{4.308298in}{1.951189in}}%
\pgfpathlineto{\pgfqpoint{4.370965in}{1.924748in}}%
\pgfpathlineto{\pgfqpoint{4.433632in}{1.898964in}}%
\pgfpathlineto{\pgfqpoint{4.496298in}{1.873822in}}%
\pgfpathlineto{\pgfqpoint{4.558965in}{1.849304in}}%
\pgfpathlineto{\pgfqpoint{4.621632in}{1.825396in}}%
\pgfpathlineto{\pgfqpoint{4.684298in}{1.802083in}}%
\pgfpathlineto{\pgfqpoint{4.746965in}{1.779349in}}%
\pgfpathlineto{\pgfqpoint{4.809632in}{1.757180in}}%
\pgfpathlineto{\pgfqpoint{4.872298in}{1.735563in}}%
\pgfpathlineto{\pgfqpoint{4.934965in}{1.714482in}}%
\pgfpathlineto{\pgfqpoint{4.997632in}{1.693926in}}%
\pgfpathlineto{\pgfqpoint{5.060298in}{1.673881in}}%
\pgfpathlineto{\pgfqpoint{5.122965in}{1.654335in}}%
\pgfpathlineto{\pgfqpoint{5.185632in}{1.635274in}}%
\pgfpathlineto{\pgfqpoint{5.248298in}{1.616687in}}%
\pgfpathlineto{\pgfqpoint{5.310965in}{1.598562in}}%
\pgfpathlineto{\pgfqpoint{5.373631in}{1.580888in}}%
\pgfpathlineto{\pgfqpoint{5.436298in}{1.563653in}}%
\pgfpathlineto{\pgfqpoint{5.498965in}{1.546847in}}%
\pgfpathlineto{\pgfqpoint{5.561631in}{1.530459in}}%
\pgfpathlineto{\pgfqpoint{5.624298in}{1.514478in}}%
\pgfpathlineto{\pgfqpoint{5.686965in}{1.498894in}}%
\pgfpathlineto{\pgfqpoint{5.749631in}{1.483698in}}%
\pgfpathlineto{\pgfqpoint{5.812298in}{1.468879in}}%
\pgfpathlineto{\pgfqpoint{5.874965in}{1.454429in}}%
\pgfpathlineto{\pgfqpoint{5.937631in}{1.440338in}}%
\pgfpathlineto{\pgfqpoint{6.000298in}{1.426598in}}%
\pgfpathlineto{\pgfqpoint{6.062965in}{1.413199in}}%
\pgfpathlineto{\pgfqpoint{6.125631in}{1.400133in}}%
\pgfpathlineto{\pgfqpoint{6.188298in}{1.387392in}}%
\pgfpathlineto{\pgfqpoint{6.250965in}{1.374968in}}%
\pgfpathlineto{\pgfqpoint{6.313631in}{1.362852in}}%
\pgfpathlineto{\pgfqpoint{6.376298in}{1.351038in}}%
\pgfpathlineto{\pgfqpoint{6.438965in}{1.339518in}}%
\pgfpathlineto{\pgfqpoint{6.501631in}{1.328284in}}%
\pgfpathlineto{\pgfqpoint{6.564298in}{1.317329in}}%
\pgfpathlineto{\pgfqpoint{6.626964in}{1.306647in}}%
\pgfpathlineto{\pgfqpoint{6.689631in}{1.296230in}}%
\pgfpathlineto{\pgfqpoint{6.752298in}{1.286072in}}%
\pgfpathlineto{\pgfqpoint{6.814964in}{1.276167in}}%
\pgfpathlineto{\pgfqpoint{6.877631in}{1.266508in}}%
\pgfpathlineto{\pgfqpoint{6.940298in}{1.257089in}}%
\pgfpathlineto{\pgfqpoint{7.002964in}{1.247904in}}%
\pgfpathlineto{\pgfqpoint{7.065631in}{1.238948in}}%
\pgfpathlineto{\pgfqpoint{7.128298in}{1.230214in}}%
\pgfpathlineto{\pgfqpoint{7.190964in}{1.221698in}}%
\pgfpathlineto{\pgfqpoint{7.253631in}{1.213393in}}%
\pgfpathlineto{\pgfqpoint{7.316298in}{1.205294in}}%
\pgfusepath{stroke}%
\end{pgfscope}%
\begin{pgfscope}%
\pgfsetrectcap%
\pgfsetmiterjoin%
\pgfsetlinewidth{0.000000pt}%
\definecolor{currentstroke}{rgb}{1.000000,1.000000,1.000000}%
\pgfsetstrokecolor{currentstroke}%
\pgfsetdash{}{0pt}%
\pgfpathmoveto{\pgfqpoint{0.802099in}{0.894194in}}%
\pgfpathlineto{\pgfqpoint{0.802099in}{5.013411in}}%
\pgfusepath{}%
\end{pgfscope}%
\begin{pgfscope}%
\pgfsetrectcap%
\pgfsetmiterjoin%
\pgfsetlinewidth{0.000000pt}%
\definecolor{currentstroke}{rgb}{1.000000,1.000000,1.000000}%
\pgfsetstrokecolor{currentstroke}%
\pgfsetdash{}{0pt}%
\pgfpathmoveto{\pgfqpoint{7.626498in}{0.894194in}}%
\pgfpathlineto{\pgfqpoint{7.626498in}{5.013411in}}%
\pgfusepath{}%
\end{pgfscope}%
\begin{pgfscope}%
\pgfsetrectcap%
\pgfsetmiterjoin%
\pgfsetlinewidth{0.000000pt}%
\definecolor{currentstroke}{rgb}{1.000000,1.000000,1.000000}%
\pgfsetstrokecolor{currentstroke}%
\pgfsetdash{}{0pt}%
\pgfpathmoveto{\pgfqpoint{0.802099in}{0.894194in}}%
\pgfpathlineto{\pgfqpoint{7.626498in}{0.894194in}}%
\pgfusepath{}%
\end{pgfscope}%
\begin{pgfscope}%
\pgfsetrectcap%
\pgfsetmiterjoin%
\pgfsetlinewidth{0.000000pt}%
\definecolor{currentstroke}{rgb}{1.000000,1.000000,1.000000}%
\pgfsetstrokecolor{currentstroke}%
\pgfsetdash{}{0pt}%
\pgfpathmoveto{\pgfqpoint{0.802099in}{5.013411in}}%
\pgfpathlineto{\pgfqpoint{7.626498in}{5.013411in}}%
\pgfusepath{}%
\end{pgfscope}%
\begin{pgfscope}%
\definecolor{textcolor}{rgb}{0.150000,0.150000,0.150000}%
\pgfsetstrokecolor{textcolor}%
\pgfsetfillcolor{textcolor}%
\pgftext[x=4.010220in, y=3.631424in, left, base]{\color{textcolor}\rmfamily\fontsize{20.000000}{24.000000}\selectfont M\(\displaystyle _z\)(t) = 4.47\(\displaystyle exp[\frac{-t}{T_2}]+\)0.55}%
\end{pgfscope}%
\begin{pgfscope}%
\definecolor{textcolor}{rgb}{0.150000,0.150000,0.150000}%
\pgfsetstrokecolor{textcolor}%
\pgfsetfillcolor{textcolor}%
\pgftext[x=4.010220in, y=3.137369in, left, base]{\color{textcolor}\rmfamily\fontsize{20.000000}{24.000000}\selectfont T\(\displaystyle _2\) = 60.9\(\displaystyle \pm\)5.7 ms}%
\end{pgfscope}%
\begin{pgfscope}%
\definecolor{textcolor}{rgb}{0.150000,0.150000,0.150000}%
\pgfsetstrokecolor{textcolor}%
\pgfsetfillcolor{textcolor}%
\pgftext[x=4.214298in,y=5.096745in,,base]{\color{textcolor}\rmfamily\fontsize{24.000000}{28.800000}\selectfont  Meiboom-Gill T\(\displaystyle _2\) measurement of Mineral Oil}%
\end{pgfscope}%
\begin{pgfscope}%
\pgfsetbuttcap%
\pgfsetroundjoin%
\definecolor{currentfill}{rgb}{0.698039,0.133333,0.133333}%
\pgfsetfillcolor{currentfill}%
\pgfsetlinewidth{0.301125pt}%
\definecolor{currentstroke}{rgb}{0.698039,0.133333,0.133333}%
\pgfsetstrokecolor{currentstroke}%
\pgfsetdash{}{0pt}%
\pgfsys@defobject{currentmarker}{\pgfqpoint{-0.048611in}{-0.048611in}}{\pgfqpoint{0.048611in}{0.048611in}}{%
\pgfpathmoveto{\pgfqpoint{0.000000in}{-0.048611in}}%
\pgfpathcurveto{\pgfqpoint{0.012892in}{-0.048611in}}{\pgfqpoint{0.025257in}{-0.043489in}}{\pgfqpoint{0.034373in}{-0.034373in}}%
\pgfpathcurveto{\pgfqpoint{0.043489in}{-0.025257in}}{\pgfqpoint{0.048611in}{-0.012892in}}{\pgfqpoint{0.048611in}{0.000000in}}%
\pgfpathcurveto{\pgfqpoint{0.048611in}{0.012892in}}{\pgfqpoint{0.043489in}{0.025257in}}{\pgfqpoint{0.034373in}{0.034373in}}%
\pgfpathcurveto{\pgfqpoint{0.025257in}{0.043489in}}{\pgfqpoint{0.012892in}{0.048611in}}{\pgfqpoint{0.000000in}{0.048611in}}%
\pgfpathcurveto{\pgfqpoint{-0.012892in}{0.048611in}}{\pgfqpoint{-0.025257in}{0.043489in}}{\pgfqpoint{-0.034373in}{0.034373in}}%
\pgfpathcurveto{\pgfqpoint{-0.043489in}{0.025257in}}{\pgfqpoint{-0.048611in}{0.012892in}}{\pgfqpoint{-0.048611in}{0.000000in}}%
\pgfpathcurveto{\pgfqpoint{-0.048611in}{-0.012892in}}{\pgfqpoint{-0.043489in}{-0.025257in}}{\pgfqpoint{-0.034373in}{-0.034373in}}%
\pgfpathcurveto{\pgfqpoint{-0.025257in}{-0.043489in}}{\pgfqpoint{-0.012892in}{-0.048611in}}{\pgfqpoint{0.000000in}{-0.048611in}}%
\pgfpathlineto{\pgfqpoint{0.000000in}{-0.048611in}}%
\pgfpathclose%
\pgfusepath{stroke,fill}%
}%
\begin{pgfscope}%
\pgfsys@transformshift{5.095627in}{4.625282in}%
\pgfsys@useobject{currentmarker}{}%
\end{pgfscope}%
\end{pgfscope}%
\begin{pgfscope}%
\definecolor{textcolor}{rgb}{0.150000,0.150000,0.150000}%
\pgfsetstrokecolor{textcolor}%
\pgfsetfillcolor{textcolor}%
\pgftext[x=5.595627in,y=4.552365in,left,base]{\color{textcolor}\rmfamily\fontsize{20.000000}{24.000000}\selectfont Data}%
\end{pgfscope}%
\begin{pgfscope}%
\pgfsetroundcap%
\pgfsetroundjoin%
\pgfsetlinewidth{1.756562pt}%
\definecolor{currentstroke}{rgb}{0.298039,0.447059,0.690196}%
\pgfsetstrokecolor{currentstroke}%
\pgfsetdash{}{0pt}%
\pgfpathmoveto{\pgfqpoint{4.817849in}{4.241873in}}%
\pgfpathlineto{\pgfqpoint{5.095627in}{4.241873in}}%
\pgfpathlineto{\pgfqpoint{5.373405in}{4.241873in}}%
\pgfusepath{stroke}%
\end{pgfscope}%
\begin{pgfscope}%
\definecolor{textcolor}{rgb}{0.150000,0.150000,0.150000}%
\pgfsetstrokecolor{textcolor}%
\pgfsetfillcolor{textcolor}%
\pgftext[x=5.595627in,y=4.144651in,left,base]{\color{textcolor}\rmfamily\fontsize{20.000000}{24.000000}\selectfont Fitted Curve}%
\end{pgfscope}%
\end{pgfpicture}%
\makeatother%
\endgroup%
}
    \caption{T2 measurement of Mineral oil using the Meiboom-Gill pulse sequence }
    \label{fig:mo_t2}
  \end{figure}

  %%%%%%%%%%%%%%%%%%%%%%%%%%%%%%%%%
  %%%%%%%%%% CuSo4 T1 %%%%%%%%%%%%%
  %%%%%%%%%%%%%%%%%%%%%%%%%%%%%%%%%
  
  \begin{figure}[h]
    \resizebox{0.45\textwidth}{!}{%% Creator: Matplotlib, PGF backend
%%
%% To include the figure in your LaTeX document, write
%%   \input{<filename>.pgf}
%%
%% Make sure the required packages are loaded in your preamble
%%   \usepackage{pgf}
%%
%% Also ensure that all the required font packages are loaded; for instance,
%% the lmodern package is sometimes necessary when using math font.
%%   \usepackage{lmodern}
%%
%% Figures using additional raster images can only be included by \input if
%% they are in the same directory as the main LaTeX file. For loading figures
%% from other directories you can use the `import` package
%%   \usepackage{import}
%%
%% and then include the figures with
%%   \import{<path to file>}{<filename>.pgf}
%%
%% Matplotlib used the following preamble
%%   \usepackage{fontspec}
%%   \setmainfont{DejaVuSerif.ttf}[Path=\detokenize{/usr/share/matplotlib/mpl-data/fonts/ttf/}]
%%   \setsansfont{DejaVuSans.ttf}[Path=\detokenize{/usr/share/matplotlib/mpl-data/fonts/ttf/}]
%%   \setmonofont{DejaVuSansMono.ttf}[Path=\detokenize{/usr/share/matplotlib/mpl-data/fonts/ttf/}]
%%
\begingroup%
\makeatletter%
\begin{pgfpicture}%
\pgfpathrectangle{\pgfpointorigin}{\pgfqpoint{8.000000in}{5.500000in}}%
\pgfusepath{use as bounding box, clip}%
\begin{pgfscope}%
\pgfsetbuttcap%
\pgfsetmiterjoin%
\definecolor{currentfill}{rgb}{1.000000,1.000000,1.000000}%
\pgfsetfillcolor{currentfill}%
\pgfsetlinewidth{0.000000pt}%
\definecolor{currentstroke}{rgb}{1.000000,1.000000,1.000000}%
\pgfsetstrokecolor{currentstroke}%
\pgfsetdash{}{0pt}%
\pgfpathmoveto{\pgfqpoint{0.000000in}{0.000000in}}%
\pgfpathlineto{\pgfqpoint{8.000000in}{0.000000in}}%
\pgfpathlineto{\pgfqpoint{8.000000in}{5.500000in}}%
\pgfpathlineto{\pgfqpoint{0.000000in}{5.500000in}}%
\pgfpathlineto{\pgfqpoint{0.000000in}{0.000000in}}%
\pgfpathclose%
\pgfusepath{fill}%
\end{pgfscope}%
\begin{pgfscope}%
\pgfsetbuttcap%
\pgfsetmiterjoin%
\definecolor{currentfill}{rgb}{0.917647,0.917647,0.949020}%
\pgfsetfillcolor{currentfill}%
\pgfsetlinewidth{0.000000pt}%
\definecolor{currentstroke}{rgb}{0.000000,0.000000,0.000000}%
\pgfsetstrokecolor{currentstroke}%
\pgfsetstrokeopacity{0.000000}%
\pgfsetdash{}{0pt}%
\pgfpathmoveto{\pgfqpoint{1.026143in}{0.898915in}}%
\pgfpathlineto{\pgfqpoint{7.788711in}{0.898915in}}%
\pgfpathlineto{\pgfqpoint{7.788711in}{5.013411in}}%
\pgfpathlineto{\pgfqpoint{1.026143in}{5.013411in}}%
\pgfpathlineto{\pgfqpoint{1.026143in}{0.898915in}}%
\pgfpathclose%
\pgfusepath{fill}%
\end{pgfscope}%
\begin{pgfscope}%
\pgfpathrectangle{\pgfqpoint{1.026143in}{0.898915in}}{\pgfqpoint{6.762568in}{4.114497in}}%
\pgfusepath{clip}%
\pgfsetroundcap%
\pgfsetroundjoin%
\pgfsetlinewidth{1.003750pt}%
\definecolor{currentstroke}{rgb}{1.000000,1.000000,1.000000}%
\pgfsetstrokecolor{currentstroke}%
\pgfsetdash{}{0pt}%
\pgfpathmoveto{\pgfqpoint{1.277131in}{0.898915in}}%
\pgfpathlineto{\pgfqpoint{1.277131in}{5.013411in}}%
\pgfusepath{stroke}%
\end{pgfscope}%
\begin{pgfscope}%
\definecolor{textcolor}{rgb}{0.150000,0.150000,0.150000}%
\pgfsetstrokecolor{textcolor}%
\pgfsetfillcolor{textcolor}%
\pgftext[x=1.277131in,y=0.801692in,,top]{\color{textcolor}\rmfamily\fontsize{20.000000}{24.000000}\selectfont 0}%
\end{pgfscope}%
\begin{pgfscope}%
\pgfpathrectangle{\pgfqpoint{1.026143in}{0.898915in}}{\pgfqpoint{6.762568in}{4.114497in}}%
\pgfusepath{clip}%
\pgfsetroundcap%
\pgfsetroundjoin%
\pgfsetlinewidth{1.003750pt}%
\definecolor{currentstroke}{rgb}{1.000000,1.000000,1.000000}%
\pgfsetstrokecolor{currentstroke}%
\pgfsetdash{}{0pt}%
\pgfpathmoveto{\pgfqpoint{2.405166in}{0.898915in}}%
\pgfpathlineto{\pgfqpoint{2.405166in}{5.013411in}}%
\pgfusepath{stroke}%
\end{pgfscope}%
\begin{pgfscope}%
\definecolor{textcolor}{rgb}{0.150000,0.150000,0.150000}%
\pgfsetstrokecolor{textcolor}%
\pgfsetfillcolor{textcolor}%
\pgftext[x=2.405166in,y=0.801692in,,top]{\color{textcolor}\rmfamily\fontsize{20.000000}{24.000000}\selectfont 20}%
\end{pgfscope}%
\begin{pgfscope}%
\pgfpathrectangle{\pgfqpoint{1.026143in}{0.898915in}}{\pgfqpoint{6.762568in}{4.114497in}}%
\pgfusepath{clip}%
\pgfsetroundcap%
\pgfsetroundjoin%
\pgfsetlinewidth{1.003750pt}%
\definecolor{currentstroke}{rgb}{1.000000,1.000000,1.000000}%
\pgfsetstrokecolor{currentstroke}%
\pgfsetdash{}{0pt}%
\pgfpathmoveto{\pgfqpoint{3.533200in}{0.898915in}}%
\pgfpathlineto{\pgfqpoint{3.533200in}{5.013411in}}%
\pgfusepath{stroke}%
\end{pgfscope}%
\begin{pgfscope}%
\definecolor{textcolor}{rgb}{0.150000,0.150000,0.150000}%
\pgfsetstrokecolor{textcolor}%
\pgfsetfillcolor{textcolor}%
\pgftext[x=3.533200in,y=0.801692in,,top]{\color{textcolor}\rmfamily\fontsize{20.000000}{24.000000}\selectfont 40}%
\end{pgfscope}%
\begin{pgfscope}%
\pgfpathrectangle{\pgfqpoint{1.026143in}{0.898915in}}{\pgfqpoint{6.762568in}{4.114497in}}%
\pgfusepath{clip}%
\pgfsetroundcap%
\pgfsetroundjoin%
\pgfsetlinewidth{1.003750pt}%
\definecolor{currentstroke}{rgb}{1.000000,1.000000,1.000000}%
\pgfsetstrokecolor{currentstroke}%
\pgfsetdash{}{0pt}%
\pgfpathmoveto{\pgfqpoint{4.661235in}{0.898915in}}%
\pgfpathlineto{\pgfqpoint{4.661235in}{5.013411in}}%
\pgfusepath{stroke}%
\end{pgfscope}%
\begin{pgfscope}%
\definecolor{textcolor}{rgb}{0.150000,0.150000,0.150000}%
\pgfsetstrokecolor{textcolor}%
\pgfsetfillcolor{textcolor}%
\pgftext[x=4.661235in,y=0.801692in,,top]{\color{textcolor}\rmfamily\fontsize{20.000000}{24.000000}\selectfont 60}%
\end{pgfscope}%
\begin{pgfscope}%
\pgfpathrectangle{\pgfqpoint{1.026143in}{0.898915in}}{\pgfqpoint{6.762568in}{4.114497in}}%
\pgfusepath{clip}%
\pgfsetroundcap%
\pgfsetroundjoin%
\pgfsetlinewidth{1.003750pt}%
\definecolor{currentstroke}{rgb}{1.000000,1.000000,1.000000}%
\pgfsetstrokecolor{currentstroke}%
\pgfsetdash{}{0pt}%
\pgfpathmoveto{\pgfqpoint{5.789270in}{0.898915in}}%
\pgfpathlineto{\pgfqpoint{5.789270in}{5.013411in}}%
\pgfusepath{stroke}%
\end{pgfscope}%
\begin{pgfscope}%
\definecolor{textcolor}{rgb}{0.150000,0.150000,0.150000}%
\pgfsetstrokecolor{textcolor}%
\pgfsetfillcolor{textcolor}%
\pgftext[x=5.789270in,y=0.801692in,,top]{\color{textcolor}\rmfamily\fontsize{20.000000}{24.000000}\selectfont 80}%
\end{pgfscope}%
\begin{pgfscope}%
\pgfpathrectangle{\pgfqpoint{1.026143in}{0.898915in}}{\pgfqpoint{6.762568in}{4.114497in}}%
\pgfusepath{clip}%
\pgfsetroundcap%
\pgfsetroundjoin%
\pgfsetlinewidth{1.003750pt}%
\definecolor{currentstroke}{rgb}{1.000000,1.000000,1.000000}%
\pgfsetstrokecolor{currentstroke}%
\pgfsetdash{}{0pt}%
\pgfpathmoveto{\pgfqpoint{6.917304in}{0.898915in}}%
\pgfpathlineto{\pgfqpoint{6.917304in}{5.013411in}}%
\pgfusepath{stroke}%
\end{pgfscope}%
\begin{pgfscope}%
\definecolor{textcolor}{rgb}{0.150000,0.150000,0.150000}%
\pgfsetstrokecolor{textcolor}%
\pgfsetfillcolor{textcolor}%
\pgftext[x=6.917304in,y=0.801692in,,top]{\color{textcolor}\rmfamily\fontsize{20.000000}{24.000000}\selectfont 100}%
\end{pgfscope}%
\begin{pgfscope}%
\definecolor{textcolor}{rgb}{0.150000,0.150000,0.150000}%
\pgfsetstrokecolor{textcolor}%
\pgfsetfillcolor{textcolor}%
\pgftext[x=4.407427in,y=0.477311in,,top]{\color{textcolor}\rmfamily\fontsize{24.000000}{28.800000}\selectfont Delay Time (ms)}%
\end{pgfscope}%
\begin{pgfscope}%
\pgfpathrectangle{\pgfqpoint{1.026143in}{0.898915in}}{\pgfqpoint{6.762568in}{4.114497in}}%
\pgfusepath{clip}%
\pgfsetroundcap%
\pgfsetroundjoin%
\pgfsetlinewidth{1.003750pt}%
\definecolor{currentstroke}{rgb}{1.000000,1.000000,1.000000}%
\pgfsetstrokecolor{currentstroke}%
\pgfsetdash{}{0pt}%
\pgfpathmoveto{\pgfqpoint{1.026143in}{1.553260in}}%
\pgfpathlineto{\pgfqpoint{7.788711in}{1.553260in}}%
\pgfusepath{stroke}%
\end{pgfscope}%
\begin{pgfscope}%
\definecolor{textcolor}{rgb}{0.150000,0.150000,0.150000}%
\pgfsetstrokecolor{textcolor}%
\pgfsetfillcolor{textcolor}%
\pgftext[x=0.528147in, y=1.447737in, left, base]{\color{textcolor}\rmfamily\fontsize{20.000000}{24.000000}\selectfont \ensuremath{-}8}%
\end{pgfscope}%
\begin{pgfscope}%
\pgfpathrectangle{\pgfqpoint{1.026143in}{0.898915in}}{\pgfqpoint{6.762568in}{4.114497in}}%
\pgfusepath{clip}%
\pgfsetroundcap%
\pgfsetroundjoin%
\pgfsetlinewidth{1.003750pt}%
\definecolor{currentstroke}{rgb}{1.000000,1.000000,1.000000}%
\pgfsetstrokecolor{currentstroke}%
\pgfsetdash{}{0pt}%
\pgfpathmoveto{\pgfqpoint{1.026143in}{2.451958in}}%
\pgfpathlineto{\pgfqpoint{7.788711in}{2.451958in}}%
\pgfusepath{stroke}%
\end{pgfscope}%
\begin{pgfscope}%
\definecolor{textcolor}{rgb}{0.150000,0.150000,0.150000}%
\pgfsetstrokecolor{textcolor}%
\pgfsetfillcolor{textcolor}%
\pgftext[x=0.528147in, y=2.346435in, left, base]{\color{textcolor}\rmfamily\fontsize{20.000000}{24.000000}\selectfont \ensuremath{-}6}%
\end{pgfscope}%
\begin{pgfscope}%
\pgfpathrectangle{\pgfqpoint{1.026143in}{0.898915in}}{\pgfqpoint{6.762568in}{4.114497in}}%
\pgfusepath{clip}%
\pgfsetroundcap%
\pgfsetroundjoin%
\pgfsetlinewidth{1.003750pt}%
\definecolor{currentstroke}{rgb}{1.000000,1.000000,1.000000}%
\pgfsetstrokecolor{currentstroke}%
\pgfsetdash{}{0pt}%
\pgfpathmoveto{\pgfqpoint{1.026143in}{3.350656in}}%
\pgfpathlineto{\pgfqpoint{7.788711in}{3.350656in}}%
\pgfusepath{stroke}%
\end{pgfscope}%
\begin{pgfscope}%
\definecolor{textcolor}{rgb}{0.150000,0.150000,0.150000}%
\pgfsetstrokecolor{textcolor}%
\pgfsetfillcolor{textcolor}%
\pgftext[x=0.528147in, y=3.245133in, left, base]{\color{textcolor}\rmfamily\fontsize{20.000000}{24.000000}\selectfont \ensuremath{-}4}%
\end{pgfscope}%
\begin{pgfscope}%
\pgfpathrectangle{\pgfqpoint{1.026143in}{0.898915in}}{\pgfqpoint{6.762568in}{4.114497in}}%
\pgfusepath{clip}%
\pgfsetroundcap%
\pgfsetroundjoin%
\pgfsetlinewidth{1.003750pt}%
\definecolor{currentstroke}{rgb}{1.000000,1.000000,1.000000}%
\pgfsetstrokecolor{currentstroke}%
\pgfsetdash{}{0pt}%
\pgfpathmoveto{\pgfqpoint{1.026143in}{4.249355in}}%
\pgfpathlineto{\pgfqpoint{7.788711in}{4.249355in}}%
\pgfusepath{stroke}%
\end{pgfscope}%
\begin{pgfscope}%
\definecolor{textcolor}{rgb}{0.150000,0.150000,0.150000}%
\pgfsetstrokecolor{textcolor}%
\pgfsetfillcolor{textcolor}%
\pgftext[x=0.528147in, y=4.143832in, left, base]{\color{textcolor}\rmfamily\fontsize{20.000000}{24.000000}\selectfont \ensuremath{-}2}%
\end{pgfscope}%
\begin{pgfscope}%
\definecolor{textcolor}{rgb}{0.150000,0.150000,0.150000}%
\pgfsetstrokecolor{textcolor}%
\pgfsetfillcolor{textcolor}%
\pgftext[x=0.472591in,y=2.956163in,,bottom,rotate=90.000000]{\color{textcolor}\rmfamily\fontsize{24.000000}{28.800000}\selectfont Pulse Amplitude (V)}%
\end{pgfscope}%
\begin{pgfscope}%
\pgfpathrectangle{\pgfqpoint{1.026143in}{0.898915in}}{\pgfqpoint{6.762568in}{4.114497in}}%
\pgfusepath{clip}%
\pgfsetbuttcap%
\pgfsetroundjoin%
\definecolor{currentfill}{rgb}{0.392157,0.584314,0.929412}%
\pgfsetfillcolor{currentfill}%
\pgfsetlinewidth{0.301125pt}%
\definecolor{currentstroke}{rgb}{0.392157,0.584314,0.929412}%
\pgfsetstrokecolor{currentstroke}%
\pgfsetdash{}{0pt}%
\pgfsys@defobject{currentmarker}{\pgfqpoint{-0.048611in}{-0.048611in}}{\pgfqpoint{0.048611in}{0.048611in}}{%
\pgfpathmoveto{\pgfqpoint{0.000000in}{-0.048611in}}%
\pgfpathcurveto{\pgfqpoint{0.012892in}{-0.048611in}}{\pgfqpoint{0.025257in}{-0.043489in}}{\pgfqpoint{0.034373in}{-0.034373in}}%
\pgfpathcurveto{\pgfqpoint{0.043489in}{-0.025257in}}{\pgfqpoint{0.048611in}{-0.012892in}}{\pgfqpoint{0.048611in}{0.000000in}}%
\pgfpathcurveto{\pgfqpoint{0.048611in}{0.012892in}}{\pgfqpoint{0.043489in}{0.025257in}}{\pgfqpoint{0.034373in}{0.034373in}}%
\pgfpathcurveto{\pgfqpoint{0.025257in}{0.043489in}}{\pgfqpoint{0.012892in}{0.048611in}}{\pgfqpoint{0.000000in}{0.048611in}}%
\pgfpathcurveto{\pgfqpoint{-0.012892in}{0.048611in}}{\pgfqpoint{-0.025257in}{0.043489in}}{\pgfqpoint{-0.034373in}{0.034373in}}%
\pgfpathcurveto{\pgfqpoint{-0.043489in}{0.025257in}}{\pgfqpoint{-0.048611in}{0.012892in}}{\pgfqpoint{-0.048611in}{0.000000in}}%
\pgfpathcurveto{\pgfqpoint{-0.048611in}{-0.012892in}}{\pgfqpoint{-0.043489in}{-0.025257in}}{\pgfqpoint{-0.034373in}{-0.034373in}}%
\pgfpathcurveto{\pgfqpoint{-0.025257in}{-0.043489in}}{\pgfqpoint{-0.012892in}{-0.048611in}}{\pgfqpoint{0.000000in}{-0.048611in}}%
\pgfpathlineto{\pgfqpoint{0.000000in}{-0.048611in}}%
\pgfpathclose%
\pgfusepath{stroke,fill}%
}%
\begin{pgfscope}%
\pgfsys@transformshift{1.333533in}{1.085937in}%
\pgfsys@useobject{currentmarker}{}%
\end{pgfscope}%
\begin{pgfscope}%
\pgfsys@transformshift{1.841148in}{1.517312in}%
\pgfsys@useobject{currentmarker}{}%
\end{pgfscope}%
\begin{pgfscope}%
\pgfsys@transformshift{2.405166in}{1.876792in}%
\pgfsys@useobject{currentmarker}{}%
\end{pgfscope}%
\begin{pgfscope}%
\pgfsys@transformshift{2.969183in}{2.272219in}%
\pgfsys@useobject{currentmarker}{}%
\end{pgfscope}%
\begin{pgfscope}%
\pgfsys@transformshift{3.533200in}{2.631698in}%
\pgfsys@useobject{currentmarker}{}%
\end{pgfscope}%
\begin{pgfscope}%
\pgfsys@transformshift{4.097218in}{2.991177in}%
\pgfsys@useobject{currentmarker}{}%
\end{pgfscope}%
\begin{pgfscope}%
\pgfsys@transformshift{4.661235in}{3.314709in}%
\pgfsys@useobject{currentmarker}{}%
\end{pgfscope}%
\begin{pgfscope}%
\pgfsys@transformshift{5.225252in}{3.638240in}%
\pgfsys@useobject{currentmarker}{}%
\end{pgfscope}%
\begin{pgfscope}%
\pgfsys@transformshift{5.789270in}{3.961771in}%
\pgfsys@useobject{currentmarker}{}%
\end{pgfscope}%
\begin{pgfscope}%
\pgfsys@transformshift{6.353287in}{4.249355in}%
\pgfsys@useobject{currentmarker}{}%
\end{pgfscope}%
\begin{pgfscope}%
\pgfsys@transformshift{6.917304in}{4.644782in}%
\pgfsys@useobject{currentmarker}{}%
\end{pgfscope}%
\begin{pgfscope}%
\pgfsys@transformshift{7.481321in}{4.770600in}%
\pgfsys@useobject{currentmarker}{}%
\end{pgfscope}%
\end{pgfscope}%
\begin{pgfscope}%
\pgfpathrectangle{\pgfqpoint{1.026143in}{0.898915in}}{\pgfqpoint{6.762568in}{4.114497in}}%
\pgfusepath{clip}%
\pgfsetroundcap%
\pgfsetroundjoin%
\pgfsetlinewidth{1.756562pt}%
\definecolor{currentstroke}{rgb}{0.298039,0.447059,0.690196}%
\pgfsetstrokecolor{currentstroke}%
\pgfsetdash{}{0pt}%
\pgfpathmoveto{\pgfqpoint{1.333533in}{1.107574in}}%
\pgfpathlineto{\pgfqpoint{1.395632in}{1.154216in}}%
\pgfpathlineto{\pgfqpoint{1.457731in}{1.200644in}}%
\pgfpathlineto{\pgfqpoint{1.519829in}{1.246859in}}%
\pgfpathlineto{\pgfqpoint{1.581928in}{1.292863in}}%
\pgfpathlineto{\pgfqpoint{1.644027in}{1.338655in}}%
\pgfpathlineto{\pgfqpoint{1.706126in}{1.384238in}}%
\pgfpathlineto{\pgfqpoint{1.768225in}{1.429612in}}%
\pgfpathlineto{\pgfqpoint{1.830324in}{1.474778in}}%
\pgfpathlineto{\pgfqpoint{1.892423in}{1.519737in}}%
\pgfpathlineto{\pgfqpoint{1.954522in}{1.564490in}}%
\pgfpathlineto{\pgfqpoint{2.016620in}{1.609038in}}%
\pgfpathlineto{\pgfqpoint{2.078719in}{1.653382in}}%
\pgfpathlineto{\pgfqpoint{2.140818in}{1.697523in}}%
\pgfpathlineto{\pgfqpoint{2.202917in}{1.741461in}}%
\pgfpathlineto{\pgfqpoint{2.265016in}{1.785198in}}%
\pgfpathlineto{\pgfqpoint{2.327115in}{1.828734in}}%
\pgfpathlineto{\pgfqpoint{2.389214in}{1.872072in}}%
\pgfpathlineto{\pgfqpoint{2.451313in}{1.915210in}}%
\pgfpathlineto{\pgfqpoint{2.513411in}{1.958151in}}%
\pgfpathlineto{\pgfqpoint{2.575510in}{2.000895in}}%
\pgfpathlineto{\pgfqpoint{2.637609in}{2.043443in}}%
\pgfpathlineto{\pgfqpoint{2.699708in}{2.085796in}}%
\pgfpathlineto{\pgfqpoint{2.761807in}{2.127955in}}%
\pgfpathlineto{\pgfqpoint{2.823906in}{2.169921in}}%
\pgfpathlineto{\pgfqpoint{2.886005in}{2.211695in}}%
\pgfpathlineto{\pgfqpoint{2.948104in}{2.253277in}}%
\pgfpathlineto{\pgfqpoint{3.010202in}{2.294669in}}%
\pgfpathlineto{\pgfqpoint{3.072301in}{2.335871in}}%
\pgfpathlineto{\pgfqpoint{3.134400in}{2.376884in}}%
\pgfpathlineto{\pgfqpoint{3.196499in}{2.417709in}}%
\pgfpathlineto{\pgfqpoint{3.258598in}{2.458347in}}%
\pgfpathlineto{\pgfqpoint{3.320697in}{2.498799in}}%
\pgfpathlineto{\pgfqpoint{3.382796in}{2.539066in}}%
\pgfpathlineto{\pgfqpoint{3.444895in}{2.579148in}}%
\pgfpathlineto{\pgfqpoint{3.506993in}{2.619046in}}%
\pgfpathlineto{\pgfqpoint{3.569092in}{2.658762in}}%
\pgfpathlineto{\pgfqpoint{3.631191in}{2.698295in}}%
\pgfpathlineto{\pgfqpoint{3.693290in}{2.737648in}}%
\pgfpathlineto{\pgfqpoint{3.755389in}{2.776820in}}%
\pgfpathlineto{\pgfqpoint{3.817488in}{2.815812in}}%
\pgfpathlineto{\pgfqpoint{3.879587in}{2.854626in}}%
\pgfpathlineto{\pgfqpoint{3.941686in}{2.893262in}}%
\pgfpathlineto{\pgfqpoint{4.003784in}{2.931721in}}%
\pgfpathlineto{\pgfqpoint{4.065883in}{2.970004in}}%
\pgfpathlineto{\pgfqpoint{4.127982in}{3.008111in}}%
\pgfpathlineto{\pgfqpoint{4.190081in}{3.046044in}}%
\pgfpathlineto{\pgfqpoint{4.252180in}{3.083803in}}%
\pgfpathlineto{\pgfqpoint{4.314279in}{3.121388in}}%
\pgfpathlineto{\pgfqpoint{4.376378in}{3.158802in}}%
\pgfpathlineto{\pgfqpoint{4.438477in}{3.196044in}}%
\pgfpathlineto{\pgfqpoint{4.500575in}{3.233116in}}%
\pgfpathlineto{\pgfqpoint{4.562674in}{3.270017in}}%
\pgfpathlineto{\pgfqpoint{4.624773in}{3.306750in}}%
\pgfpathlineto{\pgfqpoint{4.686872in}{3.343314in}}%
\pgfpathlineto{\pgfqpoint{4.748971in}{3.379711in}}%
\pgfpathlineto{\pgfqpoint{4.811070in}{3.415941in}}%
\pgfpathlineto{\pgfqpoint{4.873169in}{3.452004in}}%
\pgfpathlineto{\pgfqpoint{4.935268in}{3.487903in}}%
\pgfpathlineto{\pgfqpoint{4.997366in}{3.523637in}}%
\pgfpathlineto{\pgfqpoint{5.059465in}{3.559207in}}%
\pgfpathlineto{\pgfqpoint{5.121564in}{3.594615in}}%
\pgfpathlineto{\pgfqpoint{5.183663in}{3.629860in}}%
\pgfpathlineto{\pgfqpoint{5.245762in}{3.664943in}}%
\pgfpathlineto{\pgfqpoint{5.307861in}{3.699866in}}%
\pgfpathlineto{\pgfqpoint{5.369960in}{3.734629in}}%
\pgfpathlineto{\pgfqpoint{5.432059in}{3.769232in}}%
\pgfpathlineto{\pgfqpoint{5.494157in}{3.803677in}}%
\pgfpathlineto{\pgfqpoint{5.556256in}{3.837964in}}%
\pgfpathlineto{\pgfqpoint{5.618355in}{3.872094in}}%
\pgfpathlineto{\pgfqpoint{5.680454in}{3.906068in}}%
\pgfpathlineto{\pgfqpoint{5.742553in}{3.939886in}}%
\pgfpathlineto{\pgfqpoint{5.804652in}{3.973549in}}%
\pgfpathlineto{\pgfqpoint{5.866751in}{4.007057in}}%
\pgfpathlineto{\pgfqpoint{5.928850in}{4.040412in}}%
\pgfpathlineto{\pgfqpoint{5.990948in}{4.073615in}}%
\pgfpathlineto{\pgfqpoint{6.053047in}{4.106665in}}%
\pgfpathlineto{\pgfqpoint{6.115146in}{4.139563in}}%
\pgfpathlineto{\pgfqpoint{6.177245in}{4.172311in}}%
\pgfpathlineto{\pgfqpoint{6.239344in}{4.204909in}}%
\pgfpathlineto{\pgfqpoint{6.301443in}{4.237358in}}%
\pgfpathlineto{\pgfqpoint{6.363542in}{4.269657in}}%
\pgfpathlineto{\pgfqpoint{6.425641in}{4.301809in}}%
\pgfpathlineto{\pgfqpoint{6.487739in}{4.333814in}}%
\pgfpathlineto{\pgfqpoint{6.549838in}{4.365671in}}%
\pgfpathlineto{\pgfqpoint{6.611937in}{4.397383in}}%
\pgfpathlineto{\pgfqpoint{6.674036in}{4.428950in}}%
\pgfpathlineto{\pgfqpoint{6.736135in}{4.460371in}}%
\pgfpathlineto{\pgfqpoint{6.798234in}{4.491649in}}%
\pgfpathlineto{\pgfqpoint{6.860333in}{4.522784in}}%
\pgfpathlineto{\pgfqpoint{6.922432in}{4.553776in}}%
\pgfpathlineto{\pgfqpoint{6.984530in}{4.584625in}}%
\pgfpathlineto{\pgfqpoint{7.046629in}{4.615334in}}%
\pgfpathlineto{\pgfqpoint{7.108728in}{4.645902in}}%
\pgfpathlineto{\pgfqpoint{7.170827in}{4.676329in}}%
\pgfpathlineto{\pgfqpoint{7.232926in}{4.706617in}}%
\pgfpathlineto{\pgfqpoint{7.295025in}{4.736767in}}%
\pgfpathlineto{\pgfqpoint{7.357124in}{4.766778in}}%
\pgfpathlineto{\pgfqpoint{7.419223in}{4.796652in}}%
\pgfpathlineto{\pgfqpoint{7.481321in}{4.826389in}}%
\pgfusepath{stroke}%
\end{pgfscope}%
\begin{pgfscope}%
\pgfsetrectcap%
\pgfsetmiterjoin%
\pgfsetlinewidth{0.000000pt}%
\definecolor{currentstroke}{rgb}{1.000000,1.000000,1.000000}%
\pgfsetstrokecolor{currentstroke}%
\pgfsetdash{}{0pt}%
\pgfpathmoveto{\pgfqpoint{1.026143in}{0.898915in}}%
\pgfpathlineto{\pgfqpoint{1.026143in}{5.013411in}}%
\pgfusepath{}%
\end{pgfscope}%
\begin{pgfscope}%
\pgfsetrectcap%
\pgfsetmiterjoin%
\pgfsetlinewidth{0.000000pt}%
\definecolor{currentstroke}{rgb}{1.000000,1.000000,1.000000}%
\pgfsetstrokecolor{currentstroke}%
\pgfsetdash{}{0pt}%
\pgfpathmoveto{\pgfqpoint{7.788711in}{0.898915in}}%
\pgfpathlineto{\pgfqpoint{7.788711in}{5.013411in}}%
\pgfusepath{}%
\end{pgfscope}%
\begin{pgfscope}%
\pgfsetrectcap%
\pgfsetmiterjoin%
\pgfsetlinewidth{0.000000pt}%
\definecolor{currentstroke}{rgb}{1.000000,1.000000,1.000000}%
\pgfsetstrokecolor{currentstroke}%
\pgfsetdash{}{0pt}%
\pgfpathmoveto{\pgfqpoint{1.026143in}{0.898915in}}%
\pgfpathlineto{\pgfqpoint{7.788711in}{0.898915in}}%
\pgfusepath{}%
\end{pgfscope}%
\begin{pgfscope}%
\pgfsetrectcap%
\pgfsetmiterjoin%
\pgfsetlinewidth{0.000000pt}%
\definecolor{currentstroke}{rgb}{1.000000,1.000000,1.000000}%
\pgfsetstrokecolor{currentstroke}%
\pgfsetdash{}{0pt}%
\pgfpathmoveto{\pgfqpoint{1.026143in}{5.013411in}}%
\pgfpathlineto{\pgfqpoint{7.788711in}{5.013411in}}%
\pgfusepath{}%
\end{pgfscope}%
\begin{pgfscope}%
\definecolor{textcolor}{rgb}{0.150000,0.150000,0.150000}%
\pgfsetstrokecolor{textcolor}%
\pgfsetfillcolor{textcolor}%
\pgftext[x=2.969183in, y=1.597967in, left, base]{\color{textcolor}\rmfamily\fontsize{20.000000}{24.000000}\selectfont M\(\displaystyle _z\)(t) = 11.37(1\(\displaystyle -\)2\(\displaystyle exp[\frac{-t}{T_1}]\))\(\displaystyle +\)2.29}%
\end{pgfscope}%
\begin{pgfscope}%
\definecolor{textcolor}{rgb}{0.150000,0.150000,0.150000}%
\pgfsetstrokecolor{textcolor}%
\pgfsetfillcolor{textcolor}%
\pgftext[x=2.969183in, y=1.103911in, left, base]{\color{textcolor}\rmfamily\fontsize{20.000000}{24.000000}\selectfont T\(\displaystyle _1\) = 239.7\(\displaystyle \pm\)37.53 ms}%
\end{pgfscope}%
\begin{pgfscope}%
\definecolor{textcolor}{rgb}{0.150000,0.150000,0.150000}%
\pgfsetstrokecolor{textcolor}%
\pgfsetfillcolor{textcolor}%
\pgftext[x=4.407427in,y=5.096745in,,base]{\color{textcolor}\rmfamily\fontsize{24.000000}{28.800000}\selectfont T\(\displaystyle _1\) measurement of 0.005 M CuSO\(\displaystyle _4\)}%
\end{pgfscope}%
\begin{pgfscope}%
\pgfsetbuttcap%
\pgfsetroundjoin%
\definecolor{currentfill}{rgb}{0.392157,0.584314,0.929412}%
\pgfsetfillcolor{currentfill}%
\pgfsetlinewidth{0.301125pt}%
\definecolor{currentstroke}{rgb}{0.392157,0.584314,0.929412}%
\pgfsetstrokecolor{currentstroke}%
\pgfsetdash{}{0pt}%
\pgfsys@defobject{currentmarker}{\pgfqpoint{-0.048611in}{-0.048611in}}{\pgfqpoint{0.048611in}{0.048611in}}{%
\pgfpathmoveto{\pgfqpoint{0.000000in}{-0.048611in}}%
\pgfpathcurveto{\pgfqpoint{0.012892in}{-0.048611in}}{\pgfqpoint{0.025257in}{-0.043489in}}{\pgfqpoint{0.034373in}{-0.034373in}}%
\pgfpathcurveto{\pgfqpoint{0.043489in}{-0.025257in}}{\pgfqpoint{0.048611in}{-0.012892in}}{\pgfqpoint{0.048611in}{0.000000in}}%
\pgfpathcurveto{\pgfqpoint{0.048611in}{0.012892in}}{\pgfqpoint{0.043489in}{0.025257in}}{\pgfqpoint{0.034373in}{0.034373in}}%
\pgfpathcurveto{\pgfqpoint{0.025257in}{0.043489in}}{\pgfqpoint{0.012892in}{0.048611in}}{\pgfqpoint{0.000000in}{0.048611in}}%
\pgfpathcurveto{\pgfqpoint{-0.012892in}{0.048611in}}{\pgfqpoint{-0.025257in}{0.043489in}}{\pgfqpoint{-0.034373in}{0.034373in}}%
\pgfpathcurveto{\pgfqpoint{-0.043489in}{0.025257in}}{\pgfqpoint{-0.048611in}{0.012892in}}{\pgfqpoint{-0.048611in}{0.000000in}}%
\pgfpathcurveto{\pgfqpoint{-0.048611in}{-0.012892in}}{\pgfqpoint{-0.043489in}{-0.025257in}}{\pgfqpoint{-0.034373in}{-0.034373in}}%
\pgfpathcurveto{\pgfqpoint{-0.025257in}{-0.043489in}}{\pgfqpoint{-0.012892in}{-0.048611in}}{\pgfqpoint{0.000000in}{-0.048611in}}%
\pgfpathlineto{\pgfqpoint{0.000000in}{-0.048611in}}%
\pgfpathclose%
\pgfusepath{stroke,fill}%
}%
\begin{pgfscope}%
\pgfsys@transformshift{1.553921in}{4.625282in}%
\pgfsys@useobject{currentmarker}{}%
\end{pgfscope}%
\end{pgfscope}%
\begin{pgfscope}%
\definecolor{textcolor}{rgb}{0.150000,0.150000,0.150000}%
\pgfsetstrokecolor{textcolor}%
\pgfsetfillcolor{textcolor}%
\pgftext[x=2.053921in,y=4.552365in,left,base]{\color{textcolor}\rmfamily\fontsize{20.000000}{24.000000}\selectfont Data}%
\end{pgfscope}%
\begin{pgfscope}%
\pgfsetroundcap%
\pgfsetroundjoin%
\pgfsetlinewidth{1.756562pt}%
\definecolor{currentstroke}{rgb}{0.298039,0.447059,0.690196}%
\pgfsetstrokecolor{currentstroke}%
\pgfsetdash{}{0pt}%
\pgfpathmoveto{\pgfqpoint{1.276143in}{4.241873in}}%
\pgfpathlineto{\pgfqpoint{1.553921in}{4.241873in}}%
\pgfpathlineto{\pgfqpoint{1.831699in}{4.241873in}}%
\pgfusepath{stroke}%
\end{pgfscope}%
\begin{pgfscope}%
\definecolor{textcolor}{rgb}{0.150000,0.150000,0.150000}%
\pgfsetstrokecolor{textcolor}%
\pgfsetfillcolor{textcolor}%
\pgftext[x=2.053921in,y=4.144651in,left,base]{\color{textcolor}\rmfamily\fontsize{20.000000}{24.000000}\selectfont Fitted Curve}%
\end{pgfscope}%
\end{pgfpicture}%
\makeatother%
\endgroup%
}
    \caption{T1 measurement of 0.005 M  CuSO$_{4} \boldsymbol{\cdot} $H$_2$O using the two-pulse method }
    \label{fig:0d005_t1}
  \end{figure}

  \begin{figure}[h]
    \resizebox{0.45\textwidth}{!}{%% Creator: Matplotlib, PGF backend
%%
%% To include the figure in your LaTeX document, write
%%   \input{<filename>.pgf}
%%
%% Make sure the required packages are loaded in your preamble
%%   \usepackage{pgf}
%%
%% Also ensure that all the required font packages are loaded; for instance,
%% the lmodern package is sometimes necessary when using math font.
%%   \usepackage{lmodern}
%%
%% Figures using additional raster images can only be included by \input if
%% they are in the same directory as the main LaTeX file. For loading figures
%% from other directories you can use the `import` package
%%   \usepackage{import}
%%
%% and then include the figures with
%%   \import{<path to file>}{<filename>.pgf}
%%
%% Matplotlib used the following preamble
%%   \usepackage{fontspec}
%%   \setmainfont{DejaVuSerif.ttf}[Path=\detokenize{/usr/share/matplotlib/mpl-data/fonts/ttf/}]
%%   \setsansfont{DejaVuSans.ttf}[Path=\detokenize{/usr/share/matplotlib/mpl-data/fonts/ttf/}]
%%   \setmonofont{DejaVuSansMono.ttf}[Path=\detokenize{/usr/share/matplotlib/mpl-data/fonts/ttf/}]
%%
\begingroup%
\makeatletter%
\begin{pgfpicture}%
\pgfpathrectangle{\pgfpointorigin}{\pgfqpoint{8.000000in}{5.500000in}}%
\pgfusepath{use as bounding box, clip}%
\begin{pgfscope}%
\pgfsetbuttcap%
\pgfsetmiterjoin%
\definecolor{currentfill}{rgb}{1.000000,1.000000,1.000000}%
\pgfsetfillcolor{currentfill}%
\pgfsetlinewidth{0.000000pt}%
\definecolor{currentstroke}{rgb}{1.000000,1.000000,1.000000}%
\pgfsetstrokecolor{currentstroke}%
\pgfsetdash{}{0pt}%
\pgfpathmoveto{\pgfqpoint{0.000000in}{0.000000in}}%
\pgfpathlineto{\pgfqpoint{8.000000in}{0.000000in}}%
\pgfpathlineto{\pgfqpoint{8.000000in}{5.500000in}}%
\pgfpathlineto{\pgfqpoint{0.000000in}{5.500000in}}%
\pgfpathlineto{\pgfqpoint{0.000000in}{0.000000in}}%
\pgfpathclose%
\pgfusepath{fill}%
\end{pgfscope}%
\begin{pgfscope}%
\pgfsetbuttcap%
\pgfsetmiterjoin%
\definecolor{currentfill}{rgb}{0.917647,0.917647,0.949020}%
\pgfsetfillcolor{currentfill}%
\pgfsetlinewidth{0.000000pt}%
\definecolor{currentstroke}{rgb}{0.000000,0.000000,0.000000}%
\pgfsetstrokecolor{currentstroke}%
\pgfsetstrokeopacity{0.000000}%
\pgfsetdash{}{0pt}%
\pgfpathmoveto{\pgfqpoint{1.467902in}{0.898915in}}%
\pgfpathlineto{\pgfqpoint{7.850000in}{0.898915in}}%
\pgfpathlineto{\pgfqpoint{7.850000in}{5.013411in}}%
\pgfpathlineto{\pgfqpoint{1.467902in}{5.013411in}}%
\pgfpathlineto{\pgfqpoint{1.467902in}{0.898915in}}%
\pgfpathclose%
\pgfusepath{fill}%
\end{pgfscope}%
\begin{pgfscope}%
\pgfpathrectangle{\pgfqpoint{1.467902in}{0.898915in}}{\pgfqpoint{6.382098in}{4.114497in}}%
\pgfusepath{clip}%
\pgfsetroundcap%
\pgfsetroundjoin%
\pgfsetlinewidth{1.003750pt}%
\definecolor{currentstroke}{rgb}{1.000000,1.000000,1.000000}%
\pgfsetstrokecolor{currentstroke}%
\pgfsetdash{}{0pt}%
\pgfpathmoveto{\pgfqpoint{1.704769in}{0.898915in}}%
\pgfpathlineto{\pgfqpoint{1.704769in}{5.013411in}}%
\pgfusepath{stroke}%
\end{pgfscope}%
\begin{pgfscope}%
\definecolor{textcolor}{rgb}{0.150000,0.150000,0.150000}%
\pgfsetstrokecolor{textcolor}%
\pgfsetfillcolor{textcolor}%
\pgftext[x=1.704769in,y=0.801692in,,top]{\color{textcolor}\rmfamily\fontsize{20.000000}{24.000000}\selectfont 0}%
\end{pgfscope}%
\begin{pgfscope}%
\pgfpathrectangle{\pgfqpoint{1.467902in}{0.898915in}}{\pgfqpoint{6.382098in}{4.114497in}}%
\pgfusepath{clip}%
\pgfsetroundcap%
\pgfsetroundjoin%
\pgfsetlinewidth{1.003750pt}%
\definecolor{currentstroke}{rgb}{1.000000,1.000000,1.000000}%
\pgfsetstrokecolor{currentstroke}%
\pgfsetdash{}{0pt}%
\pgfpathmoveto{\pgfqpoint{2.769339in}{0.898915in}}%
\pgfpathlineto{\pgfqpoint{2.769339in}{5.013411in}}%
\pgfusepath{stroke}%
\end{pgfscope}%
\begin{pgfscope}%
\definecolor{textcolor}{rgb}{0.150000,0.150000,0.150000}%
\pgfsetstrokecolor{textcolor}%
\pgfsetfillcolor{textcolor}%
\pgftext[x=2.769339in,y=0.801692in,,top]{\color{textcolor}\rmfamily\fontsize{20.000000}{24.000000}\selectfont 20}%
\end{pgfscope}%
\begin{pgfscope}%
\pgfpathrectangle{\pgfqpoint{1.467902in}{0.898915in}}{\pgfqpoint{6.382098in}{4.114497in}}%
\pgfusepath{clip}%
\pgfsetroundcap%
\pgfsetroundjoin%
\pgfsetlinewidth{1.003750pt}%
\definecolor{currentstroke}{rgb}{1.000000,1.000000,1.000000}%
\pgfsetstrokecolor{currentstroke}%
\pgfsetdash{}{0pt}%
\pgfpathmoveto{\pgfqpoint{3.833909in}{0.898915in}}%
\pgfpathlineto{\pgfqpoint{3.833909in}{5.013411in}}%
\pgfusepath{stroke}%
\end{pgfscope}%
\begin{pgfscope}%
\definecolor{textcolor}{rgb}{0.150000,0.150000,0.150000}%
\pgfsetstrokecolor{textcolor}%
\pgfsetfillcolor{textcolor}%
\pgftext[x=3.833909in,y=0.801692in,,top]{\color{textcolor}\rmfamily\fontsize{20.000000}{24.000000}\selectfont 40}%
\end{pgfscope}%
\begin{pgfscope}%
\pgfpathrectangle{\pgfqpoint{1.467902in}{0.898915in}}{\pgfqpoint{6.382098in}{4.114497in}}%
\pgfusepath{clip}%
\pgfsetroundcap%
\pgfsetroundjoin%
\pgfsetlinewidth{1.003750pt}%
\definecolor{currentstroke}{rgb}{1.000000,1.000000,1.000000}%
\pgfsetstrokecolor{currentstroke}%
\pgfsetdash{}{0pt}%
\pgfpathmoveto{\pgfqpoint{4.898479in}{0.898915in}}%
\pgfpathlineto{\pgfqpoint{4.898479in}{5.013411in}}%
\pgfusepath{stroke}%
\end{pgfscope}%
\begin{pgfscope}%
\definecolor{textcolor}{rgb}{0.150000,0.150000,0.150000}%
\pgfsetstrokecolor{textcolor}%
\pgfsetfillcolor{textcolor}%
\pgftext[x=4.898479in,y=0.801692in,,top]{\color{textcolor}\rmfamily\fontsize{20.000000}{24.000000}\selectfont 60}%
\end{pgfscope}%
\begin{pgfscope}%
\pgfpathrectangle{\pgfqpoint{1.467902in}{0.898915in}}{\pgfqpoint{6.382098in}{4.114497in}}%
\pgfusepath{clip}%
\pgfsetroundcap%
\pgfsetroundjoin%
\pgfsetlinewidth{1.003750pt}%
\definecolor{currentstroke}{rgb}{1.000000,1.000000,1.000000}%
\pgfsetstrokecolor{currentstroke}%
\pgfsetdash{}{0pt}%
\pgfpathmoveto{\pgfqpoint{5.963049in}{0.898915in}}%
\pgfpathlineto{\pgfqpoint{5.963049in}{5.013411in}}%
\pgfusepath{stroke}%
\end{pgfscope}%
\begin{pgfscope}%
\definecolor{textcolor}{rgb}{0.150000,0.150000,0.150000}%
\pgfsetstrokecolor{textcolor}%
\pgfsetfillcolor{textcolor}%
\pgftext[x=5.963049in,y=0.801692in,,top]{\color{textcolor}\rmfamily\fontsize{20.000000}{24.000000}\selectfont 80}%
\end{pgfscope}%
\begin{pgfscope}%
\pgfpathrectangle{\pgfqpoint{1.467902in}{0.898915in}}{\pgfqpoint{6.382098in}{4.114497in}}%
\pgfusepath{clip}%
\pgfsetroundcap%
\pgfsetroundjoin%
\pgfsetlinewidth{1.003750pt}%
\definecolor{currentstroke}{rgb}{1.000000,1.000000,1.000000}%
\pgfsetstrokecolor{currentstroke}%
\pgfsetdash{}{0pt}%
\pgfpathmoveto{\pgfqpoint{7.027620in}{0.898915in}}%
\pgfpathlineto{\pgfqpoint{7.027620in}{5.013411in}}%
\pgfusepath{stroke}%
\end{pgfscope}%
\begin{pgfscope}%
\definecolor{textcolor}{rgb}{0.150000,0.150000,0.150000}%
\pgfsetstrokecolor{textcolor}%
\pgfsetfillcolor{textcolor}%
\pgftext[x=7.027620in,y=0.801692in,,top]{\color{textcolor}\rmfamily\fontsize{20.000000}{24.000000}\selectfont 100}%
\end{pgfscope}%
\begin{pgfscope}%
\definecolor{textcolor}{rgb}{0.150000,0.150000,0.150000}%
\pgfsetstrokecolor{textcolor}%
\pgfsetfillcolor{textcolor}%
\pgftext[x=4.658951in,y=0.477311in,,top]{\color{textcolor}\rmfamily\fontsize{24.000000}{28.800000}\selectfont Delay Time (ms)}%
\end{pgfscope}%
\begin{pgfscope}%
\pgfpathrectangle{\pgfqpoint{1.467902in}{0.898915in}}{\pgfqpoint{6.382098in}{4.114497in}}%
\pgfusepath{clip}%
\pgfsetroundcap%
\pgfsetroundjoin%
\pgfsetlinewidth{1.003750pt}%
\definecolor{currentstroke}{rgb}{1.000000,1.000000,1.000000}%
\pgfsetstrokecolor{currentstroke}%
\pgfsetdash{}{0pt}%
\pgfpathmoveto{\pgfqpoint{1.467902in}{0.909745in}}%
\pgfpathlineto{\pgfqpoint{7.850000in}{0.909745in}}%
\pgfusepath{stroke}%
\end{pgfscope}%
\begin{pgfscope}%
\definecolor{textcolor}{rgb}{0.150000,0.150000,0.150000}%
\pgfsetstrokecolor{textcolor}%
\pgfsetfillcolor{textcolor}%
\pgftext[x=0.528147in, y=0.804222in, left, base]{\color{textcolor}\rmfamily\fontsize{20.000000}{24.000000}\selectfont \ensuremath{-}10.0}%
\end{pgfscope}%
\begin{pgfscope}%
\pgfpathrectangle{\pgfqpoint{1.467902in}{0.898915in}}{\pgfqpoint{6.382098in}{4.114497in}}%
\pgfusepath{clip}%
\pgfsetroundcap%
\pgfsetroundjoin%
\pgfsetlinewidth{1.003750pt}%
\definecolor{currentstroke}{rgb}{1.000000,1.000000,1.000000}%
\pgfsetstrokecolor{currentstroke}%
\pgfsetdash{}{0pt}%
\pgfpathmoveto{\pgfqpoint{1.467902in}{1.590201in}}%
\pgfpathlineto{\pgfqpoint{7.850000in}{1.590201in}}%
\pgfusepath{stroke}%
\end{pgfscope}%
\begin{pgfscope}%
\definecolor{textcolor}{rgb}{0.150000,0.150000,0.150000}%
\pgfsetstrokecolor{textcolor}%
\pgfsetfillcolor{textcolor}%
\pgftext[x=0.704877in, y=1.484678in, left, base]{\color{textcolor}\rmfamily\fontsize{20.000000}{24.000000}\selectfont \ensuremath{-}7.5}%
\end{pgfscope}%
\begin{pgfscope}%
\pgfpathrectangle{\pgfqpoint{1.467902in}{0.898915in}}{\pgfqpoint{6.382098in}{4.114497in}}%
\pgfusepath{clip}%
\pgfsetroundcap%
\pgfsetroundjoin%
\pgfsetlinewidth{1.003750pt}%
\definecolor{currentstroke}{rgb}{1.000000,1.000000,1.000000}%
\pgfsetstrokecolor{currentstroke}%
\pgfsetdash{}{0pt}%
\pgfpathmoveto{\pgfqpoint{1.467902in}{2.270658in}}%
\pgfpathlineto{\pgfqpoint{7.850000in}{2.270658in}}%
\pgfusepath{stroke}%
\end{pgfscope}%
\begin{pgfscope}%
\definecolor{textcolor}{rgb}{0.150000,0.150000,0.150000}%
\pgfsetstrokecolor{textcolor}%
\pgfsetfillcolor{textcolor}%
\pgftext[x=0.704877in, y=2.165135in, left, base]{\color{textcolor}\rmfamily\fontsize{20.000000}{24.000000}\selectfont \ensuremath{-}5.0}%
\end{pgfscope}%
\begin{pgfscope}%
\pgfpathrectangle{\pgfqpoint{1.467902in}{0.898915in}}{\pgfqpoint{6.382098in}{4.114497in}}%
\pgfusepath{clip}%
\pgfsetroundcap%
\pgfsetroundjoin%
\pgfsetlinewidth{1.003750pt}%
\definecolor{currentstroke}{rgb}{1.000000,1.000000,1.000000}%
\pgfsetstrokecolor{currentstroke}%
\pgfsetdash{}{0pt}%
\pgfpathmoveto{\pgfqpoint{1.467902in}{2.951114in}}%
\pgfpathlineto{\pgfqpoint{7.850000in}{2.951114in}}%
\pgfusepath{stroke}%
\end{pgfscope}%
\begin{pgfscope}%
\definecolor{textcolor}{rgb}{0.150000,0.150000,0.150000}%
\pgfsetstrokecolor{textcolor}%
\pgfsetfillcolor{textcolor}%
\pgftext[x=0.704877in, y=2.845591in, left, base]{\color{textcolor}\rmfamily\fontsize{20.000000}{24.000000}\selectfont \ensuremath{-}2.5}%
\end{pgfscope}%
\begin{pgfscope}%
\pgfpathrectangle{\pgfqpoint{1.467902in}{0.898915in}}{\pgfqpoint{6.382098in}{4.114497in}}%
\pgfusepath{clip}%
\pgfsetroundcap%
\pgfsetroundjoin%
\pgfsetlinewidth{1.003750pt}%
\definecolor{currentstroke}{rgb}{1.000000,1.000000,1.000000}%
\pgfsetstrokecolor{currentstroke}%
\pgfsetdash{}{0pt}%
\pgfpathmoveto{\pgfqpoint{1.467902in}{3.631570in}}%
\pgfpathlineto{\pgfqpoint{7.850000in}{3.631570in}}%
\pgfusepath{stroke}%
\end{pgfscope}%
\begin{pgfscope}%
\definecolor{textcolor}{rgb}{0.150000,0.150000,0.150000}%
\pgfsetstrokecolor{textcolor}%
\pgfsetfillcolor{textcolor}%
\pgftext[x=0.928921in, y=3.526047in, left, base]{\color{textcolor}\rmfamily\fontsize{20.000000}{24.000000}\selectfont 0.0}%
\end{pgfscope}%
\begin{pgfscope}%
\pgfpathrectangle{\pgfqpoint{1.467902in}{0.898915in}}{\pgfqpoint{6.382098in}{4.114497in}}%
\pgfusepath{clip}%
\pgfsetroundcap%
\pgfsetroundjoin%
\pgfsetlinewidth{1.003750pt}%
\definecolor{currentstroke}{rgb}{1.000000,1.000000,1.000000}%
\pgfsetstrokecolor{currentstroke}%
\pgfsetdash{}{0pt}%
\pgfpathmoveto{\pgfqpoint{1.467902in}{4.312026in}}%
\pgfpathlineto{\pgfqpoint{7.850000in}{4.312026in}}%
\pgfusepath{stroke}%
\end{pgfscope}%
\begin{pgfscope}%
\definecolor{textcolor}{rgb}{0.150000,0.150000,0.150000}%
\pgfsetstrokecolor{textcolor}%
\pgfsetfillcolor{textcolor}%
\pgftext[x=0.928921in, y=4.206503in, left, base]{\color{textcolor}\rmfamily\fontsize{20.000000}{24.000000}\selectfont 2.5}%
\end{pgfscope}%
\begin{pgfscope}%
\pgfpathrectangle{\pgfqpoint{1.467902in}{0.898915in}}{\pgfqpoint{6.382098in}{4.114497in}}%
\pgfusepath{clip}%
\pgfsetroundcap%
\pgfsetroundjoin%
\pgfsetlinewidth{1.003750pt}%
\definecolor{currentstroke}{rgb}{1.000000,1.000000,1.000000}%
\pgfsetstrokecolor{currentstroke}%
\pgfsetdash{}{0pt}%
\pgfpathmoveto{\pgfqpoint{1.467902in}{4.992482in}}%
\pgfpathlineto{\pgfqpoint{7.850000in}{4.992482in}}%
\pgfusepath{stroke}%
\end{pgfscope}%
\begin{pgfscope}%
\definecolor{textcolor}{rgb}{0.150000,0.150000,0.150000}%
\pgfsetstrokecolor{textcolor}%
\pgfsetfillcolor{textcolor}%
\pgftext[x=0.928921in, y=4.886959in, left, base]{\color{textcolor}\rmfamily\fontsize{20.000000}{24.000000}\selectfont 5.0}%
\end{pgfscope}%
\begin{pgfscope}%
\definecolor{textcolor}{rgb}{0.150000,0.150000,0.150000}%
\pgfsetstrokecolor{textcolor}%
\pgfsetfillcolor{textcolor}%
\pgftext[x=0.472591in,y=2.956163in,,bottom,rotate=90.000000]{\color{textcolor}\rmfamily\fontsize{24.000000}{28.800000}\selectfont Pulse Amplitude (V)}%
\end{pgfscope}%
\begin{pgfscope}%
\pgfpathrectangle{\pgfqpoint{1.467902in}{0.898915in}}{\pgfqpoint{6.382098in}{4.114497in}}%
\pgfusepath{clip}%
\pgfsetbuttcap%
\pgfsetroundjoin%
\definecolor{currentfill}{rgb}{0.392157,0.584314,0.929412}%
\pgfsetfillcolor{currentfill}%
\pgfsetlinewidth{0.301125pt}%
\definecolor{currentstroke}{rgb}{0.392157,0.584314,0.929412}%
\pgfsetstrokecolor{currentstroke}%
\pgfsetdash{}{0pt}%
\pgfsys@defobject{currentmarker}{\pgfqpoint{-0.048611in}{-0.048611in}}{\pgfqpoint{0.048611in}{0.048611in}}{%
\pgfpathmoveto{\pgfqpoint{0.000000in}{-0.048611in}}%
\pgfpathcurveto{\pgfqpoint{0.012892in}{-0.048611in}}{\pgfqpoint{0.025257in}{-0.043489in}}{\pgfqpoint{0.034373in}{-0.034373in}}%
\pgfpathcurveto{\pgfqpoint{0.043489in}{-0.025257in}}{\pgfqpoint{0.048611in}{-0.012892in}}{\pgfqpoint{0.048611in}{0.000000in}}%
\pgfpathcurveto{\pgfqpoint{0.048611in}{0.012892in}}{\pgfqpoint{0.043489in}{0.025257in}}{\pgfqpoint{0.034373in}{0.034373in}}%
\pgfpathcurveto{\pgfqpoint{0.025257in}{0.043489in}}{\pgfqpoint{0.012892in}{0.048611in}}{\pgfqpoint{0.000000in}{0.048611in}}%
\pgfpathcurveto{\pgfqpoint{-0.012892in}{0.048611in}}{\pgfqpoint{-0.025257in}{0.043489in}}{\pgfqpoint{-0.034373in}{0.034373in}}%
\pgfpathcurveto{\pgfqpoint{-0.043489in}{0.025257in}}{\pgfqpoint{-0.048611in}{0.012892in}}{\pgfqpoint{-0.048611in}{0.000000in}}%
\pgfpathcurveto{\pgfqpoint{-0.048611in}{-0.012892in}}{\pgfqpoint{-0.043489in}{-0.025257in}}{\pgfqpoint{-0.034373in}{-0.034373in}}%
\pgfpathcurveto{\pgfqpoint{-0.025257in}{-0.043489in}}{\pgfqpoint{-0.012892in}{-0.048611in}}{\pgfqpoint{0.000000in}{-0.048611in}}%
\pgfpathlineto{\pgfqpoint{0.000000in}{-0.048611in}}%
\pgfpathclose%
\pgfusepath{stroke,fill}%
}%
\begin{pgfscope}%
\pgfsys@transformshift{1.757998in}{1.127491in}%
\pgfsys@useobject{currentmarker}{}%
\end{pgfscope}%
\begin{pgfscope}%
\pgfsys@transformshift{2.237054in}{1.519434in}%
\pgfsys@useobject{currentmarker}{}%
\end{pgfscope}%
\begin{pgfscope}%
\pgfsys@transformshift{2.769339in}{1.965813in}%
\pgfsys@useobject{currentmarker}{}%
\end{pgfscope}%
\begin{pgfscope}%
\pgfsys@transformshift{3.301624in}{2.335981in}%
\pgfsys@useobject{currentmarker}{}%
\end{pgfscope}%
\begin{pgfscope}%
\pgfsys@transformshift{3.833909in}{2.684375in}%
\pgfsys@useobject{currentmarker}{}%
\end{pgfscope}%
\begin{pgfscope}%
\pgfsys@transformshift{4.366194in}{2.989219in}%
\pgfsys@useobject{currentmarker}{}%
\end{pgfscope}%
\begin{pgfscope}%
\pgfsys@transformshift{4.898479in}{3.424711in}%
\pgfsys@useobject{currentmarker}{}%
\end{pgfscope}%
\begin{pgfscope}%
\pgfsys@transformshift{5.430764in}{3.925527in}%
\pgfsys@useobject{currentmarker}{}%
\end{pgfscope}%
\begin{pgfscope}%
\pgfsys@transformshift{5.963049in}{4.197710in}%
\pgfsys@useobject{currentmarker}{}%
\end{pgfscope}%
\begin{pgfscope}%
\pgfsys@transformshift{6.495335in}{4.382794in}%
\pgfsys@useobject{currentmarker}{}%
\end{pgfscope}%
\begin{pgfscope}%
\pgfsys@transformshift{7.027620in}{4.556990in}%
\pgfsys@useobject{currentmarker}{}%
\end{pgfscope}%
\begin{pgfscope}%
\pgfsys@transformshift{7.559905in}{4.720300in}%
\pgfsys@useobject{currentmarker}{}%
\end{pgfscope}%
\end{pgfscope}%
\begin{pgfscope}%
\pgfpathrectangle{\pgfqpoint{1.467902in}{0.898915in}}{\pgfqpoint{6.382098in}{4.114497in}}%
\pgfusepath{clip}%
\pgfsetroundcap%
\pgfsetroundjoin%
\pgfsetlinewidth{1.756562pt}%
\definecolor{currentstroke}{rgb}{0.298039,0.447059,0.690196}%
\pgfsetstrokecolor{currentstroke}%
\pgfsetdash{}{0pt}%
\pgfpathmoveto{\pgfqpoint{1.757998in}{1.085937in}}%
\pgfpathlineto{\pgfqpoint{1.816603in}{1.139570in}}%
\pgfpathlineto{\pgfqpoint{1.875208in}{1.192796in}}%
\pgfpathlineto{\pgfqpoint{1.933813in}{1.245616in}}%
\pgfpathlineto{\pgfqpoint{1.992418in}{1.298035in}}%
\pgfpathlineto{\pgfqpoint{2.051023in}{1.350055in}}%
\pgfpathlineto{\pgfqpoint{2.109628in}{1.401680in}}%
\pgfpathlineto{\pgfqpoint{2.168233in}{1.452912in}}%
\pgfpathlineto{\pgfqpoint{2.226839in}{1.503755in}}%
\pgfpathlineto{\pgfqpoint{2.285444in}{1.554211in}}%
\pgfpathlineto{\pgfqpoint{2.344049in}{1.604284in}}%
\pgfpathlineto{\pgfqpoint{2.402654in}{1.653976in}}%
\pgfpathlineto{\pgfqpoint{2.461259in}{1.703290in}}%
\pgfpathlineto{\pgfqpoint{2.519864in}{1.752229in}}%
\pgfpathlineto{\pgfqpoint{2.578469in}{1.800795in}}%
\pgfpathlineto{\pgfqpoint{2.637074in}{1.848993in}}%
\pgfpathlineto{\pgfqpoint{2.695680in}{1.896824in}}%
\pgfpathlineto{\pgfqpoint{2.754285in}{1.944291in}}%
\pgfpathlineto{\pgfqpoint{2.812890in}{1.991398in}}%
\pgfpathlineto{\pgfqpoint{2.871495in}{2.038146in}}%
\pgfpathlineto{\pgfqpoint{2.930100in}{2.084539in}}%
\pgfpathlineto{\pgfqpoint{2.988705in}{2.130579in}}%
\pgfpathlineto{\pgfqpoint{3.047310in}{2.176269in}}%
\pgfpathlineto{\pgfqpoint{3.105915in}{2.221612in}}%
\pgfpathlineto{\pgfqpoint{3.164521in}{2.266610in}}%
\pgfpathlineto{\pgfqpoint{3.223126in}{2.311265in}}%
\pgfpathlineto{\pgfqpoint{3.281731in}{2.355582in}}%
\pgfpathlineto{\pgfqpoint{3.340336in}{2.399561in}}%
\pgfpathlineto{\pgfqpoint{3.398941in}{2.443206in}}%
\pgfpathlineto{\pgfqpoint{3.457546in}{2.486518in}}%
\pgfpathlineto{\pgfqpoint{3.516151in}{2.529502in}}%
\pgfpathlineto{\pgfqpoint{3.574756in}{2.572159in}}%
\pgfpathlineto{\pgfqpoint{3.633361in}{2.614491in}}%
\pgfpathlineto{\pgfqpoint{3.691967in}{2.656502in}}%
\pgfpathlineto{\pgfqpoint{3.750572in}{2.698193in}}%
\pgfpathlineto{\pgfqpoint{3.809177in}{2.739567in}}%
\pgfpathlineto{\pgfqpoint{3.867782in}{2.780627in}}%
\pgfpathlineto{\pgfqpoint{3.926387in}{2.821374in}}%
\pgfpathlineto{\pgfqpoint{3.984992in}{2.861811in}}%
\pgfpathlineto{\pgfqpoint{4.043597in}{2.901941in}}%
\pgfpathlineto{\pgfqpoint{4.102202in}{2.941766in}}%
\pgfpathlineto{\pgfqpoint{4.160808in}{2.981288in}}%
\pgfpathlineto{\pgfqpoint{4.219413in}{3.020510in}}%
\pgfpathlineto{\pgfqpoint{4.278018in}{3.059433in}}%
\pgfpathlineto{\pgfqpoint{4.336623in}{3.098061in}}%
\pgfpathlineto{\pgfqpoint{4.395228in}{3.136394in}}%
\pgfpathlineto{\pgfqpoint{4.453833in}{3.174437in}}%
\pgfpathlineto{\pgfqpoint{4.512438in}{3.212190in}}%
\pgfpathlineto{\pgfqpoint{4.571043in}{3.249655in}}%
\pgfpathlineto{\pgfqpoint{4.629649in}{3.286836in}}%
\pgfpathlineto{\pgfqpoint{4.688254in}{3.323735in}}%
\pgfpathlineto{\pgfqpoint{4.746859in}{3.360352in}}%
\pgfpathlineto{\pgfqpoint{4.805464in}{3.396692in}}%
\pgfpathlineto{\pgfqpoint{4.864069in}{3.432755in}}%
\pgfpathlineto{\pgfqpoint{4.922674in}{3.468544in}}%
\pgfpathlineto{\pgfqpoint{4.981279in}{3.504060in}}%
\pgfpathlineto{\pgfqpoint{5.039884in}{3.539307in}}%
\pgfpathlineto{\pgfqpoint{5.098490in}{3.574285in}}%
\pgfpathlineto{\pgfqpoint{5.157095in}{3.608998in}}%
\pgfpathlineto{\pgfqpoint{5.215700in}{3.643447in}}%
\pgfpathlineto{\pgfqpoint{5.274305in}{3.677634in}}%
\pgfpathlineto{\pgfqpoint{5.332910in}{3.711560in}}%
\pgfpathlineto{\pgfqpoint{5.391515in}{3.745229in}}%
\pgfpathlineto{\pgfqpoint{5.450120in}{3.778642in}}%
\pgfpathlineto{\pgfqpoint{5.508725in}{3.811801in}}%
\pgfpathlineto{\pgfqpoint{5.567331in}{3.844708in}}%
\pgfpathlineto{\pgfqpoint{5.625936in}{3.877364in}}%
\pgfpathlineto{\pgfqpoint{5.684541in}{3.909772in}}%
\pgfpathlineto{\pgfqpoint{5.743146in}{3.941934in}}%
\pgfpathlineto{\pgfqpoint{5.801751in}{3.973851in}}%
\pgfpathlineto{\pgfqpoint{5.860356in}{4.005526in}}%
\pgfpathlineto{\pgfqpoint{5.918961in}{4.036959in}}%
\pgfpathlineto{\pgfqpoint{5.977566in}{4.068154in}}%
\pgfpathlineto{\pgfqpoint{6.036171in}{4.099112in}}%
\pgfpathlineto{\pgfqpoint{6.094777in}{4.129834in}}%
\pgfpathlineto{\pgfqpoint{6.153382in}{4.160322in}}%
\pgfpathlineto{\pgfqpoint{6.211987in}{4.190579in}}%
\pgfpathlineto{\pgfqpoint{6.270592in}{4.220605in}}%
\pgfpathlineto{\pgfqpoint{6.329197in}{4.250404in}}%
\pgfpathlineto{\pgfqpoint{6.387802in}{4.279975in}}%
\pgfpathlineto{\pgfqpoint{6.446407in}{4.309322in}}%
\pgfpathlineto{\pgfqpoint{6.505012in}{4.338446in}}%
\pgfpathlineto{\pgfqpoint{6.563618in}{4.367348in}}%
\pgfpathlineto{\pgfqpoint{6.622223in}{4.396031in}}%
\pgfpathlineto{\pgfqpoint{6.680828in}{4.424495in}}%
\pgfpathlineto{\pgfqpoint{6.739433in}{4.452743in}}%
\pgfpathlineto{\pgfqpoint{6.798038in}{4.480776in}}%
\pgfpathlineto{\pgfqpoint{6.856643in}{4.508596in}}%
\pgfpathlineto{\pgfqpoint{6.915248in}{4.536205in}}%
\pgfpathlineto{\pgfqpoint{6.973853in}{4.563603in}}%
\pgfpathlineto{\pgfqpoint{7.032459in}{4.590794in}}%
\pgfpathlineto{\pgfqpoint{7.091064in}{4.617777in}}%
\pgfpathlineto{\pgfqpoint{7.149669in}{4.644556in}}%
\pgfpathlineto{\pgfqpoint{7.208274in}{4.671130in}}%
\pgfpathlineto{\pgfqpoint{7.266879in}{4.697503in}}%
\pgfpathlineto{\pgfqpoint{7.325484in}{4.723675in}}%
\pgfpathlineto{\pgfqpoint{7.384089in}{4.749648in}}%
\pgfpathlineto{\pgfqpoint{7.442694in}{4.775424in}}%
\pgfpathlineto{\pgfqpoint{7.501300in}{4.801004in}}%
\pgfpathlineto{\pgfqpoint{7.559905in}{4.826389in}}%
\pgfusepath{stroke}%
\end{pgfscope}%
\begin{pgfscope}%
\pgfsetrectcap%
\pgfsetmiterjoin%
\pgfsetlinewidth{0.000000pt}%
\definecolor{currentstroke}{rgb}{1.000000,1.000000,1.000000}%
\pgfsetstrokecolor{currentstroke}%
\pgfsetdash{}{0pt}%
\pgfpathmoveto{\pgfqpoint{1.467902in}{0.898915in}}%
\pgfpathlineto{\pgfqpoint{1.467902in}{5.013411in}}%
\pgfusepath{}%
\end{pgfscope}%
\begin{pgfscope}%
\pgfsetrectcap%
\pgfsetmiterjoin%
\pgfsetlinewidth{0.000000pt}%
\definecolor{currentstroke}{rgb}{1.000000,1.000000,1.000000}%
\pgfsetstrokecolor{currentstroke}%
\pgfsetdash{}{0pt}%
\pgfpathmoveto{\pgfqpoint{7.850000in}{0.898915in}}%
\pgfpathlineto{\pgfqpoint{7.850000in}{5.013411in}}%
\pgfusepath{}%
\end{pgfscope}%
\begin{pgfscope}%
\pgfsetrectcap%
\pgfsetmiterjoin%
\pgfsetlinewidth{0.000000pt}%
\definecolor{currentstroke}{rgb}{1.000000,1.000000,1.000000}%
\pgfsetstrokecolor{currentstroke}%
\pgfsetdash{}{0pt}%
\pgfpathmoveto{\pgfqpoint{1.467902in}{0.898915in}}%
\pgfpathlineto{\pgfqpoint{7.850000in}{0.898915in}}%
\pgfusepath{}%
\end{pgfscope}%
\begin{pgfscope}%
\pgfsetrectcap%
\pgfsetmiterjoin%
\pgfsetlinewidth{0.000000pt}%
\definecolor{currentstroke}{rgb}{1.000000,1.000000,1.000000}%
\pgfsetstrokecolor{currentstroke}%
\pgfsetdash{}{0pt}%
\pgfpathmoveto{\pgfqpoint{1.467902in}{5.013411in}}%
\pgfpathlineto{\pgfqpoint{7.850000in}{5.013411in}}%
\pgfusepath{}%
\end{pgfscope}%
\begin{pgfscope}%
\definecolor{textcolor}{rgb}{0.150000,0.150000,0.150000}%
\pgfsetstrokecolor{textcolor}%
\pgfsetfillcolor{textcolor}%
\pgftext[x=3.035482in, y=1.675983in, left, base]{\color{textcolor}\rmfamily\fontsize{20.000000}{24.000000}\selectfont M\(\displaystyle _z\)(t) = 13.05(1\(\displaystyle -\)2\(\displaystyle exp[\frac{-t}{T_1}]\))\(\displaystyle +\)3.51}%
\end{pgfscope}%
\begin{pgfscope}%
\definecolor{textcolor}{rgb}{0.150000,0.150000,0.150000}%
\pgfsetstrokecolor{textcolor}%
\pgfsetfillcolor{textcolor}%
\pgftext[x=3.035482in, y=1.181928in, left, base]{\color{textcolor}\rmfamily\fontsize{20.000000}{24.000000}\selectfont T\(\displaystyle _1\) = 144.2\(\displaystyle \pm\)33.18 ms}%
\end{pgfscope}%
\begin{pgfscope}%
\definecolor{textcolor}{rgb}{0.150000,0.150000,0.150000}%
\pgfsetstrokecolor{textcolor}%
\pgfsetfillcolor{textcolor}%
\pgftext[x=4.658951in,y=5.096745in,,base]{\color{textcolor}\rmfamily\fontsize{24.000000}{28.800000}\selectfont T\(\displaystyle _1\) measurement of 0.01 M CuSO\(\displaystyle _4\)}%
\end{pgfscope}%
\begin{pgfscope}%
\pgfsetbuttcap%
\pgfsetroundjoin%
\definecolor{currentfill}{rgb}{0.392157,0.584314,0.929412}%
\pgfsetfillcolor{currentfill}%
\pgfsetlinewidth{0.301125pt}%
\definecolor{currentstroke}{rgb}{0.392157,0.584314,0.929412}%
\pgfsetstrokecolor{currentstroke}%
\pgfsetdash{}{0pt}%
\pgfsys@defobject{currentmarker}{\pgfqpoint{-0.048611in}{-0.048611in}}{\pgfqpoint{0.048611in}{0.048611in}}{%
\pgfpathmoveto{\pgfqpoint{0.000000in}{-0.048611in}}%
\pgfpathcurveto{\pgfqpoint{0.012892in}{-0.048611in}}{\pgfqpoint{0.025257in}{-0.043489in}}{\pgfqpoint{0.034373in}{-0.034373in}}%
\pgfpathcurveto{\pgfqpoint{0.043489in}{-0.025257in}}{\pgfqpoint{0.048611in}{-0.012892in}}{\pgfqpoint{0.048611in}{0.000000in}}%
\pgfpathcurveto{\pgfqpoint{0.048611in}{0.012892in}}{\pgfqpoint{0.043489in}{0.025257in}}{\pgfqpoint{0.034373in}{0.034373in}}%
\pgfpathcurveto{\pgfqpoint{0.025257in}{0.043489in}}{\pgfqpoint{0.012892in}{0.048611in}}{\pgfqpoint{0.000000in}{0.048611in}}%
\pgfpathcurveto{\pgfqpoint{-0.012892in}{0.048611in}}{\pgfqpoint{-0.025257in}{0.043489in}}{\pgfqpoint{-0.034373in}{0.034373in}}%
\pgfpathcurveto{\pgfqpoint{-0.043489in}{0.025257in}}{\pgfqpoint{-0.048611in}{0.012892in}}{\pgfqpoint{-0.048611in}{0.000000in}}%
\pgfpathcurveto{\pgfqpoint{-0.048611in}{-0.012892in}}{\pgfqpoint{-0.043489in}{-0.025257in}}{\pgfqpoint{-0.034373in}{-0.034373in}}%
\pgfpathcurveto{\pgfqpoint{-0.025257in}{-0.043489in}}{\pgfqpoint{-0.012892in}{-0.048611in}}{\pgfqpoint{0.000000in}{-0.048611in}}%
\pgfpathlineto{\pgfqpoint{0.000000in}{-0.048611in}}%
\pgfpathclose%
\pgfusepath{stroke,fill}%
}%
\begin{pgfscope}%
\pgfsys@transformshift{1.995680in}{4.625282in}%
\pgfsys@useobject{currentmarker}{}%
\end{pgfscope}%
\end{pgfscope}%
\begin{pgfscope}%
\definecolor{textcolor}{rgb}{0.150000,0.150000,0.150000}%
\pgfsetstrokecolor{textcolor}%
\pgfsetfillcolor{textcolor}%
\pgftext[x=2.495680in,y=4.552365in,left,base]{\color{textcolor}\rmfamily\fontsize{20.000000}{24.000000}\selectfont Data}%
\end{pgfscope}%
\begin{pgfscope}%
\pgfsetroundcap%
\pgfsetroundjoin%
\pgfsetlinewidth{1.756562pt}%
\definecolor{currentstroke}{rgb}{0.298039,0.447059,0.690196}%
\pgfsetstrokecolor{currentstroke}%
\pgfsetdash{}{0pt}%
\pgfpathmoveto{\pgfqpoint{1.717902in}{4.241873in}}%
\pgfpathlineto{\pgfqpoint{1.995680in}{4.241873in}}%
\pgfpathlineto{\pgfqpoint{2.273458in}{4.241873in}}%
\pgfusepath{stroke}%
\end{pgfscope}%
\begin{pgfscope}%
\definecolor{textcolor}{rgb}{0.150000,0.150000,0.150000}%
\pgfsetstrokecolor{textcolor}%
\pgfsetfillcolor{textcolor}%
\pgftext[x=2.495680in,y=4.144651in,left,base]{\color{textcolor}\rmfamily\fontsize{20.000000}{24.000000}\selectfont Fitted Curve}%
\end{pgfscope}%
\end{pgfpicture}%
\makeatother%
\endgroup%
}
    \caption{T1 measurement of 0.01 M  CuSO$_{4} \boldsymbol{\cdot} $H$_2$O using the two-pulse method }
    \label{fig:0d01_t1}
  \end{figure}

  \begin{figure}[h]
    \resizebox{0.45\textwidth}{!}{%% Creator: Matplotlib, PGF backend
%%
%% To include the figure in your LaTeX document, write
%%   \input{<filename>.pgf}
%%
%% Make sure the required packages are loaded in your preamble
%%   \usepackage{pgf}
%%
%% Also ensure that all the required font packages are loaded; for instance,
%% the lmodern package is sometimes necessary when using math font.
%%   \usepackage{lmodern}
%%
%% Figures using additional raster images can only be included by \input if
%% they are in the same directory as the main LaTeX file. For loading figures
%% from other directories you can use the `import` package
%%   \usepackage{import}
%%
%% and then include the figures with
%%   \import{<path to file>}{<filename>.pgf}
%%
%% Matplotlib used the following preamble
%%   \usepackage{fontspec}
%%   \setmainfont{DejaVuSerif.ttf}[Path=\detokenize{/usr/share/matplotlib/mpl-data/fonts/ttf/}]
%%   \setsansfont{DejaVuSans.ttf}[Path=\detokenize{/usr/share/matplotlib/mpl-data/fonts/ttf/}]
%%   \setmonofont{DejaVuSansMono.ttf}[Path=\detokenize{/usr/share/matplotlib/mpl-data/fonts/ttf/}]
%%
\begingroup%
\makeatletter%
\begin{pgfpicture}%
\pgfpathrectangle{\pgfpointorigin}{\pgfqpoint{8.000000in}{5.500000in}}%
\pgfusepath{use as bounding box, clip}%
\begin{pgfscope}%
\pgfsetbuttcap%
\pgfsetmiterjoin%
\definecolor{currentfill}{rgb}{1.000000,1.000000,1.000000}%
\pgfsetfillcolor{currentfill}%
\pgfsetlinewidth{0.000000pt}%
\definecolor{currentstroke}{rgb}{1.000000,1.000000,1.000000}%
\pgfsetstrokecolor{currentstroke}%
\pgfsetdash{}{0pt}%
\pgfpathmoveto{\pgfqpoint{0.000000in}{0.000000in}}%
\pgfpathlineto{\pgfqpoint{8.000000in}{0.000000in}}%
\pgfpathlineto{\pgfqpoint{8.000000in}{5.500000in}}%
\pgfpathlineto{\pgfqpoint{0.000000in}{5.500000in}}%
\pgfpathlineto{\pgfqpoint{0.000000in}{0.000000in}}%
\pgfpathclose%
\pgfusepath{fill}%
\end{pgfscope}%
\begin{pgfscope}%
\pgfsetbuttcap%
\pgfsetmiterjoin%
\definecolor{currentfill}{rgb}{0.917647,0.917647,0.949020}%
\pgfsetfillcolor{currentfill}%
\pgfsetlinewidth{0.000000pt}%
\definecolor{currentstroke}{rgb}{0.000000,0.000000,0.000000}%
\pgfsetstrokecolor{currentstroke}%
\pgfsetstrokeopacity{0.000000}%
\pgfsetdash{}{0pt}%
\pgfpathmoveto{\pgfqpoint{1.202874in}{0.898915in}}%
\pgfpathlineto{\pgfqpoint{7.850000in}{0.898915in}}%
\pgfpathlineto{\pgfqpoint{7.850000in}{5.013411in}}%
\pgfpathlineto{\pgfqpoint{1.202874in}{5.013411in}}%
\pgfpathlineto{\pgfqpoint{1.202874in}{0.898915in}}%
\pgfpathclose%
\pgfusepath{fill}%
\end{pgfscope}%
\begin{pgfscope}%
\pgfpathrectangle{\pgfqpoint{1.202874in}{0.898915in}}{\pgfqpoint{6.647126in}{4.114497in}}%
\pgfusepath{clip}%
\pgfsetroundcap%
\pgfsetroundjoin%
\pgfsetlinewidth{1.003750pt}%
\definecolor{currentstroke}{rgb}{1.000000,1.000000,1.000000}%
\pgfsetstrokecolor{currentstroke}%
\pgfsetdash{}{0pt}%
\pgfpathmoveto{\pgfqpoint{1.443977in}{0.898915in}}%
\pgfpathlineto{\pgfqpoint{1.443977in}{5.013411in}}%
\pgfusepath{stroke}%
\end{pgfscope}%
\begin{pgfscope}%
\definecolor{textcolor}{rgb}{0.150000,0.150000,0.150000}%
\pgfsetstrokecolor{textcolor}%
\pgfsetfillcolor{textcolor}%
\pgftext[x=1.443977in,y=0.801692in,,top]{\color{textcolor}\rmfamily\fontsize{20.000000}{24.000000}\selectfont 0}%
\end{pgfscope}%
\begin{pgfscope}%
\pgfpathrectangle{\pgfqpoint{1.202874in}{0.898915in}}{\pgfqpoint{6.647126in}{4.114497in}}%
\pgfusepath{clip}%
\pgfsetroundcap%
\pgfsetroundjoin%
\pgfsetlinewidth{1.003750pt}%
\definecolor{currentstroke}{rgb}{1.000000,1.000000,1.000000}%
\pgfsetstrokecolor{currentstroke}%
\pgfsetdash{}{0pt}%
\pgfpathmoveto{\pgfqpoint{2.664753in}{0.898915in}}%
\pgfpathlineto{\pgfqpoint{2.664753in}{5.013411in}}%
\pgfusepath{stroke}%
\end{pgfscope}%
\begin{pgfscope}%
\definecolor{textcolor}{rgb}{0.150000,0.150000,0.150000}%
\pgfsetstrokecolor{textcolor}%
\pgfsetfillcolor{textcolor}%
\pgftext[x=2.664753in,y=0.801692in,,top]{\color{textcolor}\rmfamily\fontsize{20.000000}{24.000000}\selectfont 20}%
\end{pgfscope}%
\begin{pgfscope}%
\pgfpathrectangle{\pgfqpoint{1.202874in}{0.898915in}}{\pgfqpoint{6.647126in}{4.114497in}}%
\pgfusepath{clip}%
\pgfsetroundcap%
\pgfsetroundjoin%
\pgfsetlinewidth{1.003750pt}%
\definecolor{currentstroke}{rgb}{1.000000,1.000000,1.000000}%
\pgfsetstrokecolor{currentstroke}%
\pgfsetdash{}{0pt}%
\pgfpathmoveto{\pgfqpoint{3.885530in}{0.898915in}}%
\pgfpathlineto{\pgfqpoint{3.885530in}{5.013411in}}%
\pgfusepath{stroke}%
\end{pgfscope}%
\begin{pgfscope}%
\definecolor{textcolor}{rgb}{0.150000,0.150000,0.150000}%
\pgfsetstrokecolor{textcolor}%
\pgfsetfillcolor{textcolor}%
\pgftext[x=3.885530in,y=0.801692in,,top]{\color{textcolor}\rmfamily\fontsize{20.000000}{24.000000}\selectfont 40}%
\end{pgfscope}%
\begin{pgfscope}%
\pgfpathrectangle{\pgfqpoint{1.202874in}{0.898915in}}{\pgfqpoint{6.647126in}{4.114497in}}%
\pgfusepath{clip}%
\pgfsetroundcap%
\pgfsetroundjoin%
\pgfsetlinewidth{1.003750pt}%
\definecolor{currentstroke}{rgb}{1.000000,1.000000,1.000000}%
\pgfsetstrokecolor{currentstroke}%
\pgfsetdash{}{0pt}%
\pgfpathmoveto{\pgfqpoint{5.106306in}{0.898915in}}%
\pgfpathlineto{\pgfqpoint{5.106306in}{5.013411in}}%
\pgfusepath{stroke}%
\end{pgfscope}%
\begin{pgfscope}%
\definecolor{textcolor}{rgb}{0.150000,0.150000,0.150000}%
\pgfsetstrokecolor{textcolor}%
\pgfsetfillcolor{textcolor}%
\pgftext[x=5.106306in,y=0.801692in,,top]{\color{textcolor}\rmfamily\fontsize{20.000000}{24.000000}\selectfont 60}%
\end{pgfscope}%
\begin{pgfscope}%
\pgfpathrectangle{\pgfqpoint{1.202874in}{0.898915in}}{\pgfqpoint{6.647126in}{4.114497in}}%
\pgfusepath{clip}%
\pgfsetroundcap%
\pgfsetroundjoin%
\pgfsetlinewidth{1.003750pt}%
\definecolor{currentstroke}{rgb}{1.000000,1.000000,1.000000}%
\pgfsetstrokecolor{currentstroke}%
\pgfsetdash{}{0pt}%
\pgfpathmoveto{\pgfqpoint{6.327082in}{0.898915in}}%
\pgfpathlineto{\pgfqpoint{6.327082in}{5.013411in}}%
\pgfusepath{stroke}%
\end{pgfscope}%
\begin{pgfscope}%
\definecolor{textcolor}{rgb}{0.150000,0.150000,0.150000}%
\pgfsetstrokecolor{textcolor}%
\pgfsetfillcolor{textcolor}%
\pgftext[x=6.327082in,y=0.801692in,,top]{\color{textcolor}\rmfamily\fontsize{20.000000}{24.000000}\selectfont 80}%
\end{pgfscope}%
\begin{pgfscope}%
\pgfpathrectangle{\pgfqpoint{1.202874in}{0.898915in}}{\pgfqpoint{6.647126in}{4.114497in}}%
\pgfusepath{clip}%
\pgfsetroundcap%
\pgfsetroundjoin%
\pgfsetlinewidth{1.003750pt}%
\definecolor{currentstroke}{rgb}{1.000000,1.000000,1.000000}%
\pgfsetstrokecolor{currentstroke}%
\pgfsetdash{}{0pt}%
\pgfpathmoveto{\pgfqpoint{7.547858in}{0.898915in}}%
\pgfpathlineto{\pgfqpoint{7.547858in}{5.013411in}}%
\pgfusepath{stroke}%
\end{pgfscope}%
\begin{pgfscope}%
\definecolor{textcolor}{rgb}{0.150000,0.150000,0.150000}%
\pgfsetstrokecolor{textcolor}%
\pgfsetfillcolor{textcolor}%
\pgftext[x=7.547858in,y=0.801692in,,top]{\color{textcolor}\rmfamily\fontsize{20.000000}{24.000000}\selectfont 100}%
\end{pgfscope}%
\begin{pgfscope}%
\definecolor{textcolor}{rgb}{0.150000,0.150000,0.150000}%
\pgfsetstrokecolor{textcolor}%
\pgfsetfillcolor{textcolor}%
\pgftext[x=4.526437in,y=0.477311in,,top]{\color{textcolor}\rmfamily\fontsize{24.000000}{28.800000}\selectfont Delay Time (ms)}%
\end{pgfscope}%
\begin{pgfscope}%
\pgfpathrectangle{\pgfqpoint{1.202874in}{0.898915in}}{\pgfqpoint{6.647126in}{4.114497in}}%
\pgfusepath{clip}%
\pgfsetroundcap%
\pgfsetroundjoin%
\pgfsetlinewidth{1.003750pt}%
\definecolor{currentstroke}{rgb}{1.000000,1.000000,1.000000}%
\pgfsetstrokecolor{currentstroke}%
\pgfsetdash{}{0pt}%
\pgfpathmoveto{\pgfqpoint{1.202874in}{0.975544in}}%
\pgfpathlineto{\pgfqpoint{7.850000in}{0.975544in}}%
\pgfusepath{stroke}%
\end{pgfscope}%
\begin{pgfscope}%
\definecolor{textcolor}{rgb}{0.150000,0.150000,0.150000}%
\pgfsetstrokecolor{textcolor}%
\pgfsetfillcolor{textcolor}%
\pgftext[x=0.528147in, y=0.870021in, left, base]{\color{textcolor}\rmfamily\fontsize{20.000000}{24.000000}\selectfont \ensuremath{-}10}%
\end{pgfscope}%
\begin{pgfscope}%
\pgfpathrectangle{\pgfqpoint{1.202874in}{0.898915in}}{\pgfqpoint{6.647126in}{4.114497in}}%
\pgfusepath{clip}%
\pgfsetroundcap%
\pgfsetroundjoin%
\pgfsetlinewidth{1.003750pt}%
\definecolor{currentstroke}{rgb}{1.000000,1.000000,1.000000}%
\pgfsetstrokecolor{currentstroke}%
\pgfsetdash{}{0pt}%
\pgfpathmoveto{\pgfqpoint{1.202874in}{1.972300in}}%
\pgfpathlineto{\pgfqpoint{7.850000in}{1.972300in}}%
\pgfusepath{stroke}%
\end{pgfscope}%
\begin{pgfscope}%
\definecolor{textcolor}{rgb}{0.150000,0.150000,0.150000}%
\pgfsetstrokecolor{textcolor}%
\pgfsetfillcolor{textcolor}%
\pgftext[x=0.704877in, y=1.866777in, left, base]{\color{textcolor}\rmfamily\fontsize{20.000000}{24.000000}\selectfont \ensuremath{-}5}%
\end{pgfscope}%
\begin{pgfscope}%
\pgfpathrectangle{\pgfqpoint{1.202874in}{0.898915in}}{\pgfqpoint{6.647126in}{4.114497in}}%
\pgfusepath{clip}%
\pgfsetroundcap%
\pgfsetroundjoin%
\pgfsetlinewidth{1.003750pt}%
\definecolor{currentstroke}{rgb}{1.000000,1.000000,1.000000}%
\pgfsetstrokecolor{currentstroke}%
\pgfsetdash{}{0pt}%
\pgfpathmoveto{\pgfqpoint{1.202874in}{2.969056in}}%
\pgfpathlineto{\pgfqpoint{7.850000in}{2.969056in}}%
\pgfusepath{stroke}%
\end{pgfscope}%
\begin{pgfscope}%
\definecolor{textcolor}{rgb}{0.150000,0.150000,0.150000}%
\pgfsetstrokecolor{textcolor}%
\pgfsetfillcolor{textcolor}%
\pgftext[x=0.928921in, y=2.863533in, left, base]{\color{textcolor}\rmfamily\fontsize{20.000000}{24.000000}\selectfont 0}%
\end{pgfscope}%
\begin{pgfscope}%
\pgfpathrectangle{\pgfqpoint{1.202874in}{0.898915in}}{\pgfqpoint{6.647126in}{4.114497in}}%
\pgfusepath{clip}%
\pgfsetroundcap%
\pgfsetroundjoin%
\pgfsetlinewidth{1.003750pt}%
\definecolor{currentstroke}{rgb}{1.000000,1.000000,1.000000}%
\pgfsetstrokecolor{currentstroke}%
\pgfsetdash{}{0pt}%
\pgfpathmoveto{\pgfqpoint{1.202874in}{3.965812in}}%
\pgfpathlineto{\pgfqpoint{7.850000in}{3.965812in}}%
\pgfusepath{stroke}%
\end{pgfscope}%
\begin{pgfscope}%
\definecolor{textcolor}{rgb}{0.150000,0.150000,0.150000}%
\pgfsetstrokecolor{textcolor}%
\pgfsetfillcolor{textcolor}%
\pgftext[x=0.928921in, y=3.860289in, left, base]{\color{textcolor}\rmfamily\fontsize{20.000000}{24.000000}\selectfont 5}%
\end{pgfscope}%
\begin{pgfscope}%
\pgfpathrectangle{\pgfqpoint{1.202874in}{0.898915in}}{\pgfqpoint{6.647126in}{4.114497in}}%
\pgfusepath{clip}%
\pgfsetroundcap%
\pgfsetroundjoin%
\pgfsetlinewidth{1.003750pt}%
\definecolor{currentstroke}{rgb}{1.000000,1.000000,1.000000}%
\pgfsetstrokecolor{currentstroke}%
\pgfsetdash{}{0pt}%
\pgfpathmoveto{\pgfqpoint{1.202874in}{4.962568in}}%
\pgfpathlineto{\pgfqpoint{7.850000in}{4.962568in}}%
\pgfusepath{stroke}%
\end{pgfscope}%
\begin{pgfscope}%
\definecolor{textcolor}{rgb}{0.150000,0.150000,0.150000}%
\pgfsetstrokecolor{textcolor}%
\pgfsetfillcolor{textcolor}%
\pgftext[x=0.752190in, y=4.857045in, left, base]{\color{textcolor}\rmfamily\fontsize{20.000000}{24.000000}\selectfont 10}%
\end{pgfscope}%
\begin{pgfscope}%
\definecolor{textcolor}{rgb}{0.150000,0.150000,0.150000}%
\pgfsetstrokecolor{textcolor}%
\pgfsetfillcolor{textcolor}%
\pgftext[x=0.472591in,y=2.956163in,,bottom,rotate=90.000000]{\color{textcolor}\rmfamily\fontsize{24.000000}{28.800000}\selectfont Pulse Amplitude (V)}%
\end{pgfscope}%
\begin{pgfscope}%
\pgfpathrectangle{\pgfqpoint{1.202874in}{0.898915in}}{\pgfqpoint{6.647126in}{4.114497in}}%
\pgfusepath{clip}%
\pgfsetbuttcap%
\pgfsetroundjoin%
\definecolor{currentfill}{rgb}{0.392157,0.584314,0.929412}%
\pgfsetfillcolor{currentfill}%
\pgfsetlinewidth{0.301125pt}%
\definecolor{currentstroke}{rgb}{0.392157,0.584314,0.929412}%
\pgfsetstrokecolor{currentstroke}%
\pgfsetdash{}{0pt}%
\pgfsys@defobject{currentmarker}{\pgfqpoint{-0.048611in}{-0.048611in}}{\pgfqpoint{0.048611in}{0.048611in}}{%
\pgfpathmoveto{\pgfqpoint{0.000000in}{-0.048611in}}%
\pgfpathcurveto{\pgfqpoint{0.012892in}{-0.048611in}}{\pgfqpoint{0.025257in}{-0.043489in}}{\pgfqpoint{0.034373in}{-0.034373in}}%
\pgfpathcurveto{\pgfqpoint{0.043489in}{-0.025257in}}{\pgfqpoint{0.048611in}{-0.012892in}}{\pgfqpoint{0.048611in}{0.000000in}}%
\pgfpathcurveto{\pgfqpoint{0.048611in}{0.012892in}}{\pgfqpoint{0.043489in}{0.025257in}}{\pgfqpoint{0.034373in}{0.034373in}}%
\pgfpathcurveto{\pgfqpoint{0.025257in}{0.043489in}}{\pgfqpoint{0.012892in}{0.048611in}}{\pgfqpoint{0.000000in}{0.048611in}}%
\pgfpathcurveto{\pgfqpoint{-0.012892in}{0.048611in}}{\pgfqpoint{-0.025257in}{0.043489in}}{\pgfqpoint{-0.034373in}{0.034373in}}%
\pgfpathcurveto{\pgfqpoint{-0.043489in}{0.025257in}}{\pgfqpoint{-0.048611in}{0.012892in}}{\pgfqpoint{-0.048611in}{0.000000in}}%
\pgfpathcurveto{\pgfqpoint{-0.048611in}{-0.012892in}}{\pgfqpoint{-0.043489in}{-0.025257in}}{\pgfqpoint{-0.034373in}{-0.034373in}}%
\pgfpathcurveto{\pgfqpoint{-0.025257in}{-0.043489in}}{\pgfqpoint{-0.012892in}{-0.048611in}}{\pgfqpoint{0.000000in}{-0.048611in}}%
\pgfpathlineto{\pgfqpoint{0.000000in}{-0.048611in}}%
\pgfpathclose%
\pgfusepath{stroke,fill}%
}%
\begin{pgfscope}%
\pgfsys@transformshift{1.505016in}{1.182869in}%
\pgfsys@useobject{currentmarker}{}%
\end{pgfscope}%
\begin{pgfscope}%
\pgfsys@transformshift{1.627094in}{1.501831in}%
\pgfsys@useobject{currentmarker}{}%
\end{pgfscope}%
\begin{pgfscope}%
\pgfsys@transformshift{1.749171in}{1.820793in}%
\pgfsys@useobject{currentmarker}{}%
\end{pgfscope}%
\begin{pgfscope}%
\pgfsys@transformshift{1.871249in}{2.107859in}%
\pgfsys@useobject{currentmarker}{}%
\end{pgfscope}%
\begin{pgfscope}%
\pgfsys@transformshift{2.054365in}{2.474665in}%
\pgfsys@useobject{currentmarker}{}%
\end{pgfscope}%
\begin{pgfscope}%
\pgfsys@transformshift{2.664753in}{3.702669in}%
\pgfsys@useobject{currentmarker}{}%
\end{pgfscope}%
\begin{pgfscope}%
\pgfsys@transformshift{3.275141in}{4.181112in}%
\pgfsys@useobject{currentmarker}{}%
\end{pgfscope}%
\begin{pgfscope}%
\pgfsys@transformshift{3.885530in}{4.452229in}%
\pgfsys@useobject{currentmarker}{}%
\end{pgfscope}%
\begin{pgfscope}%
\pgfsys@transformshift{4.495918in}{4.627658in}%
\pgfsys@useobject{currentmarker}{}%
\end{pgfscope}%
\begin{pgfscope}%
\pgfsys@transformshift{5.106306in}{4.707399in}%
\pgfsys@useobject{currentmarker}{}%
\end{pgfscope}%
\begin{pgfscope}%
\pgfsys@transformshift{5.716694in}{4.755243in}%
\pgfsys@useobject{currentmarker}{}%
\end{pgfscope}%
\begin{pgfscope}%
\pgfsys@transformshift{6.327082in}{4.787139in}%
\pgfsys@useobject{currentmarker}{}%
\end{pgfscope}%
\begin{pgfscope}%
\pgfsys@transformshift{6.937470in}{4.803087in}%
\pgfsys@useobject{currentmarker}{}%
\end{pgfscope}%
\begin{pgfscope}%
\pgfsys@transformshift{7.547858in}{4.803087in}%
\pgfsys@useobject{currentmarker}{}%
\end{pgfscope}%
\end{pgfscope}%
\begin{pgfscope}%
\pgfpathrectangle{\pgfqpoint{1.202874in}{0.898915in}}{\pgfqpoint{6.647126in}{4.114497in}}%
\pgfusepath{clip}%
\pgfsetroundcap%
\pgfsetroundjoin%
\pgfsetlinewidth{1.756562pt}%
\definecolor{currentstroke}{rgb}{0.298039,0.447059,0.690196}%
\pgfsetstrokecolor{currentstroke}%
\pgfsetdash{}{0pt}%
\pgfpathmoveto{\pgfqpoint{1.505016in}{1.085937in}}%
\pgfpathlineto{\pgfqpoint{1.566055in}{1.295615in}}%
\pgfpathlineto{\pgfqpoint{1.627094in}{1.493579in}}%
\pgfpathlineto{\pgfqpoint{1.688133in}{1.680483in}}%
\pgfpathlineto{\pgfqpoint{1.749171in}{1.856946in}}%
\pgfpathlineto{\pgfqpoint{1.810210in}{2.023549in}}%
\pgfpathlineto{\pgfqpoint{1.871249in}{2.180845in}}%
\pgfpathlineto{\pgfqpoint{1.932288in}{2.329353in}}%
\pgfpathlineto{\pgfqpoint{1.993327in}{2.469565in}}%
\pgfpathlineto{\pgfqpoint{2.054365in}{2.601943in}}%
\pgfpathlineto{\pgfqpoint{2.115404in}{2.726925in}}%
\pgfpathlineto{\pgfqpoint{2.176443in}{2.844925in}}%
\pgfpathlineto{\pgfqpoint{2.237482in}{2.956332in}}%
\pgfpathlineto{\pgfqpoint{2.298521in}{3.061516in}}%
\pgfpathlineto{\pgfqpoint{2.359559in}{3.160823in}}%
\pgfpathlineto{\pgfqpoint{2.420598in}{3.254582in}}%
\pgfpathlineto{\pgfqpoint{2.481637in}{3.343103in}}%
\pgfpathlineto{\pgfqpoint{2.542676in}{3.426678in}}%
\pgfpathlineto{\pgfqpoint{2.603715in}{3.505585in}}%
\pgfpathlineto{\pgfqpoint{2.664753in}{3.580082in}}%
\pgfpathlineto{\pgfqpoint{2.725792in}{3.650418in}}%
\pgfpathlineto{\pgfqpoint{2.786831in}{3.716825in}}%
\pgfpathlineto{\pgfqpoint{2.847870in}{3.779521in}}%
\pgfpathlineto{\pgfqpoint{2.908909in}{3.838715in}}%
\pgfpathlineto{\pgfqpoint{2.969947in}{3.894602in}}%
\pgfpathlineto{\pgfqpoint{3.030986in}{3.947366in}}%
\pgfpathlineto{\pgfqpoint{3.092025in}{3.997183in}}%
\pgfpathlineto{\pgfqpoint{3.153064in}{4.044216in}}%
\pgfpathlineto{\pgfqpoint{3.214103in}{4.088622in}}%
\pgfpathlineto{\pgfqpoint{3.275141in}{4.130547in}}%
\pgfpathlineto{\pgfqpoint{3.336180in}{4.170130in}}%
\pgfpathlineto{\pgfqpoint{3.397219in}{4.207501in}}%
\pgfpathlineto{\pgfqpoint{3.458258in}{4.242785in}}%
\pgfpathlineto{\pgfqpoint{3.519297in}{4.276097in}}%
\pgfpathlineto{\pgfqpoint{3.580336in}{4.307548in}}%
\pgfpathlineto{\pgfqpoint{3.641374in}{4.337242in}}%
\pgfpathlineto{\pgfqpoint{3.702413in}{4.365277in}}%
\pgfpathlineto{\pgfqpoint{3.763452in}{4.391746in}}%
\pgfpathlineto{\pgfqpoint{3.824491in}{4.416736in}}%
\pgfpathlineto{\pgfqpoint{3.885530in}{4.440330in}}%
\pgfpathlineto{\pgfqpoint{3.946568in}{4.462606in}}%
\pgfpathlineto{\pgfqpoint{4.007607in}{4.483637in}}%
\pgfpathlineto{\pgfqpoint{4.068646in}{4.503494in}}%
\pgfpathlineto{\pgfqpoint{4.129685in}{4.522241in}}%
\pgfpathlineto{\pgfqpoint{4.190724in}{4.539941in}}%
\pgfpathlineto{\pgfqpoint{4.251762in}{4.556651in}}%
\pgfpathlineto{\pgfqpoint{4.312801in}{4.572429in}}%
\pgfpathlineto{\pgfqpoint{4.373840in}{4.587324in}}%
\pgfpathlineto{\pgfqpoint{4.434879in}{4.601388in}}%
\pgfpathlineto{\pgfqpoint{4.495918in}{4.614666in}}%
\pgfpathlineto{\pgfqpoint{4.556956in}{4.627202in}}%
\pgfpathlineto{\pgfqpoint{4.617995in}{4.639038in}}%
\pgfpathlineto{\pgfqpoint{4.679034in}{4.650212in}}%
\pgfpathlineto{\pgfqpoint{4.740073in}{4.660762in}}%
\pgfpathlineto{\pgfqpoint{4.801112in}{4.670723in}}%
\pgfpathlineto{\pgfqpoint{4.862150in}{4.680128in}}%
\pgfpathlineto{\pgfqpoint{4.923189in}{4.689007in}}%
\pgfpathlineto{\pgfqpoint{4.984228in}{4.697389in}}%
\pgfpathlineto{\pgfqpoint{5.045267in}{4.705304in}}%
\pgfpathlineto{\pgfqpoint{5.106306in}{4.712776in}}%
\pgfpathlineto{\pgfqpoint{5.167344in}{4.719831in}}%
\pgfpathlineto{\pgfqpoint{5.228383in}{4.726492in}}%
\pgfpathlineto{\pgfqpoint{5.289422in}{4.732781in}}%
\pgfpathlineto{\pgfqpoint{5.350461in}{4.738718in}}%
\pgfpathlineto{\pgfqpoint{5.411500in}{4.744323in}}%
\pgfpathlineto{\pgfqpoint{5.472539in}{4.749616in}}%
\pgfpathlineto{\pgfqpoint{5.533577in}{4.754613in}}%
\pgfpathlineto{\pgfqpoint{5.594616in}{4.759330in}}%
\pgfpathlineto{\pgfqpoint{5.655655in}{4.763784in}}%
\pgfpathlineto{\pgfqpoint{5.716694in}{4.767989in}}%
\pgfpathlineto{\pgfqpoint{5.777733in}{4.771960in}}%
\pgfpathlineto{\pgfqpoint{5.838771in}{4.775708in}}%
\pgfpathlineto{\pgfqpoint{5.899810in}{4.779247in}}%
\pgfpathlineto{\pgfqpoint{5.960849in}{4.782589in}}%
\pgfpathlineto{\pgfqpoint{6.021888in}{4.785743in}}%
\pgfpathlineto{\pgfqpoint{6.082927in}{4.788722in}}%
\pgfpathlineto{\pgfqpoint{6.143965in}{4.791534in}}%
\pgfpathlineto{\pgfqpoint{6.205004in}{4.794189in}}%
\pgfpathlineto{\pgfqpoint{6.266043in}{4.796695in}}%
\pgfpathlineto{\pgfqpoint{6.327082in}{4.799062in}}%
\pgfpathlineto{\pgfqpoint{6.388121in}{4.801296in}}%
\pgfpathlineto{\pgfqpoint{6.449159in}{4.803406in}}%
\pgfpathlineto{\pgfqpoint{6.510198in}{4.805397in}}%
\pgfpathlineto{\pgfqpoint{6.571237in}{4.807278in}}%
\pgfpathlineto{\pgfqpoint{6.632276in}{4.809053in}}%
\pgfpathlineto{\pgfqpoint{6.693315in}{4.810729in}}%
\pgfpathlineto{\pgfqpoint{6.754353in}{4.812312in}}%
\pgfpathlineto{\pgfqpoint{6.815392in}{4.813806in}}%
\pgfpathlineto{\pgfqpoint{6.876431in}{4.815216in}}%
\pgfpathlineto{\pgfqpoint{6.937470in}{4.816548in}}%
\pgfpathlineto{\pgfqpoint{6.998509in}{4.817805in}}%
\pgfpathlineto{\pgfqpoint{7.059547in}{4.818993in}}%
\pgfpathlineto{\pgfqpoint{7.120586in}{4.820113in}}%
\pgfpathlineto{\pgfqpoint{7.181625in}{4.821172in}}%
\pgfpathlineto{\pgfqpoint{7.242664in}{4.822171in}}%
\pgfpathlineto{\pgfqpoint{7.303703in}{4.823114in}}%
\pgfpathlineto{\pgfqpoint{7.364741in}{4.824005in}}%
\pgfpathlineto{\pgfqpoint{7.425780in}{4.824845in}}%
\pgfpathlineto{\pgfqpoint{7.486819in}{4.825639in}}%
\pgfpathlineto{\pgfqpoint{7.547858in}{4.826389in}}%
\pgfusepath{stroke}%
\end{pgfscope}%
\begin{pgfscope}%
\pgfsetrectcap%
\pgfsetmiterjoin%
\pgfsetlinewidth{0.000000pt}%
\definecolor{currentstroke}{rgb}{1.000000,1.000000,1.000000}%
\pgfsetstrokecolor{currentstroke}%
\pgfsetdash{}{0pt}%
\pgfpathmoveto{\pgfqpoint{1.202874in}{0.898915in}}%
\pgfpathlineto{\pgfqpoint{1.202874in}{5.013411in}}%
\pgfusepath{}%
\end{pgfscope}%
\begin{pgfscope}%
\pgfsetrectcap%
\pgfsetmiterjoin%
\pgfsetlinewidth{0.000000pt}%
\definecolor{currentstroke}{rgb}{1.000000,1.000000,1.000000}%
\pgfsetstrokecolor{currentstroke}%
\pgfsetdash{}{0pt}%
\pgfpathmoveto{\pgfqpoint{7.850000in}{0.898915in}}%
\pgfpathlineto{\pgfqpoint{7.850000in}{5.013411in}}%
\pgfusepath{}%
\end{pgfscope}%
\begin{pgfscope}%
\pgfsetrectcap%
\pgfsetmiterjoin%
\pgfsetlinewidth{0.000000pt}%
\definecolor{currentstroke}{rgb}{1.000000,1.000000,1.000000}%
\pgfsetstrokecolor{currentstroke}%
\pgfsetdash{}{0pt}%
\pgfpathmoveto{\pgfqpoint{1.202874in}{0.898915in}}%
\pgfpathlineto{\pgfqpoint{7.850000in}{0.898915in}}%
\pgfusepath{}%
\end{pgfscope}%
\begin{pgfscope}%
\pgfsetrectcap%
\pgfsetmiterjoin%
\pgfsetlinewidth{0.000000pt}%
\definecolor{currentstroke}{rgb}{1.000000,1.000000,1.000000}%
\pgfsetstrokecolor{currentstroke}%
\pgfsetdash{}{0pt}%
\pgfpathmoveto{\pgfqpoint{1.202874in}{5.013411in}}%
\pgfpathlineto{\pgfqpoint{7.850000in}{5.013411in}}%
\pgfusepath{}%
\end{pgfscope}%
\begin{pgfscope}%
\definecolor{textcolor}{rgb}{0.150000,0.150000,0.150000}%
\pgfsetstrokecolor{textcolor}%
\pgfsetfillcolor{textcolor}%
\pgftext[x=2.969947in, y=2.865058in, left, base]{\color{textcolor}\rmfamily\fontsize{20.000000}{24.000000}\selectfont M\(\displaystyle _z\)(t) = 9.97(1\(\displaystyle -\)2\(\displaystyle exp[\frac{-t}{T_1}]\))\(\displaystyle +\)-0.59}%
\end{pgfscope}%
\begin{pgfscope}%
\definecolor{textcolor}{rgb}{0.150000,0.150000,0.150000}%
\pgfsetstrokecolor{textcolor}%
\pgfsetfillcolor{textcolor}%
\pgftext[x=2.969947in, y=2.371002in, left, base]{\color{textcolor}\rmfamily\fontsize{20.000000}{24.000000}\selectfont T\(\displaystyle _1\) = 17.39\(\displaystyle \pm\)0.73 ms}%
\end{pgfscope}%
\begin{pgfscope}%
\definecolor{textcolor}{rgb}{0.150000,0.150000,0.150000}%
\pgfsetstrokecolor{textcolor}%
\pgfsetfillcolor{textcolor}%
\pgftext[x=4.526437in,y=5.096745in,,base]{\color{textcolor}\rmfamily\fontsize{24.000000}{28.800000}\selectfont T\(\displaystyle _1\) measurement of 0.05 M CuSO\(\displaystyle _4\)}%
\end{pgfscope}%
\begin{pgfscope}%
\pgfsetbuttcap%
\pgfsetroundjoin%
\definecolor{currentfill}{rgb}{0.392157,0.584314,0.929412}%
\pgfsetfillcolor{currentfill}%
\pgfsetlinewidth{0.301125pt}%
\definecolor{currentstroke}{rgb}{0.392157,0.584314,0.929412}%
\pgfsetstrokecolor{currentstroke}%
\pgfsetdash{}{0pt}%
\pgfsys@defobject{currentmarker}{\pgfqpoint{-0.048611in}{-0.048611in}}{\pgfqpoint{0.048611in}{0.048611in}}{%
\pgfpathmoveto{\pgfqpoint{0.000000in}{-0.048611in}}%
\pgfpathcurveto{\pgfqpoint{0.012892in}{-0.048611in}}{\pgfqpoint{0.025257in}{-0.043489in}}{\pgfqpoint{0.034373in}{-0.034373in}}%
\pgfpathcurveto{\pgfqpoint{0.043489in}{-0.025257in}}{\pgfqpoint{0.048611in}{-0.012892in}}{\pgfqpoint{0.048611in}{0.000000in}}%
\pgfpathcurveto{\pgfqpoint{0.048611in}{0.012892in}}{\pgfqpoint{0.043489in}{0.025257in}}{\pgfqpoint{0.034373in}{0.034373in}}%
\pgfpathcurveto{\pgfqpoint{0.025257in}{0.043489in}}{\pgfqpoint{0.012892in}{0.048611in}}{\pgfqpoint{0.000000in}{0.048611in}}%
\pgfpathcurveto{\pgfqpoint{-0.012892in}{0.048611in}}{\pgfqpoint{-0.025257in}{0.043489in}}{\pgfqpoint{-0.034373in}{0.034373in}}%
\pgfpathcurveto{\pgfqpoint{-0.043489in}{0.025257in}}{\pgfqpoint{-0.048611in}{0.012892in}}{\pgfqpoint{-0.048611in}{0.000000in}}%
\pgfpathcurveto{\pgfqpoint{-0.048611in}{-0.012892in}}{\pgfqpoint{-0.043489in}{-0.025257in}}{\pgfqpoint{-0.034373in}{-0.034373in}}%
\pgfpathcurveto{\pgfqpoint{-0.025257in}{-0.043489in}}{\pgfqpoint{-0.012892in}{-0.048611in}}{\pgfqpoint{0.000000in}{-0.048611in}}%
\pgfpathlineto{\pgfqpoint{0.000000in}{-0.048611in}}%
\pgfpathclose%
\pgfusepath{stroke,fill}%
}%
\begin{pgfscope}%
\pgfsys@transformshift{5.319130in}{1.687326in}%
\pgfsys@useobject{currentmarker}{}%
\end{pgfscope}%
\end{pgfscope}%
\begin{pgfscope}%
\definecolor{textcolor}{rgb}{0.150000,0.150000,0.150000}%
\pgfsetstrokecolor{textcolor}%
\pgfsetfillcolor{textcolor}%
\pgftext[x=5.819130in,y=1.614409in,left,base]{\color{textcolor}\rmfamily\fontsize{20.000000}{24.000000}\selectfont Data}%
\end{pgfscope}%
\begin{pgfscope}%
\pgfsetroundcap%
\pgfsetroundjoin%
\pgfsetlinewidth{1.756562pt}%
\definecolor{currentstroke}{rgb}{0.298039,0.447059,0.690196}%
\pgfsetstrokecolor{currentstroke}%
\pgfsetdash{}{0pt}%
\pgfpathmoveto{\pgfqpoint{5.041352in}{1.303917in}}%
\pgfpathlineto{\pgfqpoint{5.319130in}{1.303917in}}%
\pgfpathlineto{\pgfqpoint{5.596907in}{1.303917in}}%
\pgfusepath{stroke}%
\end{pgfscope}%
\begin{pgfscope}%
\definecolor{textcolor}{rgb}{0.150000,0.150000,0.150000}%
\pgfsetstrokecolor{textcolor}%
\pgfsetfillcolor{textcolor}%
\pgftext[x=5.819130in,y=1.206694in,left,base]{\color{textcolor}\rmfamily\fontsize{20.000000}{24.000000}\selectfont Fitted Curve}%
\end{pgfscope}%
\end{pgfpicture}%
\makeatother%
\endgroup%
}
    \caption{T1 measurement of 0.05 M  CuSO$_{4} \boldsymbol{\cdot} $H$_2$O using the two-pulse method }
    \label{fig:0d05_t1}
  \end{figure}

  \begin{figure}[h]
    \resizebox{0.45\textwidth}{!}{%% Creator: Matplotlib, PGF backend
%%
%% To include the figure in your LaTeX document, write
%%   \input{<filename>.pgf}
%%
%% Make sure the required packages are loaded in your preamble
%%   \usepackage{pgf}
%%
%% Also ensure that all the required font packages are loaded; for instance,
%% the lmodern package is sometimes necessary when using math font.
%%   \usepackage{lmodern}
%%
%% Figures using additional raster images can only be included by \input if
%% they are in the same directory as the main LaTeX file. For loading figures
%% from other directories you can use the `import` package
%%   \usepackage{import}
%%
%% and then include the figures with
%%   \import{<path to file>}{<filename>.pgf}
%%
%% Matplotlib used the following preamble
%%   \usepackage{fontspec}
%%   \setmainfont{DejaVuSerif.ttf}[Path=\detokenize{/usr/share/matplotlib/mpl-data/fonts/ttf/}]
%%   \setsansfont{DejaVuSans.ttf}[Path=\detokenize{/usr/share/matplotlib/mpl-data/fonts/ttf/}]
%%   \setmonofont{DejaVuSansMono.ttf}[Path=\detokenize{/usr/share/matplotlib/mpl-data/fonts/ttf/}]
%%
\begingroup%
\makeatletter%
\begin{pgfpicture}%
\pgfpathrectangle{\pgfpointorigin}{\pgfqpoint{8.000000in}{5.500000in}}%
\pgfusepath{use as bounding box, clip}%
\begin{pgfscope}%
\pgfsetbuttcap%
\pgfsetmiterjoin%
\definecolor{currentfill}{rgb}{1.000000,1.000000,1.000000}%
\pgfsetfillcolor{currentfill}%
\pgfsetlinewidth{0.000000pt}%
\definecolor{currentstroke}{rgb}{1.000000,1.000000,1.000000}%
\pgfsetstrokecolor{currentstroke}%
\pgfsetdash{}{0pt}%
\pgfpathmoveto{\pgfqpoint{0.000000in}{0.000000in}}%
\pgfpathlineto{\pgfqpoint{8.000000in}{0.000000in}}%
\pgfpathlineto{\pgfqpoint{8.000000in}{5.500000in}}%
\pgfpathlineto{\pgfqpoint{0.000000in}{5.500000in}}%
\pgfpathlineto{\pgfqpoint{0.000000in}{0.000000in}}%
\pgfpathclose%
\pgfusepath{fill}%
\end{pgfscope}%
\begin{pgfscope}%
\pgfsetbuttcap%
\pgfsetmiterjoin%
\definecolor{currentfill}{rgb}{0.917647,0.917647,0.949020}%
\pgfsetfillcolor{currentfill}%
\pgfsetlinewidth{0.000000pt}%
\definecolor{currentstroke}{rgb}{0.000000,0.000000,0.000000}%
\pgfsetstrokecolor{currentstroke}%
\pgfsetstrokeopacity{0.000000}%
\pgfsetdash{}{0pt}%
\pgfpathmoveto{\pgfqpoint{1.291172in}{0.898915in}}%
\pgfpathlineto{\pgfqpoint{7.850000in}{0.898915in}}%
\pgfpathlineto{\pgfqpoint{7.850000in}{5.013411in}}%
\pgfpathlineto{\pgfqpoint{1.291172in}{5.013411in}}%
\pgfpathlineto{\pgfqpoint{1.291172in}{0.898915in}}%
\pgfpathclose%
\pgfusepath{fill}%
\end{pgfscope}%
\begin{pgfscope}%
\pgfpathrectangle{\pgfqpoint{1.291172in}{0.898915in}}{\pgfqpoint{6.558828in}{4.114497in}}%
\pgfusepath{clip}%
\pgfsetroundcap%
\pgfsetroundjoin%
\pgfsetlinewidth{1.003750pt}%
\definecolor{currentstroke}{rgb}{1.000000,1.000000,1.000000}%
\pgfsetstrokecolor{currentstroke}%
\pgfsetdash{}{0pt}%
\pgfpathmoveto{\pgfqpoint{1.305368in}{0.898915in}}%
\pgfpathlineto{\pgfqpoint{1.305368in}{5.013411in}}%
\pgfusepath{stroke}%
\end{pgfscope}%
\begin{pgfscope}%
\definecolor{textcolor}{rgb}{0.150000,0.150000,0.150000}%
\pgfsetstrokecolor{textcolor}%
\pgfsetfillcolor{textcolor}%
\pgftext[x=1.305368in,y=0.801692in,,top]{\color{textcolor}\rmfamily\fontsize{20.000000}{24.000000}\selectfont 0}%
\end{pgfscope}%
\begin{pgfscope}%
\pgfpathrectangle{\pgfqpoint{1.291172in}{0.898915in}}{\pgfqpoint{6.558828in}{4.114497in}}%
\pgfusepath{clip}%
\pgfsetroundcap%
\pgfsetroundjoin%
\pgfsetlinewidth{1.003750pt}%
\definecolor{currentstroke}{rgb}{1.000000,1.000000,1.000000}%
\pgfsetstrokecolor{currentstroke}%
\pgfsetdash{}{0pt}%
\pgfpathmoveto{\pgfqpoint{2.725028in}{0.898915in}}%
\pgfpathlineto{\pgfqpoint{2.725028in}{5.013411in}}%
\pgfusepath{stroke}%
\end{pgfscope}%
\begin{pgfscope}%
\definecolor{textcolor}{rgb}{0.150000,0.150000,0.150000}%
\pgfsetstrokecolor{textcolor}%
\pgfsetfillcolor{textcolor}%
\pgftext[x=2.725028in,y=0.801692in,,top]{\color{textcolor}\rmfamily\fontsize{20.000000}{24.000000}\selectfont 5}%
\end{pgfscope}%
\begin{pgfscope}%
\pgfpathrectangle{\pgfqpoint{1.291172in}{0.898915in}}{\pgfqpoint{6.558828in}{4.114497in}}%
\pgfusepath{clip}%
\pgfsetroundcap%
\pgfsetroundjoin%
\pgfsetlinewidth{1.003750pt}%
\definecolor{currentstroke}{rgb}{1.000000,1.000000,1.000000}%
\pgfsetstrokecolor{currentstroke}%
\pgfsetdash{}{0pt}%
\pgfpathmoveto{\pgfqpoint{4.144688in}{0.898915in}}%
\pgfpathlineto{\pgfqpoint{4.144688in}{5.013411in}}%
\pgfusepath{stroke}%
\end{pgfscope}%
\begin{pgfscope}%
\definecolor{textcolor}{rgb}{0.150000,0.150000,0.150000}%
\pgfsetstrokecolor{textcolor}%
\pgfsetfillcolor{textcolor}%
\pgftext[x=4.144688in,y=0.801692in,,top]{\color{textcolor}\rmfamily\fontsize{20.000000}{24.000000}\selectfont 10}%
\end{pgfscope}%
\begin{pgfscope}%
\pgfpathrectangle{\pgfqpoint{1.291172in}{0.898915in}}{\pgfqpoint{6.558828in}{4.114497in}}%
\pgfusepath{clip}%
\pgfsetroundcap%
\pgfsetroundjoin%
\pgfsetlinewidth{1.003750pt}%
\definecolor{currentstroke}{rgb}{1.000000,1.000000,1.000000}%
\pgfsetstrokecolor{currentstroke}%
\pgfsetdash{}{0pt}%
\pgfpathmoveto{\pgfqpoint{5.564348in}{0.898915in}}%
\pgfpathlineto{\pgfqpoint{5.564348in}{5.013411in}}%
\pgfusepath{stroke}%
\end{pgfscope}%
\begin{pgfscope}%
\definecolor{textcolor}{rgb}{0.150000,0.150000,0.150000}%
\pgfsetstrokecolor{textcolor}%
\pgfsetfillcolor{textcolor}%
\pgftext[x=5.564348in,y=0.801692in,,top]{\color{textcolor}\rmfamily\fontsize{20.000000}{24.000000}\selectfont 15}%
\end{pgfscope}%
\begin{pgfscope}%
\pgfpathrectangle{\pgfqpoint{1.291172in}{0.898915in}}{\pgfqpoint{6.558828in}{4.114497in}}%
\pgfusepath{clip}%
\pgfsetroundcap%
\pgfsetroundjoin%
\pgfsetlinewidth{1.003750pt}%
\definecolor{currentstroke}{rgb}{1.000000,1.000000,1.000000}%
\pgfsetstrokecolor{currentstroke}%
\pgfsetdash{}{0pt}%
\pgfpathmoveto{\pgfqpoint{6.984007in}{0.898915in}}%
\pgfpathlineto{\pgfqpoint{6.984007in}{5.013411in}}%
\pgfusepath{stroke}%
\end{pgfscope}%
\begin{pgfscope}%
\definecolor{textcolor}{rgb}{0.150000,0.150000,0.150000}%
\pgfsetstrokecolor{textcolor}%
\pgfsetfillcolor{textcolor}%
\pgftext[x=6.984007in,y=0.801692in,,top]{\color{textcolor}\rmfamily\fontsize{20.000000}{24.000000}\selectfont 20}%
\end{pgfscope}%
\begin{pgfscope}%
\definecolor{textcolor}{rgb}{0.150000,0.150000,0.150000}%
\pgfsetstrokecolor{textcolor}%
\pgfsetfillcolor{textcolor}%
\pgftext[x=4.570586in,y=0.477311in,,top]{\color{textcolor}\rmfamily\fontsize{24.000000}{28.800000}\selectfont Delay Time (ms)}%
\end{pgfscope}%
\begin{pgfscope}%
\pgfpathrectangle{\pgfqpoint{1.291172in}{0.898915in}}{\pgfqpoint{6.558828in}{4.114497in}}%
\pgfusepath{clip}%
\pgfsetroundcap%
\pgfsetroundjoin%
\pgfsetlinewidth{1.003750pt}%
\definecolor{currentstroke}{rgb}{1.000000,1.000000,1.000000}%
\pgfsetstrokecolor{currentstroke}%
\pgfsetdash{}{0pt}%
\pgfpathmoveto{\pgfqpoint{1.291172in}{1.144656in}}%
\pgfpathlineto{\pgfqpoint{7.850000in}{1.144656in}}%
\pgfusepath{stroke}%
\end{pgfscope}%
\begin{pgfscope}%
\definecolor{textcolor}{rgb}{0.150000,0.150000,0.150000}%
\pgfsetstrokecolor{textcolor}%
\pgfsetfillcolor{textcolor}%
\pgftext[x=0.528147in, y=1.039133in, left, base]{\color{textcolor}\rmfamily\fontsize{20.000000}{24.000000}\selectfont \ensuremath{-}7.5}%
\end{pgfscope}%
\begin{pgfscope}%
\pgfpathrectangle{\pgfqpoint{1.291172in}{0.898915in}}{\pgfqpoint{6.558828in}{4.114497in}}%
\pgfusepath{clip}%
\pgfsetroundcap%
\pgfsetroundjoin%
\pgfsetlinewidth{1.003750pt}%
\definecolor{currentstroke}{rgb}{1.000000,1.000000,1.000000}%
\pgfsetstrokecolor{currentstroke}%
\pgfsetdash{}{0pt}%
\pgfpathmoveto{\pgfqpoint{1.291172in}{1.774765in}}%
\pgfpathlineto{\pgfqpoint{7.850000in}{1.774765in}}%
\pgfusepath{stroke}%
\end{pgfscope}%
\begin{pgfscope}%
\definecolor{textcolor}{rgb}{0.150000,0.150000,0.150000}%
\pgfsetstrokecolor{textcolor}%
\pgfsetfillcolor{textcolor}%
\pgftext[x=0.528147in, y=1.669242in, left, base]{\color{textcolor}\rmfamily\fontsize{20.000000}{24.000000}\selectfont \ensuremath{-}5.0}%
\end{pgfscope}%
\begin{pgfscope}%
\pgfpathrectangle{\pgfqpoint{1.291172in}{0.898915in}}{\pgfqpoint{6.558828in}{4.114497in}}%
\pgfusepath{clip}%
\pgfsetroundcap%
\pgfsetroundjoin%
\pgfsetlinewidth{1.003750pt}%
\definecolor{currentstroke}{rgb}{1.000000,1.000000,1.000000}%
\pgfsetstrokecolor{currentstroke}%
\pgfsetdash{}{0pt}%
\pgfpathmoveto{\pgfqpoint{1.291172in}{2.404874in}}%
\pgfpathlineto{\pgfqpoint{7.850000in}{2.404874in}}%
\pgfusepath{stroke}%
\end{pgfscope}%
\begin{pgfscope}%
\definecolor{textcolor}{rgb}{0.150000,0.150000,0.150000}%
\pgfsetstrokecolor{textcolor}%
\pgfsetfillcolor{textcolor}%
\pgftext[x=0.528147in, y=2.299351in, left, base]{\color{textcolor}\rmfamily\fontsize{20.000000}{24.000000}\selectfont \ensuremath{-}2.5}%
\end{pgfscope}%
\begin{pgfscope}%
\pgfpathrectangle{\pgfqpoint{1.291172in}{0.898915in}}{\pgfqpoint{6.558828in}{4.114497in}}%
\pgfusepath{clip}%
\pgfsetroundcap%
\pgfsetroundjoin%
\pgfsetlinewidth{1.003750pt}%
\definecolor{currentstroke}{rgb}{1.000000,1.000000,1.000000}%
\pgfsetstrokecolor{currentstroke}%
\pgfsetdash{}{0pt}%
\pgfpathmoveto{\pgfqpoint{1.291172in}{3.034983in}}%
\pgfpathlineto{\pgfqpoint{7.850000in}{3.034983in}}%
\pgfusepath{stroke}%
\end{pgfscope}%
\begin{pgfscope}%
\definecolor{textcolor}{rgb}{0.150000,0.150000,0.150000}%
\pgfsetstrokecolor{textcolor}%
\pgfsetfillcolor{textcolor}%
\pgftext[x=0.752190in, y=2.929460in, left, base]{\color{textcolor}\rmfamily\fontsize{20.000000}{24.000000}\selectfont 0.0}%
\end{pgfscope}%
\begin{pgfscope}%
\pgfpathrectangle{\pgfqpoint{1.291172in}{0.898915in}}{\pgfqpoint{6.558828in}{4.114497in}}%
\pgfusepath{clip}%
\pgfsetroundcap%
\pgfsetroundjoin%
\pgfsetlinewidth{1.003750pt}%
\definecolor{currentstroke}{rgb}{1.000000,1.000000,1.000000}%
\pgfsetstrokecolor{currentstroke}%
\pgfsetdash{}{0pt}%
\pgfpathmoveto{\pgfqpoint{1.291172in}{3.665092in}}%
\pgfpathlineto{\pgfqpoint{7.850000in}{3.665092in}}%
\pgfusepath{stroke}%
\end{pgfscope}%
\begin{pgfscope}%
\definecolor{textcolor}{rgb}{0.150000,0.150000,0.150000}%
\pgfsetstrokecolor{textcolor}%
\pgfsetfillcolor{textcolor}%
\pgftext[x=0.752190in, y=3.559569in, left, base]{\color{textcolor}\rmfamily\fontsize{20.000000}{24.000000}\selectfont 2.5}%
\end{pgfscope}%
\begin{pgfscope}%
\pgfpathrectangle{\pgfqpoint{1.291172in}{0.898915in}}{\pgfqpoint{6.558828in}{4.114497in}}%
\pgfusepath{clip}%
\pgfsetroundcap%
\pgfsetroundjoin%
\pgfsetlinewidth{1.003750pt}%
\definecolor{currentstroke}{rgb}{1.000000,1.000000,1.000000}%
\pgfsetstrokecolor{currentstroke}%
\pgfsetdash{}{0pt}%
\pgfpathmoveto{\pgfqpoint{1.291172in}{4.295201in}}%
\pgfpathlineto{\pgfqpoint{7.850000in}{4.295201in}}%
\pgfusepath{stroke}%
\end{pgfscope}%
\begin{pgfscope}%
\definecolor{textcolor}{rgb}{0.150000,0.150000,0.150000}%
\pgfsetstrokecolor{textcolor}%
\pgfsetfillcolor{textcolor}%
\pgftext[x=0.752190in, y=4.189678in, left, base]{\color{textcolor}\rmfamily\fontsize{20.000000}{24.000000}\selectfont 5.0}%
\end{pgfscope}%
\begin{pgfscope}%
\pgfpathrectangle{\pgfqpoint{1.291172in}{0.898915in}}{\pgfqpoint{6.558828in}{4.114497in}}%
\pgfusepath{clip}%
\pgfsetroundcap%
\pgfsetroundjoin%
\pgfsetlinewidth{1.003750pt}%
\definecolor{currentstroke}{rgb}{1.000000,1.000000,1.000000}%
\pgfsetstrokecolor{currentstroke}%
\pgfsetdash{}{0pt}%
\pgfpathmoveto{\pgfqpoint{1.291172in}{4.925310in}}%
\pgfpathlineto{\pgfqpoint{7.850000in}{4.925310in}}%
\pgfusepath{stroke}%
\end{pgfscope}%
\begin{pgfscope}%
\definecolor{textcolor}{rgb}{0.150000,0.150000,0.150000}%
\pgfsetstrokecolor{textcolor}%
\pgfsetfillcolor{textcolor}%
\pgftext[x=0.752190in, y=4.819787in, left, base]{\color{textcolor}\rmfamily\fontsize{20.000000}{24.000000}\selectfont 7.5}%
\end{pgfscope}%
\begin{pgfscope}%
\definecolor{textcolor}{rgb}{0.150000,0.150000,0.150000}%
\pgfsetstrokecolor{textcolor}%
\pgfsetfillcolor{textcolor}%
\pgftext[x=0.472591in,y=2.956163in,,bottom,rotate=90.000000]{\color{textcolor}\rmfamily\fontsize{24.000000}{28.800000}\selectfont Pulse Amplitude (V)}%
\end{pgfscope}%
\begin{pgfscope}%
\pgfpathrectangle{\pgfqpoint{1.291172in}{0.898915in}}{\pgfqpoint{6.558828in}{4.114497in}}%
\pgfusepath{clip}%
\pgfsetbuttcap%
\pgfsetroundjoin%
\definecolor{currentfill}{rgb}{0.392157,0.584314,0.929412}%
\pgfsetfillcolor{currentfill}%
\pgfsetlinewidth{0.301125pt}%
\definecolor{currentstroke}{rgb}{0.392157,0.584314,0.929412}%
\pgfsetstrokecolor{currentstroke}%
\pgfsetdash{}{0pt}%
\pgfsys@defobject{currentmarker}{\pgfqpoint{-0.048611in}{-0.048611in}}{\pgfqpoint{0.048611in}{0.048611in}}{%
\pgfpathmoveto{\pgfqpoint{0.000000in}{-0.048611in}}%
\pgfpathcurveto{\pgfqpoint{0.012892in}{-0.048611in}}{\pgfqpoint{0.025257in}{-0.043489in}}{\pgfqpoint{0.034373in}{-0.034373in}}%
\pgfpathcurveto{\pgfqpoint{0.043489in}{-0.025257in}}{\pgfqpoint{0.048611in}{-0.012892in}}{\pgfqpoint{0.048611in}{0.000000in}}%
\pgfpathcurveto{\pgfqpoint{0.048611in}{0.012892in}}{\pgfqpoint{0.043489in}{0.025257in}}{\pgfqpoint{0.034373in}{0.034373in}}%
\pgfpathcurveto{\pgfqpoint{0.025257in}{0.043489in}}{\pgfqpoint{0.012892in}{0.048611in}}{\pgfqpoint{0.000000in}{0.048611in}}%
\pgfpathcurveto{\pgfqpoint{-0.012892in}{0.048611in}}{\pgfqpoint{-0.025257in}{0.043489in}}{\pgfqpoint{-0.034373in}{0.034373in}}%
\pgfpathcurveto{\pgfqpoint{-0.043489in}{0.025257in}}{\pgfqpoint{-0.048611in}{0.012892in}}{\pgfqpoint{-0.048611in}{0.000000in}}%
\pgfpathcurveto{\pgfqpoint{-0.048611in}{-0.012892in}}{\pgfqpoint{-0.043489in}{-0.025257in}}{\pgfqpoint{-0.034373in}{-0.034373in}}%
\pgfpathcurveto{\pgfqpoint{-0.025257in}{-0.043489in}}{\pgfqpoint{-0.012892in}{-0.048611in}}{\pgfqpoint{0.000000in}{-0.048611in}}%
\pgfpathlineto{\pgfqpoint{0.000000in}{-0.048611in}}%
\pgfpathclose%
\pgfusepath{stroke,fill}%
}%
\begin{pgfscope}%
\pgfsys@transformshift{1.589300in}{1.159779in}%
\pgfsys@useobject{currentmarker}{}%
\end{pgfscope}%
\begin{pgfscope}%
\pgfsys@transformshift{1.873232in}{1.563049in}%
\pgfsys@useobject{currentmarker}{}%
\end{pgfscope}%
\begin{pgfscope}%
\pgfsys@transformshift{2.441096in}{2.167953in}%
\pgfsys@useobject{currentmarker}{}%
\end{pgfscope}%
\begin{pgfscope}%
\pgfsys@transformshift{3.008960in}{2.833348in}%
\pgfsys@useobject{currentmarker}{}%
\end{pgfscope}%
\begin{pgfscope}%
\pgfsys@transformshift{3.576824in}{3.518907in}%
\pgfsys@useobject{currentmarker}{}%
\end{pgfscope}%
\begin{pgfscope}%
\pgfsys@transformshift{4.144688in}{3.881849in}%
\pgfsys@useobject{currentmarker}{}%
\end{pgfscope}%
\begin{pgfscope}%
\pgfsys@transformshift{4.712552in}{4.103648in}%
\pgfsys@useobject{currentmarker}{}%
\end{pgfscope}%
\begin{pgfscope}%
\pgfsys@transformshift{5.280416in}{4.325446in}%
\pgfsys@useobject{currentmarker}{}%
\end{pgfscope}%
\begin{pgfscope}%
\pgfsys@transformshift{5.848280in}{4.486754in}%
\pgfsys@useobject{currentmarker}{}%
\end{pgfscope}%
\begin{pgfscope}%
\pgfsys@transformshift{6.416144in}{4.607735in}%
\pgfsys@useobject{currentmarker}{}%
\end{pgfscope}%
\begin{pgfscope}%
\pgfsys@transformshift{6.984007in}{4.728716in}%
\pgfsys@useobject{currentmarker}{}%
\end{pgfscope}%
\begin{pgfscope}%
\pgfsys@transformshift{7.551871in}{4.789206in}%
\pgfsys@useobject{currentmarker}{}%
\end{pgfscope}%
\end{pgfscope}%
\begin{pgfscope}%
\pgfpathrectangle{\pgfqpoint{1.291172in}{0.898915in}}{\pgfqpoint{6.558828in}{4.114497in}}%
\pgfusepath{clip}%
\pgfsetroundcap%
\pgfsetroundjoin%
\pgfsetlinewidth{1.756562pt}%
\definecolor{currentstroke}{rgb}{0.298039,0.447059,0.690196}%
\pgfsetstrokecolor{currentstroke}%
\pgfsetdash{}{0pt}%
\pgfpathmoveto{\pgfqpoint{1.589300in}{1.085937in}}%
\pgfpathlineto{\pgfqpoint{1.649528in}{1.189342in}}%
\pgfpathlineto{\pgfqpoint{1.709756in}{1.290109in}}%
\pgfpathlineto{\pgfqpoint{1.769984in}{1.388307in}}%
\pgfpathlineto{\pgfqpoint{1.830212in}{1.484000in}}%
\pgfpathlineto{\pgfqpoint{1.890440in}{1.577253in}}%
\pgfpathlineto{\pgfqpoint{1.950668in}{1.668128in}}%
\pgfpathlineto{\pgfqpoint{2.010896in}{1.756685in}}%
\pgfpathlineto{\pgfqpoint{2.071124in}{1.842984in}}%
\pgfpathlineto{\pgfqpoint{2.131352in}{1.927082in}}%
\pgfpathlineto{\pgfqpoint{2.191580in}{2.009035in}}%
\pgfpathlineto{\pgfqpoint{2.251808in}{2.088898in}}%
\pgfpathlineto{\pgfqpoint{2.312036in}{2.166725in}}%
\pgfpathlineto{\pgfqpoint{2.372264in}{2.242566in}}%
\pgfpathlineto{\pgfqpoint{2.432492in}{2.316474in}}%
\pgfpathlineto{\pgfqpoint{2.492720in}{2.388497in}}%
\pgfpathlineto{\pgfqpoint{2.552948in}{2.458683in}}%
\pgfpathlineto{\pgfqpoint{2.613176in}{2.527079in}}%
\pgfpathlineto{\pgfqpoint{2.673404in}{2.593731in}}%
\pgfpathlineto{\pgfqpoint{2.733632in}{2.658683in}}%
\pgfpathlineto{\pgfqpoint{2.793860in}{2.721978in}}%
\pgfpathlineto{\pgfqpoint{2.854088in}{2.783660in}}%
\pgfpathlineto{\pgfqpoint{2.914316in}{2.843768in}}%
\pgfpathlineto{\pgfqpoint{2.974544in}{2.902343in}}%
\pgfpathlineto{\pgfqpoint{3.034772in}{2.959425in}}%
\pgfpathlineto{\pgfqpoint{3.095000in}{3.015051in}}%
\pgfpathlineto{\pgfqpoint{3.155228in}{3.069258in}}%
\pgfpathlineto{\pgfqpoint{3.215456in}{3.122083in}}%
\pgfpathlineto{\pgfqpoint{3.275684in}{3.173561in}}%
\pgfpathlineto{\pgfqpoint{3.335912in}{3.223726in}}%
\pgfpathlineto{\pgfqpoint{3.396140in}{3.272611in}}%
\pgfpathlineto{\pgfqpoint{3.456368in}{3.320250in}}%
\pgfpathlineto{\pgfqpoint{3.516596in}{3.366674in}}%
\pgfpathlineto{\pgfqpoint{3.576824in}{3.411914in}}%
\pgfpathlineto{\pgfqpoint{3.637052in}{3.456000in}}%
\pgfpathlineto{\pgfqpoint{3.697280in}{3.498962in}}%
\pgfpathlineto{\pgfqpoint{3.757508in}{3.540829in}}%
\pgfpathlineto{\pgfqpoint{3.817736in}{3.581627in}}%
\pgfpathlineto{\pgfqpoint{3.877964in}{3.621386in}}%
\pgfpathlineto{\pgfqpoint{3.938192in}{3.660130in}}%
\pgfpathlineto{\pgfqpoint{3.998420in}{3.697886in}}%
\pgfpathlineto{\pgfqpoint{4.058648in}{3.734679in}}%
\pgfpathlineto{\pgfqpoint{4.118876in}{3.770534in}}%
\pgfpathlineto{\pgfqpoint{4.179104in}{3.805475in}}%
\pgfpathlineto{\pgfqpoint{4.239332in}{3.839525in}}%
\pgfpathlineto{\pgfqpoint{4.299560in}{3.872706in}}%
\pgfpathlineto{\pgfqpoint{4.359788in}{3.905041in}}%
\pgfpathlineto{\pgfqpoint{4.420016in}{3.936551in}}%
\pgfpathlineto{\pgfqpoint{4.480244in}{3.967258in}}%
\pgfpathlineto{\pgfqpoint{4.540472in}{3.997181in}}%
\pgfpathlineto{\pgfqpoint{4.600700in}{4.026342in}}%
\pgfpathlineto{\pgfqpoint{4.660928in}{4.054759in}}%
\pgfpathlineto{\pgfqpoint{4.721156in}{4.082451in}}%
\pgfpathlineto{\pgfqpoint{4.781384in}{4.109437in}}%
\pgfpathlineto{\pgfqpoint{4.841612in}{4.135735in}}%
\pgfpathlineto{\pgfqpoint{4.901840in}{4.161362in}}%
\pgfpathlineto{\pgfqpoint{4.962068in}{4.186335in}}%
\pgfpathlineto{\pgfqpoint{5.022296in}{4.210672in}}%
\pgfpathlineto{\pgfqpoint{5.082524in}{4.234388in}}%
\pgfpathlineto{\pgfqpoint{5.142752in}{4.257499in}}%
\pgfpathlineto{\pgfqpoint{5.202980in}{4.280021in}}%
\pgfpathlineto{\pgfqpoint{5.263208in}{4.301968in}}%
\pgfpathlineto{\pgfqpoint{5.323436in}{4.323356in}}%
\pgfpathlineto{\pgfqpoint{5.383664in}{4.344198in}}%
\pgfpathlineto{\pgfqpoint{5.443892in}{4.364509in}}%
\pgfpathlineto{\pgfqpoint{5.504120in}{4.384302in}}%
\pgfpathlineto{\pgfqpoint{5.564348in}{4.403590in}}%
\pgfpathlineto{\pgfqpoint{5.624576in}{4.422386in}}%
\pgfpathlineto{\pgfqpoint{5.684804in}{4.440703in}}%
\pgfpathlineto{\pgfqpoint{5.745032in}{4.458552in}}%
\pgfpathlineto{\pgfqpoint{5.805260in}{4.475947in}}%
\pgfpathlineto{\pgfqpoint{5.865488in}{4.492898in}}%
\pgfpathlineto{\pgfqpoint{5.925716in}{4.509416in}}%
\pgfpathlineto{\pgfqpoint{5.985944in}{4.525513in}}%
\pgfpathlineto{\pgfqpoint{6.046172in}{4.541200in}}%
\pgfpathlineto{\pgfqpoint{6.106400in}{4.556487in}}%
\pgfpathlineto{\pgfqpoint{6.166628in}{4.571384in}}%
\pgfpathlineto{\pgfqpoint{6.226856in}{4.585901in}}%
\pgfpathlineto{\pgfqpoint{6.287084in}{4.600047in}}%
\pgfpathlineto{\pgfqpoint{6.347312in}{4.613833in}}%
\pgfpathlineto{\pgfqpoint{6.407540in}{4.627268in}}%
\pgfpathlineto{\pgfqpoint{6.467768in}{4.640360in}}%
\pgfpathlineto{\pgfqpoint{6.527996in}{4.653117in}}%
\pgfpathlineto{\pgfqpoint{6.588224in}{4.665550in}}%
\pgfpathlineto{\pgfqpoint{6.648452in}{4.677665in}}%
\pgfpathlineto{\pgfqpoint{6.708680in}{4.689472in}}%
\pgfpathlineto{\pgfqpoint{6.768908in}{4.700977in}}%
\pgfpathlineto{\pgfqpoint{6.829136in}{4.712189in}}%
\pgfpathlineto{\pgfqpoint{6.889364in}{4.723115in}}%
\pgfpathlineto{\pgfqpoint{6.949591in}{4.733763in}}%
\pgfpathlineto{\pgfqpoint{7.009819in}{4.744139in}}%
\pgfpathlineto{\pgfqpoint{7.070047in}{4.754250in}}%
\pgfpathlineto{\pgfqpoint{7.130275in}{4.764103in}}%
\pgfpathlineto{\pgfqpoint{7.190503in}{4.773705in}}%
\pgfpathlineto{\pgfqpoint{7.250731in}{4.783063in}}%
\pgfpathlineto{\pgfqpoint{7.310959in}{4.792181in}}%
\pgfpathlineto{\pgfqpoint{7.371187in}{4.801067in}}%
\pgfpathlineto{\pgfqpoint{7.431415in}{4.809727in}}%
\pgfpathlineto{\pgfqpoint{7.491643in}{4.818165in}}%
\pgfpathlineto{\pgfqpoint{7.551871in}{4.826389in}}%
\pgfusepath{stroke}%
\end{pgfscope}%
\begin{pgfscope}%
\pgfsetrectcap%
\pgfsetmiterjoin%
\pgfsetlinewidth{0.000000pt}%
\definecolor{currentstroke}{rgb}{1.000000,1.000000,1.000000}%
\pgfsetstrokecolor{currentstroke}%
\pgfsetdash{}{0pt}%
\pgfpathmoveto{\pgfqpoint{1.291172in}{0.898915in}}%
\pgfpathlineto{\pgfqpoint{1.291172in}{5.013411in}}%
\pgfusepath{}%
\end{pgfscope}%
\begin{pgfscope}%
\pgfsetrectcap%
\pgfsetmiterjoin%
\pgfsetlinewidth{0.000000pt}%
\definecolor{currentstroke}{rgb}{1.000000,1.000000,1.000000}%
\pgfsetstrokecolor{currentstroke}%
\pgfsetdash{}{0pt}%
\pgfpathmoveto{\pgfqpoint{7.850000in}{0.898915in}}%
\pgfpathlineto{\pgfqpoint{7.850000in}{5.013411in}}%
\pgfusepath{}%
\end{pgfscope}%
\begin{pgfscope}%
\pgfsetrectcap%
\pgfsetmiterjoin%
\pgfsetlinewidth{0.000000pt}%
\definecolor{currentstroke}{rgb}{1.000000,1.000000,1.000000}%
\pgfsetstrokecolor{currentstroke}%
\pgfsetdash{}{0pt}%
\pgfpathmoveto{\pgfqpoint{1.291172in}{0.898915in}}%
\pgfpathlineto{\pgfqpoint{7.850000in}{0.898915in}}%
\pgfusepath{}%
\end{pgfscope}%
\begin{pgfscope}%
\pgfsetrectcap%
\pgfsetmiterjoin%
\pgfsetlinewidth{0.000000pt}%
\definecolor{currentstroke}{rgb}{1.000000,1.000000,1.000000}%
\pgfsetstrokecolor{currentstroke}%
\pgfsetdash{}{0pt}%
\pgfpathmoveto{\pgfqpoint{1.291172in}{5.013411in}}%
\pgfpathlineto{\pgfqpoint{7.850000in}{5.013411in}}%
\pgfusepath{}%
\end{pgfscope}%
\begin{pgfscope}%
\definecolor{textcolor}{rgb}{0.150000,0.150000,0.150000}%
\pgfsetstrokecolor{textcolor}%
\pgfsetfillcolor{textcolor}%
\pgftext[x=2.725028in, y=1.764734in, left, base]{\color{textcolor}\rmfamily\fontsize{20.000000}{24.000000}\selectfont M\(\displaystyle _z\)(t) = 9.09(1\(\displaystyle -\)2\(\displaystyle exp[\frac{-t}{T_1}]\))\(\displaystyle +\)-0.73}%
\end{pgfscope}%
\begin{pgfscope}%
\definecolor{textcolor}{rgb}{0.150000,0.150000,0.150000}%
\pgfsetstrokecolor{textcolor}%
\pgfsetfillcolor{textcolor}%
\pgftext[x=2.725028in, y=1.270678in, left, base]{\color{textcolor}\rmfamily\fontsize{20.000000}{24.000000}\selectfont T\(\displaystyle _1\) = 8.211\(\displaystyle \pm\)0.58 ms}%
\end{pgfscope}%
\begin{pgfscope}%
\definecolor{textcolor}{rgb}{0.150000,0.150000,0.150000}%
\pgfsetstrokecolor{textcolor}%
\pgfsetfillcolor{textcolor}%
\pgftext[x=4.570586in,y=5.096745in,,base]{\color{textcolor}\rmfamily\fontsize{24.000000}{28.800000}\selectfont T\(\displaystyle _1\) measurement of 0.1 M CuSO\(\displaystyle _4\)}%
\end{pgfscope}%
\begin{pgfscope}%
\pgfsetbuttcap%
\pgfsetroundjoin%
\definecolor{currentfill}{rgb}{0.392157,0.584314,0.929412}%
\pgfsetfillcolor{currentfill}%
\pgfsetlinewidth{0.301125pt}%
\definecolor{currentstroke}{rgb}{0.392157,0.584314,0.929412}%
\pgfsetstrokecolor{currentstroke}%
\pgfsetdash{}{0pt}%
\pgfsys@defobject{currentmarker}{\pgfqpoint{-0.048611in}{-0.048611in}}{\pgfqpoint{0.048611in}{0.048611in}}{%
\pgfpathmoveto{\pgfqpoint{0.000000in}{-0.048611in}}%
\pgfpathcurveto{\pgfqpoint{0.012892in}{-0.048611in}}{\pgfqpoint{0.025257in}{-0.043489in}}{\pgfqpoint{0.034373in}{-0.034373in}}%
\pgfpathcurveto{\pgfqpoint{0.043489in}{-0.025257in}}{\pgfqpoint{0.048611in}{-0.012892in}}{\pgfqpoint{0.048611in}{0.000000in}}%
\pgfpathcurveto{\pgfqpoint{0.048611in}{0.012892in}}{\pgfqpoint{0.043489in}{0.025257in}}{\pgfqpoint{0.034373in}{0.034373in}}%
\pgfpathcurveto{\pgfqpoint{0.025257in}{0.043489in}}{\pgfqpoint{0.012892in}{0.048611in}}{\pgfqpoint{0.000000in}{0.048611in}}%
\pgfpathcurveto{\pgfqpoint{-0.012892in}{0.048611in}}{\pgfqpoint{-0.025257in}{0.043489in}}{\pgfqpoint{-0.034373in}{0.034373in}}%
\pgfpathcurveto{\pgfqpoint{-0.043489in}{0.025257in}}{\pgfqpoint{-0.048611in}{0.012892in}}{\pgfqpoint{-0.048611in}{0.000000in}}%
\pgfpathcurveto{\pgfqpoint{-0.048611in}{-0.012892in}}{\pgfqpoint{-0.043489in}{-0.025257in}}{\pgfqpoint{-0.034373in}{-0.034373in}}%
\pgfpathcurveto{\pgfqpoint{-0.025257in}{-0.043489in}}{\pgfqpoint{-0.012892in}{-0.048611in}}{\pgfqpoint{0.000000in}{-0.048611in}}%
\pgfpathlineto{\pgfqpoint{0.000000in}{-0.048611in}}%
\pgfpathclose%
\pgfusepath{stroke,fill}%
}%
\begin{pgfscope}%
\pgfsys@transformshift{1.818949in}{4.625282in}%
\pgfsys@useobject{currentmarker}{}%
\end{pgfscope}%
\end{pgfscope}%
\begin{pgfscope}%
\definecolor{textcolor}{rgb}{0.150000,0.150000,0.150000}%
\pgfsetstrokecolor{textcolor}%
\pgfsetfillcolor{textcolor}%
\pgftext[x=2.318949in,y=4.552365in,left,base]{\color{textcolor}\rmfamily\fontsize{20.000000}{24.000000}\selectfont Data}%
\end{pgfscope}%
\begin{pgfscope}%
\pgfsetroundcap%
\pgfsetroundjoin%
\pgfsetlinewidth{1.756562pt}%
\definecolor{currentstroke}{rgb}{0.298039,0.447059,0.690196}%
\pgfsetstrokecolor{currentstroke}%
\pgfsetdash{}{0pt}%
\pgfpathmoveto{\pgfqpoint{1.541172in}{4.241873in}}%
\pgfpathlineto{\pgfqpoint{1.818949in}{4.241873in}}%
\pgfpathlineto{\pgfqpoint{2.096727in}{4.241873in}}%
\pgfusepath{stroke}%
\end{pgfscope}%
\begin{pgfscope}%
\definecolor{textcolor}{rgb}{0.150000,0.150000,0.150000}%
\pgfsetstrokecolor{textcolor}%
\pgfsetfillcolor{textcolor}%
\pgftext[x=2.318949in,y=4.144651in,left,base]{\color{textcolor}\rmfamily\fontsize{20.000000}{24.000000}\selectfont Fitted Curve}%
\end{pgfscope}%
\end{pgfpicture}%
\makeatother%
\endgroup%
}
    \caption{T1 measurement of 0.1 M  CuSO$_{4} \boldsymbol{\cdot} $H$_2$O using the two-pulse method }
    \label{fig:0d1_t1}
  \end{figure}

  \begin{figure}[h]
    \resizebox{0.45\textwidth}{!}{%% Creator: Matplotlib, PGF backend
%%
%% To include the figure in your LaTeX document, write
%%   \input{<filename>.pgf}
%%
%% Make sure the required packages are loaded in your preamble
%%   \usepackage{pgf}
%%
%% Also ensure that all the required font packages are loaded; for instance,
%% the lmodern package is sometimes necessary when using math font.
%%   \usepackage{lmodern}
%%
%% Figures using additional raster images can only be included by \input if
%% they are in the same directory as the main LaTeX file. For loading figures
%% from other directories you can use the `import` package
%%   \usepackage{import}
%%
%% and then include the figures with
%%   \import{<path to file>}{<filename>.pgf}
%%
%% Matplotlib used the following preamble
%%   \usepackage{fontspec}
%%   \setmainfont{DejaVuSerif.ttf}[Path=\detokenize{/usr/share/matplotlib/mpl-data/fonts/ttf/}]
%%   \setsansfont{DejaVuSans.ttf}[Path=\detokenize{/usr/share/matplotlib/mpl-data/fonts/ttf/}]
%%   \setmonofont{DejaVuSansMono.ttf}[Path=\detokenize{/usr/share/matplotlib/mpl-data/fonts/ttf/}]
%%
\begingroup%
\makeatletter%
\begin{pgfpicture}%
\pgfpathrectangle{\pgfpointorigin}{\pgfqpoint{8.000000in}{5.500000in}}%
\pgfusepath{use as bounding box, clip}%
\begin{pgfscope}%
\pgfsetbuttcap%
\pgfsetmiterjoin%
\definecolor{currentfill}{rgb}{1.000000,1.000000,1.000000}%
\pgfsetfillcolor{currentfill}%
\pgfsetlinewidth{0.000000pt}%
\definecolor{currentstroke}{rgb}{1.000000,1.000000,1.000000}%
\pgfsetstrokecolor{currentstroke}%
\pgfsetdash{}{0pt}%
\pgfpathmoveto{\pgfqpoint{0.000000in}{0.000000in}}%
\pgfpathlineto{\pgfqpoint{8.000000in}{0.000000in}}%
\pgfpathlineto{\pgfqpoint{8.000000in}{5.500000in}}%
\pgfpathlineto{\pgfqpoint{0.000000in}{5.500000in}}%
\pgfpathlineto{\pgfqpoint{0.000000in}{0.000000in}}%
\pgfpathclose%
\pgfusepath{fill}%
\end{pgfscope}%
\begin{pgfscope}%
\pgfsetbuttcap%
\pgfsetmiterjoin%
\definecolor{currentfill}{rgb}{0.917647,0.917647,0.949020}%
\pgfsetfillcolor{currentfill}%
\pgfsetlinewidth{0.000000pt}%
\definecolor{currentstroke}{rgb}{0.000000,0.000000,0.000000}%
\pgfsetstrokecolor{currentstroke}%
\pgfsetstrokeopacity{0.000000}%
\pgfsetdash{}{0pt}%
\pgfpathmoveto{\pgfqpoint{1.026143in}{0.898915in}}%
\pgfpathlineto{\pgfqpoint{7.833976in}{0.898915in}}%
\pgfpathlineto{\pgfqpoint{7.833976in}{5.013411in}}%
\pgfpathlineto{\pgfqpoint{1.026143in}{5.013411in}}%
\pgfpathlineto{\pgfqpoint{1.026143in}{0.898915in}}%
\pgfpathclose%
\pgfusepath{fill}%
\end{pgfscope}%
\begin{pgfscope}%
\pgfpathrectangle{\pgfqpoint{1.026143in}{0.898915in}}{\pgfqpoint{6.807833in}{4.114497in}}%
\pgfusepath{clip}%
\pgfsetroundcap%
\pgfsetroundjoin%
\pgfsetlinewidth{1.003750pt}%
\definecolor{currentstroke}{rgb}{1.000000,1.000000,1.000000}%
\pgfsetstrokecolor{currentstroke}%
\pgfsetdash{}{0pt}%
\pgfpathmoveto{\pgfqpoint{1.898221in}{0.898915in}}%
\pgfpathlineto{\pgfqpoint{1.898221in}{5.013411in}}%
\pgfusepath{stroke}%
\end{pgfscope}%
\begin{pgfscope}%
\definecolor{textcolor}{rgb}{0.150000,0.150000,0.150000}%
\pgfsetstrokecolor{textcolor}%
\pgfsetfillcolor{textcolor}%
\pgftext[x=1.898221in,y=0.801692in,,top]{\color{textcolor}\rmfamily\fontsize{20.000000}{24.000000}\selectfont 2}%
\end{pgfscope}%
\begin{pgfscope}%
\pgfpathrectangle{\pgfqpoint{1.026143in}{0.898915in}}{\pgfqpoint{6.807833in}{4.114497in}}%
\pgfusepath{clip}%
\pgfsetroundcap%
\pgfsetroundjoin%
\pgfsetlinewidth{1.003750pt}%
\definecolor{currentstroke}{rgb}{1.000000,1.000000,1.000000}%
\pgfsetstrokecolor{currentstroke}%
\pgfsetdash{}{0pt}%
\pgfpathmoveto{\pgfqpoint{3.023483in}{0.898915in}}%
\pgfpathlineto{\pgfqpoint{3.023483in}{5.013411in}}%
\pgfusepath{stroke}%
\end{pgfscope}%
\begin{pgfscope}%
\definecolor{textcolor}{rgb}{0.150000,0.150000,0.150000}%
\pgfsetstrokecolor{textcolor}%
\pgfsetfillcolor{textcolor}%
\pgftext[x=3.023483in,y=0.801692in,,top]{\color{textcolor}\rmfamily\fontsize{20.000000}{24.000000}\selectfont 4}%
\end{pgfscope}%
\begin{pgfscope}%
\pgfpathrectangle{\pgfqpoint{1.026143in}{0.898915in}}{\pgfqpoint{6.807833in}{4.114497in}}%
\pgfusepath{clip}%
\pgfsetroundcap%
\pgfsetroundjoin%
\pgfsetlinewidth{1.003750pt}%
\definecolor{currentstroke}{rgb}{1.000000,1.000000,1.000000}%
\pgfsetstrokecolor{currentstroke}%
\pgfsetdash{}{0pt}%
\pgfpathmoveto{\pgfqpoint{4.148744in}{0.898915in}}%
\pgfpathlineto{\pgfqpoint{4.148744in}{5.013411in}}%
\pgfusepath{stroke}%
\end{pgfscope}%
\begin{pgfscope}%
\definecolor{textcolor}{rgb}{0.150000,0.150000,0.150000}%
\pgfsetstrokecolor{textcolor}%
\pgfsetfillcolor{textcolor}%
\pgftext[x=4.148744in,y=0.801692in,,top]{\color{textcolor}\rmfamily\fontsize{20.000000}{24.000000}\selectfont 6}%
\end{pgfscope}%
\begin{pgfscope}%
\pgfpathrectangle{\pgfqpoint{1.026143in}{0.898915in}}{\pgfqpoint{6.807833in}{4.114497in}}%
\pgfusepath{clip}%
\pgfsetroundcap%
\pgfsetroundjoin%
\pgfsetlinewidth{1.003750pt}%
\definecolor{currentstroke}{rgb}{1.000000,1.000000,1.000000}%
\pgfsetstrokecolor{currentstroke}%
\pgfsetdash{}{0pt}%
\pgfpathmoveto{\pgfqpoint{5.274006in}{0.898915in}}%
\pgfpathlineto{\pgfqpoint{5.274006in}{5.013411in}}%
\pgfusepath{stroke}%
\end{pgfscope}%
\begin{pgfscope}%
\definecolor{textcolor}{rgb}{0.150000,0.150000,0.150000}%
\pgfsetstrokecolor{textcolor}%
\pgfsetfillcolor{textcolor}%
\pgftext[x=5.274006in,y=0.801692in,,top]{\color{textcolor}\rmfamily\fontsize{20.000000}{24.000000}\selectfont 8}%
\end{pgfscope}%
\begin{pgfscope}%
\pgfpathrectangle{\pgfqpoint{1.026143in}{0.898915in}}{\pgfqpoint{6.807833in}{4.114497in}}%
\pgfusepath{clip}%
\pgfsetroundcap%
\pgfsetroundjoin%
\pgfsetlinewidth{1.003750pt}%
\definecolor{currentstroke}{rgb}{1.000000,1.000000,1.000000}%
\pgfsetstrokecolor{currentstroke}%
\pgfsetdash{}{0pt}%
\pgfpathmoveto{\pgfqpoint{6.399267in}{0.898915in}}%
\pgfpathlineto{\pgfqpoint{6.399267in}{5.013411in}}%
\pgfusepath{stroke}%
\end{pgfscope}%
\begin{pgfscope}%
\definecolor{textcolor}{rgb}{0.150000,0.150000,0.150000}%
\pgfsetstrokecolor{textcolor}%
\pgfsetfillcolor{textcolor}%
\pgftext[x=6.399267in,y=0.801692in,,top]{\color{textcolor}\rmfamily\fontsize{20.000000}{24.000000}\selectfont 10}%
\end{pgfscope}%
\begin{pgfscope}%
\pgfpathrectangle{\pgfqpoint{1.026143in}{0.898915in}}{\pgfqpoint{6.807833in}{4.114497in}}%
\pgfusepath{clip}%
\pgfsetroundcap%
\pgfsetroundjoin%
\pgfsetlinewidth{1.003750pt}%
\definecolor{currentstroke}{rgb}{1.000000,1.000000,1.000000}%
\pgfsetstrokecolor{currentstroke}%
\pgfsetdash{}{0pt}%
\pgfpathmoveto{\pgfqpoint{7.524529in}{0.898915in}}%
\pgfpathlineto{\pgfqpoint{7.524529in}{5.013411in}}%
\pgfusepath{stroke}%
\end{pgfscope}%
\begin{pgfscope}%
\definecolor{textcolor}{rgb}{0.150000,0.150000,0.150000}%
\pgfsetstrokecolor{textcolor}%
\pgfsetfillcolor{textcolor}%
\pgftext[x=7.524529in,y=0.801692in,,top]{\color{textcolor}\rmfamily\fontsize{20.000000}{24.000000}\selectfont 12}%
\end{pgfscope}%
\begin{pgfscope}%
\definecolor{textcolor}{rgb}{0.150000,0.150000,0.150000}%
\pgfsetstrokecolor{textcolor}%
\pgfsetfillcolor{textcolor}%
\pgftext[x=4.430060in,y=0.477311in,,top]{\color{textcolor}\rmfamily\fontsize{24.000000}{28.800000}\selectfont Delay Time (ms)}%
\end{pgfscope}%
\begin{pgfscope}%
\pgfpathrectangle{\pgfqpoint{1.026143in}{0.898915in}}{\pgfqpoint{6.807833in}{4.114497in}}%
\pgfusepath{clip}%
\pgfsetroundcap%
\pgfsetroundjoin%
\pgfsetlinewidth{1.003750pt}%
\definecolor{currentstroke}{rgb}{1.000000,1.000000,1.000000}%
\pgfsetstrokecolor{currentstroke}%
\pgfsetdash{}{0pt}%
\pgfpathmoveto{\pgfqpoint{1.026143in}{1.095485in}}%
\pgfpathlineto{\pgfqpoint{7.833976in}{1.095485in}}%
\pgfusepath{stroke}%
\end{pgfscope}%
\begin{pgfscope}%
\definecolor{textcolor}{rgb}{0.150000,0.150000,0.150000}%
\pgfsetstrokecolor{textcolor}%
\pgfsetfillcolor{textcolor}%
\pgftext[x=0.528147in, y=0.989962in, left, base]{\color{textcolor}\rmfamily\fontsize{20.000000}{24.000000}\selectfont \ensuremath{-}6}%
\end{pgfscope}%
\begin{pgfscope}%
\pgfpathrectangle{\pgfqpoint{1.026143in}{0.898915in}}{\pgfqpoint{6.807833in}{4.114497in}}%
\pgfusepath{clip}%
\pgfsetroundcap%
\pgfsetroundjoin%
\pgfsetlinewidth{1.003750pt}%
\definecolor{currentstroke}{rgb}{1.000000,1.000000,1.000000}%
\pgfsetstrokecolor{currentstroke}%
\pgfsetdash{}{0pt}%
\pgfpathmoveto{\pgfqpoint{1.026143in}{1.696417in}}%
\pgfpathlineto{\pgfqpoint{7.833976in}{1.696417in}}%
\pgfusepath{stroke}%
\end{pgfscope}%
\begin{pgfscope}%
\definecolor{textcolor}{rgb}{0.150000,0.150000,0.150000}%
\pgfsetstrokecolor{textcolor}%
\pgfsetfillcolor{textcolor}%
\pgftext[x=0.528147in, y=1.590894in, left, base]{\color{textcolor}\rmfamily\fontsize{20.000000}{24.000000}\selectfont \ensuremath{-}4}%
\end{pgfscope}%
\begin{pgfscope}%
\pgfpathrectangle{\pgfqpoint{1.026143in}{0.898915in}}{\pgfqpoint{6.807833in}{4.114497in}}%
\pgfusepath{clip}%
\pgfsetroundcap%
\pgfsetroundjoin%
\pgfsetlinewidth{1.003750pt}%
\definecolor{currentstroke}{rgb}{1.000000,1.000000,1.000000}%
\pgfsetstrokecolor{currentstroke}%
\pgfsetdash{}{0pt}%
\pgfpathmoveto{\pgfqpoint{1.026143in}{2.297349in}}%
\pgfpathlineto{\pgfqpoint{7.833976in}{2.297349in}}%
\pgfusepath{stroke}%
\end{pgfscope}%
\begin{pgfscope}%
\definecolor{textcolor}{rgb}{0.150000,0.150000,0.150000}%
\pgfsetstrokecolor{textcolor}%
\pgfsetfillcolor{textcolor}%
\pgftext[x=0.528147in, y=2.191826in, left, base]{\color{textcolor}\rmfamily\fontsize{20.000000}{24.000000}\selectfont \ensuremath{-}2}%
\end{pgfscope}%
\begin{pgfscope}%
\pgfpathrectangle{\pgfqpoint{1.026143in}{0.898915in}}{\pgfqpoint{6.807833in}{4.114497in}}%
\pgfusepath{clip}%
\pgfsetroundcap%
\pgfsetroundjoin%
\pgfsetlinewidth{1.003750pt}%
\definecolor{currentstroke}{rgb}{1.000000,1.000000,1.000000}%
\pgfsetstrokecolor{currentstroke}%
\pgfsetdash{}{0pt}%
\pgfpathmoveto{\pgfqpoint{1.026143in}{2.898281in}}%
\pgfpathlineto{\pgfqpoint{7.833976in}{2.898281in}}%
\pgfusepath{stroke}%
\end{pgfscope}%
\begin{pgfscope}%
\definecolor{textcolor}{rgb}{0.150000,0.150000,0.150000}%
\pgfsetstrokecolor{textcolor}%
\pgfsetfillcolor{textcolor}%
\pgftext[x=0.752190in, y=2.792758in, left, base]{\color{textcolor}\rmfamily\fontsize{20.000000}{24.000000}\selectfont 0}%
\end{pgfscope}%
\begin{pgfscope}%
\pgfpathrectangle{\pgfqpoint{1.026143in}{0.898915in}}{\pgfqpoint{6.807833in}{4.114497in}}%
\pgfusepath{clip}%
\pgfsetroundcap%
\pgfsetroundjoin%
\pgfsetlinewidth{1.003750pt}%
\definecolor{currentstroke}{rgb}{1.000000,1.000000,1.000000}%
\pgfsetstrokecolor{currentstroke}%
\pgfsetdash{}{0pt}%
\pgfpathmoveto{\pgfqpoint{1.026143in}{3.499213in}}%
\pgfpathlineto{\pgfqpoint{7.833976in}{3.499213in}}%
\pgfusepath{stroke}%
\end{pgfscope}%
\begin{pgfscope}%
\definecolor{textcolor}{rgb}{0.150000,0.150000,0.150000}%
\pgfsetstrokecolor{textcolor}%
\pgfsetfillcolor{textcolor}%
\pgftext[x=0.752190in, y=3.393690in, left, base]{\color{textcolor}\rmfamily\fontsize{20.000000}{24.000000}\selectfont 2}%
\end{pgfscope}%
\begin{pgfscope}%
\pgfpathrectangle{\pgfqpoint{1.026143in}{0.898915in}}{\pgfqpoint{6.807833in}{4.114497in}}%
\pgfusepath{clip}%
\pgfsetroundcap%
\pgfsetroundjoin%
\pgfsetlinewidth{1.003750pt}%
\definecolor{currentstroke}{rgb}{1.000000,1.000000,1.000000}%
\pgfsetstrokecolor{currentstroke}%
\pgfsetdash{}{0pt}%
\pgfpathmoveto{\pgfqpoint{1.026143in}{4.100145in}}%
\pgfpathlineto{\pgfqpoint{7.833976in}{4.100145in}}%
\pgfusepath{stroke}%
\end{pgfscope}%
\begin{pgfscope}%
\definecolor{textcolor}{rgb}{0.150000,0.150000,0.150000}%
\pgfsetstrokecolor{textcolor}%
\pgfsetfillcolor{textcolor}%
\pgftext[x=0.752190in, y=3.994622in, left, base]{\color{textcolor}\rmfamily\fontsize{20.000000}{24.000000}\selectfont 4}%
\end{pgfscope}%
\begin{pgfscope}%
\pgfpathrectangle{\pgfqpoint{1.026143in}{0.898915in}}{\pgfqpoint{6.807833in}{4.114497in}}%
\pgfusepath{clip}%
\pgfsetroundcap%
\pgfsetroundjoin%
\pgfsetlinewidth{1.003750pt}%
\definecolor{currentstroke}{rgb}{1.000000,1.000000,1.000000}%
\pgfsetstrokecolor{currentstroke}%
\pgfsetdash{}{0pt}%
\pgfpathmoveto{\pgfqpoint{1.026143in}{4.701077in}}%
\pgfpathlineto{\pgfqpoint{7.833976in}{4.701077in}}%
\pgfusepath{stroke}%
\end{pgfscope}%
\begin{pgfscope}%
\definecolor{textcolor}{rgb}{0.150000,0.150000,0.150000}%
\pgfsetstrokecolor{textcolor}%
\pgfsetfillcolor{textcolor}%
\pgftext[x=0.752190in, y=4.595554in, left, base]{\color{textcolor}\rmfamily\fontsize{20.000000}{24.000000}\selectfont 6}%
\end{pgfscope}%
\begin{pgfscope}%
\definecolor{textcolor}{rgb}{0.150000,0.150000,0.150000}%
\pgfsetstrokecolor{textcolor}%
\pgfsetfillcolor{textcolor}%
\pgftext[x=0.472591in,y=2.956163in,,bottom,rotate=90.000000]{\color{textcolor}\rmfamily\fontsize{24.000000}{28.800000}\selectfont Pulse Amplitude (V)}%
\end{pgfscope}%
\begin{pgfscope}%
\pgfpathrectangle{\pgfqpoint{1.026143in}{0.898915in}}{\pgfqpoint{6.807833in}{4.114497in}}%
\pgfusepath{clip}%
\pgfsetbuttcap%
\pgfsetroundjoin%
\definecolor{currentfill}{rgb}{0.392157,0.584314,0.929412}%
\pgfsetfillcolor{currentfill}%
\pgfsetlinewidth{0.301125pt}%
\definecolor{currentstroke}{rgb}{0.392157,0.584314,0.929412}%
\pgfsetstrokecolor{currentstroke}%
\pgfsetdash{}{0pt}%
\pgfsys@defobject{currentmarker}{\pgfqpoint{-0.048611in}{-0.048611in}}{\pgfqpoint{0.048611in}{0.048611in}}{%
\pgfpathmoveto{\pgfqpoint{0.000000in}{-0.048611in}}%
\pgfpathcurveto{\pgfqpoint{0.012892in}{-0.048611in}}{\pgfqpoint{0.025257in}{-0.043489in}}{\pgfqpoint{0.034373in}{-0.034373in}}%
\pgfpathcurveto{\pgfqpoint{0.043489in}{-0.025257in}}{\pgfqpoint{0.048611in}{-0.012892in}}{\pgfqpoint{0.048611in}{0.000000in}}%
\pgfpathcurveto{\pgfqpoint{0.048611in}{0.012892in}}{\pgfqpoint{0.043489in}{0.025257in}}{\pgfqpoint{0.034373in}{0.034373in}}%
\pgfpathcurveto{\pgfqpoint{0.025257in}{0.043489in}}{\pgfqpoint{0.012892in}{0.048611in}}{\pgfqpoint{0.000000in}{0.048611in}}%
\pgfpathcurveto{\pgfqpoint{-0.012892in}{0.048611in}}{\pgfqpoint{-0.025257in}{0.043489in}}{\pgfqpoint{-0.034373in}{0.034373in}}%
\pgfpathcurveto{\pgfqpoint{-0.043489in}{0.025257in}}{\pgfqpoint{-0.048611in}{0.012892in}}{\pgfqpoint{-0.048611in}{0.000000in}}%
\pgfpathcurveto{\pgfqpoint{-0.048611in}{-0.012892in}}{\pgfqpoint{-0.043489in}{-0.025257in}}{\pgfqpoint{-0.034373in}{-0.034373in}}%
\pgfpathcurveto{\pgfqpoint{-0.025257in}{-0.043489in}}{\pgfqpoint{-0.012892in}{-0.048611in}}{\pgfqpoint{0.000000in}{-0.048611in}}%
\pgfpathlineto{\pgfqpoint{0.000000in}{-0.048611in}}%
\pgfpathclose%
\pgfusepath{stroke,fill}%
}%
\begin{pgfscope}%
\pgfsys@transformshift{1.335590in}{1.215671in}%
\pgfsys@useobject{currentmarker}{}%
\end{pgfscope}%
\begin{pgfscope}%
\pgfsys@transformshift{1.616906in}{1.552193in}%
\pgfsys@useobject{currentmarker}{}%
\end{pgfscope}%
\begin{pgfscope}%
\pgfsys@transformshift{1.898221in}{1.888715in}%
\pgfsys@useobject{currentmarker}{}%
\end{pgfscope}%
\begin{pgfscope}%
\pgfsys@transformshift{2.179536in}{2.129088in}%
\pgfsys@useobject{currentmarker}{}%
\end{pgfscope}%
\begin{pgfscope}%
\pgfsys@transformshift{2.460852in}{2.489647in}%
\pgfsys@useobject{currentmarker}{}%
\end{pgfscope}%
\begin{pgfscope}%
\pgfsys@transformshift{3.023483in}{3.282878in}%
\pgfsys@useobject{currentmarker}{}%
\end{pgfscope}%
\begin{pgfscope}%
\pgfsys@transformshift{3.586113in}{3.667474in}%
\pgfsys@useobject{currentmarker}{}%
\end{pgfscope}%
\begin{pgfscope}%
\pgfsys@transformshift{4.148744in}{3.979959in}%
\pgfsys@useobject{currentmarker}{}%
\end{pgfscope}%
\begin{pgfscope}%
\pgfsys@transformshift{4.711375in}{4.172257in}%
\pgfsys@useobject{currentmarker}{}%
\end{pgfscope}%
\begin{pgfscope}%
\pgfsys@transformshift{5.274006in}{4.388593in}%
\pgfsys@useobject{currentmarker}{}%
\end{pgfscope}%
\begin{pgfscope}%
\pgfsys@transformshift{5.836637in}{4.508779in}%
\pgfsys@useobject{currentmarker}{}%
\end{pgfscope}%
\begin{pgfscope}%
\pgfsys@transformshift{6.399267in}{4.628966in}%
\pgfsys@useobject{currentmarker}{}%
\end{pgfscope}%
\begin{pgfscope}%
\pgfsys@transformshift{6.961898in}{4.749152in}%
\pgfsys@useobject{currentmarker}{}%
\end{pgfscope}%
\begin{pgfscope}%
\pgfsys@transformshift{7.524529in}{4.773189in}%
\pgfsys@useobject{currentmarker}{}%
\end{pgfscope}%
\end{pgfscope}%
\begin{pgfscope}%
\pgfpathrectangle{\pgfqpoint{1.026143in}{0.898915in}}{\pgfqpoint{6.807833in}{4.114497in}}%
\pgfusepath{clip}%
\pgfsetroundcap%
\pgfsetroundjoin%
\pgfsetlinewidth{1.756562pt}%
\definecolor{currentstroke}{rgb}{0.298039,0.447059,0.690196}%
\pgfsetstrokecolor{currentstroke}%
\pgfsetdash{}{0pt}%
\pgfpathmoveto{\pgfqpoint{1.335590in}{1.085937in}}%
\pgfpathlineto{\pgfqpoint{1.398105in}{1.192291in}}%
\pgfpathlineto{\pgfqpoint{1.460619in}{1.295833in}}%
\pgfpathlineto{\pgfqpoint{1.523134in}{1.396640in}}%
\pgfpathlineto{\pgfqpoint{1.585648in}{1.494782in}}%
\pgfpathlineto{\pgfqpoint{1.648163in}{1.590331in}}%
\pgfpathlineto{\pgfqpoint{1.710677in}{1.683354in}}%
\pgfpathlineto{\pgfqpoint{1.773192in}{1.773920in}}%
\pgfpathlineto{\pgfqpoint{1.835707in}{1.862091in}}%
\pgfpathlineto{\pgfqpoint{1.898221in}{1.947933in}}%
\pgfpathlineto{\pgfqpoint{1.960736in}{2.031506in}}%
\pgfpathlineto{\pgfqpoint{2.023250in}{2.112870in}}%
\pgfpathlineto{\pgfqpoint{2.085765in}{2.192085in}}%
\pgfpathlineto{\pgfqpoint{2.148279in}{2.269205in}}%
\pgfpathlineto{\pgfqpoint{2.210794in}{2.344288in}}%
\pgfpathlineto{\pgfqpoint{2.273308in}{2.417386in}}%
\pgfpathlineto{\pgfqpoint{2.335823in}{2.488553in}}%
\pgfpathlineto{\pgfqpoint{2.398337in}{2.557839in}}%
\pgfpathlineto{\pgfqpoint{2.460852in}{2.625293in}}%
\pgfpathlineto{\pgfqpoint{2.523366in}{2.690965in}}%
\pgfpathlineto{\pgfqpoint{2.585881in}{2.754902in}}%
\pgfpathlineto{\pgfqpoint{2.648395in}{2.817149in}}%
\pgfpathlineto{\pgfqpoint{2.710910in}{2.877751in}}%
\pgfpathlineto{\pgfqpoint{2.773425in}{2.936751in}}%
\pgfpathlineto{\pgfqpoint{2.835939in}{2.994192in}}%
\pgfpathlineto{\pgfqpoint{2.898454in}{3.050115in}}%
\pgfpathlineto{\pgfqpoint{2.960968in}{3.104560in}}%
\pgfpathlineto{\pgfqpoint{3.023483in}{3.157566in}}%
\pgfpathlineto{\pgfqpoint{3.085997in}{3.209172in}}%
\pgfpathlineto{\pgfqpoint{3.148512in}{3.259413in}}%
\pgfpathlineto{\pgfqpoint{3.211026in}{3.308327in}}%
\pgfpathlineto{\pgfqpoint{3.273541in}{3.355948in}}%
\pgfpathlineto{\pgfqpoint{3.336055in}{3.402311in}}%
\pgfpathlineto{\pgfqpoint{3.398570in}{3.447449in}}%
\pgfpathlineto{\pgfqpoint{3.461084in}{3.491393in}}%
\pgfpathlineto{\pgfqpoint{3.523599in}{3.534176in}}%
\pgfpathlineto{\pgfqpoint{3.586113in}{3.575829in}}%
\pgfpathlineto{\pgfqpoint{3.648628in}{3.616381in}}%
\pgfpathlineto{\pgfqpoint{3.711143in}{3.655861in}}%
\pgfpathlineto{\pgfqpoint{3.773657in}{3.694298in}}%
\pgfpathlineto{\pgfqpoint{3.836172in}{3.731719in}}%
\pgfpathlineto{\pgfqpoint{3.898686in}{3.768151in}}%
\pgfpathlineto{\pgfqpoint{3.961201in}{3.803620in}}%
\pgfpathlineto{\pgfqpoint{4.023715in}{3.838152in}}%
\pgfpathlineto{\pgfqpoint{4.086230in}{3.871771in}}%
\pgfpathlineto{\pgfqpoint{4.148744in}{3.904502in}}%
\pgfpathlineto{\pgfqpoint{4.211259in}{3.936368in}}%
\pgfpathlineto{\pgfqpoint{4.273773in}{3.967391in}}%
\pgfpathlineto{\pgfqpoint{4.336288in}{3.997595in}}%
\pgfpathlineto{\pgfqpoint{4.398802in}{4.027001in}}%
\pgfpathlineto{\pgfqpoint{4.461317in}{4.055629in}}%
\pgfpathlineto{\pgfqpoint{4.523831in}{4.083501in}}%
\pgfpathlineto{\pgfqpoint{4.586346in}{4.110636in}}%
\pgfpathlineto{\pgfqpoint{4.648861in}{4.137054in}}%
\pgfpathlineto{\pgfqpoint{4.711375in}{4.162774in}}%
\pgfpathlineto{\pgfqpoint{4.773890in}{4.187815in}}%
\pgfpathlineto{\pgfqpoint{4.836404in}{4.212193in}}%
\pgfpathlineto{\pgfqpoint{4.898919in}{4.235927in}}%
\pgfpathlineto{\pgfqpoint{4.961433in}{4.259035in}}%
\pgfpathlineto{\pgfqpoint{5.023948in}{4.281531in}}%
\pgfpathlineto{\pgfqpoint{5.086462in}{4.303433in}}%
\pgfpathlineto{\pgfqpoint{5.148977in}{4.324756in}}%
\pgfpathlineto{\pgfqpoint{5.211491in}{4.345515in}}%
\pgfpathlineto{\pgfqpoint{5.274006in}{4.365726in}}%
\pgfpathlineto{\pgfqpoint{5.336520in}{4.385403in}}%
\pgfpathlineto{\pgfqpoint{5.399035in}{4.404560in}}%
\pgfpathlineto{\pgfqpoint{5.461549in}{4.423210in}}%
\pgfpathlineto{\pgfqpoint{5.524064in}{4.441368in}}%
\pgfpathlineto{\pgfqpoint{5.586579in}{4.459046in}}%
\pgfpathlineto{\pgfqpoint{5.649093in}{4.476256in}}%
\pgfpathlineto{\pgfqpoint{5.711608in}{4.493012in}}%
\pgfpathlineto{\pgfqpoint{5.774122in}{4.509325in}}%
\pgfpathlineto{\pgfqpoint{5.836637in}{4.525207in}}%
\pgfpathlineto{\pgfqpoint{5.899151in}{4.540669in}}%
\pgfpathlineto{\pgfqpoint{5.961666in}{4.555722in}}%
\pgfpathlineto{\pgfqpoint{6.024180in}{4.570378in}}%
\pgfpathlineto{\pgfqpoint{6.086695in}{4.584646in}}%
\pgfpathlineto{\pgfqpoint{6.149209in}{4.598538in}}%
\pgfpathlineto{\pgfqpoint{6.211724in}{4.612062in}}%
\pgfpathlineto{\pgfqpoint{6.274238in}{4.625228in}}%
\pgfpathlineto{\pgfqpoint{6.336753in}{4.638047in}}%
\pgfpathlineto{\pgfqpoint{6.399267in}{4.650527in}}%
\pgfpathlineto{\pgfqpoint{6.461782in}{4.662677in}}%
\pgfpathlineto{\pgfqpoint{6.524297in}{4.674507in}}%
\pgfpathlineto{\pgfqpoint{6.586811in}{4.686023in}}%
\pgfpathlineto{\pgfqpoint{6.649326in}{4.697235in}}%
\pgfpathlineto{\pgfqpoint{6.711840in}{4.708151in}}%
\pgfpathlineto{\pgfqpoint{6.774355in}{4.718778in}}%
\pgfpathlineto{\pgfqpoint{6.836869in}{4.729125in}}%
\pgfpathlineto{\pgfqpoint{6.899384in}{4.739198in}}%
\pgfpathlineto{\pgfqpoint{6.961898in}{4.749005in}}%
\pgfpathlineto{\pgfqpoint{7.024413in}{4.758552in}}%
\pgfpathlineto{\pgfqpoint{7.086927in}{4.767848in}}%
\pgfpathlineto{\pgfqpoint{7.149442in}{4.776897in}}%
\pgfpathlineto{\pgfqpoint{7.211956in}{4.785708in}}%
\pgfpathlineto{\pgfqpoint{7.274471in}{4.794286in}}%
\pgfpathlineto{\pgfqpoint{7.336985in}{4.802637in}}%
\pgfpathlineto{\pgfqpoint{7.399500in}{4.810767in}}%
\pgfpathlineto{\pgfqpoint{7.462015in}{4.818683in}}%
\pgfpathlineto{\pgfqpoint{7.524529in}{4.826389in}}%
\pgfusepath{stroke}%
\end{pgfscope}%
\begin{pgfscope}%
\pgfsetrectcap%
\pgfsetmiterjoin%
\pgfsetlinewidth{0.000000pt}%
\definecolor{currentstroke}{rgb}{1.000000,1.000000,1.000000}%
\pgfsetstrokecolor{currentstroke}%
\pgfsetdash{}{0pt}%
\pgfpathmoveto{\pgfqpoint{1.026143in}{0.898915in}}%
\pgfpathlineto{\pgfqpoint{1.026143in}{5.013411in}}%
\pgfusepath{}%
\end{pgfscope}%
\begin{pgfscope}%
\pgfsetrectcap%
\pgfsetmiterjoin%
\pgfsetlinewidth{0.000000pt}%
\definecolor{currentstroke}{rgb}{1.000000,1.000000,1.000000}%
\pgfsetstrokecolor{currentstroke}%
\pgfsetdash{}{0pt}%
\pgfpathmoveto{\pgfqpoint{7.833976in}{0.898915in}}%
\pgfpathlineto{\pgfqpoint{7.833976in}{5.013411in}}%
\pgfusepath{}%
\end{pgfscope}%
\begin{pgfscope}%
\pgfsetrectcap%
\pgfsetmiterjoin%
\pgfsetlinewidth{0.000000pt}%
\definecolor{currentstroke}{rgb}{1.000000,1.000000,1.000000}%
\pgfsetstrokecolor{currentstroke}%
\pgfsetdash{}{0pt}%
\pgfpathmoveto{\pgfqpoint{1.026143in}{0.898915in}}%
\pgfpathlineto{\pgfqpoint{7.833976in}{0.898915in}}%
\pgfusepath{}%
\end{pgfscope}%
\begin{pgfscope}%
\pgfsetrectcap%
\pgfsetmiterjoin%
\pgfsetlinewidth{0.000000pt}%
\definecolor{currentstroke}{rgb}{1.000000,1.000000,1.000000}%
\pgfsetstrokecolor{currentstroke}%
\pgfsetdash{}{0pt}%
\pgfpathmoveto{\pgfqpoint{1.026143in}{5.013411in}}%
\pgfpathlineto{\pgfqpoint{7.833976in}{5.013411in}}%
\pgfusepath{}%
\end{pgfscope}%
\begin{pgfscope}%
\definecolor{textcolor}{rgb}{0.150000,0.150000,0.150000}%
\pgfsetstrokecolor{textcolor}%
\pgfsetfillcolor{textcolor}%
\pgftext[x=3.023483in, y=1.589540in, left, base]{\color{textcolor}\rmfamily\fontsize{20.000000}{24.000000}\selectfont M\(\displaystyle _z\)(t) = 8.52(1\(\displaystyle -\)2\(\displaystyle exp[\frac{-t}{T_1}]\))\(\displaystyle +\)-1.16}%
\end{pgfscope}%
\begin{pgfscope}%
\definecolor{textcolor}{rgb}{0.150000,0.150000,0.150000}%
\pgfsetstrokecolor{textcolor}%
\pgfsetfillcolor{textcolor}%
\pgftext[x=3.023483in, y=1.095485in, left, base]{\color{textcolor}\rmfamily\fontsize{20.000000}{24.000000}\selectfont T\(\displaystyle _1\) = 4.148\(\displaystyle \pm\)0.31 ms}%
\end{pgfscope}%
\begin{pgfscope}%
\definecolor{textcolor}{rgb}{0.150000,0.150000,0.150000}%
\pgfsetstrokecolor{textcolor}%
\pgfsetfillcolor{textcolor}%
\pgftext[x=4.430060in,y=5.096745in,,base]{\color{textcolor}\rmfamily\fontsize{24.000000}{28.800000}\selectfont T\(\displaystyle _1\) measurement of 0.2 M CuSO\(\displaystyle _4\)}%
\end{pgfscope}%
\begin{pgfscope}%
\pgfsetbuttcap%
\pgfsetroundjoin%
\definecolor{currentfill}{rgb}{0.392157,0.584314,0.929412}%
\pgfsetfillcolor{currentfill}%
\pgfsetlinewidth{0.301125pt}%
\definecolor{currentstroke}{rgb}{0.392157,0.584314,0.929412}%
\pgfsetstrokecolor{currentstroke}%
\pgfsetdash{}{0pt}%
\pgfsys@defobject{currentmarker}{\pgfqpoint{-0.048611in}{-0.048611in}}{\pgfqpoint{0.048611in}{0.048611in}}{%
\pgfpathmoveto{\pgfqpoint{0.000000in}{-0.048611in}}%
\pgfpathcurveto{\pgfqpoint{0.012892in}{-0.048611in}}{\pgfqpoint{0.025257in}{-0.043489in}}{\pgfqpoint{0.034373in}{-0.034373in}}%
\pgfpathcurveto{\pgfqpoint{0.043489in}{-0.025257in}}{\pgfqpoint{0.048611in}{-0.012892in}}{\pgfqpoint{0.048611in}{0.000000in}}%
\pgfpathcurveto{\pgfqpoint{0.048611in}{0.012892in}}{\pgfqpoint{0.043489in}{0.025257in}}{\pgfqpoint{0.034373in}{0.034373in}}%
\pgfpathcurveto{\pgfqpoint{0.025257in}{0.043489in}}{\pgfqpoint{0.012892in}{0.048611in}}{\pgfqpoint{0.000000in}{0.048611in}}%
\pgfpathcurveto{\pgfqpoint{-0.012892in}{0.048611in}}{\pgfqpoint{-0.025257in}{0.043489in}}{\pgfqpoint{-0.034373in}{0.034373in}}%
\pgfpathcurveto{\pgfqpoint{-0.043489in}{0.025257in}}{\pgfqpoint{-0.048611in}{0.012892in}}{\pgfqpoint{-0.048611in}{0.000000in}}%
\pgfpathcurveto{\pgfqpoint{-0.048611in}{-0.012892in}}{\pgfqpoint{-0.043489in}{-0.025257in}}{\pgfqpoint{-0.034373in}{-0.034373in}}%
\pgfpathcurveto{\pgfqpoint{-0.025257in}{-0.043489in}}{\pgfqpoint{-0.012892in}{-0.048611in}}{\pgfqpoint{0.000000in}{-0.048611in}}%
\pgfpathlineto{\pgfqpoint{0.000000in}{-0.048611in}}%
\pgfpathclose%
\pgfusepath{stroke,fill}%
}%
\begin{pgfscope}%
\pgfsys@transformshift{1.553921in}{4.625282in}%
\pgfsys@useobject{currentmarker}{}%
\end{pgfscope}%
\end{pgfscope}%
\begin{pgfscope}%
\definecolor{textcolor}{rgb}{0.150000,0.150000,0.150000}%
\pgfsetstrokecolor{textcolor}%
\pgfsetfillcolor{textcolor}%
\pgftext[x=2.053921in,y=4.552365in,left,base]{\color{textcolor}\rmfamily\fontsize{20.000000}{24.000000}\selectfont Data}%
\end{pgfscope}%
\begin{pgfscope}%
\pgfsetroundcap%
\pgfsetroundjoin%
\pgfsetlinewidth{1.756562pt}%
\definecolor{currentstroke}{rgb}{0.298039,0.447059,0.690196}%
\pgfsetstrokecolor{currentstroke}%
\pgfsetdash{}{0pt}%
\pgfpathmoveto{\pgfqpoint{1.276143in}{4.241873in}}%
\pgfpathlineto{\pgfqpoint{1.553921in}{4.241873in}}%
\pgfpathlineto{\pgfqpoint{1.831699in}{4.241873in}}%
\pgfusepath{stroke}%
\end{pgfscope}%
\begin{pgfscope}%
\definecolor{textcolor}{rgb}{0.150000,0.150000,0.150000}%
\pgfsetstrokecolor{textcolor}%
\pgfsetfillcolor{textcolor}%
\pgftext[x=2.053921in,y=4.144651in,left,base]{\color{textcolor}\rmfamily\fontsize{20.000000}{24.000000}\selectfont Fitted Curve}%
\end{pgfscope}%
\end{pgfpicture}%
\makeatother%
\endgroup%
}
    \caption{T1 measurement of 0.2 M  CuSO$_{4} \boldsymbol{\cdot} $H$_2$O using the two-pulse method }
    \label{fig:0d2_t1}
  \end{figure}

  \begin{figure}[h]
    \resizebox{0.45\textwidth}{!}{%% Creator: Matplotlib, PGF backend
%%
%% To include the figure in your LaTeX document, write
%%   \input{<filename>.pgf}
%%
%% Make sure the required packages are loaded in your preamble
%%   \usepackage{pgf}
%%
%% Also ensure that all the required font packages are loaded; for instance,
%% the lmodern package is sometimes necessary when using math font.
%%   \usepackage{lmodern}
%%
%% Figures using additional raster images can only be included by \input if
%% they are in the same directory as the main LaTeX file. For loading figures
%% from other directories you can use the `import` package
%%   \usepackage{import}
%%
%% and then include the figures with
%%   \import{<path to file>}{<filename>.pgf}
%%
%% Matplotlib used the following preamble
%%   \usepackage{fontspec}
%%   \setmainfont{DejaVuSerif.ttf}[Path=\detokenize{/usr/share/matplotlib/mpl-data/fonts/ttf/}]
%%   \setsansfont{DejaVuSans.ttf}[Path=\detokenize{/usr/share/matplotlib/mpl-data/fonts/ttf/}]
%%   \setmonofont{DejaVuSansMono.ttf}[Path=\detokenize{/usr/share/matplotlib/mpl-data/fonts/ttf/}]
%%
\begingroup%
\makeatletter%
\begin{pgfpicture}%
\pgfpathrectangle{\pgfpointorigin}{\pgfqpoint{8.000000in}{5.500000in}}%
\pgfusepath{use as bounding box, clip}%
\begin{pgfscope}%
\pgfsetbuttcap%
\pgfsetmiterjoin%
\definecolor{currentfill}{rgb}{1.000000,1.000000,1.000000}%
\pgfsetfillcolor{currentfill}%
\pgfsetlinewidth{0.000000pt}%
\definecolor{currentstroke}{rgb}{1.000000,1.000000,1.000000}%
\pgfsetstrokecolor{currentstroke}%
\pgfsetdash{}{0pt}%
\pgfpathmoveto{\pgfqpoint{0.000000in}{0.000000in}}%
\pgfpathlineto{\pgfqpoint{8.000000in}{0.000000in}}%
\pgfpathlineto{\pgfqpoint{8.000000in}{5.500000in}}%
\pgfpathlineto{\pgfqpoint{0.000000in}{5.500000in}}%
\pgfpathlineto{\pgfqpoint{0.000000in}{0.000000in}}%
\pgfpathclose%
\pgfusepath{fill}%
\end{pgfscope}%
\begin{pgfscope}%
\pgfsetbuttcap%
\pgfsetmiterjoin%
\definecolor{currentfill}{rgb}{0.917647,0.917647,0.949020}%
\pgfsetfillcolor{currentfill}%
\pgfsetlinewidth{0.000000pt}%
\definecolor{currentstroke}{rgb}{0.000000,0.000000,0.000000}%
\pgfsetstrokecolor{currentstroke}%
\pgfsetstrokeopacity{0.000000}%
\pgfsetdash{}{0pt}%
\pgfpathmoveto{\pgfqpoint{1.026143in}{0.898915in}}%
\pgfpathlineto{\pgfqpoint{7.657218in}{0.898915in}}%
\pgfpathlineto{\pgfqpoint{7.657218in}{5.013411in}}%
\pgfpathlineto{\pgfqpoint{1.026143in}{5.013411in}}%
\pgfpathlineto{\pgfqpoint{1.026143in}{0.898915in}}%
\pgfpathclose%
\pgfusepath{fill}%
\end{pgfscope}%
\begin{pgfscope}%
\pgfpathrectangle{\pgfqpoint{1.026143in}{0.898915in}}{\pgfqpoint{6.631075in}{4.114497in}}%
\pgfusepath{clip}%
\pgfsetroundcap%
\pgfsetroundjoin%
\pgfsetlinewidth{1.003750pt}%
\definecolor{currentstroke}{rgb}{1.000000,1.000000,1.000000}%
\pgfsetstrokecolor{currentstroke}%
\pgfsetdash{}{0pt}%
\pgfpathmoveto{\pgfqpoint{1.251249in}{0.898915in}}%
\pgfpathlineto{\pgfqpoint{1.251249in}{5.013411in}}%
\pgfusepath{stroke}%
\end{pgfscope}%
\begin{pgfscope}%
\definecolor{textcolor}{rgb}{0.150000,0.150000,0.150000}%
\pgfsetstrokecolor{textcolor}%
\pgfsetfillcolor{textcolor}%
\pgftext[x=1.251249in,y=0.801692in,,top]{\color{textcolor}\rmfamily\fontsize{20.000000}{24.000000}\selectfont 0}%
\end{pgfscope}%
\begin{pgfscope}%
\pgfpathrectangle{\pgfqpoint{1.026143in}{0.898915in}}{\pgfqpoint{6.631075in}{4.114497in}}%
\pgfusepath{clip}%
\pgfsetroundcap%
\pgfsetroundjoin%
\pgfsetlinewidth{1.003750pt}%
\definecolor{currentstroke}{rgb}{1.000000,1.000000,1.000000}%
\pgfsetstrokecolor{currentstroke}%
\pgfsetdash{}{0pt}%
\pgfpathmoveto{\pgfqpoint{2.777388in}{0.898915in}}%
\pgfpathlineto{\pgfqpoint{2.777388in}{5.013411in}}%
\pgfusepath{stroke}%
\end{pgfscope}%
\begin{pgfscope}%
\definecolor{textcolor}{rgb}{0.150000,0.150000,0.150000}%
\pgfsetstrokecolor{textcolor}%
\pgfsetfillcolor{textcolor}%
\pgftext[x=2.777388in,y=0.801692in,,top]{\color{textcolor}\rmfamily\fontsize{20.000000}{24.000000}\selectfont 2}%
\end{pgfscope}%
\begin{pgfscope}%
\pgfpathrectangle{\pgfqpoint{1.026143in}{0.898915in}}{\pgfqpoint{6.631075in}{4.114497in}}%
\pgfusepath{clip}%
\pgfsetroundcap%
\pgfsetroundjoin%
\pgfsetlinewidth{1.003750pt}%
\definecolor{currentstroke}{rgb}{1.000000,1.000000,1.000000}%
\pgfsetstrokecolor{currentstroke}%
\pgfsetdash{}{0pt}%
\pgfpathmoveto{\pgfqpoint{4.303527in}{0.898915in}}%
\pgfpathlineto{\pgfqpoint{4.303527in}{5.013411in}}%
\pgfusepath{stroke}%
\end{pgfscope}%
\begin{pgfscope}%
\definecolor{textcolor}{rgb}{0.150000,0.150000,0.150000}%
\pgfsetstrokecolor{textcolor}%
\pgfsetfillcolor{textcolor}%
\pgftext[x=4.303527in,y=0.801692in,,top]{\color{textcolor}\rmfamily\fontsize{20.000000}{24.000000}\selectfont 4}%
\end{pgfscope}%
\begin{pgfscope}%
\pgfpathrectangle{\pgfqpoint{1.026143in}{0.898915in}}{\pgfqpoint{6.631075in}{4.114497in}}%
\pgfusepath{clip}%
\pgfsetroundcap%
\pgfsetroundjoin%
\pgfsetlinewidth{1.003750pt}%
\definecolor{currentstroke}{rgb}{1.000000,1.000000,1.000000}%
\pgfsetstrokecolor{currentstroke}%
\pgfsetdash{}{0pt}%
\pgfpathmoveto{\pgfqpoint{5.829666in}{0.898915in}}%
\pgfpathlineto{\pgfqpoint{5.829666in}{5.013411in}}%
\pgfusepath{stroke}%
\end{pgfscope}%
\begin{pgfscope}%
\definecolor{textcolor}{rgb}{0.150000,0.150000,0.150000}%
\pgfsetstrokecolor{textcolor}%
\pgfsetfillcolor{textcolor}%
\pgftext[x=5.829666in,y=0.801692in,,top]{\color{textcolor}\rmfamily\fontsize{20.000000}{24.000000}\selectfont 6}%
\end{pgfscope}%
\begin{pgfscope}%
\pgfpathrectangle{\pgfqpoint{1.026143in}{0.898915in}}{\pgfqpoint{6.631075in}{4.114497in}}%
\pgfusepath{clip}%
\pgfsetroundcap%
\pgfsetroundjoin%
\pgfsetlinewidth{1.003750pt}%
\definecolor{currentstroke}{rgb}{1.000000,1.000000,1.000000}%
\pgfsetstrokecolor{currentstroke}%
\pgfsetdash{}{0pt}%
\pgfpathmoveto{\pgfqpoint{7.355805in}{0.898915in}}%
\pgfpathlineto{\pgfqpoint{7.355805in}{5.013411in}}%
\pgfusepath{stroke}%
\end{pgfscope}%
\begin{pgfscope}%
\definecolor{textcolor}{rgb}{0.150000,0.150000,0.150000}%
\pgfsetstrokecolor{textcolor}%
\pgfsetfillcolor{textcolor}%
\pgftext[x=7.355805in,y=0.801692in,,top]{\color{textcolor}\rmfamily\fontsize{20.000000}{24.000000}\selectfont 8}%
\end{pgfscope}%
\begin{pgfscope}%
\definecolor{textcolor}{rgb}{0.150000,0.150000,0.150000}%
\pgfsetstrokecolor{textcolor}%
\pgfsetfillcolor{textcolor}%
\pgftext[x=4.341681in,y=0.477311in,,top]{\color{textcolor}\rmfamily\fontsize{24.000000}{28.800000}\selectfont Delay Time (ms)}%
\end{pgfscope}%
\begin{pgfscope}%
\pgfpathrectangle{\pgfqpoint{1.026143in}{0.898915in}}{\pgfqpoint{6.631075in}{4.114497in}}%
\pgfusepath{clip}%
\pgfsetroundcap%
\pgfsetroundjoin%
\pgfsetlinewidth{1.003750pt}%
\definecolor{currentstroke}{rgb}{1.000000,1.000000,1.000000}%
\pgfsetstrokecolor{currentstroke}%
\pgfsetdash{}{0pt}%
\pgfpathmoveto{\pgfqpoint{1.026143in}{1.151367in}}%
\pgfpathlineto{\pgfqpoint{7.657218in}{1.151367in}}%
\pgfusepath{stroke}%
\end{pgfscope}%
\begin{pgfscope}%
\definecolor{textcolor}{rgb}{0.150000,0.150000,0.150000}%
\pgfsetstrokecolor{textcolor}%
\pgfsetfillcolor{textcolor}%
\pgftext[x=0.528147in, y=1.045844in, left, base]{\color{textcolor}\rmfamily\fontsize{20.000000}{24.000000}\selectfont \ensuremath{-}6}%
\end{pgfscope}%
\begin{pgfscope}%
\pgfpathrectangle{\pgfqpoint{1.026143in}{0.898915in}}{\pgfqpoint{6.631075in}{4.114497in}}%
\pgfusepath{clip}%
\pgfsetroundcap%
\pgfsetroundjoin%
\pgfsetlinewidth{1.003750pt}%
\definecolor{currentstroke}{rgb}{1.000000,1.000000,1.000000}%
\pgfsetstrokecolor{currentstroke}%
\pgfsetdash{}{0pt}%
\pgfpathmoveto{\pgfqpoint{1.026143in}{1.767903in}}%
\pgfpathlineto{\pgfqpoint{7.657218in}{1.767903in}}%
\pgfusepath{stroke}%
\end{pgfscope}%
\begin{pgfscope}%
\definecolor{textcolor}{rgb}{0.150000,0.150000,0.150000}%
\pgfsetstrokecolor{textcolor}%
\pgfsetfillcolor{textcolor}%
\pgftext[x=0.528147in, y=1.662380in, left, base]{\color{textcolor}\rmfamily\fontsize{20.000000}{24.000000}\selectfont \ensuremath{-}4}%
\end{pgfscope}%
\begin{pgfscope}%
\pgfpathrectangle{\pgfqpoint{1.026143in}{0.898915in}}{\pgfqpoint{6.631075in}{4.114497in}}%
\pgfusepath{clip}%
\pgfsetroundcap%
\pgfsetroundjoin%
\pgfsetlinewidth{1.003750pt}%
\definecolor{currentstroke}{rgb}{1.000000,1.000000,1.000000}%
\pgfsetstrokecolor{currentstroke}%
\pgfsetdash{}{0pt}%
\pgfpathmoveto{\pgfqpoint{1.026143in}{2.384439in}}%
\pgfpathlineto{\pgfqpoint{7.657218in}{2.384439in}}%
\pgfusepath{stroke}%
\end{pgfscope}%
\begin{pgfscope}%
\definecolor{textcolor}{rgb}{0.150000,0.150000,0.150000}%
\pgfsetstrokecolor{textcolor}%
\pgfsetfillcolor{textcolor}%
\pgftext[x=0.528147in, y=2.278916in, left, base]{\color{textcolor}\rmfamily\fontsize{20.000000}{24.000000}\selectfont \ensuremath{-}2}%
\end{pgfscope}%
\begin{pgfscope}%
\pgfpathrectangle{\pgfqpoint{1.026143in}{0.898915in}}{\pgfqpoint{6.631075in}{4.114497in}}%
\pgfusepath{clip}%
\pgfsetroundcap%
\pgfsetroundjoin%
\pgfsetlinewidth{1.003750pt}%
\definecolor{currentstroke}{rgb}{1.000000,1.000000,1.000000}%
\pgfsetstrokecolor{currentstroke}%
\pgfsetdash{}{0pt}%
\pgfpathmoveto{\pgfqpoint{1.026143in}{3.000975in}}%
\pgfpathlineto{\pgfqpoint{7.657218in}{3.000975in}}%
\pgfusepath{stroke}%
\end{pgfscope}%
\begin{pgfscope}%
\definecolor{textcolor}{rgb}{0.150000,0.150000,0.150000}%
\pgfsetstrokecolor{textcolor}%
\pgfsetfillcolor{textcolor}%
\pgftext[x=0.752190in, y=2.895452in, left, base]{\color{textcolor}\rmfamily\fontsize{20.000000}{24.000000}\selectfont 0}%
\end{pgfscope}%
\begin{pgfscope}%
\pgfpathrectangle{\pgfqpoint{1.026143in}{0.898915in}}{\pgfqpoint{6.631075in}{4.114497in}}%
\pgfusepath{clip}%
\pgfsetroundcap%
\pgfsetroundjoin%
\pgfsetlinewidth{1.003750pt}%
\definecolor{currentstroke}{rgb}{1.000000,1.000000,1.000000}%
\pgfsetstrokecolor{currentstroke}%
\pgfsetdash{}{0pt}%
\pgfpathmoveto{\pgfqpoint{1.026143in}{3.617510in}}%
\pgfpathlineto{\pgfqpoint{7.657218in}{3.617510in}}%
\pgfusepath{stroke}%
\end{pgfscope}%
\begin{pgfscope}%
\definecolor{textcolor}{rgb}{0.150000,0.150000,0.150000}%
\pgfsetstrokecolor{textcolor}%
\pgfsetfillcolor{textcolor}%
\pgftext[x=0.752190in, y=3.511987in, left, base]{\color{textcolor}\rmfamily\fontsize{20.000000}{24.000000}\selectfont 2}%
\end{pgfscope}%
\begin{pgfscope}%
\pgfpathrectangle{\pgfqpoint{1.026143in}{0.898915in}}{\pgfqpoint{6.631075in}{4.114497in}}%
\pgfusepath{clip}%
\pgfsetroundcap%
\pgfsetroundjoin%
\pgfsetlinewidth{1.003750pt}%
\definecolor{currentstroke}{rgb}{1.000000,1.000000,1.000000}%
\pgfsetstrokecolor{currentstroke}%
\pgfsetdash{}{0pt}%
\pgfpathmoveto{\pgfqpoint{1.026143in}{4.234046in}}%
\pgfpathlineto{\pgfqpoint{7.657218in}{4.234046in}}%
\pgfusepath{stroke}%
\end{pgfscope}%
\begin{pgfscope}%
\definecolor{textcolor}{rgb}{0.150000,0.150000,0.150000}%
\pgfsetstrokecolor{textcolor}%
\pgfsetfillcolor{textcolor}%
\pgftext[x=0.752190in, y=4.128523in, left, base]{\color{textcolor}\rmfamily\fontsize{20.000000}{24.000000}\selectfont 4}%
\end{pgfscope}%
\begin{pgfscope}%
\pgfpathrectangle{\pgfqpoint{1.026143in}{0.898915in}}{\pgfqpoint{6.631075in}{4.114497in}}%
\pgfusepath{clip}%
\pgfsetroundcap%
\pgfsetroundjoin%
\pgfsetlinewidth{1.003750pt}%
\definecolor{currentstroke}{rgb}{1.000000,1.000000,1.000000}%
\pgfsetstrokecolor{currentstroke}%
\pgfsetdash{}{0pt}%
\pgfpathmoveto{\pgfqpoint{1.026143in}{4.850582in}}%
\pgfpathlineto{\pgfqpoint{7.657218in}{4.850582in}}%
\pgfusepath{stroke}%
\end{pgfscope}%
\begin{pgfscope}%
\definecolor{textcolor}{rgb}{0.150000,0.150000,0.150000}%
\pgfsetstrokecolor{textcolor}%
\pgfsetfillcolor{textcolor}%
\pgftext[x=0.752190in, y=4.745059in, left, base]{\color{textcolor}\rmfamily\fontsize{20.000000}{24.000000}\selectfont 6}%
\end{pgfscope}%
\begin{pgfscope}%
\definecolor{textcolor}{rgb}{0.150000,0.150000,0.150000}%
\pgfsetstrokecolor{textcolor}%
\pgfsetfillcolor{textcolor}%
\pgftext[x=0.472591in,y=2.956163in,,bottom,rotate=90.000000]{\color{textcolor}\rmfamily\fontsize{24.000000}{28.800000}\selectfont Pulse Amplitude (V)}%
\end{pgfscope}%
\begin{pgfscope}%
\pgfpathrectangle{\pgfqpoint{1.026143in}{0.898915in}}{\pgfqpoint{6.631075in}{4.114497in}}%
\pgfusepath{clip}%
\pgfsetbuttcap%
\pgfsetroundjoin%
\definecolor{currentfill}{rgb}{0.392157,0.584314,0.929412}%
\pgfsetfillcolor{currentfill}%
\pgfsetlinewidth{0.301125pt}%
\definecolor{currentstroke}{rgb}{0.392157,0.584314,0.929412}%
\pgfsetstrokecolor{currentstroke}%
\pgfsetdash{}{0pt}%
\pgfsys@defobject{currentmarker}{\pgfqpoint{-0.048611in}{-0.048611in}}{\pgfqpoint{0.048611in}{0.048611in}}{%
\pgfpathmoveto{\pgfqpoint{0.000000in}{-0.048611in}}%
\pgfpathcurveto{\pgfqpoint{0.012892in}{-0.048611in}}{\pgfqpoint{0.025257in}{-0.043489in}}{\pgfqpoint{0.034373in}{-0.034373in}}%
\pgfpathcurveto{\pgfqpoint{0.043489in}{-0.025257in}}{\pgfqpoint{0.048611in}{-0.012892in}}{\pgfqpoint{0.048611in}{0.000000in}}%
\pgfpathcurveto{\pgfqpoint{0.048611in}{0.012892in}}{\pgfqpoint{0.043489in}{0.025257in}}{\pgfqpoint{0.034373in}{0.034373in}}%
\pgfpathcurveto{\pgfqpoint{0.025257in}{0.043489in}}{\pgfqpoint{0.012892in}{0.048611in}}{\pgfqpoint{0.000000in}{0.048611in}}%
\pgfpathcurveto{\pgfqpoint{-0.012892in}{0.048611in}}{\pgfqpoint{-0.025257in}{0.043489in}}{\pgfqpoint{-0.034373in}{0.034373in}}%
\pgfpathcurveto{\pgfqpoint{-0.043489in}{0.025257in}}{\pgfqpoint{-0.048611in}{0.012892in}}{\pgfqpoint{-0.048611in}{0.000000in}}%
\pgfpathcurveto{\pgfqpoint{-0.048611in}{-0.012892in}}{\pgfqpoint{-0.043489in}{-0.025257in}}{\pgfqpoint{-0.034373in}{-0.034373in}}%
\pgfpathcurveto{\pgfqpoint{-0.025257in}{-0.043489in}}{\pgfqpoint{-0.012892in}{-0.048611in}}{\pgfqpoint{0.000000in}{-0.048611in}}%
\pgfpathlineto{\pgfqpoint{0.000000in}{-0.048611in}}%
\pgfpathclose%
\pgfusepath{stroke,fill}%
}%
\begin{pgfscope}%
\pgfsys@transformshift{1.327556in}{1.225352in}%
\pgfsys@useobject{currentmarker}{}%
\end{pgfscope}%
\begin{pgfscope}%
\pgfsys@transformshift{1.403863in}{1.323997in}%
\pgfsys@useobject{currentmarker}{}%
\end{pgfscope}%
\begin{pgfscope}%
\pgfsys@transformshift{1.556477in}{1.644596in}%
\pgfsys@useobject{currentmarker}{}%
\end{pgfscope}%
\begin{pgfscope}%
\pgfsys@transformshift{1.709091in}{1.965195in}%
\pgfsys@useobject{currentmarker}{}%
\end{pgfscope}%
\begin{pgfscope}%
\pgfsys@transformshift{1.861705in}{2.236470in}%
\pgfsys@useobject{currentmarker}{}%
\end{pgfscope}%
\begin{pgfscope}%
\pgfsys@transformshift{2.014318in}{2.532408in}%
\pgfsys@useobject{currentmarker}{}%
\end{pgfscope}%
\begin{pgfscope}%
\pgfsys@transformshift{2.777388in}{3.814802in}%
\pgfsys@useobject{currentmarker}{}%
\end{pgfscope}%
\begin{pgfscope}%
\pgfsys@transformshift{3.540458in}{4.234046in}%
\pgfsys@useobject{currentmarker}{}%
\end{pgfscope}%
\begin{pgfscope}%
\pgfsys@transformshift{4.303527in}{4.480660in}%
\pgfsys@useobject{currentmarker}{}%
\end{pgfscope}%
\begin{pgfscope}%
\pgfsys@transformshift{5.066597in}{4.653290in}%
\pgfsys@useobject{currentmarker}{}%
\end{pgfscope}%
\begin{pgfscope}%
\pgfsys@transformshift{5.829666in}{4.702613in}%
\pgfsys@useobject{currentmarker}{}%
\end{pgfscope}%
\begin{pgfscope}%
\pgfsys@transformshift{6.592736in}{4.776598in}%
\pgfsys@useobject{currentmarker}{}%
\end{pgfscope}%
\begin{pgfscope}%
\pgfsys@transformshift{7.355805in}{4.801259in}%
\pgfsys@useobject{currentmarker}{}%
\end{pgfscope}%
\end{pgfscope}%
\begin{pgfscope}%
\pgfpathrectangle{\pgfqpoint{1.026143in}{0.898915in}}{\pgfqpoint{6.631075in}{4.114497in}}%
\pgfusepath{clip}%
\pgfsetroundcap%
\pgfsetroundjoin%
\pgfsetlinewidth{1.756562pt}%
\definecolor{currentstroke}{rgb}{0.298039,0.447059,0.690196}%
\pgfsetstrokecolor{currentstroke}%
\pgfsetdash{}{0pt}%
\pgfpathmoveto{\pgfqpoint{1.327556in}{1.085937in}}%
\pgfpathlineto{\pgfqpoint{1.388447in}{1.258109in}}%
\pgfpathlineto{\pgfqpoint{1.449339in}{1.422434in}}%
\pgfpathlineto{\pgfqpoint{1.510230in}{1.579270in}}%
\pgfpathlineto{\pgfqpoint{1.571121in}{1.728958in}}%
\pgfpathlineto{\pgfqpoint{1.632013in}{1.871824in}}%
\pgfpathlineto{\pgfqpoint{1.692904in}{2.008179in}}%
\pgfpathlineto{\pgfqpoint{1.753796in}{2.138319in}}%
\pgfpathlineto{\pgfqpoint{1.814687in}{2.262528in}}%
\pgfpathlineto{\pgfqpoint{1.875579in}{2.381077in}}%
\pgfpathlineto{\pgfqpoint{1.936470in}{2.494222in}}%
\pgfpathlineto{\pgfqpoint{1.997361in}{2.602211in}}%
\pgfpathlineto{\pgfqpoint{2.058253in}{2.705278in}}%
\pgfpathlineto{\pgfqpoint{2.119144in}{2.803648in}}%
\pgfpathlineto{\pgfqpoint{2.180036in}{2.897535in}}%
\pgfpathlineto{\pgfqpoint{2.240927in}{2.987143in}}%
\pgfpathlineto{\pgfqpoint{2.301818in}{3.072667in}}%
\pgfpathlineto{\pgfqpoint{2.362710in}{3.154293in}}%
\pgfpathlineto{\pgfqpoint{2.423601in}{3.232199in}}%
\pgfpathlineto{\pgfqpoint{2.484493in}{3.306554in}}%
\pgfpathlineto{\pgfqpoint{2.545384in}{3.377521in}}%
\pgfpathlineto{\pgfqpoint{2.606275in}{3.445253in}}%
\pgfpathlineto{\pgfqpoint{2.667167in}{3.509898in}}%
\pgfpathlineto{\pgfqpoint{2.728058in}{3.571598in}}%
\pgfpathlineto{\pgfqpoint{2.788950in}{3.630485in}}%
\pgfpathlineto{\pgfqpoint{2.849841in}{3.686688in}}%
\pgfpathlineto{\pgfqpoint{2.910732in}{3.740330in}}%
\pgfpathlineto{\pgfqpoint{2.971624in}{3.791527in}}%
\pgfpathlineto{\pgfqpoint{3.032515in}{3.840391in}}%
\pgfpathlineto{\pgfqpoint{3.093407in}{3.887028in}}%
\pgfpathlineto{\pgfqpoint{3.154298in}{3.931539in}}%
\pgfpathlineto{\pgfqpoint{3.215190in}{3.974022in}}%
\pgfpathlineto{\pgfqpoint{3.276081in}{4.014569in}}%
\pgfpathlineto{\pgfqpoint{3.336972in}{4.053267in}}%
\pgfpathlineto{\pgfqpoint{3.397864in}{4.090202in}}%
\pgfpathlineto{\pgfqpoint{3.458755in}{4.125454in}}%
\pgfpathlineto{\pgfqpoint{3.519647in}{4.159099in}}%
\pgfpathlineto{\pgfqpoint{3.580538in}{4.191211in}}%
\pgfpathlineto{\pgfqpoint{3.641429in}{4.221859in}}%
\pgfpathlineto{\pgfqpoint{3.702321in}{4.251110in}}%
\pgfpathlineto{\pgfqpoint{3.763212in}{4.279029in}}%
\pgfpathlineto{\pgfqpoint{3.824104in}{4.305674in}}%
\pgfpathlineto{\pgfqpoint{3.884995in}{4.331106in}}%
\pgfpathlineto{\pgfqpoint{3.945886in}{4.355378in}}%
\pgfpathlineto{\pgfqpoint{4.006778in}{4.378544in}}%
\pgfpathlineto{\pgfqpoint{4.067669in}{4.400655in}}%
\pgfpathlineto{\pgfqpoint{4.128561in}{4.421758in}}%
\pgfpathlineto{\pgfqpoint{4.189452in}{4.441898in}}%
\pgfpathlineto{\pgfqpoint{4.250343in}{4.461122in}}%
\pgfpathlineto{\pgfqpoint{4.311235in}{4.479468in}}%
\pgfpathlineto{\pgfqpoint{4.372126in}{4.496979in}}%
\pgfpathlineto{\pgfqpoint{4.433018in}{4.513692in}}%
\pgfpathlineto{\pgfqpoint{4.493909in}{4.529643in}}%
\pgfpathlineto{\pgfqpoint{4.554801in}{4.544867in}}%
\pgfpathlineto{\pgfqpoint{4.615692in}{4.559397in}}%
\pgfpathlineto{\pgfqpoint{4.676583in}{4.573265in}}%
\pgfpathlineto{\pgfqpoint{4.737475in}{4.586501in}}%
\pgfpathlineto{\pgfqpoint{4.798366in}{4.599134in}}%
\pgfpathlineto{\pgfqpoint{4.859258in}{4.611191in}}%
\pgfpathlineto{\pgfqpoint{4.920149in}{4.622698in}}%
\pgfpathlineto{\pgfqpoint{4.981040in}{4.633681in}}%
\pgfpathlineto{\pgfqpoint{5.041932in}{4.644164in}}%
\pgfpathlineto{\pgfqpoint{5.102823in}{4.654168in}}%
\pgfpathlineto{\pgfqpoint{5.163715in}{4.663717in}}%
\pgfpathlineto{\pgfqpoint{5.224606in}{4.672831in}}%
\pgfpathlineto{\pgfqpoint{5.285497in}{4.681529in}}%
\pgfpathlineto{\pgfqpoint{5.346389in}{4.689831in}}%
\pgfpathlineto{\pgfqpoint{5.407280in}{4.697754in}}%
\pgfpathlineto{\pgfqpoint{5.468172in}{4.705317in}}%
\pgfpathlineto{\pgfqpoint{5.529063in}{4.712534in}}%
\pgfpathlineto{\pgfqpoint{5.589955in}{4.719423in}}%
\pgfpathlineto{\pgfqpoint{5.650846in}{4.725998in}}%
\pgfpathlineto{\pgfqpoint{5.711737in}{4.732273in}}%
\pgfpathlineto{\pgfqpoint{5.772629in}{4.738262in}}%
\pgfpathlineto{\pgfqpoint{5.833520in}{4.743978in}}%
\pgfpathlineto{\pgfqpoint{5.894412in}{4.749434in}}%
\pgfpathlineto{\pgfqpoint{5.955303in}{4.754641in}}%
\pgfpathlineto{\pgfqpoint{6.016194in}{4.759610in}}%
\pgfpathlineto{\pgfqpoint{6.077086in}{4.764354in}}%
\pgfpathlineto{\pgfqpoint{6.137977in}{4.768881in}}%
\pgfpathlineto{\pgfqpoint{6.198869in}{4.773201in}}%
\pgfpathlineto{\pgfqpoint{6.259760in}{4.777325in}}%
\pgfpathlineto{\pgfqpoint{6.320651in}{4.781261in}}%
\pgfpathlineto{\pgfqpoint{6.381543in}{4.785018in}}%
\pgfpathlineto{\pgfqpoint{6.442434in}{4.788603in}}%
\pgfpathlineto{\pgfqpoint{6.503326in}{4.792025in}}%
\pgfpathlineto{\pgfqpoint{6.564217in}{4.795291in}}%
\pgfpathlineto{\pgfqpoint{6.625108in}{4.798408in}}%
\pgfpathlineto{\pgfqpoint{6.686000in}{4.801383in}}%
\pgfpathlineto{\pgfqpoint{6.746891in}{4.804222in}}%
\pgfpathlineto{\pgfqpoint{6.807783in}{4.806932in}}%
\pgfpathlineto{\pgfqpoint{6.868674in}{4.809519in}}%
\pgfpathlineto{\pgfqpoint{6.929566in}{4.811987in}}%
\pgfpathlineto{\pgfqpoint{6.990457in}{4.814343in}}%
\pgfpathlineto{\pgfqpoint{7.051348in}{4.816592in}}%
\pgfpathlineto{\pgfqpoint{7.112240in}{4.818738in}}%
\pgfpathlineto{\pgfqpoint{7.173131in}{4.820787in}}%
\pgfpathlineto{\pgfqpoint{7.234023in}{4.822742in}}%
\pgfpathlineto{\pgfqpoint{7.294914in}{4.824608in}}%
\pgfpathlineto{\pgfqpoint{7.355805in}{4.826389in}}%
\pgfusepath{stroke}%
\end{pgfscope}%
\begin{pgfscope}%
\pgfsetrectcap%
\pgfsetmiterjoin%
\pgfsetlinewidth{0.000000pt}%
\definecolor{currentstroke}{rgb}{1.000000,1.000000,1.000000}%
\pgfsetstrokecolor{currentstroke}%
\pgfsetdash{}{0pt}%
\pgfpathmoveto{\pgfqpoint{1.026143in}{0.898915in}}%
\pgfpathlineto{\pgfqpoint{1.026143in}{5.013411in}}%
\pgfusepath{}%
\end{pgfscope}%
\begin{pgfscope}%
\pgfsetrectcap%
\pgfsetmiterjoin%
\pgfsetlinewidth{0.000000pt}%
\definecolor{currentstroke}{rgb}{1.000000,1.000000,1.000000}%
\pgfsetstrokecolor{currentstroke}%
\pgfsetdash{}{0pt}%
\pgfpathmoveto{\pgfqpoint{7.657218in}{0.898915in}}%
\pgfpathlineto{\pgfqpoint{7.657218in}{5.013411in}}%
\pgfusepath{}%
\end{pgfscope}%
\begin{pgfscope}%
\pgfsetrectcap%
\pgfsetmiterjoin%
\pgfsetlinewidth{0.000000pt}%
\definecolor{currentstroke}{rgb}{1.000000,1.000000,1.000000}%
\pgfsetstrokecolor{currentstroke}%
\pgfsetdash{}{0pt}%
\pgfpathmoveto{\pgfqpoint{1.026143in}{0.898915in}}%
\pgfpathlineto{\pgfqpoint{7.657218in}{0.898915in}}%
\pgfusepath{}%
\end{pgfscope}%
\begin{pgfscope}%
\pgfsetrectcap%
\pgfsetmiterjoin%
\pgfsetlinewidth{0.000000pt}%
\definecolor{currentstroke}{rgb}{1.000000,1.000000,1.000000}%
\pgfsetstrokecolor{currentstroke}%
\pgfsetdash{}{0pt}%
\pgfpathmoveto{\pgfqpoint{1.026143in}{5.013411in}}%
\pgfpathlineto{\pgfqpoint{7.657218in}{5.013411in}}%
\pgfusepath{}%
\end{pgfscope}%
\begin{pgfscope}%
\definecolor{textcolor}{rgb}{0.150000,0.150000,0.150000}%
\pgfsetstrokecolor{textcolor}%
\pgfsetfillcolor{textcolor}%
\pgftext[x=3.349690in, y=2.416093in, left, base]{\color{textcolor}\rmfamily\fontsize{20.000000}{24.000000}\selectfont M\(\displaystyle _z\)(t) = 6.5(1\(\displaystyle -\)2\(\displaystyle exp[\frac{-t}{T_1}]\))\(\displaystyle +\)-0.45}%
\end{pgfscope}%
\begin{pgfscope}%
\definecolor{textcolor}{rgb}{0.150000,0.150000,0.150000}%
\pgfsetstrokecolor{textcolor}%
\pgfsetfillcolor{textcolor}%
\pgftext[x=3.349690in, y=1.922037in, left, base]{\color{textcolor}\rmfamily\fontsize{20.000000}{24.000000}\selectfont T\(\displaystyle _1\) = 1.710\(\displaystyle \pm\)0.11 ms}%
\end{pgfscope}%
\begin{pgfscope}%
\definecolor{textcolor}{rgb}{0.150000,0.150000,0.150000}%
\pgfsetstrokecolor{textcolor}%
\pgfsetfillcolor{textcolor}%
\pgftext[x=4.341681in,y=5.096745in,,base]{\color{textcolor}\rmfamily\fontsize{24.000000}{28.800000}\selectfont T\(\displaystyle _1\) measurement of 0.5 M CuSO\(\displaystyle _4\)}%
\end{pgfscope}%
\begin{pgfscope}%
\pgfsetbuttcap%
\pgfsetroundjoin%
\definecolor{currentfill}{rgb}{0.392157,0.584314,0.929412}%
\pgfsetfillcolor{currentfill}%
\pgfsetlinewidth{0.301125pt}%
\definecolor{currentstroke}{rgb}{0.392157,0.584314,0.929412}%
\pgfsetstrokecolor{currentstroke}%
\pgfsetdash{}{0pt}%
\pgfsys@defobject{currentmarker}{\pgfqpoint{-0.048611in}{-0.048611in}}{\pgfqpoint{0.048611in}{0.048611in}}{%
\pgfpathmoveto{\pgfqpoint{0.000000in}{-0.048611in}}%
\pgfpathcurveto{\pgfqpoint{0.012892in}{-0.048611in}}{\pgfqpoint{0.025257in}{-0.043489in}}{\pgfqpoint{0.034373in}{-0.034373in}}%
\pgfpathcurveto{\pgfqpoint{0.043489in}{-0.025257in}}{\pgfqpoint{0.048611in}{-0.012892in}}{\pgfqpoint{0.048611in}{0.000000in}}%
\pgfpathcurveto{\pgfqpoint{0.048611in}{0.012892in}}{\pgfqpoint{0.043489in}{0.025257in}}{\pgfqpoint{0.034373in}{0.034373in}}%
\pgfpathcurveto{\pgfqpoint{0.025257in}{0.043489in}}{\pgfqpoint{0.012892in}{0.048611in}}{\pgfqpoint{0.000000in}{0.048611in}}%
\pgfpathcurveto{\pgfqpoint{-0.012892in}{0.048611in}}{\pgfqpoint{-0.025257in}{0.043489in}}{\pgfqpoint{-0.034373in}{0.034373in}}%
\pgfpathcurveto{\pgfqpoint{-0.043489in}{0.025257in}}{\pgfqpoint{-0.048611in}{0.012892in}}{\pgfqpoint{-0.048611in}{0.000000in}}%
\pgfpathcurveto{\pgfqpoint{-0.048611in}{-0.012892in}}{\pgfqpoint{-0.043489in}{-0.025257in}}{\pgfqpoint{-0.034373in}{-0.034373in}}%
\pgfpathcurveto{\pgfqpoint{-0.025257in}{-0.043489in}}{\pgfqpoint{-0.012892in}{-0.048611in}}{\pgfqpoint{0.000000in}{-0.048611in}}%
\pgfpathlineto{\pgfqpoint{0.000000in}{-0.048611in}}%
\pgfpathclose%
\pgfusepath{stroke,fill}%
}%
\begin{pgfscope}%
\pgfsys@transformshift{5.126348in}{1.687326in}%
\pgfsys@useobject{currentmarker}{}%
\end{pgfscope}%
\end{pgfscope}%
\begin{pgfscope}%
\definecolor{textcolor}{rgb}{0.150000,0.150000,0.150000}%
\pgfsetstrokecolor{textcolor}%
\pgfsetfillcolor{textcolor}%
\pgftext[x=5.626348in,y=1.614409in,left,base]{\color{textcolor}\rmfamily\fontsize{20.000000}{24.000000}\selectfont Data}%
\end{pgfscope}%
\begin{pgfscope}%
\pgfsetroundcap%
\pgfsetroundjoin%
\pgfsetlinewidth{1.756562pt}%
\definecolor{currentstroke}{rgb}{0.298039,0.447059,0.690196}%
\pgfsetstrokecolor{currentstroke}%
\pgfsetdash{}{0pt}%
\pgfpathmoveto{\pgfqpoint{4.848570in}{1.303917in}}%
\pgfpathlineto{\pgfqpoint{5.126348in}{1.303917in}}%
\pgfpathlineto{\pgfqpoint{5.404125in}{1.303917in}}%
\pgfusepath{stroke}%
\end{pgfscope}%
\begin{pgfscope}%
\definecolor{textcolor}{rgb}{0.150000,0.150000,0.150000}%
\pgfsetstrokecolor{textcolor}%
\pgfsetfillcolor{textcolor}%
\pgftext[x=5.626348in,y=1.206694in,left,base]{\color{textcolor}\rmfamily\fontsize{20.000000}{24.000000}\selectfont Fitted Curve}%
\end{pgfscope}%
\end{pgfpicture}%
\makeatother%
\endgroup%
}
    \caption{T1 measurement of 0.5 M  CuSO$_{4} \boldsymbol{\cdot} $H$_2$O using thetwo-pulse method }
    \label{fig:0d5_t1}
  \end{figure}

  \begin{figure}[h]
    \resizebox{0.45\textwidth}{!}{%% Creator: Matplotlib, PGF backend
%%
%% To include the figure in your LaTeX document, write
%%   \input{<filename>.pgf}
%%
%% Make sure the required packages are loaded in your preamble
%%   \usepackage{pgf}
%%
%% Also ensure that all the required font packages are loaded; for instance,
%% the lmodern package is sometimes necessary when using math font.
%%   \usepackage{lmodern}
%%
%% Figures using additional raster images can only be included by \input if
%% they are in the same directory as the main LaTeX file. For loading figures
%% from other directories you can use the `import` package
%%   \usepackage{import}
%%
%% and then include the figures with
%%   \import{<path to file>}{<filename>.pgf}
%%
%% Matplotlib used the following preamble
%%   \usepackage{fontspec}
%%   \setmainfont{DejaVuSerif.ttf}[Path=\detokenize{/usr/share/matplotlib/mpl-data/fonts/ttf/}]
%%   \setsansfont{DejaVuSans.ttf}[Path=\detokenize{/usr/share/matplotlib/mpl-data/fonts/ttf/}]
%%   \setmonofont{DejaVuSansMono.ttf}[Path=\detokenize{/usr/share/matplotlib/mpl-data/fonts/ttf/}]
%%
\begingroup%
\makeatletter%
\begin{pgfpicture}%
\pgfpathrectangle{\pgfpointorigin}{\pgfqpoint{8.000000in}{5.500000in}}%
\pgfusepath{use as bounding box, clip}%
\begin{pgfscope}%
\pgfsetbuttcap%
\pgfsetmiterjoin%
\definecolor{currentfill}{rgb}{1.000000,1.000000,1.000000}%
\pgfsetfillcolor{currentfill}%
\pgfsetlinewidth{0.000000pt}%
\definecolor{currentstroke}{rgb}{1.000000,1.000000,1.000000}%
\pgfsetstrokecolor{currentstroke}%
\pgfsetdash{}{0pt}%
\pgfpathmoveto{\pgfqpoint{0.000000in}{0.000000in}}%
\pgfpathlineto{\pgfqpoint{8.000000in}{0.000000in}}%
\pgfpathlineto{\pgfqpoint{8.000000in}{5.500000in}}%
\pgfpathlineto{\pgfqpoint{0.000000in}{5.500000in}}%
\pgfpathlineto{\pgfqpoint{0.000000in}{0.000000in}}%
\pgfpathclose%
\pgfusepath{fill}%
\end{pgfscope}%
\begin{pgfscope}%
\pgfsetbuttcap%
\pgfsetmiterjoin%
\definecolor{currentfill}{rgb}{0.917647,0.917647,0.949020}%
\pgfsetfillcolor{currentfill}%
\pgfsetlinewidth{0.000000pt}%
\definecolor{currentstroke}{rgb}{0.000000,0.000000,0.000000}%
\pgfsetstrokecolor{currentstroke}%
\pgfsetstrokeopacity{0.000000}%
\pgfsetdash{}{0pt}%
\pgfpathmoveto{\pgfqpoint{1.291172in}{0.898915in}}%
\pgfpathlineto{\pgfqpoint{7.816307in}{0.898915in}}%
\pgfpathlineto{\pgfqpoint{7.816307in}{5.013411in}}%
\pgfpathlineto{\pgfqpoint{1.291172in}{5.013411in}}%
\pgfpathlineto{\pgfqpoint{1.291172in}{0.898915in}}%
\pgfpathclose%
\pgfusepath{fill}%
\end{pgfscope}%
\begin{pgfscope}%
\pgfpathrectangle{\pgfqpoint{1.291172in}{0.898915in}}{\pgfqpoint{6.525136in}{4.114497in}}%
\pgfusepath{clip}%
\pgfsetroundcap%
\pgfsetroundjoin%
\pgfsetlinewidth{1.003750pt}%
\definecolor{currentstroke}{rgb}{1.000000,1.000000,1.000000}%
\pgfsetstrokecolor{currentstroke}%
\pgfsetdash{}{0pt}%
\pgfpathmoveto{\pgfqpoint{1.383219in}{0.898915in}}%
\pgfpathlineto{\pgfqpoint{1.383219in}{5.013411in}}%
\pgfusepath{stroke}%
\end{pgfscope}%
\begin{pgfscope}%
\definecolor{textcolor}{rgb}{0.150000,0.150000,0.150000}%
\pgfsetstrokecolor{textcolor}%
\pgfsetfillcolor{textcolor}%
\pgftext[x=1.383219in,y=0.801692in,,top]{\color{textcolor}\rmfamily\fontsize{20.000000}{24.000000}\selectfont 0.0}%
\end{pgfscope}%
\begin{pgfscope}%
\pgfpathrectangle{\pgfqpoint{1.291172in}{0.898915in}}{\pgfqpoint{6.525136in}{4.114497in}}%
\pgfusepath{clip}%
\pgfsetroundcap%
\pgfsetroundjoin%
\pgfsetlinewidth{1.003750pt}%
\definecolor{currentstroke}{rgb}{1.000000,1.000000,1.000000}%
\pgfsetstrokecolor{currentstroke}%
\pgfsetdash{}{0pt}%
\pgfpathmoveto{\pgfqpoint{2.405967in}{0.898915in}}%
\pgfpathlineto{\pgfqpoint{2.405967in}{5.013411in}}%
\pgfusepath{stroke}%
\end{pgfscope}%
\begin{pgfscope}%
\definecolor{textcolor}{rgb}{0.150000,0.150000,0.150000}%
\pgfsetstrokecolor{textcolor}%
\pgfsetfillcolor{textcolor}%
\pgftext[x=2.405967in,y=0.801692in,,top]{\color{textcolor}\rmfamily\fontsize{20.000000}{24.000000}\selectfont 0.5}%
\end{pgfscope}%
\begin{pgfscope}%
\pgfpathrectangle{\pgfqpoint{1.291172in}{0.898915in}}{\pgfqpoint{6.525136in}{4.114497in}}%
\pgfusepath{clip}%
\pgfsetroundcap%
\pgfsetroundjoin%
\pgfsetlinewidth{1.003750pt}%
\definecolor{currentstroke}{rgb}{1.000000,1.000000,1.000000}%
\pgfsetstrokecolor{currentstroke}%
\pgfsetdash{}{0pt}%
\pgfpathmoveto{\pgfqpoint{3.428716in}{0.898915in}}%
\pgfpathlineto{\pgfqpoint{3.428716in}{5.013411in}}%
\pgfusepath{stroke}%
\end{pgfscope}%
\begin{pgfscope}%
\definecolor{textcolor}{rgb}{0.150000,0.150000,0.150000}%
\pgfsetstrokecolor{textcolor}%
\pgfsetfillcolor{textcolor}%
\pgftext[x=3.428716in,y=0.801692in,,top]{\color{textcolor}\rmfamily\fontsize{20.000000}{24.000000}\selectfont 1.0}%
\end{pgfscope}%
\begin{pgfscope}%
\pgfpathrectangle{\pgfqpoint{1.291172in}{0.898915in}}{\pgfqpoint{6.525136in}{4.114497in}}%
\pgfusepath{clip}%
\pgfsetroundcap%
\pgfsetroundjoin%
\pgfsetlinewidth{1.003750pt}%
\definecolor{currentstroke}{rgb}{1.000000,1.000000,1.000000}%
\pgfsetstrokecolor{currentstroke}%
\pgfsetdash{}{0pt}%
\pgfpathmoveto{\pgfqpoint{4.451465in}{0.898915in}}%
\pgfpathlineto{\pgfqpoint{4.451465in}{5.013411in}}%
\pgfusepath{stroke}%
\end{pgfscope}%
\begin{pgfscope}%
\definecolor{textcolor}{rgb}{0.150000,0.150000,0.150000}%
\pgfsetstrokecolor{textcolor}%
\pgfsetfillcolor{textcolor}%
\pgftext[x=4.451465in,y=0.801692in,,top]{\color{textcolor}\rmfamily\fontsize{20.000000}{24.000000}\selectfont 1.5}%
\end{pgfscope}%
\begin{pgfscope}%
\pgfpathrectangle{\pgfqpoint{1.291172in}{0.898915in}}{\pgfqpoint{6.525136in}{4.114497in}}%
\pgfusepath{clip}%
\pgfsetroundcap%
\pgfsetroundjoin%
\pgfsetlinewidth{1.003750pt}%
\definecolor{currentstroke}{rgb}{1.000000,1.000000,1.000000}%
\pgfsetstrokecolor{currentstroke}%
\pgfsetdash{}{0pt}%
\pgfpathmoveto{\pgfqpoint{5.474213in}{0.898915in}}%
\pgfpathlineto{\pgfqpoint{5.474213in}{5.013411in}}%
\pgfusepath{stroke}%
\end{pgfscope}%
\begin{pgfscope}%
\definecolor{textcolor}{rgb}{0.150000,0.150000,0.150000}%
\pgfsetstrokecolor{textcolor}%
\pgfsetfillcolor{textcolor}%
\pgftext[x=5.474213in,y=0.801692in,,top]{\color{textcolor}\rmfamily\fontsize{20.000000}{24.000000}\selectfont 2.0}%
\end{pgfscope}%
\begin{pgfscope}%
\pgfpathrectangle{\pgfqpoint{1.291172in}{0.898915in}}{\pgfqpoint{6.525136in}{4.114497in}}%
\pgfusepath{clip}%
\pgfsetroundcap%
\pgfsetroundjoin%
\pgfsetlinewidth{1.003750pt}%
\definecolor{currentstroke}{rgb}{1.000000,1.000000,1.000000}%
\pgfsetstrokecolor{currentstroke}%
\pgfsetdash{}{0pt}%
\pgfpathmoveto{\pgfqpoint{6.496962in}{0.898915in}}%
\pgfpathlineto{\pgfqpoint{6.496962in}{5.013411in}}%
\pgfusepath{stroke}%
\end{pgfscope}%
\begin{pgfscope}%
\definecolor{textcolor}{rgb}{0.150000,0.150000,0.150000}%
\pgfsetstrokecolor{textcolor}%
\pgfsetfillcolor{textcolor}%
\pgftext[x=6.496962in,y=0.801692in,,top]{\color{textcolor}\rmfamily\fontsize{20.000000}{24.000000}\selectfont 2.5}%
\end{pgfscope}%
\begin{pgfscope}%
\pgfpathrectangle{\pgfqpoint{1.291172in}{0.898915in}}{\pgfqpoint{6.525136in}{4.114497in}}%
\pgfusepath{clip}%
\pgfsetroundcap%
\pgfsetroundjoin%
\pgfsetlinewidth{1.003750pt}%
\definecolor{currentstroke}{rgb}{1.000000,1.000000,1.000000}%
\pgfsetstrokecolor{currentstroke}%
\pgfsetdash{}{0pt}%
\pgfpathmoveto{\pgfqpoint{7.519710in}{0.898915in}}%
\pgfpathlineto{\pgfqpoint{7.519710in}{5.013411in}}%
\pgfusepath{stroke}%
\end{pgfscope}%
\begin{pgfscope}%
\definecolor{textcolor}{rgb}{0.150000,0.150000,0.150000}%
\pgfsetstrokecolor{textcolor}%
\pgfsetfillcolor{textcolor}%
\pgftext[x=7.519710in,y=0.801692in,,top]{\color{textcolor}\rmfamily\fontsize{20.000000}{24.000000}\selectfont 3.0}%
\end{pgfscope}%
\begin{pgfscope}%
\definecolor{textcolor}{rgb}{0.150000,0.150000,0.150000}%
\pgfsetstrokecolor{textcolor}%
\pgfsetfillcolor{textcolor}%
\pgftext[x=4.553739in,y=0.477311in,,top]{\color{textcolor}\rmfamily\fontsize{24.000000}{28.800000}\selectfont Delay Time (ms)}%
\end{pgfscope}%
\begin{pgfscope}%
\pgfpathrectangle{\pgfqpoint{1.291172in}{0.898915in}}{\pgfqpoint{6.525136in}{4.114497in}}%
\pgfusepath{clip}%
\pgfsetroundcap%
\pgfsetroundjoin%
\pgfsetlinewidth{1.003750pt}%
\definecolor{currentstroke}{rgb}{1.000000,1.000000,1.000000}%
\pgfsetstrokecolor{currentstroke}%
\pgfsetdash{}{0pt}%
\pgfpathmoveto{\pgfqpoint{1.291172in}{1.562066in}}%
\pgfpathlineto{\pgfqpoint{7.816307in}{1.562066in}}%
\pgfusepath{stroke}%
\end{pgfscope}%
\begin{pgfscope}%
\definecolor{textcolor}{rgb}{0.150000,0.150000,0.150000}%
\pgfsetstrokecolor{textcolor}%
\pgfsetfillcolor{textcolor}%
\pgftext[x=0.528147in, y=1.456543in, left, base]{\color{textcolor}\rmfamily\fontsize{20.000000}{24.000000}\selectfont \ensuremath{-}5.0}%
\end{pgfscope}%
\begin{pgfscope}%
\pgfpathrectangle{\pgfqpoint{1.291172in}{0.898915in}}{\pgfqpoint{6.525136in}{4.114497in}}%
\pgfusepath{clip}%
\pgfsetroundcap%
\pgfsetroundjoin%
\pgfsetlinewidth{1.003750pt}%
\definecolor{currentstroke}{rgb}{1.000000,1.000000,1.000000}%
\pgfsetstrokecolor{currentstroke}%
\pgfsetdash{}{0pt}%
\pgfpathmoveto{\pgfqpoint{1.291172in}{2.260745in}}%
\pgfpathlineto{\pgfqpoint{7.816307in}{2.260745in}}%
\pgfusepath{stroke}%
\end{pgfscope}%
\begin{pgfscope}%
\definecolor{textcolor}{rgb}{0.150000,0.150000,0.150000}%
\pgfsetstrokecolor{textcolor}%
\pgfsetfillcolor{textcolor}%
\pgftext[x=0.528147in, y=2.155222in, left, base]{\color{textcolor}\rmfamily\fontsize{20.000000}{24.000000}\selectfont \ensuremath{-}2.5}%
\end{pgfscope}%
\begin{pgfscope}%
\pgfpathrectangle{\pgfqpoint{1.291172in}{0.898915in}}{\pgfqpoint{6.525136in}{4.114497in}}%
\pgfusepath{clip}%
\pgfsetroundcap%
\pgfsetroundjoin%
\pgfsetlinewidth{1.003750pt}%
\definecolor{currentstroke}{rgb}{1.000000,1.000000,1.000000}%
\pgfsetstrokecolor{currentstroke}%
\pgfsetdash{}{0pt}%
\pgfpathmoveto{\pgfqpoint{1.291172in}{2.959424in}}%
\pgfpathlineto{\pgfqpoint{7.816307in}{2.959424in}}%
\pgfusepath{stroke}%
\end{pgfscope}%
\begin{pgfscope}%
\definecolor{textcolor}{rgb}{0.150000,0.150000,0.150000}%
\pgfsetstrokecolor{textcolor}%
\pgfsetfillcolor{textcolor}%
\pgftext[x=0.752190in, y=2.853901in, left, base]{\color{textcolor}\rmfamily\fontsize{20.000000}{24.000000}\selectfont 0.0}%
\end{pgfscope}%
\begin{pgfscope}%
\pgfpathrectangle{\pgfqpoint{1.291172in}{0.898915in}}{\pgfqpoint{6.525136in}{4.114497in}}%
\pgfusepath{clip}%
\pgfsetroundcap%
\pgfsetroundjoin%
\pgfsetlinewidth{1.003750pt}%
\definecolor{currentstroke}{rgb}{1.000000,1.000000,1.000000}%
\pgfsetstrokecolor{currentstroke}%
\pgfsetdash{}{0pt}%
\pgfpathmoveto{\pgfqpoint{1.291172in}{3.658103in}}%
\pgfpathlineto{\pgfqpoint{7.816307in}{3.658103in}}%
\pgfusepath{stroke}%
\end{pgfscope}%
\begin{pgfscope}%
\definecolor{textcolor}{rgb}{0.150000,0.150000,0.150000}%
\pgfsetstrokecolor{textcolor}%
\pgfsetfillcolor{textcolor}%
\pgftext[x=0.752190in, y=3.552580in, left, base]{\color{textcolor}\rmfamily\fontsize{20.000000}{24.000000}\selectfont 2.5}%
\end{pgfscope}%
\begin{pgfscope}%
\pgfpathrectangle{\pgfqpoint{1.291172in}{0.898915in}}{\pgfqpoint{6.525136in}{4.114497in}}%
\pgfusepath{clip}%
\pgfsetroundcap%
\pgfsetroundjoin%
\pgfsetlinewidth{1.003750pt}%
\definecolor{currentstroke}{rgb}{1.000000,1.000000,1.000000}%
\pgfsetstrokecolor{currentstroke}%
\pgfsetdash{}{0pt}%
\pgfpathmoveto{\pgfqpoint{1.291172in}{4.356782in}}%
\pgfpathlineto{\pgfqpoint{7.816307in}{4.356782in}}%
\pgfusepath{stroke}%
\end{pgfscope}%
\begin{pgfscope}%
\definecolor{textcolor}{rgb}{0.150000,0.150000,0.150000}%
\pgfsetstrokecolor{textcolor}%
\pgfsetfillcolor{textcolor}%
\pgftext[x=0.752190in, y=4.251259in, left, base]{\color{textcolor}\rmfamily\fontsize{20.000000}{24.000000}\selectfont 5.0}%
\end{pgfscope}%
\begin{pgfscope}%
\definecolor{textcolor}{rgb}{0.150000,0.150000,0.150000}%
\pgfsetstrokecolor{textcolor}%
\pgfsetfillcolor{textcolor}%
\pgftext[x=0.472591in,y=2.956163in,,bottom,rotate=90.000000]{\color{textcolor}\rmfamily\fontsize{24.000000}{28.800000}\selectfont Pulse Amplitude (V)}%
\end{pgfscope}%
\begin{pgfscope}%
\pgfpathrectangle{\pgfqpoint{1.291172in}{0.898915in}}{\pgfqpoint{6.525136in}{4.114497in}}%
\pgfusepath{clip}%
\pgfsetbuttcap%
\pgfsetroundjoin%
\definecolor{currentfill}{rgb}{0.392157,0.584314,0.929412}%
\pgfsetfillcolor{currentfill}%
\pgfsetlinewidth{0.301125pt}%
\definecolor{currentstroke}{rgb}{0.392157,0.584314,0.929412}%
\pgfsetstrokecolor{currentstroke}%
\pgfsetdash{}{0pt}%
\pgfsys@defobject{currentmarker}{\pgfqpoint{-0.048611in}{-0.048611in}}{\pgfqpoint{0.048611in}{0.048611in}}{%
\pgfpathmoveto{\pgfqpoint{0.000000in}{-0.048611in}}%
\pgfpathcurveto{\pgfqpoint{0.012892in}{-0.048611in}}{\pgfqpoint{0.025257in}{-0.043489in}}{\pgfqpoint{0.034373in}{-0.034373in}}%
\pgfpathcurveto{\pgfqpoint{0.043489in}{-0.025257in}}{\pgfqpoint{0.048611in}{-0.012892in}}{\pgfqpoint{0.048611in}{0.000000in}}%
\pgfpathcurveto{\pgfqpoint{0.048611in}{0.012892in}}{\pgfqpoint{0.043489in}{0.025257in}}{\pgfqpoint{0.034373in}{0.034373in}}%
\pgfpathcurveto{\pgfqpoint{0.025257in}{0.043489in}}{\pgfqpoint{0.012892in}{0.048611in}}{\pgfqpoint{0.000000in}{0.048611in}}%
\pgfpathcurveto{\pgfqpoint{-0.012892in}{0.048611in}}{\pgfqpoint{-0.025257in}{0.043489in}}{\pgfqpoint{-0.034373in}{0.034373in}}%
\pgfpathcurveto{\pgfqpoint{-0.043489in}{0.025257in}}{\pgfqpoint{-0.048611in}{0.012892in}}{\pgfqpoint{-0.048611in}{0.000000in}}%
\pgfpathcurveto{\pgfqpoint{-0.048611in}{-0.012892in}}{\pgfqpoint{-0.043489in}{-0.025257in}}{\pgfqpoint{-0.034373in}{-0.034373in}}%
\pgfpathcurveto{\pgfqpoint{-0.025257in}{-0.043489in}}{\pgfqpoint{-0.012892in}{-0.048611in}}{\pgfqpoint{0.000000in}{-0.048611in}}%
\pgfpathlineto{\pgfqpoint{0.000000in}{-0.048611in}}%
\pgfpathclose%
\pgfusepath{stroke,fill}%
}%
\begin{pgfscope}%
\pgfsys@transformshift{1.587769in}{1.215521in}%
\pgfsys@useobject{currentmarker}{}%
\end{pgfscope}%
\begin{pgfscope}%
\pgfsys@transformshift{1.792318in}{1.584423in}%
\pgfsys@useobject{currentmarker}{}%
\end{pgfscope}%
\begin{pgfscope}%
\pgfsys@transformshift{1.996868in}{1.908611in}%
\pgfsys@useobject{currentmarker}{}%
\end{pgfscope}%
\begin{pgfscope}%
\pgfsys@transformshift{2.201418in}{2.221619in}%
\pgfsys@useobject{currentmarker}{}%
\end{pgfscope}%
\begin{pgfscope}%
\pgfsys@transformshift{2.405967in}{2.512269in}%
\pgfsys@useobject{currentmarker}{}%
\end{pgfscope}%
\begin{pgfscope}%
\pgfsys@transformshift{2.610517in}{2.747026in}%
\pgfsys@useobject{currentmarker}{}%
\end{pgfscope}%
\begin{pgfscope}%
\pgfsys@transformshift{2.815067in}{3.317148in}%
\pgfsys@useobject{currentmarker}{}%
\end{pgfscope}%
\begin{pgfscope}%
\pgfsys@transformshift{3.019617in}{3.529546in}%
\pgfsys@useobject{currentmarker}{}%
\end{pgfscope}%
\begin{pgfscope}%
\pgfsys@transformshift{3.224166in}{3.686050in}%
\pgfsys@useobject{currentmarker}{}%
\end{pgfscope}%
\begin{pgfscope}%
\pgfsys@transformshift{3.428716in}{3.831375in}%
\pgfsys@useobject{currentmarker}{}%
\end{pgfscope}%
\begin{pgfscope}%
\pgfsys@transformshift{4.451465in}{4.334424in}%
\pgfsys@useobject{currentmarker}{}%
\end{pgfscope}%
\begin{pgfscope}%
\pgfsys@transformshift{5.474213in}{4.569181in}%
\pgfsys@useobject{currentmarker}{}%
\end{pgfscope}%
\begin{pgfscope}%
\pgfsys@transformshift{6.496962in}{4.725685in}%
\pgfsys@useobject{currentmarker}{}%
\end{pgfscope}%
\begin{pgfscope}%
\pgfsys@transformshift{7.519710in}{4.792758in}%
\pgfsys@useobject{currentmarker}{}%
\end{pgfscope}%
\end{pgfscope}%
\begin{pgfscope}%
\pgfpathrectangle{\pgfqpoint{1.291172in}{0.898915in}}{\pgfqpoint{6.525136in}{4.114497in}}%
\pgfusepath{clip}%
\pgfsetroundcap%
\pgfsetroundjoin%
\pgfsetlinewidth{1.756562pt}%
\definecolor{currentstroke}{rgb}{0.298039,0.447059,0.690196}%
\pgfsetstrokecolor{currentstroke}%
\pgfsetdash{}{0pt}%
\pgfpathmoveto{\pgfqpoint{1.587769in}{1.085937in}}%
\pgfpathlineto{\pgfqpoint{1.647687in}{1.231711in}}%
\pgfpathlineto{\pgfqpoint{1.707606in}{1.371925in}}%
\pgfpathlineto{\pgfqpoint{1.767524in}{1.506790in}}%
\pgfpathlineto{\pgfqpoint{1.827443in}{1.636512in}}%
\pgfpathlineto{\pgfqpoint{1.887362in}{1.761285in}}%
\pgfpathlineto{\pgfqpoint{1.947280in}{1.881299in}}%
\pgfpathlineto{\pgfqpoint{2.007199in}{1.996736in}}%
\pgfpathlineto{\pgfqpoint{2.067117in}{2.107769in}}%
\pgfpathlineto{\pgfqpoint{2.127036in}{2.214567in}}%
\pgfpathlineto{\pgfqpoint{2.186955in}{2.317292in}}%
\pgfpathlineto{\pgfqpoint{2.246873in}{2.416098in}}%
\pgfpathlineto{\pgfqpoint{2.306792in}{2.511136in}}%
\pgfpathlineto{\pgfqpoint{2.366710in}{2.602548in}}%
\pgfpathlineto{\pgfqpoint{2.426629in}{2.690474in}}%
\pgfpathlineto{\pgfqpoint{2.486548in}{2.775046in}}%
\pgfpathlineto{\pgfqpoint{2.546466in}{2.856392in}}%
\pgfpathlineto{\pgfqpoint{2.606385in}{2.934636in}}%
\pgfpathlineto{\pgfqpoint{2.666303in}{3.009895in}}%
\pgfpathlineto{\pgfqpoint{2.726222in}{3.082283in}}%
\pgfpathlineto{\pgfqpoint{2.786141in}{3.151911in}}%
\pgfpathlineto{\pgfqpoint{2.846059in}{3.218882in}}%
\pgfpathlineto{\pgfqpoint{2.905978in}{3.283299in}}%
\pgfpathlineto{\pgfqpoint{2.965896in}{3.345259in}}%
\pgfpathlineto{\pgfqpoint{3.025815in}{3.404856in}}%
\pgfpathlineto{\pgfqpoint{3.085734in}{3.462179in}}%
\pgfpathlineto{\pgfqpoint{3.145652in}{3.517316in}}%
\pgfpathlineto{\pgfqpoint{3.205571in}{3.570350in}}%
\pgfpathlineto{\pgfqpoint{3.265489in}{3.621361in}}%
\pgfpathlineto{\pgfqpoint{3.325408in}{3.670426in}}%
\pgfpathlineto{\pgfqpoint{3.385327in}{3.717620in}}%
\pgfpathlineto{\pgfqpoint{3.445245in}{3.763013in}}%
\pgfpathlineto{\pgfqpoint{3.505164in}{3.806676in}}%
\pgfpathlineto{\pgfqpoint{3.565082in}{3.848672in}}%
\pgfpathlineto{\pgfqpoint{3.625001in}{3.889067in}}%
\pgfpathlineto{\pgfqpoint{3.684920in}{3.927921in}}%
\pgfpathlineto{\pgfqpoint{3.744838in}{3.965293in}}%
\pgfpathlineto{\pgfqpoint{3.804757in}{4.001240in}}%
\pgfpathlineto{\pgfqpoint{3.864675in}{4.035816in}}%
\pgfpathlineto{\pgfqpoint{3.924594in}{4.069072in}}%
\pgfpathlineto{\pgfqpoint{3.984513in}{4.101060in}}%
\pgfpathlineto{\pgfqpoint{4.044431in}{4.131829in}}%
\pgfpathlineto{\pgfqpoint{4.104350in}{4.161423in}}%
\pgfpathlineto{\pgfqpoint{4.164268in}{4.189889in}}%
\pgfpathlineto{\pgfqpoint{4.224187in}{4.217269in}}%
\pgfpathlineto{\pgfqpoint{4.284106in}{4.243604in}}%
\pgfpathlineto{\pgfqpoint{4.344024in}{4.268935in}}%
\pgfpathlineto{\pgfqpoint{4.403943in}{4.293300in}}%
\pgfpathlineto{\pgfqpoint{4.463861in}{4.316735in}}%
\pgfpathlineto{\pgfqpoint{4.523780in}{4.339277in}}%
\pgfpathlineto{\pgfqpoint{4.583699in}{4.360959in}}%
\pgfpathlineto{\pgfqpoint{4.643617in}{4.381813in}}%
\pgfpathlineto{\pgfqpoint{4.703536in}{4.401873in}}%
\pgfpathlineto{\pgfqpoint{4.763454in}{4.421167in}}%
\pgfpathlineto{\pgfqpoint{4.823373in}{4.439725in}}%
\pgfpathlineto{\pgfqpoint{4.883292in}{4.457575in}}%
\pgfpathlineto{\pgfqpoint{4.943210in}{4.474745in}}%
\pgfpathlineto{\pgfqpoint{5.003129in}{4.491260in}}%
\pgfpathlineto{\pgfqpoint{5.063047in}{4.507144in}}%
\pgfpathlineto{\pgfqpoint{5.122966in}{4.522423in}}%
\pgfpathlineto{\pgfqpoint{5.182885in}{4.537119in}}%
\pgfpathlineto{\pgfqpoint{5.242803in}{4.551254in}}%
\pgfpathlineto{\pgfqpoint{5.302722in}{4.564851in}}%
\pgfpathlineto{\pgfqpoint{5.362640in}{4.577928in}}%
\pgfpathlineto{\pgfqpoint{5.422559in}{4.590507in}}%
\pgfpathlineto{\pgfqpoint{5.482478in}{4.602606in}}%
\pgfpathlineto{\pgfqpoint{5.542396in}{4.614244in}}%
\pgfpathlineto{\pgfqpoint{5.602315in}{4.625438in}}%
\pgfpathlineto{\pgfqpoint{5.662233in}{4.636204in}}%
\pgfpathlineto{\pgfqpoint{5.722152in}{4.646560in}}%
\pgfpathlineto{\pgfqpoint{5.782071in}{4.656521in}}%
\pgfpathlineto{\pgfqpoint{5.841989in}{4.666103in}}%
\pgfpathlineto{\pgfqpoint{5.901908in}{4.675318in}}%
\pgfpathlineto{\pgfqpoint{5.961826in}{4.684182in}}%
\pgfpathlineto{\pgfqpoint{6.021745in}{4.692708in}}%
\pgfpathlineto{\pgfqpoint{6.081664in}{4.700909in}}%
\pgfpathlineto{\pgfqpoint{6.141582in}{4.708797in}}%
\pgfpathlineto{\pgfqpoint{6.201501in}{4.716384in}}%
\pgfpathlineto{\pgfqpoint{6.261419in}{4.723682in}}%
\pgfpathlineto{\pgfqpoint{6.321338in}{4.730701in}}%
\pgfpathlineto{\pgfqpoint{6.381257in}{4.737453in}}%
\pgfpathlineto{\pgfqpoint{6.441175in}{4.743947in}}%
\pgfpathlineto{\pgfqpoint{6.501094in}{4.750194in}}%
\pgfpathlineto{\pgfqpoint{6.561012in}{4.756202in}}%
\pgfpathlineto{\pgfqpoint{6.620931in}{4.761981in}}%
\pgfpathlineto{\pgfqpoint{6.680850in}{4.767539in}}%
\pgfpathlineto{\pgfqpoint{6.740768in}{4.772886in}}%
\pgfpathlineto{\pgfqpoint{6.800687in}{4.778028in}}%
\pgfpathlineto{\pgfqpoint{6.860606in}{4.782975in}}%
\pgfpathlineto{\pgfqpoint{6.920524in}{4.787733in}}%
\pgfpathlineto{\pgfqpoint{6.980443in}{4.792309in}}%
\pgfpathlineto{\pgfqpoint{7.040361in}{4.796711in}}%
\pgfpathlineto{\pgfqpoint{7.100280in}{4.800945in}}%
\pgfpathlineto{\pgfqpoint{7.160199in}{4.805017in}}%
\pgfpathlineto{\pgfqpoint{7.220117in}{4.808934in}}%
\pgfpathlineto{\pgfqpoint{7.280036in}{4.812702in}}%
\pgfpathlineto{\pgfqpoint{7.339954in}{4.816326in}}%
\pgfpathlineto{\pgfqpoint{7.399873in}{4.819811in}}%
\pgfpathlineto{\pgfqpoint{7.459792in}{4.823164in}}%
\pgfpathlineto{\pgfqpoint{7.519710in}{4.826389in}}%
\pgfusepath{stroke}%
\end{pgfscope}%
\begin{pgfscope}%
\pgfsetrectcap%
\pgfsetmiterjoin%
\pgfsetlinewidth{0.000000pt}%
\definecolor{currentstroke}{rgb}{1.000000,1.000000,1.000000}%
\pgfsetstrokecolor{currentstroke}%
\pgfsetdash{}{0pt}%
\pgfpathmoveto{\pgfqpoint{1.291172in}{0.898915in}}%
\pgfpathlineto{\pgfqpoint{1.291172in}{5.013411in}}%
\pgfusepath{}%
\end{pgfscope}%
\begin{pgfscope}%
\pgfsetrectcap%
\pgfsetmiterjoin%
\pgfsetlinewidth{0.000000pt}%
\definecolor{currentstroke}{rgb}{1.000000,1.000000,1.000000}%
\pgfsetstrokecolor{currentstroke}%
\pgfsetdash{}{0pt}%
\pgfpathmoveto{\pgfqpoint{7.816307in}{0.898915in}}%
\pgfpathlineto{\pgfqpoint{7.816307in}{5.013411in}}%
\pgfusepath{}%
\end{pgfscope}%
\begin{pgfscope}%
\pgfsetrectcap%
\pgfsetmiterjoin%
\pgfsetlinewidth{0.000000pt}%
\definecolor{currentstroke}{rgb}{1.000000,1.000000,1.000000}%
\pgfsetstrokecolor{currentstroke}%
\pgfsetdash{}{0pt}%
\pgfpathmoveto{\pgfqpoint{1.291172in}{0.898915in}}%
\pgfpathlineto{\pgfqpoint{7.816307in}{0.898915in}}%
\pgfusepath{}%
\end{pgfscope}%
\begin{pgfscope}%
\pgfsetrectcap%
\pgfsetmiterjoin%
\pgfsetlinewidth{0.000000pt}%
\definecolor{currentstroke}{rgb}{1.000000,1.000000,1.000000}%
\pgfsetstrokecolor{currentstroke}%
\pgfsetdash{}{0pt}%
\pgfpathmoveto{\pgfqpoint{1.291172in}{5.013411in}}%
\pgfpathlineto{\pgfqpoint{7.816307in}{5.013411in}}%
\pgfusepath{}%
\end{pgfscope}%
\begin{pgfscope}%
\definecolor{textcolor}{rgb}{0.150000,0.150000,0.150000}%
\pgfsetstrokecolor{textcolor}%
\pgfsetfillcolor{textcolor}%
\pgftext[x=3.224166in, y=2.475329in, left, base]{\color{textcolor}\rmfamily\fontsize{20.000000}{24.000000}\selectfont M\(\displaystyle _z\)(t) = 7.81(1\(\displaystyle -\)2\(\displaystyle exp[\frac{-t}{T_1}]\))\(\displaystyle +\)-0.84}%
\end{pgfscope}%
\begin{pgfscope}%
\definecolor{textcolor}{rgb}{0.150000,0.150000,0.150000}%
\pgfsetstrokecolor{textcolor}%
\pgfsetfillcolor{textcolor}%
\pgftext[x=3.224166in, y=1.981273in, left, base]{\color{textcolor}\rmfamily\fontsize{20.000000}{24.000000}\selectfont T\(\displaystyle _1\) = 0.7532\(\displaystyle \pm\)0.0535 ms}%
\end{pgfscope}%
\begin{pgfscope}%
\definecolor{textcolor}{rgb}{0.150000,0.150000,0.150000}%
\pgfsetstrokecolor{textcolor}%
\pgfsetfillcolor{textcolor}%
\pgftext[x=4.553739in,y=5.096745in,,base]{\color{textcolor}\rmfamily\fontsize{24.000000}{28.800000}\selectfont T\(\displaystyle _1\) measurement of 1 M CuSO\(\displaystyle _4\)}%
\end{pgfscope}%
\begin{pgfscope}%
\pgfsetbuttcap%
\pgfsetroundjoin%
\definecolor{currentfill}{rgb}{0.392157,0.584314,0.929412}%
\pgfsetfillcolor{currentfill}%
\pgfsetlinewidth{0.301125pt}%
\definecolor{currentstroke}{rgb}{0.392157,0.584314,0.929412}%
\pgfsetstrokecolor{currentstroke}%
\pgfsetdash{}{0pt}%
\pgfsys@defobject{currentmarker}{\pgfqpoint{-0.048611in}{-0.048611in}}{\pgfqpoint{0.048611in}{0.048611in}}{%
\pgfpathmoveto{\pgfqpoint{0.000000in}{-0.048611in}}%
\pgfpathcurveto{\pgfqpoint{0.012892in}{-0.048611in}}{\pgfqpoint{0.025257in}{-0.043489in}}{\pgfqpoint{0.034373in}{-0.034373in}}%
\pgfpathcurveto{\pgfqpoint{0.043489in}{-0.025257in}}{\pgfqpoint{0.048611in}{-0.012892in}}{\pgfqpoint{0.048611in}{0.000000in}}%
\pgfpathcurveto{\pgfqpoint{0.048611in}{0.012892in}}{\pgfqpoint{0.043489in}{0.025257in}}{\pgfqpoint{0.034373in}{0.034373in}}%
\pgfpathcurveto{\pgfqpoint{0.025257in}{0.043489in}}{\pgfqpoint{0.012892in}{0.048611in}}{\pgfqpoint{0.000000in}{0.048611in}}%
\pgfpathcurveto{\pgfqpoint{-0.012892in}{0.048611in}}{\pgfqpoint{-0.025257in}{0.043489in}}{\pgfqpoint{-0.034373in}{0.034373in}}%
\pgfpathcurveto{\pgfqpoint{-0.043489in}{0.025257in}}{\pgfqpoint{-0.048611in}{0.012892in}}{\pgfqpoint{-0.048611in}{0.000000in}}%
\pgfpathcurveto{\pgfqpoint{-0.048611in}{-0.012892in}}{\pgfqpoint{-0.043489in}{-0.025257in}}{\pgfqpoint{-0.034373in}{-0.034373in}}%
\pgfpathcurveto{\pgfqpoint{-0.025257in}{-0.043489in}}{\pgfqpoint{-0.012892in}{-0.048611in}}{\pgfqpoint{0.000000in}{-0.048611in}}%
\pgfpathlineto{\pgfqpoint{0.000000in}{-0.048611in}}%
\pgfpathclose%
\pgfusepath{stroke,fill}%
}%
\begin{pgfscope}%
\pgfsys@transformshift{5.285437in}{1.687326in}%
\pgfsys@useobject{currentmarker}{}%
\end{pgfscope}%
\end{pgfscope}%
\begin{pgfscope}%
\definecolor{textcolor}{rgb}{0.150000,0.150000,0.150000}%
\pgfsetstrokecolor{textcolor}%
\pgfsetfillcolor{textcolor}%
\pgftext[x=5.785437in,y=1.614409in,left,base]{\color{textcolor}\rmfamily\fontsize{20.000000}{24.000000}\selectfont Data}%
\end{pgfscope}%
\begin{pgfscope}%
\pgfsetroundcap%
\pgfsetroundjoin%
\pgfsetlinewidth{1.756562pt}%
\definecolor{currentstroke}{rgb}{0.298039,0.447059,0.690196}%
\pgfsetstrokecolor{currentstroke}%
\pgfsetdash{}{0pt}%
\pgfpathmoveto{\pgfqpoint{5.007659in}{1.303917in}}%
\pgfpathlineto{\pgfqpoint{5.285437in}{1.303917in}}%
\pgfpathlineto{\pgfqpoint{5.563215in}{1.303917in}}%
\pgfusepath{stroke}%
\end{pgfscope}%
\begin{pgfscope}%
\definecolor{textcolor}{rgb}{0.150000,0.150000,0.150000}%
\pgfsetstrokecolor{textcolor}%
\pgfsetfillcolor{textcolor}%
\pgftext[x=5.785437in,y=1.206694in,left,base]{\color{textcolor}\rmfamily\fontsize{20.000000}{24.000000}\selectfont Fitted Curve}%
\end{pgfscope}%
\end{pgfpicture}%
\makeatother%
\endgroup%
}
    \caption{T1 measurement of 1.0 M  CuSO$_{4} \boldsymbol{\cdot} $H$_2$O using the two-pulse method }
    \label{fig:1d0_t1}
  \end{figure}


  %%%%%%%%%%%%%%%%%%%%%%%%%%%%%%%%%
  %%%%%%%%%% CuSo4 T2 %%%%%%%%%%%%%
  %%%%%%%%%%%%%%%%%%%%%%%%%%%%%%%%%

  \begin{figure}[h]
    \resizebox{0.45\textwidth}{!}{%% Creator: Matplotlib, PGF backend
%%
%% To include the figure in your LaTeX document, write
%%   \input{<filename>.pgf}
%%
%% Make sure the required packages are loaded in your preamble
%%   \usepackage{pgf}
%%
%% Also ensure that all the required font packages are loaded; for instance,
%% the lmodern package is sometimes necessary when using math font.
%%   \usepackage{lmodern}
%%
%% Figures using additional raster images can only be included by \input if
%% they are in the same directory as the main LaTeX file. For loading figures
%% from other directories you can use the `import` package
%%   \usepackage{import}
%%
%% and then include the figures with
%%   \import{<path to file>}{<filename>.pgf}
%%
%% Matplotlib used the following preamble
%%   \usepackage{fontspec}
%%   \setmainfont{DejaVuSerif.ttf}[Path=\detokenize{/usr/share/matplotlib/mpl-data/fonts/ttf/}]
%%   \setsansfont{DejaVuSans.ttf}[Path=\detokenize{/usr/share/matplotlib/mpl-data/fonts/ttf/}]
%%   \setmonofont{DejaVuSansMono.ttf}[Path=\detokenize{/usr/share/matplotlib/mpl-data/fonts/ttf/}]
%%
\begingroup%
\makeatletter%
\begin{pgfpicture}%
\pgfpathrectangle{\pgfpointorigin}{\pgfqpoint{8.000000in}{5.500000in}}%
\pgfusepath{use as bounding box, clip}%
\begin{pgfscope}%
\pgfsetbuttcap%
\pgfsetmiterjoin%
\definecolor{currentfill}{rgb}{1.000000,1.000000,1.000000}%
\pgfsetfillcolor{currentfill}%
\pgfsetlinewidth{0.000000pt}%
\definecolor{currentstroke}{rgb}{1.000000,1.000000,1.000000}%
\pgfsetstrokecolor{currentstroke}%
\pgfsetdash{}{0pt}%
\pgfpathmoveto{\pgfqpoint{0.000000in}{0.000000in}}%
\pgfpathlineto{\pgfqpoint{8.000000in}{0.000000in}}%
\pgfpathlineto{\pgfqpoint{8.000000in}{5.500000in}}%
\pgfpathlineto{\pgfqpoint{0.000000in}{5.500000in}}%
\pgfpathlineto{\pgfqpoint{0.000000in}{0.000000in}}%
\pgfpathclose%
\pgfusepath{fill}%
\end{pgfscope}%
\begin{pgfscope}%
\pgfsetbuttcap%
\pgfsetmiterjoin%
\definecolor{currentfill}{rgb}{0.917647,0.917647,0.949020}%
\pgfsetfillcolor{currentfill}%
\pgfsetlinewidth{0.000000pt}%
\definecolor{currentstroke}{rgb}{0.000000,0.000000,0.000000}%
\pgfsetstrokecolor{currentstroke}%
\pgfsetstrokeopacity{0.000000}%
\pgfsetdash{}{0pt}%
\pgfpathmoveto{\pgfqpoint{0.802099in}{0.894194in}}%
\pgfpathlineto{\pgfqpoint{7.657874in}{0.894194in}}%
\pgfpathlineto{\pgfqpoint{7.657874in}{5.013411in}}%
\pgfpathlineto{\pgfqpoint{0.802099in}{5.013411in}}%
\pgfpathlineto{\pgfqpoint{0.802099in}{0.894194in}}%
\pgfpathclose%
\pgfusepath{fill}%
\end{pgfscope}%
\begin{pgfscope}%
\pgfpathrectangle{\pgfqpoint{0.802099in}{0.894194in}}{\pgfqpoint{6.855775in}{4.119217in}}%
\pgfusepath{clip}%
\pgfsetroundcap%
\pgfsetroundjoin%
\pgfsetlinewidth{1.003750pt}%
\definecolor{currentstroke}{rgb}{1.000000,1.000000,1.000000}%
\pgfsetstrokecolor{currentstroke}%
\pgfsetdash{}{0pt}%
\pgfpathmoveto{\pgfqpoint{0.993869in}{0.894194in}}%
\pgfpathlineto{\pgfqpoint{0.993869in}{5.013411in}}%
\pgfusepath{stroke}%
\end{pgfscope}%
\begin{pgfscope}%
\definecolor{textcolor}{rgb}{0.150000,0.150000,0.150000}%
\pgfsetstrokecolor{textcolor}%
\pgfsetfillcolor{textcolor}%
\pgftext[x=0.993869in,y=0.796972in,,top]{\color{textcolor}\rmfamily\fontsize{20.000000}{24.000000}\selectfont 0}%
\end{pgfscope}%
\begin{pgfscope}%
\pgfpathrectangle{\pgfqpoint{0.802099in}{0.894194in}}{\pgfqpoint{6.855775in}{4.119217in}}%
\pgfusepath{clip}%
\pgfsetroundcap%
\pgfsetroundjoin%
\pgfsetlinewidth{1.003750pt}%
\definecolor{currentstroke}{rgb}{1.000000,1.000000,1.000000}%
\pgfsetstrokecolor{currentstroke}%
\pgfsetdash{}{0pt}%
\pgfpathmoveto{\pgfqpoint{2.492072in}{0.894194in}}%
\pgfpathlineto{\pgfqpoint{2.492072in}{5.013411in}}%
\pgfusepath{stroke}%
\end{pgfscope}%
\begin{pgfscope}%
\definecolor{textcolor}{rgb}{0.150000,0.150000,0.150000}%
\pgfsetstrokecolor{textcolor}%
\pgfsetfillcolor{textcolor}%
\pgftext[x=2.492072in,y=0.796972in,,top]{\color{textcolor}\rmfamily\fontsize{20.000000}{24.000000}\selectfont 50}%
\end{pgfscope}%
\begin{pgfscope}%
\pgfpathrectangle{\pgfqpoint{0.802099in}{0.894194in}}{\pgfqpoint{6.855775in}{4.119217in}}%
\pgfusepath{clip}%
\pgfsetroundcap%
\pgfsetroundjoin%
\pgfsetlinewidth{1.003750pt}%
\definecolor{currentstroke}{rgb}{1.000000,1.000000,1.000000}%
\pgfsetstrokecolor{currentstroke}%
\pgfsetdash{}{0pt}%
\pgfpathmoveto{\pgfqpoint{3.990274in}{0.894194in}}%
\pgfpathlineto{\pgfqpoint{3.990274in}{5.013411in}}%
\pgfusepath{stroke}%
\end{pgfscope}%
\begin{pgfscope}%
\definecolor{textcolor}{rgb}{0.150000,0.150000,0.150000}%
\pgfsetstrokecolor{textcolor}%
\pgfsetfillcolor{textcolor}%
\pgftext[x=3.990274in,y=0.796972in,,top]{\color{textcolor}\rmfamily\fontsize{20.000000}{24.000000}\selectfont 100}%
\end{pgfscope}%
\begin{pgfscope}%
\pgfpathrectangle{\pgfqpoint{0.802099in}{0.894194in}}{\pgfqpoint{6.855775in}{4.119217in}}%
\pgfusepath{clip}%
\pgfsetroundcap%
\pgfsetroundjoin%
\pgfsetlinewidth{1.003750pt}%
\definecolor{currentstroke}{rgb}{1.000000,1.000000,1.000000}%
\pgfsetstrokecolor{currentstroke}%
\pgfsetdash{}{0pt}%
\pgfpathmoveto{\pgfqpoint{5.488477in}{0.894194in}}%
\pgfpathlineto{\pgfqpoint{5.488477in}{5.013411in}}%
\pgfusepath{stroke}%
\end{pgfscope}%
\begin{pgfscope}%
\definecolor{textcolor}{rgb}{0.150000,0.150000,0.150000}%
\pgfsetstrokecolor{textcolor}%
\pgfsetfillcolor{textcolor}%
\pgftext[x=5.488477in,y=0.796972in,,top]{\color{textcolor}\rmfamily\fontsize{20.000000}{24.000000}\selectfont 150}%
\end{pgfscope}%
\begin{pgfscope}%
\pgfpathrectangle{\pgfqpoint{0.802099in}{0.894194in}}{\pgfqpoint{6.855775in}{4.119217in}}%
\pgfusepath{clip}%
\pgfsetroundcap%
\pgfsetroundjoin%
\pgfsetlinewidth{1.003750pt}%
\definecolor{currentstroke}{rgb}{1.000000,1.000000,1.000000}%
\pgfsetstrokecolor{currentstroke}%
\pgfsetdash{}{0pt}%
\pgfpathmoveto{\pgfqpoint{6.986679in}{0.894194in}}%
\pgfpathlineto{\pgfqpoint{6.986679in}{5.013411in}}%
\pgfusepath{stroke}%
\end{pgfscope}%
\begin{pgfscope}%
\definecolor{textcolor}{rgb}{0.150000,0.150000,0.150000}%
\pgfsetstrokecolor{textcolor}%
\pgfsetfillcolor{textcolor}%
\pgftext[x=6.986679in,y=0.796972in,,top]{\color{textcolor}\rmfamily\fontsize{20.000000}{24.000000}\selectfont 200}%
\end{pgfscope}%
\begin{pgfscope}%
\definecolor{textcolor}{rgb}{0.150000,0.150000,0.150000}%
\pgfsetstrokecolor{textcolor}%
\pgfsetfillcolor{textcolor}%
\pgftext[x=4.229987in,y=0.472591in,,top]{\color{textcolor}\rmfamily\fontsize{24.000000}{28.800000}\selectfont Time (ms)}%
\end{pgfscope}%
\begin{pgfscope}%
\pgfpathrectangle{\pgfqpoint{0.802099in}{0.894194in}}{\pgfqpoint{6.855775in}{4.119217in}}%
\pgfusepath{clip}%
\pgfsetroundcap%
\pgfsetroundjoin%
\pgfsetlinewidth{1.003750pt}%
\definecolor{currentstroke}{rgb}{1.000000,1.000000,1.000000}%
\pgfsetstrokecolor{currentstroke}%
\pgfsetdash{}{0pt}%
\pgfpathmoveto{\pgfqpoint{0.802099in}{1.081432in}}%
\pgfpathlineto{\pgfqpoint{7.657874in}{1.081432in}}%
\pgfusepath{stroke}%
\end{pgfscope}%
\begin{pgfscope}%
\definecolor{textcolor}{rgb}{0.150000,0.150000,0.150000}%
\pgfsetstrokecolor{textcolor}%
\pgfsetfillcolor{textcolor}%
\pgftext[x=0.528147in, y=0.975909in, left, base]{\color{textcolor}\rmfamily\fontsize{20.000000}{24.000000}\selectfont 2}%
\end{pgfscope}%
\begin{pgfscope}%
\pgfpathrectangle{\pgfqpoint{0.802099in}{0.894194in}}{\pgfqpoint{6.855775in}{4.119217in}}%
\pgfusepath{clip}%
\pgfsetroundcap%
\pgfsetroundjoin%
\pgfsetlinewidth{1.003750pt}%
\definecolor{currentstroke}{rgb}{1.000000,1.000000,1.000000}%
\pgfsetstrokecolor{currentstroke}%
\pgfsetdash{}{0pt}%
\pgfpathmoveto{\pgfqpoint{0.802099in}{1.700394in}}%
\pgfpathlineto{\pgfqpoint{7.657874in}{1.700394in}}%
\pgfusepath{stroke}%
\end{pgfscope}%
\begin{pgfscope}%
\definecolor{textcolor}{rgb}{0.150000,0.150000,0.150000}%
\pgfsetstrokecolor{textcolor}%
\pgfsetfillcolor{textcolor}%
\pgftext[x=0.528147in, y=1.594871in, left, base]{\color{textcolor}\rmfamily\fontsize{20.000000}{24.000000}\selectfont 3}%
\end{pgfscope}%
\begin{pgfscope}%
\pgfpathrectangle{\pgfqpoint{0.802099in}{0.894194in}}{\pgfqpoint{6.855775in}{4.119217in}}%
\pgfusepath{clip}%
\pgfsetroundcap%
\pgfsetroundjoin%
\pgfsetlinewidth{1.003750pt}%
\definecolor{currentstroke}{rgb}{1.000000,1.000000,1.000000}%
\pgfsetstrokecolor{currentstroke}%
\pgfsetdash{}{0pt}%
\pgfpathmoveto{\pgfqpoint{0.802099in}{2.319356in}}%
\pgfpathlineto{\pgfqpoint{7.657874in}{2.319356in}}%
\pgfusepath{stroke}%
\end{pgfscope}%
\begin{pgfscope}%
\definecolor{textcolor}{rgb}{0.150000,0.150000,0.150000}%
\pgfsetstrokecolor{textcolor}%
\pgfsetfillcolor{textcolor}%
\pgftext[x=0.528147in, y=2.213833in, left, base]{\color{textcolor}\rmfamily\fontsize{20.000000}{24.000000}\selectfont 4}%
\end{pgfscope}%
\begin{pgfscope}%
\pgfpathrectangle{\pgfqpoint{0.802099in}{0.894194in}}{\pgfqpoint{6.855775in}{4.119217in}}%
\pgfusepath{clip}%
\pgfsetroundcap%
\pgfsetroundjoin%
\pgfsetlinewidth{1.003750pt}%
\definecolor{currentstroke}{rgb}{1.000000,1.000000,1.000000}%
\pgfsetstrokecolor{currentstroke}%
\pgfsetdash{}{0pt}%
\pgfpathmoveto{\pgfqpoint{0.802099in}{2.938319in}}%
\pgfpathlineto{\pgfqpoint{7.657874in}{2.938319in}}%
\pgfusepath{stroke}%
\end{pgfscope}%
\begin{pgfscope}%
\definecolor{textcolor}{rgb}{0.150000,0.150000,0.150000}%
\pgfsetstrokecolor{textcolor}%
\pgfsetfillcolor{textcolor}%
\pgftext[x=0.528147in, y=2.832796in, left, base]{\color{textcolor}\rmfamily\fontsize{20.000000}{24.000000}\selectfont 5}%
\end{pgfscope}%
\begin{pgfscope}%
\pgfpathrectangle{\pgfqpoint{0.802099in}{0.894194in}}{\pgfqpoint{6.855775in}{4.119217in}}%
\pgfusepath{clip}%
\pgfsetroundcap%
\pgfsetroundjoin%
\pgfsetlinewidth{1.003750pt}%
\definecolor{currentstroke}{rgb}{1.000000,1.000000,1.000000}%
\pgfsetstrokecolor{currentstroke}%
\pgfsetdash{}{0pt}%
\pgfpathmoveto{\pgfqpoint{0.802099in}{3.557281in}}%
\pgfpathlineto{\pgfqpoint{7.657874in}{3.557281in}}%
\pgfusepath{stroke}%
\end{pgfscope}%
\begin{pgfscope}%
\definecolor{textcolor}{rgb}{0.150000,0.150000,0.150000}%
\pgfsetstrokecolor{textcolor}%
\pgfsetfillcolor{textcolor}%
\pgftext[x=0.528147in, y=3.451758in, left, base]{\color{textcolor}\rmfamily\fontsize{20.000000}{24.000000}\selectfont 6}%
\end{pgfscope}%
\begin{pgfscope}%
\pgfpathrectangle{\pgfqpoint{0.802099in}{0.894194in}}{\pgfqpoint{6.855775in}{4.119217in}}%
\pgfusepath{clip}%
\pgfsetroundcap%
\pgfsetroundjoin%
\pgfsetlinewidth{1.003750pt}%
\definecolor{currentstroke}{rgb}{1.000000,1.000000,1.000000}%
\pgfsetstrokecolor{currentstroke}%
\pgfsetdash{}{0pt}%
\pgfpathmoveto{\pgfqpoint{0.802099in}{4.176243in}}%
\pgfpathlineto{\pgfqpoint{7.657874in}{4.176243in}}%
\pgfusepath{stroke}%
\end{pgfscope}%
\begin{pgfscope}%
\definecolor{textcolor}{rgb}{0.150000,0.150000,0.150000}%
\pgfsetstrokecolor{textcolor}%
\pgfsetfillcolor{textcolor}%
\pgftext[x=0.528147in, y=4.070720in, left, base]{\color{textcolor}\rmfamily\fontsize{20.000000}{24.000000}\selectfont 7}%
\end{pgfscope}%
\begin{pgfscope}%
\pgfpathrectangle{\pgfqpoint{0.802099in}{0.894194in}}{\pgfqpoint{6.855775in}{4.119217in}}%
\pgfusepath{clip}%
\pgfsetroundcap%
\pgfsetroundjoin%
\pgfsetlinewidth{1.003750pt}%
\definecolor{currentstroke}{rgb}{1.000000,1.000000,1.000000}%
\pgfsetstrokecolor{currentstroke}%
\pgfsetdash{}{0pt}%
\pgfpathmoveto{\pgfqpoint{0.802099in}{4.795206in}}%
\pgfpathlineto{\pgfqpoint{7.657874in}{4.795206in}}%
\pgfusepath{stroke}%
\end{pgfscope}%
\begin{pgfscope}%
\definecolor{textcolor}{rgb}{0.150000,0.150000,0.150000}%
\pgfsetstrokecolor{textcolor}%
\pgfsetfillcolor{textcolor}%
\pgftext[x=0.528147in, y=4.689683in, left, base]{\color{textcolor}\rmfamily\fontsize{20.000000}{24.000000}\selectfont 8}%
\end{pgfscope}%
\begin{pgfscope}%
\definecolor{textcolor}{rgb}{0.150000,0.150000,0.150000}%
\pgfsetstrokecolor{textcolor}%
\pgfsetfillcolor{textcolor}%
\pgftext[x=0.472591in,y=2.953803in,,bottom,rotate=90.000000]{\color{textcolor}\rmfamily\fontsize{24.000000}{28.800000}\selectfont Pulse Amplitude (V)}%
\end{pgfscope}%
\begin{pgfscope}%
\pgfpathrectangle{\pgfqpoint{0.802099in}{0.894194in}}{\pgfqpoint{6.855775in}{4.119217in}}%
\pgfusepath{clip}%
\pgfsetbuttcap%
\pgfsetroundjoin%
\definecolor{currentfill}{rgb}{0.698039,0.133333,0.133333}%
\pgfsetfillcolor{currentfill}%
\pgfsetlinewidth{0.301125pt}%
\definecolor{currentstroke}{rgb}{0.698039,0.133333,0.133333}%
\pgfsetstrokecolor{currentstroke}%
\pgfsetdash{}{0pt}%
\pgfsys@defobject{currentmarker}{\pgfqpoint{-0.048611in}{-0.048611in}}{\pgfqpoint{0.048611in}{0.048611in}}{%
\pgfpathmoveto{\pgfqpoint{0.000000in}{-0.048611in}}%
\pgfpathcurveto{\pgfqpoint{0.012892in}{-0.048611in}}{\pgfqpoint{0.025257in}{-0.043489in}}{\pgfqpoint{0.034373in}{-0.034373in}}%
\pgfpathcurveto{\pgfqpoint{0.043489in}{-0.025257in}}{\pgfqpoint{0.048611in}{-0.012892in}}{\pgfqpoint{0.048611in}{0.000000in}}%
\pgfpathcurveto{\pgfqpoint{0.048611in}{0.012892in}}{\pgfqpoint{0.043489in}{0.025257in}}{\pgfqpoint{0.034373in}{0.034373in}}%
\pgfpathcurveto{\pgfqpoint{0.025257in}{0.043489in}}{\pgfqpoint{0.012892in}{0.048611in}}{\pgfqpoint{0.000000in}{0.048611in}}%
\pgfpathcurveto{\pgfqpoint{-0.012892in}{0.048611in}}{\pgfqpoint{-0.025257in}{0.043489in}}{\pgfqpoint{-0.034373in}{0.034373in}}%
\pgfpathcurveto{\pgfqpoint{-0.043489in}{0.025257in}}{\pgfqpoint{-0.048611in}{0.012892in}}{\pgfqpoint{-0.048611in}{0.000000in}}%
\pgfpathcurveto{\pgfqpoint{-0.048611in}{-0.012892in}}{\pgfqpoint{-0.043489in}{-0.025257in}}{\pgfqpoint{-0.034373in}{-0.034373in}}%
\pgfpathcurveto{\pgfqpoint{-0.025257in}{-0.043489in}}{\pgfqpoint{-0.012892in}{-0.048611in}}{\pgfqpoint{0.000000in}{-0.048611in}}%
\pgfpathlineto{\pgfqpoint{0.000000in}{-0.048611in}}%
\pgfpathclose%
\pgfusepath{stroke,fill}%
}%
\begin{pgfscope}%
\pgfsys@transformshift{1.113726in}{4.795206in}%
\pgfsys@useobject{currentmarker}{}%
\end{pgfscope}%
\begin{pgfscope}%
\pgfsys@transformshift{1.593150in}{4.300036in}%
\pgfsys@useobject{currentmarker}{}%
\end{pgfscope}%
\begin{pgfscope}%
\pgfsys@transformshift{2.072575in}{3.903900in}%
\pgfsys@useobject{currentmarker}{}%
\end{pgfscope}%
\begin{pgfscope}%
\pgfsys@transformshift{2.552000in}{3.309696in}%
\pgfsys@useobject{currentmarker}{}%
\end{pgfscope}%
\begin{pgfscope}%
\pgfsys@transformshift{3.031425in}{2.864043in}%
\pgfsys@useobject{currentmarker}{}%
\end{pgfscope}%
\begin{pgfscope}%
\pgfsys@transformshift{3.510850in}{2.765009in}%
\pgfsys@useobject{currentmarker}{}%
\end{pgfscope}%
\begin{pgfscope}%
\pgfsys@transformshift{3.990274in}{2.418390in}%
\pgfsys@useobject{currentmarker}{}%
\end{pgfscope}%
\begin{pgfscope}%
\pgfsys@transformshift{4.469699in}{2.121288in}%
\pgfsys@useobject{currentmarker}{}%
\end{pgfscope}%
\begin{pgfscope}%
\pgfsys@transformshift{4.949124in}{1.824186in}%
\pgfsys@useobject{currentmarker}{}%
\end{pgfscope}%
\begin{pgfscope}%
\pgfsys@transformshift{5.428549in}{1.725152in}%
\pgfsys@useobject{currentmarker}{}%
\end{pgfscope}%
\begin{pgfscope}%
\pgfsys@transformshift{5.907974in}{1.576601in}%
\pgfsys@useobject{currentmarker}{}%
\end{pgfscope}%
\begin{pgfscope}%
\pgfsys@transformshift{6.387398in}{1.378534in}%
\pgfsys@useobject{currentmarker}{}%
\end{pgfscope}%
\begin{pgfscope}%
\pgfsys@transformshift{6.866823in}{1.279500in}%
\pgfsys@useobject{currentmarker}{}%
\end{pgfscope}%
\begin{pgfscope}%
\pgfsys@transformshift{7.346248in}{1.081432in}%
\pgfsys@useobject{currentmarker}{}%
\end{pgfscope}%
\end{pgfscope}%
\begin{pgfscope}%
\pgfpathrectangle{\pgfqpoint{0.802099in}{0.894194in}}{\pgfqpoint{6.855775in}{4.119217in}}%
\pgfusepath{clip}%
\pgfsetroundcap%
\pgfsetroundjoin%
\pgfsetlinewidth{1.756562pt}%
\definecolor{currentstroke}{rgb}{0.298039,0.447059,0.690196}%
\pgfsetstrokecolor{currentstroke}%
\pgfsetdash{}{0pt}%
\pgfpathmoveto{\pgfqpoint{1.113726in}{4.826174in}}%
\pgfpathlineto{\pgfqpoint{1.176680in}{4.750261in}}%
\pgfpathlineto{\pgfqpoint{1.239635in}{4.675608in}}%
\pgfpathlineto{\pgfqpoint{1.302590in}{4.602194in}}%
\pgfpathlineto{\pgfqpoint{1.365545in}{4.530000in}}%
\pgfpathlineto{\pgfqpoint{1.428499in}{4.459005in}}%
\pgfpathlineto{\pgfqpoint{1.491454in}{4.389188in}}%
\pgfpathlineto{\pgfqpoint{1.554409in}{4.320531in}}%
\pgfpathlineto{\pgfqpoint{1.617364in}{4.253014in}}%
\pgfpathlineto{\pgfqpoint{1.680319in}{4.186618in}}%
\pgfpathlineto{\pgfqpoint{1.743273in}{4.121324in}}%
\pgfpathlineto{\pgfqpoint{1.806228in}{4.057115in}}%
\pgfpathlineto{\pgfqpoint{1.869183in}{3.993971in}}%
\pgfpathlineto{\pgfqpoint{1.932138in}{3.931877in}}%
\pgfpathlineto{\pgfqpoint{1.995092in}{3.870813in}}%
\pgfpathlineto{\pgfqpoint{2.058047in}{3.810764in}}%
\pgfpathlineto{\pgfqpoint{2.121002in}{3.751711in}}%
\pgfpathlineto{\pgfqpoint{2.183957in}{3.693639in}}%
\pgfpathlineto{\pgfqpoint{2.246911in}{3.636532in}}%
\pgfpathlineto{\pgfqpoint{2.309866in}{3.580372in}}%
\pgfpathlineto{\pgfqpoint{2.372821in}{3.525146in}}%
\pgfpathlineto{\pgfqpoint{2.435776in}{3.470836in}}%
\pgfpathlineto{\pgfqpoint{2.498731in}{3.417428in}}%
\pgfpathlineto{\pgfqpoint{2.561685in}{3.364907in}}%
\pgfpathlineto{\pgfqpoint{2.624640in}{3.313258in}}%
\pgfpathlineto{\pgfqpoint{2.687595in}{3.262467in}}%
\pgfpathlineto{\pgfqpoint{2.750550in}{3.212519in}}%
\pgfpathlineto{\pgfqpoint{2.813504in}{3.163400in}}%
\pgfpathlineto{\pgfqpoint{2.876459in}{3.115097in}}%
\pgfpathlineto{\pgfqpoint{2.939414in}{3.067597in}}%
\pgfpathlineto{\pgfqpoint{3.002369in}{3.020885in}}%
\pgfpathlineto{\pgfqpoint{3.065323in}{2.974948in}}%
\pgfpathlineto{\pgfqpoint{3.128278in}{2.929775in}}%
\pgfpathlineto{\pgfqpoint{3.191233in}{2.885351in}}%
\pgfpathlineto{\pgfqpoint{3.254188in}{2.841665in}}%
\pgfpathlineto{\pgfqpoint{3.317143in}{2.798705in}}%
\pgfpathlineto{\pgfqpoint{3.380097in}{2.756458in}}%
\pgfpathlineto{\pgfqpoint{3.443052in}{2.714912in}}%
\pgfpathlineto{\pgfqpoint{3.506007in}{2.674057in}}%
\pgfpathlineto{\pgfqpoint{3.568962in}{2.633879in}}%
\pgfpathlineto{\pgfqpoint{3.631916in}{2.594369in}}%
\pgfpathlineto{\pgfqpoint{3.694871in}{2.555515in}}%
\pgfpathlineto{\pgfqpoint{3.757826in}{2.517306in}}%
\pgfpathlineto{\pgfqpoint{3.820781in}{2.479732in}}%
\pgfpathlineto{\pgfqpoint{3.883736in}{2.442781in}}%
\pgfpathlineto{\pgfqpoint{3.946690in}{2.406444in}}%
\pgfpathlineto{\pgfqpoint{4.009645in}{2.370711in}}%
\pgfpathlineto{\pgfqpoint{4.072600in}{2.335571in}}%
\pgfpathlineto{\pgfqpoint{4.135555in}{2.301014in}}%
\pgfpathlineto{\pgfqpoint{4.198509in}{2.267031in}}%
\pgfpathlineto{\pgfqpoint{4.261464in}{2.233613in}}%
\pgfpathlineto{\pgfqpoint{4.324419in}{2.200749in}}%
\pgfpathlineto{\pgfqpoint{4.387374in}{2.168431in}}%
\pgfpathlineto{\pgfqpoint{4.450328in}{2.136650in}}%
\pgfpathlineto{\pgfqpoint{4.513283in}{2.105396in}}%
\pgfpathlineto{\pgfqpoint{4.576238in}{2.074662in}}%
\pgfpathlineto{\pgfqpoint{4.639193in}{2.044438in}}%
\pgfpathlineto{\pgfqpoint{4.702148in}{2.014715in}}%
\pgfpathlineto{\pgfqpoint{4.765102in}{1.985486in}}%
\pgfpathlineto{\pgfqpoint{4.828057in}{1.956743in}}%
\pgfpathlineto{\pgfqpoint{4.891012in}{1.928477in}}%
\pgfpathlineto{\pgfqpoint{4.953967in}{1.900680in}}%
\pgfpathlineto{\pgfqpoint{5.016921in}{1.873345in}}%
\pgfpathlineto{\pgfqpoint{5.079876in}{1.846464in}}%
\pgfpathlineto{\pgfqpoint{5.142831in}{1.820029in}}%
\pgfpathlineto{\pgfqpoint{5.205786in}{1.794033in}}%
\pgfpathlineto{\pgfqpoint{5.268740in}{1.768468in}}%
\pgfpathlineto{\pgfqpoint{5.331695in}{1.743328in}}%
\pgfpathlineto{\pgfqpoint{5.394650in}{1.718606in}}%
\pgfpathlineto{\pgfqpoint{5.457605in}{1.694294in}}%
\pgfpathlineto{\pgfqpoint{5.520560in}{1.670386in}}%
\pgfpathlineto{\pgfqpoint{5.583514in}{1.646875in}}%
\pgfpathlineto{\pgfqpoint{5.646469in}{1.623754in}}%
\pgfpathlineto{\pgfqpoint{5.709424in}{1.601017in}}%
\pgfpathlineto{\pgfqpoint{5.772379in}{1.578658in}}%
\pgfpathlineto{\pgfqpoint{5.835333in}{1.556670in}}%
\pgfpathlineto{\pgfqpoint{5.898288in}{1.535047in}}%
\pgfpathlineto{\pgfqpoint{5.961243in}{1.513783in}}%
\pgfpathlineto{\pgfqpoint{6.024198in}{1.492873in}}%
\pgfpathlineto{\pgfqpoint{6.087153in}{1.472309in}}%
\pgfpathlineto{\pgfqpoint{6.150107in}{1.452087in}}%
\pgfpathlineto{\pgfqpoint{6.213062in}{1.432201in}}%
\pgfpathlineto{\pgfqpoint{6.276017in}{1.412645in}}%
\pgfpathlineto{\pgfqpoint{6.338972in}{1.393413in}}%
\pgfpathlineto{\pgfqpoint{6.401926in}{1.374501in}}%
\pgfpathlineto{\pgfqpoint{6.464881in}{1.355903in}}%
\pgfpathlineto{\pgfqpoint{6.527836in}{1.337614in}}%
\pgfpathlineto{\pgfqpoint{6.590791in}{1.319629in}}%
\pgfpathlineto{\pgfqpoint{6.653745in}{1.301942in}}%
\pgfpathlineto{\pgfqpoint{6.716700in}{1.284549in}}%
\pgfpathlineto{\pgfqpoint{6.779655in}{1.267445in}}%
\pgfpathlineto{\pgfqpoint{6.842610in}{1.250624in}}%
\pgfpathlineto{\pgfqpoint{6.905565in}{1.234083in}}%
\pgfpathlineto{\pgfqpoint{6.968519in}{1.217817in}}%
\pgfpathlineto{\pgfqpoint{7.031474in}{1.201821in}}%
\pgfpathlineto{\pgfqpoint{7.094429in}{1.186090in}}%
\pgfpathlineto{\pgfqpoint{7.157384in}{1.170621in}}%
\pgfpathlineto{\pgfqpoint{7.220338in}{1.155409in}}%
\pgfpathlineto{\pgfqpoint{7.283293in}{1.140449in}}%
\pgfpathlineto{\pgfqpoint{7.346248in}{1.125737in}}%
\pgfusepath{stroke}%
\end{pgfscope}%
\begin{pgfscope}%
\pgfsetrectcap%
\pgfsetmiterjoin%
\pgfsetlinewidth{0.000000pt}%
\definecolor{currentstroke}{rgb}{1.000000,1.000000,1.000000}%
\pgfsetstrokecolor{currentstroke}%
\pgfsetdash{}{0pt}%
\pgfpathmoveto{\pgfqpoint{0.802099in}{0.894194in}}%
\pgfpathlineto{\pgfqpoint{0.802099in}{5.013411in}}%
\pgfusepath{}%
\end{pgfscope}%
\begin{pgfscope}%
\pgfsetrectcap%
\pgfsetmiterjoin%
\pgfsetlinewidth{0.000000pt}%
\definecolor{currentstroke}{rgb}{1.000000,1.000000,1.000000}%
\pgfsetstrokecolor{currentstroke}%
\pgfsetdash{}{0pt}%
\pgfpathmoveto{\pgfqpoint{7.657874in}{0.894194in}}%
\pgfpathlineto{\pgfqpoint{7.657874in}{5.013411in}}%
\pgfusepath{}%
\end{pgfscope}%
\begin{pgfscope}%
\pgfsetrectcap%
\pgfsetmiterjoin%
\pgfsetlinewidth{0.000000pt}%
\definecolor{currentstroke}{rgb}{1.000000,1.000000,1.000000}%
\pgfsetstrokecolor{currentstroke}%
\pgfsetdash{}{0pt}%
\pgfpathmoveto{\pgfqpoint{0.802099in}{0.894194in}}%
\pgfpathlineto{\pgfqpoint{7.657874in}{0.894194in}}%
\pgfusepath{}%
\end{pgfscope}%
\begin{pgfscope}%
\pgfsetrectcap%
\pgfsetmiterjoin%
\pgfsetlinewidth{0.000000pt}%
\definecolor{currentstroke}{rgb}{1.000000,1.000000,1.000000}%
\pgfsetstrokecolor{currentstroke}%
\pgfsetdash{}{0pt}%
\pgfpathmoveto{\pgfqpoint{0.802099in}{5.013411in}}%
\pgfpathlineto{\pgfqpoint{7.657874in}{5.013411in}}%
\pgfusepath{}%
\end{pgfscope}%
\begin{pgfscope}%
\definecolor{textcolor}{rgb}{0.150000,0.150000,0.150000}%
\pgfsetstrokecolor{textcolor}%
\pgfsetfillcolor{textcolor}%
\pgftext[x=3.990274in, y=3.741855in, left, base]{\color{textcolor}\rmfamily\fontsize{20.000000}{24.000000}\selectfont M\(\displaystyle _z\)(t) = 7.63\(\displaystyle exp[\frac{-t}{T_2}]+\)0.66}%
\end{pgfscope}%
\begin{pgfscope}%
\definecolor{textcolor}{rgb}{0.150000,0.150000,0.150000}%
\pgfsetstrokecolor{textcolor}%
\pgfsetfillcolor{textcolor}%
\pgftext[x=3.990274in, y=3.247800in, left, base]{\color{textcolor}\rmfamily\fontsize{20.000000}{24.000000}\selectfont T\(\displaystyle _2\) = 125.4\(\displaystyle \pm\)10.0 ms}%
\end{pgfscope}%
\begin{pgfscope}%
\definecolor{textcolor}{rgb}{0.150000,0.150000,0.150000}%
\pgfsetstrokecolor{textcolor}%
\pgfsetfillcolor{textcolor}%
\pgftext[x=4.229987in,y=5.096745in,,base]{\color{textcolor}\rmfamily\fontsize{24.000000}{28.800000}\selectfont  Meiboom-Gill T\(\displaystyle _2\) measurement of 0.005 M CuSO\(\displaystyle _4\)}%
\end{pgfscope}%
\begin{pgfscope}%
\pgfsetbuttcap%
\pgfsetroundjoin%
\definecolor{currentfill}{rgb}{0.698039,0.133333,0.133333}%
\pgfsetfillcolor{currentfill}%
\pgfsetlinewidth{0.301125pt}%
\definecolor{currentstroke}{rgb}{0.698039,0.133333,0.133333}%
\pgfsetstrokecolor{currentstroke}%
\pgfsetdash{}{0pt}%
\pgfsys@defobject{currentmarker}{\pgfqpoint{-0.048611in}{-0.048611in}}{\pgfqpoint{0.048611in}{0.048611in}}{%
\pgfpathmoveto{\pgfqpoint{0.000000in}{-0.048611in}}%
\pgfpathcurveto{\pgfqpoint{0.012892in}{-0.048611in}}{\pgfqpoint{0.025257in}{-0.043489in}}{\pgfqpoint{0.034373in}{-0.034373in}}%
\pgfpathcurveto{\pgfqpoint{0.043489in}{-0.025257in}}{\pgfqpoint{0.048611in}{-0.012892in}}{\pgfqpoint{0.048611in}{0.000000in}}%
\pgfpathcurveto{\pgfqpoint{0.048611in}{0.012892in}}{\pgfqpoint{0.043489in}{0.025257in}}{\pgfqpoint{0.034373in}{0.034373in}}%
\pgfpathcurveto{\pgfqpoint{0.025257in}{0.043489in}}{\pgfqpoint{0.012892in}{0.048611in}}{\pgfqpoint{0.000000in}{0.048611in}}%
\pgfpathcurveto{\pgfqpoint{-0.012892in}{0.048611in}}{\pgfqpoint{-0.025257in}{0.043489in}}{\pgfqpoint{-0.034373in}{0.034373in}}%
\pgfpathcurveto{\pgfqpoint{-0.043489in}{0.025257in}}{\pgfqpoint{-0.048611in}{0.012892in}}{\pgfqpoint{-0.048611in}{0.000000in}}%
\pgfpathcurveto{\pgfqpoint{-0.048611in}{-0.012892in}}{\pgfqpoint{-0.043489in}{-0.025257in}}{\pgfqpoint{-0.034373in}{-0.034373in}}%
\pgfpathcurveto{\pgfqpoint{-0.025257in}{-0.043489in}}{\pgfqpoint{-0.012892in}{-0.048611in}}{\pgfqpoint{0.000000in}{-0.048611in}}%
\pgfpathlineto{\pgfqpoint{0.000000in}{-0.048611in}}%
\pgfpathclose%
\pgfusepath{stroke,fill}%
}%
\begin{pgfscope}%
\pgfsys@transformshift{5.127004in}{4.625282in}%
\pgfsys@useobject{currentmarker}{}%
\end{pgfscope}%
\end{pgfscope}%
\begin{pgfscope}%
\definecolor{textcolor}{rgb}{0.150000,0.150000,0.150000}%
\pgfsetstrokecolor{textcolor}%
\pgfsetfillcolor{textcolor}%
\pgftext[x=5.627004in,y=4.552365in,left,base]{\color{textcolor}\rmfamily\fontsize{20.000000}{24.000000}\selectfont Data}%
\end{pgfscope}%
\begin{pgfscope}%
\pgfsetroundcap%
\pgfsetroundjoin%
\pgfsetlinewidth{1.756562pt}%
\definecolor{currentstroke}{rgb}{0.298039,0.447059,0.690196}%
\pgfsetstrokecolor{currentstroke}%
\pgfsetdash{}{0pt}%
\pgfpathmoveto{\pgfqpoint{4.849226in}{4.241873in}}%
\pgfpathlineto{\pgfqpoint{5.127004in}{4.241873in}}%
\pgfpathlineto{\pgfqpoint{5.404782in}{4.241873in}}%
\pgfusepath{stroke}%
\end{pgfscope}%
\begin{pgfscope}%
\definecolor{textcolor}{rgb}{0.150000,0.150000,0.150000}%
\pgfsetstrokecolor{textcolor}%
\pgfsetfillcolor{textcolor}%
\pgftext[x=5.627004in,y=4.144651in,left,base]{\color{textcolor}\rmfamily\fontsize{20.000000}{24.000000}\selectfont Fitted Curve}%
\end{pgfscope}%
\end{pgfpicture}%
\makeatother%
\endgroup%
}
    \caption{T2 measurement of 0.005 M CuSO$_{4} \boldsymbol{\cdot} $H$_2$O using the Meiboom-Gill pulse sequence }
    \label{fig:0d005_t2}
  \end{figure}

  \begin{figure}[h]
    \resizebox{0.45\textwidth}{!}{%% Creator: Matplotlib, PGF backend
%%
%% To include the figure in your LaTeX document, write
%%   \input{<filename>.pgf}
%%
%% Make sure the required packages are loaded in your preamble
%%   \usepackage{pgf}
%%
%% Also ensure that all the required font packages are loaded; for instance,
%% the lmodern package is sometimes necessary when using math font.
%%   \usepackage{lmodern}
%%
%% Figures using additional raster images can only be included by \input if
%% they are in the same directory as the main LaTeX file. For loading figures
%% from other directories you can use the `import` package
%%   \usepackage{import}
%%
%% and then include the figures with
%%   \import{<path to file>}{<filename>.pgf}
%%
%% Matplotlib used the following preamble
%%   \usepackage{fontspec}
%%   \setmainfont{DejaVuSerif.ttf}[Path=\detokenize{/usr/share/matplotlib/mpl-data/fonts/ttf/}]
%%   \setsansfont{DejaVuSans.ttf}[Path=\detokenize{/usr/share/matplotlib/mpl-data/fonts/ttf/}]
%%   \setmonofont{DejaVuSansMono.ttf}[Path=\detokenize{/usr/share/matplotlib/mpl-data/fonts/ttf/}]
%%
\begingroup%
\makeatletter%
\begin{pgfpicture}%
\pgfpathrectangle{\pgfpointorigin}{\pgfqpoint{8.000000in}{5.500000in}}%
\pgfusepath{use as bounding box, clip}%
\begin{pgfscope}%
\pgfsetbuttcap%
\pgfsetmiterjoin%
\definecolor{currentfill}{rgb}{1.000000,1.000000,1.000000}%
\pgfsetfillcolor{currentfill}%
\pgfsetlinewidth{0.000000pt}%
\definecolor{currentstroke}{rgb}{1.000000,1.000000,1.000000}%
\pgfsetstrokecolor{currentstroke}%
\pgfsetdash{}{0pt}%
\pgfpathmoveto{\pgfqpoint{0.000000in}{0.000000in}}%
\pgfpathlineto{\pgfqpoint{8.000000in}{0.000000in}}%
\pgfpathlineto{\pgfqpoint{8.000000in}{5.500000in}}%
\pgfpathlineto{\pgfqpoint{0.000000in}{5.500000in}}%
\pgfpathlineto{\pgfqpoint{0.000000in}{0.000000in}}%
\pgfpathclose%
\pgfusepath{fill}%
\end{pgfscope}%
\begin{pgfscope}%
\pgfsetbuttcap%
\pgfsetmiterjoin%
\definecolor{currentfill}{rgb}{0.917647,0.917647,0.949020}%
\pgfsetfillcolor{currentfill}%
\pgfsetlinewidth{0.000000pt}%
\definecolor{currentstroke}{rgb}{0.000000,0.000000,0.000000}%
\pgfsetstrokecolor{currentstroke}%
\pgfsetstrokeopacity{0.000000}%
\pgfsetdash{}{0pt}%
\pgfpathmoveto{\pgfqpoint{0.802099in}{0.894194in}}%
\pgfpathlineto{\pgfqpoint{7.534428in}{0.894194in}}%
\pgfpathlineto{\pgfqpoint{7.534428in}{5.013411in}}%
\pgfpathlineto{\pgfqpoint{0.802099in}{5.013411in}}%
\pgfpathlineto{\pgfqpoint{0.802099in}{0.894194in}}%
\pgfpathclose%
\pgfusepath{fill}%
\end{pgfscope}%
\begin{pgfscope}%
\pgfpathrectangle{\pgfqpoint{0.802099in}{0.894194in}}{\pgfqpoint{6.732329in}{4.119217in}}%
\pgfusepath{clip}%
\pgfsetroundcap%
\pgfsetroundjoin%
\pgfsetlinewidth{1.003750pt}%
\definecolor{currentstroke}{rgb}{1.000000,1.000000,1.000000}%
\pgfsetstrokecolor{currentstroke}%
\pgfsetdash{}{0pt}%
\pgfpathmoveto{\pgfqpoint{0.990416in}{0.894194in}}%
\pgfpathlineto{\pgfqpoint{0.990416in}{5.013411in}}%
\pgfusepath{stroke}%
\end{pgfscope}%
\begin{pgfscope}%
\definecolor{textcolor}{rgb}{0.150000,0.150000,0.150000}%
\pgfsetstrokecolor{textcolor}%
\pgfsetfillcolor{textcolor}%
\pgftext[x=0.990416in,y=0.796972in,,top]{\color{textcolor}\rmfamily\fontsize{20.000000}{24.000000}\selectfont 0}%
\end{pgfscope}%
\begin{pgfscope}%
\pgfpathrectangle{\pgfqpoint{0.802099in}{0.894194in}}{\pgfqpoint{6.732329in}{4.119217in}}%
\pgfusepath{clip}%
\pgfsetroundcap%
\pgfsetroundjoin%
\pgfsetlinewidth{1.003750pt}%
\definecolor{currentstroke}{rgb}{1.000000,1.000000,1.000000}%
\pgfsetstrokecolor{currentstroke}%
\pgfsetdash{}{0pt}%
\pgfpathmoveto{\pgfqpoint{2.461642in}{0.894194in}}%
\pgfpathlineto{\pgfqpoint{2.461642in}{5.013411in}}%
\pgfusepath{stroke}%
\end{pgfscope}%
\begin{pgfscope}%
\definecolor{textcolor}{rgb}{0.150000,0.150000,0.150000}%
\pgfsetstrokecolor{textcolor}%
\pgfsetfillcolor{textcolor}%
\pgftext[x=2.461642in,y=0.796972in,,top]{\color{textcolor}\rmfamily\fontsize{20.000000}{24.000000}\selectfont 50}%
\end{pgfscope}%
\begin{pgfscope}%
\pgfpathrectangle{\pgfqpoint{0.802099in}{0.894194in}}{\pgfqpoint{6.732329in}{4.119217in}}%
\pgfusepath{clip}%
\pgfsetroundcap%
\pgfsetroundjoin%
\pgfsetlinewidth{1.003750pt}%
\definecolor{currentstroke}{rgb}{1.000000,1.000000,1.000000}%
\pgfsetstrokecolor{currentstroke}%
\pgfsetdash{}{0pt}%
\pgfpathmoveto{\pgfqpoint{3.932868in}{0.894194in}}%
\pgfpathlineto{\pgfqpoint{3.932868in}{5.013411in}}%
\pgfusepath{stroke}%
\end{pgfscope}%
\begin{pgfscope}%
\definecolor{textcolor}{rgb}{0.150000,0.150000,0.150000}%
\pgfsetstrokecolor{textcolor}%
\pgfsetfillcolor{textcolor}%
\pgftext[x=3.932868in,y=0.796972in,,top]{\color{textcolor}\rmfamily\fontsize{20.000000}{24.000000}\selectfont 100}%
\end{pgfscope}%
\begin{pgfscope}%
\pgfpathrectangle{\pgfqpoint{0.802099in}{0.894194in}}{\pgfqpoint{6.732329in}{4.119217in}}%
\pgfusepath{clip}%
\pgfsetroundcap%
\pgfsetroundjoin%
\pgfsetlinewidth{1.003750pt}%
\definecolor{currentstroke}{rgb}{1.000000,1.000000,1.000000}%
\pgfsetstrokecolor{currentstroke}%
\pgfsetdash{}{0pt}%
\pgfpathmoveto{\pgfqpoint{5.404094in}{0.894194in}}%
\pgfpathlineto{\pgfqpoint{5.404094in}{5.013411in}}%
\pgfusepath{stroke}%
\end{pgfscope}%
\begin{pgfscope}%
\definecolor{textcolor}{rgb}{0.150000,0.150000,0.150000}%
\pgfsetstrokecolor{textcolor}%
\pgfsetfillcolor{textcolor}%
\pgftext[x=5.404094in,y=0.796972in,,top]{\color{textcolor}\rmfamily\fontsize{20.000000}{24.000000}\selectfont 150}%
\end{pgfscope}%
\begin{pgfscope}%
\pgfpathrectangle{\pgfqpoint{0.802099in}{0.894194in}}{\pgfqpoint{6.732329in}{4.119217in}}%
\pgfusepath{clip}%
\pgfsetroundcap%
\pgfsetroundjoin%
\pgfsetlinewidth{1.003750pt}%
\definecolor{currentstroke}{rgb}{1.000000,1.000000,1.000000}%
\pgfsetstrokecolor{currentstroke}%
\pgfsetdash{}{0pt}%
\pgfpathmoveto{\pgfqpoint{6.875319in}{0.894194in}}%
\pgfpathlineto{\pgfqpoint{6.875319in}{5.013411in}}%
\pgfusepath{stroke}%
\end{pgfscope}%
\begin{pgfscope}%
\definecolor{textcolor}{rgb}{0.150000,0.150000,0.150000}%
\pgfsetstrokecolor{textcolor}%
\pgfsetfillcolor{textcolor}%
\pgftext[x=6.875319in,y=0.796972in,,top]{\color{textcolor}\rmfamily\fontsize{20.000000}{24.000000}\selectfont 200}%
\end{pgfscope}%
\begin{pgfscope}%
\definecolor{textcolor}{rgb}{0.150000,0.150000,0.150000}%
\pgfsetstrokecolor{textcolor}%
\pgfsetfillcolor{textcolor}%
\pgftext[x=4.168264in,y=0.472591in,,top]{\color{textcolor}\rmfamily\fontsize{24.000000}{28.800000}\selectfont Time (ms)}%
\end{pgfscope}%
\begin{pgfscope}%
\pgfpathrectangle{\pgfqpoint{0.802099in}{0.894194in}}{\pgfqpoint{6.732329in}{4.119217in}}%
\pgfusepath{clip}%
\pgfsetroundcap%
\pgfsetroundjoin%
\pgfsetlinewidth{1.003750pt}%
\definecolor{currentstroke}{rgb}{1.000000,1.000000,1.000000}%
\pgfsetstrokecolor{currentstroke}%
\pgfsetdash{}{0pt}%
\pgfpathmoveto{\pgfqpoint{0.802099in}{1.718835in}}%
\pgfpathlineto{\pgfqpoint{7.534428in}{1.718835in}}%
\pgfusepath{stroke}%
\end{pgfscope}%
\begin{pgfscope}%
\definecolor{textcolor}{rgb}{0.150000,0.150000,0.150000}%
\pgfsetstrokecolor{textcolor}%
\pgfsetfillcolor{textcolor}%
\pgftext[x=0.528147in, y=1.613312in, left, base]{\color{textcolor}\rmfamily\fontsize{20.000000}{24.000000}\selectfont 2}%
\end{pgfscope}%
\begin{pgfscope}%
\pgfpathrectangle{\pgfqpoint{0.802099in}{0.894194in}}{\pgfqpoint{6.732329in}{4.119217in}}%
\pgfusepath{clip}%
\pgfsetroundcap%
\pgfsetroundjoin%
\pgfsetlinewidth{1.003750pt}%
\definecolor{currentstroke}{rgb}{1.000000,1.000000,1.000000}%
\pgfsetstrokecolor{currentstroke}%
\pgfsetdash{}{0pt}%
\pgfpathmoveto{\pgfqpoint{0.802099in}{2.714777in}}%
\pgfpathlineto{\pgfqpoint{7.534428in}{2.714777in}}%
\pgfusepath{stroke}%
\end{pgfscope}%
\begin{pgfscope}%
\definecolor{textcolor}{rgb}{0.150000,0.150000,0.150000}%
\pgfsetstrokecolor{textcolor}%
\pgfsetfillcolor{textcolor}%
\pgftext[x=0.528147in, y=2.609254in, left, base]{\color{textcolor}\rmfamily\fontsize{20.000000}{24.000000}\selectfont 4}%
\end{pgfscope}%
\begin{pgfscope}%
\pgfpathrectangle{\pgfqpoint{0.802099in}{0.894194in}}{\pgfqpoint{6.732329in}{4.119217in}}%
\pgfusepath{clip}%
\pgfsetroundcap%
\pgfsetroundjoin%
\pgfsetlinewidth{1.003750pt}%
\definecolor{currentstroke}{rgb}{1.000000,1.000000,1.000000}%
\pgfsetstrokecolor{currentstroke}%
\pgfsetdash{}{0pt}%
\pgfpathmoveto{\pgfqpoint{0.802099in}{3.710719in}}%
\pgfpathlineto{\pgfqpoint{7.534428in}{3.710719in}}%
\pgfusepath{stroke}%
\end{pgfscope}%
\begin{pgfscope}%
\definecolor{textcolor}{rgb}{0.150000,0.150000,0.150000}%
\pgfsetstrokecolor{textcolor}%
\pgfsetfillcolor{textcolor}%
\pgftext[x=0.528147in, y=3.605196in, left, base]{\color{textcolor}\rmfamily\fontsize{20.000000}{24.000000}\selectfont 6}%
\end{pgfscope}%
\begin{pgfscope}%
\pgfpathrectangle{\pgfqpoint{0.802099in}{0.894194in}}{\pgfqpoint{6.732329in}{4.119217in}}%
\pgfusepath{clip}%
\pgfsetroundcap%
\pgfsetroundjoin%
\pgfsetlinewidth{1.003750pt}%
\definecolor{currentstroke}{rgb}{1.000000,1.000000,1.000000}%
\pgfsetstrokecolor{currentstroke}%
\pgfsetdash{}{0pt}%
\pgfpathmoveto{\pgfqpoint{0.802099in}{4.706661in}}%
\pgfpathlineto{\pgfqpoint{7.534428in}{4.706661in}}%
\pgfusepath{stroke}%
\end{pgfscope}%
\begin{pgfscope}%
\definecolor{textcolor}{rgb}{0.150000,0.150000,0.150000}%
\pgfsetstrokecolor{textcolor}%
\pgfsetfillcolor{textcolor}%
\pgftext[x=0.528147in, y=4.601138in, left, base]{\color{textcolor}\rmfamily\fontsize{20.000000}{24.000000}\selectfont 8}%
\end{pgfscope}%
\begin{pgfscope}%
\definecolor{textcolor}{rgb}{0.150000,0.150000,0.150000}%
\pgfsetstrokecolor{textcolor}%
\pgfsetfillcolor{textcolor}%
\pgftext[x=0.472591in,y=2.953803in,,bottom,rotate=90.000000]{\color{textcolor}\rmfamily\fontsize{24.000000}{28.800000}\selectfont Pulse Amplitude (V)}%
\end{pgfscope}%
\begin{pgfscope}%
\pgfpathrectangle{\pgfqpoint{0.802099in}{0.894194in}}{\pgfqpoint{6.732329in}{4.119217in}}%
\pgfusepath{clip}%
\pgfsetbuttcap%
\pgfsetroundjoin%
\definecolor{currentfill}{rgb}{0.698039,0.133333,0.133333}%
\pgfsetfillcolor{currentfill}%
\pgfsetlinewidth{0.301125pt}%
\definecolor{currentstroke}{rgb}{0.698039,0.133333,0.133333}%
\pgfsetstrokecolor{currentstroke}%
\pgfsetdash{}{0pt}%
\pgfsys@defobject{currentmarker}{\pgfqpoint{-0.048611in}{-0.048611in}}{\pgfqpoint{0.048611in}{0.048611in}}{%
\pgfpathmoveto{\pgfqpoint{0.000000in}{-0.048611in}}%
\pgfpathcurveto{\pgfqpoint{0.012892in}{-0.048611in}}{\pgfqpoint{0.025257in}{-0.043489in}}{\pgfqpoint{0.034373in}{-0.034373in}}%
\pgfpathcurveto{\pgfqpoint{0.043489in}{-0.025257in}}{\pgfqpoint{0.048611in}{-0.012892in}}{\pgfqpoint{0.048611in}{0.000000in}}%
\pgfpathcurveto{\pgfqpoint{0.048611in}{0.012892in}}{\pgfqpoint{0.043489in}{0.025257in}}{\pgfqpoint{0.034373in}{0.034373in}}%
\pgfpathcurveto{\pgfqpoint{0.025257in}{0.043489in}}{\pgfqpoint{0.012892in}{0.048611in}}{\pgfqpoint{0.000000in}{0.048611in}}%
\pgfpathcurveto{\pgfqpoint{-0.012892in}{0.048611in}}{\pgfqpoint{-0.025257in}{0.043489in}}{\pgfqpoint{-0.034373in}{0.034373in}}%
\pgfpathcurveto{\pgfqpoint{-0.043489in}{0.025257in}}{\pgfqpoint{-0.048611in}{0.012892in}}{\pgfqpoint{-0.048611in}{0.000000in}}%
\pgfpathcurveto{\pgfqpoint{-0.048611in}{-0.012892in}}{\pgfqpoint{-0.043489in}{-0.025257in}}{\pgfqpoint{-0.034373in}{-0.034373in}}%
\pgfpathcurveto{\pgfqpoint{-0.025257in}{-0.043489in}}{\pgfqpoint{-0.012892in}{-0.048611in}}{\pgfqpoint{0.000000in}{-0.048611in}}%
\pgfpathlineto{\pgfqpoint{0.000000in}{-0.048611in}}%
\pgfpathclose%
\pgfusepath{stroke,fill}%
}%
\begin{pgfscope}%
\pgfsys@transformshift{1.108114in}{4.826174in}%
\pgfsys@useobject{currentmarker}{}%
\end{pgfscope}%
\begin{pgfscope}%
\pgfsys@transformshift{1.578907in}{4.258487in}%
\pgfsys@useobject{currentmarker}{}%
\end{pgfscope}%
\begin{pgfscope}%
\pgfsys@transformshift{2.049699in}{3.511531in}%
\pgfsys@useobject{currentmarker}{}%
\end{pgfscope}%
\begin{pgfscope}%
\pgfsys@transformshift{2.520491in}{3.152991in}%
\pgfsys@useobject{currentmarker}{}%
\end{pgfscope}%
\begin{pgfscope}%
\pgfsys@transformshift{2.991283in}{2.754614in}%
\pgfsys@useobject{currentmarker}{}%
\end{pgfscope}%
\begin{pgfscope}%
\pgfsys@transformshift{3.462076in}{2.435913in}%
\pgfsys@useobject{currentmarker}{}%
\end{pgfscope}%
\begin{pgfscope}%
\pgfsys@transformshift{3.932868in}{2.157049in}%
\pgfsys@useobject{currentmarker}{}%
\end{pgfscope}%
\begin{pgfscope}%
\pgfsys@transformshift{4.403660in}{1.957861in}%
\pgfsys@useobject{currentmarker}{}%
\end{pgfscope}%
\begin{pgfscope}%
\pgfsys@transformshift{4.874452in}{2.216806in}%
\pgfsys@useobject{currentmarker}{}%
\end{pgfscope}%
\begin{pgfscope}%
\pgfsys@transformshift{5.345244in}{1.519646in}%
\pgfsys@useobject{currentmarker}{}%
\end{pgfscope}%
\begin{pgfscope}%
\pgfsys@transformshift{5.816037in}{1.439971in}%
\pgfsys@useobject{currentmarker}{}%
\end{pgfscope}%
\begin{pgfscope}%
\pgfsys@transformshift{6.286829in}{1.280620in}%
\pgfsys@useobject{currentmarker}{}%
\end{pgfscope}%
\begin{pgfscope}%
\pgfsys@transformshift{6.757621in}{1.240782in}%
\pgfsys@useobject{currentmarker}{}%
\end{pgfscope}%
\begin{pgfscope}%
\pgfsys@transformshift{7.228413in}{1.081432in}%
\pgfsys@useobject{currentmarker}{}%
\end{pgfscope}%
\end{pgfscope}%
\begin{pgfscope}%
\pgfpathrectangle{\pgfqpoint{0.802099in}{0.894194in}}{\pgfqpoint{6.732329in}{4.119217in}}%
\pgfusepath{clip}%
\pgfsetroundcap%
\pgfsetroundjoin%
\pgfsetlinewidth{1.756562pt}%
\definecolor{currentstroke}{rgb}{0.298039,0.447059,0.690196}%
\pgfsetstrokecolor{currentstroke}%
\pgfsetdash{}{0pt}%
\pgfpathmoveto{\pgfqpoint{1.108114in}{4.797762in}}%
\pgfpathlineto{\pgfqpoint{1.169936in}{4.709357in}}%
\pgfpathlineto{\pgfqpoint{1.231757in}{4.622845in}}%
\pgfpathlineto{\pgfqpoint{1.293578in}{4.538185in}}%
\pgfpathlineto{\pgfqpoint{1.355399in}{4.455339in}}%
\pgfpathlineto{\pgfqpoint{1.417220in}{4.374266in}}%
\pgfpathlineto{\pgfqpoint{1.479042in}{4.294930in}}%
\pgfpathlineto{\pgfqpoint{1.540863in}{4.217292in}}%
\pgfpathlineto{\pgfqpoint{1.602684in}{4.141317in}}%
\pgfpathlineto{\pgfqpoint{1.664505in}{4.066969in}}%
\pgfpathlineto{\pgfqpoint{1.726326in}{3.994213in}}%
\pgfpathlineto{\pgfqpoint{1.788148in}{3.923015in}}%
\pgfpathlineto{\pgfqpoint{1.849969in}{3.853342in}}%
\pgfpathlineto{\pgfqpoint{1.911790in}{3.785161in}}%
\pgfpathlineto{\pgfqpoint{1.973611in}{3.718440in}}%
\pgfpathlineto{\pgfqpoint{2.035432in}{3.653147in}}%
\pgfpathlineto{\pgfqpoint{2.097254in}{3.589253in}}%
\pgfpathlineto{\pgfqpoint{2.159075in}{3.526727in}}%
\pgfpathlineto{\pgfqpoint{2.220896in}{3.465540in}}%
\pgfpathlineto{\pgfqpoint{2.282717in}{3.405663in}}%
\pgfpathlineto{\pgfqpoint{2.344538in}{3.347069in}}%
\pgfpathlineto{\pgfqpoint{2.406360in}{3.289729in}}%
\pgfpathlineto{\pgfqpoint{2.468181in}{3.233617in}}%
\pgfpathlineto{\pgfqpoint{2.530002in}{3.178707in}}%
\pgfpathlineto{\pgfqpoint{2.591823in}{3.124972in}}%
\pgfpathlineto{\pgfqpoint{2.653644in}{3.072388in}}%
\pgfpathlineto{\pgfqpoint{2.715466in}{3.020931in}}%
\pgfpathlineto{\pgfqpoint{2.777287in}{2.970575in}}%
\pgfpathlineto{\pgfqpoint{2.839108in}{2.921298in}}%
\pgfpathlineto{\pgfqpoint{2.900929in}{2.873075in}}%
\pgfpathlineto{\pgfqpoint{2.962750in}{2.825886in}}%
\pgfpathlineto{\pgfqpoint{3.024572in}{2.779707in}}%
\pgfpathlineto{\pgfqpoint{3.086393in}{2.734517in}}%
\pgfpathlineto{\pgfqpoint{3.148214in}{2.690294in}}%
\pgfpathlineto{\pgfqpoint{3.210035in}{2.647019in}}%
\pgfpathlineto{\pgfqpoint{3.271856in}{2.604670in}}%
\pgfpathlineto{\pgfqpoint{3.333678in}{2.563228in}}%
\pgfpathlineto{\pgfqpoint{3.395499in}{2.522674in}}%
\pgfpathlineto{\pgfqpoint{3.457320in}{2.482988in}}%
\pgfpathlineto{\pgfqpoint{3.519141in}{2.444152in}}%
\pgfpathlineto{\pgfqpoint{3.580962in}{2.406147in}}%
\pgfpathlineto{\pgfqpoint{3.642784in}{2.368957in}}%
\pgfpathlineto{\pgfqpoint{3.704605in}{2.332562in}}%
\pgfpathlineto{\pgfqpoint{3.766426in}{2.296947in}}%
\pgfpathlineto{\pgfqpoint{3.828247in}{2.262095in}}%
\pgfpathlineto{\pgfqpoint{3.890068in}{2.227989in}}%
\pgfpathlineto{\pgfqpoint{3.951890in}{2.194614in}}%
\pgfpathlineto{\pgfqpoint{4.013711in}{2.161953in}}%
\pgfpathlineto{\pgfqpoint{4.075532in}{2.129992in}}%
\pgfpathlineto{\pgfqpoint{4.137353in}{2.098715in}}%
\pgfpathlineto{\pgfqpoint{4.199174in}{2.068107in}}%
\pgfpathlineto{\pgfqpoint{4.260996in}{2.038156in}}%
\pgfpathlineto{\pgfqpoint{4.322817in}{2.008845in}}%
\pgfpathlineto{\pgfqpoint{4.384638in}{1.980163in}}%
\pgfpathlineto{\pgfqpoint{4.446459in}{1.952094in}}%
\pgfpathlineto{\pgfqpoint{4.508281in}{1.924627in}}%
\pgfpathlineto{\pgfqpoint{4.570102in}{1.897747in}}%
\pgfpathlineto{\pgfqpoint{4.631923in}{1.871444in}}%
\pgfpathlineto{\pgfqpoint{4.693744in}{1.845703in}}%
\pgfpathlineto{\pgfqpoint{4.755565in}{1.820514in}}%
\pgfpathlineto{\pgfqpoint{4.817387in}{1.795864in}}%
\pgfpathlineto{\pgfqpoint{4.879208in}{1.771743in}}%
\pgfpathlineto{\pgfqpoint{4.941029in}{1.748137in}}%
\pgfpathlineto{\pgfqpoint{5.002850in}{1.725037in}}%
\pgfpathlineto{\pgfqpoint{5.064671in}{1.702432in}}%
\pgfpathlineto{\pgfqpoint{5.126493in}{1.680311in}}%
\pgfpathlineto{\pgfqpoint{5.188314in}{1.658664in}}%
\pgfpathlineto{\pgfqpoint{5.250135in}{1.637480in}}%
\pgfpathlineto{\pgfqpoint{5.311956in}{1.616750in}}%
\pgfpathlineto{\pgfqpoint{5.373777in}{1.596463in}}%
\pgfpathlineto{\pgfqpoint{5.435599in}{1.576611in}}%
\pgfpathlineto{\pgfqpoint{5.497420in}{1.557185in}}%
\pgfpathlineto{\pgfqpoint{5.559241in}{1.538174in}}%
\pgfpathlineto{\pgfqpoint{5.621062in}{1.519570in}}%
\pgfpathlineto{\pgfqpoint{5.682883in}{1.501365in}}%
\pgfpathlineto{\pgfqpoint{5.744705in}{1.483550in}}%
\pgfpathlineto{\pgfqpoint{5.806526in}{1.466116in}}%
\pgfpathlineto{\pgfqpoint{5.868347in}{1.449055in}}%
\pgfpathlineto{\pgfqpoint{5.930168in}{1.432360in}}%
\pgfpathlineto{\pgfqpoint{5.991989in}{1.416022in}}%
\pgfpathlineto{\pgfqpoint{6.053811in}{1.400034in}}%
\pgfpathlineto{\pgfqpoint{6.115632in}{1.384389in}}%
\pgfpathlineto{\pgfqpoint{6.177453in}{1.369078in}}%
\pgfpathlineto{\pgfqpoint{6.239274in}{1.354096in}}%
\pgfpathlineto{\pgfqpoint{6.301095in}{1.339434in}}%
\pgfpathlineto{\pgfqpoint{6.362917in}{1.325086in}}%
\pgfpathlineto{\pgfqpoint{6.424738in}{1.311046in}}%
\pgfpathlineto{\pgfqpoint{6.486559in}{1.297306in}}%
\pgfpathlineto{\pgfqpoint{6.548380in}{1.283860in}}%
\pgfpathlineto{\pgfqpoint{6.610201in}{1.270703in}}%
\pgfpathlineto{\pgfqpoint{6.672023in}{1.257827in}}%
\pgfpathlineto{\pgfqpoint{6.733844in}{1.245226in}}%
\pgfpathlineto{\pgfqpoint{6.795665in}{1.232896in}}%
\pgfpathlineto{\pgfqpoint{6.857486in}{1.220830in}}%
\pgfpathlineto{\pgfqpoint{6.919307in}{1.209022in}}%
\pgfpathlineto{\pgfqpoint{6.981129in}{1.197467in}}%
\pgfpathlineto{\pgfqpoint{7.042950in}{1.186159in}}%
\pgfpathlineto{\pgfqpoint{7.104771in}{1.175094in}}%
\pgfpathlineto{\pgfqpoint{7.166592in}{1.164265in}}%
\pgfpathlineto{\pgfqpoint{7.228413in}{1.153668in}}%
\pgfusepath{stroke}%
\end{pgfscope}%
\begin{pgfscope}%
\pgfsetrectcap%
\pgfsetmiterjoin%
\pgfsetlinewidth{0.000000pt}%
\definecolor{currentstroke}{rgb}{1.000000,1.000000,1.000000}%
\pgfsetstrokecolor{currentstroke}%
\pgfsetdash{}{0pt}%
\pgfpathmoveto{\pgfqpoint{0.802099in}{0.894194in}}%
\pgfpathlineto{\pgfqpoint{0.802099in}{5.013411in}}%
\pgfusepath{}%
\end{pgfscope}%
\begin{pgfscope}%
\pgfsetrectcap%
\pgfsetmiterjoin%
\pgfsetlinewidth{0.000000pt}%
\definecolor{currentstroke}{rgb}{1.000000,1.000000,1.000000}%
\pgfsetstrokecolor{currentstroke}%
\pgfsetdash{}{0pt}%
\pgfpathmoveto{\pgfqpoint{7.534428in}{0.894194in}}%
\pgfpathlineto{\pgfqpoint{7.534428in}{5.013411in}}%
\pgfusepath{}%
\end{pgfscope}%
\begin{pgfscope}%
\pgfsetrectcap%
\pgfsetmiterjoin%
\pgfsetlinewidth{0.000000pt}%
\definecolor{currentstroke}{rgb}{1.000000,1.000000,1.000000}%
\pgfsetstrokecolor{currentstroke}%
\pgfsetdash{}{0pt}%
\pgfpathmoveto{\pgfqpoint{0.802099in}{0.894194in}}%
\pgfpathlineto{\pgfqpoint{7.534428in}{0.894194in}}%
\pgfusepath{}%
\end{pgfscope}%
\begin{pgfscope}%
\pgfsetrectcap%
\pgfsetmiterjoin%
\pgfsetlinewidth{0.000000pt}%
\definecolor{currentstroke}{rgb}{1.000000,1.000000,1.000000}%
\pgfsetstrokecolor{currentstroke}%
\pgfsetdash{}{0pt}%
\pgfpathmoveto{\pgfqpoint{0.802099in}{5.013411in}}%
\pgfpathlineto{\pgfqpoint{7.534428in}{5.013411in}}%
\pgfusepath{}%
\end{pgfscope}%
\begin{pgfscope}%
\definecolor{textcolor}{rgb}{0.150000,0.150000,0.150000}%
\pgfsetstrokecolor{textcolor}%
\pgfsetfillcolor{textcolor}%
\pgftext[x=3.638623in, y=3.706803in, left, base]{\color{textcolor}\rmfamily\fontsize{20.000000}{24.000000}\selectfont M\(\displaystyle _{x,y}\)(t) = 8.64\(\displaystyle exp[\frac{-2t}{T_2}]+\)-0.11}%
\end{pgfscope}%
\begin{pgfscope}%
\definecolor{textcolor}{rgb}{0.150000,0.150000,0.150000}%
\pgfsetstrokecolor{textcolor}%
\pgfsetfillcolor{textcolor}%
\pgftext[x=3.638623in, y=3.212748in, left, base]{\color{textcolor}\rmfamily\fontsize{20.000000}{24.000000}\selectfont T\(\displaystyle _2\) = 194.1\(\displaystyle \pm\)26.67 ms}%
\end{pgfscope}%
\begin{pgfscope}%
\definecolor{textcolor}{rgb}{0.150000,0.150000,0.150000}%
\pgfsetstrokecolor{textcolor}%
\pgfsetfillcolor{textcolor}%
\pgftext[x=4.168264in,y=5.096745in,,base]{\color{textcolor}\rmfamily\fontsize{24.000000}{28.800000}\selectfont MG T\(\displaystyle _2\) measurement of 0.01 M CuSO\(\displaystyle _4\)}%
\end{pgfscope}%
\begin{pgfscope}%
\pgfsetbuttcap%
\pgfsetroundjoin%
\definecolor{currentfill}{rgb}{0.698039,0.133333,0.133333}%
\pgfsetfillcolor{currentfill}%
\pgfsetlinewidth{0.301125pt}%
\definecolor{currentstroke}{rgb}{0.698039,0.133333,0.133333}%
\pgfsetstrokecolor{currentstroke}%
\pgfsetdash{}{0pt}%
\pgfsys@defobject{currentmarker}{\pgfqpoint{-0.048611in}{-0.048611in}}{\pgfqpoint{0.048611in}{0.048611in}}{%
\pgfpathmoveto{\pgfqpoint{0.000000in}{-0.048611in}}%
\pgfpathcurveto{\pgfqpoint{0.012892in}{-0.048611in}}{\pgfqpoint{0.025257in}{-0.043489in}}{\pgfqpoint{0.034373in}{-0.034373in}}%
\pgfpathcurveto{\pgfqpoint{0.043489in}{-0.025257in}}{\pgfqpoint{0.048611in}{-0.012892in}}{\pgfqpoint{0.048611in}{0.000000in}}%
\pgfpathcurveto{\pgfqpoint{0.048611in}{0.012892in}}{\pgfqpoint{0.043489in}{0.025257in}}{\pgfqpoint{0.034373in}{0.034373in}}%
\pgfpathcurveto{\pgfqpoint{0.025257in}{0.043489in}}{\pgfqpoint{0.012892in}{0.048611in}}{\pgfqpoint{0.000000in}{0.048611in}}%
\pgfpathcurveto{\pgfqpoint{-0.012892in}{0.048611in}}{\pgfqpoint{-0.025257in}{0.043489in}}{\pgfqpoint{-0.034373in}{0.034373in}}%
\pgfpathcurveto{\pgfqpoint{-0.043489in}{0.025257in}}{\pgfqpoint{-0.048611in}{0.012892in}}{\pgfqpoint{-0.048611in}{0.000000in}}%
\pgfpathcurveto{\pgfqpoint{-0.048611in}{-0.012892in}}{\pgfqpoint{-0.043489in}{-0.025257in}}{\pgfqpoint{-0.034373in}{-0.034373in}}%
\pgfpathcurveto{\pgfqpoint{-0.025257in}{-0.043489in}}{\pgfqpoint{-0.012892in}{-0.048611in}}{\pgfqpoint{0.000000in}{-0.048611in}}%
\pgfpathlineto{\pgfqpoint{0.000000in}{-0.048611in}}%
\pgfpathclose%
\pgfusepath{stroke,fill}%
}%
\begin{pgfscope}%
\pgfsys@transformshift{5.003558in}{4.625282in}%
\pgfsys@useobject{currentmarker}{}%
\end{pgfscope}%
\end{pgfscope}%
\begin{pgfscope}%
\definecolor{textcolor}{rgb}{0.150000,0.150000,0.150000}%
\pgfsetstrokecolor{textcolor}%
\pgfsetfillcolor{textcolor}%
\pgftext[x=5.503558in,y=4.552365in,left,base]{\color{textcolor}\rmfamily\fontsize{20.000000}{24.000000}\selectfont Data}%
\end{pgfscope}%
\begin{pgfscope}%
\pgfsetroundcap%
\pgfsetroundjoin%
\pgfsetlinewidth{1.756562pt}%
\definecolor{currentstroke}{rgb}{0.298039,0.447059,0.690196}%
\pgfsetstrokecolor{currentstroke}%
\pgfsetdash{}{0pt}%
\pgfpathmoveto{\pgfqpoint{4.725780in}{4.241873in}}%
\pgfpathlineto{\pgfqpoint{5.003558in}{4.241873in}}%
\pgfpathlineto{\pgfqpoint{5.281336in}{4.241873in}}%
\pgfusepath{stroke}%
\end{pgfscope}%
\begin{pgfscope}%
\definecolor{textcolor}{rgb}{0.150000,0.150000,0.150000}%
\pgfsetstrokecolor{textcolor}%
\pgfsetfillcolor{textcolor}%
\pgftext[x=5.503558in,y=4.144651in,left,base]{\color{textcolor}\rmfamily\fontsize{20.000000}{24.000000}\selectfont Fitted Curve}%
\end{pgfscope}%
\end{pgfpicture}%
\makeatother%
\endgroup%
}
    \caption{T2 measurement of 0.01 M CuSO$_{4} \boldsymbol{\cdot} $H$_2$O using the Meiboom-Gill pulse sequence }
    \label{fig:0d01_t2}
  \end{figure}

  \begin{figure}[h]
    \resizebox{0.45\textwidth}{!}{%% Creator: Matplotlib, PGF backend
%%
%% To include the figure in your LaTeX document, write
%%   \input{<filename>.pgf}
%%
%% Make sure the required packages are loaded in your preamble
%%   \usepackage{pgf}
%%
%% Also ensure that all the required font packages are loaded; for instance,
%% the lmodern package is sometimes necessary when using math font.
%%   \usepackage{lmodern}
%%
%% Figures using additional raster images can only be included by \input if
%% they are in the same directory as the main LaTeX file. For loading figures
%% from other directories you can use the `import` package
%%   \usepackage{import}
%%
%% and then include the figures with
%%   \import{<path to file>}{<filename>.pgf}
%%
%% Matplotlib used the following preamble
%%   \usepackage{fontspec}
%%   \setmainfont{DejaVuSerif.ttf}[Path=\detokenize{/usr/share/matplotlib/mpl-data/fonts/ttf/}]
%%   \setsansfont{DejaVuSans.ttf}[Path=\detokenize{/usr/share/matplotlib/mpl-data/fonts/ttf/}]
%%   \setmonofont{DejaVuSansMono.ttf}[Path=\detokenize{/usr/share/matplotlib/mpl-data/fonts/ttf/}]
%%
\begingroup%
\makeatletter%
\begin{pgfpicture}%
\pgfpathrectangle{\pgfpointorigin}{\pgfqpoint{8.000000in}{5.500000in}}%
\pgfusepath{use as bounding box, clip}%
\begin{pgfscope}%
\pgfsetbuttcap%
\pgfsetmiterjoin%
\definecolor{currentfill}{rgb}{1.000000,1.000000,1.000000}%
\pgfsetfillcolor{currentfill}%
\pgfsetlinewidth{0.000000pt}%
\definecolor{currentstroke}{rgb}{1.000000,1.000000,1.000000}%
\pgfsetstrokecolor{currentstroke}%
\pgfsetdash{}{0pt}%
\pgfpathmoveto{\pgfqpoint{0.000000in}{0.000000in}}%
\pgfpathlineto{\pgfqpoint{8.000000in}{0.000000in}}%
\pgfpathlineto{\pgfqpoint{8.000000in}{5.500000in}}%
\pgfpathlineto{\pgfqpoint{0.000000in}{5.500000in}}%
\pgfpathlineto{\pgfqpoint{0.000000in}{0.000000in}}%
\pgfpathclose%
\pgfusepath{fill}%
\end{pgfscope}%
\begin{pgfscope}%
\pgfsetbuttcap%
\pgfsetmiterjoin%
\definecolor{currentfill}{rgb}{0.917647,0.917647,0.949020}%
\pgfsetfillcolor{currentfill}%
\pgfsetlinewidth{0.000000pt}%
\definecolor{currentstroke}{rgb}{0.000000,0.000000,0.000000}%
\pgfsetstrokecolor{currentstroke}%
\pgfsetstrokeopacity{0.000000}%
\pgfsetdash{}{0pt}%
\pgfpathmoveto{\pgfqpoint{0.802099in}{0.894194in}}%
\pgfpathlineto{\pgfqpoint{7.814178in}{0.894194in}}%
\pgfpathlineto{\pgfqpoint{7.814178in}{5.013411in}}%
\pgfpathlineto{\pgfqpoint{0.802099in}{5.013411in}}%
\pgfpathlineto{\pgfqpoint{0.802099in}{0.894194in}}%
\pgfpathclose%
\pgfusepath{fill}%
\end{pgfscope}%
\begin{pgfscope}%
\pgfpathrectangle{\pgfqpoint{0.802099in}{0.894194in}}{\pgfqpoint{7.012079in}{4.119217in}}%
\pgfusepath{clip}%
\pgfsetroundcap%
\pgfsetroundjoin%
\pgfsetlinewidth{1.003750pt}%
\definecolor{currentstroke}{rgb}{1.000000,1.000000,1.000000}%
\pgfsetstrokecolor{currentstroke}%
\pgfsetdash{}{0pt}%
\pgfpathmoveto{\pgfqpoint{1.014587in}{0.894194in}}%
\pgfpathlineto{\pgfqpoint{1.014587in}{5.013411in}}%
\pgfusepath{stroke}%
\end{pgfscope}%
\begin{pgfscope}%
\definecolor{textcolor}{rgb}{0.150000,0.150000,0.150000}%
\pgfsetstrokecolor{textcolor}%
\pgfsetfillcolor{textcolor}%
\pgftext[x=1.014587in,y=0.796972in,,top]{\color{textcolor}\rmfamily\fontsize{20.000000}{24.000000}\selectfont 0}%
\end{pgfscope}%
\begin{pgfscope}%
\pgfpathrectangle{\pgfqpoint{0.802099in}{0.894194in}}{\pgfqpoint{7.012079in}{4.119217in}}%
\pgfusepath{clip}%
\pgfsetroundcap%
\pgfsetroundjoin%
\pgfsetlinewidth{1.003750pt}%
\definecolor{currentstroke}{rgb}{1.000000,1.000000,1.000000}%
\pgfsetstrokecolor{currentstroke}%
\pgfsetdash{}{0pt}%
\pgfpathmoveto{\pgfqpoint{2.342632in}{0.894194in}}%
\pgfpathlineto{\pgfqpoint{2.342632in}{5.013411in}}%
\pgfusepath{stroke}%
\end{pgfscope}%
\begin{pgfscope}%
\definecolor{textcolor}{rgb}{0.150000,0.150000,0.150000}%
\pgfsetstrokecolor{textcolor}%
\pgfsetfillcolor{textcolor}%
\pgftext[x=2.342632in,y=0.796972in,,top]{\color{textcolor}\rmfamily\fontsize{20.000000}{24.000000}\selectfont 10}%
\end{pgfscope}%
\begin{pgfscope}%
\pgfpathrectangle{\pgfqpoint{0.802099in}{0.894194in}}{\pgfqpoint{7.012079in}{4.119217in}}%
\pgfusepath{clip}%
\pgfsetroundcap%
\pgfsetroundjoin%
\pgfsetlinewidth{1.003750pt}%
\definecolor{currentstroke}{rgb}{1.000000,1.000000,1.000000}%
\pgfsetstrokecolor{currentstroke}%
\pgfsetdash{}{0pt}%
\pgfpathmoveto{\pgfqpoint{3.670677in}{0.894194in}}%
\pgfpathlineto{\pgfqpoint{3.670677in}{5.013411in}}%
\pgfusepath{stroke}%
\end{pgfscope}%
\begin{pgfscope}%
\definecolor{textcolor}{rgb}{0.150000,0.150000,0.150000}%
\pgfsetstrokecolor{textcolor}%
\pgfsetfillcolor{textcolor}%
\pgftext[x=3.670677in,y=0.796972in,,top]{\color{textcolor}\rmfamily\fontsize{20.000000}{24.000000}\selectfont 20}%
\end{pgfscope}%
\begin{pgfscope}%
\pgfpathrectangle{\pgfqpoint{0.802099in}{0.894194in}}{\pgfqpoint{7.012079in}{4.119217in}}%
\pgfusepath{clip}%
\pgfsetroundcap%
\pgfsetroundjoin%
\pgfsetlinewidth{1.003750pt}%
\definecolor{currentstroke}{rgb}{1.000000,1.000000,1.000000}%
\pgfsetstrokecolor{currentstroke}%
\pgfsetdash{}{0pt}%
\pgfpathmoveto{\pgfqpoint{4.998722in}{0.894194in}}%
\pgfpathlineto{\pgfqpoint{4.998722in}{5.013411in}}%
\pgfusepath{stroke}%
\end{pgfscope}%
\begin{pgfscope}%
\definecolor{textcolor}{rgb}{0.150000,0.150000,0.150000}%
\pgfsetstrokecolor{textcolor}%
\pgfsetfillcolor{textcolor}%
\pgftext[x=4.998722in,y=0.796972in,,top]{\color{textcolor}\rmfamily\fontsize{20.000000}{24.000000}\selectfont 30}%
\end{pgfscope}%
\begin{pgfscope}%
\pgfpathrectangle{\pgfqpoint{0.802099in}{0.894194in}}{\pgfqpoint{7.012079in}{4.119217in}}%
\pgfusepath{clip}%
\pgfsetroundcap%
\pgfsetroundjoin%
\pgfsetlinewidth{1.003750pt}%
\definecolor{currentstroke}{rgb}{1.000000,1.000000,1.000000}%
\pgfsetstrokecolor{currentstroke}%
\pgfsetdash{}{0pt}%
\pgfpathmoveto{\pgfqpoint{6.326768in}{0.894194in}}%
\pgfpathlineto{\pgfqpoint{6.326768in}{5.013411in}}%
\pgfusepath{stroke}%
\end{pgfscope}%
\begin{pgfscope}%
\definecolor{textcolor}{rgb}{0.150000,0.150000,0.150000}%
\pgfsetstrokecolor{textcolor}%
\pgfsetfillcolor{textcolor}%
\pgftext[x=6.326768in,y=0.796972in,,top]{\color{textcolor}\rmfamily\fontsize{20.000000}{24.000000}\selectfont 40}%
\end{pgfscope}%
\begin{pgfscope}%
\pgfpathrectangle{\pgfqpoint{0.802099in}{0.894194in}}{\pgfqpoint{7.012079in}{4.119217in}}%
\pgfusepath{clip}%
\pgfsetroundcap%
\pgfsetroundjoin%
\pgfsetlinewidth{1.003750pt}%
\definecolor{currentstroke}{rgb}{1.000000,1.000000,1.000000}%
\pgfsetstrokecolor{currentstroke}%
\pgfsetdash{}{0pt}%
\pgfpathmoveto{\pgfqpoint{7.654813in}{0.894194in}}%
\pgfpathlineto{\pgfqpoint{7.654813in}{5.013411in}}%
\pgfusepath{stroke}%
\end{pgfscope}%
\begin{pgfscope}%
\definecolor{textcolor}{rgb}{0.150000,0.150000,0.150000}%
\pgfsetstrokecolor{textcolor}%
\pgfsetfillcolor{textcolor}%
\pgftext[x=7.654813in,y=0.796972in,,top]{\color{textcolor}\rmfamily\fontsize{20.000000}{24.000000}\selectfont 50}%
\end{pgfscope}%
\begin{pgfscope}%
\definecolor{textcolor}{rgb}{0.150000,0.150000,0.150000}%
\pgfsetstrokecolor{textcolor}%
\pgfsetfillcolor{textcolor}%
\pgftext[x=4.308139in,y=0.472591in,,top]{\color{textcolor}\rmfamily\fontsize{24.000000}{28.800000}\selectfont Time (ms)}%
\end{pgfscope}%
\begin{pgfscope}%
\pgfpathrectangle{\pgfqpoint{0.802099in}{0.894194in}}{\pgfqpoint{7.012079in}{4.119217in}}%
\pgfusepath{clip}%
\pgfsetroundcap%
\pgfsetroundjoin%
\pgfsetlinewidth{1.003750pt}%
\definecolor{currentstroke}{rgb}{1.000000,1.000000,1.000000}%
\pgfsetstrokecolor{currentstroke}%
\pgfsetdash{}{0pt}%
\pgfpathmoveto{\pgfqpoint{0.802099in}{1.635700in}}%
\pgfpathlineto{\pgfqpoint{7.814178in}{1.635700in}}%
\pgfusepath{stroke}%
\end{pgfscope}%
\begin{pgfscope}%
\definecolor{textcolor}{rgb}{0.150000,0.150000,0.150000}%
\pgfsetstrokecolor{textcolor}%
\pgfsetfillcolor{textcolor}%
\pgftext[x=0.528147in, y=1.530177in, left, base]{\color{textcolor}\rmfamily\fontsize{20.000000}{24.000000}\selectfont 2}%
\end{pgfscope}%
\begin{pgfscope}%
\pgfpathrectangle{\pgfqpoint{0.802099in}{0.894194in}}{\pgfqpoint{7.012079in}{4.119217in}}%
\pgfusepath{clip}%
\pgfsetroundcap%
\pgfsetroundjoin%
\pgfsetlinewidth{1.003750pt}%
\definecolor{currentstroke}{rgb}{1.000000,1.000000,1.000000}%
\pgfsetstrokecolor{currentstroke}%
\pgfsetdash{}{0pt}%
\pgfpathmoveto{\pgfqpoint{0.802099in}{2.501745in}}%
\pgfpathlineto{\pgfqpoint{7.814178in}{2.501745in}}%
\pgfusepath{stroke}%
\end{pgfscope}%
\begin{pgfscope}%
\definecolor{textcolor}{rgb}{0.150000,0.150000,0.150000}%
\pgfsetstrokecolor{textcolor}%
\pgfsetfillcolor{textcolor}%
\pgftext[x=0.528147in, y=2.396222in, left, base]{\color{textcolor}\rmfamily\fontsize{20.000000}{24.000000}\selectfont 4}%
\end{pgfscope}%
\begin{pgfscope}%
\pgfpathrectangle{\pgfqpoint{0.802099in}{0.894194in}}{\pgfqpoint{7.012079in}{4.119217in}}%
\pgfusepath{clip}%
\pgfsetroundcap%
\pgfsetroundjoin%
\pgfsetlinewidth{1.003750pt}%
\definecolor{currentstroke}{rgb}{1.000000,1.000000,1.000000}%
\pgfsetstrokecolor{currentstroke}%
\pgfsetdash{}{0pt}%
\pgfpathmoveto{\pgfqpoint{0.802099in}{3.367790in}}%
\pgfpathlineto{\pgfqpoint{7.814178in}{3.367790in}}%
\pgfusepath{stroke}%
\end{pgfscope}%
\begin{pgfscope}%
\definecolor{textcolor}{rgb}{0.150000,0.150000,0.150000}%
\pgfsetstrokecolor{textcolor}%
\pgfsetfillcolor{textcolor}%
\pgftext[x=0.528147in, y=3.262267in, left, base]{\color{textcolor}\rmfamily\fontsize{20.000000}{24.000000}\selectfont 6}%
\end{pgfscope}%
\begin{pgfscope}%
\pgfpathrectangle{\pgfqpoint{0.802099in}{0.894194in}}{\pgfqpoint{7.012079in}{4.119217in}}%
\pgfusepath{clip}%
\pgfsetroundcap%
\pgfsetroundjoin%
\pgfsetlinewidth{1.003750pt}%
\definecolor{currentstroke}{rgb}{1.000000,1.000000,1.000000}%
\pgfsetstrokecolor{currentstroke}%
\pgfsetdash{}{0pt}%
\pgfpathmoveto{\pgfqpoint{0.802099in}{4.233836in}}%
\pgfpathlineto{\pgfqpoint{7.814178in}{4.233836in}}%
\pgfusepath{stroke}%
\end{pgfscope}%
\begin{pgfscope}%
\definecolor{textcolor}{rgb}{0.150000,0.150000,0.150000}%
\pgfsetstrokecolor{textcolor}%
\pgfsetfillcolor{textcolor}%
\pgftext[x=0.528147in, y=4.128313in, left, base]{\color{textcolor}\rmfamily\fontsize{20.000000}{24.000000}\selectfont 8}%
\end{pgfscope}%
\begin{pgfscope}%
\definecolor{textcolor}{rgb}{0.150000,0.150000,0.150000}%
\pgfsetstrokecolor{textcolor}%
\pgfsetfillcolor{textcolor}%
\pgftext[x=0.472591in,y=2.953803in,,bottom,rotate=90.000000]{\color{textcolor}\rmfamily\fontsize{24.000000}{28.800000}\selectfont Pulse Amplitude (V)}%
\end{pgfscope}%
\begin{pgfscope}%
\pgfpathrectangle{\pgfqpoint{0.802099in}{0.894194in}}{\pgfqpoint{7.012079in}{4.119217in}}%
\pgfusepath{clip}%
\pgfsetbuttcap%
\pgfsetroundjoin%
\definecolor{currentfill}{rgb}{0.698039,0.133333,0.133333}%
\pgfsetfillcolor{currentfill}%
\pgfsetlinewidth{0.301125pt}%
\definecolor{currentstroke}{rgb}{0.698039,0.133333,0.133333}%
\pgfsetstrokecolor{currentstroke}%
\pgfsetdash{}{0pt}%
\pgfsys@defobject{currentmarker}{\pgfqpoint{-0.048611in}{-0.048611in}}{\pgfqpoint{0.048611in}{0.048611in}}{%
\pgfpathmoveto{\pgfqpoint{0.000000in}{-0.048611in}}%
\pgfpathcurveto{\pgfqpoint{0.012892in}{-0.048611in}}{\pgfqpoint{0.025257in}{-0.043489in}}{\pgfqpoint{0.034373in}{-0.034373in}}%
\pgfpathcurveto{\pgfqpoint{0.043489in}{-0.025257in}}{\pgfqpoint{0.048611in}{-0.012892in}}{\pgfqpoint{0.048611in}{0.000000in}}%
\pgfpathcurveto{\pgfqpoint{0.048611in}{0.012892in}}{\pgfqpoint{0.043489in}{0.025257in}}{\pgfqpoint{0.034373in}{0.034373in}}%
\pgfpathcurveto{\pgfqpoint{0.025257in}{0.043489in}}{\pgfqpoint{0.012892in}{0.048611in}}{\pgfqpoint{0.000000in}{0.048611in}}%
\pgfpathcurveto{\pgfqpoint{-0.012892in}{0.048611in}}{\pgfqpoint{-0.025257in}{0.043489in}}{\pgfqpoint{-0.034373in}{0.034373in}}%
\pgfpathcurveto{\pgfqpoint{-0.043489in}{0.025257in}}{\pgfqpoint{-0.048611in}{0.012892in}}{\pgfqpoint{-0.048611in}{0.000000in}}%
\pgfpathcurveto{\pgfqpoint{-0.048611in}{-0.012892in}}{\pgfqpoint{-0.043489in}{-0.025257in}}{\pgfqpoint{-0.034373in}{-0.034373in}}%
\pgfpathcurveto{\pgfqpoint{-0.025257in}{-0.043489in}}{\pgfqpoint{-0.012892in}{-0.048611in}}{\pgfqpoint{0.000000in}{-0.048611in}}%
\pgfpathlineto{\pgfqpoint{0.000000in}{-0.048611in}}%
\pgfpathclose%
\pgfusepath{stroke,fill}%
}%
\begin{pgfscope}%
\pgfsys@transformshift{1.120830in}{4.822746in}%
\pgfsys@useobject{currentmarker}{}%
\end{pgfscope}%
\begin{pgfscope}%
\pgfsys@transformshift{1.652048in}{4.095268in}%
\pgfsys@useobject{currentmarker}{}%
\end{pgfscope}%
\begin{pgfscope}%
\pgfsys@transformshift{2.183267in}{3.437074in}%
\pgfsys@useobject{currentmarker}{}%
\end{pgfscope}%
\begin{pgfscope}%
\pgfsys@transformshift{2.714485in}{2.917447in}%
\pgfsys@useobject{currentmarker}{}%
\end{pgfscope}%
\begin{pgfscope}%
\pgfsys@transformshift{3.245703in}{2.501745in}%
\pgfsys@useobject{currentmarker}{}%
\end{pgfscope}%
\begin{pgfscope}%
\pgfsys@transformshift{3.776921in}{2.189969in}%
\pgfsys@useobject{currentmarker}{}%
\end{pgfscope}%
\begin{pgfscope}%
\pgfsys@transformshift{4.308139in}{1.947477in}%
\pgfsys@useobject{currentmarker}{}%
\end{pgfscope}%
\begin{pgfscope}%
\pgfsys@transformshift{4.839357in}{1.774268in}%
\pgfsys@useobject{currentmarker}{}%
\end{pgfscope}%
\begin{pgfscope}%
\pgfsys@transformshift{5.370575in}{1.601059in}%
\pgfsys@useobject{currentmarker}{}%
\end{pgfscope}%
\begin{pgfscope}%
\pgfsys@transformshift{5.901793in}{1.427850in}%
\pgfsys@useobject{currentmarker}{}%
\end{pgfscope}%
\begin{pgfscope}%
\pgfsys@transformshift{6.433011in}{1.323924in}%
\pgfsys@useobject{currentmarker}{}%
\end{pgfscope}%
\begin{pgfscope}%
\pgfsys@transformshift{6.964229in}{1.185357in}%
\pgfsys@useobject{currentmarker}{}%
\end{pgfscope}%
\begin{pgfscope}%
\pgfsys@transformshift{7.495448in}{1.081432in}%
\pgfsys@useobject{currentmarker}{}%
\end{pgfscope}%
\end{pgfscope}%
\begin{pgfscope}%
\pgfpathrectangle{\pgfqpoint{0.802099in}{0.894194in}}{\pgfqpoint{7.012079in}{4.119217in}}%
\pgfusepath{clip}%
\pgfsetroundcap%
\pgfsetroundjoin%
\pgfsetlinewidth{1.756562pt}%
\definecolor{currentstroke}{rgb}{0.298039,0.447059,0.690196}%
\pgfsetstrokecolor{currentstroke}%
\pgfsetdash{}{0pt}%
\pgfpathmoveto{\pgfqpoint{1.120830in}{4.826174in}}%
\pgfpathlineto{\pgfqpoint{1.185220in}{4.724828in}}%
\pgfpathlineto{\pgfqpoint{1.249610in}{4.626041in}}%
\pgfpathlineto{\pgfqpoint{1.314001in}{4.529748in}}%
\pgfpathlineto{\pgfqpoint{1.378391in}{4.435887in}}%
\pgfpathlineto{\pgfqpoint{1.442781in}{4.344397in}}%
\pgfpathlineto{\pgfqpoint{1.507171in}{4.255216in}}%
\pgfpathlineto{\pgfqpoint{1.571561in}{4.168288in}}%
\pgfpathlineto{\pgfqpoint{1.635951in}{4.083554in}}%
\pgfpathlineto{\pgfqpoint{1.700341in}{4.000961in}}%
\pgfpathlineto{\pgfqpoint{1.764731in}{3.920453in}}%
\pgfpathlineto{\pgfqpoint{1.829121in}{3.841978in}}%
\pgfpathlineto{\pgfqpoint{1.893511in}{3.765484in}}%
\pgfpathlineto{\pgfqpoint{1.957901in}{3.690923in}}%
\pgfpathlineto{\pgfqpoint{2.022291in}{3.618244in}}%
\pgfpathlineto{\pgfqpoint{2.086681in}{3.547400in}}%
\pgfpathlineto{\pgfqpoint{2.151071in}{3.478345in}}%
\pgfpathlineto{\pgfqpoint{2.215462in}{3.411034in}}%
\pgfpathlineto{\pgfqpoint{2.279852in}{3.345423in}}%
\pgfpathlineto{\pgfqpoint{2.344242in}{3.281469in}}%
\pgfpathlineto{\pgfqpoint{2.408632in}{3.219129in}}%
\pgfpathlineto{\pgfqpoint{2.473022in}{3.158364in}}%
\pgfpathlineto{\pgfqpoint{2.537412in}{3.099133in}}%
\pgfpathlineto{\pgfqpoint{2.601802in}{3.041398in}}%
\pgfpathlineto{\pgfqpoint{2.666192in}{2.985121in}}%
\pgfpathlineto{\pgfqpoint{2.730582in}{2.930265in}}%
\pgfpathlineto{\pgfqpoint{2.794972in}{2.876794in}}%
\pgfpathlineto{\pgfqpoint{2.859362in}{2.824674in}}%
\pgfpathlineto{\pgfqpoint{2.923752in}{2.773869in}}%
\pgfpathlineto{\pgfqpoint{2.988142in}{2.724348in}}%
\pgfpathlineto{\pgfqpoint{3.052532in}{2.676077in}}%
\pgfpathlineto{\pgfqpoint{3.116923in}{2.629024in}}%
\pgfpathlineto{\pgfqpoint{3.181313in}{2.583161in}}%
\pgfpathlineto{\pgfqpoint{3.245703in}{2.538455in}}%
\pgfpathlineto{\pgfqpoint{3.310093in}{2.494878in}}%
\pgfpathlineto{\pgfqpoint{3.374483in}{2.452401in}}%
\pgfpathlineto{\pgfqpoint{3.438873in}{2.410998in}}%
\pgfpathlineto{\pgfqpoint{3.503263in}{2.370639in}}%
\pgfpathlineto{\pgfqpoint{3.567653in}{2.331300in}}%
\pgfpathlineto{\pgfqpoint{3.632043in}{2.292954in}}%
\pgfpathlineto{\pgfqpoint{3.696433in}{2.255577in}}%
\pgfpathlineto{\pgfqpoint{3.760823in}{2.219143in}}%
\pgfpathlineto{\pgfqpoint{3.825213in}{2.183630in}}%
\pgfpathlineto{\pgfqpoint{3.889603in}{2.149013in}}%
\pgfpathlineto{\pgfqpoint{3.953994in}{2.115270in}}%
\pgfpathlineto{\pgfqpoint{4.018384in}{2.082379in}}%
\pgfpathlineto{\pgfqpoint{4.082774in}{2.050319in}}%
\pgfpathlineto{\pgfqpoint{4.147164in}{2.019069in}}%
\pgfpathlineto{\pgfqpoint{4.211554in}{1.988608in}}%
\pgfpathlineto{\pgfqpoint{4.275944in}{1.958915in}}%
\pgfpathlineto{\pgfqpoint{4.340334in}{1.929973in}}%
\pgfpathlineto{\pgfqpoint{4.404724in}{1.901762in}}%
\pgfpathlineto{\pgfqpoint{4.469114in}{1.874262in}}%
\pgfpathlineto{\pgfqpoint{4.533504in}{1.847458in}}%
\pgfpathlineto{\pgfqpoint{4.597894in}{1.821330in}}%
\pgfpathlineto{\pgfqpoint{4.662284in}{1.795862in}}%
\pgfpathlineto{\pgfqpoint{4.726674in}{1.771037in}}%
\pgfpathlineto{\pgfqpoint{4.791064in}{1.746839in}}%
\pgfpathlineto{\pgfqpoint{4.855455in}{1.723252in}}%
\pgfpathlineto{\pgfqpoint{4.919845in}{1.700260in}}%
\pgfpathlineto{\pgfqpoint{4.984235in}{1.677850in}}%
\pgfpathlineto{\pgfqpoint{5.048625in}{1.656005in}}%
\pgfpathlineto{\pgfqpoint{5.113015in}{1.634712in}}%
\pgfpathlineto{\pgfqpoint{5.177405in}{1.613956in}}%
\pgfpathlineto{\pgfqpoint{5.241795in}{1.593725in}}%
\pgfpathlineto{\pgfqpoint{5.306185in}{1.574004in}}%
\pgfpathlineto{\pgfqpoint{5.370575in}{1.554781in}}%
\pgfpathlineto{\pgfqpoint{5.434965in}{1.536044in}}%
\pgfpathlineto{\pgfqpoint{5.499355in}{1.517780in}}%
\pgfpathlineto{\pgfqpoint{5.563745in}{1.499977in}}%
\pgfpathlineto{\pgfqpoint{5.628135in}{1.482624in}}%
\pgfpathlineto{\pgfqpoint{5.692525in}{1.465709in}}%
\pgfpathlineto{\pgfqpoint{5.756916in}{1.449221in}}%
\pgfpathlineto{\pgfqpoint{5.821306in}{1.433150in}}%
\pgfpathlineto{\pgfqpoint{5.885696in}{1.417484in}}%
\pgfpathlineto{\pgfqpoint{5.950086in}{1.402214in}}%
\pgfpathlineto{\pgfqpoint{6.014476in}{1.387329in}}%
\pgfpathlineto{\pgfqpoint{6.078866in}{1.372820in}}%
\pgfpathlineto{\pgfqpoint{6.143256in}{1.358678in}}%
\pgfpathlineto{\pgfqpoint{6.207646in}{1.344893in}}%
\pgfpathlineto{\pgfqpoint{6.272036in}{1.331456in}}%
\pgfpathlineto{\pgfqpoint{6.336426in}{1.318358in}}%
\pgfpathlineto{\pgfqpoint{6.400816in}{1.305591in}}%
\pgfpathlineto{\pgfqpoint{6.465206in}{1.293146in}}%
\pgfpathlineto{\pgfqpoint{6.529596in}{1.281016in}}%
\pgfpathlineto{\pgfqpoint{6.593987in}{1.269192in}}%
\pgfpathlineto{\pgfqpoint{6.658377in}{1.257667in}}%
\pgfpathlineto{\pgfqpoint{6.722767in}{1.246432in}}%
\pgfpathlineto{\pgfqpoint{6.787157in}{1.235481in}}%
\pgfpathlineto{\pgfqpoint{6.851547in}{1.224807in}}%
\pgfpathlineto{\pgfqpoint{6.915937in}{1.214402in}}%
\pgfpathlineto{\pgfqpoint{6.980327in}{1.204261in}}%
\pgfpathlineto{\pgfqpoint{7.044717in}{1.194375in}}%
\pgfpathlineto{\pgfqpoint{7.109107in}{1.184738in}}%
\pgfpathlineto{\pgfqpoint{7.173497in}{1.175346in}}%
\pgfpathlineto{\pgfqpoint{7.237887in}{1.166190in}}%
\pgfpathlineto{\pgfqpoint{7.302277in}{1.157265in}}%
\pgfpathlineto{\pgfqpoint{7.366667in}{1.148566in}}%
\pgfpathlineto{\pgfqpoint{7.431057in}{1.140087in}}%
\pgfpathlineto{\pgfqpoint{7.495448in}{1.131822in}}%
\pgfusepath{stroke}%
\end{pgfscope}%
\begin{pgfscope}%
\pgfsetrectcap%
\pgfsetmiterjoin%
\pgfsetlinewidth{0.000000pt}%
\definecolor{currentstroke}{rgb}{1.000000,1.000000,1.000000}%
\pgfsetstrokecolor{currentstroke}%
\pgfsetdash{}{0pt}%
\pgfpathmoveto{\pgfqpoint{0.802099in}{0.894194in}}%
\pgfpathlineto{\pgfqpoint{0.802099in}{5.013411in}}%
\pgfusepath{}%
\end{pgfscope}%
\begin{pgfscope}%
\pgfsetrectcap%
\pgfsetmiterjoin%
\pgfsetlinewidth{0.000000pt}%
\definecolor{currentstroke}{rgb}{1.000000,1.000000,1.000000}%
\pgfsetstrokecolor{currentstroke}%
\pgfsetdash{}{0pt}%
\pgfpathmoveto{\pgfqpoint{7.814178in}{0.894194in}}%
\pgfpathlineto{\pgfqpoint{7.814178in}{5.013411in}}%
\pgfusepath{}%
\end{pgfscope}%
\begin{pgfscope}%
\pgfsetrectcap%
\pgfsetmiterjoin%
\pgfsetlinewidth{0.000000pt}%
\definecolor{currentstroke}{rgb}{1.000000,1.000000,1.000000}%
\pgfsetstrokecolor{currentstroke}%
\pgfsetdash{}{0pt}%
\pgfpathmoveto{\pgfqpoint{0.802099in}{0.894194in}}%
\pgfpathlineto{\pgfqpoint{7.814178in}{0.894194in}}%
\pgfusepath{}%
\end{pgfscope}%
\begin{pgfscope}%
\pgfsetrectcap%
\pgfsetmiterjoin%
\pgfsetlinewidth{0.000000pt}%
\definecolor{currentstroke}{rgb}{1.000000,1.000000,1.000000}%
\pgfsetstrokecolor{currentstroke}%
\pgfsetdash{}{0pt}%
\pgfpathmoveto{\pgfqpoint{0.802099in}{5.013411in}}%
\pgfpathlineto{\pgfqpoint{7.814178in}{5.013411in}}%
\pgfusepath{}%
\end{pgfscope}%
\begin{pgfscope}%
\definecolor{textcolor}{rgb}{0.150000,0.150000,0.150000}%
\pgfsetstrokecolor{textcolor}%
\pgfsetfillcolor{textcolor}%
\pgftext[x=3.936286in, y=3.428823in, left, base]{\color{textcolor}\rmfamily\fontsize{20.000000}{24.000000}\selectfont M\(\displaystyle _z\)(t) = 9.67\(\displaystyle exp[\frac{-t}{T_2}]+\)0.1}%
\end{pgfscope}%
\begin{pgfscope}%
\definecolor{textcolor}{rgb}{0.150000,0.150000,0.150000}%
\pgfsetstrokecolor{textcolor}%
\pgfsetfillcolor{textcolor}%
\pgftext[x=3.936286in, y=2.934768in, left, base]{\color{textcolor}\rmfamily\fontsize{20.000000}{24.000000}\selectfont T\(\displaystyle _2\) = 18.95\(\displaystyle \pm\)0.53 ms}%
\end{pgfscope}%
\begin{pgfscope}%
\definecolor{textcolor}{rgb}{0.150000,0.150000,0.150000}%
\pgfsetstrokecolor{textcolor}%
\pgfsetfillcolor{textcolor}%
\pgftext[x=4.308139in,y=5.096745in,,base]{\color{textcolor}\rmfamily\fontsize{24.000000}{28.800000}\selectfont MG T\(\displaystyle _2\) measurement of 0.05 M CuSO\(\displaystyle _4\)}%
\end{pgfscope}%
\begin{pgfscope}%
\pgfsetbuttcap%
\pgfsetroundjoin%
\definecolor{currentfill}{rgb}{0.698039,0.133333,0.133333}%
\pgfsetfillcolor{currentfill}%
\pgfsetlinewidth{0.301125pt}%
\definecolor{currentstroke}{rgb}{0.698039,0.133333,0.133333}%
\pgfsetstrokecolor{currentstroke}%
\pgfsetdash{}{0pt}%
\pgfsys@defobject{currentmarker}{\pgfqpoint{-0.048611in}{-0.048611in}}{\pgfqpoint{0.048611in}{0.048611in}}{%
\pgfpathmoveto{\pgfqpoint{0.000000in}{-0.048611in}}%
\pgfpathcurveto{\pgfqpoint{0.012892in}{-0.048611in}}{\pgfqpoint{0.025257in}{-0.043489in}}{\pgfqpoint{0.034373in}{-0.034373in}}%
\pgfpathcurveto{\pgfqpoint{0.043489in}{-0.025257in}}{\pgfqpoint{0.048611in}{-0.012892in}}{\pgfqpoint{0.048611in}{0.000000in}}%
\pgfpathcurveto{\pgfqpoint{0.048611in}{0.012892in}}{\pgfqpoint{0.043489in}{0.025257in}}{\pgfqpoint{0.034373in}{0.034373in}}%
\pgfpathcurveto{\pgfqpoint{0.025257in}{0.043489in}}{\pgfqpoint{0.012892in}{0.048611in}}{\pgfqpoint{0.000000in}{0.048611in}}%
\pgfpathcurveto{\pgfqpoint{-0.012892in}{0.048611in}}{\pgfqpoint{-0.025257in}{0.043489in}}{\pgfqpoint{-0.034373in}{0.034373in}}%
\pgfpathcurveto{\pgfqpoint{-0.043489in}{0.025257in}}{\pgfqpoint{-0.048611in}{0.012892in}}{\pgfqpoint{-0.048611in}{0.000000in}}%
\pgfpathcurveto{\pgfqpoint{-0.048611in}{-0.012892in}}{\pgfqpoint{-0.043489in}{-0.025257in}}{\pgfqpoint{-0.034373in}{-0.034373in}}%
\pgfpathcurveto{\pgfqpoint{-0.025257in}{-0.043489in}}{\pgfqpoint{-0.012892in}{-0.048611in}}{\pgfqpoint{0.000000in}{-0.048611in}}%
\pgfpathlineto{\pgfqpoint{0.000000in}{-0.048611in}}%
\pgfpathclose%
\pgfusepath{stroke,fill}%
}%
\begin{pgfscope}%
\pgfsys@transformshift{5.283308in}{4.625282in}%
\pgfsys@useobject{currentmarker}{}%
\end{pgfscope}%
\end{pgfscope}%
\begin{pgfscope}%
\definecolor{textcolor}{rgb}{0.150000,0.150000,0.150000}%
\pgfsetstrokecolor{textcolor}%
\pgfsetfillcolor{textcolor}%
\pgftext[x=5.783308in,y=4.552365in,left,base]{\color{textcolor}\rmfamily\fontsize{20.000000}{24.000000}\selectfont Data}%
\end{pgfscope}%
\begin{pgfscope}%
\pgfsetroundcap%
\pgfsetroundjoin%
\pgfsetlinewidth{1.756562pt}%
\definecolor{currentstroke}{rgb}{0.298039,0.447059,0.690196}%
\pgfsetstrokecolor{currentstroke}%
\pgfsetdash{}{0pt}%
\pgfpathmoveto{\pgfqpoint{5.005530in}{4.241873in}}%
\pgfpathlineto{\pgfqpoint{5.283308in}{4.241873in}}%
\pgfpathlineto{\pgfqpoint{5.561086in}{4.241873in}}%
\pgfusepath{stroke}%
\end{pgfscope}%
\begin{pgfscope}%
\definecolor{textcolor}{rgb}{0.150000,0.150000,0.150000}%
\pgfsetstrokecolor{textcolor}%
\pgfsetfillcolor{textcolor}%
\pgftext[x=5.783308in,y=4.144651in,left,base]{\color{textcolor}\rmfamily\fontsize{20.000000}{24.000000}\selectfont Fitted Curve}%
\end{pgfscope}%
\end{pgfpicture}%
\makeatother%
\endgroup%
}
    \caption{T2 measurement of 0.05 M CuSO$_{4} \boldsymbol{\cdot} $H$_2$O using the Meiboom-Gill pulse sequence }
    \label{fig:0d05_t2}
  \end{figure}

  \begin{figure}[h]
    \resizebox{0.45\textwidth}{!}{%% Creator: Matplotlib, PGF backend
%%
%% To include the figure in your LaTeX document, write
%%   \input{<filename>.pgf}
%%
%% Make sure the required packages are loaded in your preamble
%%   \usepackage{pgf}
%%
%% Also ensure that all the required font packages are loaded; for instance,
%% the lmodern package is sometimes necessary when using math font.
%%   \usepackage{lmodern}
%%
%% Figures using additional raster images can only be included by \input if
%% they are in the same directory as the main LaTeX file. For loading figures
%% from other directories you can use the `import` package
%%   \usepackage{import}
%%
%% and then include the figures with
%%   \import{<path to file>}{<filename>.pgf}
%%
%% Matplotlib used the following preamble
%%   \usepackage{fontspec}
%%   \setmainfont{DejaVuSerif.ttf}[Path=\detokenize{/usr/share/matplotlib/mpl-data/fonts/ttf/}]
%%   \setsansfont{DejaVuSans.ttf}[Path=\detokenize{/usr/share/matplotlib/mpl-data/fonts/ttf/}]
%%   \setmonofont{DejaVuSansMono.ttf}[Path=\detokenize{/usr/share/matplotlib/mpl-data/fonts/ttf/}]
%%
\begingroup%
\makeatletter%
\begin{pgfpicture}%
\pgfpathrectangle{\pgfpointorigin}{\pgfqpoint{8.000000in}{5.500000in}}%
\pgfusepath{use as bounding box, clip}%
\begin{pgfscope}%
\pgfsetbuttcap%
\pgfsetmiterjoin%
\definecolor{currentfill}{rgb}{1.000000,1.000000,1.000000}%
\pgfsetfillcolor{currentfill}%
\pgfsetlinewidth{0.000000pt}%
\definecolor{currentstroke}{rgb}{1.000000,1.000000,1.000000}%
\pgfsetstrokecolor{currentstroke}%
\pgfsetdash{}{0pt}%
\pgfpathmoveto{\pgfqpoint{0.000000in}{0.000000in}}%
\pgfpathlineto{\pgfqpoint{8.000000in}{0.000000in}}%
\pgfpathlineto{\pgfqpoint{8.000000in}{5.500000in}}%
\pgfpathlineto{\pgfqpoint{0.000000in}{5.500000in}}%
\pgfpathlineto{\pgfqpoint{0.000000in}{0.000000in}}%
\pgfpathclose%
\pgfusepath{fill}%
\end{pgfscope}%
\begin{pgfscope}%
\pgfsetbuttcap%
\pgfsetmiterjoin%
\definecolor{currentfill}{rgb}{0.917647,0.917647,0.949020}%
\pgfsetfillcolor{currentfill}%
\pgfsetlinewidth{0.000000pt}%
\definecolor{currentstroke}{rgb}{0.000000,0.000000,0.000000}%
\pgfsetstrokecolor{currentstroke}%
\pgfsetstrokeopacity{0.000000}%
\pgfsetdash{}{0pt}%
\pgfpathmoveto{\pgfqpoint{0.802099in}{0.894194in}}%
\pgfpathlineto{\pgfqpoint{7.810889in}{0.894194in}}%
\pgfpathlineto{\pgfqpoint{7.810889in}{5.013411in}}%
\pgfpathlineto{\pgfqpoint{0.802099in}{5.013411in}}%
\pgfpathlineto{\pgfqpoint{0.802099in}{0.894194in}}%
\pgfpathclose%
\pgfusepath{fill}%
\end{pgfscope}%
\begin{pgfscope}%
\pgfpathrectangle{\pgfqpoint{0.802099in}{0.894194in}}{\pgfqpoint{7.008789in}{4.119217in}}%
\pgfusepath{clip}%
\pgfsetroundcap%
\pgfsetroundjoin%
\pgfsetlinewidth{1.003750pt}%
\definecolor{currentstroke}{rgb}{1.000000,1.000000,1.000000}%
\pgfsetstrokecolor{currentstroke}%
\pgfsetdash{}{0pt}%
\pgfpathmoveto{\pgfqpoint{0.875618in}{0.894194in}}%
\pgfpathlineto{\pgfqpoint{0.875618in}{5.013411in}}%
\pgfusepath{stroke}%
\end{pgfscope}%
\begin{pgfscope}%
\definecolor{textcolor}{rgb}{0.150000,0.150000,0.150000}%
\pgfsetstrokecolor{textcolor}%
\pgfsetfillcolor{textcolor}%
\pgftext[x=0.875618in,y=0.796972in,,top]{\color{textcolor}\rmfamily\fontsize{20.000000}{24.000000}\selectfont 0}%
\end{pgfscope}%
\begin{pgfscope}%
\pgfpathrectangle{\pgfqpoint{0.802099in}{0.894194in}}{\pgfqpoint{7.008789in}{4.119217in}}%
\pgfusepath{clip}%
\pgfsetroundcap%
\pgfsetroundjoin%
\pgfsetlinewidth{1.003750pt}%
\definecolor{currentstroke}{rgb}{1.000000,1.000000,1.000000}%
\pgfsetstrokecolor{currentstroke}%
\pgfsetdash{}{0pt}%
\pgfpathmoveto{\pgfqpoint{2.407259in}{0.894194in}}%
\pgfpathlineto{\pgfqpoint{2.407259in}{5.013411in}}%
\pgfusepath{stroke}%
\end{pgfscope}%
\begin{pgfscope}%
\definecolor{textcolor}{rgb}{0.150000,0.150000,0.150000}%
\pgfsetstrokecolor{textcolor}%
\pgfsetfillcolor{textcolor}%
\pgftext[x=2.407259in,y=0.796972in,,top]{\color{textcolor}\rmfamily\fontsize{20.000000}{24.000000}\selectfont 5}%
\end{pgfscope}%
\begin{pgfscope}%
\pgfpathrectangle{\pgfqpoint{0.802099in}{0.894194in}}{\pgfqpoint{7.008789in}{4.119217in}}%
\pgfusepath{clip}%
\pgfsetroundcap%
\pgfsetroundjoin%
\pgfsetlinewidth{1.003750pt}%
\definecolor{currentstroke}{rgb}{1.000000,1.000000,1.000000}%
\pgfsetstrokecolor{currentstroke}%
\pgfsetdash{}{0pt}%
\pgfpathmoveto{\pgfqpoint{3.938900in}{0.894194in}}%
\pgfpathlineto{\pgfqpoint{3.938900in}{5.013411in}}%
\pgfusepath{stroke}%
\end{pgfscope}%
\begin{pgfscope}%
\definecolor{textcolor}{rgb}{0.150000,0.150000,0.150000}%
\pgfsetstrokecolor{textcolor}%
\pgfsetfillcolor{textcolor}%
\pgftext[x=3.938900in,y=0.796972in,,top]{\color{textcolor}\rmfamily\fontsize{20.000000}{24.000000}\selectfont 10}%
\end{pgfscope}%
\begin{pgfscope}%
\pgfpathrectangle{\pgfqpoint{0.802099in}{0.894194in}}{\pgfqpoint{7.008789in}{4.119217in}}%
\pgfusepath{clip}%
\pgfsetroundcap%
\pgfsetroundjoin%
\pgfsetlinewidth{1.003750pt}%
\definecolor{currentstroke}{rgb}{1.000000,1.000000,1.000000}%
\pgfsetstrokecolor{currentstroke}%
\pgfsetdash{}{0pt}%
\pgfpathmoveto{\pgfqpoint{5.470541in}{0.894194in}}%
\pgfpathlineto{\pgfqpoint{5.470541in}{5.013411in}}%
\pgfusepath{stroke}%
\end{pgfscope}%
\begin{pgfscope}%
\definecolor{textcolor}{rgb}{0.150000,0.150000,0.150000}%
\pgfsetstrokecolor{textcolor}%
\pgfsetfillcolor{textcolor}%
\pgftext[x=5.470541in,y=0.796972in,,top]{\color{textcolor}\rmfamily\fontsize{20.000000}{24.000000}\selectfont 15}%
\end{pgfscope}%
\begin{pgfscope}%
\pgfpathrectangle{\pgfqpoint{0.802099in}{0.894194in}}{\pgfqpoint{7.008789in}{4.119217in}}%
\pgfusepath{clip}%
\pgfsetroundcap%
\pgfsetroundjoin%
\pgfsetlinewidth{1.003750pt}%
\definecolor{currentstroke}{rgb}{1.000000,1.000000,1.000000}%
\pgfsetstrokecolor{currentstroke}%
\pgfsetdash{}{0pt}%
\pgfpathmoveto{\pgfqpoint{7.002182in}{0.894194in}}%
\pgfpathlineto{\pgfqpoint{7.002182in}{5.013411in}}%
\pgfusepath{stroke}%
\end{pgfscope}%
\begin{pgfscope}%
\definecolor{textcolor}{rgb}{0.150000,0.150000,0.150000}%
\pgfsetstrokecolor{textcolor}%
\pgfsetfillcolor{textcolor}%
\pgftext[x=7.002182in,y=0.796972in,,top]{\color{textcolor}\rmfamily\fontsize{20.000000}{24.000000}\selectfont 20}%
\end{pgfscope}%
\begin{pgfscope}%
\definecolor{textcolor}{rgb}{0.150000,0.150000,0.150000}%
\pgfsetstrokecolor{textcolor}%
\pgfsetfillcolor{textcolor}%
\pgftext[x=4.306494in,y=0.472591in,,top]{\color{textcolor}\rmfamily\fontsize{24.000000}{28.800000}\selectfont Time (ms)}%
\end{pgfscope}%
\begin{pgfscope}%
\pgfpathrectangle{\pgfqpoint{0.802099in}{0.894194in}}{\pgfqpoint{7.008789in}{4.119217in}}%
\pgfusepath{clip}%
\pgfsetroundcap%
\pgfsetroundjoin%
\pgfsetlinewidth{1.003750pt}%
\definecolor{currentstroke}{rgb}{1.000000,1.000000,1.000000}%
\pgfsetstrokecolor{currentstroke}%
\pgfsetdash{}{0pt}%
\pgfpathmoveto{\pgfqpoint{0.802099in}{1.666548in}}%
\pgfpathlineto{\pgfqpoint{7.810889in}{1.666548in}}%
\pgfusepath{stroke}%
\end{pgfscope}%
\begin{pgfscope}%
\definecolor{textcolor}{rgb}{0.150000,0.150000,0.150000}%
\pgfsetstrokecolor{textcolor}%
\pgfsetfillcolor{textcolor}%
\pgftext[x=0.528147in, y=1.561025in, left, base]{\color{textcolor}\rmfamily\fontsize{20.000000}{24.000000}\selectfont 2}%
\end{pgfscope}%
\begin{pgfscope}%
\pgfpathrectangle{\pgfqpoint{0.802099in}{0.894194in}}{\pgfqpoint{7.008789in}{4.119217in}}%
\pgfusepath{clip}%
\pgfsetroundcap%
\pgfsetroundjoin%
\pgfsetlinewidth{1.003750pt}%
\definecolor{currentstroke}{rgb}{1.000000,1.000000,1.000000}%
\pgfsetstrokecolor{currentstroke}%
\pgfsetdash{}{0pt}%
\pgfpathmoveto{\pgfqpoint{0.802099in}{2.641741in}}%
\pgfpathlineto{\pgfqpoint{7.810889in}{2.641741in}}%
\pgfusepath{stroke}%
\end{pgfscope}%
\begin{pgfscope}%
\definecolor{textcolor}{rgb}{0.150000,0.150000,0.150000}%
\pgfsetstrokecolor{textcolor}%
\pgfsetfillcolor{textcolor}%
\pgftext[x=0.528147in, y=2.536218in, left, base]{\color{textcolor}\rmfamily\fontsize{20.000000}{24.000000}\selectfont 4}%
\end{pgfscope}%
\begin{pgfscope}%
\pgfpathrectangle{\pgfqpoint{0.802099in}{0.894194in}}{\pgfqpoint{7.008789in}{4.119217in}}%
\pgfusepath{clip}%
\pgfsetroundcap%
\pgfsetroundjoin%
\pgfsetlinewidth{1.003750pt}%
\definecolor{currentstroke}{rgb}{1.000000,1.000000,1.000000}%
\pgfsetstrokecolor{currentstroke}%
\pgfsetdash{}{0pt}%
\pgfpathmoveto{\pgfqpoint{0.802099in}{3.616934in}}%
\pgfpathlineto{\pgfqpoint{7.810889in}{3.616934in}}%
\pgfusepath{stroke}%
\end{pgfscope}%
\begin{pgfscope}%
\definecolor{textcolor}{rgb}{0.150000,0.150000,0.150000}%
\pgfsetstrokecolor{textcolor}%
\pgfsetfillcolor{textcolor}%
\pgftext[x=0.528147in, y=3.511411in, left, base]{\color{textcolor}\rmfamily\fontsize{20.000000}{24.000000}\selectfont 6}%
\end{pgfscope}%
\begin{pgfscope}%
\pgfpathrectangle{\pgfqpoint{0.802099in}{0.894194in}}{\pgfqpoint{7.008789in}{4.119217in}}%
\pgfusepath{clip}%
\pgfsetroundcap%
\pgfsetroundjoin%
\pgfsetlinewidth{1.003750pt}%
\definecolor{currentstroke}{rgb}{1.000000,1.000000,1.000000}%
\pgfsetstrokecolor{currentstroke}%
\pgfsetdash{}{0pt}%
\pgfpathmoveto{\pgfqpoint{0.802099in}{4.592128in}}%
\pgfpathlineto{\pgfqpoint{7.810889in}{4.592128in}}%
\pgfusepath{stroke}%
\end{pgfscope}%
\begin{pgfscope}%
\definecolor{textcolor}{rgb}{0.150000,0.150000,0.150000}%
\pgfsetstrokecolor{textcolor}%
\pgfsetfillcolor{textcolor}%
\pgftext[x=0.528147in, y=4.486605in, left, base]{\color{textcolor}\rmfamily\fontsize{20.000000}{24.000000}\selectfont 8}%
\end{pgfscope}%
\begin{pgfscope}%
\definecolor{textcolor}{rgb}{0.150000,0.150000,0.150000}%
\pgfsetstrokecolor{textcolor}%
\pgfsetfillcolor{textcolor}%
\pgftext[x=0.472591in,y=2.953803in,,bottom,rotate=90.000000]{\color{textcolor}\rmfamily\fontsize{24.000000}{28.800000}\selectfont Pulse Amplitude (V)}%
\end{pgfscope}%
\begin{pgfscope}%
\pgfpathrectangle{\pgfqpoint{0.802099in}{0.894194in}}{\pgfqpoint{7.008789in}{4.119217in}}%
\pgfusepath{clip}%
\pgfsetbuttcap%
\pgfsetroundjoin%
\definecolor{currentfill}{rgb}{0.698039,0.133333,0.133333}%
\pgfsetfillcolor{currentfill}%
\pgfsetlinewidth{0.301125pt}%
\definecolor{currentstroke}{rgb}{0.698039,0.133333,0.133333}%
\pgfsetstrokecolor{currentstroke}%
\pgfsetdash{}{0pt}%
\pgfsys@defobject{currentmarker}{\pgfqpoint{-0.048611in}{-0.048611in}}{\pgfqpoint{0.048611in}{0.048611in}}{%
\pgfpathmoveto{\pgfqpoint{0.000000in}{-0.048611in}}%
\pgfpathcurveto{\pgfqpoint{0.012892in}{-0.048611in}}{\pgfqpoint{0.025257in}{-0.043489in}}{\pgfqpoint{0.034373in}{-0.034373in}}%
\pgfpathcurveto{\pgfqpoint{0.043489in}{-0.025257in}}{\pgfqpoint{0.048611in}{-0.012892in}}{\pgfqpoint{0.048611in}{0.000000in}}%
\pgfpathcurveto{\pgfqpoint{0.048611in}{0.012892in}}{\pgfqpoint{0.043489in}{0.025257in}}{\pgfqpoint{0.034373in}{0.034373in}}%
\pgfpathcurveto{\pgfqpoint{0.025257in}{0.043489in}}{\pgfqpoint{0.012892in}{0.048611in}}{\pgfqpoint{0.000000in}{0.048611in}}%
\pgfpathcurveto{\pgfqpoint{-0.012892in}{0.048611in}}{\pgfqpoint{-0.025257in}{0.043489in}}{\pgfqpoint{-0.034373in}{0.034373in}}%
\pgfpathcurveto{\pgfqpoint{-0.043489in}{0.025257in}}{\pgfqpoint{-0.048611in}{0.012892in}}{\pgfqpoint{-0.048611in}{0.000000in}}%
\pgfpathcurveto{\pgfqpoint{-0.048611in}{-0.012892in}}{\pgfqpoint{-0.043489in}{-0.025257in}}{\pgfqpoint{-0.034373in}{-0.034373in}}%
\pgfpathcurveto{\pgfqpoint{-0.025257in}{-0.043489in}}{\pgfqpoint{-0.012892in}{-0.048611in}}{\pgfqpoint{0.000000in}{-0.048611in}}%
\pgfpathlineto{\pgfqpoint{0.000000in}{-0.048611in}}%
\pgfpathclose%
\pgfusepath{stroke,fill}%
}%
\begin{pgfscope}%
\pgfsys@transformshift{1.120681in}{4.826174in}%
\pgfsys@useobject{currentmarker}{}%
\end{pgfscope}%
\begin{pgfscope}%
\pgfsys@transformshift{1.610806in}{3.928996in}%
\pgfsys@useobject{currentmarker}{}%
\end{pgfscope}%
\begin{pgfscope}%
\pgfsys@transformshift{2.100931in}{3.382888in}%
\pgfsys@useobject{currentmarker}{}%
\end{pgfscope}%
\begin{pgfscope}%
\pgfsys@transformshift{2.591056in}{2.953803in}%
\pgfsys@useobject{currentmarker}{}%
\end{pgfscope}%
\begin{pgfscope}%
\pgfsys@transformshift{3.081181in}{2.641741in}%
\pgfsys@useobject{currentmarker}{}%
\end{pgfscope}%
\begin{pgfscope}%
\pgfsys@transformshift{3.571306in}{2.324803in}%
\pgfsys@useobject{currentmarker}{}%
\end{pgfscope}%
\begin{pgfscope}%
\pgfsys@transformshift{4.061432in}{2.095633in}%
\pgfsys@useobject{currentmarker}{}%
\end{pgfscope}%
\begin{pgfscope}%
\pgfsys@transformshift{4.551557in}{1.861586in}%
\pgfsys@useobject{currentmarker}{}%
\end{pgfscope}%
\begin{pgfscope}%
\pgfsys@transformshift{5.041682in}{1.705555in}%
\pgfsys@useobject{currentmarker}{}%
\end{pgfscope}%
\begin{pgfscope}%
\pgfsys@transformshift{5.531807in}{1.549524in}%
\pgfsys@useobject{currentmarker}{}%
\end{pgfscope}%
\begin{pgfscope}%
\pgfsys@transformshift{6.021932in}{1.432501in}%
\pgfsys@useobject{currentmarker}{}%
\end{pgfscope}%
\begin{pgfscope}%
\pgfsys@transformshift{6.512057in}{1.315478in}%
\pgfsys@useobject{currentmarker}{}%
\end{pgfscope}%
\begin{pgfscope}%
\pgfsys@transformshift{7.002182in}{1.237463in}%
\pgfsys@useobject{currentmarker}{}%
\end{pgfscope}%
\begin{pgfscope}%
\pgfsys@transformshift{7.492308in}{1.081432in}%
\pgfsys@useobject{currentmarker}{}%
\end{pgfscope}%
\end{pgfscope}%
\begin{pgfscope}%
\pgfpathrectangle{\pgfqpoint{0.802099in}{0.894194in}}{\pgfqpoint{7.008789in}{4.119217in}}%
\pgfusepath{clip}%
\pgfsetroundcap%
\pgfsetroundjoin%
\pgfsetlinewidth{1.756562pt}%
\definecolor{currentstroke}{rgb}{0.298039,0.447059,0.690196}%
\pgfsetstrokecolor{currentstroke}%
\pgfsetdash{}{0pt}%
\pgfpathmoveto{\pgfqpoint{1.120681in}{4.719021in}}%
\pgfpathlineto{\pgfqpoint{1.185041in}{4.620128in}}%
\pgfpathlineto{\pgfqpoint{1.249401in}{4.523787in}}%
\pgfpathlineto{\pgfqpoint{1.313760in}{4.429933in}}%
\pgfpathlineto{\pgfqpoint{1.378120in}{4.338501in}}%
\pgfpathlineto{\pgfqpoint{1.442480in}{4.249429in}}%
\pgfpathlineto{\pgfqpoint{1.506840in}{4.162655in}}%
\pgfpathlineto{\pgfqpoint{1.571200in}{4.078121in}}%
\pgfpathlineto{\pgfqpoint{1.635560in}{3.995769in}}%
\pgfpathlineto{\pgfqpoint{1.699920in}{3.915542in}}%
\pgfpathlineto{\pgfqpoint{1.764279in}{3.837385in}}%
\pgfpathlineto{\pgfqpoint{1.828639in}{3.761246in}}%
\pgfpathlineto{\pgfqpoint{1.892999in}{3.687072in}}%
\pgfpathlineto{\pgfqpoint{1.957359in}{3.614812in}}%
\pgfpathlineto{\pgfqpoint{2.021719in}{3.544417in}}%
\pgfpathlineto{\pgfqpoint{2.086079in}{3.475839in}}%
\pgfpathlineto{\pgfqpoint{2.150439in}{3.409031in}}%
\pgfpathlineto{\pgfqpoint{2.214799in}{3.343947in}}%
\pgfpathlineto{\pgfqpoint{2.279158in}{3.280543in}}%
\pgfpathlineto{\pgfqpoint{2.343518in}{3.218775in}}%
\pgfpathlineto{\pgfqpoint{2.407878in}{3.158601in}}%
\pgfpathlineto{\pgfqpoint{2.472238in}{3.099980in}}%
\pgfpathlineto{\pgfqpoint{2.536598in}{3.042873in}}%
\pgfpathlineto{\pgfqpoint{2.600958in}{2.987239in}}%
\pgfpathlineto{\pgfqpoint{2.665318in}{2.933041in}}%
\pgfpathlineto{\pgfqpoint{2.729677in}{2.880241in}}%
\pgfpathlineto{\pgfqpoint{2.794037in}{2.828805in}}%
\pgfpathlineto{\pgfqpoint{2.858397in}{2.778696in}}%
\pgfpathlineto{\pgfqpoint{2.922757in}{2.729880in}}%
\pgfpathlineto{\pgfqpoint{2.987117in}{2.682324in}}%
\pgfpathlineto{\pgfqpoint{3.051477in}{2.635995in}}%
\pgfpathlineto{\pgfqpoint{3.115837in}{2.590862in}}%
\pgfpathlineto{\pgfqpoint{3.180196in}{2.546894in}}%
\pgfpathlineto{\pgfqpoint{3.244556in}{2.504061in}}%
\pgfpathlineto{\pgfqpoint{3.308916in}{2.462333in}}%
\pgfpathlineto{\pgfqpoint{3.373276in}{2.421682in}}%
\pgfpathlineto{\pgfqpoint{3.437636in}{2.382080in}}%
\pgfpathlineto{\pgfqpoint{3.501996in}{2.343501in}}%
\pgfpathlineto{\pgfqpoint{3.566356in}{2.305917in}}%
\pgfpathlineto{\pgfqpoint{3.630716in}{2.269303in}}%
\pgfpathlineto{\pgfqpoint{3.695075in}{2.233634in}}%
\pgfpathlineto{\pgfqpoint{3.759435in}{2.198885in}}%
\pgfpathlineto{\pgfqpoint{3.823795in}{2.165034in}}%
\pgfpathlineto{\pgfqpoint{3.888155in}{2.132056in}}%
\pgfpathlineto{\pgfqpoint{3.952515in}{2.099929in}}%
\pgfpathlineto{\pgfqpoint{4.016875in}{2.068631in}}%
\pgfpathlineto{\pgfqpoint{4.081235in}{2.038141in}}%
\pgfpathlineto{\pgfqpoint{4.145594in}{2.008438in}}%
\pgfpathlineto{\pgfqpoint{4.209954in}{1.979501in}}%
\pgfpathlineto{\pgfqpoint{4.274314in}{1.951312in}}%
\pgfpathlineto{\pgfqpoint{4.338674in}{1.923850in}}%
\pgfpathlineto{\pgfqpoint{4.403034in}{1.897096in}}%
\pgfpathlineto{\pgfqpoint{4.467394in}{1.871033in}}%
\pgfpathlineto{\pgfqpoint{4.531754in}{1.845643in}}%
\pgfpathlineto{\pgfqpoint{4.596114in}{1.820908in}}%
\pgfpathlineto{\pgfqpoint{4.660473in}{1.796812in}}%
\pgfpathlineto{\pgfqpoint{4.724833in}{1.773337in}}%
\pgfpathlineto{\pgfqpoint{4.789193in}{1.750468in}}%
\pgfpathlineto{\pgfqpoint{4.853553in}{1.728190in}}%
\pgfpathlineto{\pgfqpoint{4.917913in}{1.706486in}}%
\pgfpathlineto{\pgfqpoint{4.982273in}{1.685342in}}%
\pgfpathlineto{\pgfqpoint{5.046633in}{1.664745in}}%
\pgfpathlineto{\pgfqpoint{5.110992in}{1.644678in}}%
\pgfpathlineto{\pgfqpoint{5.175352in}{1.625130in}}%
\pgfpathlineto{\pgfqpoint{5.239712in}{1.606086in}}%
\pgfpathlineto{\pgfqpoint{5.304072in}{1.587534in}}%
\pgfpathlineto{\pgfqpoint{5.368432in}{1.569461in}}%
\pgfpathlineto{\pgfqpoint{5.432792in}{1.551853in}}%
\pgfpathlineto{\pgfqpoint{5.497152in}{1.534701in}}%
\pgfpathlineto{\pgfqpoint{5.561512in}{1.517991in}}%
\pgfpathlineto{\pgfqpoint{5.625871in}{1.501712in}}%
\pgfpathlineto{\pgfqpoint{5.690231in}{1.485854in}}%
\pgfpathlineto{\pgfqpoint{5.754591in}{1.470404in}}%
\pgfpathlineto{\pgfqpoint{5.818951in}{1.455354in}}%
\pgfpathlineto{\pgfqpoint{5.883311in}{1.440692in}}%
\pgfpathlineto{\pgfqpoint{5.947671in}{1.426408in}}%
\pgfpathlineto{\pgfqpoint{6.012031in}{1.412493in}}%
\pgfpathlineto{\pgfqpoint{6.076390in}{1.398937in}}%
\pgfpathlineto{\pgfqpoint{6.140750in}{1.385731in}}%
\pgfpathlineto{\pgfqpoint{6.205110in}{1.372866in}}%
\pgfpathlineto{\pgfqpoint{6.269470in}{1.360333in}}%
\pgfpathlineto{\pgfqpoint{6.333830in}{1.348123in}}%
\pgfpathlineto{\pgfqpoint{6.398190in}{1.336228in}}%
\pgfpathlineto{\pgfqpoint{6.462550in}{1.324641in}}%
\pgfpathlineto{\pgfqpoint{6.526910in}{1.313352in}}%
\pgfpathlineto{\pgfqpoint{6.591269in}{1.302355in}}%
\pgfpathlineto{\pgfqpoint{6.655629in}{1.291642in}}%
\pgfpathlineto{\pgfqpoint{6.719989in}{1.281205in}}%
\pgfpathlineto{\pgfqpoint{6.784349in}{1.271037in}}%
\pgfpathlineto{\pgfqpoint{6.848709in}{1.261132in}}%
\pgfpathlineto{\pgfqpoint{6.913069in}{1.251483in}}%
\pgfpathlineto{\pgfqpoint{6.977429in}{1.242082in}}%
\pgfpathlineto{\pgfqpoint{7.041788in}{1.232924in}}%
\pgfpathlineto{\pgfqpoint{7.106148in}{1.224003in}}%
\pgfpathlineto{\pgfqpoint{7.170508in}{1.215312in}}%
\pgfpathlineto{\pgfqpoint{7.234868in}{1.206845in}}%
\pgfpathlineto{\pgfqpoint{7.299228in}{1.198596in}}%
\pgfpathlineto{\pgfqpoint{7.363588in}{1.190561in}}%
\pgfpathlineto{\pgfqpoint{7.427948in}{1.182733in}}%
\pgfpathlineto{\pgfqpoint{7.492308in}{1.175107in}}%
\pgfusepath{stroke}%
\end{pgfscope}%
\begin{pgfscope}%
\pgfsetrectcap%
\pgfsetmiterjoin%
\pgfsetlinewidth{0.000000pt}%
\definecolor{currentstroke}{rgb}{1.000000,1.000000,1.000000}%
\pgfsetstrokecolor{currentstroke}%
\pgfsetdash{}{0pt}%
\pgfpathmoveto{\pgfqpoint{0.802099in}{0.894194in}}%
\pgfpathlineto{\pgfqpoint{0.802099in}{5.013411in}}%
\pgfusepath{}%
\end{pgfscope}%
\begin{pgfscope}%
\pgfsetrectcap%
\pgfsetmiterjoin%
\pgfsetlinewidth{0.000000pt}%
\definecolor{currentstroke}{rgb}{1.000000,1.000000,1.000000}%
\pgfsetstrokecolor{currentstroke}%
\pgfsetdash{}{0pt}%
\pgfpathmoveto{\pgfqpoint{7.810889in}{0.894194in}}%
\pgfpathlineto{\pgfqpoint{7.810889in}{5.013411in}}%
\pgfusepath{}%
\end{pgfscope}%
\begin{pgfscope}%
\pgfsetrectcap%
\pgfsetmiterjoin%
\pgfsetlinewidth{0.000000pt}%
\definecolor{currentstroke}{rgb}{1.000000,1.000000,1.000000}%
\pgfsetstrokecolor{currentstroke}%
\pgfsetdash{}{0pt}%
\pgfpathmoveto{\pgfqpoint{0.802099in}{0.894194in}}%
\pgfpathlineto{\pgfqpoint{7.810889in}{0.894194in}}%
\pgfusepath{}%
\end{pgfscope}%
\begin{pgfscope}%
\pgfsetrectcap%
\pgfsetmiterjoin%
\pgfsetlinewidth{0.000000pt}%
\definecolor{currentstroke}{rgb}{1.000000,1.000000,1.000000}%
\pgfsetstrokecolor{currentstroke}%
\pgfsetdash{}{0pt}%
\pgfpathmoveto{\pgfqpoint{0.802099in}{5.013411in}}%
\pgfpathlineto{\pgfqpoint{7.810889in}{5.013411in}}%
\pgfusepath{}%
\end{pgfscope}%
\begin{pgfscope}%
\definecolor{textcolor}{rgb}{0.150000,0.150000,0.150000}%
\pgfsetstrokecolor{textcolor}%
\pgfsetfillcolor{textcolor}%
\pgftext[x=3.938900in, y=3.623393in, left, base]{\color{textcolor}\rmfamily\fontsize{20.000000}{24.000000}\selectfont M\(\displaystyle _z\)(t) = 8.68\(\displaystyle exp[\frac{-2t}{T_2}]+\)0.4}%
\end{pgfscope}%
\begin{pgfscope}%
\definecolor{textcolor}{rgb}{0.150000,0.150000,0.150000}%
\pgfsetstrokecolor{textcolor}%
\pgfsetfillcolor{textcolor}%
\pgftext[x=3.938900in, y=3.129338in, left, base]{\color{textcolor}\rmfamily\fontsize{20.000000}{24.000000}\selectfont T\(\displaystyle _2\) = 16.07\(\displaystyle \pm\)0.84 ms}%
\end{pgfscope}%
\begin{pgfscope}%
\definecolor{textcolor}{rgb}{0.150000,0.150000,0.150000}%
\pgfsetstrokecolor{textcolor}%
\pgfsetfillcolor{textcolor}%
\pgftext[x=4.306494in,y=5.096745in,,base]{\color{textcolor}\rmfamily\fontsize{24.000000}{28.800000}\selectfont MG T\(\displaystyle _2\) measurement of 0.1 M CuSO\(\displaystyle _4\)}%
\end{pgfscope}%
\begin{pgfscope}%
\pgfsetbuttcap%
\pgfsetroundjoin%
\definecolor{currentfill}{rgb}{0.698039,0.133333,0.133333}%
\pgfsetfillcolor{currentfill}%
\pgfsetlinewidth{0.301125pt}%
\definecolor{currentstroke}{rgb}{0.698039,0.133333,0.133333}%
\pgfsetstrokecolor{currentstroke}%
\pgfsetdash{}{0pt}%
\pgfsys@defobject{currentmarker}{\pgfqpoint{-0.048611in}{-0.048611in}}{\pgfqpoint{0.048611in}{0.048611in}}{%
\pgfpathmoveto{\pgfqpoint{0.000000in}{-0.048611in}}%
\pgfpathcurveto{\pgfqpoint{0.012892in}{-0.048611in}}{\pgfqpoint{0.025257in}{-0.043489in}}{\pgfqpoint{0.034373in}{-0.034373in}}%
\pgfpathcurveto{\pgfqpoint{0.043489in}{-0.025257in}}{\pgfqpoint{0.048611in}{-0.012892in}}{\pgfqpoint{0.048611in}{0.000000in}}%
\pgfpathcurveto{\pgfqpoint{0.048611in}{0.012892in}}{\pgfqpoint{0.043489in}{0.025257in}}{\pgfqpoint{0.034373in}{0.034373in}}%
\pgfpathcurveto{\pgfqpoint{0.025257in}{0.043489in}}{\pgfqpoint{0.012892in}{0.048611in}}{\pgfqpoint{0.000000in}{0.048611in}}%
\pgfpathcurveto{\pgfqpoint{-0.012892in}{0.048611in}}{\pgfqpoint{-0.025257in}{0.043489in}}{\pgfqpoint{-0.034373in}{0.034373in}}%
\pgfpathcurveto{\pgfqpoint{-0.043489in}{0.025257in}}{\pgfqpoint{-0.048611in}{0.012892in}}{\pgfqpoint{-0.048611in}{0.000000in}}%
\pgfpathcurveto{\pgfqpoint{-0.048611in}{-0.012892in}}{\pgfqpoint{-0.043489in}{-0.025257in}}{\pgfqpoint{-0.034373in}{-0.034373in}}%
\pgfpathcurveto{\pgfqpoint{-0.025257in}{-0.043489in}}{\pgfqpoint{-0.012892in}{-0.048611in}}{\pgfqpoint{0.000000in}{-0.048611in}}%
\pgfpathlineto{\pgfqpoint{0.000000in}{-0.048611in}}%
\pgfpathclose%
\pgfusepath{stroke,fill}%
}%
\begin{pgfscope}%
\pgfsys@transformshift{5.280019in}{4.625282in}%
\pgfsys@useobject{currentmarker}{}%
\end{pgfscope}%
\end{pgfscope}%
\begin{pgfscope}%
\definecolor{textcolor}{rgb}{0.150000,0.150000,0.150000}%
\pgfsetstrokecolor{textcolor}%
\pgfsetfillcolor{textcolor}%
\pgftext[x=5.780019in,y=4.552365in,left,base]{\color{textcolor}\rmfamily\fontsize{20.000000}{24.000000}\selectfont Data}%
\end{pgfscope}%
\begin{pgfscope}%
\pgfsetroundcap%
\pgfsetroundjoin%
\pgfsetlinewidth{1.756562pt}%
\definecolor{currentstroke}{rgb}{0.298039,0.447059,0.690196}%
\pgfsetstrokecolor{currentstroke}%
\pgfsetdash{}{0pt}%
\pgfpathmoveto{\pgfqpoint{5.002241in}{4.241873in}}%
\pgfpathlineto{\pgfqpoint{5.280019in}{4.241873in}}%
\pgfpathlineto{\pgfqpoint{5.557796in}{4.241873in}}%
\pgfusepath{stroke}%
\end{pgfscope}%
\begin{pgfscope}%
\definecolor{textcolor}{rgb}{0.150000,0.150000,0.150000}%
\pgfsetstrokecolor{textcolor}%
\pgfsetfillcolor{textcolor}%
\pgftext[x=5.780019in,y=4.144651in,left,base]{\color{textcolor}\rmfamily\fontsize{20.000000}{24.000000}\selectfont Fitted Curve}%
\end{pgfscope}%
\end{pgfpicture}%
\makeatother%
\endgroup%
}
    \caption{T2 measurement of 0.1 M CuSO$_{4} \boldsymbol{\cdot} $H$_2$O using the Meiboom-Gill pulse sequence }
    \label{fig:0d1_t2}
  \end{figure}

  \begin{figure}[h]
    \resizebox{0.45\textwidth}{!}{%% Creator: Matplotlib, PGF backend
%%
%% To include the figure in your LaTeX document, write
%%   \input{<filename>.pgf}
%%
%% Make sure the required packages are loaded in your preamble
%%   \usepackage{pgf}
%%
%% Also ensure that all the required font packages are loaded; for instance,
%% the lmodern package is sometimes necessary when using math font.
%%   \usepackage{lmodern}
%%
%% Figures using additional raster images can only be included by \input if
%% they are in the same directory as the main LaTeX file. For loading figures
%% from other directories you can use the `import` package
%%   \usepackage{import}
%%
%% and then include the figures with
%%   \import{<path to file>}{<filename>.pgf}
%%
%% Matplotlib used the following preamble
%%   \usepackage{fontspec}
%%   \setmainfont{DejaVuSerif.ttf}[Path=\detokenize{/usr/share/matplotlib/mpl-data/fonts/ttf/}]
%%   \setsansfont{DejaVuSans.ttf}[Path=\detokenize{/usr/share/matplotlib/mpl-data/fonts/ttf/}]
%%   \setmonofont{DejaVuSansMono.ttf}[Path=\detokenize{/usr/share/matplotlib/mpl-data/fonts/ttf/}]
%%
\begingroup%
\makeatletter%
\begin{pgfpicture}%
\pgfpathrectangle{\pgfpointorigin}{\pgfqpoint{8.000000in}{5.500000in}}%
\pgfusepath{use as bounding box, clip}%
\begin{pgfscope}%
\pgfsetbuttcap%
\pgfsetmiterjoin%
\definecolor{currentfill}{rgb}{1.000000,1.000000,1.000000}%
\pgfsetfillcolor{currentfill}%
\pgfsetlinewidth{0.000000pt}%
\definecolor{currentstroke}{rgb}{1.000000,1.000000,1.000000}%
\pgfsetstrokecolor{currentstroke}%
\pgfsetdash{}{0pt}%
\pgfpathmoveto{\pgfqpoint{0.000000in}{0.000000in}}%
\pgfpathlineto{\pgfqpoint{8.000000in}{0.000000in}}%
\pgfpathlineto{\pgfqpoint{8.000000in}{5.500000in}}%
\pgfpathlineto{\pgfqpoint{0.000000in}{5.500000in}}%
\pgfpathlineto{\pgfqpoint{0.000000in}{0.000000in}}%
\pgfpathclose%
\pgfusepath{fill}%
\end{pgfscope}%
\begin{pgfscope}%
\pgfsetbuttcap%
\pgfsetmiterjoin%
\definecolor{currentfill}{rgb}{0.917647,0.917647,0.949020}%
\pgfsetfillcolor{currentfill}%
\pgfsetlinewidth{0.000000pt}%
\definecolor{currentstroke}{rgb}{0.000000,0.000000,0.000000}%
\pgfsetstrokecolor{currentstroke}%
\pgfsetstrokeopacity{0.000000}%
\pgfsetdash{}{0pt}%
\pgfpathmoveto{\pgfqpoint{0.802099in}{0.894194in}}%
\pgfpathlineto{\pgfqpoint{7.737742in}{0.894194in}}%
\pgfpathlineto{\pgfqpoint{7.737742in}{5.013411in}}%
\pgfpathlineto{\pgfqpoint{0.802099in}{5.013411in}}%
\pgfpathlineto{\pgfqpoint{0.802099in}{0.894194in}}%
\pgfpathclose%
\pgfusepath{fill}%
\end{pgfscope}%
\begin{pgfscope}%
\pgfpathrectangle{\pgfqpoint{0.802099in}{0.894194in}}{\pgfqpoint{6.935642in}{4.119217in}}%
\pgfusepath{clip}%
\pgfsetroundcap%
\pgfsetroundjoin%
\pgfsetlinewidth{1.003750pt}%
\definecolor{currentstroke}{rgb}{1.000000,1.000000,1.000000}%
\pgfsetstrokecolor{currentstroke}%
\pgfsetdash{}{0pt}%
\pgfpathmoveto{\pgfqpoint{0.896123in}{0.894194in}}%
\pgfpathlineto{\pgfqpoint{0.896123in}{5.013411in}}%
\pgfusepath{stroke}%
\end{pgfscope}%
\begin{pgfscope}%
\definecolor{textcolor}{rgb}{0.150000,0.150000,0.150000}%
\pgfsetstrokecolor{textcolor}%
\pgfsetfillcolor{textcolor}%
\pgftext[x=0.896123in,y=0.796972in,,top]{\color{textcolor}\rmfamily\fontsize{20.000000}{24.000000}\selectfont 0}%
\end{pgfscope}%
\begin{pgfscope}%
\pgfpathrectangle{\pgfqpoint{0.802099in}{0.894194in}}{\pgfqpoint{6.935642in}{4.119217in}}%
\pgfusepath{clip}%
\pgfsetroundcap%
\pgfsetroundjoin%
\pgfsetlinewidth{1.003750pt}%
\definecolor{currentstroke}{rgb}{1.000000,1.000000,1.000000}%
\pgfsetstrokecolor{currentstroke}%
\pgfsetdash{}{0pt}%
\pgfpathmoveto{\pgfqpoint{2.002286in}{0.894194in}}%
\pgfpathlineto{\pgfqpoint{2.002286in}{5.013411in}}%
\pgfusepath{stroke}%
\end{pgfscope}%
\begin{pgfscope}%
\definecolor{textcolor}{rgb}{0.150000,0.150000,0.150000}%
\pgfsetstrokecolor{textcolor}%
\pgfsetfillcolor{textcolor}%
\pgftext[x=2.002286in,y=0.796972in,,top]{\color{textcolor}\rmfamily\fontsize{20.000000}{24.000000}\selectfont 1}%
\end{pgfscope}%
\begin{pgfscope}%
\pgfpathrectangle{\pgfqpoint{0.802099in}{0.894194in}}{\pgfqpoint{6.935642in}{4.119217in}}%
\pgfusepath{clip}%
\pgfsetroundcap%
\pgfsetroundjoin%
\pgfsetlinewidth{1.003750pt}%
\definecolor{currentstroke}{rgb}{1.000000,1.000000,1.000000}%
\pgfsetstrokecolor{currentstroke}%
\pgfsetdash{}{0pt}%
\pgfpathmoveto{\pgfqpoint{3.108449in}{0.894194in}}%
\pgfpathlineto{\pgfqpoint{3.108449in}{5.013411in}}%
\pgfusepath{stroke}%
\end{pgfscope}%
\begin{pgfscope}%
\definecolor{textcolor}{rgb}{0.150000,0.150000,0.150000}%
\pgfsetstrokecolor{textcolor}%
\pgfsetfillcolor{textcolor}%
\pgftext[x=3.108449in,y=0.796972in,,top]{\color{textcolor}\rmfamily\fontsize{20.000000}{24.000000}\selectfont 2}%
\end{pgfscope}%
\begin{pgfscope}%
\pgfpathrectangle{\pgfqpoint{0.802099in}{0.894194in}}{\pgfqpoint{6.935642in}{4.119217in}}%
\pgfusepath{clip}%
\pgfsetroundcap%
\pgfsetroundjoin%
\pgfsetlinewidth{1.003750pt}%
\definecolor{currentstroke}{rgb}{1.000000,1.000000,1.000000}%
\pgfsetstrokecolor{currentstroke}%
\pgfsetdash{}{0pt}%
\pgfpathmoveto{\pgfqpoint{4.214612in}{0.894194in}}%
\pgfpathlineto{\pgfqpoint{4.214612in}{5.013411in}}%
\pgfusepath{stroke}%
\end{pgfscope}%
\begin{pgfscope}%
\definecolor{textcolor}{rgb}{0.150000,0.150000,0.150000}%
\pgfsetstrokecolor{textcolor}%
\pgfsetfillcolor{textcolor}%
\pgftext[x=4.214612in,y=0.796972in,,top]{\color{textcolor}\rmfamily\fontsize{20.000000}{24.000000}\selectfont 3}%
\end{pgfscope}%
\begin{pgfscope}%
\pgfpathrectangle{\pgfqpoint{0.802099in}{0.894194in}}{\pgfqpoint{6.935642in}{4.119217in}}%
\pgfusepath{clip}%
\pgfsetroundcap%
\pgfsetroundjoin%
\pgfsetlinewidth{1.003750pt}%
\definecolor{currentstroke}{rgb}{1.000000,1.000000,1.000000}%
\pgfsetstrokecolor{currentstroke}%
\pgfsetdash{}{0pt}%
\pgfpathmoveto{\pgfqpoint{5.320775in}{0.894194in}}%
\pgfpathlineto{\pgfqpoint{5.320775in}{5.013411in}}%
\pgfusepath{stroke}%
\end{pgfscope}%
\begin{pgfscope}%
\definecolor{textcolor}{rgb}{0.150000,0.150000,0.150000}%
\pgfsetstrokecolor{textcolor}%
\pgfsetfillcolor{textcolor}%
\pgftext[x=5.320775in,y=0.796972in,,top]{\color{textcolor}\rmfamily\fontsize{20.000000}{24.000000}\selectfont 4}%
\end{pgfscope}%
\begin{pgfscope}%
\pgfpathrectangle{\pgfqpoint{0.802099in}{0.894194in}}{\pgfqpoint{6.935642in}{4.119217in}}%
\pgfusepath{clip}%
\pgfsetroundcap%
\pgfsetroundjoin%
\pgfsetlinewidth{1.003750pt}%
\definecolor{currentstroke}{rgb}{1.000000,1.000000,1.000000}%
\pgfsetstrokecolor{currentstroke}%
\pgfsetdash{}{0pt}%
\pgfpathmoveto{\pgfqpoint{6.426938in}{0.894194in}}%
\pgfpathlineto{\pgfqpoint{6.426938in}{5.013411in}}%
\pgfusepath{stroke}%
\end{pgfscope}%
\begin{pgfscope}%
\definecolor{textcolor}{rgb}{0.150000,0.150000,0.150000}%
\pgfsetstrokecolor{textcolor}%
\pgfsetfillcolor{textcolor}%
\pgftext[x=6.426938in,y=0.796972in,,top]{\color{textcolor}\rmfamily\fontsize{20.000000}{24.000000}\selectfont 5}%
\end{pgfscope}%
\begin{pgfscope}%
\pgfpathrectangle{\pgfqpoint{0.802099in}{0.894194in}}{\pgfqpoint{6.935642in}{4.119217in}}%
\pgfusepath{clip}%
\pgfsetroundcap%
\pgfsetroundjoin%
\pgfsetlinewidth{1.003750pt}%
\definecolor{currentstroke}{rgb}{1.000000,1.000000,1.000000}%
\pgfsetstrokecolor{currentstroke}%
\pgfsetdash{}{0pt}%
\pgfpathmoveto{\pgfqpoint{7.533101in}{0.894194in}}%
\pgfpathlineto{\pgfqpoint{7.533101in}{5.013411in}}%
\pgfusepath{stroke}%
\end{pgfscope}%
\begin{pgfscope}%
\definecolor{textcolor}{rgb}{0.150000,0.150000,0.150000}%
\pgfsetstrokecolor{textcolor}%
\pgfsetfillcolor{textcolor}%
\pgftext[x=7.533101in,y=0.796972in,,top]{\color{textcolor}\rmfamily\fontsize{20.000000}{24.000000}\selectfont 6}%
\end{pgfscope}%
\begin{pgfscope}%
\definecolor{textcolor}{rgb}{0.150000,0.150000,0.150000}%
\pgfsetstrokecolor{textcolor}%
\pgfsetfillcolor{textcolor}%
\pgftext[x=4.269920in,y=0.472591in,,top]{\color{textcolor}\rmfamily\fontsize{24.000000}{28.800000}\selectfont Time (ms)}%
\end{pgfscope}%
\begin{pgfscope}%
\pgfpathrectangle{\pgfqpoint{0.802099in}{0.894194in}}{\pgfqpoint{6.935642in}{4.119217in}}%
\pgfusepath{clip}%
\pgfsetroundcap%
\pgfsetroundjoin%
\pgfsetlinewidth{1.003750pt}%
\definecolor{currentstroke}{rgb}{1.000000,1.000000,1.000000}%
\pgfsetstrokecolor{currentstroke}%
\pgfsetdash{}{0pt}%
\pgfpathmoveto{\pgfqpoint{0.802099in}{1.492277in}}%
\pgfpathlineto{\pgfqpoint{7.737742in}{1.492277in}}%
\pgfusepath{stroke}%
\end{pgfscope}%
\begin{pgfscope}%
\definecolor{textcolor}{rgb}{0.150000,0.150000,0.150000}%
\pgfsetstrokecolor{textcolor}%
\pgfsetfillcolor{textcolor}%
\pgftext[x=0.528147in, y=1.386754in, left, base]{\color{textcolor}\rmfamily\fontsize{20.000000}{24.000000}\selectfont 3}%
\end{pgfscope}%
\begin{pgfscope}%
\pgfpathrectangle{\pgfqpoint{0.802099in}{0.894194in}}{\pgfqpoint{6.935642in}{4.119217in}}%
\pgfusepath{clip}%
\pgfsetroundcap%
\pgfsetroundjoin%
\pgfsetlinewidth{1.003750pt}%
\definecolor{currentstroke}{rgb}{1.000000,1.000000,1.000000}%
\pgfsetstrokecolor{currentstroke}%
\pgfsetdash{}{0pt}%
\pgfpathmoveto{\pgfqpoint{0.802099in}{2.282363in}}%
\pgfpathlineto{\pgfqpoint{7.737742in}{2.282363in}}%
\pgfusepath{stroke}%
\end{pgfscope}%
\begin{pgfscope}%
\definecolor{textcolor}{rgb}{0.150000,0.150000,0.150000}%
\pgfsetstrokecolor{textcolor}%
\pgfsetfillcolor{textcolor}%
\pgftext[x=0.528147in, y=2.176840in, left, base]{\color{textcolor}\rmfamily\fontsize{20.000000}{24.000000}\selectfont 4}%
\end{pgfscope}%
\begin{pgfscope}%
\pgfpathrectangle{\pgfqpoint{0.802099in}{0.894194in}}{\pgfqpoint{6.935642in}{4.119217in}}%
\pgfusepath{clip}%
\pgfsetroundcap%
\pgfsetroundjoin%
\pgfsetlinewidth{1.003750pt}%
\definecolor{currentstroke}{rgb}{1.000000,1.000000,1.000000}%
\pgfsetstrokecolor{currentstroke}%
\pgfsetdash{}{0pt}%
\pgfpathmoveto{\pgfqpoint{0.802099in}{3.072450in}}%
\pgfpathlineto{\pgfqpoint{7.737742in}{3.072450in}}%
\pgfusepath{stroke}%
\end{pgfscope}%
\begin{pgfscope}%
\definecolor{textcolor}{rgb}{0.150000,0.150000,0.150000}%
\pgfsetstrokecolor{textcolor}%
\pgfsetfillcolor{textcolor}%
\pgftext[x=0.528147in, y=2.966927in, left, base]{\color{textcolor}\rmfamily\fontsize{20.000000}{24.000000}\selectfont 5}%
\end{pgfscope}%
\begin{pgfscope}%
\pgfpathrectangle{\pgfqpoint{0.802099in}{0.894194in}}{\pgfqpoint{6.935642in}{4.119217in}}%
\pgfusepath{clip}%
\pgfsetroundcap%
\pgfsetroundjoin%
\pgfsetlinewidth{1.003750pt}%
\definecolor{currentstroke}{rgb}{1.000000,1.000000,1.000000}%
\pgfsetstrokecolor{currentstroke}%
\pgfsetdash{}{0pt}%
\pgfpathmoveto{\pgfqpoint{0.802099in}{3.862537in}}%
\pgfpathlineto{\pgfqpoint{7.737742in}{3.862537in}}%
\pgfusepath{stroke}%
\end{pgfscope}%
\begin{pgfscope}%
\definecolor{textcolor}{rgb}{0.150000,0.150000,0.150000}%
\pgfsetstrokecolor{textcolor}%
\pgfsetfillcolor{textcolor}%
\pgftext[x=0.528147in, y=3.757014in, left, base]{\color{textcolor}\rmfamily\fontsize{20.000000}{24.000000}\selectfont 6}%
\end{pgfscope}%
\begin{pgfscope}%
\pgfpathrectangle{\pgfqpoint{0.802099in}{0.894194in}}{\pgfqpoint{6.935642in}{4.119217in}}%
\pgfusepath{clip}%
\pgfsetroundcap%
\pgfsetroundjoin%
\pgfsetlinewidth{1.003750pt}%
\definecolor{currentstroke}{rgb}{1.000000,1.000000,1.000000}%
\pgfsetstrokecolor{currentstroke}%
\pgfsetdash{}{0pt}%
\pgfpathmoveto{\pgfqpoint{0.802099in}{4.652624in}}%
\pgfpathlineto{\pgfqpoint{7.737742in}{4.652624in}}%
\pgfusepath{stroke}%
\end{pgfscope}%
\begin{pgfscope}%
\definecolor{textcolor}{rgb}{0.150000,0.150000,0.150000}%
\pgfsetstrokecolor{textcolor}%
\pgfsetfillcolor{textcolor}%
\pgftext[x=0.528147in, y=4.547101in, left, base]{\color{textcolor}\rmfamily\fontsize{20.000000}{24.000000}\selectfont 7}%
\end{pgfscope}%
\begin{pgfscope}%
\definecolor{textcolor}{rgb}{0.150000,0.150000,0.150000}%
\pgfsetstrokecolor{textcolor}%
\pgfsetfillcolor{textcolor}%
\pgftext[x=0.472591in,y=2.953803in,,bottom,rotate=90.000000]{\color{textcolor}\rmfamily\fontsize{24.000000}{28.800000}\selectfont Pulse Amplitude (V)}%
\end{pgfscope}%
\begin{pgfscope}%
\pgfpathrectangle{\pgfqpoint{0.802099in}{0.894194in}}{\pgfqpoint{6.935642in}{4.119217in}}%
\pgfusepath{clip}%
\pgfsetbuttcap%
\pgfsetroundjoin%
\definecolor{currentfill}{rgb}{0.698039,0.133333,0.133333}%
\pgfsetfillcolor{currentfill}%
\pgfsetlinewidth{0.301125pt}%
\definecolor{currentstroke}{rgb}{0.698039,0.133333,0.133333}%
\pgfsetstrokecolor{currentstroke}%
\pgfsetdash{}{0pt}%
\pgfsys@defobject{currentmarker}{\pgfqpoint{-0.048611in}{-0.048611in}}{\pgfqpoint{0.048611in}{0.048611in}}{%
\pgfpathmoveto{\pgfqpoint{0.000000in}{-0.048611in}}%
\pgfpathcurveto{\pgfqpoint{0.012892in}{-0.048611in}}{\pgfqpoint{0.025257in}{-0.043489in}}{\pgfqpoint{0.034373in}{-0.034373in}}%
\pgfpathcurveto{\pgfqpoint{0.043489in}{-0.025257in}}{\pgfqpoint{0.048611in}{-0.012892in}}{\pgfqpoint{0.048611in}{0.000000in}}%
\pgfpathcurveto{\pgfqpoint{0.048611in}{0.012892in}}{\pgfqpoint{0.043489in}{0.025257in}}{\pgfqpoint{0.034373in}{0.034373in}}%
\pgfpathcurveto{\pgfqpoint{0.025257in}{0.043489in}}{\pgfqpoint{0.012892in}{0.048611in}}{\pgfqpoint{0.000000in}{0.048611in}}%
\pgfpathcurveto{\pgfqpoint{-0.012892in}{0.048611in}}{\pgfqpoint{-0.025257in}{0.043489in}}{\pgfqpoint{-0.034373in}{0.034373in}}%
\pgfpathcurveto{\pgfqpoint{-0.043489in}{0.025257in}}{\pgfqpoint{-0.048611in}{0.012892in}}{\pgfqpoint{-0.048611in}{0.000000in}}%
\pgfpathcurveto{\pgfqpoint{-0.048611in}{-0.012892in}}{\pgfqpoint{-0.043489in}{-0.025257in}}{\pgfqpoint{-0.034373in}{-0.034373in}}%
\pgfpathcurveto{\pgfqpoint{-0.025257in}{-0.043489in}}{\pgfqpoint{-0.012892in}{-0.048611in}}{\pgfqpoint{0.000000in}{-0.048611in}}%
\pgfpathlineto{\pgfqpoint{0.000000in}{-0.048611in}}%
\pgfpathclose%
\pgfusepath{stroke,fill}%
}%
\begin{pgfscope}%
\pgfsys@transformshift{1.117356in}{4.810641in}%
\pgfsys@useobject{currentmarker}{}%
\end{pgfscope}%
\begin{pgfscope}%
\pgfsys@transformshift{1.670437in}{4.368192in}%
\pgfsys@useobject{currentmarker}{}%
\end{pgfscope}%
\begin{pgfscope}%
\pgfsys@transformshift{2.112903in}{3.925744in}%
\pgfsys@useobject{currentmarker}{}%
\end{pgfscope}%
\begin{pgfscope}%
\pgfsys@transformshift{2.555368in}{3.483295in}%
\pgfsys@useobject{currentmarker}{}%
\end{pgfscope}%
\begin{pgfscope}%
\pgfsys@transformshift{3.108449in}{3.104054in}%
\pgfsys@useobject{currentmarker}{}%
\end{pgfscope}%
\begin{pgfscope}%
\pgfsys@transformshift{3.661531in}{2.851226in}%
\pgfsys@useobject{currentmarker}{}%
\end{pgfscope}%
\begin{pgfscope}%
\pgfsys@transformshift{3.993380in}{2.598398in}%
\pgfsys@useobject{currentmarker}{}%
\end{pgfscope}%
\begin{pgfscope}%
\pgfsys@transformshift{4.546461in}{2.282363in}%
\pgfsys@useobject{currentmarker}{}%
\end{pgfscope}%
\begin{pgfscope}%
\pgfsys@transformshift{5.099543in}{2.092743in}%
\pgfsys@useobject{currentmarker}{}%
\end{pgfscope}%
\begin{pgfscope}%
\pgfsys@transformshift{5.542008in}{1.776708in}%
\pgfsys@useobject{currentmarker}{}%
\end{pgfscope}%
\begin{pgfscope}%
\pgfsys@transformshift{5.984473in}{1.587087in}%
\pgfsys@useobject{currentmarker}{}%
\end{pgfscope}%
\begin{pgfscope}%
\pgfsys@transformshift{6.537555in}{1.397466in}%
\pgfsys@useobject{currentmarker}{}%
\end{pgfscope}%
\begin{pgfscope}%
\pgfsys@transformshift{6.980020in}{1.271052in}%
\pgfsys@useobject{currentmarker}{}%
\end{pgfscope}%
\begin{pgfscope}%
\pgfsys@transformshift{7.422485in}{1.081432in}%
\pgfsys@useobject{currentmarker}{}%
\end{pgfscope}%
\end{pgfscope}%
\begin{pgfscope}%
\pgfpathrectangle{\pgfqpoint{0.802099in}{0.894194in}}{\pgfqpoint{6.935642in}{4.119217in}}%
\pgfusepath{clip}%
\pgfsetroundcap%
\pgfsetroundjoin%
\pgfsetlinewidth{1.756562pt}%
\definecolor{currentstroke}{rgb}{0.298039,0.447059,0.690196}%
\pgfsetstrokecolor{currentstroke}%
\pgfsetdash{}{0pt}%
\pgfpathmoveto{\pgfqpoint{1.117356in}{4.826174in}}%
\pgfpathlineto{\pgfqpoint{1.181044in}{4.762896in}}%
\pgfpathlineto{\pgfqpoint{1.244732in}{4.700358in}}%
\pgfpathlineto{\pgfqpoint{1.308420in}{4.638550in}}%
\pgfpathlineto{\pgfqpoint{1.372109in}{4.577465in}}%
\pgfpathlineto{\pgfqpoint{1.435797in}{4.517093in}}%
\pgfpathlineto{\pgfqpoint{1.499485in}{4.457428in}}%
\pgfpathlineto{\pgfqpoint{1.563173in}{4.398460in}}%
\pgfpathlineto{\pgfqpoint{1.626861in}{4.340181in}}%
\pgfpathlineto{\pgfqpoint{1.690549in}{4.282583in}}%
\pgfpathlineto{\pgfqpoint{1.754238in}{4.225658in}}%
\pgfpathlineto{\pgfqpoint{1.817926in}{4.169399in}}%
\pgfpathlineto{\pgfqpoint{1.881614in}{4.113797in}}%
\pgfpathlineto{\pgfqpoint{1.945302in}{4.058845in}}%
\pgfpathlineto{\pgfqpoint{2.008990in}{4.004535in}}%
\pgfpathlineto{\pgfqpoint{2.072679in}{3.950860in}}%
\pgfpathlineto{\pgfqpoint{2.136367in}{3.897813in}}%
\pgfpathlineto{\pgfqpoint{2.200055in}{3.845385in}}%
\pgfpathlineto{\pgfqpoint{2.263743in}{3.793571in}}%
\pgfpathlineto{\pgfqpoint{2.327431in}{3.742361in}}%
\pgfpathlineto{\pgfqpoint{2.391119in}{3.691751in}}%
\pgfpathlineto{\pgfqpoint{2.454808in}{3.641732in}}%
\pgfpathlineto{\pgfqpoint{2.518496in}{3.592297in}}%
\pgfpathlineto{\pgfqpoint{2.582184in}{3.543441in}}%
\pgfpathlineto{\pgfqpoint{2.645872in}{3.495155in}}%
\pgfpathlineto{\pgfqpoint{2.709560in}{3.447434in}}%
\pgfpathlineto{\pgfqpoint{2.773248in}{3.400270in}}%
\pgfpathlineto{\pgfqpoint{2.836937in}{3.353658in}}%
\pgfpathlineto{\pgfqpoint{2.900625in}{3.307590in}}%
\pgfpathlineto{\pgfqpoint{2.964313in}{3.262061in}}%
\pgfpathlineto{\pgfqpoint{3.028001in}{3.217065in}}%
\pgfpathlineto{\pgfqpoint{3.091689in}{3.172594in}}%
\pgfpathlineto{\pgfqpoint{3.155377in}{3.128642in}}%
\pgfpathlineto{\pgfqpoint{3.219066in}{3.085205in}}%
\pgfpathlineto{\pgfqpoint{3.282754in}{3.042275in}}%
\pgfpathlineto{\pgfqpoint{3.346442in}{2.999847in}}%
\pgfpathlineto{\pgfqpoint{3.410130in}{2.957915in}}%
\pgfpathlineto{\pgfqpoint{3.473818in}{2.916473in}}%
\pgfpathlineto{\pgfqpoint{3.537507in}{2.875515in}}%
\pgfpathlineto{\pgfqpoint{3.601195in}{2.835036in}}%
\pgfpathlineto{\pgfqpoint{3.664883in}{2.795030in}}%
\pgfpathlineto{\pgfqpoint{3.728571in}{2.755492in}}%
\pgfpathlineto{\pgfqpoint{3.792259in}{2.716416in}}%
\pgfpathlineto{\pgfqpoint{3.855947in}{2.677796in}}%
\pgfpathlineto{\pgfqpoint{3.919636in}{2.639628in}}%
\pgfpathlineto{\pgfqpoint{3.983324in}{2.601906in}}%
\pgfpathlineto{\pgfqpoint{4.047012in}{2.564625in}}%
\pgfpathlineto{\pgfqpoint{4.110700in}{2.527780in}}%
\pgfpathlineto{\pgfqpoint{4.174388in}{2.491365in}}%
\pgfpathlineto{\pgfqpoint{4.238076in}{2.455376in}}%
\pgfpathlineto{\pgfqpoint{4.301765in}{2.419808in}}%
\pgfpathlineto{\pgfqpoint{4.365453in}{2.384655in}}%
\pgfpathlineto{\pgfqpoint{4.429141in}{2.349913in}}%
\pgfpathlineto{\pgfqpoint{4.492829in}{2.315578in}}%
\pgfpathlineto{\pgfqpoint{4.556517in}{2.281643in}}%
\pgfpathlineto{\pgfqpoint{4.620205in}{2.248105in}}%
\pgfpathlineto{\pgfqpoint{4.683894in}{2.214960in}}%
\pgfpathlineto{\pgfqpoint{4.747582in}{2.182201in}}%
\pgfpathlineto{\pgfqpoint{4.811270in}{2.149826in}}%
\pgfpathlineto{\pgfqpoint{4.874958in}{2.117828in}}%
\pgfpathlineto{\pgfqpoint{4.938646in}{2.086205in}}%
\pgfpathlineto{\pgfqpoint{5.002334in}{2.054952in}}%
\pgfpathlineto{\pgfqpoint{5.066023in}{2.024063in}}%
\pgfpathlineto{\pgfqpoint{5.129711in}{1.993536in}}%
\pgfpathlineto{\pgfqpoint{5.193399in}{1.963366in}}%
\pgfpathlineto{\pgfqpoint{5.257087in}{1.933548in}}%
\pgfpathlineto{\pgfqpoint{5.320775in}{1.904079in}}%
\pgfpathlineto{\pgfqpoint{5.384464in}{1.874954in}}%
\pgfpathlineto{\pgfqpoint{5.448152in}{1.846169in}}%
\pgfpathlineto{\pgfqpoint{5.511840in}{1.817721in}}%
\pgfpathlineto{\pgfqpoint{5.575528in}{1.789606in}}%
\pgfpathlineto{\pgfqpoint{5.639216in}{1.761819in}}%
\pgfpathlineto{\pgfqpoint{5.702904in}{1.734357in}}%
\pgfpathlineto{\pgfqpoint{5.766593in}{1.707215in}}%
\pgfpathlineto{\pgfqpoint{5.830281in}{1.680391in}}%
\pgfpathlineto{\pgfqpoint{5.893969in}{1.653881in}}%
\pgfpathlineto{\pgfqpoint{5.957657in}{1.627680in}}%
\pgfpathlineto{\pgfqpoint{6.021345in}{1.601786in}}%
\pgfpathlineto{\pgfqpoint{6.085033in}{1.576194in}}%
\pgfpathlineto{\pgfqpoint{6.148722in}{1.550902in}}%
\pgfpathlineto{\pgfqpoint{6.212410in}{1.525905in}}%
\pgfpathlineto{\pgfqpoint{6.276098in}{1.501200in}}%
\pgfpathlineto{\pgfqpoint{6.339786in}{1.476784in}}%
\pgfpathlineto{\pgfqpoint{6.403474in}{1.452653in}}%
\pgfpathlineto{\pgfqpoint{6.467162in}{1.428805in}}%
\pgfpathlineto{\pgfqpoint{6.530851in}{1.405235in}}%
\pgfpathlineto{\pgfqpoint{6.594539in}{1.381940in}}%
\pgfpathlineto{\pgfqpoint{6.658227in}{1.358918in}}%
\pgfpathlineto{\pgfqpoint{6.721915in}{1.336165in}}%
\pgfpathlineto{\pgfqpoint{6.785603in}{1.313678in}}%
\pgfpathlineto{\pgfqpoint{6.849292in}{1.291454in}}%
\pgfpathlineto{\pgfqpoint{6.912980in}{1.269489in}}%
\pgfpathlineto{\pgfqpoint{6.976668in}{1.247781in}}%
\pgfpathlineto{\pgfqpoint{7.040356in}{1.226327in}}%
\pgfpathlineto{\pgfqpoint{7.104044in}{1.205124in}}%
\pgfpathlineto{\pgfqpoint{7.167732in}{1.184168in}}%
\pgfpathlineto{\pgfqpoint{7.231421in}{1.163457in}}%
\pgfpathlineto{\pgfqpoint{7.295109in}{1.142989in}}%
\pgfpathlineto{\pgfqpoint{7.358797in}{1.122760in}}%
\pgfpathlineto{\pgfqpoint{7.422485in}{1.102767in}}%
\pgfusepath{stroke}%
\end{pgfscope}%
\begin{pgfscope}%
\pgfsetrectcap%
\pgfsetmiterjoin%
\pgfsetlinewidth{0.000000pt}%
\definecolor{currentstroke}{rgb}{1.000000,1.000000,1.000000}%
\pgfsetstrokecolor{currentstroke}%
\pgfsetdash{}{0pt}%
\pgfpathmoveto{\pgfqpoint{0.802099in}{0.894194in}}%
\pgfpathlineto{\pgfqpoint{0.802099in}{5.013411in}}%
\pgfusepath{}%
\end{pgfscope}%
\begin{pgfscope}%
\pgfsetrectcap%
\pgfsetmiterjoin%
\pgfsetlinewidth{0.000000pt}%
\definecolor{currentstroke}{rgb}{1.000000,1.000000,1.000000}%
\pgfsetstrokecolor{currentstroke}%
\pgfsetdash{}{0pt}%
\pgfpathmoveto{\pgfqpoint{7.737742in}{0.894194in}}%
\pgfpathlineto{\pgfqpoint{7.737742in}{5.013411in}}%
\pgfusepath{}%
\end{pgfscope}%
\begin{pgfscope}%
\pgfsetrectcap%
\pgfsetmiterjoin%
\pgfsetlinewidth{0.000000pt}%
\definecolor{currentstroke}{rgb}{1.000000,1.000000,1.000000}%
\pgfsetstrokecolor{currentstroke}%
\pgfsetdash{}{0pt}%
\pgfpathmoveto{\pgfqpoint{0.802099in}{0.894194in}}%
\pgfpathlineto{\pgfqpoint{7.737742in}{0.894194in}}%
\pgfusepath{}%
\end{pgfscope}%
\begin{pgfscope}%
\pgfsetrectcap%
\pgfsetmiterjoin%
\pgfsetlinewidth{0.000000pt}%
\definecolor{currentstroke}{rgb}{1.000000,1.000000,1.000000}%
\pgfsetstrokecolor{currentstroke}%
\pgfsetdash{}{0pt}%
\pgfpathmoveto{\pgfqpoint{0.802099in}{5.013411in}}%
\pgfpathlineto{\pgfqpoint{7.737742in}{5.013411in}}%
\pgfusepath{}%
\end{pgfscope}%
\begin{pgfscope}%
\definecolor{textcolor}{rgb}{0.150000,0.150000,0.150000}%
\pgfsetstrokecolor{textcolor}%
\pgfsetfillcolor{textcolor}%
\pgftext[x=3.938072in, y=3.566506in, left, base]{\color{textcolor}\rmfamily\fontsize{20.000000}{24.000000}\selectfont M\(\displaystyle _z\)(t) = 7.14\(\displaystyle exp[\frac{-t}{T_2}]+\)0.37}%
\end{pgfscope}%
\begin{pgfscope}%
\definecolor{textcolor}{rgb}{0.150000,0.150000,0.150000}%
\pgfsetstrokecolor{textcolor}%
\pgfsetfillcolor{textcolor}%
\pgftext[x=3.938072in, y=3.072450in, left, base]{\color{textcolor}\rmfamily\fontsize{20.000000}{24.000000}\selectfont T\(\displaystyle _2\) = 4.897\(\displaystyle \pm\)0.36 ms}%
\end{pgfscope}%
\begin{pgfscope}%
\definecolor{textcolor}{rgb}{0.150000,0.150000,0.150000}%
\pgfsetstrokecolor{textcolor}%
\pgfsetfillcolor{textcolor}%
\pgftext[x=4.269920in,y=5.096745in,,base]{\color{textcolor}\rmfamily\fontsize{24.000000}{28.800000}\selectfont MG T\(\displaystyle _2\) measurement of 0.2 M CuSO\(\displaystyle _4\)}%
\end{pgfscope}%
\begin{pgfscope}%
\pgfsetbuttcap%
\pgfsetroundjoin%
\definecolor{currentfill}{rgb}{0.698039,0.133333,0.133333}%
\pgfsetfillcolor{currentfill}%
\pgfsetlinewidth{0.301125pt}%
\definecolor{currentstroke}{rgb}{0.698039,0.133333,0.133333}%
\pgfsetstrokecolor{currentstroke}%
\pgfsetdash{}{0pt}%
\pgfsys@defobject{currentmarker}{\pgfqpoint{-0.048611in}{-0.048611in}}{\pgfqpoint{0.048611in}{0.048611in}}{%
\pgfpathmoveto{\pgfqpoint{0.000000in}{-0.048611in}}%
\pgfpathcurveto{\pgfqpoint{0.012892in}{-0.048611in}}{\pgfqpoint{0.025257in}{-0.043489in}}{\pgfqpoint{0.034373in}{-0.034373in}}%
\pgfpathcurveto{\pgfqpoint{0.043489in}{-0.025257in}}{\pgfqpoint{0.048611in}{-0.012892in}}{\pgfqpoint{0.048611in}{0.000000in}}%
\pgfpathcurveto{\pgfqpoint{0.048611in}{0.012892in}}{\pgfqpoint{0.043489in}{0.025257in}}{\pgfqpoint{0.034373in}{0.034373in}}%
\pgfpathcurveto{\pgfqpoint{0.025257in}{0.043489in}}{\pgfqpoint{0.012892in}{0.048611in}}{\pgfqpoint{0.000000in}{0.048611in}}%
\pgfpathcurveto{\pgfqpoint{-0.012892in}{0.048611in}}{\pgfqpoint{-0.025257in}{0.043489in}}{\pgfqpoint{-0.034373in}{0.034373in}}%
\pgfpathcurveto{\pgfqpoint{-0.043489in}{0.025257in}}{\pgfqpoint{-0.048611in}{0.012892in}}{\pgfqpoint{-0.048611in}{0.000000in}}%
\pgfpathcurveto{\pgfqpoint{-0.048611in}{-0.012892in}}{\pgfqpoint{-0.043489in}{-0.025257in}}{\pgfqpoint{-0.034373in}{-0.034373in}}%
\pgfpathcurveto{\pgfqpoint{-0.025257in}{-0.043489in}}{\pgfqpoint{-0.012892in}{-0.048611in}}{\pgfqpoint{0.000000in}{-0.048611in}}%
\pgfpathlineto{\pgfqpoint{0.000000in}{-0.048611in}}%
\pgfpathclose%
\pgfusepath{stroke,fill}%
}%
\begin{pgfscope}%
\pgfsys@transformshift{5.206871in}{4.625282in}%
\pgfsys@useobject{currentmarker}{}%
\end{pgfscope}%
\end{pgfscope}%
\begin{pgfscope}%
\definecolor{textcolor}{rgb}{0.150000,0.150000,0.150000}%
\pgfsetstrokecolor{textcolor}%
\pgfsetfillcolor{textcolor}%
\pgftext[x=5.706871in,y=4.552365in,left,base]{\color{textcolor}\rmfamily\fontsize{20.000000}{24.000000}\selectfont Data}%
\end{pgfscope}%
\begin{pgfscope}%
\pgfsetroundcap%
\pgfsetroundjoin%
\pgfsetlinewidth{1.756562pt}%
\definecolor{currentstroke}{rgb}{0.298039,0.447059,0.690196}%
\pgfsetstrokecolor{currentstroke}%
\pgfsetdash{}{0pt}%
\pgfpathmoveto{\pgfqpoint{4.929093in}{4.241873in}}%
\pgfpathlineto{\pgfqpoint{5.206871in}{4.241873in}}%
\pgfpathlineto{\pgfqpoint{5.484649in}{4.241873in}}%
\pgfusepath{stroke}%
\end{pgfscope}%
\begin{pgfscope}%
\definecolor{textcolor}{rgb}{0.150000,0.150000,0.150000}%
\pgfsetstrokecolor{textcolor}%
\pgfsetfillcolor{textcolor}%
\pgftext[x=5.706871in,y=4.144651in,left,base]{\color{textcolor}\rmfamily\fontsize{20.000000}{24.000000}\selectfont Fitted Curve}%
\end{pgfscope}%
\end{pgfpicture}%
\makeatother%
\endgroup%
}
    \caption{T2 measurement of 0.2 M CuSO$_{4} \boldsymbol{\cdot} $H$_2$O using the Meiboom-Gill pulse sequence }
    \label{fig:0d2_t2}
  \end{figure}

  \begin{figure}[h]
    \resizebox{0.45\textwidth}{!}{%% Creator: Matplotlib, PGF backend
%%
%% To include the figure in your LaTeX document, write
%%   \input{<filename>.pgf}
%%
%% Make sure the required packages are loaded in your preamble
%%   \usepackage{pgf}
%%
%% Also ensure that all the required font packages are loaded; for instance,
%% the lmodern package is sometimes necessary when using math font.
%%   \usepackage{lmodern}
%%
%% Figures using additional raster images can only be included by \input if
%% they are in the same directory as the main LaTeX file. For loading figures
%% from other directories you can use the `import` package
%%   \usepackage{import}
%%
%% and then include the figures with
%%   \import{<path to file>}{<filename>.pgf}
%%
%% Matplotlib used the following preamble
%%   \usepackage{fontspec}
%%   \setmainfont{DejaVuSerif.ttf}[Path=\detokenize{/usr/share/matplotlib/mpl-data/fonts/ttf/}]
%%   \setsansfont{DejaVuSans.ttf}[Path=\detokenize{/usr/share/matplotlib/mpl-data/fonts/ttf/}]
%%   \setmonofont{DejaVuSansMono.ttf}[Path=\detokenize{/usr/share/matplotlib/mpl-data/fonts/ttf/}]
%%
\begingroup%
\makeatletter%
\begin{pgfpicture}%
\pgfpathrectangle{\pgfpointorigin}{\pgfqpoint{8.000000in}{5.500000in}}%
\pgfusepath{use as bounding box, clip}%
\begin{pgfscope}%
\pgfsetbuttcap%
\pgfsetmiterjoin%
\definecolor{currentfill}{rgb}{1.000000,1.000000,1.000000}%
\pgfsetfillcolor{currentfill}%
\pgfsetlinewidth{0.000000pt}%
\definecolor{currentstroke}{rgb}{1.000000,1.000000,1.000000}%
\pgfsetstrokecolor{currentstroke}%
\pgfsetdash{}{0pt}%
\pgfpathmoveto{\pgfqpoint{0.000000in}{0.000000in}}%
\pgfpathlineto{\pgfqpoint{8.000000in}{0.000000in}}%
\pgfpathlineto{\pgfqpoint{8.000000in}{5.500000in}}%
\pgfpathlineto{\pgfqpoint{0.000000in}{5.500000in}}%
\pgfpathlineto{\pgfqpoint{0.000000in}{0.000000in}}%
\pgfpathclose%
\pgfusepath{fill}%
\end{pgfscope}%
\begin{pgfscope}%
\pgfsetbuttcap%
\pgfsetmiterjoin%
\definecolor{currentfill}{rgb}{0.917647,0.917647,0.949020}%
\pgfsetfillcolor{currentfill}%
\pgfsetlinewidth{0.000000pt}%
\definecolor{currentstroke}{rgb}{0.000000,0.000000,0.000000}%
\pgfsetstrokecolor{currentstroke}%
\pgfsetstrokeopacity{0.000000}%
\pgfsetdash{}{0pt}%
\pgfpathmoveto{\pgfqpoint{0.802099in}{0.894194in}}%
\pgfpathlineto{\pgfqpoint{7.850000in}{0.894194in}}%
\pgfpathlineto{\pgfqpoint{7.850000in}{5.013411in}}%
\pgfpathlineto{\pgfqpoint{0.802099in}{5.013411in}}%
\pgfpathlineto{\pgfqpoint{0.802099in}{0.894194in}}%
\pgfpathclose%
\pgfusepath{fill}%
\end{pgfscope}%
\begin{pgfscope}%
\pgfpathrectangle{\pgfqpoint{0.802099in}{0.894194in}}{\pgfqpoint{7.047901in}{4.119217in}}%
\pgfusepath{clip}%
\pgfsetroundcap%
\pgfsetroundjoin%
\pgfsetlinewidth{1.003750pt}%
\definecolor{currentstroke}{rgb}{1.000000,1.000000,1.000000}%
\pgfsetstrokecolor{currentstroke}%
\pgfsetdash{}{0pt}%
\pgfpathmoveto{\pgfqpoint{0.876028in}{0.894194in}}%
\pgfpathlineto{\pgfqpoint{0.876028in}{5.013411in}}%
\pgfusepath{stroke}%
\end{pgfscope}%
\begin{pgfscope}%
\definecolor{textcolor}{rgb}{0.150000,0.150000,0.150000}%
\pgfsetstrokecolor{textcolor}%
\pgfsetfillcolor{textcolor}%
\pgftext[x=0.876028in,y=0.796972in,,top]{\color{textcolor}\rmfamily\fontsize{20.000000}{24.000000}\selectfont 0.00}%
\end{pgfscope}%
\begin{pgfscope}%
\pgfpathrectangle{\pgfqpoint{0.802099in}{0.894194in}}{\pgfqpoint{7.047901in}{4.119217in}}%
\pgfusepath{clip}%
\pgfsetroundcap%
\pgfsetroundjoin%
\pgfsetlinewidth{1.003750pt}%
\definecolor{currentstroke}{rgb}{1.000000,1.000000,1.000000}%
\pgfsetstrokecolor{currentstroke}%
\pgfsetdash{}{0pt}%
\pgfpathmoveto{\pgfqpoint{1.902821in}{0.894194in}}%
\pgfpathlineto{\pgfqpoint{1.902821in}{5.013411in}}%
\pgfusepath{stroke}%
\end{pgfscope}%
\begin{pgfscope}%
\definecolor{textcolor}{rgb}{0.150000,0.150000,0.150000}%
\pgfsetstrokecolor{textcolor}%
\pgfsetfillcolor{textcolor}%
\pgftext[x=1.902821in,y=0.796972in,,top]{\color{textcolor}\rmfamily\fontsize{20.000000}{24.000000}\selectfont 0.25}%
\end{pgfscope}%
\begin{pgfscope}%
\pgfpathrectangle{\pgfqpoint{0.802099in}{0.894194in}}{\pgfqpoint{7.047901in}{4.119217in}}%
\pgfusepath{clip}%
\pgfsetroundcap%
\pgfsetroundjoin%
\pgfsetlinewidth{1.003750pt}%
\definecolor{currentstroke}{rgb}{1.000000,1.000000,1.000000}%
\pgfsetstrokecolor{currentstroke}%
\pgfsetdash{}{0pt}%
\pgfpathmoveto{\pgfqpoint{2.929613in}{0.894194in}}%
\pgfpathlineto{\pgfqpoint{2.929613in}{5.013411in}}%
\pgfusepath{stroke}%
\end{pgfscope}%
\begin{pgfscope}%
\definecolor{textcolor}{rgb}{0.150000,0.150000,0.150000}%
\pgfsetstrokecolor{textcolor}%
\pgfsetfillcolor{textcolor}%
\pgftext[x=2.929613in,y=0.796972in,,top]{\color{textcolor}\rmfamily\fontsize{20.000000}{24.000000}\selectfont 0.50}%
\end{pgfscope}%
\begin{pgfscope}%
\pgfpathrectangle{\pgfqpoint{0.802099in}{0.894194in}}{\pgfqpoint{7.047901in}{4.119217in}}%
\pgfusepath{clip}%
\pgfsetroundcap%
\pgfsetroundjoin%
\pgfsetlinewidth{1.003750pt}%
\definecolor{currentstroke}{rgb}{1.000000,1.000000,1.000000}%
\pgfsetstrokecolor{currentstroke}%
\pgfsetdash{}{0pt}%
\pgfpathmoveto{\pgfqpoint{3.956405in}{0.894194in}}%
\pgfpathlineto{\pgfqpoint{3.956405in}{5.013411in}}%
\pgfusepath{stroke}%
\end{pgfscope}%
\begin{pgfscope}%
\definecolor{textcolor}{rgb}{0.150000,0.150000,0.150000}%
\pgfsetstrokecolor{textcolor}%
\pgfsetfillcolor{textcolor}%
\pgftext[x=3.956405in,y=0.796972in,,top]{\color{textcolor}\rmfamily\fontsize{20.000000}{24.000000}\selectfont 0.75}%
\end{pgfscope}%
\begin{pgfscope}%
\pgfpathrectangle{\pgfqpoint{0.802099in}{0.894194in}}{\pgfqpoint{7.047901in}{4.119217in}}%
\pgfusepath{clip}%
\pgfsetroundcap%
\pgfsetroundjoin%
\pgfsetlinewidth{1.003750pt}%
\definecolor{currentstroke}{rgb}{1.000000,1.000000,1.000000}%
\pgfsetstrokecolor{currentstroke}%
\pgfsetdash{}{0pt}%
\pgfpathmoveto{\pgfqpoint{4.983197in}{0.894194in}}%
\pgfpathlineto{\pgfqpoint{4.983197in}{5.013411in}}%
\pgfusepath{stroke}%
\end{pgfscope}%
\begin{pgfscope}%
\definecolor{textcolor}{rgb}{0.150000,0.150000,0.150000}%
\pgfsetstrokecolor{textcolor}%
\pgfsetfillcolor{textcolor}%
\pgftext[x=4.983197in,y=0.796972in,,top]{\color{textcolor}\rmfamily\fontsize{20.000000}{24.000000}\selectfont 1.00}%
\end{pgfscope}%
\begin{pgfscope}%
\pgfpathrectangle{\pgfqpoint{0.802099in}{0.894194in}}{\pgfqpoint{7.047901in}{4.119217in}}%
\pgfusepath{clip}%
\pgfsetroundcap%
\pgfsetroundjoin%
\pgfsetlinewidth{1.003750pt}%
\definecolor{currentstroke}{rgb}{1.000000,1.000000,1.000000}%
\pgfsetstrokecolor{currentstroke}%
\pgfsetdash{}{0pt}%
\pgfpathmoveto{\pgfqpoint{6.009989in}{0.894194in}}%
\pgfpathlineto{\pgfqpoint{6.009989in}{5.013411in}}%
\pgfusepath{stroke}%
\end{pgfscope}%
\begin{pgfscope}%
\definecolor{textcolor}{rgb}{0.150000,0.150000,0.150000}%
\pgfsetstrokecolor{textcolor}%
\pgfsetfillcolor{textcolor}%
\pgftext[x=6.009989in,y=0.796972in,,top]{\color{textcolor}\rmfamily\fontsize{20.000000}{24.000000}\selectfont 1.25}%
\end{pgfscope}%
\begin{pgfscope}%
\pgfpathrectangle{\pgfqpoint{0.802099in}{0.894194in}}{\pgfqpoint{7.047901in}{4.119217in}}%
\pgfusepath{clip}%
\pgfsetroundcap%
\pgfsetroundjoin%
\pgfsetlinewidth{1.003750pt}%
\definecolor{currentstroke}{rgb}{1.000000,1.000000,1.000000}%
\pgfsetstrokecolor{currentstroke}%
\pgfsetdash{}{0pt}%
\pgfpathmoveto{\pgfqpoint{7.036781in}{0.894194in}}%
\pgfpathlineto{\pgfqpoint{7.036781in}{5.013411in}}%
\pgfusepath{stroke}%
\end{pgfscope}%
\begin{pgfscope}%
\definecolor{textcolor}{rgb}{0.150000,0.150000,0.150000}%
\pgfsetstrokecolor{textcolor}%
\pgfsetfillcolor{textcolor}%
\pgftext[x=7.036781in,y=0.796972in,,top]{\color{textcolor}\rmfamily\fontsize{20.000000}{24.000000}\selectfont 1.50}%
\end{pgfscope}%
\begin{pgfscope}%
\definecolor{textcolor}{rgb}{0.150000,0.150000,0.150000}%
\pgfsetstrokecolor{textcolor}%
\pgfsetfillcolor{textcolor}%
\pgftext[x=4.326050in,y=0.472591in,,top]{\color{textcolor}\rmfamily\fontsize{24.000000}{28.800000}\selectfont Time (ms)}%
\end{pgfscope}%
\begin{pgfscope}%
\pgfpathrectangle{\pgfqpoint{0.802099in}{0.894194in}}{\pgfqpoint{7.047901in}{4.119217in}}%
\pgfusepath{clip}%
\pgfsetroundcap%
\pgfsetroundjoin%
\pgfsetlinewidth{1.003750pt}%
\definecolor{currentstroke}{rgb}{1.000000,1.000000,1.000000}%
\pgfsetstrokecolor{currentstroke}%
\pgfsetdash{}{0pt}%
\pgfpathmoveto{\pgfqpoint{0.802099in}{1.828475in}}%
\pgfpathlineto{\pgfqpoint{7.850000in}{1.828475in}}%
\pgfusepath{stroke}%
\end{pgfscope}%
\begin{pgfscope}%
\definecolor{textcolor}{rgb}{0.150000,0.150000,0.150000}%
\pgfsetstrokecolor{textcolor}%
\pgfsetfillcolor{textcolor}%
\pgftext[x=0.528147in, y=1.722952in, left, base]{\color{textcolor}\rmfamily\fontsize{20.000000}{24.000000}\selectfont 3}%
\end{pgfscope}%
\begin{pgfscope}%
\pgfpathrectangle{\pgfqpoint{0.802099in}{0.894194in}}{\pgfqpoint{7.047901in}{4.119217in}}%
\pgfusepath{clip}%
\pgfsetroundcap%
\pgfsetroundjoin%
\pgfsetlinewidth{1.003750pt}%
\definecolor{currentstroke}{rgb}{1.000000,1.000000,1.000000}%
\pgfsetstrokecolor{currentstroke}%
\pgfsetdash{}{0pt}%
\pgfpathmoveto{\pgfqpoint{0.802099in}{2.963508in}}%
\pgfpathlineto{\pgfqpoint{7.850000in}{2.963508in}}%
\pgfusepath{stroke}%
\end{pgfscope}%
\begin{pgfscope}%
\definecolor{textcolor}{rgb}{0.150000,0.150000,0.150000}%
\pgfsetstrokecolor{textcolor}%
\pgfsetfillcolor{textcolor}%
\pgftext[x=0.528147in, y=2.857985in, left, base]{\color{textcolor}\rmfamily\fontsize{20.000000}{24.000000}\selectfont 4}%
\end{pgfscope}%
\begin{pgfscope}%
\pgfpathrectangle{\pgfqpoint{0.802099in}{0.894194in}}{\pgfqpoint{7.047901in}{4.119217in}}%
\pgfusepath{clip}%
\pgfsetroundcap%
\pgfsetroundjoin%
\pgfsetlinewidth{1.003750pt}%
\definecolor{currentstroke}{rgb}{1.000000,1.000000,1.000000}%
\pgfsetstrokecolor{currentstroke}%
\pgfsetdash{}{0pt}%
\pgfpathmoveto{\pgfqpoint{0.802099in}{4.098540in}}%
\pgfpathlineto{\pgfqpoint{7.850000in}{4.098540in}}%
\pgfusepath{stroke}%
\end{pgfscope}%
\begin{pgfscope}%
\definecolor{textcolor}{rgb}{0.150000,0.150000,0.150000}%
\pgfsetstrokecolor{textcolor}%
\pgfsetfillcolor{textcolor}%
\pgftext[x=0.528147in, y=3.993017in, left, base]{\color{textcolor}\rmfamily\fontsize{20.000000}{24.000000}\selectfont 5}%
\end{pgfscope}%
\begin{pgfscope}%
\definecolor{textcolor}{rgb}{0.150000,0.150000,0.150000}%
\pgfsetstrokecolor{textcolor}%
\pgfsetfillcolor{textcolor}%
\pgftext[x=0.472591in,y=2.953803in,,bottom,rotate=90.000000]{\color{textcolor}\rmfamily\fontsize{24.000000}{28.800000}\selectfont Pulse Amplitude (V)}%
\end{pgfscope}%
\begin{pgfscope}%
\pgfpathrectangle{\pgfqpoint{0.802099in}{0.894194in}}{\pgfqpoint{7.047901in}{4.119217in}}%
\pgfusepath{clip}%
\pgfsetbuttcap%
\pgfsetroundjoin%
\definecolor{currentfill}{rgb}{0.698039,0.133333,0.133333}%
\pgfsetfillcolor{currentfill}%
\pgfsetlinewidth{0.301125pt}%
\definecolor{currentstroke}{rgb}{0.698039,0.133333,0.133333}%
\pgfsetstrokecolor{currentstroke}%
\pgfsetdash{}{0pt}%
\pgfsys@defobject{currentmarker}{\pgfqpoint{-0.048611in}{-0.048611in}}{\pgfqpoint{0.048611in}{0.048611in}}{%
\pgfpathmoveto{\pgfqpoint{0.000000in}{-0.048611in}}%
\pgfpathcurveto{\pgfqpoint{0.012892in}{-0.048611in}}{\pgfqpoint{0.025257in}{-0.043489in}}{\pgfqpoint{0.034373in}{-0.034373in}}%
\pgfpathcurveto{\pgfqpoint{0.043489in}{-0.025257in}}{\pgfqpoint{0.048611in}{-0.012892in}}{\pgfqpoint{0.048611in}{0.000000in}}%
\pgfpathcurveto{\pgfqpoint{0.048611in}{0.012892in}}{\pgfqpoint{0.043489in}{0.025257in}}{\pgfqpoint{0.034373in}{0.034373in}}%
\pgfpathcurveto{\pgfqpoint{0.025257in}{0.043489in}}{\pgfqpoint{0.012892in}{0.048611in}}{\pgfqpoint{0.000000in}{0.048611in}}%
\pgfpathcurveto{\pgfqpoint{-0.012892in}{0.048611in}}{\pgfqpoint{-0.025257in}{0.043489in}}{\pgfqpoint{-0.034373in}{0.034373in}}%
\pgfpathcurveto{\pgfqpoint{-0.043489in}{0.025257in}}{\pgfqpoint{-0.048611in}{0.012892in}}{\pgfqpoint{-0.048611in}{0.000000in}}%
\pgfpathcurveto{\pgfqpoint{-0.048611in}{-0.012892in}}{\pgfqpoint{-0.043489in}{-0.025257in}}{\pgfqpoint{-0.034373in}{-0.034373in}}%
\pgfpathcurveto{\pgfqpoint{-0.025257in}{-0.043489in}}{\pgfqpoint{-0.012892in}{-0.048611in}}{\pgfqpoint{0.000000in}{-0.048611in}}%
\pgfpathlineto{\pgfqpoint{0.000000in}{-0.048611in}}%
\pgfpathclose%
\pgfusepath{stroke,fill}%
}%
\begin{pgfscope}%
\pgfsys@transformshift{1.122459in}{4.779560in}%
\pgfsys@useobject{currentmarker}{}%
\end{pgfscope}%
\begin{pgfscope}%
\pgfsys@transformshift{1.615319in}{4.370948in}%
\pgfsys@useobject{currentmarker}{}%
\end{pgfscope}%
\begin{pgfscope}%
\pgfsys@transformshift{2.108179in}{4.189343in}%
\pgfsys@useobject{currentmarker}{}%
\end{pgfscope}%
\begin{pgfscope}%
\pgfsys@transformshift{2.601039in}{3.871534in}%
\pgfsys@useobject{currentmarker}{}%
\end{pgfscope}%
\begin{pgfscope}%
\pgfsys@transformshift{3.093899in}{3.417521in}%
\pgfsys@useobject{currentmarker}{}%
\end{pgfscope}%
\begin{pgfscope}%
\pgfsys@transformshift{3.586759in}{3.099712in}%
\pgfsys@useobject{currentmarker}{}%
\end{pgfscope}%
\begin{pgfscope}%
\pgfsys@transformshift{4.161763in}{2.781903in}%
\pgfsys@useobject{currentmarker}{}%
\end{pgfscope}%
\begin{pgfscope}%
\pgfsys@transformshift{4.572480in}{2.464094in}%
\pgfsys@useobject{currentmarker}{}%
\end{pgfscope}%
\begin{pgfscope}%
\pgfsys@transformshift{5.065340in}{2.191686in}%
\pgfsys@useobject{currentmarker}{}%
\end{pgfscope}%
\begin{pgfscope}%
\pgfsys@transformshift{5.558200in}{1.964679in}%
\pgfsys@useobject{currentmarker}{}%
\end{pgfscope}%
\begin{pgfscope}%
\pgfsys@transformshift{6.215347in}{1.692271in}%
\pgfsys@useobject{currentmarker}{}%
\end{pgfscope}%
\begin{pgfscope}%
\pgfsys@transformshift{6.543921in}{1.510666in}%
\pgfsys@useobject{currentmarker}{}%
\end{pgfscope}%
\begin{pgfscope}%
\pgfsys@transformshift{7.036781in}{1.329061in}%
\pgfsys@useobject{currentmarker}{}%
\end{pgfscope}%
\begin{pgfscope}%
\pgfsys@transformshift{7.529641in}{1.102055in}%
\pgfsys@useobject{currentmarker}{}%
\end{pgfscope}%
\end{pgfscope}%
\begin{pgfscope}%
\pgfpathrectangle{\pgfqpoint{0.802099in}{0.894194in}}{\pgfqpoint{7.047901in}{4.119217in}}%
\pgfusepath{clip}%
\pgfsetroundcap%
\pgfsetroundjoin%
\pgfsetlinewidth{1.756562pt}%
\definecolor{currentstroke}{rgb}{0.298039,0.447059,0.690196}%
\pgfsetstrokecolor{currentstroke}%
\pgfsetdash{}{0pt}%
\pgfpathmoveto{\pgfqpoint{1.122459in}{4.826174in}}%
\pgfpathlineto{\pgfqpoint{1.187178in}{4.775712in}}%
\pgfpathlineto{\pgfqpoint{1.251897in}{4.725563in}}%
\pgfpathlineto{\pgfqpoint{1.316616in}{4.675723in}}%
\pgfpathlineto{\pgfqpoint{1.381335in}{4.626191in}}%
\pgfpathlineto{\pgfqpoint{1.446054in}{4.576966in}}%
\pgfpathlineto{\pgfqpoint{1.510773in}{4.528045in}}%
\pgfpathlineto{\pgfqpoint{1.575492in}{4.479426in}}%
\pgfpathlineto{\pgfqpoint{1.640211in}{4.431108in}}%
\pgfpathlineto{\pgfqpoint{1.704930in}{4.383089in}}%
\pgfpathlineto{\pgfqpoint{1.769649in}{4.335367in}}%
\pgfpathlineto{\pgfqpoint{1.834368in}{4.287939in}}%
\pgfpathlineto{\pgfqpoint{1.899087in}{4.240805in}}%
\pgfpathlineto{\pgfqpoint{1.963806in}{4.193962in}}%
\pgfpathlineto{\pgfqpoint{2.028525in}{4.147409in}}%
\pgfpathlineto{\pgfqpoint{2.093244in}{4.101144in}}%
\pgfpathlineto{\pgfqpoint{2.157963in}{4.055165in}}%
\pgfpathlineto{\pgfqpoint{2.222682in}{4.009470in}}%
\pgfpathlineto{\pgfqpoint{2.287401in}{3.964057in}}%
\pgfpathlineto{\pgfqpoint{2.352120in}{3.918925in}}%
\pgfpathlineto{\pgfqpoint{2.416839in}{3.874073in}}%
\pgfpathlineto{\pgfqpoint{2.481558in}{3.829497in}}%
\pgfpathlineto{\pgfqpoint{2.546277in}{3.785198in}}%
\pgfpathlineto{\pgfqpoint{2.610996in}{3.741172in}}%
\pgfpathlineto{\pgfqpoint{2.675715in}{3.697418in}}%
\pgfpathlineto{\pgfqpoint{2.740434in}{3.653935in}}%
\pgfpathlineto{\pgfqpoint{2.805153in}{3.610720in}}%
\pgfpathlineto{\pgfqpoint{2.869872in}{3.567773in}}%
\pgfpathlineto{\pgfqpoint{2.934591in}{3.525091in}}%
\pgfpathlineto{\pgfqpoint{2.999310in}{3.482674in}}%
\pgfpathlineto{\pgfqpoint{3.064029in}{3.440518in}}%
\pgfpathlineto{\pgfqpoint{3.128748in}{3.398623in}}%
\pgfpathlineto{\pgfqpoint{3.193467in}{3.356987in}}%
\pgfpathlineto{\pgfqpoint{3.258186in}{3.315609in}}%
\pgfpathlineto{\pgfqpoint{3.322905in}{3.274486in}}%
\pgfpathlineto{\pgfqpoint{3.387624in}{3.233618in}}%
\pgfpathlineto{\pgfqpoint{3.452343in}{3.193002in}}%
\pgfpathlineto{\pgfqpoint{3.517062in}{3.152637in}}%
\pgfpathlineto{\pgfqpoint{3.581781in}{3.112522in}}%
\pgfpathlineto{\pgfqpoint{3.646500in}{3.072655in}}%
\pgfpathlineto{\pgfqpoint{3.711219in}{3.033034in}}%
\pgfpathlineto{\pgfqpoint{3.775938in}{2.993659in}}%
\pgfpathlineto{\pgfqpoint{3.840657in}{2.954526in}}%
\pgfpathlineto{\pgfqpoint{3.905376in}{2.915636in}}%
\pgfpathlineto{\pgfqpoint{3.970095in}{2.876986in}}%
\pgfpathlineto{\pgfqpoint{4.034814in}{2.838575in}}%
\pgfpathlineto{\pgfqpoint{4.099533in}{2.800402in}}%
\pgfpathlineto{\pgfqpoint{4.164252in}{2.762465in}}%
\pgfpathlineto{\pgfqpoint{4.228971in}{2.724762in}}%
\pgfpathlineto{\pgfqpoint{4.293690in}{2.687292in}}%
\pgfpathlineto{\pgfqpoint{4.358409in}{2.650054in}}%
\pgfpathlineto{\pgfqpoint{4.423128in}{2.613046in}}%
\pgfpathlineto{\pgfqpoint{4.487847in}{2.576267in}}%
\pgfpathlineto{\pgfqpoint{4.552566in}{2.539715in}}%
\pgfpathlineto{\pgfqpoint{4.617285in}{2.503390in}}%
\pgfpathlineto{\pgfqpoint{4.682004in}{2.467288in}}%
\pgfpathlineto{\pgfqpoint{4.746723in}{2.431411in}}%
\pgfpathlineto{\pgfqpoint{4.811442in}{2.395754in}}%
\pgfpathlineto{\pgfqpoint{4.876161in}{2.360319in}}%
\pgfpathlineto{\pgfqpoint{4.940880in}{2.325102in}}%
\pgfpathlineto{\pgfqpoint{5.005599in}{2.290103in}}%
\pgfpathlineto{\pgfqpoint{5.070318in}{2.255321in}}%
\pgfpathlineto{\pgfqpoint{5.135037in}{2.220753in}}%
\pgfpathlineto{\pgfqpoint{5.199756in}{2.186400in}}%
\pgfpathlineto{\pgfqpoint{5.264475in}{2.152258in}}%
\pgfpathlineto{\pgfqpoint{5.329194in}{2.118328in}}%
\pgfpathlineto{\pgfqpoint{5.393913in}{2.084608in}}%
\pgfpathlineto{\pgfqpoint{5.458632in}{2.051096in}}%
\pgfpathlineto{\pgfqpoint{5.523351in}{2.017791in}}%
\pgfpathlineto{\pgfqpoint{5.588070in}{1.984692in}}%
\pgfpathlineto{\pgfqpoint{5.652789in}{1.951798in}}%
\pgfpathlineto{\pgfqpoint{5.717509in}{1.919107in}}%
\pgfpathlineto{\pgfqpoint{5.782228in}{1.886618in}}%
\pgfpathlineto{\pgfqpoint{5.846947in}{1.854330in}}%
\pgfpathlineto{\pgfqpoint{5.911666in}{1.822242in}}%
\pgfpathlineto{\pgfqpoint{5.976385in}{1.790352in}}%
\pgfpathlineto{\pgfqpoint{6.041104in}{1.758659in}}%
\pgfpathlineto{\pgfqpoint{6.105823in}{1.727162in}}%
\pgfpathlineto{\pgfqpoint{6.170542in}{1.695860in}}%
\pgfpathlineto{\pgfqpoint{6.235261in}{1.664752in}}%
\pgfpathlineto{\pgfqpoint{6.299980in}{1.633836in}}%
\pgfpathlineto{\pgfqpoint{6.364699in}{1.603110in}}%
\pgfpathlineto{\pgfqpoint{6.429418in}{1.572575in}}%
\pgfpathlineto{\pgfqpoint{6.494137in}{1.542229in}}%
\pgfpathlineto{\pgfqpoint{6.558856in}{1.512070in}}%
\pgfpathlineto{\pgfqpoint{6.623575in}{1.482098in}}%
\pgfpathlineto{\pgfqpoint{6.688294in}{1.452311in}}%
\pgfpathlineto{\pgfqpoint{6.753013in}{1.422708in}}%
\pgfpathlineto{\pgfqpoint{6.817732in}{1.393289in}}%
\pgfpathlineto{\pgfqpoint{6.882451in}{1.364051in}}%
\pgfpathlineto{\pgfqpoint{6.947170in}{1.334993in}}%
\pgfpathlineto{\pgfqpoint{7.011889in}{1.306116in}}%
\pgfpathlineto{\pgfqpoint{7.076608in}{1.277417in}}%
\pgfpathlineto{\pgfqpoint{7.141327in}{1.248896in}}%
\pgfpathlineto{\pgfqpoint{7.206046in}{1.220550in}}%
\pgfpathlineto{\pgfqpoint{7.270765in}{1.192380in}}%
\pgfpathlineto{\pgfqpoint{7.335484in}{1.164385in}}%
\pgfpathlineto{\pgfqpoint{7.400203in}{1.136562in}}%
\pgfpathlineto{\pgfqpoint{7.464922in}{1.108911in}}%
\pgfpathlineto{\pgfqpoint{7.529641in}{1.081432in}}%
\pgfusepath{stroke}%
\end{pgfscope}%
\begin{pgfscope}%
\pgfsetrectcap%
\pgfsetmiterjoin%
\pgfsetlinewidth{0.000000pt}%
\definecolor{currentstroke}{rgb}{1.000000,1.000000,1.000000}%
\pgfsetstrokecolor{currentstroke}%
\pgfsetdash{}{0pt}%
\pgfpathmoveto{\pgfqpoint{0.802099in}{0.894194in}}%
\pgfpathlineto{\pgfqpoint{0.802099in}{5.013411in}}%
\pgfusepath{}%
\end{pgfscope}%
\begin{pgfscope}%
\pgfsetrectcap%
\pgfsetmiterjoin%
\pgfsetlinewidth{0.000000pt}%
\definecolor{currentstroke}{rgb}{1.000000,1.000000,1.000000}%
\pgfsetstrokecolor{currentstroke}%
\pgfsetdash{}{0pt}%
\pgfpathmoveto{\pgfqpoint{7.850000in}{0.894194in}}%
\pgfpathlineto{\pgfqpoint{7.850000in}{5.013411in}}%
\pgfusepath{}%
\end{pgfscope}%
\begin{pgfscope}%
\pgfsetrectcap%
\pgfsetmiterjoin%
\pgfsetlinewidth{0.000000pt}%
\definecolor{currentstroke}{rgb}{1.000000,1.000000,1.000000}%
\pgfsetstrokecolor{currentstroke}%
\pgfsetdash{}{0pt}%
\pgfpathmoveto{\pgfqpoint{0.802099in}{0.894194in}}%
\pgfpathlineto{\pgfqpoint{7.850000in}{0.894194in}}%
\pgfusepath{}%
\end{pgfscope}%
\begin{pgfscope}%
\pgfsetrectcap%
\pgfsetmiterjoin%
\pgfsetlinewidth{0.000000pt}%
\definecolor{currentstroke}{rgb}{1.000000,1.000000,1.000000}%
\pgfsetstrokecolor{currentstroke}%
\pgfsetdash{}{0pt}%
\pgfpathmoveto{\pgfqpoint{0.802099in}{5.013411in}}%
\pgfpathlineto{\pgfqpoint{7.850000in}{5.013411in}}%
\pgfusepath{}%
\end{pgfscope}%
\begin{pgfscope}%
\definecolor{textcolor}{rgb}{0.150000,0.150000,0.150000}%
\pgfsetstrokecolor{textcolor}%
\pgfsetfillcolor{textcolor}%
\pgftext[x=1.286745in, y=1.755015in, left, base]{\color{textcolor}\rmfamily\fontsize{20.000000}{24.000000}\selectfont M\(\displaystyle _{x,y}\)(t) = 7.36\(\displaystyle exp[\frac{-2t}{T_2}]+\)-1.55}%
\end{pgfscope}%
\begin{pgfscope}%
\definecolor{textcolor}{rgb}{0.150000,0.150000,0.150000}%
\pgfsetstrokecolor{textcolor}%
\pgfsetfillcolor{textcolor}%
\pgftext[x=1.286745in, y=1.260959in, left, base]{\color{textcolor}\rmfamily\fontsize{20.000000}{24.000000}\selectfont T\(\displaystyle _2\) = 5.081\(\displaystyle \pm\)0.90 ms}%
\end{pgfscope}%
\begin{pgfscope}%
\definecolor{textcolor}{rgb}{0.150000,0.150000,0.150000}%
\pgfsetstrokecolor{textcolor}%
\pgfsetfillcolor{textcolor}%
\pgftext[x=4.326050in,y=5.096745in,,base]{\color{textcolor}\rmfamily\fontsize{24.000000}{28.800000}\selectfont MG T\(\displaystyle _2\) measurement of 0.5 M CuSO\(\displaystyle _4\)}%
\end{pgfscope}%
\begin{pgfscope}%
\pgfsetbuttcap%
\pgfsetroundjoin%
\definecolor{currentfill}{rgb}{0.698039,0.133333,0.133333}%
\pgfsetfillcolor{currentfill}%
\pgfsetlinewidth{0.301125pt}%
\definecolor{currentstroke}{rgb}{0.698039,0.133333,0.133333}%
\pgfsetstrokecolor{currentstroke}%
\pgfsetdash{}{0pt}%
\pgfsys@defobject{currentmarker}{\pgfqpoint{-0.048611in}{-0.048611in}}{\pgfqpoint{0.048611in}{0.048611in}}{%
\pgfpathmoveto{\pgfqpoint{0.000000in}{-0.048611in}}%
\pgfpathcurveto{\pgfqpoint{0.012892in}{-0.048611in}}{\pgfqpoint{0.025257in}{-0.043489in}}{\pgfqpoint{0.034373in}{-0.034373in}}%
\pgfpathcurveto{\pgfqpoint{0.043489in}{-0.025257in}}{\pgfqpoint{0.048611in}{-0.012892in}}{\pgfqpoint{0.048611in}{0.000000in}}%
\pgfpathcurveto{\pgfqpoint{0.048611in}{0.012892in}}{\pgfqpoint{0.043489in}{0.025257in}}{\pgfqpoint{0.034373in}{0.034373in}}%
\pgfpathcurveto{\pgfqpoint{0.025257in}{0.043489in}}{\pgfqpoint{0.012892in}{0.048611in}}{\pgfqpoint{0.000000in}{0.048611in}}%
\pgfpathcurveto{\pgfqpoint{-0.012892in}{0.048611in}}{\pgfqpoint{-0.025257in}{0.043489in}}{\pgfqpoint{-0.034373in}{0.034373in}}%
\pgfpathcurveto{\pgfqpoint{-0.043489in}{0.025257in}}{\pgfqpoint{-0.048611in}{0.012892in}}{\pgfqpoint{-0.048611in}{0.000000in}}%
\pgfpathcurveto{\pgfqpoint{-0.048611in}{-0.012892in}}{\pgfqpoint{-0.043489in}{-0.025257in}}{\pgfqpoint{-0.034373in}{-0.034373in}}%
\pgfpathcurveto{\pgfqpoint{-0.025257in}{-0.043489in}}{\pgfqpoint{-0.012892in}{-0.048611in}}{\pgfqpoint{0.000000in}{-0.048611in}}%
\pgfpathlineto{\pgfqpoint{0.000000in}{-0.048611in}}%
\pgfpathclose%
\pgfusepath{stroke,fill}%
}%
\begin{pgfscope}%
\pgfsys@transformshift{5.319130in}{4.625282in}%
\pgfsys@useobject{currentmarker}{}%
\end{pgfscope}%
\end{pgfscope}%
\begin{pgfscope}%
\definecolor{textcolor}{rgb}{0.150000,0.150000,0.150000}%
\pgfsetstrokecolor{textcolor}%
\pgfsetfillcolor{textcolor}%
\pgftext[x=5.819130in,y=4.552365in,left,base]{\color{textcolor}\rmfamily\fontsize{20.000000}{24.000000}\selectfont Data}%
\end{pgfscope}%
\begin{pgfscope}%
\pgfsetroundcap%
\pgfsetroundjoin%
\pgfsetlinewidth{1.756562pt}%
\definecolor{currentstroke}{rgb}{0.298039,0.447059,0.690196}%
\pgfsetstrokecolor{currentstroke}%
\pgfsetdash{}{0pt}%
\pgfpathmoveto{\pgfqpoint{5.041352in}{4.241873in}}%
\pgfpathlineto{\pgfqpoint{5.319130in}{4.241873in}}%
\pgfpathlineto{\pgfqpoint{5.596907in}{4.241873in}}%
\pgfusepath{stroke}%
\end{pgfscope}%
\begin{pgfscope}%
\definecolor{textcolor}{rgb}{0.150000,0.150000,0.150000}%
\pgfsetstrokecolor{textcolor}%
\pgfsetfillcolor{textcolor}%
\pgftext[x=5.819130in,y=4.144651in,left,base]{\color{textcolor}\rmfamily\fontsize{20.000000}{24.000000}\selectfont Fitted Curve}%
\end{pgfscope}%
\end{pgfpicture}%
\makeatother%
\endgroup%
}
    \caption{T2 measurement of 0.5 M CuSO$_{4} \boldsymbol{\cdot} $H$_2$O using the Meiboom-Gill pulse sequence }
    \label{fig:0d5_t2}
  \end{figure}

  \begin{figure}[h]
    \resizebox{0.45\textwidth}{!}{%% Creator: Matplotlib, PGF backend
%%
%% To include the figure in your LaTeX document, write
%%   \input{<filename>.pgf}
%%
%% Make sure the required packages are loaded in your preamble
%%   \usepackage{pgf}
%%
%% Also ensure that all the required font packages are loaded; for instance,
%% the lmodern package is sometimes necessary when using math font.
%%   \usepackage{lmodern}
%%
%% Figures using additional raster images can only be included by \input if
%% they are in the same directory as the main LaTeX file. For loading figures
%% from other directories you can use the `import` package
%%   \usepackage{import}
%%
%% and then include the figures with
%%   \import{<path to file>}{<filename>.pgf}
%%
%% Matplotlib used the following preamble
%%   \usepackage{fontspec}
%%   \setmainfont{DejaVuSerif.ttf}[Path=\detokenize{/usr/share/matplotlib/mpl-data/fonts/ttf/}]
%%   \setsansfont{DejaVuSans.ttf}[Path=\detokenize{/usr/share/matplotlib/mpl-data/fonts/ttf/}]
%%   \setmonofont{DejaVuSansMono.ttf}[Path=\detokenize{/usr/share/matplotlib/mpl-data/fonts/ttf/}]
%%
\begingroup%
\makeatletter%
\begin{pgfpicture}%
\pgfpathrectangle{\pgfpointorigin}{\pgfqpoint{8.000000in}{5.500000in}}%
\pgfusepath{use as bounding box, clip}%
\begin{pgfscope}%
\pgfsetbuttcap%
\pgfsetmiterjoin%
\definecolor{currentfill}{rgb}{1.000000,1.000000,1.000000}%
\pgfsetfillcolor{currentfill}%
\pgfsetlinewidth{0.000000pt}%
\definecolor{currentstroke}{rgb}{1.000000,1.000000,1.000000}%
\pgfsetstrokecolor{currentstroke}%
\pgfsetdash{}{0pt}%
\pgfpathmoveto{\pgfqpoint{0.000000in}{0.000000in}}%
\pgfpathlineto{\pgfqpoint{8.000000in}{0.000000in}}%
\pgfpathlineto{\pgfqpoint{8.000000in}{5.500000in}}%
\pgfpathlineto{\pgfqpoint{0.000000in}{5.500000in}}%
\pgfpathlineto{\pgfqpoint{0.000000in}{0.000000in}}%
\pgfpathclose%
\pgfusepath{fill}%
\end{pgfscope}%
\begin{pgfscope}%
\pgfsetbuttcap%
\pgfsetmiterjoin%
\definecolor{currentfill}{rgb}{0.917647,0.917647,0.949020}%
\pgfsetfillcolor{currentfill}%
\pgfsetlinewidth{0.000000pt}%
\definecolor{currentstroke}{rgb}{0.000000,0.000000,0.000000}%
\pgfsetstrokecolor{currentstroke}%
\pgfsetstrokeopacity{0.000000}%
\pgfsetdash{}{0pt}%
\pgfpathmoveto{\pgfqpoint{0.802099in}{0.894194in}}%
\pgfpathlineto{\pgfqpoint{7.553939in}{0.894194in}}%
\pgfpathlineto{\pgfqpoint{7.553939in}{5.013411in}}%
\pgfpathlineto{\pgfqpoint{0.802099in}{5.013411in}}%
\pgfpathlineto{\pgfqpoint{0.802099in}{0.894194in}}%
\pgfpathclose%
\pgfusepath{fill}%
\end{pgfscope}%
\begin{pgfscope}%
\pgfpathrectangle{\pgfqpoint{0.802099in}{0.894194in}}{\pgfqpoint{6.751839in}{4.119217in}}%
\pgfusepath{clip}%
\pgfsetroundcap%
\pgfsetroundjoin%
\pgfsetlinewidth{1.003750pt}%
\definecolor{currentstroke}{rgb}{1.000000,1.000000,1.000000}%
\pgfsetstrokecolor{currentstroke}%
\pgfsetdash{}{0pt}%
\pgfpathmoveto{\pgfqpoint{1.784185in}{0.894194in}}%
\pgfpathlineto{\pgfqpoint{1.784185in}{5.013411in}}%
\pgfusepath{stroke}%
\end{pgfscope}%
\begin{pgfscope}%
\definecolor{textcolor}{rgb}{0.150000,0.150000,0.150000}%
\pgfsetstrokecolor{textcolor}%
\pgfsetfillcolor{textcolor}%
\pgftext[x=1.784185in,y=0.796972in,,top]{\color{textcolor}\rmfamily\fontsize{20.000000}{24.000000}\selectfont 0.5}%
\end{pgfscope}%
\begin{pgfscope}%
\pgfpathrectangle{\pgfqpoint{0.802099in}{0.894194in}}{\pgfqpoint{6.751839in}{4.119217in}}%
\pgfusepath{clip}%
\pgfsetroundcap%
\pgfsetroundjoin%
\pgfsetlinewidth{1.003750pt}%
\definecolor{currentstroke}{rgb}{1.000000,1.000000,1.000000}%
\pgfsetstrokecolor{currentstroke}%
\pgfsetdash{}{0pt}%
\pgfpathmoveto{\pgfqpoint{3.318694in}{0.894194in}}%
\pgfpathlineto{\pgfqpoint{3.318694in}{5.013411in}}%
\pgfusepath{stroke}%
\end{pgfscope}%
\begin{pgfscope}%
\definecolor{textcolor}{rgb}{0.150000,0.150000,0.150000}%
\pgfsetstrokecolor{textcolor}%
\pgfsetfillcolor{textcolor}%
\pgftext[x=3.318694in,y=0.796972in,,top]{\color{textcolor}\rmfamily\fontsize{20.000000}{24.000000}\selectfont 1.0}%
\end{pgfscope}%
\begin{pgfscope}%
\pgfpathrectangle{\pgfqpoint{0.802099in}{0.894194in}}{\pgfqpoint{6.751839in}{4.119217in}}%
\pgfusepath{clip}%
\pgfsetroundcap%
\pgfsetroundjoin%
\pgfsetlinewidth{1.003750pt}%
\definecolor{currentstroke}{rgb}{1.000000,1.000000,1.000000}%
\pgfsetstrokecolor{currentstroke}%
\pgfsetdash{}{0pt}%
\pgfpathmoveto{\pgfqpoint{4.853203in}{0.894194in}}%
\pgfpathlineto{\pgfqpoint{4.853203in}{5.013411in}}%
\pgfusepath{stroke}%
\end{pgfscope}%
\begin{pgfscope}%
\definecolor{textcolor}{rgb}{0.150000,0.150000,0.150000}%
\pgfsetstrokecolor{textcolor}%
\pgfsetfillcolor{textcolor}%
\pgftext[x=4.853203in,y=0.796972in,,top]{\color{textcolor}\rmfamily\fontsize{20.000000}{24.000000}\selectfont 1.5}%
\end{pgfscope}%
\begin{pgfscope}%
\pgfpathrectangle{\pgfqpoint{0.802099in}{0.894194in}}{\pgfqpoint{6.751839in}{4.119217in}}%
\pgfusepath{clip}%
\pgfsetroundcap%
\pgfsetroundjoin%
\pgfsetlinewidth{1.003750pt}%
\definecolor{currentstroke}{rgb}{1.000000,1.000000,1.000000}%
\pgfsetstrokecolor{currentstroke}%
\pgfsetdash{}{0pt}%
\pgfpathmoveto{\pgfqpoint{6.387712in}{0.894194in}}%
\pgfpathlineto{\pgfqpoint{6.387712in}{5.013411in}}%
\pgfusepath{stroke}%
\end{pgfscope}%
\begin{pgfscope}%
\definecolor{textcolor}{rgb}{0.150000,0.150000,0.150000}%
\pgfsetstrokecolor{textcolor}%
\pgfsetfillcolor{textcolor}%
\pgftext[x=6.387712in,y=0.796972in,,top]{\color{textcolor}\rmfamily\fontsize{20.000000}{24.000000}\selectfont 2.0}%
\end{pgfscope}%
\begin{pgfscope}%
\definecolor{textcolor}{rgb}{0.150000,0.150000,0.150000}%
\pgfsetstrokecolor{textcolor}%
\pgfsetfillcolor{textcolor}%
\pgftext[x=4.178019in,y=0.472591in,,top]{\color{textcolor}\rmfamily\fontsize{24.000000}{28.800000}\selectfont Time (ms)}%
\end{pgfscope}%
\begin{pgfscope}%
\pgfpathrectangle{\pgfqpoint{0.802099in}{0.894194in}}{\pgfqpoint{6.751839in}{4.119217in}}%
\pgfusepath{clip}%
\pgfsetroundcap%
\pgfsetroundjoin%
\pgfsetlinewidth{1.003750pt}%
\definecolor{currentstroke}{rgb}{1.000000,1.000000,1.000000}%
\pgfsetstrokecolor{currentstroke}%
\pgfsetdash{}{0pt}%
\pgfpathmoveto{\pgfqpoint{0.802099in}{1.005008in}}%
\pgfpathlineto{\pgfqpoint{7.553939in}{1.005008in}}%
\pgfusepath{stroke}%
\end{pgfscope}%
\begin{pgfscope}%
\definecolor{textcolor}{rgb}{0.150000,0.150000,0.150000}%
\pgfsetstrokecolor{textcolor}%
\pgfsetfillcolor{textcolor}%
\pgftext[x=0.528147in, y=0.899485in, left, base]{\color{textcolor}\rmfamily\fontsize{20.000000}{24.000000}\selectfont 0}%
\end{pgfscope}%
\begin{pgfscope}%
\pgfpathrectangle{\pgfqpoint{0.802099in}{0.894194in}}{\pgfqpoint{6.751839in}{4.119217in}}%
\pgfusepath{clip}%
\pgfsetroundcap%
\pgfsetroundjoin%
\pgfsetlinewidth{1.003750pt}%
\definecolor{currentstroke}{rgb}{1.000000,1.000000,1.000000}%
\pgfsetstrokecolor{currentstroke}%
\pgfsetdash{}{0pt}%
\pgfpathmoveto{\pgfqpoint{0.802099in}{1.641869in}}%
\pgfpathlineto{\pgfqpoint{7.553939in}{1.641869in}}%
\pgfusepath{stroke}%
\end{pgfscope}%
\begin{pgfscope}%
\definecolor{textcolor}{rgb}{0.150000,0.150000,0.150000}%
\pgfsetstrokecolor{textcolor}%
\pgfsetfillcolor{textcolor}%
\pgftext[x=0.528147in, y=1.536346in, left, base]{\color{textcolor}\rmfamily\fontsize{20.000000}{24.000000}\selectfont 1}%
\end{pgfscope}%
\begin{pgfscope}%
\pgfpathrectangle{\pgfqpoint{0.802099in}{0.894194in}}{\pgfqpoint{6.751839in}{4.119217in}}%
\pgfusepath{clip}%
\pgfsetroundcap%
\pgfsetroundjoin%
\pgfsetlinewidth{1.003750pt}%
\definecolor{currentstroke}{rgb}{1.000000,1.000000,1.000000}%
\pgfsetstrokecolor{currentstroke}%
\pgfsetdash{}{0pt}%
\pgfpathmoveto{\pgfqpoint{0.802099in}{2.278730in}}%
\pgfpathlineto{\pgfqpoint{7.553939in}{2.278730in}}%
\pgfusepath{stroke}%
\end{pgfscope}%
\begin{pgfscope}%
\definecolor{textcolor}{rgb}{0.150000,0.150000,0.150000}%
\pgfsetstrokecolor{textcolor}%
\pgfsetfillcolor{textcolor}%
\pgftext[x=0.528147in, y=2.173207in, left, base]{\color{textcolor}\rmfamily\fontsize{20.000000}{24.000000}\selectfont 2}%
\end{pgfscope}%
\begin{pgfscope}%
\pgfpathrectangle{\pgfqpoint{0.802099in}{0.894194in}}{\pgfqpoint{6.751839in}{4.119217in}}%
\pgfusepath{clip}%
\pgfsetroundcap%
\pgfsetroundjoin%
\pgfsetlinewidth{1.003750pt}%
\definecolor{currentstroke}{rgb}{1.000000,1.000000,1.000000}%
\pgfsetstrokecolor{currentstroke}%
\pgfsetdash{}{0pt}%
\pgfpathmoveto{\pgfqpoint{0.802099in}{2.915591in}}%
\pgfpathlineto{\pgfqpoint{7.553939in}{2.915591in}}%
\pgfusepath{stroke}%
\end{pgfscope}%
\begin{pgfscope}%
\definecolor{textcolor}{rgb}{0.150000,0.150000,0.150000}%
\pgfsetstrokecolor{textcolor}%
\pgfsetfillcolor{textcolor}%
\pgftext[x=0.528147in, y=2.810068in, left, base]{\color{textcolor}\rmfamily\fontsize{20.000000}{24.000000}\selectfont 3}%
\end{pgfscope}%
\begin{pgfscope}%
\pgfpathrectangle{\pgfqpoint{0.802099in}{0.894194in}}{\pgfqpoint{6.751839in}{4.119217in}}%
\pgfusepath{clip}%
\pgfsetroundcap%
\pgfsetroundjoin%
\pgfsetlinewidth{1.003750pt}%
\definecolor{currentstroke}{rgb}{1.000000,1.000000,1.000000}%
\pgfsetstrokecolor{currentstroke}%
\pgfsetdash{}{0pt}%
\pgfpathmoveto{\pgfqpoint{0.802099in}{3.552452in}}%
\pgfpathlineto{\pgfqpoint{7.553939in}{3.552452in}}%
\pgfusepath{stroke}%
\end{pgfscope}%
\begin{pgfscope}%
\definecolor{textcolor}{rgb}{0.150000,0.150000,0.150000}%
\pgfsetstrokecolor{textcolor}%
\pgfsetfillcolor{textcolor}%
\pgftext[x=0.528147in, y=3.446929in, left, base]{\color{textcolor}\rmfamily\fontsize{20.000000}{24.000000}\selectfont 4}%
\end{pgfscope}%
\begin{pgfscope}%
\pgfpathrectangle{\pgfqpoint{0.802099in}{0.894194in}}{\pgfqpoint{6.751839in}{4.119217in}}%
\pgfusepath{clip}%
\pgfsetroundcap%
\pgfsetroundjoin%
\pgfsetlinewidth{1.003750pt}%
\definecolor{currentstroke}{rgb}{1.000000,1.000000,1.000000}%
\pgfsetstrokecolor{currentstroke}%
\pgfsetdash{}{0pt}%
\pgfpathmoveto{\pgfqpoint{0.802099in}{4.189313in}}%
\pgfpathlineto{\pgfqpoint{7.553939in}{4.189313in}}%
\pgfusepath{stroke}%
\end{pgfscope}%
\begin{pgfscope}%
\definecolor{textcolor}{rgb}{0.150000,0.150000,0.150000}%
\pgfsetstrokecolor{textcolor}%
\pgfsetfillcolor{textcolor}%
\pgftext[x=0.528147in, y=4.083790in, left, base]{\color{textcolor}\rmfamily\fontsize{20.000000}{24.000000}\selectfont 5}%
\end{pgfscope}%
\begin{pgfscope}%
\pgfpathrectangle{\pgfqpoint{0.802099in}{0.894194in}}{\pgfqpoint{6.751839in}{4.119217in}}%
\pgfusepath{clip}%
\pgfsetroundcap%
\pgfsetroundjoin%
\pgfsetlinewidth{1.003750pt}%
\definecolor{currentstroke}{rgb}{1.000000,1.000000,1.000000}%
\pgfsetstrokecolor{currentstroke}%
\pgfsetdash{}{0pt}%
\pgfpathmoveto{\pgfqpoint{0.802099in}{4.826174in}}%
\pgfpathlineto{\pgfqpoint{7.553939in}{4.826174in}}%
\pgfusepath{stroke}%
\end{pgfscope}%
\begin{pgfscope}%
\definecolor{textcolor}{rgb}{0.150000,0.150000,0.150000}%
\pgfsetstrokecolor{textcolor}%
\pgfsetfillcolor{textcolor}%
\pgftext[x=0.528147in, y=4.720651in, left, base]{\color{textcolor}\rmfamily\fontsize{20.000000}{24.000000}\selectfont 6}%
\end{pgfscope}%
\begin{pgfscope}%
\definecolor{textcolor}{rgb}{0.150000,0.150000,0.150000}%
\pgfsetstrokecolor{textcolor}%
\pgfsetfillcolor{textcolor}%
\pgftext[x=0.472591in,y=2.953803in,,bottom,rotate=90.000000]{\color{textcolor}\rmfamily\fontsize{24.000000}{28.800000}\selectfont Pulse Amplitude (V)}%
\end{pgfscope}%
\begin{pgfscope}%
\pgfpathrectangle{\pgfqpoint{0.802099in}{0.894194in}}{\pgfqpoint{6.751839in}{4.119217in}}%
\pgfusepath{clip}%
\pgfsetbuttcap%
\pgfsetroundjoin%
\definecolor{currentfill}{rgb}{0.698039,0.133333,0.133333}%
\pgfsetfillcolor{currentfill}%
\pgfsetlinewidth{0.301125pt}%
\definecolor{currentstroke}{rgb}{0.698039,0.133333,0.133333}%
\pgfsetstrokecolor{currentstroke}%
\pgfsetdash{}{0pt}%
\pgfsys@defobject{currentmarker}{\pgfqpoint{-0.048611in}{-0.048611in}}{\pgfqpoint{0.048611in}{0.048611in}}{%
\pgfpathmoveto{\pgfqpoint{0.000000in}{-0.048611in}}%
\pgfpathcurveto{\pgfqpoint{0.012892in}{-0.048611in}}{\pgfqpoint{0.025257in}{-0.043489in}}{\pgfqpoint{0.034373in}{-0.034373in}}%
\pgfpathcurveto{\pgfqpoint{0.043489in}{-0.025257in}}{\pgfqpoint{0.048611in}{-0.012892in}}{\pgfqpoint{0.048611in}{0.000000in}}%
\pgfpathcurveto{\pgfqpoint{0.048611in}{0.012892in}}{\pgfqpoint{0.043489in}{0.025257in}}{\pgfqpoint{0.034373in}{0.034373in}}%
\pgfpathcurveto{\pgfqpoint{0.025257in}{0.043489in}}{\pgfqpoint{0.012892in}{0.048611in}}{\pgfqpoint{0.000000in}{0.048611in}}%
\pgfpathcurveto{\pgfqpoint{-0.012892in}{0.048611in}}{\pgfqpoint{-0.025257in}{0.043489in}}{\pgfqpoint{-0.034373in}{0.034373in}}%
\pgfpathcurveto{\pgfqpoint{-0.043489in}{0.025257in}}{\pgfqpoint{-0.048611in}{0.012892in}}{\pgfqpoint{-0.048611in}{0.000000in}}%
\pgfpathcurveto{\pgfqpoint{-0.048611in}{-0.012892in}}{\pgfqpoint{-0.043489in}{-0.025257in}}{\pgfqpoint{-0.034373in}{-0.034373in}}%
\pgfpathcurveto{\pgfqpoint{-0.025257in}{-0.043489in}}{\pgfqpoint{-0.012892in}{-0.048611in}}{\pgfqpoint{0.000000in}{-0.048611in}}%
\pgfpathlineto{\pgfqpoint{0.000000in}{-0.048611in}}%
\pgfpathclose%
\pgfusepath{stroke,fill}%
}%
\begin{pgfscope}%
\pgfsys@transformshift{1.109001in}{4.826174in}%
\pgfsys@useobject{currentmarker}{}%
\end{pgfscope}%
\begin{pgfscope}%
\pgfsys@transformshift{1.722805in}{4.036467in}%
\pgfsys@useobject{currentmarker}{}%
\end{pgfscope}%
\begin{pgfscope}%
\pgfsys@transformshift{2.336608in}{3.348657in}%
\pgfsys@useobject{currentmarker}{}%
\end{pgfscope}%
\begin{pgfscope}%
\pgfsys@transformshift{2.950412in}{2.864642in}%
\pgfsys@useobject{currentmarker}{}%
\end{pgfscope}%
\begin{pgfscope}%
\pgfsys@transformshift{3.686976in}{2.431577in}%
\pgfsys@useobject{currentmarker}{}%
\end{pgfscope}%
\begin{pgfscope}%
\pgfsys@transformshift{4.178019in}{2.125884in}%
\pgfsys@useobject{currentmarker}{}%
\end{pgfscope}%
\begin{pgfscope}%
\pgfsys@transformshift{4.791823in}{1.871139in}%
\pgfsys@useobject{currentmarker}{}%
\end{pgfscope}%
\begin{pgfscope}%
\pgfsys@transformshift{5.405626in}{1.616395in}%
\pgfsys@useobject{currentmarker}{}%
\end{pgfscope}%
\begin{pgfscope}%
\pgfsys@transformshift{6.019430in}{1.387125in}%
\pgfsys@useobject{currentmarker}{}%
\end{pgfscope}%
\begin{pgfscope}%
\pgfsys@transformshift{6.633233in}{1.285227in}%
\pgfsys@useobject{currentmarker}{}%
\end{pgfscope}%
\begin{pgfscope}%
\pgfsys@transformshift{7.247037in}{1.081432in}%
\pgfsys@useobject{currentmarker}{}%
\end{pgfscope}%
\end{pgfscope}%
\begin{pgfscope}%
\pgfpathrectangle{\pgfqpoint{0.802099in}{0.894194in}}{\pgfqpoint{6.751839in}{4.119217in}}%
\pgfusepath{clip}%
\pgfsetroundcap%
\pgfsetroundjoin%
\pgfsetlinewidth{1.756562pt}%
\definecolor{currentstroke}{rgb}{0.298039,0.447059,0.690196}%
\pgfsetstrokecolor{currentstroke}%
\pgfsetdash{}{0pt}%
\pgfpathmoveto{\pgfqpoint{1.109001in}{4.802008in}}%
\pgfpathlineto{\pgfqpoint{1.171002in}{4.717468in}}%
\pgfpathlineto{\pgfqpoint{1.233002in}{4.634604in}}%
\pgfpathlineto{\pgfqpoint{1.295002in}{4.553383in}}%
\pgfpathlineto{\pgfqpoint{1.357003in}{4.473773in}}%
\pgfpathlineto{\pgfqpoint{1.419003in}{4.395740in}}%
\pgfpathlineto{\pgfqpoint{1.481003in}{4.319255in}}%
\pgfpathlineto{\pgfqpoint{1.543004in}{4.244286in}}%
\pgfpathlineto{\pgfqpoint{1.605004in}{4.170804in}}%
\pgfpathlineto{\pgfqpoint{1.667004in}{4.098778in}}%
\pgfpathlineto{\pgfqpoint{1.729005in}{4.028180in}}%
\pgfpathlineto{\pgfqpoint{1.791005in}{3.958982in}}%
\pgfpathlineto{\pgfqpoint{1.853006in}{3.891156in}}%
\pgfpathlineto{\pgfqpoint{1.915006in}{3.824674in}}%
\pgfpathlineto{\pgfqpoint{1.977006in}{3.759511in}}%
\pgfpathlineto{\pgfqpoint{2.039007in}{3.695639in}}%
\pgfpathlineto{\pgfqpoint{2.101007in}{3.633034in}}%
\pgfpathlineto{\pgfqpoint{2.163007in}{3.571670in}}%
\pgfpathlineto{\pgfqpoint{2.225008in}{3.511522in}}%
\pgfpathlineto{\pgfqpoint{2.287008in}{3.452567in}}%
\pgfpathlineto{\pgfqpoint{2.349008in}{3.394781in}}%
\pgfpathlineto{\pgfqpoint{2.411009in}{3.338141in}}%
\pgfpathlineto{\pgfqpoint{2.473009in}{3.282623in}}%
\pgfpathlineto{\pgfqpoint{2.535010in}{3.228206in}}%
\pgfpathlineto{\pgfqpoint{2.597010in}{3.174868in}}%
\pgfpathlineto{\pgfqpoint{2.659010in}{3.122587in}}%
\pgfpathlineto{\pgfqpoint{2.721011in}{3.071343in}}%
\pgfpathlineto{\pgfqpoint{2.783011in}{3.021115in}}%
\pgfpathlineto{\pgfqpoint{2.845011in}{2.971883in}}%
\pgfpathlineto{\pgfqpoint{2.907012in}{2.923626in}}%
\pgfpathlineto{\pgfqpoint{2.969012in}{2.876327in}}%
\pgfpathlineto{\pgfqpoint{3.031012in}{2.829965in}}%
\pgfpathlineto{\pgfqpoint{3.093013in}{2.784522in}}%
\pgfpathlineto{\pgfqpoint{3.155013in}{2.739980in}}%
\pgfpathlineto{\pgfqpoint{3.217014in}{2.696322in}}%
\pgfpathlineto{\pgfqpoint{3.279014in}{2.653528in}}%
\pgfpathlineto{\pgfqpoint{3.341014in}{2.611584in}}%
\pgfpathlineto{\pgfqpoint{3.403015in}{2.570470in}}%
\pgfpathlineto{\pgfqpoint{3.465015in}{2.530172in}}%
\pgfpathlineto{\pgfqpoint{3.527015in}{2.490673in}}%
\pgfpathlineto{\pgfqpoint{3.589016in}{2.451957in}}%
\pgfpathlineto{\pgfqpoint{3.651016in}{2.414009in}}%
\pgfpathlineto{\pgfqpoint{3.713016in}{2.376813in}}%
\pgfpathlineto{\pgfqpoint{3.775017in}{2.340354in}}%
\pgfpathlineto{\pgfqpoint{3.837017in}{2.304618in}}%
\pgfpathlineto{\pgfqpoint{3.899018in}{2.269591in}}%
\pgfpathlineto{\pgfqpoint{3.961018in}{2.235258in}}%
\pgfpathlineto{\pgfqpoint{4.023018in}{2.201606in}}%
\pgfpathlineto{\pgfqpoint{4.085019in}{2.168621in}}%
\pgfpathlineto{\pgfqpoint{4.147019in}{2.136289in}}%
\pgfpathlineto{\pgfqpoint{4.209019in}{2.104599in}}%
\pgfpathlineto{\pgfqpoint{4.271020in}{2.073537in}}%
\pgfpathlineto{\pgfqpoint{4.333020in}{2.043091in}}%
\pgfpathlineto{\pgfqpoint{4.395020in}{2.013249in}}%
\pgfpathlineto{\pgfqpoint{4.457021in}{1.983998in}}%
\pgfpathlineto{\pgfqpoint{4.519021in}{1.955327in}}%
\pgfpathlineto{\pgfqpoint{4.581021in}{1.927225in}}%
\pgfpathlineto{\pgfqpoint{4.643022in}{1.899679in}}%
\pgfpathlineto{\pgfqpoint{4.705022in}{1.872680in}}%
\pgfpathlineto{\pgfqpoint{4.767023in}{1.846216in}}%
\pgfpathlineto{\pgfqpoint{4.829023in}{1.820277in}}%
\pgfpathlineto{\pgfqpoint{4.891023in}{1.794852in}}%
\pgfpathlineto{\pgfqpoint{4.953024in}{1.769931in}}%
\pgfpathlineto{\pgfqpoint{5.015024in}{1.745504in}}%
\pgfpathlineto{\pgfqpoint{5.077024in}{1.721562in}}%
\pgfpathlineto{\pgfqpoint{5.139025in}{1.698094in}}%
\pgfpathlineto{\pgfqpoint{5.201025in}{1.675091in}}%
\pgfpathlineto{\pgfqpoint{5.263025in}{1.652545in}}%
\pgfpathlineto{\pgfqpoint{5.325026in}{1.630445in}}%
\pgfpathlineto{\pgfqpoint{5.387026in}{1.608784in}}%
\pgfpathlineto{\pgfqpoint{5.449027in}{1.587552in}}%
\pgfpathlineto{\pgfqpoint{5.511027in}{1.566741in}}%
\pgfpathlineto{\pgfqpoint{5.573027in}{1.546342in}}%
\pgfpathlineto{\pgfqpoint{5.635028in}{1.526348in}}%
\pgfpathlineto{\pgfqpoint{5.697028in}{1.506750in}}%
\pgfpathlineto{\pgfqpoint{5.759028in}{1.487541in}}%
\pgfpathlineto{\pgfqpoint{5.821029in}{1.468713in}}%
\pgfpathlineto{\pgfqpoint{5.883029in}{1.450258in}}%
\pgfpathlineto{\pgfqpoint{5.945029in}{1.432169in}}%
\pgfpathlineto{\pgfqpoint{6.007030in}{1.414438in}}%
\pgfpathlineto{\pgfqpoint{6.069030in}{1.397059in}}%
\pgfpathlineto{\pgfqpoint{6.131031in}{1.380025in}}%
\pgfpathlineto{\pgfqpoint{6.193031in}{1.363328in}}%
\pgfpathlineto{\pgfqpoint{6.255031in}{1.346963in}}%
\pgfpathlineto{\pgfqpoint{6.317032in}{1.330921in}}%
\pgfpathlineto{\pgfqpoint{6.379032in}{1.315198in}}%
\pgfpathlineto{\pgfqpoint{6.441032in}{1.299787in}}%
\pgfpathlineto{\pgfqpoint{6.503033in}{1.284681in}}%
\pgfpathlineto{\pgfqpoint{6.565033in}{1.269874in}}%
\pgfpathlineto{\pgfqpoint{6.627033in}{1.255361in}}%
\pgfpathlineto{\pgfqpoint{6.689034in}{1.241136in}}%
\pgfpathlineto{\pgfqpoint{6.751034in}{1.227193in}}%
\pgfpathlineto{\pgfqpoint{6.813035in}{1.213526in}}%
\pgfpathlineto{\pgfqpoint{6.875035in}{1.200131in}}%
\pgfpathlineto{\pgfqpoint{6.937035in}{1.187001in}}%
\pgfpathlineto{\pgfqpoint{6.999036in}{1.174131in}}%
\pgfpathlineto{\pgfqpoint{7.061036in}{1.161516in}}%
\pgfpathlineto{\pgfqpoint{7.123036in}{1.149151in}}%
\pgfpathlineto{\pgfqpoint{7.185037in}{1.137032in}}%
\pgfpathlineto{\pgfqpoint{7.247037in}{1.125153in}}%
\pgfusepath{stroke}%
\end{pgfscope}%
\begin{pgfscope}%
\pgfsetrectcap%
\pgfsetmiterjoin%
\pgfsetlinewidth{0.000000pt}%
\definecolor{currentstroke}{rgb}{1.000000,1.000000,1.000000}%
\pgfsetstrokecolor{currentstroke}%
\pgfsetdash{}{0pt}%
\pgfpathmoveto{\pgfqpoint{0.802099in}{0.894194in}}%
\pgfpathlineto{\pgfqpoint{0.802099in}{5.013411in}}%
\pgfusepath{}%
\end{pgfscope}%
\begin{pgfscope}%
\pgfsetrectcap%
\pgfsetmiterjoin%
\pgfsetlinewidth{0.000000pt}%
\definecolor{currentstroke}{rgb}{1.000000,1.000000,1.000000}%
\pgfsetstrokecolor{currentstroke}%
\pgfsetdash{}{0pt}%
\pgfpathmoveto{\pgfqpoint{7.553939in}{0.894194in}}%
\pgfpathlineto{\pgfqpoint{7.553939in}{5.013411in}}%
\pgfusepath{}%
\end{pgfscope}%
\begin{pgfscope}%
\pgfsetrectcap%
\pgfsetmiterjoin%
\pgfsetlinewidth{0.000000pt}%
\definecolor{currentstroke}{rgb}{1.000000,1.000000,1.000000}%
\pgfsetstrokecolor{currentstroke}%
\pgfsetdash{}{0pt}%
\pgfpathmoveto{\pgfqpoint{0.802099in}{0.894194in}}%
\pgfpathlineto{\pgfqpoint{7.553939in}{0.894194in}}%
\pgfusepath{}%
\end{pgfscope}%
\begin{pgfscope}%
\pgfsetrectcap%
\pgfsetmiterjoin%
\pgfsetlinewidth{0.000000pt}%
\definecolor{currentstroke}{rgb}{1.000000,1.000000,1.000000}%
\pgfsetstrokecolor{currentstroke}%
\pgfsetdash{}{0pt}%
\pgfpathmoveto{\pgfqpoint{0.802099in}{5.013411in}}%
\pgfpathlineto{\pgfqpoint{7.553939in}{5.013411in}}%
\pgfusepath{}%
\end{pgfscope}%
\begin{pgfscope}%
\definecolor{textcolor}{rgb}{0.150000,0.150000,0.150000}%
\pgfsetstrokecolor{textcolor}%
\pgfsetfillcolor{textcolor}%
\pgftext[x=3.625596in, y=3.728077in, left, base]{\color{textcolor}\rmfamily\fontsize{20.000000}{24.000000}\selectfont M\(\displaystyle _{x,y}\)(t) = 8.84\(\displaystyle exp[\frac{-2t}{T_2}]+\)-0.73}%
\end{pgfscope}%
\begin{pgfscope}%
\definecolor{textcolor}{rgb}{0.150000,0.150000,0.150000}%
\pgfsetstrokecolor{textcolor}%
\pgfsetfillcolor{textcolor}%
\pgftext[x=3.625596in, y=3.234022in, left, base]{\color{textcolor}\rmfamily\fontsize{20.000000}{24.000000}\selectfont T\(\displaystyle _2\) = 2.0176\(\displaystyle \pm\)0.0792 ms}%
\end{pgfscope}%
\begin{pgfscope}%
\definecolor{textcolor}{rgb}{0.150000,0.150000,0.150000}%
\pgfsetstrokecolor{textcolor}%
\pgfsetfillcolor{textcolor}%
\pgftext[x=4.178019in,y=5.096745in,,base]{\color{textcolor}\rmfamily\fontsize{24.000000}{28.800000}\selectfont MG T\(\displaystyle _2\) measurement of 1 M CuSO\(\displaystyle _4\)}%
\end{pgfscope}%
\begin{pgfscope}%
\pgfsetbuttcap%
\pgfsetroundjoin%
\definecolor{currentfill}{rgb}{0.698039,0.133333,0.133333}%
\pgfsetfillcolor{currentfill}%
\pgfsetlinewidth{0.301125pt}%
\definecolor{currentstroke}{rgb}{0.698039,0.133333,0.133333}%
\pgfsetstrokecolor{currentstroke}%
\pgfsetdash{}{0pt}%
\pgfsys@defobject{currentmarker}{\pgfqpoint{-0.048611in}{-0.048611in}}{\pgfqpoint{0.048611in}{0.048611in}}{%
\pgfpathmoveto{\pgfqpoint{0.000000in}{-0.048611in}}%
\pgfpathcurveto{\pgfqpoint{0.012892in}{-0.048611in}}{\pgfqpoint{0.025257in}{-0.043489in}}{\pgfqpoint{0.034373in}{-0.034373in}}%
\pgfpathcurveto{\pgfqpoint{0.043489in}{-0.025257in}}{\pgfqpoint{0.048611in}{-0.012892in}}{\pgfqpoint{0.048611in}{0.000000in}}%
\pgfpathcurveto{\pgfqpoint{0.048611in}{0.012892in}}{\pgfqpoint{0.043489in}{0.025257in}}{\pgfqpoint{0.034373in}{0.034373in}}%
\pgfpathcurveto{\pgfqpoint{0.025257in}{0.043489in}}{\pgfqpoint{0.012892in}{0.048611in}}{\pgfqpoint{0.000000in}{0.048611in}}%
\pgfpathcurveto{\pgfqpoint{-0.012892in}{0.048611in}}{\pgfqpoint{-0.025257in}{0.043489in}}{\pgfqpoint{-0.034373in}{0.034373in}}%
\pgfpathcurveto{\pgfqpoint{-0.043489in}{0.025257in}}{\pgfqpoint{-0.048611in}{0.012892in}}{\pgfqpoint{-0.048611in}{0.000000in}}%
\pgfpathcurveto{\pgfqpoint{-0.048611in}{-0.012892in}}{\pgfqpoint{-0.043489in}{-0.025257in}}{\pgfqpoint{-0.034373in}{-0.034373in}}%
\pgfpathcurveto{\pgfqpoint{-0.025257in}{-0.043489in}}{\pgfqpoint{-0.012892in}{-0.048611in}}{\pgfqpoint{0.000000in}{-0.048611in}}%
\pgfpathlineto{\pgfqpoint{0.000000in}{-0.048611in}}%
\pgfpathclose%
\pgfusepath{stroke,fill}%
}%
\begin{pgfscope}%
\pgfsys@transformshift{5.023069in}{4.625282in}%
\pgfsys@useobject{currentmarker}{}%
\end{pgfscope}%
\end{pgfscope}%
\begin{pgfscope}%
\definecolor{textcolor}{rgb}{0.150000,0.150000,0.150000}%
\pgfsetstrokecolor{textcolor}%
\pgfsetfillcolor{textcolor}%
\pgftext[x=5.523069in,y=4.552365in,left,base]{\color{textcolor}\rmfamily\fontsize{20.000000}{24.000000}\selectfont Data}%
\end{pgfscope}%
\begin{pgfscope}%
\pgfsetroundcap%
\pgfsetroundjoin%
\pgfsetlinewidth{1.756562pt}%
\definecolor{currentstroke}{rgb}{0.298039,0.447059,0.690196}%
\pgfsetstrokecolor{currentstroke}%
\pgfsetdash{}{0pt}%
\pgfpathmoveto{\pgfqpoint{4.745291in}{4.241873in}}%
\pgfpathlineto{\pgfqpoint{5.023069in}{4.241873in}}%
\pgfpathlineto{\pgfqpoint{5.300846in}{4.241873in}}%
\pgfusepath{stroke}%
\end{pgfscope}%
\begin{pgfscope}%
\definecolor{textcolor}{rgb}{0.150000,0.150000,0.150000}%
\pgfsetstrokecolor{textcolor}%
\pgfsetfillcolor{textcolor}%
\pgftext[x=5.523069in,y=4.144651in,left,base]{\color{textcolor}\rmfamily\fontsize{20.000000}{24.000000}\selectfont Fitted Curve}%
\end{pgfscope}%
\end{pgfpicture}%
\makeatother%
\endgroup%
}
    \caption{T2 measurement of 1.0 M CuSO$_{4} \boldsymbol{\cdot} $H$_2$O using the Meiboom-Gill pulse sequence }
    \label{fig:1d0_t2}
  \end{figure}

  %%%%%%%%%%%%%%%%%%%%%%%%%%%%%%%%%
  %%%%%%%%%% CuSo4 T1 %%%%%%%%%%%%%
  %%%%%%%%%%%%%%%%%%%%%%%%%%%%%%%%%


  \begin{figure}[h]
    \resizebox{0.45\textwidth}{!}{%% Creator: Matplotlib, PGF backend
%%
%% To include the figure in your LaTeX document, write
%%   \input{<filename>.pgf}
%%
%% Make sure the required packages are loaded in your preamble
%%   \usepackage{pgf}
%%
%% Also ensure that all the required font packages are loaded; for instance,
%% the lmodern package is sometimes necessary when using math font.
%%   \usepackage{lmodern}
%%
%% Figures using additional raster images can only be included by \input if
%% they are in the same directory as the main LaTeX file. For loading figures
%% from other directories you can use the `import` package
%%   \usepackage{import}
%%
%% and then include the figures with
%%   \import{<path to file>}{<filename>.pgf}
%%
%% Matplotlib used the following preamble
%%   \usepackage{fontspec}
%%   \setmainfont{DejaVuSerif.ttf}[Path=\detokenize{/usr/share/matplotlib/mpl-data/fonts/ttf/}]
%%   \setsansfont{DejaVuSans.ttf}[Path=\detokenize{/usr/share/matplotlib/mpl-data/fonts/ttf/}]
%%   \setmonofont{DejaVuSansMono.ttf}[Path=\detokenize{/usr/share/matplotlib/mpl-data/fonts/ttf/}]
%%
\begingroup%
\makeatletter%
\begin{pgfpicture}%
\pgfpathrectangle{\pgfpointorigin}{\pgfqpoint{8.000000in}{5.500000in}}%
\pgfusepath{use as bounding box, clip}%
\begin{pgfscope}%
\pgfsetbuttcap%
\pgfsetmiterjoin%
\definecolor{currentfill}{rgb}{1.000000,1.000000,1.000000}%
\pgfsetfillcolor{currentfill}%
\pgfsetlinewidth{0.000000pt}%
\definecolor{currentstroke}{rgb}{1.000000,1.000000,1.000000}%
\pgfsetstrokecolor{currentstroke}%
\pgfsetdash{}{0pt}%
\pgfpathmoveto{\pgfqpoint{0.000000in}{0.000000in}}%
\pgfpathlineto{\pgfqpoint{8.000000in}{0.000000in}}%
\pgfpathlineto{\pgfqpoint{8.000000in}{5.500000in}}%
\pgfpathlineto{\pgfqpoint{0.000000in}{5.500000in}}%
\pgfpathlineto{\pgfqpoint{0.000000in}{0.000000in}}%
\pgfpathclose%
\pgfusepath{fill}%
\end{pgfscope}%
\begin{pgfscope}%
\pgfsetbuttcap%
\pgfsetmiterjoin%
\definecolor{currentfill}{rgb}{0.917647,0.917647,0.949020}%
\pgfsetfillcolor{currentfill}%
\pgfsetlinewidth{0.000000pt}%
\definecolor{currentstroke}{rgb}{0.000000,0.000000,0.000000}%
\pgfsetstrokecolor{currentstroke}%
\pgfsetstrokeopacity{0.000000}%
\pgfsetdash{}{0pt}%
\pgfpathmoveto{\pgfqpoint{1.666655in}{0.898915in}}%
\pgfpathlineto{\pgfqpoint{7.850000in}{0.898915in}}%
\pgfpathlineto{\pgfqpoint{7.850000in}{5.013411in}}%
\pgfpathlineto{\pgfqpoint{1.666655in}{5.013411in}}%
\pgfpathlineto{\pgfqpoint{1.666655in}{0.898915in}}%
\pgfpathclose%
\pgfusepath{fill}%
\end{pgfscope}%
\begin{pgfscope}%
\pgfpathrectangle{\pgfqpoint{1.666655in}{0.898915in}}{\pgfqpoint{6.183345in}{4.114497in}}%
\pgfusepath{clip}%
\pgfsetroundcap%
\pgfsetroundjoin%
\pgfsetlinewidth{1.003750pt}%
\definecolor{currentstroke}{rgb}{1.000000,1.000000,1.000000}%
\pgfsetstrokecolor{currentstroke}%
\pgfsetdash{}{0pt}%
\pgfpathmoveto{\pgfqpoint{1.947716in}{0.898915in}}%
\pgfpathlineto{\pgfqpoint{1.947716in}{5.013411in}}%
\pgfusepath{stroke}%
\end{pgfscope}%
\begin{pgfscope}%
\definecolor{textcolor}{rgb}{0.150000,0.150000,0.150000}%
\pgfsetstrokecolor{textcolor}%
\pgfsetfillcolor{textcolor}%
\pgftext[x=1.947716in,y=0.801692in,,top]{\color{textcolor}\rmfamily\fontsize{20.000000}{24.000000}\selectfont 0.0}%
\end{pgfscope}%
\begin{pgfscope}%
\pgfpathrectangle{\pgfqpoint{1.666655in}{0.898915in}}{\pgfqpoint{6.183345in}{4.114497in}}%
\pgfusepath{clip}%
\pgfsetroundcap%
\pgfsetroundjoin%
\pgfsetlinewidth{1.003750pt}%
\definecolor{currentstroke}{rgb}{1.000000,1.000000,1.000000}%
\pgfsetstrokecolor{currentstroke}%
\pgfsetdash{}{0pt}%
\pgfpathmoveto{\pgfqpoint{3.071961in}{0.898915in}}%
\pgfpathlineto{\pgfqpoint{3.071961in}{5.013411in}}%
\pgfusepath{stroke}%
\end{pgfscope}%
\begin{pgfscope}%
\definecolor{textcolor}{rgb}{0.150000,0.150000,0.150000}%
\pgfsetstrokecolor{textcolor}%
\pgfsetfillcolor{textcolor}%
\pgftext[x=3.071961in,y=0.801692in,,top]{\color{textcolor}\rmfamily\fontsize{20.000000}{24.000000}\selectfont 0.2}%
\end{pgfscope}%
\begin{pgfscope}%
\pgfpathrectangle{\pgfqpoint{1.666655in}{0.898915in}}{\pgfqpoint{6.183345in}{4.114497in}}%
\pgfusepath{clip}%
\pgfsetroundcap%
\pgfsetroundjoin%
\pgfsetlinewidth{1.003750pt}%
\definecolor{currentstroke}{rgb}{1.000000,1.000000,1.000000}%
\pgfsetstrokecolor{currentstroke}%
\pgfsetdash{}{0pt}%
\pgfpathmoveto{\pgfqpoint{4.196205in}{0.898915in}}%
\pgfpathlineto{\pgfqpoint{4.196205in}{5.013411in}}%
\pgfusepath{stroke}%
\end{pgfscope}%
\begin{pgfscope}%
\definecolor{textcolor}{rgb}{0.150000,0.150000,0.150000}%
\pgfsetstrokecolor{textcolor}%
\pgfsetfillcolor{textcolor}%
\pgftext[x=4.196205in,y=0.801692in,,top]{\color{textcolor}\rmfamily\fontsize{20.000000}{24.000000}\selectfont 0.4}%
\end{pgfscope}%
\begin{pgfscope}%
\pgfpathrectangle{\pgfqpoint{1.666655in}{0.898915in}}{\pgfqpoint{6.183345in}{4.114497in}}%
\pgfusepath{clip}%
\pgfsetroundcap%
\pgfsetroundjoin%
\pgfsetlinewidth{1.003750pt}%
\definecolor{currentstroke}{rgb}{1.000000,1.000000,1.000000}%
\pgfsetstrokecolor{currentstroke}%
\pgfsetdash{}{0pt}%
\pgfpathmoveto{\pgfqpoint{5.320450in}{0.898915in}}%
\pgfpathlineto{\pgfqpoint{5.320450in}{5.013411in}}%
\pgfusepath{stroke}%
\end{pgfscope}%
\begin{pgfscope}%
\definecolor{textcolor}{rgb}{0.150000,0.150000,0.150000}%
\pgfsetstrokecolor{textcolor}%
\pgfsetfillcolor{textcolor}%
\pgftext[x=5.320450in,y=0.801692in,,top]{\color{textcolor}\rmfamily\fontsize{20.000000}{24.000000}\selectfont 0.6}%
\end{pgfscope}%
\begin{pgfscope}%
\pgfpathrectangle{\pgfqpoint{1.666655in}{0.898915in}}{\pgfqpoint{6.183345in}{4.114497in}}%
\pgfusepath{clip}%
\pgfsetroundcap%
\pgfsetroundjoin%
\pgfsetlinewidth{1.003750pt}%
\definecolor{currentstroke}{rgb}{1.000000,1.000000,1.000000}%
\pgfsetstrokecolor{currentstroke}%
\pgfsetdash{}{0pt}%
\pgfpathmoveto{\pgfqpoint{6.444694in}{0.898915in}}%
\pgfpathlineto{\pgfqpoint{6.444694in}{5.013411in}}%
\pgfusepath{stroke}%
\end{pgfscope}%
\begin{pgfscope}%
\definecolor{textcolor}{rgb}{0.150000,0.150000,0.150000}%
\pgfsetstrokecolor{textcolor}%
\pgfsetfillcolor{textcolor}%
\pgftext[x=6.444694in,y=0.801692in,,top]{\color{textcolor}\rmfamily\fontsize{20.000000}{24.000000}\selectfont 0.8}%
\end{pgfscope}%
\begin{pgfscope}%
\pgfpathrectangle{\pgfqpoint{1.666655in}{0.898915in}}{\pgfqpoint{6.183345in}{4.114497in}}%
\pgfusepath{clip}%
\pgfsetroundcap%
\pgfsetroundjoin%
\pgfsetlinewidth{1.003750pt}%
\definecolor{currentstroke}{rgb}{1.000000,1.000000,1.000000}%
\pgfsetstrokecolor{currentstroke}%
\pgfsetdash{}{0pt}%
\pgfpathmoveto{\pgfqpoint{7.568939in}{0.898915in}}%
\pgfpathlineto{\pgfqpoint{7.568939in}{5.013411in}}%
\pgfusepath{stroke}%
\end{pgfscope}%
\begin{pgfscope}%
\definecolor{textcolor}{rgb}{0.150000,0.150000,0.150000}%
\pgfsetstrokecolor{textcolor}%
\pgfsetfillcolor{textcolor}%
\pgftext[x=7.568939in,y=0.801692in,,top]{\color{textcolor}\rmfamily\fontsize{20.000000}{24.000000}\selectfont 1.0}%
\end{pgfscope}%
\begin{pgfscope}%
\definecolor{textcolor}{rgb}{0.150000,0.150000,0.150000}%
\pgfsetstrokecolor{textcolor}%
\pgfsetfillcolor{textcolor}%
\pgftext[x=4.758327in,y=0.477311in,,top]{\color{textcolor}\rmfamily\fontsize{24.000000}{28.800000}\selectfont Molarity (mol/L)}%
\end{pgfscope}%
\begin{pgfscope}%
\pgfpathrectangle{\pgfqpoint{1.666655in}{0.898915in}}{\pgfqpoint{6.183345in}{4.114497in}}%
\pgfusepath{clip}%
\pgfsetroundcap%
\pgfsetroundjoin%
\pgfsetlinewidth{1.003750pt}%
\definecolor{currentstroke}{rgb}{1.000000,1.000000,1.000000}%
\pgfsetstrokecolor{currentstroke}%
\pgfsetdash{}{0pt}%
\pgfpathmoveto{\pgfqpoint{1.666655in}{1.125811in}}%
\pgfpathlineto{\pgfqpoint{7.850000in}{1.125811in}}%
\pgfusepath{stroke}%
\end{pgfscope}%
\begin{pgfscope}%
\definecolor{textcolor}{rgb}{0.150000,0.150000,0.150000}%
\pgfsetstrokecolor{textcolor}%
\pgfsetfillcolor{textcolor}%
\pgftext[x=0.950943in, y=1.020288in, left, base]{\color{textcolor}\rmfamily\fontsize{20.000000}{24.000000}\selectfont 0.00}%
\end{pgfscope}%
\begin{pgfscope}%
\pgfpathrectangle{\pgfqpoint{1.666655in}{0.898915in}}{\pgfqpoint{6.183345in}{4.114497in}}%
\pgfusepath{clip}%
\pgfsetroundcap%
\pgfsetroundjoin%
\pgfsetlinewidth{1.003750pt}%
\definecolor{currentstroke}{rgb}{1.000000,1.000000,1.000000}%
\pgfsetstrokecolor{currentstroke}%
\pgfsetdash{}{0pt}%
\pgfpathmoveto{\pgfqpoint{1.666655in}{1.776384in}}%
\pgfpathlineto{\pgfqpoint{7.850000in}{1.776384in}}%
\pgfusepath{stroke}%
\end{pgfscope}%
\begin{pgfscope}%
\definecolor{textcolor}{rgb}{0.150000,0.150000,0.150000}%
\pgfsetstrokecolor{textcolor}%
\pgfsetfillcolor{textcolor}%
\pgftext[x=0.950943in, y=1.670861in, left, base]{\color{textcolor}\rmfamily\fontsize{20.000000}{24.000000}\selectfont 0.25}%
\end{pgfscope}%
\begin{pgfscope}%
\pgfpathrectangle{\pgfqpoint{1.666655in}{0.898915in}}{\pgfqpoint{6.183345in}{4.114497in}}%
\pgfusepath{clip}%
\pgfsetroundcap%
\pgfsetroundjoin%
\pgfsetlinewidth{1.003750pt}%
\definecolor{currentstroke}{rgb}{1.000000,1.000000,1.000000}%
\pgfsetstrokecolor{currentstroke}%
\pgfsetdash{}{0pt}%
\pgfpathmoveto{\pgfqpoint{1.666655in}{2.426957in}}%
\pgfpathlineto{\pgfqpoint{7.850000in}{2.426957in}}%
\pgfusepath{stroke}%
\end{pgfscope}%
\begin{pgfscope}%
\definecolor{textcolor}{rgb}{0.150000,0.150000,0.150000}%
\pgfsetstrokecolor{textcolor}%
\pgfsetfillcolor{textcolor}%
\pgftext[x=0.950943in, y=2.321434in, left, base]{\color{textcolor}\rmfamily\fontsize{20.000000}{24.000000}\selectfont 0.50}%
\end{pgfscope}%
\begin{pgfscope}%
\pgfpathrectangle{\pgfqpoint{1.666655in}{0.898915in}}{\pgfqpoint{6.183345in}{4.114497in}}%
\pgfusepath{clip}%
\pgfsetroundcap%
\pgfsetroundjoin%
\pgfsetlinewidth{1.003750pt}%
\definecolor{currentstroke}{rgb}{1.000000,1.000000,1.000000}%
\pgfsetstrokecolor{currentstroke}%
\pgfsetdash{}{0pt}%
\pgfpathmoveto{\pgfqpoint{1.666655in}{3.077530in}}%
\pgfpathlineto{\pgfqpoint{7.850000in}{3.077530in}}%
\pgfusepath{stroke}%
\end{pgfscope}%
\begin{pgfscope}%
\definecolor{textcolor}{rgb}{0.150000,0.150000,0.150000}%
\pgfsetstrokecolor{textcolor}%
\pgfsetfillcolor{textcolor}%
\pgftext[x=0.950943in, y=2.972007in, left, base]{\color{textcolor}\rmfamily\fontsize{20.000000}{24.000000}\selectfont 0.75}%
\end{pgfscope}%
\begin{pgfscope}%
\pgfpathrectangle{\pgfqpoint{1.666655in}{0.898915in}}{\pgfqpoint{6.183345in}{4.114497in}}%
\pgfusepath{clip}%
\pgfsetroundcap%
\pgfsetroundjoin%
\pgfsetlinewidth{1.003750pt}%
\definecolor{currentstroke}{rgb}{1.000000,1.000000,1.000000}%
\pgfsetstrokecolor{currentstroke}%
\pgfsetdash{}{0pt}%
\pgfpathmoveto{\pgfqpoint{1.666655in}{3.728102in}}%
\pgfpathlineto{\pgfqpoint{7.850000in}{3.728102in}}%
\pgfusepath{stroke}%
\end{pgfscope}%
\begin{pgfscope}%
\definecolor{textcolor}{rgb}{0.150000,0.150000,0.150000}%
\pgfsetstrokecolor{textcolor}%
\pgfsetfillcolor{textcolor}%
\pgftext[x=0.950943in, y=3.622579in, left, base]{\color{textcolor}\rmfamily\fontsize{20.000000}{24.000000}\selectfont 1.00}%
\end{pgfscope}%
\begin{pgfscope}%
\pgfpathrectangle{\pgfqpoint{1.666655in}{0.898915in}}{\pgfqpoint{6.183345in}{4.114497in}}%
\pgfusepath{clip}%
\pgfsetroundcap%
\pgfsetroundjoin%
\pgfsetlinewidth{1.003750pt}%
\definecolor{currentstroke}{rgb}{1.000000,1.000000,1.000000}%
\pgfsetstrokecolor{currentstroke}%
\pgfsetdash{}{0pt}%
\pgfpathmoveto{\pgfqpoint{1.666655in}{4.378675in}}%
\pgfpathlineto{\pgfqpoint{7.850000in}{4.378675in}}%
\pgfusepath{stroke}%
\end{pgfscope}%
\begin{pgfscope}%
\definecolor{textcolor}{rgb}{0.150000,0.150000,0.150000}%
\pgfsetstrokecolor{textcolor}%
\pgfsetfillcolor{textcolor}%
\pgftext[x=0.950943in, y=4.273152in, left, base]{\color{textcolor}\rmfamily\fontsize{20.000000}{24.000000}\selectfont 1.25}%
\end{pgfscope}%
\begin{pgfscope}%
\definecolor{textcolor}{rgb}{0.150000,0.150000,0.150000}%
\pgfsetstrokecolor{textcolor}%
\pgfsetfillcolor{textcolor}%
\pgftext[x=0.895387in,y=2.956163in,,bottom,rotate=90.000000]{\color{textcolor}\rmfamily\fontsize{24.000000}{28.800000}\selectfont \(\displaystyle \frac{1}{T_i}\) \(\displaystyle  (\frac{1}{ms})\)}%
\end{pgfscope}%
\begin{pgfscope}%
\pgfpathrectangle{\pgfqpoint{1.666655in}{0.898915in}}{\pgfqpoint{6.183345in}{4.114497in}}%
\pgfusepath{clip}%
\pgfsetbuttcap%
\pgfsetroundjoin%
\definecolor{currentfill}{rgb}{0.392157,0.584314,0.929412}%
\pgfsetfillcolor{currentfill}%
\pgfsetlinewidth{0.301125pt}%
\definecolor{currentstroke}{rgb}{0.392157,0.584314,0.929412}%
\pgfsetstrokecolor{currentstroke}%
\pgfsetdash{}{0pt}%
\pgfsys@defobject{currentmarker}{\pgfqpoint{-0.048611in}{-0.048611in}}{\pgfqpoint{0.048611in}{0.048611in}}{%
\pgfpathmoveto{\pgfqpoint{0.000000in}{-0.048611in}}%
\pgfpathcurveto{\pgfqpoint{0.012892in}{-0.048611in}}{\pgfqpoint{0.025257in}{-0.043489in}}{\pgfqpoint{0.034373in}{-0.034373in}}%
\pgfpathcurveto{\pgfqpoint{0.043489in}{-0.025257in}}{\pgfqpoint{0.048611in}{-0.012892in}}{\pgfqpoint{0.048611in}{0.000000in}}%
\pgfpathcurveto{\pgfqpoint{0.048611in}{0.012892in}}{\pgfqpoint{0.043489in}{0.025257in}}{\pgfqpoint{0.034373in}{0.034373in}}%
\pgfpathcurveto{\pgfqpoint{0.025257in}{0.043489in}}{\pgfqpoint{0.012892in}{0.048611in}}{\pgfqpoint{0.000000in}{0.048611in}}%
\pgfpathcurveto{\pgfqpoint{-0.012892in}{0.048611in}}{\pgfqpoint{-0.025257in}{0.043489in}}{\pgfqpoint{-0.034373in}{0.034373in}}%
\pgfpathcurveto{\pgfqpoint{-0.043489in}{0.025257in}}{\pgfqpoint{-0.048611in}{0.012892in}}{\pgfqpoint{-0.048611in}{0.000000in}}%
\pgfpathcurveto{\pgfqpoint{-0.048611in}{-0.012892in}}{\pgfqpoint{-0.043489in}{-0.025257in}}{\pgfqpoint{-0.034373in}{-0.034373in}}%
\pgfpathcurveto{\pgfqpoint{-0.025257in}{-0.043489in}}{\pgfqpoint{-0.012892in}{-0.048611in}}{\pgfqpoint{0.000000in}{-0.048611in}}%
\pgfpathlineto{\pgfqpoint{0.000000in}{-0.048611in}}%
\pgfpathclose%
\pgfusepath{stroke,fill}%
}%
\begin{pgfscope}%
\pgfsys@transformshift{1.975822in}{1.136666in}%
\pgfsys@useobject{currentmarker}{}%
\end{pgfscope}%
\begin{pgfscope}%
\pgfsys@transformshift{2.003928in}{1.143851in}%
\pgfsys@useobject{currentmarker}{}%
\end{pgfscope}%
\begin{pgfscope}%
\pgfsys@transformshift{2.228777in}{1.275414in}%
\pgfsys@useobject{currentmarker}{}%
\end{pgfscope}%
\begin{pgfscope}%
\pgfsys@transformshift{2.509838in}{1.442732in}%
\pgfsys@useobject{currentmarker}{}%
\end{pgfscope}%
\begin{pgfscope}%
\pgfsys@transformshift{3.071961in}{1.753085in}%
\pgfsys@useobject{currentmarker}{}%
\end{pgfscope}%
\begin{pgfscope}%
\pgfsys@transformshift{4.758327in}{2.646999in}%
\pgfsys@useobject{currentmarker}{}%
\end{pgfscope}%
\begin{pgfscope}%
\pgfsys@transformshift{7.568939in}{4.580632in}%
\pgfsys@useobject{currentmarker}{}%
\end{pgfscope}%
\end{pgfscope}%
\begin{pgfscope}%
\pgfpathrectangle{\pgfqpoint{1.666655in}{0.898915in}}{\pgfqpoint{6.183345in}{4.114497in}}%
\pgfusepath{clip}%
\pgfsetbuttcap%
\pgfsetroundjoin%
\definecolor{currentfill}{rgb}{0.698039,0.133333,0.133333}%
\pgfsetfillcolor{currentfill}%
\pgfsetlinewidth{0.301125pt}%
\definecolor{currentstroke}{rgb}{0.698039,0.133333,0.133333}%
\pgfsetstrokecolor{currentstroke}%
\pgfsetdash{}{0pt}%
\pgfsys@defobject{currentmarker}{\pgfqpoint{-0.048611in}{-0.048611in}}{\pgfqpoint{0.048611in}{0.048611in}}{%
\pgfpathmoveto{\pgfqpoint{0.000000in}{-0.048611in}}%
\pgfpathcurveto{\pgfqpoint{0.012892in}{-0.048611in}}{\pgfqpoint{0.025257in}{-0.043489in}}{\pgfqpoint{0.034373in}{-0.034373in}}%
\pgfpathcurveto{\pgfqpoint{0.043489in}{-0.025257in}}{\pgfqpoint{0.048611in}{-0.012892in}}{\pgfqpoint{0.048611in}{0.000000in}}%
\pgfpathcurveto{\pgfqpoint{0.048611in}{0.012892in}}{\pgfqpoint{0.043489in}{0.025257in}}{\pgfqpoint{0.034373in}{0.034373in}}%
\pgfpathcurveto{\pgfqpoint{0.025257in}{0.043489in}}{\pgfqpoint{0.012892in}{0.048611in}}{\pgfqpoint{0.000000in}{0.048611in}}%
\pgfpathcurveto{\pgfqpoint{-0.012892in}{0.048611in}}{\pgfqpoint{-0.025257in}{0.043489in}}{\pgfqpoint{-0.034373in}{0.034373in}}%
\pgfpathcurveto{\pgfqpoint{-0.043489in}{0.025257in}}{\pgfqpoint{-0.048611in}{0.012892in}}{\pgfqpoint{-0.048611in}{0.000000in}}%
\pgfpathcurveto{\pgfqpoint{-0.048611in}{-0.012892in}}{\pgfqpoint{-0.043489in}{-0.025257in}}{\pgfqpoint{-0.034373in}{-0.034373in}}%
\pgfpathcurveto{\pgfqpoint{-0.025257in}{-0.043489in}}{\pgfqpoint{-0.012892in}{-0.048611in}}{\pgfqpoint{0.000000in}{-0.048611in}}%
\pgfpathlineto{\pgfqpoint{0.000000in}{-0.048611in}}%
\pgfpathclose%
\pgfusepath{stroke,fill}%
}%
\begin{pgfscope}%
\pgfsys@transformshift{1.975822in}{1.146550in}%
\pgfsys@useobject{currentmarker}{}%
\end{pgfscope}%
\begin{pgfscope}%
\pgfsys@transformshift{2.003928in}{1.152622in}%
\pgfsys@useobject{currentmarker}{}%
\end{pgfscope}%
\begin{pgfscope}%
\pgfsys@transformshift{2.228777in}{1.263084in}%
\pgfsys@useobject{currentmarker}{}%
\end{pgfscope}%
\begin{pgfscope}%
\pgfsys@transformshift{2.509838in}{1.449672in}%
\pgfsys@useobject{currentmarker}{}%
\end{pgfscope}%
\begin{pgfscope}%
\pgfsys@transformshift{3.071961in}{1.657190in}%
\pgfsys@useobject{currentmarker}{}%
\end{pgfscope}%
\begin{pgfscope}%
\pgfsys@transformshift{4.758327in}{2.149995in}%
\pgfsys@useobject{currentmarker}{}%
\end{pgfscope}%
\begin{pgfscope}%
\pgfsys@transformshift{7.568939in}{3.705277in}%
\pgfsys@useobject{currentmarker}{}%
\end{pgfscope}%
\end{pgfscope}%
\begin{pgfscope}%
\pgfpathrectangle{\pgfqpoint{1.666655in}{0.898915in}}{\pgfqpoint{6.183345in}{4.114497in}}%
\pgfusepath{clip}%
\pgfsetroundcap%
\pgfsetroundjoin%
\pgfsetlinewidth{1.756562pt}%
\definecolor{currentstroke}{rgb}{0.392157,0.584314,0.929412}%
\pgfsetstrokecolor{currentstroke}%
\pgfsetdash{}{0pt}%
\pgfpathmoveto{\pgfqpoint{1.947716in}{1.085937in}}%
\pgfpathlineto{\pgfqpoint{2.004496in}{1.120480in}}%
\pgfpathlineto{\pgfqpoint{2.061276in}{1.155023in}}%
\pgfpathlineto{\pgfqpoint{2.118056in}{1.189566in}}%
\pgfpathlineto{\pgfqpoint{2.174836in}{1.224109in}}%
\pgfpathlineto{\pgfqpoint{2.231616in}{1.258651in}}%
\pgfpathlineto{\pgfqpoint{2.288396in}{1.293194in}}%
\pgfpathlineto{\pgfqpoint{2.345176in}{1.327737in}}%
\pgfpathlineto{\pgfqpoint{2.401956in}{1.362280in}}%
\pgfpathlineto{\pgfqpoint{2.458736in}{1.396823in}}%
\pgfpathlineto{\pgfqpoint{2.515516in}{1.431366in}}%
\pgfpathlineto{\pgfqpoint{2.572296in}{1.465909in}}%
\pgfpathlineto{\pgfqpoint{2.629076in}{1.500451in}}%
\pgfpathlineto{\pgfqpoint{2.685856in}{1.534994in}}%
\pgfpathlineto{\pgfqpoint{2.742636in}{1.569537in}}%
\pgfpathlineto{\pgfqpoint{2.799416in}{1.604080in}}%
\pgfpathlineto{\pgfqpoint{2.856196in}{1.638623in}}%
\pgfpathlineto{\pgfqpoint{2.912976in}{1.673166in}}%
\pgfpathlineto{\pgfqpoint{2.969757in}{1.707708in}}%
\pgfpathlineto{\pgfqpoint{3.026537in}{1.742251in}}%
\pgfpathlineto{\pgfqpoint{3.083317in}{1.776794in}}%
\pgfpathlineto{\pgfqpoint{3.140097in}{1.811337in}}%
\pgfpathlineto{\pgfqpoint{3.196877in}{1.845880in}}%
\pgfpathlineto{\pgfqpoint{3.253657in}{1.880423in}}%
\pgfpathlineto{\pgfqpoint{3.310437in}{1.914966in}}%
\pgfpathlineto{\pgfqpoint{3.367217in}{1.949508in}}%
\pgfpathlineto{\pgfqpoint{3.423997in}{1.984051in}}%
\pgfpathlineto{\pgfqpoint{3.480777in}{2.018594in}}%
\pgfpathlineto{\pgfqpoint{3.537557in}{2.053137in}}%
\pgfpathlineto{\pgfqpoint{3.594337in}{2.087680in}}%
\pgfpathlineto{\pgfqpoint{3.651117in}{2.122223in}}%
\pgfpathlineto{\pgfqpoint{3.707897in}{2.156766in}}%
\pgfpathlineto{\pgfqpoint{3.764677in}{2.191308in}}%
\pgfpathlineto{\pgfqpoint{3.821457in}{2.225851in}}%
\pgfpathlineto{\pgfqpoint{3.878237in}{2.260394in}}%
\pgfpathlineto{\pgfqpoint{3.935017in}{2.294937in}}%
\pgfpathlineto{\pgfqpoint{3.991797in}{2.329480in}}%
\pgfpathlineto{\pgfqpoint{4.048577in}{2.364023in}}%
\pgfpathlineto{\pgfqpoint{4.105357in}{2.398565in}}%
\pgfpathlineto{\pgfqpoint{4.162137in}{2.433108in}}%
\pgfpathlineto{\pgfqpoint{4.218917in}{2.467651in}}%
\pgfpathlineto{\pgfqpoint{4.275697in}{2.502194in}}%
\pgfpathlineto{\pgfqpoint{4.332477in}{2.536737in}}%
\pgfpathlineto{\pgfqpoint{4.389257in}{2.571280in}}%
\pgfpathlineto{\pgfqpoint{4.446037in}{2.605823in}}%
\pgfpathlineto{\pgfqpoint{4.502817in}{2.640365in}}%
\pgfpathlineto{\pgfqpoint{4.559597in}{2.674908in}}%
\pgfpathlineto{\pgfqpoint{4.616377in}{2.709451in}}%
\pgfpathlineto{\pgfqpoint{4.673157in}{2.743994in}}%
\pgfpathlineto{\pgfqpoint{4.729937in}{2.778537in}}%
\pgfpathlineto{\pgfqpoint{4.786717in}{2.813080in}}%
\pgfpathlineto{\pgfqpoint{4.843497in}{2.847623in}}%
\pgfpathlineto{\pgfqpoint{4.900278in}{2.882165in}}%
\pgfpathlineto{\pgfqpoint{4.957058in}{2.916708in}}%
\pgfpathlineto{\pgfqpoint{5.013838in}{2.951251in}}%
\pgfpathlineto{\pgfqpoint{5.070618in}{2.985794in}}%
\pgfpathlineto{\pgfqpoint{5.127398in}{3.020337in}}%
\pgfpathlineto{\pgfqpoint{5.184178in}{3.054880in}}%
\pgfpathlineto{\pgfqpoint{5.240958in}{3.089423in}}%
\pgfpathlineto{\pgfqpoint{5.297738in}{3.123965in}}%
\pgfpathlineto{\pgfqpoint{5.354518in}{3.158508in}}%
\pgfpathlineto{\pgfqpoint{5.411298in}{3.193051in}}%
\pgfpathlineto{\pgfqpoint{5.468078in}{3.227594in}}%
\pgfpathlineto{\pgfqpoint{5.524858in}{3.262137in}}%
\pgfpathlineto{\pgfqpoint{5.581638in}{3.296680in}}%
\pgfpathlineto{\pgfqpoint{5.638418in}{3.331222in}}%
\pgfpathlineto{\pgfqpoint{5.695198in}{3.365765in}}%
\pgfpathlineto{\pgfqpoint{5.751978in}{3.400308in}}%
\pgfpathlineto{\pgfqpoint{5.808758in}{3.434851in}}%
\pgfpathlineto{\pgfqpoint{5.865538in}{3.469394in}}%
\pgfpathlineto{\pgfqpoint{5.922318in}{3.503937in}}%
\pgfpathlineto{\pgfqpoint{5.979098in}{3.538480in}}%
\pgfpathlineto{\pgfqpoint{6.035878in}{3.573022in}}%
\pgfpathlineto{\pgfqpoint{6.092658in}{3.607565in}}%
\pgfpathlineto{\pgfqpoint{6.149438in}{3.642108in}}%
\pgfpathlineto{\pgfqpoint{6.206218in}{3.676651in}}%
\pgfpathlineto{\pgfqpoint{6.262998in}{3.711194in}}%
\pgfpathlineto{\pgfqpoint{6.319778in}{3.745737in}}%
\pgfpathlineto{\pgfqpoint{6.376558in}{3.780280in}}%
\pgfpathlineto{\pgfqpoint{6.433338in}{3.814822in}}%
\pgfpathlineto{\pgfqpoint{6.490118in}{3.849365in}}%
\pgfpathlineto{\pgfqpoint{6.546898in}{3.883908in}}%
\pgfpathlineto{\pgfqpoint{6.603678in}{3.918451in}}%
\pgfpathlineto{\pgfqpoint{6.660458in}{3.952994in}}%
\pgfpathlineto{\pgfqpoint{6.717238in}{3.987537in}}%
\pgfpathlineto{\pgfqpoint{6.774018in}{4.022079in}}%
\pgfpathlineto{\pgfqpoint{6.830798in}{4.056622in}}%
\pgfpathlineto{\pgfqpoint{6.887579in}{4.091165in}}%
\pgfpathlineto{\pgfqpoint{6.944359in}{4.125708in}}%
\pgfpathlineto{\pgfqpoint{7.001139in}{4.160251in}}%
\pgfpathlineto{\pgfqpoint{7.057919in}{4.194794in}}%
\pgfpathlineto{\pgfqpoint{7.114699in}{4.229337in}}%
\pgfpathlineto{\pgfqpoint{7.171479in}{4.263879in}}%
\pgfpathlineto{\pgfqpoint{7.228259in}{4.298422in}}%
\pgfpathlineto{\pgfqpoint{7.285039in}{4.332965in}}%
\pgfpathlineto{\pgfqpoint{7.341819in}{4.367508in}}%
\pgfpathlineto{\pgfqpoint{7.398599in}{4.402051in}}%
\pgfpathlineto{\pgfqpoint{7.455379in}{4.436594in}}%
\pgfpathlineto{\pgfqpoint{7.512159in}{4.471137in}}%
\pgfpathlineto{\pgfqpoint{7.568939in}{4.505679in}}%
\pgfusepath{stroke}%
\end{pgfscope}%
\begin{pgfscope}%
\pgfpathrectangle{\pgfqpoint{1.666655in}{0.898915in}}{\pgfqpoint{6.183345in}{4.114497in}}%
\pgfusepath{clip}%
\pgfsetbuttcap%
\pgfsetroundjoin%
\pgfsetlinewidth{1.756562pt}%
\definecolor{currentstroke}{rgb}{0.000000,0.000000,1.000000}%
\pgfsetstrokecolor{currentstroke}%
\pgfsetdash{}{0pt}%
\pgfpathmoveto{\pgfqpoint{1.975822in}{1.134966in}}%
\pgfpathlineto{\pgfqpoint{1.975822in}{1.138366in}}%
\pgfusepath{stroke}%
\end{pgfscope}%
\begin{pgfscope}%
\pgfpathrectangle{\pgfqpoint{1.666655in}{0.898915in}}{\pgfqpoint{6.183345in}{4.114497in}}%
\pgfusepath{clip}%
\pgfsetbuttcap%
\pgfsetroundjoin%
\pgfsetlinewidth{1.756562pt}%
\definecolor{currentstroke}{rgb}{0.000000,0.000000,1.000000}%
\pgfsetstrokecolor{currentstroke}%
\pgfsetdash{}{0pt}%
\pgfpathmoveto{\pgfqpoint{2.003928in}{1.139700in}}%
\pgfpathlineto{\pgfqpoint{2.003928in}{1.148002in}}%
\pgfusepath{stroke}%
\end{pgfscope}%
\begin{pgfscope}%
\pgfpathrectangle{\pgfqpoint{1.666655in}{0.898915in}}{\pgfqpoint{6.183345in}{4.114497in}}%
\pgfusepath{clip}%
\pgfsetbuttcap%
\pgfsetroundjoin%
\pgfsetlinewidth{1.756562pt}%
\definecolor{currentstroke}{rgb}{0.000000,0.000000,1.000000}%
\pgfsetstrokecolor{currentstroke}%
\pgfsetdash{}{0pt}%
\pgfpathmoveto{\pgfqpoint{2.228777in}{1.269125in}}%
\pgfpathlineto{\pgfqpoint{2.228777in}{1.281702in}}%
\pgfusepath{stroke}%
\end{pgfscope}%
\begin{pgfscope}%
\pgfpathrectangle{\pgfqpoint{1.666655in}{0.898915in}}{\pgfqpoint{6.183345in}{4.114497in}}%
\pgfusepath{clip}%
\pgfsetbuttcap%
\pgfsetroundjoin%
\pgfsetlinewidth{1.756562pt}%
\definecolor{currentstroke}{rgb}{0.000000,0.000000,1.000000}%
\pgfsetstrokecolor{currentstroke}%
\pgfsetdash{}{0pt}%
\pgfpathmoveto{\pgfqpoint{2.509838in}{1.420333in}}%
\pgfpathlineto{\pgfqpoint{2.509838in}{1.465132in}}%
\pgfusepath{stroke}%
\end{pgfscope}%
\begin{pgfscope}%
\pgfpathrectangle{\pgfqpoint{1.666655in}{0.898915in}}{\pgfqpoint{6.183345in}{4.114497in}}%
\pgfusepath{clip}%
\pgfsetbuttcap%
\pgfsetroundjoin%
\pgfsetlinewidth{1.756562pt}%
\definecolor{currentstroke}{rgb}{0.000000,0.000000,1.000000}%
\pgfsetstrokecolor{currentstroke}%
\pgfsetdash{}{0pt}%
\pgfpathmoveto{\pgfqpoint{3.071961in}{1.704808in}}%
\pgfpathlineto{\pgfqpoint{3.071961in}{1.801362in}}%
\pgfusepath{stroke}%
\end{pgfscope}%
\begin{pgfscope}%
\pgfpathrectangle{\pgfqpoint{1.666655in}{0.898915in}}{\pgfqpoint{6.183345in}{4.114497in}}%
\pgfusepath{clip}%
\pgfsetbuttcap%
\pgfsetroundjoin%
\pgfsetlinewidth{1.756562pt}%
\definecolor{currentstroke}{rgb}{0.000000,0.000000,1.000000}%
\pgfsetstrokecolor{currentstroke}%
\pgfsetdash{}{0pt}%
\pgfpathmoveto{\pgfqpoint{4.758327in}{2.548860in}}%
\pgfpathlineto{\pgfqpoint{4.758327in}{2.745138in}}%
\pgfusepath{stroke}%
\end{pgfscope}%
\begin{pgfscope}%
\pgfpathrectangle{\pgfqpoint{1.666655in}{0.898915in}}{\pgfqpoint{6.183345in}{4.114497in}}%
\pgfusepath{clip}%
\pgfsetbuttcap%
\pgfsetroundjoin%
\pgfsetlinewidth{1.756562pt}%
\definecolor{currentstroke}{rgb}{0.000000,0.000000,1.000000}%
\pgfsetstrokecolor{currentstroke}%
\pgfsetdash{}{0pt}%
\pgfpathmoveto{\pgfqpoint{7.568939in}{4.334876in}}%
\pgfpathlineto{\pgfqpoint{7.568939in}{4.826389in}}%
\pgfusepath{stroke}%
\end{pgfscope}%
\begin{pgfscope}%
\pgfpathrectangle{\pgfqpoint{1.666655in}{0.898915in}}{\pgfqpoint{6.183345in}{4.114497in}}%
\pgfusepath{clip}%
\pgfsetbuttcap%
\pgfsetroundjoin%
\definecolor{currentfill}{rgb}{0.000000,0.000000,1.000000}%
\pgfsetfillcolor{currentfill}%
\pgfsetlinewidth{1.003750pt}%
\definecolor{currentstroke}{rgb}{0.000000,0.000000,1.000000}%
\pgfsetstrokecolor{currentstroke}%
\pgfsetdash{}{0pt}%
\pgfsys@defobject{currentmarker}{\pgfqpoint{-0.069444in}{-0.000000in}}{\pgfqpoint{0.069444in}{0.000000in}}{%
\pgfpathmoveto{\pgfqpoint{0.069444in}{-0.000000in}}%
\pgfpathlineto{\pgfqpoint{-0.069444in}{0.000000in}}%
\pgfusepath{stroke,fill}%
}%
\begin{pgfscope}%
\pgfsys@transformshift{1.975822in}{1.134966in}%
\pgfsys@useobject{currentmarker}{}%
\end{pgfscope}%
\begin{pgfscope}%
\pgfsys@transformshift{2.003928in}{1.139700in}%
\pgfsys@useobject{currentmarker}{}%
\end{pgfscope}%
\begin{pgfscope}%
\pgfsys@transformshift{2.228777in}{1.269125in}%
\pgfsys@useobject{currentmarker}{}%
\end{pgfscope}%
\begin{pgfscope}%
\pgfsys@transformshift{2.509838in}{1.420333in}%
\pgfsys@useobject{currentmarker}{}%
\end{pgfscope}%
\begin{pgfscope}%
\pgfsys@transformshift{3.071961in}{1.704808in}%
\pgfsys@useobject{currentmarker}{}%
\end{pgfscope}%
\begin{pgfscope}%
\pgfsys@transformshift{4.758327in}{2.548860in}%
\pgfsys@useobject{currentmarker}{}%
\end{pgfscope}%
\begin{pgfscope}%
\pgfsys@transformshift{7.568939in}{4.334876in}%
\pgfsys@useobject{currentmarker}{}%
\end{pgfscope}%
\end{pgfscope}%
\begin{pgfscope}%
\pgfpathrectangle{\pgfqpoint{1.666655in}{0.898915in}}{\pgfqpoint{6.183345in}{4.114497in}}%
\pgfusepath{clip}%
\pgfsetbuttcap%
\pgfsetroundjoin%
\definecolor{currentfill}{rgb}{0.000000,0.000000,1.000000}%
\pgfsetfillcolor{currentfill}%
\pgfsetlinewidth{1.003750pt}%
\definecolor{currentstroke}{rgb}{0.000000,0.000000,1.000000}%
\pgfsetstrokecolor{currentstroke}%
\pgfsetdash{}{0pt}%
\pgfsys@defobject{currentmarker}{\pgfqpoint{-0.069444in}{-0.000000in}}{\pgfqpoint{0.069444in}{0.000000in}}{%
\pgfpathmoveto{\pgfqpoint{0.069444in}{-0.000000in}}%
\pgfpathlineto{\pgfqpoint{-0.069444in}{0.000000in}}%
\pgfusepath{stroke,fill}%
}%
\begin{pgfscope}%
\pgfsys@transformshift{1.975822in}{1.138366in}%
\pgfsys@useobject{currentmarker}{}%
\end{pgfscope}%
\begin{pgfscope}%
\pgfsys@transformshift{2.003928in}{1.148002in}%
\pgfsys@useobject{currentmarker}{}%
\end{pgfscope}%
\begin{pgfscope}%
\pgfsys@transformshift{2.228777in}{1.281702in}%
\pgfsys@useobject{currentmarker}{}%
\end{pgfscope}%
\begin{pgfscope}%
\pgfsys@transformshift{2.509838in}{1.465132in}%
\pgfsys@useobject{currentmarker}{}%
\end{pgfscope}%
\begin{pgfscope}%
\pgfsys@transformshift{3.071961in}{1.801362in}%
\pgfsys@useobject{currentmarker}{}%
\end{pgfscope}%
\begin{pgfscope}%
\pgfsys@transformshift{4.758327in}{2.745138in}%
\pgfsys@useobject{currentmarker}{}%
\end{pgfscope}%
\begin{pgfscope}%
\pgfsys@transformshift{7.568939in}{4.826389in}%
\pgfsys@useobject{currentmarker}{}%
\end{pgfscope}%
\end{pgfscope}%
\begin{pgfscope}%
\pgfpathrectangle{\pgfqpoint{1.666655in}{0.898915in}}{\pgfqpoint{6.183345in}{4.114497in}}%
\pgfusepath{clip}%
\pgfsetroundcap%
\pgfsetroundjoin%
\pgfsetlinewidth{1.756562pt}%
\definecolor{currentstroke}{rgb}{0.698039,0.133333,0.133333}%
\pgfsetstrokecolor{currentstroke}%
\pgfsetdash{}{0pt}%
\pgfpathmoveto{\pgfqpoint{1.947716in}{1.127790in}}%
\pgfpathlineto{\pgfqpoint{2.004496in}{1.152866in}}%
\pgfpathlineto{\pgfqpoint{2.061276in}{1.177942in}}%
\pgfpathlineto{\pgfqpoint{2.118056in}{1.203017in}}%
\pgfpathlineto{\pgfqpoint{2.174836in}{1.228093in}}%
\pgfpathlineto{\pgfqpoint{2.231616in}{1.253169in}}%
\pgfpathlineto{\pgfqpoint{2.288396in}{1.278245in}}%
\pgfpathlineto{\pgfqpoint{2.345176in}{1.303320in}}%
\pgfpathlineto{\pgfqpoint{2.401956in}{1.328396in}}%
\pgfpathlineto{\pgfqpoint{2.458736in}{1.353472in}}%
\pgfpathlineto{\pgfqpoint{2.515516in}{1.378548in}}%
\pgfpathlineto{\pgfqpoint{2.572296in}{1.403623in}}%
\pgfpathlineto{\pgfqpoint{2.629076in}{1.428699in}}%
\pgfpathlineto{\pgfqpoint{2.685856in}{1.453775in}}%
\pgfpathlineto{\pgfqpoint{2.742636in}{1.478851in}}%
\pgfpathlineto{\pgfqpoint{2.799416in}{1.503926in}}%
\pgfpathlineto{\pgfqpoint{2.856196in}{1.529002in}}%
\pgfpathlineto{\pgfqpoint{2.912976in}{1.554078in}}%
\pgfpathlineto{\pgfqpoint{2.969757in}{1.579154in}}%
\pgfpathlineto{\pgfqpoint{3.026537in}{1.604229in}}%
\pgfpathlineto{\pgfqpoint{3.083317in}{1.629305in}}%
\pgfpathlineto{\pgfqpoint{3.140097in}{1.654381in}}%
\pgfpathlineto{\pgfqpoint{3.196877in}{1.679457in}}%
\pgfpathlineto{\pgfqpoint{3.253657in}{1.704532in}}%
\pgfpathlineto{\pgfqpoint{3.310437in}{1.729608in}}%
\pgfpathlineto{\pgfqpoint{3.367217in}{1.754684in}}%
\pgfpathlineto{\pgfqpoint{3.423997in}{1.779759in}}%
\pgfpathlineto{\pgfqpoint{3.480777in}{1.804835in}}%
\pgfpathlineto{\pgfqpoint{3.537557in}{1.829911in}}%
\pgfpathlineto{\pgfqpoint{3.594337in}{1.854987in}}%
\pgfpathlineto{\pgfqpoint{3.651117in}{1.880062in}}%
\pgfpathlineto{\pgfqpoint{3.707897in}{1.905138in}}%
\pgfpathlineto{\pgfqpoint{3.764677in}{1.930214in}}%
\pgfpathlineto{\pgfqpoint{3.821457in}{1.955290in}}%
\pgfpathlineto{\pgfqpoint{3.878237in}{1.980365in}}%
\pgfpathlineto{\pgfqpoint{3.935017in}{2.005441in}}%
\pgfpathlineto{\pgfqpoint{3.991797in}{2.030517in}}%
\pgfpathlineto{\pgfqpoint{4.048577in}{2.055593in}}%
\pgfpathlineto{\pgfqpoint{4.105357in}{2.080668in}}%
\pgfpathlineto{\pgfqpoint{4.162137in}{2.105744in}}%
\pgfpathlineto{\pgfqpoint{4.218917in}{2.130820in}}%
\pgfpathlineto{\pgfqpoint{4.275697in}{2.155896in}}%
\pgfpathlineto{\pgfqpoint{4.332477in}{2.180971in}}%
\pgfpathlineto{\pgfqpoint{4.389257in}{2.206047in}}%
\pgfpathlineto{\pgfqpoint{4.446037in}{2.231123in}}%
\pgfpathlineto{\pgfqpoint{4.502817in}{2.256199in}}%
\pgfpathlineto{\pgfqpoint{4.559597in}{2.281274in}}%
\pgfpathlineto{\pgfqpoint{4.616377in}{2.306350in}}%
\pgfpathlineto{\pgfqpoint{4.673157in}{2.331426in}}%
\pgfpathlineto{\pgfqpoint{4.729937in}{2.356502in}}%
\pgfpathlineto{\pgfqpoint{4.786717in}{2.381577in}}%
\pgfpathlineto{\pgfqpoint{4.843497in}{2.406653in}}%
\pgfpathlineto{\pgfqpoint{4.900278in}{2.431729in}}%
\pgfpathlineto{\pgfqpoint{4.957058in}{2.456805in}}%
\pgfpathlineto{\pgfqpoint{5.013838in}{2.481880in}}%
\pgfpathlineto{\pgfqpoint{5.070618in}{2.506956in}}%
\pgfpathlineto{\pgfqpoint{5.127398in}{2.532032in}}%
\pgfpathlineto{\pgfqpoint{5.184178in}{2.557107in}}%
\pgfpathlineto{\pgfqpoint{5.240958in}{2.582183in}}%
\pgfpathlineto{\pgfqpoint{5.297738in}{2.607259in}}%
\pgfpathlineto{\pgfqpoint{5.354518in}{2.632335in}}%
\pgfpathlineto{\pgfqpoint{5.411298in}{2.657410in}}%
\pgfpathlineto{\pgfqpoint{5.468078in}{2.682486in}}%
\pgfpathlineto{\pgfqpoint{5.524858in}{2.707562in}}%
\pgfpathlineto{\pgfqpoint{5.581638in}{2.732638in}}%
\pgfpathlineto{\pgfqpoint{5.638418in}{2.757713in}}%
\pgfpathlineto{\pgfqpoint{5.695198in}{2.782789in}}%
\pgfpathlineto{\pgfqpoint{5.751978in}{2.807865in}}%
\pgfpathlineto{\pgfqpoint{5.808758in}{2.832941in}}%
\pgfpathlineto{\pgfqpoint{5.865538in}{2.858016in}}%
\pgfpathlineto{\pgfqpoint{5.922318in}{2.883092in}}%
\pgfpathlineto{\pgfqpoint{5.979098in}{2.908168in}}%
\pgfpathlineto{\pgfqpoint{6.035878in}{2.933244in}}%
\pgfpathlineto{\pgfqpoint{6.092658in}{2.958319in}}%
\pgfpathlineto{\pgfqpoint{6.149438in}{2.983395in}}%
\pgfpathlineto{\pgfqpoint{6.206218in}{3.008471in}}%
\pgfpathlineto{\pgfqpoint{6.262998in}{3.033547in}}%
\pgfpathlineto{\pgfqpoint{6.319778in}{3.058622in}}%
\pgfpathlineto{\pgfqpoint{6.376558in}{3.083698in}}%
\pgfpathlineto{\pgfqpoint{6.433338in}{3.108774in}}%
\pgfpathlineto{\pgfqpoint{6.490118in}{3.133850in}}%
\pgfpathlineto{\pgfqpoint{6.546898in}{3.158925in}}%
\pgfpathlineto{\pgfqpoint{6.603678in}{3.184001in}}%
\pgfpathlineto{\pgfqpoint{6.660458in}{3.209077in}}%
\pgfpathlineto{\pgfqpoint{6.717238in}{3.234153in}}%
\pgfpathlineto{\pgfqpoint{6.774018in}{3.259228in}}%
\pgfpathlineto{\pgfqpoint{6.830798in}{3.284304in}}%
\pgfpathlineto{\pgfqpoint{6.887579in}{3.309380in}}%
\pgfpathlineto{\pgfqpoint{6.944359in}{3.334456in}}%
\pgfpathlineto{\pgfqpoint{7.001139in}{3.359531in}}%
\pgfpathlineto{\pgfqpoint{7.057919in}{3.384607in}}%
\pgfpathlineto{\pgfqpoint{7.114699in}{3.409683in}}%
\pgfpathlineto{\pgfqpoint{7.171479in}{3.434758in}}%
\pgfpathlineto{\pgfqpoint{7.228259in}{3.459834in}}%
\pgfpathlineto{\pgfqpoint{7.285039in}{3.484910in}}%
\pgfpathlineto{\pgfqpoint{7.341819in}{3.509986in}}%
\pgfpathlineto{\pgfqpoint{7.398599in}{3.535061in}}%
\pgfpathlineto{\pgfqpoint{7.455379in}{3.560137in}}%
\pgfpathlineto{\pgfqpoint{7.512159in}{3.585213in}}%
\pgfpathlineto{\pgfqpoint{7.568939in}{3.610289in}}%
\pgfusepath{stroke}%
\end{pgfscope}%
\begin{pgfscope}%
\pgfpathrectangle{\pgfqpoint{1.666655in}{0.898915in}}{\pgfqpoint{6.183345in}{4.114497in}}%
\pgfusepath{clip}%
\pgfsetbuttcap%
\pgfsetroundjoin%
\pgfsetlinewidth{1.756562pt}%
\definecolor{currentstroke}{rgb}{1.000000,0.000000,0.000000}%
\pgfsetstrokecolor{currentstroke}%
\pgfsetdash{}{0pt}%
\pgfpathmoveto{\pgfqpoint{1.975822in}{1.144890in}}%
\pgfpathlineto{\pgfqpoint{1.975822in}{1.148211in}}%
\pgfusepath{stroke}%
\end{pgfscope}%
\begin{pgfscope}%
\pgfpathrectangle{\pgfqpoint{1.666655in}{0.898915in}}{\pgfqpoint{6.183345in}{4.114497in}}%
\pgfusepath{clip}%
\pgfsetbuttcap%
\pgfsetroundjoin%
\pgfsetlinewidth{1.756562pt}%
\definecolor{currentstroke}{rgb}{1.000000,0.000000,0.000000}%
\pgfsetstrokecolor{currentstroke}%
\pgfsetdash{}{0pt}%
\pgfpathmoveto{\pgfqpoint{2.003928in}{1.148939in}}%
\pgfpathlineto{\pgfqpoint{2.003928in}{1.156306in}}%
\pgfusepath{stroke}%
\end{pgfscope}%
\begin{pgfscope}%
\pgfpathrectangle{\pgfqpoint{1.666655in}{0.898915in}}{\pgfqpoint{6.183345in}{4.114497in}}%
\pgfusepath{clip}%
\pgfsetbuttcap%
\pgfsetroundjoin%
\pgfsetlinewidth{1.756562pt}%
\definecolor{currentstroke}{rgb}{1.000000,0.000000,0.000000}%
\pgfsetstrokecolor{currentstroke}%
\pgfsetdash{}{0pt}%
\pgfpathmoveto{\pgfqpoint{2.228777in}{1.259221in}}%
\pgfpathlineto{\pgfqpoint{2.228777in}{1.266948in}}%
\pgfusepath{stroke}%
\end{pgfscope}%
\begin{pgfscope}%
\pgfpathrectangle{\pgfqpoint{1.666655in}{0.898915in}}{\pgfqpoint{6.183345in}{4.114497in}}%
\pgfusepath{clip}%
\pgfsetbuttcap%
\pgfsetroundjoin%
\pgfsetlinewidth{1.756562pt}%
\definecolor{currentstroke}{rgb}{1.000000,0.000000,0.000000}%
\pgfsetstrokecolor{currentstroke}%
\pgfsetdash{}{0pt}%
\pgfpathmoveto{\pgfqpoint{2.509838in}{1.432682in}}%
\pgfpathlineto{\pgfqpoint{2.509838in}{1.466662in}}%
\pgfusepath{stroke}%
\end{pgfscope}%
\begin{pgfscope}%
\pgfpathrectangle{\pgfqpoint{1.666655in}{0.898915in}}{\pgfqpoint{6.183345in}{4.114497in}}%
\pgfusepath{clip}%
\pgfsetbuttcap%
\pgfsetroundjoin%
\pgfsetlinewidth{1.756562pt}%
\definecolor{currentstroke}{rgb}{1.000000,0.000000,0.000000}%
\pgfsetstrokecolor{currentstroke}%
\pgfsetdash{}{0pt}%
\pgfpathmoveto{\pgfqpoint{3.071961in}{1.617085in}}%
\pgfpathlineto{\pgfqpoint{3.071961in}{1.697295in}}%
\pgfusepath{stroke}%
\end{pgfscope}%
\begin{pgfscope}%
\pgfpathrectangle{\pgfqpoint{1.666655in}{0.898915in}}{\pgfqpoint{6.183345in}{4.114497in}}%
\pgfusepath{clip}%
\pgfsetbuttcap%
\pgfsetroundjoin%
\pgfsetlinewidth{1.756562pt}%
\definecolor{currentstroke}{rgb}{1.000000,0.000000,0.000000}%
\pgfsetstrokecolor{currentstroke}%
\pgfsetdash{}{0pt}%
\pgfpathmoveto{\pgfqpoint{4.758327in}{1.968099in}}%
\pgfpathlineto{\pgfqpoint{4.758327in}{2.331892in}}%
\pgfusepath{stroke}%
\end{pgfscope}%
\begin{pgfscope}%
\pgfpathrectangle{\pgfqpoint{1.666655in}{0.898915in}}{\pgfqpoint{6.183345in}{4.114497in}}%
\pgfusepath{clip}%
\pgfsetbuttcap%
\pgfsetroundjoin%
\pgfsetlinewidth{1.756562pt}%
\definecolor{currentstroke}{rgb}{1.000000,0.000000,0.000000}%
\pgfsetstrokecolor{currentstroke}%
\pgfsetdash{}{0pt}%
\pgfpathmoveto{\pgfqpoint{7.568939in}{3.603905in}}%
\pgfpathlineto{\pgfqpoint{7.568939in}{3.806650in}}%
\pgfusepath{stroke}%
\end{pgfscope}%
\begin{pgfscope}%
\pgfpathrectangle{\pgfqpoint{1.666655in}{0.898915in}}{\pgfqpoint{6.183345in}{4.114497in}}%
\pgfusepath{clip}%
\pgfsetbuttcap%
\pgfsetroundjoin%
\definecolor{currentfill}{rgb}{1.000000,0.000000,0.000000}%
\pgfsetfillcolor{currentfill}%
\pgfsetlinewidth{1.003750pt}%
\definecolor{currentstroke}{rgb}{1.000000,0.000000,0.000000}%
\pgfsetstrokecolor{currentstroke}%
\pgfsetdash{}{0pt}%
\pgfsys@defobject{currentmarker}{\pgfqpoint{-0.069444in}{-0.000000in}}{\pgfqpoint{0.069444in}{0.000000in}}{%
\pgfpathmoveto{\pgfqpoint{0.069444in}{-0.000000in}}%
\pgfpathlineto{\pgfqpoint{-0.069444in}{0.000000in}}%
\pgfusepath{stroke,fill}%
}%
\begin{pgfscope}%
\pgfsys@transformshift{1.975822in}{1.144890in}%
\pgfsys@useobject{currentmarker}{}%
\end{pgfscope}%
\begin{pgfscope}%
\pgfsys@transformshift{2.003928in}{1.148939in}%
\pgfsys@useobject{currentmarker}{}%
\end{pgfscope}%
\begin{pgfscope}%
\pgfsys@transformshift{2.228777in}{1.259221in}%
\pgfsys@useobject{currentmarker}{}%
\end{pgfscope}%
\begin{pgfscope}%
\pgfsys@transformshift{2.509838in}{1.432682in}%
\pgfsys@useobject{currentmarker}{}%
\end{pgfscope}%
\begin{pgfscope}%
\pgfsys@transformshift{3.071961in}{1.617085in}%
\pgfsys@useobject{currentmarker}{}%
\end{pgfscope}%
\begin{pgfscope}%
\pgfsys@transformshift{4.758327in}{1.968099in}%
\pgfsys@useobject{currentmarker}{}%
\end{pgfscope}%
\begin{pgfscope}%
\pgfsys@transformshift{7.568939in}{3.603905in}%
\pgfsys@useobject{currentmarker}{}%
\end{pgfscope}%
\end{pgfscope}%
\begin{pgfscope}%
\pgfpathrectangle{\pgfqpoint{1.666655in}{0.898915in}}{\pgfqpoint{6.183345in}{4.114497in}}%
\pgfusepath{clip}%
\pgfsetbuttcap%
\pgfsetroundjoin%
\definecolor{currentfill}{rgb}{1.000000,0.000000,0.000000}%
\pgfsetfillcolor{currentfill}%
\pgfsetlinewidth{1.003750pt}%
\definecolor{currentstroke}{rgb}{1.000000,0.000000,0.000000}%
\pgfsetstrokecolor{currentstroke}%
\pgfsetdash{}{0pt}%
\pgfsys@defobject{currentmarker}{\pgfqpoint{-0.069444in}{-0.000000in}}{\pgfqpoint{0.069444in}{0.000000in}}{%
\pgfpathmoveto{\pgfqpoint{0.069444in}{-0.000000in}}%
\pgfpathlineto{\pgfqpoint{-0.069444in}{0.000000in}}%
\pgfusepath{stroke,fill}%
}%
\begin{pgfscope}%
\pgfsys@transformshift{1.975822in}{1.148211in}%
\pgfsys@useobject{currentmarker}{}%
\end{pgfscope}%
\begin{pgfscope}%
\pgfsys@transformshift{2.003928in}{1.156306in}%
\pgfsys@useobject{currentmarker}{}%
\end{pgfscope}%
\begin{pgfscope}%
\pgfsys@transformshift{2.228777in}{1.266948in}%
\pgfsys@useobject{currentmarker}{}%
\end{pgfscope}%
\begin{pgfscope}%
\pgfsys@transformshift{2.509838in}{1.466662in}%
\pgfsys@useobject{currentmarker}{}%
\end{pgfscope}%
\begin{pgfscope}%
\pgfsys@transformshift{3.071961in}{1.697295in}%
\pgfsys@useobject{currentmarker}{}%
\end{pgfscope}%
\begin{pgfscope}%
\pgfsys@transformshift{4.758327in}{2.331892in}%
\pgfsys@useobject{currentmarker}{}%
\end{pgfscope}%
\begin{pgfscope}%
\pgfsys@transformshift{7.568939in}{3.806650in}%
\pgfsys@useobject{currentmarker}{}%
\end{pgfscope}%
\end{pgfscope}%
\begin{pgfscope}%
\pgfsetrectcap%
\pgfsetmiterjoin%
\pgfsetlinewidth{0.000000pt}%
\definecolor{currentstroke}{rgb}{1.000000,1.000000,1.000000}%
\pgfsetstrokecolor{currentstroke}%
\pgfsetdash{}{0pt}%
\pgfpathmoveto{\pgfqpoint{1.666655in}{0.898915in}}%
\pgfpathlineto{\pgfqpoint{1.666655in}{5.013411in}}%
\pgfusepath{}%
\end{pgfscope}%
\begin{pgfscope}%
\pgfsetrectcap%
\pgfsetmiterjoin%
\pgfsetlinewidth{0.000000pt}%
\definecolor{currentstroke}{rgb}{1.000000,1.000000,1.000000}%
\pgfsetstrokecolor{currentstroke}%
\pgfsetdash{}{0pt}%
\pgfpathmoveto{\pgfqpoint{7.850000in}{0.898915in}}%
\pgfpathlineto{\pgfqpoint{7.850000in}{5.013411in}}%
\pgfusepath{}%
\end{pgfscope}%
\begin{pgfscope}%
\pgfsetrectcap%
\pgfsetmiterjoin%
\pgfsetlinewidth{0.000000pt}%
\definecolor{currentstroke}{rgb}{1.000000,1.000000,1.000000}%
\pgfsetstrokecolor{currentstroke}%
\pgfsetdash{}{0pt}%
\pgfpathmoveto{\pgfqpoint{1.666655in}{0.898915in}}%
\pgfpathlineto{\pgfqpoint{7.850000in}{0.898915in}}%
\pgfusepath{}%
\end{pgfscope}%
\begin{pgfscope}%
\pgfsetrectcap%
\pgfsetmiterjoin%
\pgfsetlinewidth{0.000000pt}%
\definecolor{currentstroke}{rgb}{1.000000,1.000000,1.000000}%
\pgfsetstrokecolor{currentstroke}%
\pgfsetdash{}{0pt}%
\pgfpathmoveto{\pgfqpoint{1.666655in}{5.013411in}}%
\pgfpathlineto{\pgfqpoint{7.850000in}{5.013411in}}%
\pgfusepath{}%
\end{pgfscope}%
\begin{pgfscope}%
\definecolor{textcolor}{rgb}{0.150000,0.150000,0.150000}%
\pgfsetstrokecolor{textcolor}%
\pgfsetfillcolor{textcolor}%
\pgftext[x=4.196205in, y=1.849301in, left, base]{\color{textcolor}\rmfamily\fontsize{20.000000}{24.000000}\selectfont \(\displaystyle \frac{1}{T_1} =\) 1.31\(\displaystyle x +\) -0.0153}%
\end{pgfscope}%
\begin{pgfscope}%
\definecolor{textcolor}{rgb}{0.150000,0.150000,0.150000}%
\pgfsetstrokecolor{textcolor}%
\pgfsetfillcolor{textcolor}%
\pgftext[x=4.196205in, y=1.151833in, left, base]{\color{textcolor}\rmfamily\fontsize{20.000000}{24.000000}\selectfont \(\displaystyle \frac{1}{T_2} =\) 0.954\(\displaystyle x +\) 0.000761}%
\end{pgfscope}%
\begin{pgfscope}%
\definecolor{textcolor}{rgb}{0.150000,0.150000,0.150000}%
\pgfsetstrokecolor{textcolor}%
\pgfsetfillcolor{textcolor}%
\pgftext[x=4.758327in,y=5.096745in,,base]{\color{textcolor}\rmfamily\fontsize{24.000000}{28.800000}\selectfont  CuSO\(\displaystyle _4\)  }%
\end{pgfscope}%
\begin{pgfscope}%
\pgfsetbuttcap%
\pgfsetroundjoin%
\definecolor{currentfill}{rgb}{0.392157,0.584314,0.929412}%
\pgfsetfillcolor{currentfill}%
\pgfsetlinewidth{0.301125pt}%
\definecolor{currentstroke}{rgb}{0.392157,0.584314,0.929412}%
\pgfsetstrokecolor{currentstroke}%
\pgfsetdash{}{0pt}%
\pgfsys@defobject{currentmarker}{\pgfqpoint{-0.048611in}{-0.048611in}}{\pgfqpoint{0.048611in}{0.048611in}}{%
\pgfpathmoveto{\pgfqpoint{0.000000in}{-0.048611in}}%
\pgfpathcurveto{\pgfqpoint{0.012892in}{-0.048611in}}{\pgfqpoint{0.025257in}{-0.043489in}}{\pgfqpoint{0.034373in}{-0.034373in}}%
\pgfpathcurveto{\pgfqpoint{0.043489in}{-0.025257in}}{\pgfqpoint{0.048611in}{-0.012892in}}{\pgfqpoint{0.048611in}{0.000000in}}%
\pgfpathcurveto{\pgfqpoint{0.048611in}{0.012892in}}{\pgfqpoint{0.043489in}{0.025257in}}{\pgfqpoint{0.034373in}{0.034373in}}%
\pgfpathcurveto{\pgfqpoint{0.025257in}{0.043489in}}{\pgfqpoint{0.012892in}{0.048611in}}{\pgfqpoint{0.000000in}{0.048611in}}%
\pgfpathcurveto{\pgfqpoint{-0.012892in}{0.048611in}}{\pgfqpoint{-0.025257in}{0.043489in}}{\pgfqpoint{-0.034373in}{0.034373in}}%
\pgfpathcurveto{\pgfqpoint{-0.043489in}{0.025257in}}{\pgfqpoint{-0.048611in}{0.012892in}}{\pgfqpoint{-0.048611in}{0.000000in}}%
\pgfpathcurveto{\pgfqpoint{-0.048611in}{-0.012892in}}{\pgfqpoint{-0.043489in}{-0.025257in}}{\pgfqpoint{-0.034373in}{-0.034373in}}%
\pgfpathcurveto{\pgfqpoint{-0.025257in}{-0.043489in}}{\pgfqpoint{-0.012892in}{-0.048611in}}{\pgfqpoint{0.000000in}{-0.048611in}}%
\pgfpathlineto{\pgfqpoint{0.000000in}{-0.048611in}}%
\pgfpathclose%
\pgfusepath{stroke,fill}%
}%
\begin{pgfscope}%
\pgfsys@transformshift{2.194433in}{4.625282in}%
\pgfsys@useobject{currentmarker}{}%
\end{pgfscope}%
\end{pgfscope}%
\begin{pgfscope}%
\definecolor{textcolor}{rgb}{0.150000,0.150000,0.150000}%
\pgfsetstrokecolor{textcolor}%
\pgfsetfillcolor{textcolor}%
\pgftext[x=2.694433in,y=4.552365in,left,base]{\color{textcolor}\rmfamily\fontsize{20.000000}{24.000000}\selectfont Data T1}%
\end{pgfscope}%
\begin{pgfscope}%
\pgfsetroundcap%
\pgfsetroundjoin%
\pgfsetlinewidth{1.756562pt}%
\definecolor{currentstroke}{rgb}{0.392157,0.584314,0.929412}%
\pgfsetstrokecolor{currentstroke}%
\pgfsetdash{}{0pt}%
\pgfpathmoveto{\pgfqpoint{1.916655in}{4.241873in}}%
\pgfpathlineto{\pgfqpoint{2.194433in}{4.241873in}}%
\pgfpathlineto{\pgfqpoint{2.472210in}{4.241873in}}%
\pgfusepath{stroke}%
\end{pgfscope}%
\begin{pgfscope}%
\definecolor{textcolor}{rgb}{0.150000,0.150000,0.150000}%
\pgfsetstrokecolor{textcolor}%
\pgfsetfillcolor{textcolor}%
\pgftext[x=2.694433in,y=4.144651in,left,base]{\color{textcolor}\rmfamily\fontsize{20.000000}{24.000000}\selectfont T\(\displaystyle _1\) Fit}%
\end{pgfscope}%
\begin{pgfscope}%
\pgfsetbuttcap%
\pgfsetroundjoin%
\definecolor{currentfill}{rgb}{0.698039,0.133333,0.133333}%
\pgfsetfillcolor{currentfill}%
\pgfsetlinewidth{0.301125pt}%
\definecolor{currentstroke}{rgb}{0.698039,0.133333,0.133333}%
\pgfsetstrokecolor{currentstroke}%
\pgfsetdash{}{0pt}%
\pgfsys@defobject{currentmarker}{\pgfqpoint{-0.048611in}{-0.048611in}}{\pgfqpoint{0.048611in}{0.048611in}}{%
\pgfpathmoveto{\pgfqpoint{0.000000in}{-0.048611in}}%
\pgfpathcurveto{\pgfqpoint{0.012892in}{-0.048611in}}{\pgfqpoint{0.025257in}{-0.043489in}}{\pgfqpoint{0.034373in}{-0.034373in}}%
\pgfpathcurveto{\pgfqpoint{0.043489in}{-0.025257in}}{\pgfqpoint{0.048611in}{-0.012892in}}{\pgfqpoint{0.048611in}{0.000000in}}%
\pgfpathcurveto{\pgfqpoint{0.048611in}{0.012892in}}{\pgfqpoint{0.043489in}{0.025257in}}{\pgfqpoint{0.034373in}{0.034373in}}%
\pgfpathcurveto{\pgfqpoint{0.025257in}{0.043489in}}{\pgfqpoint{0.012892in}{0.048611in}}{\pgfqpoint{0.000000in}{0.048611in}}%
\pgfpathcurveto{\pgfqpoint{-0.012892in}{0.048611in}}{\pgfqpoint{-0.025257in}{0.043489in}}{\pgfqpoint{-0.034373in}{0.034373in}}%
\pgfpathcurveto{\pgfqpoint{-0.043489in}{0.025257in}}{\pgfqpoint{-0.048611in}{0.012892in}}{\pgfqpoint{-0.048611in}{0.000000in}}%
\pgfpathcurveto{\pgfqpoint{-0.048611in}{-0.012892in}}{\pgfqpoint{-0.043489in}{-0.025257in}}{\pgfqpoint{-0.034373in}{-0.034373in}}%
\pgfpathcurveto{\pgfqpoint{-0.025257in}{-0.043489in}}{\pgfqpoint{-0.012892in}{-0.048611in}}{\pgfqpoint{0.000000in}{-0.048611in}}%
\pgfpathlineto{\pgfqpoint{0.000000in}{-0.048611in}}%
\pgfpathclose%
\pgfusepath{stroke,fill}%
}%
\begin{pgfscope}%
\pgfsys@transformshift{2.194433in}{3.809853in}%
\pgfsys@useobject{currentmarker}{}%
\end{pgfscope}%
\end{pgfscope}%
\begin{pgfscope}%
\definecolor{textcolor}{rgb}{0.150000,0.150000,0.150000}%
\pgfsetstrokecolor{textcolor}%
\pgfsetfillcolor{textcolor}%
\pgftext[x=2.694433in,y=3.736936in,left,base]{\color{textcolor}\rmfamily\fontsize{20.000000}{24.000000}\selectfont Data T2}%
\end{pgfscope}%
\begin{pgfscope}%
\pgfsetroundcap%
\pgfsetroundjoin%
\pgfsetlinewidth{1.756562pt}%
\definecolor{currentstroke}{rgb}{0.698039,0.133333,0.133333}%
\pgfsetstrokecolor{currentstroke}%
\pgfsetdash{}{0pt}%
\pgfpathmoveto{\pgfqpoint{1.916655in}{3.426444in}}%
\pgfpathlineto{\pgfqpoint{2.194433in}{3.426444in}}%
\pgfpathlineto{\pgfqpoint{2.472210in}{3.426444in}}%
\pgfusepath{stroke}%
\end{pgfscope}%
\begin{pgfscope}%
\definecolor{textcolor}{rgb}{0.150000,0.150000,0.150000}%
\pgfsetstrokecolor{textcolor}%
\pgfsetfillcolor{textcolor}%
\pgftext[x=2.694433in,y=3.329222in,left,base]{\color{textcolor}\rmfamily\fontsize{20.000000}{24.000000}\selectfont T\(\displaystyle _2\) Fit}%
\end{pgfscope}%
\end{pgfpicture}%
\makeatother%
\endgroup%
}
    \caption{Inverse Time constants of CuSO$_{4} \boldsymbol{\cdot} $H$_2$O vs Molarity. Here 1/T$_i$ are proportional to $\gamma^2_p \mu^2_{eff} N_{ion} \eta / NkT$. Where $N_{ion}$ is the number density and is related to Molarity by $N_{ion} = N_A M$ }
    \label{fig:cu_linear}
  \end{figure}


\bibliography{nmr_DA}{}



\end{document}